% TO DO:
% number tabular also in superscripts
% line the baseline of images
\documentclass[12pt]{article}
\usepackage[a3paper,margin=1.5cm]{geometry}
\usepackage{fontspec}
\newfontfamily{\Junicode}{Junicode}[Numbers=Lining]
%\fontspec[Numbers=Lining]{Junicode}
\newcommand{\J}[1]{{\Junicode #1}}
% \usepackage{polyglossia}
% \setmainlanguage{polish}
% \setotherlanguage{english}
%\usepackage{csquotes}

\usepackage{metalogo}
% \usepackage[polish]{varioref}
% % dla varioref!
% \def\eob{ę}
\usepackage{xcolor}


\usepackage{relsize}

%\usepackage{float}

\usepackage{caption}

\usepackage[verbose]{hyperref}

\usepackage{graphicx}
% [hyphens]: options clash
\usepackage{url}
%\usepackage{natbib}

% program name
\newcommand{\pname}[1]{\textsf{#1}}


% file name
\newcommand{\fname}[1]{\texttt{#1}}

\newcommand{\uname}[1]{\texttt{'#1'}}
\newcommand{\ucode}[1]{\texttt{U+#1}}
\newcommand{\usi}[1]{\texttt{#1}}

% Aletheia
\newcommand{\aname}[1]{\texttt{#1}}
\newcommand{\acode}[1]{\texttt{#1}}

% MUFI
\newcommand{\mname}[1]{\texttt{'#1 \textsc{<mufi>'}}}
\newcommand{\mcode}[1]{\texttt{M+#1}}



%\usepackage{draftwatermark}
% \usepackage[doublespacing]{setspace}

\usepackage[draft]{fixme}

% nie działa:?
%\renewcommand{\topfraction}{0.9}
\renewcommand{\floatpagefraction}{0.9}	% require fuller float pages
\renewcommand{\topfraction}{0.9}	% max fraction of floats at top
\setcounter{topnumber}{5}
\setcounter{totalnumber}{5}  

\renewcommand{\labelenumii}{\arabic{enumii}.}

% \vrefwarning

% https://tex.stackexchange.com/questions/54136/hyperref-link-spans-a-pagebreak-looks-ugly
% nie zawsze działa!!!

% retrieve absolute page numbers (physical pages, as opposed to the
% ‘logical’ page number that is normally typeset when a page number is
% requested;
% \usepackage{zref-abspage}

\usepackage{expex}


\lingset{glhangstyle=none}
\defineglwlevels{pismo,nr}
\newcommand{\bg}{\begingl}

% 1 height
% 2 image

%\newcommand{\PTglyph}[2]{\includegraphics[height=#1ex]{glyphs/#2}}
\newcommand{\PTglyph}[2]{\includegraphics[height=8ex]{glyphs/#2}}
\newcommand{\PTglyphid}[1]{#1}

\parindent0pt

\begin{document}
%\gappto\captionslingua{\renewcommand{\chaptername}{Caput}}
%\gappto\captionspolish{\renewcommand{\figurename}{Ilustracja}}


\title{POLONIA TYPOGRAPHICA
  SAECULI SEDECIMI\\
  {\relsize{-2} TŁOCZNIE POLSKIE XVI STULECIA\\ MONOGRAFIE I PODOBIZNY
    ZASOBÓW DRUKARSKICH}\\Summary: Reconstructed font tables --- the index\\
  (draft)}

\author{Janusz S. Bień (editor)}

\date{\today}

\maketitle

\catcode`\&=11
\catcode`\|=11
\catcode`\_=11

\def\apostrof{`}


% dodac indeks!:
\catcode`\`=\active
\def`#1{\fbox{{\znak#1}}}

\def\Hb#1{{\fontspec{Junicode}#1}}

\section{Introduction}
\label{sec:introduction}

For more information about \textsc{POLONIA TYPOGRAPHICA SAECULI
  SEDECIMI} please consult
e.g. \url{https://github.com/jsbien/early_fonts_inventory}.

\end{document}



%%% Local Variables: 
%%% coding: utf-8-unix
%%% mode: latex
%%% TeX-master: t
%%% TeX-PDF-mode: t
%%% TeX-engine: xetex
%%% End: 
