% TO DO:
% https://en.wikipedia.org/wiki/Latin_delta
% https://en.wikipedia.org/wiki/Insular_script
% number tabular also in superscripts ???
% line the baseline of images ???
\documentclass[12pt]{article}
\usepackage[a3paper,margin=1.5cm]{geometry}
\usepackage{fontspec}
\newfontfamily{\Junicode}{Junicode}[Numbers=Lining]
%\fontspec[Numbers=Lining]{Junicode}
\newcommand{\J}[1]{{\Junicode #1}}
% \usepackage{polyglossia}
% \setmainlanguage{polish}
% \setotherlanguage{english}
%\usepackage{csquotes}

\usepackage{metalogo}
% \usepackage[polish]{varioref}
% % dla varioref!
% \def\eob{ę}
\usepackage{xcolor}


\usepackage{relsize}

%\usepackage{float}

\usepackage{caption}

\usepackage[verbose]{hyperref}

\usepackage{graphicx}
% [hyphens]: options clash
\usepackage{url}
%\usepackage{natbib}

% program name
\newcommand{\pname}[1]{\textsf{#1}}


% file name
\newcommand{\fname}[1]{\texttt{#1}}

\newcommand{\uname}[1]{\texttt{'#1'}}
\newcommand{\ucode}[1]{\texttt{U+#1}}
\newcommand{\usi}[1]{\texttt{#1}}

% Aletheia
\newcommand{\aname}[1]{\texttt{#1}}
\newcommand{\acode}[1]{\texttt{#1}}

% MUFI
\newcommand{\mname}[1]{\texttt{'#1 \textsc{<mufi>'}}}
\newcommand{\mcode}[1]{\texttt{M+#1}}



%\usepackage{draftwatermark}
% \usepackage[doublespacing]{setspace}

\usepackage[draft]{fixme}

% nie działa:?
%\renewcommand{\topfraction}{0.9}
\renewcommand{\floatpagefraction}{0.9}	% require fuller float pages
\renewcommand{\topfraction}{0.9}	% max fraction of floats at top
\setcounter{topnumber}{5}
\setcounter{totalnumber}{5}  

\renewcommand{\labelenumii}{\arabic{enumii}.}

% \vrefwarning

% https://tex.stackexchange.com/questions/54136/hyperref-link-spans-a-pagebreak-looks-ugly
% nie zawsze działa!!!

% retrieve absolute page numbers (physical pages, as opposed to the
% ‘logical’ page number that is normally typeset when a page number is
% requested;
% \usepackage{zref-abspage}

\usepackage{expex}


\lingset{glhangstyle=none}
\defineglwlevels{pismo,nr}
\newcommand{\bg}{\begingl}

% 1 height
% 2 image

%\newcommand{\PTglyph}[2]{\includegraphics[height=#1ex]{glyphs/#2}}
% \newcommand{\PTglyph}[2]{\includegraphics[height=8ex]{glyphs/#2}}
\newcommand{\PTglyph}[2]{\includegraphics[height=7ex]{glyphs/#2}}
%\newcommand{\PTglyph}[2]{\includegraphics[height=6ex]{glyphs/#2}}
\newcommand{\PTglyphid}[1]{#1}

\parindent0pt

\begin{document}
%\gappto\captionslingua{\renewcommand{\chaptername}{Caput}}
%\gappto\captionspolish{\renewcommand{\figurename}{Ilustracja}}


\title{POLONIA TYPOGRAPHICA
  SAECULI SEDECIMI\\
  {\relsize{-2} TŁOCZNIE POLSKIE XVI STULECIA\\ MONOGRAFIE I PODOBIZNY
    ZASOBÓW DRUKARSKICH}\\Summary: Reconstructed font tables\\
  (draft)}

\author{Janusz S. Bień (editor)}

\date{\today}

\maketitle

\catcode`\&=11
\catcode`\|=11
\catcode`\_=11

\def\apostrof{`}


% dodac indeks!:
% \catcode`\`=\active
% \def`#1{\fbox{{\znak#1}}}

\def\Hb#1{{\fontspec{Junicode}#1}}

\newcommand{\alfa}{\textit{alfa} (\J{α})}
\renewcommand{\beta}{\textit{beta} (\J{β})}
\renewcommand{\delta}{\textit{delta} (\J{δ})}
\renewcommand{\epsilon}{\textit{epsilon} (\J{ε})}
\renewcommand{\eta}{\textit{eta} (\J{η})}
\renewcommand{\theta}{\textit{theta} (\J{θ})}
\renewcommand{\gamma}{\textit{gamma} (\J{γ})}
\renewcommand{\chi}{\textit{chi} (\J{χ})}
\renewcommand{\kappa}{\textit{kappa} (\J{κ})}
\renewcommand{\iota}{\textit{iota} (\J{ι})}
\renewcommand{\lambda}{\textit{lambda} (\J{λ})}


\section{Introduction}
\label{sec:introduction}

For more information about \textsc{POLONIA TYPOGRAPHICA SAECULI
  SEDECIMI} please consult
e.g. \url{https://github.com/jsbien/early_fonts_inventory}.

\bigskip

\bigskip \textbf{Please remember this is a draft!} Some details still
require verification.

\subsection{CATALOGUS LIBRORUM}
\label{sec:catalogus-librorum}

By \textit{CATALOGUS LIBRORUM} we mean the list of publications
included in the booklet accompanying the plate in question. We
reference it by the item number (sometimes supplemented by a letter)
preceeded by the fascicule Roman number, e.g. `VIII:4a'.


\subsection{Haebler's font classification}
\label{sec:haebl-font-class}


Symbols such as \Hb{M¹⁶}, \Hb{M¹⁸}, \Hb{M⁴⁸}, \Hb{M⁶⁰}, \Hb{M⁹¹},
\Hb{Q|u}, \Hb{C}, \Hb{F7}, \Hb{G}, \Hb{K5} are explained in the
publication:

\begin{quote}
  Konrad Haebler:\\ «Typenrepertorium der Wiegendrucke» (the series
  \textit{Sammlung Bibliothekswissenschaftlicher Arbeiten})
  \begin{enumerate}
  \item Abteilung I: Deutschland und seine Nachbarlaender (1905).
  \item Abteilung II: Italien, die Niederlande, Frankreich, Spanien und Portugal (1908).
  \item Abteilung III:
    \begin{enumerate}
    \item Tabellen I: Antiqua-Typen (1909).
    \item Tabellen II: Gotische Typen" (1910)
    \end{enumerate}
  \item Ergänzungsband I (suplement, 1922).
  \item Ergänzungsband II (suplement, 1924).
  \end{enumerate}
\end{quote}
Reprinted in 1968, digitized in 2020 by Kujawsko-Pomorska Biblioteka
Cyfrowa (\url{https://kpbc.umk.pl/publication/222829}). Some volumes
digitized earlier by Google Books and Hathi Trust Digital Library but,
as of today, they seem available only for searching.

\noindent
Cf. also \url{https://tw.staatsbibliothek-berlin.de/}.

\subsection{Wierzbowski's bibliography}
\label{sec:wierzb-bibl}

The abbreviations in the form ``Wierzb. 891,'' refer to items (not
page numbers) in the bibliography \textit{Bibliographia Polonica XV ac
  XVII ss. quae in bibliotheca Universitatis Caesareare Varsoviensis
  asservantur} by Teodor Wierzbowski, published in 1889 and available
in the Polona digital library as
\url{https://polona.pl/item/96996417}.

\subsection{Estreicher's bibliography}
\label{sec:estr-bibl}

The abbreviations ``Estr.,'' refer Estreichers' bibliography, called
"most outstanding bibliography of Polish books, and probably one of
the most famous bibliographies in the
world".\footnote{\url{https://en.wikipedia.org/wiki/Karol_Estreicher_(senior)}}. It
was started by Karol Estrecher (1827--1908), continued by his son
Stanisław Estreicher (1869--1939) and finished by his grandson Karol
Estrecher(1906--1984).For most of the volumes the copyright has
expired, so they were reprinted, they are also (original or reprints)
available in several digital librarries,
e.g. \url{https://kpbc.umk.pl/dlibra/publication/13947} .
Additionally there is also \textit{ELEKTRONICZNA BAZA BIBLIOGRAFII
  ESTREICHERA} (EBBE), an electronic version of the
bibliographies\footnote{\url{https://www.estreicher.uj.edu.pl/}}; the
database includes also the scan of all the volumes, but there are some
restrictions on their usage.

The bibliography by Karol Estreicher (senior) consists of several
parts. The volumes have two numbers: the number in the whole
bibliography and in the specific part.
  \begin{itemize}
  \item Bibliografia polska. Cz. 1, Stólecie [!] XIX., volumes 7
  \item 
    Bibliografia polska. Cz. 2, Stólecie [!] XV-XIX spis chronologiczny. : volumes 3, one in
    two volumes (global volume numbers 8--11)
      \item 
    Bibliografia polska. Cz. 3, Stólecie [!] XV-XVIII w układzie
    abecadłowym: volumes 22 (global volume numbers 12--33)
  \item 
    Bibliografia polska. Cz. 4, Bibliografia polska XIX. stólecia [!] :
    lata 1881-1900: volumes 4
  \end{itemize}
  There are  also two unnumbered volume
  \begin{itemize}
  \item Bibliografia polska XV.-XVI. stólecia : zestawienie chronologiczne
    7200 druków w kształcie rejestru do Bibliografii, tudzież spis
    abecadłowy tych dzieł, które dochowały się w bibliotekach polskich
  \item 
    Bibliografia polska XIX. stulecia. Zeszyt dodatkowy, 1871-1873
  \end{itemize}

  The volumes authored by Stanisław Estreicher (with some use of his
  father manuscripts) include volumes 34--36 suplementing the part 3.

  The references to the bibliography contains the global volume number
  and the page number, e.g.  the reference ``Estr. XV. 242;
  XXXIV. 55'' refer to the mentions of Zaborowski's \textit{Tractatus
    contra malos divites et usurarios} which can be seen on page 242
  of volume 15\footnote{See e.g.
    \url{https://www.estreicher.uj.edu.pl/_skany/Bibliografia_Staropolska/15_Tom_XV/0257_0242.jpg}}
  and page 55 of volume 34\footnote{See e.g.
    \url{https://www.estreicher.uj.edu.pl/_skany/Bibliografia_Staropolska/34_Tom_XXXIV/0057_0055.jpg}}.

\subsection{Other abbreviations}
\label{sec:other-abbreviations}

Knihopis

Knih.
Knihopis ćeskosłovenskych tiskń od doby nejstarsi aź do konce XVIH.
stoleti. Red. Z. Tobolka. DilII. Tisky z let 1501—1800. Ć. 1 i nast. V Praze
1936 i nast.
https://www.digitalniknihovna.cz/nkp/periodical/uuid:595d5a00-baf9-11e3-b74a-5ef3fc9ae867

BM Germ.: Short-title catalogue
Short-title catalogue of book printed in the German-speaking countries
and German books printed in other countries from 1455 to 1600 now in
the British Museum. London 1962.

Bohonos Ossol.

Drukarze IV 

Boh. Ossol.
Drukarze IV

M. Bohonos: Katalog starych druków Biblioteki Zakładu Narodowego
im. Ossolińskich. Polonica wieku XVI. Z materiałów rejestracyjnych
zebranych zespołowo pod kierownictwem Kazimierza Zatheya opraco-
wała... Wrocław 1965.

Drukarze dawnej Polski od XV do XVIII wieku. T. 4: Pomorze. Oprac.
A. Kawecka-Gryczowa oraz K. Korotajowa. Wrocław 1962.

War.???

The abbreviation ``K.'' occuring in some descriptions stands for
\textit{Karty}, i.e. \textit{sheets}.

\newpage
\section{Font tables}
\label{sec:font-tables}

%%%%%%%%%%%%%%%%%%%%%%%%%%%%%%%%%%%%%%%%%%%%%%%%%%%%%%%%%%%%%%%%%%%%%%%%%%%%%%
% Tab. 01 Aleksander Augezdecki pismo 1
%%%%%%%%%%%%%%%%%%%%%%%%%%%%%%%%%%%%%%%%%%%%%%%%%%%%%%%%%%%%%%%%%%%%%%%%%%%%%%

% 01,Augezdecki-01_PT08_402.djvu,Augezdecki,01,08,402

% Author "Paulina Buchwald-Pelcowa"
% Title "Aleksander Augezdecki 1549-1561"
% Editor "Alodia Kawecka Gryczowa"
% Series "Polonia Typographica Saeculi Sedecimi: zbiór podobizn zasobu drukarskiego tłoczni polskich XVI stulecia"
% Fascicule "VIII"
% Publisher "Zakład Narodowy imienia Ossolińskich — Wydawnictwo"
% Addres "Kraków  Wrocław Warszawa"
% Year "1972"
% Note "1. Pisma tekstowe, szwabacha M⁸¹. Stopień 20 ww. = 102—103 mm (tercja). — Tabl. 402—404. [402]"
% Note1 "Character set table prepared by Paulina Buchwald-Pelcowa"
% Note2 "Scan (prepared by Biblioteka Uniwersytecka w Warszawie from their own copy) converted to DjVu with didjvu by Janusz S. Bień"
% URL "https://github.com/jsbien/early_fonts_inventory/"

\newcommand{\pismoPL}[1]{{\relsize{2}\Junicode\textbf{#1}}}
%\textbf{1. Pisma tekstowe, szwabacha M⁸¹. Stopień 20 ww. = 102—103 mm (tercja). — Tabl. 402—404.}

\pismoPL{Aleksander Augezdecki 1. Pisma tekstowe, szwabacha M⁸¹. Stopień 20 ww. = 102—103 mm (tercja). — Tabl. 402—404.}

\newcommand{\pismoEN}[1]{{\relsize{1}\Junicode\begin{quote}#1\end{quote}}}

% \begin{quote}
%   1. Schwabacher text script. Typeface M⁸¹. Type size 20 lines = 102—103
%   mm (tertia) - Plate 402-404.
% \end{quote}

\pismoEN{Aleksander Augezdecki 1. Schwabacher text script. Typeface M⁸¹. Type size 20 lines = 102—103 mm (tertia) - Plate 402-404.}

  
    \medskip

    \newcommand{\plate}[3]{\textbf{Plate #1} (fasc. #2, #3)}
    
%    \textbf{Plate 402} 

\plate{402}{VIII}{1972}
    
%     Original font table
    Prepared by Paulina Buchwald-Pelcowa.

    \medskip
    
    % Biblioteka Czartoryskich. Krakéw. P.B. P.
% 4*, [TESTAMENTUM NOVUM. Evangelium secundum Matthaeum. Trad. polon. Stanislaus Murzynowski]:
% Ewangelia Sw. Mateusza. Krélewiec, [Aleksander Augezdeckij, 1551. 4°.
% Karta LXXXIIIb.
% Estreicher XIII 26. Wierzbowski 135.
% Pismo 1: tekst i zestaw wraz z zestawem liter ze znakami diakrytycznymi polskimi. — Pismo 2: szwabacha tekstowa w marginaliach. — Pismo 3: szwabacha
% komentarzowa w marginaliach. — Pismo 7: nagtowek. — Rubryki «, 8, y, 6, 7 w zestawie — Cyfry 1 z pismem 3. — Cyfry 3 z pismem 3. — Przerywniki
% 5 z pismem 1 i 3.


      \bigskip

%        \textsc{Primary example:}

      \newcommand{\exampleBib}[1]{{\relsize{2}\Junicode\textbf{The
            example:}\\[2ex] CATALOGUS LIBRORUM \textbf{#1}}}
      \newcommand{\exampleBibExtra}[1]{{\relsize{2}\Junicode\textit{The
            plate contains also an example without a font
            table:}\\[2ex] CATALOGUS LIBRORUM \textbf{#1}}}

      \exampleBib{VIII:4a}
      \bigskip

\newcommand{\exampleDesc}[1]{{\relsize{0}\Junicode#1}}
\newcommand{\exampleDig}[1]{{\relsize{0}\Junicode \textbf{Digitization(s) [JSB]:} #1}}


\exampleDesc{[TESTAMENTUM NOVUM. Evangelium secundum Matthaeum. Trad. polon. Stanislaus Murzynowski]:
Ewangelia Sw. Mateusza. Królewiec, [Aleksander Augezdecki], 1551. 4°.}

\newcommand{\examplePage}[1]{{\relsize{0}\Junicode#1}}
\newcommand{\examplePageEN}[1]{{\relsize{0}\Junicode#1}}

\medskip
\examplePage{\textit{Karta LXXXIIIb.}}

\newcommand{\exampleLib}[1]{{\relsize{0}\Junicode \textbf{Library:} #1}}

\bigskip
\exampleLib{Biblioteka Czartoryskich. Kraków.}

\bigskip
\newcommand{\exampleRef}[1]{{\relsize{0}\Junicode \textbf{References:} #1}}

\exampleRef{\textit{Estreicher XIII 26. Wierzbowski 135.}}

\bigskip
\exampleDig{\url{https://www.dbc.wroc.pl/publication/33779} page 202.}

\newcommand{\examplePL}[1]{{\relsize{0}\Junicode#1}}

%      Pismo 1: tekst i zestaw wraz z zestawem liter ze znakami diakrytycznymi polskimi.

\bigskip

      \examplePL{Pismo 1: tekst i zestaw wraz z zestawem liter ze znakami diakrytycznymi polskimi.}
      
      \medskip

      % \begin{quote}
      %   Font 1. The text and the table including letters with Polish diacritical marks.
      % \end{quote}

\newcommand{\exampleEN}[1]{{\relsize{0}\Junicode\begin{quote}#1\end{quote}}}

\exampleEN{Font 1. The text and the table including letters with Polish diacritical marks.}


  % \textit{Testamentu Nowego Czesc Pierwsza Czterzei Euangelistowie
  %   swieći, Mattheusz, Marek, Lukasz I Ian, Z Greckiego ięzyka na
  %   Polski przelozeni i wykladem krotkiem obiasnieni}, 1551, page
  % LXXXIIIb.  Digitization
  % \url{https://www.dbc.wroc.pl/publication/33779} page 202.

  
  \bigskip
  
\newcommand{\fontID}[2]{{\relsize{1}\Junicode\textbf{Font identifier} (JSB): #1 (table #2)}}

%    \textbf{Font identifier} (JSB): Au-01 (table 01)

    \fontID{Au-01}{01}

\newcommand{\fontstat}[1]{{\relsize{1}\Junicode\textbf{Statistics} (JSB): #1 glyphs.}}

\fontstat{?}

    % \exdisplay \bg \gla
\input {t01_glyphs}
%//
%\glpismo
\glpismo
% 1
{\PTglyphid{Au-010101}}
% 2
{\PTglyphid{Au-010102}}
% 3
{\PTglyphid{Au-010103}}
% 4
{\PTglyphid{Au-010104}}
% 5
{\PTglyphid{Au-010105}}
% 6
{\PTglyphid{Au-010106}}
% 7
{\PTglyphid{Au-010107}}
% 8
{\PTglyphid{Au-010108}}
% 9
{\PTglyphid{Au-010109}}
% 10
{\PTglyphid{Au-010110}}
% 11
{\PTglyphid{Au-010111}}
% 12
{\PTglyphid{Au-010112}}
% 13
{\PTglyphid{Au-010113}}
% 14
{\PTglyphid{Au-010114}}
% 15
{\PTglyphid{Au-010115}}
% 16
{\PTglyphid{Au-010116}}
% 17
{\PTglyphid{Au-010117}}
% 18
{\PTglyphid{Au-010118}}
% 19
{\PTglyphid{Au-010119}}
% 20
{\PTglyphid{Au-010120}}
% 21
{\PTglyphid{Au-010121}}
% 22
{\PTglyphid{Au-010122}}
% 23
{\PTglyphid{Au-010123}}
% 24
{\PTglyphid{Au-010124}}
% 25
{\PTglyphid{Au-010125}}
% 26
{\PTglyphid{Au-010126}}
% 27
{\PTglyphid{Au-010127}}
% 28
{\PTglyphid{Au-010128}}
% 29
{\PTglyphid{Au-010201}}
% 30
{\PTglyphid{Au-010202}}
% 31
{\PTglyphid{Au-010203}}
% 32
{\PTglyphid{Au-010204}}
% 33
{\PTglyphid{Au-010205}}
% 34
{\PTglyphid{Au-010206}}
% 35
{\PTglyphid{Au-010207}}
% 36
{\PTglyphid{Au-010208}}
% 37
{\PTglyphid{Au-010209}}
% 38
{\PTglyphid{Au-010210}}
% 39
{\PTglyphid{Au-010211}}
% 40
{\PTglyphid{Au-010212}}
% 41
{\PTglyphid{Au-010213}}
% 42
{\PTglyphid{Au-010214}}
% 43
{\PTglyphid{Au-010215}}
% 44
{\PTglyphid{Au-010216}}
% 45
{\PTglyphid{Au-010217}}
% 46
{\PTglyphid{Au-010218}}
% 47
{\PTglyphid{Au-010219}}
% 48
{\PTglyphid{Au-010220}}
% 49
{\PTglyphid{Au-010221}}
% 50
{\PTglyphid{Au-010222}}
% 51
{\PTglyphid{Au-010223}}
% 52
{\PTglyphid{Au-010224}}
% 53
{\PTglyphid{Au-010225}}
% 54
{\PTglyphid{Au-010226}}
% 55
{\PTglyphid{Au-010227}}
% 56
{\PTglyphid{Au-010228}}
% 57
{\PTglyphid{Au-010229}}
% 58
{\PTglyphid{Au-010230}}
% 59
{\PTglyphid{Au-010231}}
% 60
{\PTglyphid{Au-010232}}
% 61
{\PTglyphid{Au-010233}}
% 62
{\PTglyphid{Au-010234}}
% 63
{\PTglyphid{Au-010235}}
% 64
{\PTglyphid{Au-010236}}
% 65
{\PTglyphid{Au-010237}}
% 66
{\PTglyphid{Au-010238}}
% 67
{\PTglyphid{Au-010239}}
% 68
{\PTglyphid{Au-010240}}
% 69
{\PTglyphid{Au-010241}}
% 70
{\PTglyphid{Au-010301}}
% 71
{\PTglyphid{Au-010302}}
% 72
{\PTglyphid{Au-010303}}
% 73
{\PTglyphid{Au-010304}}
% 74
{\PTglyphid{Au-010305}}
% 75
{\PTglyphid{Au-010306}}
% 76
{\PTglyphid{Au-010307}}
% 77
{\PTglyphid{Au-010308}}
% 78
{\PTglyphid{Au-010309}}
% 79
{\PTglyphid{Au-010310}}
% 80
{\PTglyphid{Au-010311}}
% 81
{\PTglyphid{Au-010312}}
% 82
{\PTglyphid{Au-010313}}
% 83
{\PTglyphid{Au-010314}}
% 84
{\PTglyphid{Au-010315}}
% 85
{\PTglyphid{Au-010316}}
% 86
{\PTglyphid{Au-010317}}
% 87
{\PTglyphid{Au-010318}}
% 88
{\PTglyphid{Au-010319}}
% 89
{\PTglyphid{Au-010320}}
% 90
{\PTglyphid{Au-010321}}
% 91
{\PTglyphid{Au-010322}}
% 92
{\PTglyphid{Au-010323}}
% 93
{\PTglyphid{Au-010324}}
% 94
{\PTglyphid{Au-010325}}
% 95
{\PTglyphid{Au-010326}}
% 96
{\PTglyphid{Au-010327}}
% 97
{\PTglyphid{Au-010328}}
% 98
{\PTglyphid{Au-010329}}
% 99
{\PTglyphid{Au-010330}}
% 100
{\PTglyphid{Au-010331}}
% 101
{\PTglyphid{Au-010332}}
% 102
{\PTglyphid{Au-010333}}
% 103
{\PTglyphid{Au-010334}}
% 104
{\PTglyphid{Au-010335}}
% 105
{\PTglyphid{Au-010336}}
% 106
{\PTglyphid{Au-010337}}
% 107
{\PTglyphid{Au-010339}}
% 108
{\PTglyphid{Au-010340}}
% 109
{\PTglyphid{Au-010341}}
% 110
{\PTglyphid{Au-010342}}
% 111
{\PTglyphid{Au-010343}}
% 112
{\PTglyphid{Au-010344}}
% 113
{\PTglyphid{Au-010401}}
% 114
{\PTglyphid{Au-010402}}
% 115
{\PTglyphid{Au-010403}}
% 116
{\PTglyphid{Au-010404}}
% 117
{\PTglyphid{Au-010405}}
% 118
{\PTglyphid{Au-010406}}
% 119
{\PTglyphid{Au-010407}}
% 120
{\PTglyphid{Au-010408}}
% 121
{\PTglyphid{Au-010409}}
% 122
{\PTglyphid{Au-010410}}
% 123
{\PTglyphid{Au-010411}}
% 124
{\PTglyphid{Au-010412}}
% 125
{\PTglyphid{Au-010413}}
% 126
{\PTglyphid{Au-010414}}
% 127
{\PTglyphid{Au-010415}}
% 128
{\PTglyphid{Au-010416}}
% 129
{\PTglyphid{Au-010417}}
% 130
{\PTglyphid{Au-010418}}
% 131
{\PTglyphid{Au-010419}}
% 132
{\PTglyphid{Au-010420}}
% 133
{\PTglyphid{Au-010421}}
% 134
{\PTglyphid{Au-010422}}
% 135
{\PTglyphid{Au-010423}}
% 136
{\PTglyphid{Au-010424}}
% 137
{\PTglyphid{Au-010425}}
% 138
{\PTglyphid{Au-010501}}
% 139
{\PTglyphid{Au-010502}}
% 140
{\PTglyphid{Au-010503}}
% 141
{\PTglyphid{Au-010504}}
% 142
{\PTglyphid{Au-010505}}
% 143
{\PTglyphid{Au-010506}}
% 144
{\PTglyphid{Au-010507}}
% 145
{\PTglyphid{Au-010508}}
% 146
{\PTglyphid{Au-010509}}
% 147
{\PTglyphid{Au-010510}}
% 148
{\PTglyphid{Au-010511}}
% 149
{\PTglyphid{Au-010512}}
% 150
{\PTglyphid{Au-010513}}
% 151
{\PTglyphid{Au-010514}}
% 152
{\PTglyphid{Au-010515}}
% 153
{\PTglyphid{Au-010516}}
% 154
{\PTglyphid{Au-010517}}
% 155
{\PTglyphid{Au-010518}}
% 156
{\PTglyphid{Au-010519}}
% 157
{\PTglyphid{Au-010520}}
% 158
{\PTglyphid{Au-010522}}
% 159
{\PTglyphid{Au-010523}}
% 160
{\PTglyphid{Au-010524}}
% 161
{\PTglyphid{Au-010525}}
% 162
{\PTglyphid{Au-010526}}
% 163
{\PTglyphid{Au-010527}}
% 164
{\PTglyphid{Au-010528}}
% 165
{\PTglyphid{Au-010529}}
% 166
{\PTglyphid{Au-010530}}
% 167
{\PTglyphid{Au-010531}}
% 168
{\PTglyphid{Au-010532}}
% 169
{\PTglyphid{Au-010533}}
% 170
{\PTglyphid{Au-010534}}
% 171
{\PTglyphid{Au-010535}}
% 172
{\PTglyphid{Au-010536}}
% 173
{\PTglyphid{Au-010537}}
% 174
{\PTglyphid{Au-010538}}
% 175
{\PTglyphid{Au-010539}}
% 176
{\PTglyphid{Au-010540}}
% 177
{\PTglyphid{Au-010541}}
% 178
{\PTglyphid{Au-010542}}
% 179
{\PTglyphid{Au-010601}}
% 180
{\PTglyphid{Au-010602}}
% 181
{\PTglyphid{Au-010603}}
% 182
{\PTglyphid{Au-010604}}
% 183
{\PTglyphid{Au-010605}}
% 184
{\PTglyphid{Au-010606}}
% 185
{\PTglyphid{Au-010607}}
% 186
{\PTglyphid{Au-010608}}
% 187
{\PTglyphid{Au-010609}}
% 188
{\PTglyphid{Au-010610}}
//
%%% Local Variables:
%%% mode: latex
%%% TeX-engine: luatex
%%% TeX-master: shared
%%% End:

% //
%\endgl \xe

%\end{flushleft}

\newpage
%%%%%%%%%%%%%%%%%%%%%%%%%%%%%%%%%%%%%%%%%%%%%%%%%%%%%%%%%%%%%%%%%%%%%%%%%%%%%%
% Tab. 02 Aleksander Augezdecki pismo 1a
%%%%%%%%%%%%%%%%%%%%%%%%%%%%%%%%%%%%%%%%%%%%%%%%%%%%%%%%%%%%%%%%%%%%%%%%%%%%%%

% Author "Paulina Buchwald-Pelcowa"
% Title "Aleksander Augezdecki 1549-1561"
% Editor "Alodia Kawecka Gryczowa"
% Series "Polonia Typographica Saeculi Sedecimi: zbiór podobizn zasobu drukarskiego tłoczni polskich XVI stulecia"
% Fascicule "VIII"
% Publisher "Zakład Narodowy imienia Ossolińskich — Wydawnictwo"
% Addres "Kraków  Wrocław Warszawa"
% Year "1972"
% Note "1. Pisma tekstowe, szwabacha M⁸¹. Stopień 20 ww. = 102—103 mm (tercja). — Tabl. 402—404. [403]"
% Note1 "Character set table prepared by Paulina Buchwald-Pelcowa"
% Note2 "Scan (prepared by Biblioteka Uniwersytecka w Warszawie from their own copy) converted to DjVu with didjvu by Janusz S. Bień"
    
\pismoPL{Aleksander Augezdecki 1. Pisma tekstowe, szwabacha M⁸¹. Stopień 20 ww. = 102—103 mm (tercja). — Tabl. 402—404.}

\pismoEN{Aleksander Augezdecki 1. Schwabacher text script. Typeface M⁸¹. Type size 20 lines = 102—103 mm (tertia) - Plate 402-404.}

\medskip

\plate{403}{VIII}{1972}

Prepared by Paulina Buchwald-Pelcowa.

\bigskip

\exampleBib{VIII:25}

\medskip
\bigskip

\exampleDesc{PIESNE Chwal Bozskych. Szamotuly, Aleksander Augezdecki, [25 I 1560—] 7 VI 1561. 2°. War. A.}

\medskip
\examplePage{\textit{Karta *₂b}}

\bigskip
\exampleLib{Biblioteka Czartoryskich. Kraków.}

\bigskip
\exampleRef{\textit{Estreicher XIX 91. Knihopis 12860.}}

\bigskip
\exampleDig{\url{https://cyfrowe.mnk.pl/dlibra/publication/13639/}, page 8.}

    \examplePL{Pismo 1: tekst i zestaw liter ze znakami diakrytycznymi czeskimi.}

    \medskip

    \exampleEN{Font 1. The text and the table of letters with Czech diacritical marks}

\bigskip

    \fontID{Au-01a}{02}

\fontstat{?}

\bigskip

% \exdisplay \bg \gla
\input {t02_glyphs.tex}
%//
%\glpismo
\glpismo
% 1
{\PTglyphid{Au-020101}}
% 2
{\PTglyphid{Au-020102}}
% 3
{\PTglyphid{Au-020103}}
% 4
{\PTglyphid{Au-020104}}
% 5
{\PTglyphid{Au-020105}}
% 6
{\PTglyphid{Au-020106}}
% 7
{\PTglyphid{Au-020107}}
% 8
{\PTglyphid{Au-020108}}
% 9
{\PTglyphid{Au-020109}}
% 10
{\PTglyphid{Au-020110}}
% 11
{\PTglyphid{Au-020201}}
% 12
{\PTglyphid{Au-020202}}
% 13
{\PTglyphid{Au-020203}}
% 14
{\PTglyphid{Au-020204}}
% 15
{\PTglyphid{Au-020205}}
% 16
{\PTglyphid{Au-020206}}
% 17
{\PTglyphid{Au-020207}}
% 18
{\PTglyphid{Au-020208}}
% 19
{\PTglyphid{Au-020210}}
% 20
{\PTglyphid{Au-020211}}
% 21
{\PTglyphid{Au-020212}}
% 22
{\PTglyphid{Au-020213}}
% 23
{\PTglyphid{Au-020214}}
% 24
{\PTglyphid{Au-020215}}
% 25
{\PTglyphid{Au-020216}}
% 26
{\PTglyphid{Au-020217}}
% 27
{\PTglyphid{Au-020219}}
% 28
{\PTglyphid{Au-020220}}
% 29
{\PTglyphid{Au-020221}}
% 30
{\PTglyphid{Au-020222}}
% 31
{\PTglyphid{Au-020223}}
% 32
{\PTglyphid{Au-020224}}
% 33
{\PTglyphid{Au-020225}}
% 34
{\PTglyphid{Au-020226}}
% 35
{\PTglyphid{Au-020227}}
% 36
{\PTglyphid{Au-020228}}
% 37
{\PTglyphid{Au-020229}}
% 38
{\PTglyphid{Au-020230}}
% 39
{\PTglyphid{Au-020231}}
% 40
{\PTglyphid{Au-020232}}
% 41
{\PTglyphid{Au-020233}}
% 42
{\PTglyphid{Au-020234}}
% 43
{\PTglyphid{Au-020235}}
//
%%% Local Variables:
%%% mode: latex
%%% TeX-engine: luatex
%%% TeX-master: shared
%%% End:

% //
%\endgl \xe


\newpage
%%%%%%%%%%%%%%%%%%%%%%%%%%%%%%%%%%%%%%%%%%%%%%%%%%%%%%%%%%%%%%%%%%%%%%%%%%%%%%
% Tab. 03 Aleksander Augezdecki pismo 1b
%%%%%%%%%%%%%%%%%%%%%%%%%%%%%%%%%%%%%%%%%%%%%%%%%%%%%%%%%%%%%%%%%%%%%%%%%%%%%%

\pismoPL{Aleksander Augezdecki 1. Pisma tekstowe, szwabacha M⁸¹. Stopień 20 ww. = 102—103 mm (tercja). — Tabl. 402—404.}

\pismoEN{Aleksander Augezdecki 1. Schwabacher text script. Typeface M⁸¹. Type size 20 lines = 102—103 mm (tertia) - Plate 402-404.}

\medskip

\plate{404}{VIII}{1972}

Prepared by Paulina Buchwald-Pelcowa.

\bigskip

\exampleBib{VIII:12}

\bigskip
\exampleDesc{CHRISTOPHORUS RUDOLFF: Die Coss. 4°. Krélewiec, Aleksander Augezdecki, 1553. 4°.}

\medskip
\examplePage{\textit{Karta 63a.}}

  \bigskip
\exampleLib{Biblioteka Czartoryskich. Kraków.}

\bigskip
\exampleRef{\textit{BM Germ.: Short-title catalogue s. 759}}

\bigskip
\exampleDig{\url{https://ds.ub.uni-bielefeld.de/viewer/api/v1/records/2014414/sections/LOG_0000/pdf/}}

\medskip

    \examplePL{Pismo 1: tekst i zestaw liter ze znakami diakrytycznymi niemieckimi.}

    \medskip

    \exampleEN{Font 1. The text and the table of letters with German diacritical marks [and mathematical notation]}

      \bigskip

%https://www.unicode.org/L2/L2024/24141-n5277-leibniz.pdf
%      Biblioteka Czartoryskich. Krakéw.
% P.B.P. 
% 12. CHRISTOPHORUS RUDOLFF: Die Coss. 4°. Krélewiec, Aleksander Augezdecki, 1553. 4°.
% Pismo 1: tekst i zestaw liter ze znakami diakrytycznymi niemieckimi. — Cyfry 4 z pismem 1.
% Karta 63a.
% BM Germ.: Short-title catalogue s. 759,

% https://www.unicode.org/L2/L2024/24141-n5277-leibniz.pdf
% https://www.researchgate.net/publication/230735404_From_the_second_unknown_to_the_symbolic_equation

%       12 karta 63a

%       1553
% CHRISTOPHORUS RUDOLFF: Die Coss. Ed.
% Michael Stifel. 4*. K. 104. (Arkusze A—Z Aa— Cc) —
% BM Germ. 759. Arkusze Dd—Zz Aaa—Zzz
% Aaaa—Zzzz Aaaaa—Zzzzz Aaaaaa—LlIII zob.
% poz. 138. *
% W niektórych egzemplarzach różnice w foliacji i sygnacji. Egzem-
% plarz B. Nar. XVI. Qu 115 k. ... 90,9, 92, 93, 95, 95—212...; egzem-
% płarz B. Czart. 52811 II k. ... 89—93, 95, 95—212..., karta Ca
% błędnie sygnowana B;.
% Pisma 1, 4—6, 13, 15. — Rubryka a, $, y. — Cyfry 1—4. —
% Przerywnik 2. — Inicjały 3, 15, 16, 21, 22, 27. — Drzeworyty
% 20, 21—23. - Li2.
      
% Digitization %%%%%which variant????
% https://ds.ub.uni-bielefeld.de/viewer/image/2014414/1/LOG_0000/
% https://old.maa.org/press/periodicals/convergence/mathematical-treasures-rudolffs-arithmetic-and-algebra
% https://old.maa.org/sites/default/files/images/upload_library/46/Swetz_2012_Math_Treasures/ColumbiaU/1302100032.png

\bigskip

    \fontID{Au-01b}{03}

    \fontstat{?}

    \bigskip
% \exdisplay \bg \gla
\input {t03_glyphs.tex}
%//
%\glpismo
\input {t03_glyphids.tex}
% //
%\endgl \xe


\newpage
%%%%%%%%%%%%%%%%%%%%%%%%%%%%%%%%%%%%%%%%%%%%%%%%%%%%%%%%%%%%%%%%%%%%%%%%%%%%%%
% Tab. 04 Aleksander Augezdecki pismo 2
%%%%%%%%%%%%%%%%%%%%%%%%%%%%%%%%%%%%%%%%%%%%%%%%%%%%%%%%%%%%%%%%%%%%%%%%%%%%%%

% Author "Paulina Buchwald-Pelcowa"
% Title "Aleksander Augezdecki 1549-1561"
% Editor "Alodia Kawecka Gryczowa"
% Series "Polonia Typographica Saeculi Sedecimi: zbiór podobizn zasobu drukarskiego tłoczni polskich XVI stulecia"
% Fascicule "VIII"
% Publisher "Zakład Narodowy imienia Ossolińskich — Wydawnictwo"
% Addres "Kraków  Wrocław Warszawa"
% Year "1972"
% Note "2 Pismo tekstowe, szwabacha M⁸¹. Stopień 20 ww. = 86—87 mm (cycero). — Tabl. 405, 406. [406]"
% Note1 "Character set table prepared by Paulina Buchwald-Pelcowa"

\pismoPL{Aleksander Augezdecki 2. Pisma tekstowe, szwabacha M⁸¹. Stopień 20 ww. = 86—87 mm (cycero). — Tabl. 402—404.}

\pismoEN{Aleksander Augezdecki 2. Schwabacher text script. Typeface M⁸¹. Type size 20 lines = 86—87 mm (cicero) - Plate 402-404.}

\medskip

\plate{405}{VIII}{1972}

Prepared by Paulina Buchwald-Pelcowa.

\bigskip

\exampleBib{VIII:6}

\bigskip
\exampleDesc{[TESTAMENTUM NOVUM. Trad. polon. Stanislaus Murzynowski]: Testamentu Nowego część pierwsza.
Królewiec, Aleksander Augezdecki, X 1551. 4°.}

\medskip
\examplePage{\textit{Karta B₁a.}}

  \bigskip
\exampleLib{Biblioteka Czartoryskich. Kraków.}

\bigskip
\exampleRef{\textit{Estreicher XIII 26, Wierzbowski 1288.}}

\bigskip
\exampleDig{\url{https://www.dbc.wroc.pl/publication/33779} page 13.}


\medskip

    \examplePL{Pismo 2: tekst i zestaw wraz z zestawem liter ze znakami diakrytycznymi polskimi.}

    \medskip

    \exampleEN{Font 2. The text and the table including letters with Polish diacritical marks.}


\bigskip

    \fontID{Au-02}{04}

    \fontstat{?}

\bigskip

% \exdisplay \bg \gla
\input {t04_glyphs.tex}
%//
%\glpismo
\glpismo
% 1
{\PTglyphid{Au-020101}}
% 2
{\PTglyphid{Au-020102}}
% 3
{\PTglyphid{Au-020103}}
% 4
{\PTglyphid{Au-020104}}
% 5
{\PTglyphid{Au-020105}}
% 6
{\PTglyphid{Au-020106}}
% 7
{\PTglyphid{Au-020107}}
% 8
{\PTglyphid{Au-020108}}
% 9
{\PTglyphid{Au-020109}}
% 10
{\PTglyphid{Au-020110}}
% 11
{\PTglyphid{Au-020111}}
% 12
{\PTglyphid{Au-020112}}
% 13
{\PTglyphid{Au-020113}}
% 14
{\PTglyphid{Au-020114}}
% 15
{\PTglyphid{Au-020115}}
% 16
{\PTglyphid{Au-020116}}
% 17
{\PTglyphid{Au-020117}}
% 18
{\PTglyphid{Au-020118}}
% 19
{\PTglyphid{Au-020119}}
% 20
{\PTglyphid{Au-020120}}
% 21
{\PTglyphid{Au-020121}}
% 22
{\PTglyphid{Au-020122}}
% 23
{\PTglyphid{Au-020123}}
% 24
{\PTglyphid{Au-020124}}
% 25
{\PTglyphid{Au-020125}}
% 26
{\PTglyphid{Au-020126}}
% 27
{\PTglyphid{Au-020127}}
% 28
{\PTglyphid{Au-020128}}
% 29
{\PTglyphid{Au-020201}}
% 30
{\PTglyphid{Au-020202}}
% 31
{\PTglyphid{Au-020203}}
% 32
{\PTglyphid{Au-020204}}
% 33
{\PTglyphid{Au-020205}}
% 34
{\PTglyphid{Au-020206}}
% 35
{\PTglyphid{Au-020207}}
% 36
{\PTglyphid{Au-020208}}
% 37
{\PTglyphid{Au-020209}}
% 38
{\PTglyphid{Au-020210}}
% 39
{\PTglyphid{Au-020211}}
% 40
{\PTglyphid{Au-020212}}
% 41
{\PTglyphid{Au-020213}}
% 42
{\PTglyphid{Au-020214}}
% 43
{\PTglyphid{Au-020215}}
% 44
{\PTglyphid{Au-020216}}
% 45
{\PTglyphid{Au-020217}}
% 46
{\PTglyphid{Au-020218}}
% 47
{\PTglyphid{Au-020219}}
% 48
{\PTglyphid{Au-020220}}
% 49
{\PTglyphid{Au-020221}}
% 50
{\PTglyphid{Au-020222}}
% 51
{\PTglyphid{Au-020223}}
% 52
{\PTglyphid{Au-020224}}
% 53
{\PTglyphid{Au-020225}}
% 54
{\PTglyphid{Au-020226}}
% 55
{\PTglyphid{Au-020227}}
% 56
{\PTglyphid{Au-020228}}
% 57
{\PTglyphid{Au-020229}}
% 58
{\PTglyphid{Au-020230}}
% 59
{\PTglyphid{Au-020231}}
% 60
{\PTglyphid{Au-020232}}
% 61
{\PTglyphid{Au-020233}}
% 62
{\PTglyphid{Au-020234}}
% 63
{\PTglyphid{Au-020235}}
% 64
{\PTglyphid{Au-020236}}
% 65
{\PTglyphid{Au-020237}}
% 66
{\PTglyphid{Au-020238}}
% 67
{\PTglyphid{Au-020239}}
% 68
{\PTglyphid{Au-020240}}
% 69
{\PTglyphid{Au-020241}}
% 70
{\PTglyphid{Au-020242}}
% 71
{\PTglyphid{Au-020301}}
% 72
{\PTglyphid{Au-020302}}
% 73
{\PTglyphid{Au-020303}}
% 74
{\PTglyphid{Au-020304}}
% 75
{\PTglyphid{Au-020305}}
% 76
{\PTglyphid{Au-020306}}
% 77
{\PTglyphid{Au-020307}}
% 78
{\PTglyphid{Au-020308}}
% 79
{\PTglyphid{Au-020309}}
% 80
{\PTglyphid{Au-020310}}
% 81
{\PTglyphid{Au-020311}}
% 82
{\PTglyphid{Au-020312}}
% 83
{\PTglyphid{Au-020313}}
% 84
{\PTglyphid{Au-020314}}
% 85
{\PTglyphid{Au-020315}}
% 86
{\PTglyphid{Au-020316}}
% 87
{\PTglyphid{Au-020317}}
% 88
{\PTglyphid{Au-020318}}
% 89
{\PTglyphid{Au-020319}}
% 90
{\PTglyphid{Au-020320}}
% 91
{\PTglyphid{Au-020321}}
% 92
{\PTglyphid{Au-020322}}
% 93
{\PTglyphid{Au-020323}}
% 94
{\PTglyphid{Au-020401}}
% 95
{\PTglyphid{Au-020402}}
% 96
{\PTglyphid{Au-020403}}
% 97
{\PTglyphid{Au-020404}}
% 98
{\PTglyphid{Au-020405}}
% 99
{\PTglyphid{Au-020406}}
% 100
{\PTglyphid{Au-020407}}
% 101
{\PTglyphid{Au-020408}}
% 102
{\PTglyphid{Au-020409}}
% 103
{\PTglyphid{Au-020410}}
% 104
{\PTglyphid{Au-020411}}
% 105
{\PTglyphid{Au-020412}}
% 106
{\PTglyphid{Au-020413}}
% 107
{\PTglyphid{Au-020414}}
% 108
{\PTglyphid{Au-020415}}
% 109
{\PTglyphid{Au-020416}}
% 110
{\PTglyphid{Au-020417}}
% 111
{\PTglyphid{Au-020418}}
% 112
{\PTglyphid{Au-020419}}
% 113
{\PTglyphid{Au-020420}}
% 114
{\PTglyphid{Au-020421}}
% 115
{\PTglyphid{Au-020422}}
% 116
{\PTglyphid{Au-020423}}
% 117
{\PTglyphid{Au-020424}}
% 118
{\PTglyphid{Au-020425}}
% 119
{\PTglyphid{Au-020426}}
% 120
{\PTglyphid{Au-020427}}
% 121
{\PTglyphid{Au-020428}}
% 122
{\PTglyphid{Au-020429}}
% 123
{\PTglyphid{Au-020501}}
% 124
{\PTglyphid{Au-020502}}
% 125
{\PTglyphid{Au-020503}}
% 126
{\PTglyphid{Au-020504}}
% 127
{\PTglyphid{Au-020505}}
% 128
{\PTglyphid{Au-020506}}
% 129
{\PTglyphid{Au-020507}}
% 130
{\PTglyphid{Au-020508}}
% 131
{\PTglyphid{Au-020509}}
% 132
{\PTglyphid{Au-020510}}
% 133
{\PTglyphid{Au-020511}}
% 134
{\PTglyphid{Au-020512}}
% 135
{\PTglyphid{Au-020513}}
% 136
{\PTglyphid{Au-020514}}
% 137
{\PTglyphid{Au-020515}}
% 138
{\PTglyphid{Au-020516}}
% 139
{\PTglyphid{Au-020517}}
% 140
{\PTglyphid{Au-020518}}
% 141
{\PTglyphid{Au-020519}}
% 142
{\PTglyphid{Au-020520}}
% 143
{\PTglyphid{Au-020521}}
% 144
{\PTglyphid{Au-020522}}
% 145
{\PTglyphid{Au-020523}}
% 146
{\PTglyphid{Au-020524}}
% 147
{\PTglyphid{Au-020525}}
% 148
{\PTglyphid{Au-020526}}
% 149
{\PTglyphid{Au-020527}}
% 150
{\PTglyphid{Au-020528}}
% 151
{\PTglyphid{Au-020529}}
% 152
{\PTglyphid{Au-020530}}
% 153
{\PTglyphid{Au-020531}}
% 154
{\PTglyphid{Au-020532}}
% 155
{\PTglyphid{Au-020533}}
% 156
{\PTglyphid{Au-020534}}
% 157
{\PTglyphid{Au-020535}}
% 158
{\PTglyphid{Au-020536}}
//
\endgl \xe
%%% Local Variables:
%%% mode: latex
%%% TeX-engine: luatex
%%% TeX-master: shared
%%% End:

% //
%\endgl \xe

%   *[TESTAMENTUM NOVUM. Evangelium secundum Matthaeum. Trad. polon. Stanislaus Murzynowski]: Ewangelia św. Mateusza. 4�. K. 104.
% (Arkusze A—B A—B A—Y). Druk czarno-czerwony. — Estr. XIII 26. Wierzb. 135. Arkusze
% Z aa—kk zob. poz. 4.
% W egz. B. Nar. XVI Qu. 6471 w pierwszym marginalium na lewym
% marginesie w. 4, zamiast majuskuły A wydrukowano minuskułę a.
% Pisma 1—3, 5—7, 10—12. — Cyfry 1—3, 5. — Przerywnik I, 5. —
% Inicjały 1, 3—8, 28.
% Warsz. B. U. 28.2.4.4. z notatkami J. Maleckiego. [4�


\newpage
%%%%%%%%%%%%%%%%%%%%%%%%%%%%%%%%%%%%%%%%%%%%%%%%%%%%%%%%%%%%%%%%%%%%%%%%%%%%%%
% Tab. 05 Aleksander Augezdecki pismo 2a
%%%%%%%%%%%%%%%%%%%%%%%%%%%%%%%%%%%%%%%%%%%%%%%%%%%%%%%%%%%%%%%%%%%%%%%%%%%%%%

 % Author "Paulina Buchwald-Pelcowa"
% Title "Aleksander Augezdecki 1549-1561"
% Editor "Alodia Kawecka Gryczowa"
% Series "Polonia Typographica Saeculi Sedecimi: zbiór podobizn zasobu drukarskiego tłoczni polskich XVI stulecia"
% Fascicule "VIII"
% Publisher "Zakład Narodowy imienia Ossolińskich — Wydawnictwo"
% Addres "Kraków  Wrocław Warszawa"
% Year "1972"
% Note "2 Pismo tekstowe, szwabacha M⁸¹. Stopień 20 ww. = 86—87 mm (cycero). — Tabl. 405, 406. [405]"
% Note1 "Character set table prepared by Paulina Buchwald-Pelcowa"
% Note2 "Scan (prepared by Biblioteka Uniwersytecka w Warszawie from their own copy) converted to DjVu with didjvu by Janusz S. Bień"
% URL "https://github.com/jsbien/early_fonts_inventory/"
% .

\pismoPL{Aleksander Augezdecki 2. Pisma tekstowe, szwabacha M⁸¹. Stopień 20 ww. = 86—87 mm (cycero). — Tabl. 405, 406.}

\pismoEN{Aleksander Augezdecki 2. Schwabacher text script. Typeface M⁸¹. Type size 20 lines = 86—87 mm (cicero) - Plate 405, 406.}

\medskip

\plate{406}{VIII}{1972}

Prepared by Paulina Buchwald-Pelcowa.

\bigskip

\exampleBib{VIII:25}

\bigskip
\exampleDesc{PIESNĚ Chwal Bożských. Szamotuly, Aleksander Augezdecki, [25 I 1560—] 7 VI 1561. 2°. War. A.}

\medskip
\examplePage{\textit{Karta *₄a.}}

  \bigskip
\exampleLib{Biblioteka Czartoryskich. Kraków.}

\bigskip
\exampleRef{\textit{Estreicher XIX 91. Knihopis 12860.}}

\bigskip
\exampleDig{\url{https://cyfrowe.mnk.pl/dlibra/publication/13639/}, page 10.}

\medskip

    \examplePL{Pismo 2: tekst i zestaw liter ze znakami diakrytycznymi czeskimi.}

    \medskip

    \exampleEN{Font 2. The text and the table including letters with Czech diacritical marks.}


\bigskip

    \fontID{Au-02a}{05}

    \fontstat{?}

\bigskip


% \exdisplay \bg \gla
\input {t05_glyphs.tex}
%//
%\glpismo
\input {t05_glyphids.tex}
% //
%\endgl \xe

\newpage
%%%%%%%%%%%%%%%%%%%%%%%%%%%%%%%%%%%%%%%%%%%%%%%%%%%%%%%%%%%%%%%%%%%%%%%%%%%%%%
% Tab. 06 Aleksander Augezdecki pismo 3
%%%%%%%%%%%%%%%%%%%%%%%%%%%%%%%%%%%%%%%%%%%%%%%%%%%%%%%%%%%%%%%%%%%%%%%%%%%%%%

% Note "3. Pismo komentarzowe, szwabacha M⁸¹ + M⁸⁷. Stopień 20 ww. = 67 mm (garmond). — Tabl. 407."
% Note1 "Character set table prepared by Maria Błońska"

\pismoPL{Aleksander Augezdecki 3. Pismo komentarzowe, szwabacha M⁸¹ + M⁸⁷. Stopień 20 ww. = 67 mm (garmond). — Tabl. 407.}

\pismoEN{Aleksander Augezdecki 3. Schwabacher comment script. Typeface M⁸¹ + M⁸⁷. Type size 20 lines = 67 mm (garmond). - Plate 407.}

\medskip

\plate{407}{VIII}{1972}

Prepared by Paulina Buchwald-Pelcowa.\\
The font table prepared by Paulina Buchwald-Pelcowa and Maria Błońska.

\bigskip

 \exampleBib{VIII:4$^a$}

\bigskip
\exampleDesc{[TESTAMENTUM NOVUM. Evangelium secundum Matthaeum. Trad. polon. Stanislaus Murzynowski]:
Ewangelia Św. Mateusza. Królewiec, [Aleksander Augezdecki], 1551. 4°.}

\medskip
\examplePage{\textit{Karta B₂a.}}

  \bigskip
\exampleLib{Biblioteka Czartoryskich. Kraków.}

\bigskip
\exampleRef{\textit{Estreicher XIII 26. Wierzbowski 135.}}

\bigskip
\exampleDig{\url{https://crispa.uw.edu.pl/object/files/318618/display/Default} page 18}

\medskip

    \examplePL{Pismo 3: tekst i zestaw.}

    \medskip

    \exampleEN{Font 3. The text and the font table}


\bigskip

    \fontID{Au-03}{06}

    \fontstat{?}

\bigskip


% \exdisplay \bg \gla
 \input {t06_glyphs.tex}
%//
%\glpismo
 \input {t06_glyphids.tex}
% //
%\endgl \xe

\newpage
%%%%%%%%%%%%%%%%%%%%%%%%%%%%%%%%%%%%%%%%%%%%%%%%%%%%%%%%%%%%%%%%%%%%%%%%%%%%%%
% BRAK w PT tabeli pisma 4!
% Tab. 07 Aleksander Augezdecki pismo 5
%%%%%%%%%%%%%%%%%%%%%%%%%%%%%%%%%%%%%%%%%%%%%%%%%%%%%%%%%%%%%%%%%%%%%%%%%%%%%%

% Note "5. Pismo nagłówkowe, tekstura M³⁰. Stopień I w. = 9 mm. — Tabl. 408."
% Note1 "Character set table prepared by Paulina Buchwald-Pelcowa"

\pismoPL{Aleksander Augezdecki 5. Pismo nagłówkowe, tekstura M³⁰. Stopień 1 w. = 9 mm. — Tabl. 408.}

\pismoEN{Aleksander Augezdecki 5. Display font, Textura M³⁰. Type size 1 line = 9 mm. - Plate 408.}
% https://www.adfontes.uzh.ch/en/tutorium/schriften-lesen/schriftgeschichte/gotische-minuskeln-textura-und-textualis/
\medskip

\plate{408}{VIII}{1972}

Prepared by Paulina Buchwald-Pelcowa [layout confusing, misinterpretation possible --- JSB].\\

\bigskip

\fontID{Au-05}{07}

    \fontstat{97}

% \exdisplay \bg \gla
 \input {t07_glyphs.tex}
%//
%\glpismo
 \glpismo
% 1
{\PTglyphid{Au-05_0101}}
% 2
{\PTglyphid{Au-05_0102}}
% 3
{\PTglyphid{Au-05_0103}}
% 4
{\PTglyphid{Au-05_0104}}
% 5
{\PTglyphid{Au-05_0105}}
% 6
{\PTglyphid{Au-05_0106}}
% 7
{\PTglyphid{Au-05_0107}}
% 8
{\PTglyphid{Au-05_0108}}
% 9
{\PTglyphid{Au-05_0109}}
% 10
{\PTglyphid{Au-05_0110}}
% 11
{\PTglyphid{Au-05_0111}}
% 12
{\PTglyphid{Au-05_0112}}
% 13
{\PTglyphid{Au-05_0113}}
% 14
{\PTglyphid{Au-05_0114}}
% 15
{\PTglyphid{Au-05_0115}}
% 16
{\PTglyphid{Au-05_0116}}
% 17
{\PTglyphid{Au-05_0117}}
% 18
{\PTglyphid{Au-05_0118}}
% 19
{\PTglyphid{Au-05_0119}}
% 20
{\PTglyphid{Au-05_0120}}
% 21
{\PTglyphid{Au-05_0121}}
% 22
{\PTglyphid{Au-05_0122}}
% 23
{\PTglyphid{Au-05_0123}}
% 24
{\PTglyphid{Au-05_0124}}
% 25
{\PTglyphid{Au-05_0201}}
% 26
{\PTglyphid{Au-05_0202}}
% 27
{\PTglyphid{Au-05_0203}}
% 28
{\PTglyphid{Au-05_0204}}
% 29
{\PTglyphid{Au-05_0301}}
% 30
{\PTglyphid{Au-05_0302}}
% 31
{\PTglyphid{Au-05_0303}}
% 32
{\PTglyphid{Au-05_0304}}
% 33
{\PTglyphid{Au-05_0305}}
% 34
{\PTglyphid{Au-05_0306}}
% 35
{\PTglyphid{Au-05_0307}}
% 36
{\PTglyphid{Au-05_0308}}
% 37
{\PTglyphid{Au-05_0309}}
% 38
{\PTglyphid{Au-05_0310}}
% 39
{\PTglyphid{Au-05_0311}}
% 40
{\PTglyphid{Au-05_0312}}
% 41
{\PTglyphid{Au-05_0313}}
% 42
{\PTglyphid{Au-05_0314}}
% 43
{\PTglyphid{Au-05_0315}}
% 44
{\PTglyphid{Au-05_0316}}
% 45
{\PTglyphid{Au-05_0317}}
% 46
{\PTglyphid{Au-05_0318}}
% 47
{\PTglyphid{Au-05_0319}}
% 48
{\PTglyphid{Au-05_0320}}
% 49
{\PTglyphid{Au-05_0321}}
% 50
{\PTglyphid{Au-05_0322}}
% 51
{\PTglyphid{Au-05_0323}}
% 52
{\PTglyphid{Au-05_0324}}
% 53
{\PTglyphid{Au-05_0325}}
% 54
{\PTglyphid{Au-05_0326}}
% 55
{\PTglyphid{Au-05_0327}}
% 56
{\PTglyphid{Au-05_0328}}
% 57
{\PTglyphid{Au-05_0329}}
% 58
{\PTglyphid{Au-05_0330}}
% 59
{\PTglyphid{Au-05_0331}}
% 60
{\PTglyphid{Au-05_0332}}
% 61
{\PTglyphid{Au-05_0333}}
% 62
{\PTglyphid{Au-05_0334}}
% 63
{\PTglyphid{Au-05_0335}}
% 64
{\PTglyphid{Au-05_0336}}
% 65
{\PTglyphid{Au-05_0337}}
% 66
{\PTglyphid{Au-05_0338}}
% 67
{\PTglyphid{Au-05_0339}}
% 68
{\PTglyphid{Au-05_0340}}
% 69
{\PTglyphid{Au-05_0341}}
% 70
{\PTglyphid{Au-05_0401}}
% 71
{\PTglyphid{Au-05_0402}}
% 72
{\PTglyphid{Au-05_0403}}
% 73
{\PTglyphid{Au-05_0404}}
% 74
{\PTglyphid{Au-05_0405}}
% 75
{\PTglyphid{Au-05_0406}}
% 76
{\PTglyphid{Au-05_0407}}
% 77
{\PTglyphid{Au-05_0408}}
% 78
{\PTglyphid{Au-05_0501}}
% 79
{\PTglyphid{Au-05_0502}}
% 80
{\PTglyphid{Au-05_0503}}
% 81
{\PTglyphid{Au-05_0504}}
% 82
{\PTglyphid{Au-05_0505}}
% 83
{\PTglyphid{Au-05_0506}}
% 84
{\PTglyphid{Au-05_0507}}
% 85
{\PTglyphid{Au-05_0508}}
% 86
{\PTglyphid{Au-05_0509}}
% 87
{\PTglyphid{Au-05_0510}}
% 88
{\PTglyphid{Au-05_0511}}
% 89
{\PTglyphid{Au-05_0512}}
% 90
{\PTglyphid{Au-05_0513}}
% 91
{\PTglyphid{Au-05_0514}}
% 92
{\PTglyphid{Au-05_0515}}
% 93
{\PTglyphid{Au-05_0516}}
% 94
{\PTglyphid{Au-05_0517}}
% 95
{\PTglyphid{Au-05_0518}}
% 96
{\PTglyphid{Au-05_0519}}
% 97
{\PTglyphid{Au-05_0520}}
//
\endgl \xe
%%% Local Variables:
%%% mode: latex
%%% TeX-engine: luatex
%%% TeX-master: shared
%%% End:

% //
%\endgl \xe

\newpage
%%%%%%%%%%%%%%%%%%%%%%%%%%%%%%%%%%%%%%%%%%%%%%%%%%%%%%%%%%%%%%%%%%%%%%%%%%%%%%
% Tab. 08 Aleksander Augezdecki pismo 6
%%%%%%%%%%%%%%%%%%%%%%%%%%%%%%%%%%%%%%%%%%%%%%%%%%%%%%%%%%%%%%%%%%%%%%%%%%%%%%

% Note "6. Pismo nagłówkowe, tekstura M²⁹. Stopień I w. = 7 mm. — Tabl. 408."
% Note1 "Character set table prepared by Paulina Buchwald-Pelcowa"

\pismoPL{Aleksander Augezdecki 6. Pismo nagłówkowe, tekstura M²⁹. Stopień 1 w. = 7 mm. — Tabl. 408.}

\pismoEN{Aleksander Augezdecki 6. Display font, Textura M²⁹. Type size 1 line = 7 mm. - Plate 408.}
% https://www.adfontes.uzh.ch/en/tutorium/schriften-lesen/schriftgeschichte/gotische-minuskeln-textura-und-textualis/
\medskip

\plate{408}{VIII}{1972}

Prepared by Paulina Buchwald-Pelcowa [layout confusing, misinterpretation possible --- JSB].\\

\bigskip

\fontID{Au-06}{08}

% \exdisplay \bg \gla
 \input {t08_glyphs.tex}
%//
%\glpismo%
 \input {t08_glyphids.tex}
% //
%\endgl \xe

 
 \newpage

%%%%%%%%%%%%%%%%%%%%%%%%%%%%%%%%%%%%%%%%%%%%%%%%%%%%%%%%%%%%%%%%%%%%%%%%%%%%%% 
% Tab. 09 Aleksander Augezdecki pismo 7
%%%%%%%%%%%%%%%%%%%%%%%%%%%%%%%%%%%%%%%%%%%%%%%%%%%%%%%%%%%%%%%%%%%%%%%%%%%%%%

% Note "7. Pismo tekstowe, antykwa Qu/(G, I). Stopień 20 ww. = 101/102 mm (tercja). — Tabl. 409."
% Note1 "Character set table prepared by Paulina Buchwald-Pelcowa"


\pismoPL{Aleksander Augezdecki 7. Pismo tekstowe, antykwa Qu/(G, I). Stopień 20 ww. = 101/102 mm (tercja). — Tabl. 409.}

\pismoEN{Aleksander Augezdecki 7. Text Roman type Qu/(G, I). Type size 20 lines = 101/102 mm (tertia). — Tabl. 409.}
% https://www.adfontes.uzh.ch/en/tutorium/schriften-lesen/schriftgeschichte/gotische-minuskeln-textura-und-textualis/
\medskip

\plate{409}{VIII}{1972}

Prepared by Paulina Buchwald-Pelcowa.

\bigskip

 \exampleBib{VIII:4$^a$}

\bigskip
\exampleDesc{[TESTAMENTUM NOVUM. Evangelium secundum Matthaeum. Trad. polon. Stanislaus Murzynowski]:
Ewangelia Św. Mateusza. Królewiec, [Aleksander Augezdecki], 1551. 4°.}

\medskip
\examplePage{\textit{Karta A₂a.}}

  \bigskip
\exampleLib{Biblioteka Czartoryskich. Kraków.}

\bigskip
\exampleRef{\textit{Estreicher XIII 26. Wierzbowski 135.}}

\bigskip
\exampleDig{\url{https://crispa.uw.edu.pl/object/files/318618/display/Default} page 10}

\medskip

    \examplePL{Pismo 7: tekst i zestaw.}

    \medskip

    \exampleEN{Font 7. The text and the font table}


\bigskip


\fontID{Au-07}{09}

% \exdisplay \bg \gla
 \input {t09_glyphs.tex}
%//
%\glpismo%
 \input {t09_glyphids.tex}
% //
%\endgl \xe

 \newpage

%%%%%%%%%%%%%%%%%%%%%%%%%%%%%%%%%%%%%%%%%%%%%%%%%%%%%%%%%%%%%%%%%%%%%%%%%%%%%% 
% Tab. 10 Aleksander Augezdecki pismo 8
%%%%%%%%%%%%%%%%%%%%%%%%%%%%%%%%%%%%%%%%%%%%%%%%%%%%%%%%%%%%%%%%%%%%%%%%%%%%%%

% Note "8. Pismo tekstowe, antykwa Qu/(G). Stopień 20 ww. = 101 mm (tercja). — Tabl. 410."
% Note1 "Character set table prepared by Paulina Buchwald-Pelcowa"


\pismoPL{Aleksander Augezdecki 8. Pismo tekstowe, antykwa Qu/(G, I). Stopień 20 ww. = 101/102 mm (tercja). — Tabl. 410.}

\pismoEN{Aleksander Augezdecki 8. Text Roman type Qu/(G, I). Type size 20 lines = 101/102 mm (tertia). — Tabl. 410.}
% https://www.adfontes.uzh.ch/en/tutorium/schriften-lesen/schriftgeschichte/gotische-minuskeln-textura-und-textualis/
\medskip

\plate{410}{VIII}{1972}

Prepared by Paulina Buchwald-Pelcowa.

\bigskip

 \exampleBib{VIII:26}

\bigskip
\exampleDesc{[CONFESSIO AUGUSTANA. Trad. polon. Martinus Kwiatkowski]: Confessio Augustanae fidei to jest wyznanie
wiary krześciańskiej. Szamotuły, Aleksander Augezdecki, [po 13 V] 1561. 4°.}

\medskip
\examplePage{\textit{Karta B₁a.}}

  \bigskip
\exampleLib{Biblioteka Zakł. Nar. im. Ossolinskich. Wroclaw.}

\bigskip
\exampleRef{\textit{Estreicher XIV 355. Wierzbowski 1393. Bohonos Ossol. 442. Drukarze IV 31 i 85.}}

\bigskip
\exampleDig{\url{https://dbc.wroc.pl/dlibra/publication/15990/edition/14101} page 13}

\medskip

    \examplePL{Pismo 8: kolumna [? -- JSB]  i pierwszy zestaw.}

    \medskip

    \exampleEN{Font 7. The text column [?] and the first font table.}


\bigskip


\fontID{Au-08}{10}

% \exdisplay \bg \gla
 \input {t10_glyphs.tex}
%//
%\glpismo%
 \input {t10_glyphids.tex}
% //
%\endgl \xe

 \newpage

%%%%%%%%%%%%%%%%%%%%%%%%%%%%%%%%%%%%%%%%%%%%%%%%%%%%%%%%%%%%%%%%%%%%%%%%%%%%%% 
% Tab. 11 Aleksander Augezdecki pismo 9
%%%%%%%%%%%%%%%%%%%%%%%%%%%%%%%%%%%%%%%%%%%%%%%%%%%%%%%%%%%%%%%%%%%%%%%%%%%%%%

% Note "9. Wersaliki tytułowe, antykwa. Wysokość 8 mm. — Tabl. 408."
% Note1 "Character set table prepared by Paulina Buchwald-Pelcowa"

\pismoPL{Aleksander Augezdecki 9. Wersaliki tytułowe, antykwa. Wysokość 8 mm. — Tabl. 408.}

\pismoEN{Aleksander Augezdecki 9. Roman title capitals.  Type size 1 [line] = 8 mm. - Plate 408.}
% https://www.adfontes.uzh.ch/en/tutorium/schriften-lesen/schriftgeschichte/gotische-minuskeln-textura-und-textualis/
\medskip

\plate{408}{VIII}{1972}

Prepared by Paulina Buchwald-Pelcowa [layout confusing, misinterpretation possible --- JSB].\\

\bigskip

\fontID{Au-09}{11}

% \exdisplay \bg \gla
 \input {t11_glyphs.tex}
%//
%\glpismo%
 \input {t11_glyphids.tex}
% //
%\endgl \xe


\newpage

%%%%%%%%%%%%%%%%%%%%%%%%%%%%%%%%%%%%%%%%%%%%%%%%%%%%%%%%%%%%%%%%%%%%%%%%%%%%%% 
% Tab. 12 Aleksander Augezdecki pismo 10
%%%%%%%%%%%%%%%%%%%%%%%%%%%%%%%%%%%%%%%%%%%%%%%%%%%%%%%%%%%%%%%%%%%%%%%%%%%%%%

% Note "10.Wersaliki tytułowe, antykwa. Wysokość 6—7 mm. — Tabl. 408."
% Note1 "Character set table prepared by Paulina Buchwald-Pelcowa"

\pismoPL{Aleksander Augezdecki 10. Wersaliki tytułowe, antykwa. Wysokość 6—7 mm. — Tabl. 408.}

\pismoEN{Aleksander Augezdecki 10. Roman title capitals.  Type size [1 line] = 6--7 mm. - Plate 408.}
% https://www.adfontes.uzh.ch/en/tutorium/schriften-lesen/schriftgeschichte/gotische-minuskeln-textura-und-textualis/
\medskip

\plate{408}{VIII}{1972}

Prepared by Paulina Buchwald-Pelcowa [layout confusing, misinterpretation possible --- JSB].\\

\bigskip

\fontID{Au-10}{12}

% \exdisplay \bg \gla
 \input {t12_glyphs.tex}
%//
%\glpismo%
 \input {t12_glyphids.tex}
% //
%\endgl \xe

 \newpage
 
%%%%%%%%%%%%%%%%%%%%%%%%%%%%%%%%%%%%%%%%%%%%%%%%%%%%%%%%%%%%%%%%%%%%%%%%%%%%%% 
% Tab. 13 Aleksander Augezdecki pismo 11
%%%%%%%%%%%%%%%%%%%%%%%%%%%%%%%%%%%%%%%%%%%%%%%%%%%%%%%%%%%%%%%%%%%%%%%%%%%%%%

% Note "11. Wersaliki tytułowe, antykwa. Wysokość 4,5—5 mm:— Tabl. 408."
% Note1 "Character set table prepared by Paulina Buchwald-Pelcowa"

\pismoPL{Aleksander Augezdecki 11. Wersaliki tytułowe, antykwa. Wysokość 4,5—5 mm:— Tabl. 408.}

\pismoEN{Aleksander Augezdecki 11. Roman title capitals. Type size [1 line] = 4.5--5 mm. - Plate 408.}
% https://www.adfontes.uzh.ch/en/tutorium/schriften-lesen/schriftgeschichte/gotische-minuskeln-textura-und-textualis/
\medskip

\plate{408}{VIII}{1972}

Prepared by Paulina Buchwald-Pelcowa [layout confusing, misinterpretation possible --- JSB].\\

\bigskip

\fontID{Au-11}{13}

% \exdisplay \bg \gla
 \input {t13_glyphs.tex}
%//
%\glpismo%
 \input {t13_glyphids.tex}
% //
%\endgl \xe

\newpage
 
%%%%%%%%%%%%%%%%%%%%%%%%%%%%%%%%%%%%%%%%%%%%%%%%%%%%%%%%%%%%%%%%%%%%%%%%%%%%%% 
% Tab. 14 Aleksander Augezdecki pismo 12
%%%%%%%%%%%%%%%%%%%%%%%%%%%%%%%%%%%%%%%%%%%%%%%%%%%%%%%%%%%%%%%%%%%%%%%%%%%%%%

% Note "12. Wersaliki, antykwa. Wysokość 2—2,5 mm. — Tabl. 408."
% Note1 "Character set table prepared by Paulina Buchwald-Pelcowa"

\pismoPL{Aleksander Augezdecki 12. Wersaliki, antykwa. Wysokość 2—2,5 mm. — Tabl. 408.}

\pismoEN{Aleksander Augezdecki 12. Roman capitals. Type size [1 line] = 2--2.5 mm. - Plate 408.}
% https://www.adfontes.uzh.ch/en/tutorium/schriften-lesen/schriftgeschichte/gotische-minuskeln-textura-und-textualis/
\medskip

\plate{408}{VIII}{1972}

Prepared by Paulina Buchwald-Pelcowa [layout confusing, misinterpretation possible --- JSB].\\

\bigskip

\fontID{Au-12}{14}

% \exdisplay \bg \gla
 \input {t14_glyphs.tex}
%//
%\glpismo%
 \input {t14_glyphids.tex}
% //
%\endgl \xe

\newpage
 

%%%%%%%%%%%%%%%%%%%%%%%%%%%%%%%%%%%%%%%%%%%%%%%%%%%%%%%%%%%%%%%%%%%%%%%%%%%%%% 
% Tab. 15 Aleksander Augezdecki pismo 13
%%%%%%%%%%%%%%%%%%%%%%%%%%%%%%%%%%%%%%%%%%%%%%%%%%%%%%%%%%%%%%%%%%%%%%%%%%%%%%

% Note "13. Pismo nagłówkowe, fraktura H. Schönspergera. Stopień 1 w. = 8 mm. — Tabl. 412."

\pismoPL{Aleksander Augezdecki 13. Pismo  nagłówkowe,
  fraktura H. Schönspergera. Wysokość 1 w. = 8 mm. — Tabl. 412.}

\pismoEN{Aleksander Augezdecki 13. H. Schönsperger's Fracture, header font. Type size 1 line = 8 mm. - Plate 412.}
% https://www.adfontes.uzh.ch/en/tutorium/schriften-lesen/schriftgeschichte/gotische-minuskeln-textura-und-textualis/
\medskip

\plate{412}{VIII}{1972}

Prepared by Paulina Buchwald-Pelcowa [layout confusing, misinterpretation possible --- JSB].\\

\bigskip

\fontID{Au-13}{15}

% \exdisplay \bg \gla
 \input {t15_glyphs.tex}
%//
%\glpismo%
 \input {t15_glyphids.tex}
% //
%\endgl \xe


\newpage
 

%%%%%%%%%%%%%%%%%%%%%%%%%%%%%%%%%%%%%%%%%%%%%%%%%%%%%%%%%%%%%%%%%%%%%%%%%%%%%% 
% Tab. 16 Aleksander Augezdecki pismo 14
%%%%%%%%%%%%%%%%%%%%%%%%%%%%%%%%%%%%%%%%%%%%%%%%%%%%%%%%%%%%%%%%%%%%%%%%%%%%%%

% join!

% Note "14. Pismo tekstowe i nagłówkowe, fraktura H. Schönspergera. Stopień 20 ww. = 153 mm. — Tabl. 410, 411.[410]"
% Note1 "Character set table prepared by Paulina Buchwald-Pelcowa"

\pismoPL{Aleksander Augezdecki 14. Pismo tekstowe i nagłówkowe, fraktura H. Schönspergera. Stopień 20 ww. = 153 mm. — Tabl. 410, 411.}

\pismoEN{Aleksander Augezdecki 14. H. Schönsperger's Fracture, text and header font. Type size 20 lines = 153 mm. - Plate 410.}


\plate{410}{VIII}{1972}

Prepared by Paulina Buchwald-Pelcowa [layout confusing, misinterpretation possible --- JSB].\\

\bigskip

\fontID{Au-14}{16}

    \fontstat{144}


% \exdisplay \bg \gla
 \input {t16_glyphs.tex}
%//
%\glpismo%
 \input {t16_glyphids.tex}
% //
%\endgl \xe


\newpage

%%%%%%%%%%%%%%%%%%%%%%%%%%%%%%%%%%%%%%%%%%%%%%%%%%%%%%%%%%%%%%%%%%%%%%%%%%%%%% 
% Tab. 17 Aleksander Augezdecki pismo 15
%%%%%%%%%%%%%%%%%%%%%%%%%%%%%%%%%%%%%%%%%%%%%%%%%%%%%%%%%%%%%%%%%%%%%%%%%%%%%%

% Note "15. Pismo tytułowe i nagłówkowe, fraktura H. Schönspergera. Wysokość 1 w. = 11—12 mm bez przedłużek. — Tabl. 412."
% Note1 "Character set table prepared by Paulina Buchwald-Pelcowa"

\pismoPL{Aleksander Augezdecki 15. Pismo tytułowe i nagłówkowe,
  fraktura H. Schönspergera. Wysokość 1 w. = 11—12 mm bez
  przedłużek. — Tabl. 412.}

\pismoEN{Aleksander Augezdecki 15. H. Schönsperger's Fracture, title
  and header font. Type size 1 line = 11--12 mm without ascenders and
  descenders. - Plate 412.}

\plate{412}{VIII}{1972}

Prepared by Paulina Buchwald-Pelcowa [layout confusing, misinterpretation possible --- JSB].\\

\bigskip

\fontID{Au-15}{17}

\fontstat{109}

% \exdisplay \bg \gla
 \input {t17_glyphs.tex}
%//
%\glpismo%
 \input {t17_glyphids.tex}
% //
%\endgl \xe

 \newpage

%%%%%%%%%%%%%%%%%%%%%%%%%%%%%%%%%%%%%%%%%%%%%%%%%%%%%%%%%%%%%%%%%%%%%%%%%%%%%% 
% Tab. 18 Aleksander Augezdecki pismo 16
%%%%%%%%%%%%%%%%%%%%%%%%%%%%%%%%%%%%%%%%%%%%%%%%%%%%%%%%%%%%%%%%%%%%%%%%%%%%%%

% Note "16. Pismo tekstowe, szwabacha M⁸¹. Stopień 20 ww. = 88 mm. — Tabl. 413."
% Note1 "Character set table prepared by Paulina Buchwald-Pelcowa"


\pismoPL{Aleksander Augezdecki 16. Pismo tekstowe, szwabacha M⁸¹. Stopień 20 ww. = 88 mm. — Tabl. 413.}

\pismoEN{Aleksander Augezdecki 16. Text type, Schwabacher M⁸¹. Type size 20 lines = 88 mm. — Tabl. 413.}
% https://www.adfontes.uzh.ch/en/tutorium/schriften-lesen/schriftgeschichte/gotische-minuskeln-textura-und-textualis/
\medskip

\plate{413}{VIII}{1972}

Prepared by Paulina Buchwald-Pelcowa.

\bigskip

 \exampleBib{VIII:20}

\bigskip
\exampleDesc{IACOBUS KUCHLER: Epithalamion de nuptiis Andreae comitis in Gorka.
Szamotuly, Aleksander Augezdecki, 20 X 1558. 4°.}

\medskip
\examplePage{\textit{Karta A₃a.}}

  \bigskip
\exampleLib{Książnica Miejska im. Kopernika. Torun}

\bigskip
\exampleRef{\textit{Drukarze IV 29.}}

% \bigskip
% \exampleDig{\url{https://dbc.wroc.pl/dlibra/publication/15990/edition/14101} page 13}

\medskip

    \examplePL{Pismo 16: tekst i zestaw.}

    \medskip

    \exampleEN{Font 16. A text and the font table.}


\bigskip


\fontID{Au-16}{18}

\fontstat{74}

% \exdisplay \bg \gla
 \input {t18_glyphs.tex}
%//
%\glpismo%
 \input {t18_glyphids.tex}
% //
%\endgl \xe

 \newpage

%%%%%%%%%%%%%%%%%%%%%%%%%%%%%%%%%%%%%%%%%%%%%%%%%%%%%%%%%%%%%%%%%%%%%%%%%%%%%%% 
% Tab. 19 Aleksander Augezdecki pismo 17
%%%%%%%%%%%%%%%%%%%%%%%%%%%%%%%%%%%%%%%%%%%%%%%%%%%%%%%%%%%%%%%%%%%%%%%%%%%%%%

% Note "17. Pismo tytułowe i nagłówkowe, tekstura M⁶³. Wysokość I w.= 12— 13 mm. — Tabl. 412."
% Note1 "Character set table prepared by Paulina Buchwald-Pelcowa"

\pismoPL{Aleksander Augezdecki 17. Pismo tytułowe i nagłówkowe, tekstura M⁶³. Wysokość 1 w.= 12—13 mm. — Tabl. 412.}

\pismoEN{Aleksander Augezdecki 17. Title
  and header font. Type size 1 line = 12-13 mm.- Plate 412.}

\plate{412}{VIII}{1972}

Prepared by Paulina Buchwald-Pelcowa [layout confusing, misinterpretation possible --- JSB].\\

\bigskip

\fontID{Au-17}{19}

\fontstat{67}

% \exdisplay \bg \gla
 \input {t19_glyphs.tex}
%//
%\glpismo%
 \input {t19_glyphids.tex}
% //
%\endgl \xe

  \newpage

%%%%%%%%%%%%%%%%%%%%%%%%%%%%%%%%%%%%%%%%%%%%%%%%%%%%%%%%%%%%%%%%%%%%%%%%%%%%%%% 
% Tab. 20 Jan Haller pismo 1
%%%%%%%%%%%%%%%%%%%%%%%%%%%%%%%%%%%%%%%%%%%%%%%%%%%%%%%%%%%%%%%%%%%%%%%%%%%%%%

% Note "1. Pismo kanonowe. Krój M¹⁹, Stopień 10 ww. = 128/129 mm. — Tabl. 164."
% Note1 "Character set table prepared by Maria Błońska"

  \pismoPL{Jan Haller 1. Pismo kanonowe. Krój M¹⁹. Stopień 10 ww. =
    128/129 mm. — Tabl. 164. (Występuje u Hochfedera jako pismo
    8. Tabl. 25, u Unglera jako pismo 13,  Tabl. 72). }


  
  \pismoEN{Jan Haller 1. Canon [?] font. Typeface M¹⁹. Type size 10
    ww. = 128/129 mm. — Plate 164. (Used by Hochfeder as font 8, plate
    24, and by Ungler as font 13, plate 72.)}

\plate{164}{IV}{1962}

Prepared by Helena Kapełuś.\\
The font table prepared by Maria Błońska.

\bigskip

\exampleBib{IV:121}

\bigskip
\exampleDesc{MISSALE Vladislaviense. Kraków, Jan Haller, 29. XI. 1515 — 1. II. 1516. 2⁰.}

\medskip
\examplePage{\textit{Karta 1 (Canon) niepełna: wiersze 1——15.}}

  \bigskip
\exampleLib{Biblioteka Czartoryskich. Kraków.}

\bigskip
\exampleRef{\textit{Estreicher XXII. 434}}

\bigskip
\exampleDig{\url{https://cyfrowe.mnk.pl/dlibra/publication/22776/}, page 403.}

% \medskip

%     \examplePL{Pismo 2: tekst i zestaw liter ze znakami diakrytycznymi czeskimi.}

%     \medskip

%     \exampleEN{Font 2. The text and the table including letters with Czech diacritical marks.}


\bigskip


\fontID{Ha-01}{20}

\fontstat{85}

% \exdisplay \bg \gla
 \input {t20_glyphs.tex}
%//
%\glpismo%
 \input {t20_glyphids.tex}
% //
%\endgl \xe

  \newpage

%%%%%%%%%%%%%%%%%%%%%%%%%%%%%%%%%%%%%%%%%%%%%%%%%%%%%%%%%%%%%%%%%%%%%%%%%%%%%%% 
% Tab. 21 Jan Haller pismo 2
%%%%%%%%%%%%%%%%%%%%%%%%%%%%%%%%%%%%%%%%%%%%%%%%%%%%%%%%%%%%%%%%%%%%%%%%%%%%%%

% BAD:???
% python3 PT_chunks.py renumbered-lines-m21/
% OK:
% python3 PT_chunks-test.py renumbered-lines-m21/

  
% Note "2. Pismo mszalne większe. Krój M¹⁸. Stopień 10 ww. = 77 mm. — Tabl. 165."
% Note1 "Character set table prepared by Maria Błońska"

  \pismoPL{Jan Haller 2. Pismo mszalne większe. Krój M¹⁸. Stopień 10
    ww. = 77 mm. — Tabl. 165. (Występuje u Hochfedera jako pismo
    7. Tabl. 24, u Unglera jako pismo 12. Tabl. 72).}


  
\pismoEN{Jan Haller 2. Larger mass [?] font. Typeface M¹⁸. Type size 10 ww. =
    77 mm. — Plate 165. (Used by Hochfeder as font 7, plate
    24, and by Ungler as font 12, plate 72.)}

\plate{165}{IV}{1962}

Prepared by Helena Kapełuś.\\
The font table prepared by Maria Błońska.

\bigskip

\exampleBib{IV:2}

\bigskip
\exampleDesc{IOANNES LASKI: Commune Poloniae Regni privilegium. Kraków, Jan Haller, [po 27. I. 1506]. 2⁰}

\medskip
\examplePage{\textit{Karta a₃b.}}

  \bigskip
\exampleLib{Biblioteka Zakł. Nar. im. Ossolińskich. Wrocław.}

\bigskip
\exampleRef{\textit{Estreicher XXI. 79. Wierzbowski 9.}}

\bigskip
%\exampleDig{\url{https://www.wbc.poznan.pl/dlibra/publication/493453/}, page ???}???
\exampleDig{\url{https://dlibra.biblioteka.tarnow.pl/publication/196}, page 54.}

\medskip

    \examplePL{Pismo 2: wiersz 1—4, 17.}

    \medskip

    \exampleEN{Font 2: lines 1--4, 17}


\bigskip


\fontID{Ha-02}{21}

\fontstat{111}

% \exdisplay \bg \gla
 \input {t21_glyphs.tex}
%//
%\glpismo%
 \input {t21_glyphids.tex}
% //
%\endgl \xe

  \newpage

%%%%%%%%%%%%%%%%%%%%%%%%%%%%%%%%%%%%%%%%%%%%%%%%%%%%%%%%%%%%%%%%%%%%%%%%%%%%%%% 
% Tab. 22 Jan Haller pismo 3
%%%%%%%%%%%%%%%%%%%%%%%%%%%%%%%%%%%%%%%%%%%%%%%%%%%%%%%%%%%%%%%%%%%%%%%%%%%%%%

% Note "3. Pismo mszalne mniejsze. Krój M²³. Stopień 20 ww. = 136/137 mm. — Tabl. 166."
% Note1 "Character set table prepared by Maria Błońska"


  \pismoPL{Jan Haller 3. Pismo mszalne większe. Krój M²³. Stopień 10
    ww. = 136/137 mm. — Tabl. 166. (Występuje u Hochfedera jako pismo
    11. Tabl. 24, u Unglera jako pismo 11).}


  
\pismoEN{Jan Haller 3. Smaller mass [?] font. Typeface M²³. Type size 10 ww. =
    136/137 mm. — Plate 166. (Used by Hochfeder as font 11, plate
    11, and by Ungler as font 11.)}

\plate{166}{IV}{1962}

Prepared by Helena Kapełuś.\\
The font table prepared by Maria Błońska.

\bigskip

\exampleBib{IV:151}

\bigskip
\exampleDesc{AGENDA Cracoviensis. Kraków, Jan Haller, [1517]. 4⁰.}

\medskip
\examplePage{\textit{Karta M₂b.}}

  \bigskip
\exampleLib{Biblioteka Zakł. Nar. im. Ossolińskich. Wrocław.}

\bigskip
\exampleRef{\textit{Estreicher XII. 70.}}

\bigskip
%\exampleDig{\url{https://www.wbc.poznan.pl/dlibra/publication/493453/}, page ???}???
\exampleDig{\url{https://dbc.wroc.pl/dlibra/publication/11760/}, page 242.}

% \medskip

%     \examplePL{Pismo 2: wiersz 1—4, 17.}

%     \medskip

%     \exampleEN{Font 2: lines 1--4, 17}


\bigskip


\fontID{Ha-03}{22}

\fontstat{115}

% \exdisplay \bg \gla
 \input {t22_glyphs.tex}
%//
%\glpismo%
 \input {t22_glyphids.tex}
% //
%\endgl \xe

  \newpage

%%%%%%%%%%%%%%%%%%%%%%%%%%%%%%%%%%%%%%%%%%%%%%%%%%%%%%%%%%%%%%%%%%%%%%%%%%%%%%% 
% Tab. 23 Jan Haller pismo 4
%%%%%%%%%%%%%%%%%%%%%%%%%%%%%%%%%%%%%%%%%%%%%%%%%%%%%%%%%%%%%%%%%%%%%%%%%%%%%%

% Note "4. Pismo nagłówkowe. Krój M⁸³. Stopień 20 ww. =112/114 mm. — Tabl. 167."
% Note1 "Character set table prepared by Maria Błońska"


  \pismoPL{Jan Haller 4. Pismo nagłówkowe. Krój M⁸³. Stopień 20 ww. =
    112/114 mm. — Tabl. 167. (Występuje u Hochfedera jako pismo
    2. Tabl. 19).}

  
  \pismoEN{Jan Haller 4. Header font. Typeface M⁸³. Type size 20 ww. =
    112/114 mm. — Plate 167. (Used by Hochfeder as font 2, plate 19.)}

\plate{167}{IV}{1962}

Prepared by Helena Kapełuś.\\
The font table prepared by Maria Błońska.

\bigskip

\exampleBib{IV:21}

\bigskip
\exampleDesc{STATUTA conventus generalis Cracoviensis. [Kraków, Jan Haller, po 2. IIT. 1507]. 2*. Wyd. A.}

\medskip
\examplePage{\textit{Karta [1] a.}}

  \bigskip
\exampleLib{Biblioteka Narodowa. Warszawa.}

\bigskip
\exampleRef{\textit{Estreicher XXIX. 249.}}

\bigskip
%\exampleDig{\url{https://www.wbc.poznan.pl/dlibra/publication/493453/}, page ???}???
\exampleDig{\url{https://wbc.poznan.pl/dlibra/publication/493454} planned (as of \today)}

% \medskip

%     \examplePL{Pismo 2: wiersz 1—4, 17.}

%     \medskip

%     \exampleEN{Font 2: lines 1--4, 17}


\bigskip


\fontID{Ha-04}{23}

\fontstat{137}

% \exdisplay \bg \gla
 \input {t23_glyphs.tex}
%//
%\glpismo%
 \input {t23_glyphids.tex}
% //
%\endgl \xe

  \newpage

%%%%%%%%%%%%%%%%%%%%%%%%%%%%%%%%%%%%%%%%%%%%%%%%%%%%%%%%%%%%%%%%%%%%%%%%%%%%%%% 
% Tab. 24 Jan Haller pismo 5
%%%%%%%%%%%%%%%%%%%%%%%%%%%%%%%%%%%%%%%%%%%%%%%%%%%%%%%%%%%%%%%%%%%%%%%%%%%%%%

% Note "5. Pismo tekstowe. Krój M⁴⁸. Stopień 20 ww. =81/82 mm. — Tabl. 168."
% Note1 "Character set table prepared by Maria Błońska"

  \pismoPL{Jan Haller 5. Pismo tekstowe. Krój M⁴⁸. Stopień 20
    ww. =81/82 mm. — Tabl. 168. (Występuje u Hochfedera jako pismo
    9. Tabl. 26, u Unglera jako pismo 4.  Tabl. 115).}


  
\pismoEN{Jan Haller 5. Text font. Typeface M⁴⁸. Type size 20 ww. =
    81/82 mm. — Plate 168. (Used by Hochfeder as font 9, plate
    26, and by Ungler as font 4, plate 115.)}

\plate{168}{IV}{1962}

Prepared by Helena Kapełuś.\\
The font table prepared by Maria Błońska.

\bigskip

\exampleBib{IV:17}

\bigskip
\exampleDesc{17. IOANNES SACRANUS: Hlucidarius errorum ritus Ruthenici. [Kraków, Jan Haller, ok. 1507]. 4⁰.}
  

\medskip
\examplePage{\textit{Karta 7 a.}}

  \bigskip
\exampleLib{Biblioteka Narodowa. Warszawa.}

\bigskip
\exampleRef{\textit{Estreicher XVII. 13. Wierzbowski 773.}}

\bigskip
\exampleDig{\url{http://old.mbc.malopolska.pl/dlibra/docmetadata?id=83069}
page 13,\\
\url{https://www.dbc.wroc.pl/dlibra/publication/3624/edition/3513}
page 7.}
  % https://www.dbc.wroc.pl/dlibra/publication/3624/edition/3513?language=en
  % http://old.mbc.malopolska.pl/dlibra/docmetadata?id=83069&from=publication
  
  % \medskip
\bigskip

    \examplePL{Pismo 4: naglówek.}

    \medskip

    \exampleEN{Font 4: the header}


\bigskip


\fontID{Ha-05}{24}

\fontstat{135}

% \exdisplay \bg \gla
 \input {t24_glyphs.tex}
%//
%\glpismo%
 \input {t24_glyphids.tex}
% //
%\endgl \xe

  \newpage

%%%%%%%%%%%%%%%%%%%%%%%%%%%%%%%%%%%%%%%%%%%%%%%%%%%%%%%%%%%%%%%%%%%%%%%%%%%%%%% 
% Tab. 25 Jan Haller pismo 6
%%%%%%%%%%%%%%%%%%%%%%%%%%%%%%%%%%%%%%%%%%%%%%%%%%%%%%%%%%%%%%%%%%%%%%%%%%%%%%

% Note "6. Pismo komentarzowe. Krój M”. Stopień 20 ww. = 82 mm (interliniowane). — Tabl. 169."
% Note1 "Character set table prepared by Maria Błońska"


  \pismoPL{Jan Haller 6. Pismo komentarzowe. Krój M⁹¹. Stopień 20
    ww. = 82 mm (interliniowane). — Tabl. 169. (Występuje u Hochfedera
    jako pismo 4. Tabl. 21).}



  
\pismoEN{Jan Haller 6. Text font. Typeface M⁹¹. Type size 20 ww. =
    82 mm (with extra leading). — Plate 169. (Used by Hochfeder as font 4, plate
    21.)}

\plate{169}{IV}{1962}

Prepared by Helena Kapełuś.\\
The font table prepared by Maria Błońska.

\bigskip

\exampleBib{IV:23}

\bigskip \exampleDesc{THEOREMATA Autoris causarum cum
  annotationibus ac expositione Iacobi de Gostynin. Kraków, Jan
  Haller, 23. III. 1507. 4⁰.}
  

\medskip
\examplePage{\textit{Karta b₃a.}}

  \bigskip
\exampleLib{Biblioteka Zakl. Nar. im. Ossolińskich. Wrocław.}

\bigskip
\exampleRef{\textit{Estreicher XV. 85. Wierzbowski 848.}}

\bigskip
\exampleDig{\url{https://www.wbc.poznan.pl/dlibra/publication/310579/} page 25.}
  
  % \medskip
\bigskip

    \examplePL{Pismo 4: naglówek. [Pismo 6: tekst - JSB]}

    \medskip

    \exampleEN{Font 4: the header [Font 6: the text - JSB]}


\bigskip


\fontID{Ha-06}{25}

\fontstat{94}

% \exdisplay \bg \gla
 \input {t25_glyphs.tex}
%//
%\glpismo%
 \input {t25_glyphids.tex}
% //
%\endgl \xe

  

  \newpage

%%%%%%%%%%%%%%%%%%%%%%%%%%%%%%%%%%%%%%%%%%%%%%%%%%%%%%%%%%%%%%%%%%%%%%%%%%%%%%% 
% Tab. 26 Jan Haller pismo 7
%%%%%%%%%%%%%%%%%%%%%%%%%%%%%%%%%%%%%%%%%%%%%%%%%%%%%%%%%%%%%%%%%%%%%%%%%%%%%%

% Note "6. Pismo komentarzowe. Krój M”. Stopień 20 ww. = 82 mm (interliniowane). — Tabl. 169."
% Note1 "Character set table prepared by Maria Błońska"


  \pismoPL{Jan Haller 6. Pismo komentarzowe. Krój M⁹¹. Stopień 20
    ww. = 82 mm (interliniowane). — Tabl. 169. (Występuje u Hochfedera
    jako pismo 4. Tabl. 21).}



  
\pismoEN{Jan Haller 6. Text font. Typeface M⁹¹. Type size 20 ww. =
    82 mm (with extra leading). — Plate 169. (Used by Hochfeder as font 4, plate
    21.)}

\plate{170}{IV}{1962}

Prepared by Helena Kapełuś.\\
The font table prepared by Maria Błońska.

\bigskip

\exampleBib{IV:83}

\bigskip \exampleDesc{PETRUS ROSELLI: Quaestiones in libros priorum
  Analyticorum et Elenchorum Aristotelis. Ed. Ioannes de Stobnica.
  Kraków, Jan Haller, 24. V. 1511. 4⁰.}
  

\medskip
\examplePage{\textit{Karta 89}}

  \bigskip
\exampleLib{Biblioteka Narodowa. Warszawa.}


\bigskip
\exampleRef{\textit{Estreicher XXVI. 363. Wierzbowski 18.}}

% \bigskip
% \exampleDig{\url{} page} 
% podobne ale nie identyczne:
% PETRUS ROSELLI: Quaestiones in libros priorum
%   Analyticorum et Elenchorum Aristotelis. 
%   https://www.dbc.wroc.pl/dlibra/publication/15987/edition/13965

  
  % \medskip
\bigskip

    \examplePL{Pismo 4: naglówek. [Pismo 7: tekst - JSB]}

    \medskip

    \exampleEN{Font 4: the header [Font 7: the text - JSB]}


\bigskip


\fontID{Ha-07}{26}

\fontstat{129}

% \exdisplay \bg \gla
 \exdisplay \bg \gla
% 1
{\PTglyph{5}{t26_l01g01.png}}
% 2
{\PTglyph{5}{t26_l01g02.png}}
//
%%% Local Variables:
%%% mode: latex
%%% TeX-engine: luatex
%%% TeX-master: shared
%%% End:

%//
%\glpismo%
 \input {t26_glyphids.tex}
% //
%\endgl \xe

 \newpage

%%%%%%%%%%%%%%%%%%%%%%%%%%%%%%%%%%%%%%%%%%%%%%%%%%%%%%%%%%%%%%%%%%%%%%%%%%%%%%% 
% Tab. 27 Jan Haller pismo 8
%%%%%%%%%%%%%%%%%%%%%%%%%%%%%%%%%%%%%%%%%%%%%%%%%%%%%%%%%%%%%%%%%%%%%%%%%%%%%%

% Note "8. Pismo tekstowe. Krój M⁹¹. Stopień 20 ww. = 72 mm. — Tabl. 171."
% Note1 "Character set table prepared by Maria Błońska"

  \pismoPL{Jan Haller 8. Pismo tekstowe. Krój M⁹¹. Stopień 20 ww. = 72 mm. — Tabl. 171.}
  
\pismoEN{Jan Haller 8. Text font. Typeface M⁹¹. Type size 20 ww. =
    72 mm . — Plate 171.}

\plate{171}{IV}{1962}

Prepared by Helena Kapełuś.\\
The font table prepared by Maria Błońska.

\bigskip

\exampleBib{IV:201}

\bigskip


\exampleDesc{MICHAEL VRATISLAVIENSIS: Epithoma conclusionum
  theologicalium pro introductione in IV libros Petri
  Lombardi. Kraków, Jan Haller, 1521. 4⁰.}
  

\medskip
\examplePage{\textit{Karta a²a}}

  \bigskip
\exampleLib{Biblioteka Narodowa. Warszawa.}


\bigskip
\exampleRef{\textit{Estreicher XXXIII. 357. Wierzbowski 57.}}

\bigskip
\exampleDig{\url{https://www.dbc.wroc.pl/dlibra/publication/8876/} page 7} 

% https://polona.pl/preview/1f9dd490-e381-429e-9784-174169b32fb7
% Epithoma conclusionum theologicalium pro introductione in quatuor libros Sententiarum magistri Petri Lombardi [...] in [...] studio Cracouien[si] elucubratum

% https://www.dbc.wroc.pl/dlibra/publication/8876/edition/8006
% Epithoma conclusionum theologicalium pro introductione in quatuor libros sententiarum magistri Petri Lombardi [...]
% page 7


% https://polona2.pl/item/epithoma-conclusionum-theologicalium-pro-introductione-in-quatuor-libros-sententiarum,OTIxNzA2MTg/2/#info:metadata

% https://platforma.bk.pan.pl/en/search_results/1283666


\bigskip

    \examplePL{Pismo 12: wiersz 1, 4. [Pismo 8: tekst - JSB]}

    \medskip

    \exampleEN{Font 12: the lines 1, 4. [Font 8: the text - JSB]}


\bigskip


\fontID{Ha-08}{27}

\fontstat{121}

% \exdisplay \bg \gla
 \input {t27_glyphs.tex}
%//
%\glpismo%
 \input {t27_glyphids.tex}
% //
%\endgl \xe

 \newpage

%%%%%%%%%%%%%%%%%%%%%%%%%%%%%%%%%%%%%%%%%%%%%%%%%%%%%%%%%%%%%%%%%%%%%%%%%%%%%%% 
% Tab. 28 Jan Haller pismo 9
%%%%%%%%%%%%%%%%%%%%%%%%%%%%%%%%%%%%%%%%%%%%%%%%%%%%%%%%%%%%%%%%%%%%%%%%%%%%%%

% Note "9. Pismo tekstowe. Krój M⁴⁹. Stopień 20 ww. = 72 mm. — Tabl. 172."
% Note1 "Character set table prepared by Maria Błońska"

 
  \pismoPL{Jan Haller 9. Pismo tekstowe. Krój M⁴⁹. Stopień 20 ww. = 72 mm. — Tabl. 172.}
  
\pismoEN{Jan Haller 9. Text font. Typeface M⁴¹. Type size 20 ww. =
    72 mm . — Plate 172.}

\plate{172}{IV}{1962}

Prepared by Helena Kapełuś.\\
The font table prepared by Maria Błońska.

\bigskip

\exampleBib{IV:135}

\bigskip


\exampleDesc{MICHAEL VRATISLAVIENSIS: Expositio hymnorumque interpretatio. Kraków, Jan Haller, 1516. 4⁰.}
  

\medskip
\examplePage{\textit{Karta 3 b}}

  \bigskip
\exampleLib{Biblioteka Zakł. Nar. im. Ossolińskich. Wrocław.}


\bigskip
\exampleRef{\textit{Estreicher XXXIII. 358.}}

% \bigskip
% \exampleDig{\url page 7} 

\bigskip

\examplePL{Pismo 3: naglówek. — Pismo 9: tekst i pierwszy zestaw. —
  Rubryka \beta{} z pismem 9. — Cyfry 7: z pismem 9.  z pismem 13. [\ldots]}

    \medskip

    \exampleEN{Font 3: the header. — Font 9: the text and the first character set. — Rubric \beta{} with digits from font 9}


\bigskip


\fontID{Ha-09}{28}

\fontstat{119}

% \exdisplay \bg \gla
 \input {t28_glyphs.tex}
%//
%\glpismo%
 \input {t28_glyphids.tex}
% //
%\endgl \xe


 \newpage

%%%%%%%%%%%%%%%%%%%%%%%%%%%%%%%%%%%%%%%%%%%%%%%%%%%%%%%%%%%%%%%%%%%%%%%%%%%%%%% 
 % Tab. 29 Jan Haller pismo 10
%%%%%%%%%%%%%%%%%%%%%%%%%%%%%%%%%%%%%%%%%%%%%%%%%%%%%%%%%%%%%%%%%%%%%%%%%%%%%%

% Note "10. Pismo komentarzowe. Krój M⁸⁷. Stopień 20 ww. = 72 mm. — Tabl. 173."
% Note1 "Character set table prepared by Maria Błońska"


  \pismoPL{Jan Haller 10. Pismo komentarzowe. Krój M⁸⁷. Stopień 20 ww. = 72 mm. — Tabl. 173.}
  
  \pismoEN{Jan Haller 10. Text font. Typeface M⁸⁷. Type size 20 ww. =
    72 mm . — Plate 173.}

\plate{173}{IV}{1962}

Prepared by Helena Kapełuś.\\
The font table prepared by Maria Błońska.

\bigskip

\exampleBib{IV:208}

\bigskip


\exampleDesc{IACOBUS FABER STAPULENSIS: Introductio in libros De Anima Aristotelis. Kraków, Jan Haller, 1522. 4⁰}
  

\medskip
\examplePage{\textit{Karta A₂b}}

  \bigskip
\exampleLib{Biblioteka Zakł. Nar. im. Ossolińskich. Wrocław.}


\bigskip
\exampleRef{\textit{Estreicher. XIV. 150. Wierzbowski 993.}}

% \bigskip
% \exampleDig{\url page 7} 

\bigskip

\examplePL{Rubryka \eta{}: z pismem 10.}

    \medskip

    \exampleEN{Rubric \eta{} with font 10.}


\bigskip


\fontID{Ha-10}{29}

\fontstat{102}

% \exdisplay \bg \gla
 \input {t29_glyphs.tex}
%//
%\glpismo%
 \input {t29_glyphids.tex}
% //
%\endgl \xe

 
 \newpage

%%%%%%%%%%%%%%%%%%%%%%%%%%%%%%%%%%%%%%%%%%%%%%%%%%%%%%%%%%%%%%%%%%%%%%%%%%%%%%% 
 % Tab. 30 Jan Haller pismo 11
%%%%%%%%%%%%%%%%%%%%%%%%%%%%%%%%%%%%%%%%%%%%%%%%%%%%%%%%%%%%%%%%%%%%%%%%%%%%%%

% Note "11. Pismo tekstowe antykwowe. Krój Qlu (C). Stopień 20 ww. = 92 mm. — - Tabl. 174."
% Note1 "Character set table prepared by Maria Błońska"


  \pismoPL{Jan Haller 11. Pismo tekstowe antykwowe. Krój Q|u (C). Stopień 20 ww. = 92 mm. — - Tabl. 174.}
  
  \pismoEN{Jan Haller 11. Roman text font. Typeface Q|u (C). Type size 20 ww. =
    92 mm . — Plate 174.}

\plate{174}{IV}{1962}

Prepared by Helena Kapełuś.\\
The font table prepared by Maria Błońska.

\bigskip

\exampleBib{IV:163}

\bigskip


\exampleDesc{STANISLAUS ZABOROWSKI: Grammatices rudimenta. Kraków, Jan Haller, 1518. 4⁰}
  
% https://cyfrowe.mnk.pl/dlibra/publication/24392/edition/24082?language=en
\medskip
\examplePage{\textit{Karta K₃a}}

  \bigskip
\exampleLib{Biblioteka Narodowa. Warszawa.}



\bigskip
\exampleRef{\textit{Estreicher.XXXIV. 45. Wierzbowski 957.}}

% \bigskip
\exampleDig{\url{https://polona.pl/preview/dfb921b2-a937-4ee8-ad01-0724aa52c76a}
  page 121}

\bigskip

\examplePL{Pismo 11 (tekst łaciński). — Rubryki \theta, \iota. — Cyfry 6.}

    \medskip

    \exampleEN{Font 11 (Latin text). — Rubric \theta{}, \iota. — Digits 6.}


\bigskip

\exampleBib{IV:16}

\bigskip


\exampleDesc{STANISLAUS ZABOROWSKI: Orthographia seu modus recte scribendi. Kraków, Jan Haller, IV. 1518. 4⁰.}




\medskip
\examplePage{\textit{Karta [5] a}}

  \bigskip
\exampleLib{Biblioteka Narodowa. Warszawa.}



\bigskip
\exampleRef{\textit{Estreicher.XXXIV. 52. Wierzbowski 48.}}

% \bigskip
\exampleDig{\url{https://polona.pl/preview/2db2e5da-ec87-426c-8e0d-e6b09093407a}
  page 17}
% https://zpe.gov.pl/kronika/759455

\bigskip

\examplePL{Pismo 11 (tekst polski). — Rubryka \theta{}. — Inicjał 60 (V).}

    \medskip

    \exampleEN{Font 11 (Polish text). — Rubric \theta{}. — Initial 60. (V)}


\bigskip


\fontID{Ha-11}{30}

\fontstat{112}

% \exdisplay \bg \gla
 \input {t30_glyphs.tex}
%//
%\glpismo%
 \input {t30_glyphids.tex}
% //
%\endgl \xe


 
 \newpage

%%%%%%%%%%%%%%%%%%%%%%%%%%%%%%%%%%%%%%%%%%%%%%%%%%%%%%%%%%%%%%%%%%%%%%%%%%%%%%% 
 % Tab. 31 Jan Haller pismo 12 
%%%%%%%%%%%%%%%%%%%%%%%%%%%%%%%%%%%%%%%%%%%%%%%%%%%%%%%%%%%%%%%%%%%%%%%%%%%%%%%

% Note "12. Pismo mszalne. Krój M⁶⁰. Stopień 20 ww. = 176 mm. — Tabl. 175."
% Note1 "Character set table prepared by Maria Błońska"

  \pismoPL{Jan Haller 12. Pismo mszalne. Krój M⁶⁰. Stopień 20 ww. = 176 mm. — Tabl. 175.}
  
  \pismoEN{Jan Haller 12. Missal font. Typeface M⁶⁰. Type size 20 ww. =
    176 mm . — Plate 175.}

\plate{175}{IV}{1962}

Prepared by Helena Kapełuś.\\
The font table prepared by Maria Błońska.

\bigskip

\exampleBib{IV:173}

\bigskip

\exampleDesc{PSALTERIUM secundum morem Ecclesiae Cracoviensis. Kraków, Jan Haller, 1518. 2⁰.}
  

\medskip
\examplePage{\textit{Karta ostatnia, kolofon.}}

\examplePageEN{\textit{Last page, colophon}}

  
  \bigskip
\exampleLib{Biblioteka Narodowa. Warszawa.}

\bigskip
\exampleRef{\textit{Estreicher XXV. 387.}}

% \bigskip
% \exampleDig{\url{https://polona.pl/preview/dfb921b2-a937-4ee8-ad01-0724aa52c76a}
%   page 121}

\bigskip

\examplePL{Pismo 12. — Rubryki \chi{}, \lambda{}. — Cyfry 8.}

    \medskip

    \exampleEN{Font 12. — Rubric \chi{}, \lambda. — Digits 8.}


\bigskip

\bigskip


\fontID{Ha-12}{31}

\fontstat{100}

% \exdisplay \bg \gla
 \input {t31_glyphs.tex}
%//
%\glpismo%
 \input {t31_glyphids.tex}
% //
%\endgl \xe

 {
 \relsize{-1}
 
\exampleBibExtra{IV:26}

\bigskip

\exampleDesc{MICHAEL VRATISLAVIENSIS: Introductorium astronomiae Cracoviense almanach.
Kraków, Jan Haller, 29. V. 1507. 4⁰.}




\medskip
\examplePage{\textit{Karta A₆b}}

  \bigskip
\exampleLib{Biblioteka Zakl. Nar. im. Ossolińskich. Wrocław}


\bigskip
\exampleRef{\textit{Estreicher XXXIII. 356.}}

% https://polona2.pl/item/introductoriu-m-astronomie-cracouiense-elucidans-almanach,NDQzMzg2MDM/0/#info:metadata
% 1517
% https://dbc.wroc.pl/dlibra/publication/8925/edition/8046?language=pl
% 1507!!!
% https://wbc.poznan.pl/dlibra/publication/402464/edition/314800
% 1506


% \bigskip
% \exampleDig{\url{https://dbc.wroc.pl/dlibra/publication/8925}
%   page brak?!}

\bigskip

\examplePL{Znaki Zodiaku. — Pismo 5. — Rubryka \beta{} z pismem 5. — Cyfry 2.}

    \medskip

    \exampleEN{Znaki Zodiaku. — Font 5. — Rubric \beta{} with font 5. — Digits 2}
  }
  
 \newpage

%%%%%%%%%%%%%%%%%%%%%%%%%%%%%%%%%%%%%%%%%%%%%%%%%%%%%%%%%%%%%%%%%%%%%%%%%%%%%%% 
 % Tab. 32 Jan Haller pismo 13
%%%%%%%%%%%%%%%%%%%%%%%%%%%%%%%%%%%%%%%%%%%%%%%%%%%%%%%%%%%%%%%%%%%%%%%%%%%%%%%

% Note "13.  Pismo tekstowe. Krój M¹⁸. Stopień 20 ww. = 112 mm (interliniowane). — Tabl. 172."
% Note1 "Character set table prepared by Maria Błońska"

  \pismoPL{Jan Haller 13.  Pismo tekstowe. Krój M¹⁸. Stopień 20 ww. = 112 mm (interliniowane). — Tabl. 172.}
  
  \pismoEN{Jan Haller 13. Text font. Typeface M¹⁸. Type size 20 ww. =
    112 mm (with extra leading). — Plate 172.}

\plate{172}{IV}{1962}

Prepared by Helena Kapełuś.\\
The font table prepared by Maria Błońska.


\examplePL{Pismo 13: drugi zestaw. — Cyfry 9: z pismem 13.}

    \medskip

    \exampleEN{Font 13. — second character set,  — Digits 9: with font 13.}


\bigskip


\fontID{Ha-13}{32}

\fontstat{96}

% \exdisplay \bg \gla
 \input {t32_glyphs.tex}
%//
%\glpismo%
 \input {t32_glyphids.tex}
% //
%\endgl \xe

\end{document}

\newpage

%%%%%%%%%%%%%%%%%%%%%%%%%%%%%%%%%%%%%%%%%%%%%%%%%%%%%%%%%%%%%%%%%%%%%%%%%%%%%%% 
 % Tab. 33,Hochfeder-02_PT01_020bis.djvu,Hochfeder,02,01,020
%%%%%%%%%%%%%%%%%%%%%%%%%%%%%%%%%%%%%%%%%%%%%%%%%%%%%%%%%%%%%%%%%%%%%%%%%%%%%%%

% Fascicule "I"
% Edition "Wydanie II przejrzane i uzupełnione przez Marię Błońską"
% Publisher "Instytut Badań Literackich Polskiej Akademii Nauk — Biblioteka Narodowa"
% Addres "Warszawa"
% Year "1968"
% Note "2. Pismo nagłówkowe gotyckie. Krój M⁸⁹. Stopień 20 ww. = 111/113 mm - Tabl. 20bis. (Występuje u Hallera jako pismo 4. Tabl. 167) [20bis]"
% Note1 "Character set table prepared by Maria Błońska and Anna Wolińska"


python3 renumber_glyphs.py glyphs-edited/m10 glyph-test
python3 glyphids2tex.py glyph-test/ glyphs4tex names.csv
python3 glyph2tex.py glyph-test glyphs4tex/

git:
cd glyphs-final/
cd ../glyphs4tex/

???:
https://www.wbc.poznan.pl/dlibra/publication/504801/edition/473399?language=pl
Priorum analyticorum Aristotelis [...] libri duo castigate. Impressi secundum exemplar Jacobi Stapulensis
Haller, Jan
1510

https://en.wikipedia.org/wiki/Traditional_point-size_names
https://ewangelie.uw.edu.pl/przeklady
  
*ITESTAMENTUM NOVUM. Trad. polon. Stanisłaus Murzynowski]: Testamentu Nowego część
pierwsza. 4. K. 184. (Arkusze A—B A—B Z aa—zz
aaa—rrr A). Druk czarno-czerwony. — Estr. XIII 26.
Wierzb. 1288. Piek. Kórn. 103. Boh. Ossol. 173.
Arkusze A—Y w poz. 4. |
Pisma 1—3, 5, 6, 8, 10, 12 (sporadycznie majuskuły). — Cyfry 1,
2, 5. — Przerywniki 1, 5. — Inicjały 1, 3—12, 28. — Drzeworyty
2—5. [6.

część druga!
https://wbc.poznan.pl/dlibra/publication/516957/edition/472068?language=pl


\begin{description}
\item[3]
  \begin{figure}[H]
    \centering
%    \PTfont{1}{0.9}{Hochfeder-03_PT01_021.png}
    \caption{\Junicode Hochfeder: 3. Gothic text script [Pismo
      tekstowe gotyckie]. Typeface [Krój] M⁸⁸. Type size [Stopień] 20
      lines = 76/77 mm - Plate 20, 26. (Occurs in Haller's inventory
      as font 7. Plate. 170)}
    \label{fig:Hochfeder-03_PT01_021.png}
  \end{figure}

Glyphs: 123.

\end{description}

  
  
%   *[TESTAMENTUM NOVUM. Evangelium secundum Matthaeum. Trad. polon. Stanislaus Murzynowski]: Ewangelia św. Mateusza. 4�. K. 104.
% (Arkusze A—B A—B A—Y). Druk czarno-czerwony. — Estr. XIII 26. Wierzb. 135. Arkusze
% Z aa—kk zob. poz. 4.
% W egz. B. Nar. XVI Qu. 6471 w pierwszym marginalium na lewym
% marginesie w. 4, zamiast majuskuły A wydrukowano minuskułę a.
% Pisma 1—3, 5—7, 10—12. — Cyfry 1—3, 5. — Przerywnik I, 5. —
% Inicjały 1, 3—8, 28.
% Warsz. B. U. 28.2.4.4. z notatkami J. Maleckiego. [4�




B1a

  \begin{center}
%  \includegraphics[scale=0.5]{van_gogh.jpg}
  \captionof{figure}{Imagen}
  \label{fig:VanGogh}
\end{center}

\end{enumerate}

\end{document}

\begin{description}
\item[3]
  \begin{figure}[H]
    \centering
%    \PTfont{1}{0.9}{Hochfeder-03_PT01_021.png}
    \caption{\Junicode Hochfeder: 3. Gothic text script [Pismo
      tekstowe gotyckie]. Typeface [Krój] M⁸⁸. Type size [Stopień] 20
      lines = 76/77 mm - Plate 20, 26. (Occurs in Haller's inventory
      as font 7. Plate. 170)}
    \label{fig:Hochfeder-03_PT01_021.png}
  \end{figure}

Glyphs: 123.

\end{description}


\usepackage{expex}

% 1 number
% 2 height
% 3 image
% 4 font
% 5 comment
\newcommand{\PTglyph}[3]{\includegraphics[height=#2ex]{glyphs/#3}}
% 1 numer
% 2 font identifier
\newcommand{\PTpismo}[2]{{\tiny #2}}

\lingset{glhangstyle=none}
\defineglwlevels{pismo,nr}
\newcommand{\bg}{\begingl}


% 22+36+33+29+3= 123

\item[Font 4] 
  \begin{figure}[H]
    \centering
    \PTfont{2}{0.9}{Hochfeder-04_PT01_022.png}
    \caption{\Junicode Hochfeder: 4. Pismo tekstowe gotyckie. Krój M⁸⁸. Stopień
      20 ww. = 76/77 mm - Tabl. 20, 26. (Występuje u Hallera jako
      pismo 7. Tabl. 170)}
    \label{fig:Hochfeder-04_PT01_022.png}
  \end{figure}

 Glyphs: 75.

% 14+23+21+17= 75 (198)

\item[Font 5] 
  \begin{figure}[H]
    \centering
    \PTfont{3}{0.9}{Hochfeder-05_PT01_023.png}
    \caption{\Junicode Hochfeder: 5. Pismo tekstowe gotyckie. Krój M⁸⁸. Stopień
      20 ww. = 88/89 mm - Tabl. 23}
    \label{fig:Hochfeder-05_PT01_023.png}
  \end{figure}

  Glyphs: 107.

%  21+39+36+11= 107 (305)

  
\item[Font 6] 
  
  \begin{figure}[H]
    \centering
    \PTfont{4}{0.9}{Hochfeder-06_PT01_024.png}
    \caption{\Junicode Hochfeder: 6. Pismo tekstowe gotyckie. Krój M⁸⁸. Stopień
      20 ww. = 79/80 mm - Tabl. 24}
    \label{fig:Hochfeder-06_PT01_024.png}
  \end{figure}

  Glyphs: 108.

%  25+44+39= 108 (413)

  
\item[Font 7] 
  
  \begin{figure}[H]
    \centering
    \PTfont{5}{0.9}{Hochfeder-07_PT01_024.png}
    \caption{\Junicode Hochfeder: 7. Pismo mszalne gotyckie. Krój M¹⁸. Stopień
      20 ww. = 154/156 mm - Tabl. 24, 25. (Występuje u Hallera jako
      pismo 2. Tabl.165; u Unglera jako pismo13. Tabl. 72)}
    \label{fig:Hochfeder-07_PT01_024.png}
  \end{figure}

Glyphs: 114.
  
% 25+39+37+13= 114 (527)

\item[Font 8] 
  
  \begin{figure}[H]
    \centering
    \PTfont{6}{0.9}{Hochfeder-08_PT01_020.png}
    \caption{\Junicode Hochfeder: 8. Pismo kanonowe gotyckie. Krój M¹⁸. Stopień
      20 ww. = 128/129 mm - Tabl. 20, 26. (Występuje u Hallera jako
      pismo 1. Tabl. 164; u Unglera jako pismo 13. Tabl. 72)}
    \label{fig:Hochfeder-08_PT01_020.png}
  \end{figure}

Glyphs: 114.
  
%  23+39+39+13= 114 (641)

  
\item[Font 9] 
  
  \begin{figure}[H]
    \centering
    \PTfont{7}{0.9}{Hochfeder-09_PT01_027.png}
    \caption{\Junicode Hochfeder: 9. Pismo tekstowe gotyckie. Krój M⁴⁸. Stopień
      20 ww. = 81/82 mm - Tabl. 27. (Występuje u Hallera jako pismo
      5. Tabl. 168; u Unglera jako pismo4. Tabl. 115)}
    \label{fig:Hochfeder-09_PT01_027.png}
  \end{figure}

Glyphs: 111.

%  21+33+30+27= 111 (752)

  
\item[Font 10] 
  
  \begin{figure}[H]
    \centering
    \PTfont{8}{0.9}{Hochfeder-10_PT01_028.png}
    \caption{\Junicode Hochfeder: 10. Pismo komentarzowe gotyckie. Krój
      M¹⁶. Stopień 20 ww. = 69 mm - Tabl. 28}
    \label{fig:Hochfeder-10_PT01_028.png}
  \end{figure}

Glyphs: 107.

% 22+31+27+27= 107 (859)

  
\item[Font 11] 
  
  \begin{figure}[H]
    \centering
    \PTfont{9}{0.9}{Hochfeder-11_PT01_028.png}
    \caption{\Junicode Hochfeder: 11. Pismo mszalne gotyckie. Krój M²³. Stopień
      20 ww. = 154/156 mm - Tabl. 25, 28. (Występuje u Hallera jako
      pismo 3. Tabl. 166; u Unglera jako pismo 11)}
    \label{fig:Hochfeder-11_PT01_028.png}
  \end{figure}

Glyphs: 94.

  
% 23+38+33= 94 (953)
  
\end{description}

\section{Zeszyt III: Ungler ---  pierwsza drukarnia}
\label{sec:zeszyt-iii}

Font tables: 10. Glyphs: 934
  
\begin{description}
\item[Font 1]

  \begin{figure}[H]
    \centering
    \PTfont{11}{0.9}{Ungler1-01_PT03_112.png}
    % \includegraphics[width=0.9\textwidth]{img/Ungler1_pismo01_tab112.png}
    \caption{\Junicode Ungler (1. drukarnia): 1. Pismo tekstowe gotyckie. Krój M⁹¹. Stopień 20 ww. ==
      76/77 mm. — Tabl. 112.}
    \label{fig:Ungler1_pismo01_tab112.png}
  \end{figure}

Glyphs: 120.
  
%  20+29+27+27+5+12=\textbf{120} U1 1

\item[Font 2] 
  \begin{figure}[H]
    \centering
    \PTfont{10}{0.9}{Ungler1-02_PT03_113.png}
%    \includegraphics[width=0.9\textwidth]{img/Ungler1_pismo02_tab113.png}
    \caption{\Junicode Ungler (1. drukarnia): 2. Pismo tekstowe gotyckie. Krój M⁹¹. Stopień 20 ww. ==
      76/77 mm. — Tabl. 113.}
    \label{fig:Ungler1_pismo02_tab113.png}
  \end{figure}

  Glyphs: 122.
  

  % 30+49+37+9=\textbf{125} U1 2
  % total 245

  
\item[Font 3] 
  \begin{figure}[H]
    \centering
    \PTfont{11}{0.9}{Ungler1-03_PT03_114.png}
 %   \includegraphics[width=0.9\textwidth]{img/Ungler1_pismo03_tab114.png}
    \caption{\Junicode Ungler (1. drukarnia): 3. Pismo tekstowe gotyckie. Krój M⁹¹. Stopień 20
      ww. == 72/73 mm. — Tabl. 114.}
    \label{fig:Ungler1_pismo03_tab114.png}
  \end{figure}

Glyphs: 129
  

  
  % 23+35+31+30+10=129
  % total 374

  
\item[Font 4] 
  
  \begin{figure}[H]
    \centering
    \PTfont{12}{0.9}{Ungler1-04_PT03_115.png}
%    \includegraphics[width=0.9\textwidth]{img/Ungler1_pismo04_tab115.png}
    \caption{\Junicode Ungler (1. drukarnia): 4. Pismo tekstowe gotyckie. Krój M⁴⁸. Stopień 20
      ww. == 79/81 mm. — Tabl. 115.}
    \label{fig:Ungler1_pismo04_tab115.png}
  \end{figure}

Glyphs: 77.
  
  % 23+31+23=77
  % total 451

\item[Font 5] 
  
  \begin{figure}[H]
    \centering
    \PTfont{13}{0.9}{Ungler1-05_PT03_116.png}
 %   \includegraphics[width=0.9\textwidth]{img/Ungler1_pismo05_tab116.png}
    \caption{\Junicode Ungler (1. drukarnia): 5. Pismo tekstowe gotyckie. Krój M⁴⁸. Stopień 20
      ww. == 89/90 mm. — Tabl. 116.}
    \label{fig:Ungler1_pismo05_tab116.png}
  \end{figure}

Glyphs: 95.

  
  % 24+40+31=95
  % total 546

\item[Font 6] 
  
  \begin{figure}[H]
    \centering
    \PTfont{14}{0.9}{Ungler1-06_PT03_117.png}
  %  \includegraphics[width=0.9\textwidth]{img/Ungler1_pismo06_tab117.png}
    \caption{\Junicode Ungler (1. drukarnia): 6. Pismo tekstowe gotyckie (stany a i b). Krój M⁴⁸. Stopień 20 ww. ==
      85/86 mm. — Tabl. 117.}
    \label{fig:Ungler1_pismo06_tab117.png}
  \end{figure}

Glyphs: 131.
  
  % 30+38+37+26=131
  % total 677

\item[Font 7]
  
  \begin{figure}[H]
    \centering
    \PTfont{15}{0.9}{Ungler1-07_PT03_118.png}
   % \includegraphics[width=0.9\textwidth]{img/Ungler1_pismo07_tab118.png}
    \caption{\Junicode Ungler (1. drukarnia): 7. Pismo nagłówkowe i tekstowe gotyckie. Krój M¹⁸ raz przekreślone.
      % przekreślone we wzorze?
      Stopień 20 ww. == 104/106 mm. — Tabl. 118.}
    \label{fig:Ungler1_pismo07_tab118.png}
%    20+30+28+28=106
  \end{figure}

  Glyphs: 106.

\item[Font 8] 
  
  \begin{figure}[H]
    \centering
    \PTfont{16}{0.9}{Ungler1-08_PT03_120.png}
    %\includegraphics[width=0.9\textwidth]{img/Ungler1_pismo08_tab120.png}
    \caption{\Junicode Ungler (1. drukarnia): 8. Pismo nagłówkowe gotyckie. Krój M⁶⁹. Stopień
      10 ww. == ok. 80mm. — Tabl. 120.}
    \label{fig:Ungler1_pismo08_tab120.png}
  \end{figure}

 Glyphs: 42.

  
  % 15+27=42
  % total 719

\item[Font 9] 
  
  \begin{figure}[H]
    \centering
%    \includegraphics[width=0.9\textwidth]{img/Ungler1_pismo09_tab120.png}
    \PTfont{17}{0.9}{Ungler1-09_PT03_120.png}
    \caption{\Junicode Ungler (1. drukarnia): 9.  Pismo nagłówkowe gotyckie. Krój M²³. Stopień
      10 ww. == ok. 70 mm. — Tabl. 120.}
    \label{fig:Ungler1_pismo09_tab120.png}
  \end{figure}

   Glyphs: 107.

  % 22+28+27+20+10=107
  % total 826

  
\item[10]
  
  \begin{figure}[H]
    \centering
    \PTfont{18}{0.9}{Ungler1-10_PT03_119.png}
%    \includegraphics[width=0.9\textwidth]{img/Ungler1_pismo10_tab119.png}
    \caption{\Junicode Ungler (1. drukarnia): 10. Pismo tekstowe antykwowe. Krój Qu/ pośredni między G i
      K⁵. Stopień 20 ww. == 95/96 mm. — Tabl. 119.}
    \label{fig:Ungler1_pismo10_tab119.png}
  \end{figure}

     Glyphs: 107.

  % 11.
  % Pismo mszalne małe gotyckie. Krój M²³ Stopień 20 ww. — 135/136 mm.
  % (Występuje u Hochfedera jako pismo 11. Tabl. 24, u Hallera jako pismo 3.
  % Tabl. 53).

  % sprawdzić Hochfeder nie ta tablica???

  % 12
  % Pismo mszalne duże gotyckie. Krój M¹⁸. Stopień 20 ww. — 160 mm.
  % Tabl. 72. (Występuje u Hochfedera jako pismo 7. Tabl. 24, u Hallera
  % jako pismo 2. Tabl. 30).

  % u Hochfera jest tablica!

  % sprawdzic

  % Anna Wolińska

  % 13.
  % Pismo kanonowe gotyckie. Krój M¹⁹. Stopień  20 ww. — 135 mm. —
  % Tabl. 72. (Występuje u Hochfedera jako pismo 8. Tabl. 25, u Hallera
  % jako pismo 1. Tabl. 30).

  % sprawdzić
  % Tab.72 bez tabeli!

  % 23+28+27+20+10=108

  % Ungler1 total 934
\end{description}

\section{Zeszyt IV: Haller}
\label{sec:zeszyt-iv}

Font tables: 14. Glyphs: 1287.

% BRAK 154 inicjały?

% 159 inicjały?


\begin{description}
\item[Font 1]

  \begin{figure}[H]
    \centering
    \PTfont{19}{0.9}{Haller-01_PT04_164.png}
 %   \includegraphics[width=0.9\textwidth]{img/Haller_pismo01_tab164}
    \caption{\Junicode Haller: 1. Pismo kanonowe. Krój M¹⁹, Stopień 10 ww. =
      128/129 mm. — Tabl. 164.  (Występuje u Hochfedera jako pismo
      8. Tabl. 25, u Unglera jako pismo 13.  Tabl. 72).}
    \label{fig:Haller_pismo01tab164}
  \end{figure}
  Glyphs: 85.

%   16+26+27+16=85

\item[Font 2] 
  \begin{figure}[H]
    \centering
    \PTfont{20}{0.9}{Haller-02_PT04_165.png}
%    \includegraphics[width=0.9\textwidth]{img/Haller_pismo02_tab165}
    \caption{\Junicode Haller: 2. Pismo mszalne większe. Krój M¹⁸. Stopień 10 ww. = 77 mm. — Tabl. 165.
      (Występuje u Hochfedera jako pismo 7. Tabl. 24, u Unglera jako pismo 12.
      Tabl. 72).}
    \label{fig:Haller_pismo02tab165}
  \end{figure}

  Glyphs: 111.
  
  % 21+40+37+13=111
  % total 196

  
\item[Font 3] 
  \begin{figure}[H]
    \centering
    \PTfont{21}{0.9}{Haller-03_PT04_166.png}
  %  \includegraphics[width=0.9\textwidth]{img/Haller_pismo03_tab166}
    \caption{\Junicode Haller: 3. Pismo mszalne mniejsze. Krój M²³. Stopień 20 ww. = 136/137 mm. —
      Tabl. 166. (Występuje u Hochfedera jako pismo 11. Tabl. 24, u Unglera
      jako pismo 11).}
    \label{fig:Haller_pismo03tab166}
  \end{figure}

  Glyphs: 111.

  
  % 23+39+43+16=121
  % total 317
\item[Font 4] 
  \begin{figure}[H]
    \centering
    \PTfont{22}{0.9}{Haller-04_PT04_167.png}
 %   \includegraphics[width=0.9\textwidth]{img/Haller_pismo04_tab167}
    \caption{\Junicode Haller: 4. Pismo nagłówkowe. Krój M⁸³. Stopień 20 ww. =112/114 mm. —
      Tabl. 167. (Występuje u Hochfedera jako pismo 2. Tabl. 19).}
    \label{fig:Haller_pismo04tab167}
  \end{figure}

  Glyphs: 135.

  
  % 25+40?+39+31=135
  % total 452

\item[Font 5] 
  \begin{figure}[H]
    \centering
    \PTfont{23}{0.9}{Haller-05_PT04_168.png}
%    \includegraphics[width=0.9\textwidth]{img/Haller_pismo05_tab168}
    \caption{\Junicode Haller: 5. Pismo tekstowe. Krój M⁴⁸. Stopień 20 ww. =81/82 mm. —
      Tabl. 168.  (Występuje u Hochfedera jako pismo 9. Tabl. 26, u
      Unglera jako pismo 4.  Tabl. 115).}
    \label{fig:Haller_pismo05tab168}
  \end{figure}

 Glyphs: 135.

  % 27+41+39+28=135
  % total 587

\item[Font 6] 
  \begin{figure}[H]
    \centering
    \PTfont{24}{0.9}{Haller-06_PT04_169.png}
    %\includegraphics[width=0.9\textwidth]{img/Haller_pismo06_tab169}
    \caption{\Junicode Haller: 6. Pismo komentarzowe. Krój M”. Stopień 20 ww. =
      82 mm (interliniowane). — Tabl. 169. (Występuje u Hochfedera
      jako pismo 4. Tabl. 21).}
    \label{fig:Haller_pismo06tab169}
    % 19+45+29=93\\
    % total 680
  \end{figure}

   Glyphs: 93.

  
\item[Font 7]
  
  \begin{figure}[H]
    \centering
    \PTfont{25}{0.9}{Haller-07_PT04_170.png}
%    \includegraphics[width=0.9\textwidth]{img/Haller_pismo07_tab170}
    \caption{\Junicode Haller: 7. Pismo tekstowe. Krój M⁸⁸. Stopień 20 ww. =82
      mm. — Tabl. 170.}
    \label{fig:Haller_pismo07tab170}
%    22+49+47+1+10=129\\total 809
  \end{figure}

   Glyphs: 129.

   
 \item[Font 8] 

  \begin{figure}[H]
    \centering
    \PTfont{26}{0.9}{Haller-08_PT04_171.png}
    %\includegraphics[width=0.9\textwidth]{img/Haller_pismo08_tab171}
    \caption{\Junicode Haller: 8. Pismo tekstowe. Krój M⁹¹. Stopień 20 ww. = 72 mm. —
      Tabl. 171.}
    \label{fig:Haller_pismo08tab171}
%    24+44+43+2=113\\ total 922
  \end{figure}

 Glyphs: 113.

 
\item[Font 9] 
  
  \begin{figure}[H]
    \centering
    \PTfont{27}{0.9}{Haller-09_PT04_172.png}
   % \includegraphics[width=0.9\textwidth]{img/Haller_pismo09_tab172}
    \caption{\Junicode Haller: 9. Pismo tekstowe. Krój M⁴⁹. Stopień 20 ww. = 72
      mm. — Tabl. 172.}
    \label{fig:Haller_pismo09tab172}
%    22+38+36+20=116\\ total 1038
  \end{figure}

 Glyphs: 116.

\item[Font 10] 
  
  \begin{figure}[H]
    \centering
    \PTfont{28}{0.9}{Haller-10_PT04_173.png}
  %  \includegraphics[width=0.9\textwidth]{img/Haller_pismo10_tab173}
    \caption{\Junicode Haller: 10. Pismo komentarzowe. Krój M⁸⁷. Stopień 20 ww. = 72 mm. —
      Tabl. 173.}
    \label{fig:Haller_pismo10tab173}
%    22+40+40=102\\ total 1140
  \end{figure}

 Glyphs: 102.

 
\item[Fonts 11a,11b]  
  \begin{figure}[H]
    \centering
    \PTfont{29}{0.9}{Haller-11_PT04_174.png}
    \PTfont{29}{0.9}{Haller-11pl_PT04_174.png}
    % \includegraphics[width=0.9\textwidth]{img/Haller_pismo11a_tab174}
    % \includegraphics[width=0.9\textwidth]{img/Haller_pismo11b_tab174}
    \caption{\Junicode Haller: 11. Pismo tekstowe antykwowe. Krój Qlu (C). Stopień 20
      ww. = 92 mm. — - Tabl. 174.}
    \label{fig:Haller_pismo11tab174}
  %  25+40+36+10+1+46=158\\ total 1298
  \end{figure}

 Glyphs: 158.

\item[Font 12] 
  
  \begin{figure}[H]
    \centering
    \PTfont{30}{0.9}{Haller-12_PT04_175.png}
 %   \includegraphics[width=0.9\textwidth]{img/Haller_pismo12_tab175}
    \caption{\Junicode Haller: 12. Pismo mszalne. Krój M⁶⁰. Stopień 20 ww. = 176 mm. —
      Tabl. 175.}
    \label{fig:Haller_pismo12tab175}
 %   22+37+37+4=100\\ total 1398
  \end{figure}

 Glyphs: 100.

 
\item[Font 13] 
  
  \begin{figure}[H]
    \centering
    \PTfont{31}{0.9}{Haller-13_PT04_172.png}
%    \includegraphics[width=0.9\textwidth]{img/Haller_pismo13_tab172}
    \caption{\Junicode Haller: 13.  Pismo tekstowe. Krój M¹⁸. Stopień 20 ww. = 112 mm
      (interliniowane). — Tabl. 172. (Występuje u Wietora, w drukarni
      wiedeńskiej).}
    \label{fig:Haller_pismo13tab172}
%    22+36+37+10=105\\total 1503
  \end{figure}

   Glyphs: 105.
  
%  14 i 15 bez zestawów
\end{description}


\section{Zeszyt V: Ungler --- druga drukarnia}
\label{sec:zeszyt-v}

Font tables: 11. Glyphs: 1047.

\begin{description}
\item[Font 1] \begin{figure}[H]
    \centering
      \PTfont{33}{0.9}{Ungler2-01_PT05_239.png}
%    \includegraphics[width=0.9\textwidth]{img/Ungler2_pismo01_tab239.png}
    \caption{\Junicode Ungler (2. drukarnia): 1. Pismo nagłówkowe i tekstowe, rotunda M²³. Stopień
      20 ww. = 132 mm. — Tabl. 239.}
    \label{fig:Ungler2_pismo01_tab239.png}
%    19+7+32+31+10=99
  \end{figure}
Glyphs: 99.
  
\item[Font 2] 
  \begin{figure}[H]
    \centering
      \PTfont{34}{0.9}{Ungler2-02_PT05_240.png}
 %   \includegraphics[width=0.9\textwidth]{img/Ungler2_pismo02_tab240.png}
    \caption{\Junicode Ungler (2. drukarnia): 2. Pismo tekstowe, antykwa Q/u (zbliżone do
      F⁵). Stopień 20 ww. =92— 94 mm. — Tabl. 240.}
    \label{fig:Ungler2_pismo02_tab240.png}
%    20+25+37+35+33=150\\ total 249
  \end{figure}

  Glyphs: 150.


\item[Font 3] 
  \begin{figure}[H]
    \centering
      \PTfont{35}{0.9}{Ungler2-03_PT05_241.png}
  %  \includegraphics[width=0.9\textwidth]{img/Ungler2_pismo03_tab241.png}
    \caption{\Junicode Ungler (2. drukarnia): 3. Pismo tekstowe, rotunda M⁴⁹, Stopień 20 ww. =
      62/63 mm. — Tabl. 241.}
    \label{fig:Ungler2_pismo03_tab241.png}
%    27+42+39+39+10=157\\ total 406
  \end{figure}

 Glyphs: 157.


\item[Font 4] 
  \begin{figure}[H]
    \centering
      \PTfont{36}{0.9}{Ungler2-04_PT05_242.png}
   % \includegraphics[width=0.9\textwidth]{img/Ungler2_pismo04_tab242.png}
    \caption{\Junicode Ungler (2. drukarnia): 4. Pismo tekstowe i nagłówkowe, szwabacha
      M⁸¹. Stopień 20 ww. = = 103 mm. — Tabl. 242.}
    \label{fig:Ungler2_pismo04_tab242.png}
%    24+31+21+31+29=136\\ total 542
  \end{figure}

   Glyphs: 136.


\item[Font 5] 
  \begin{figure}[H]
    \centering
      \PTfont{37}{0.9}{Ungler2-05_PT05_243.png}
    %\includegraphics[width=0.9\textwidth]{img/Ungler2_pismo05_tab243.png}
    \caption{\Junicode Ungler (2. drukarnia): 5. Pismo tekstowe, antykwa Q/u (C). Stopień 20 ww. =94—95
      mm. — Tabl. 243.}
    \label{fig:Ungler2_pismo05_tab243.png}
%    21+31+29=81\\ total 623
  \end{figure}

   Glyphs: 81.


\item[Font 6] 
  \begin{figure}[H]
    \centering
      \PTfont{38}{0.9}{Ungler2-06_PT05_241.png}
    %\includegraphics[width=0.9\textwidth]{img/Ungler2_pismo06_tab241.png}
    \caption{\Junicode Ungler (2. drukarnia): 6. Pismo tekstowe, rotunda M¹⁸. Stopień 20 ww. = 76
      mm. — Tabl. 241.}
    \label{fig:Ungler2_pismo06_tab241.png}
%    20+33+34+8=95\\ total 718
  \end{figure}


   Glyphs: 95.

  % Tylko teksty przykładowe

  % 7. Pismo tytułowe, fraktura. Stopień 1 w. =11 mm. —
  % Tabl. 218, 219.}

  % 7. . Pismo tytułowe, fraktura. Stopień 1 w. =11 mm. — Tabl. 218, 219.
  % 8. Pismo nagłówkowe, rotunda M* (1). Stopień 1 w. =4/5 mm bez przedłużek. — Tabl. 218.

\item[Font 9] 
  \begin{figure}[H]
    \centering
      \PTfont{39}{0.9}{Ungler2-09_PT05_244.png}
   % \includegraphics[width=0.9\textwidth]{img/Ungler2_pismo09_tab244.png}
    \caption{\Junicode Ungler (2. drukarnia): 9. Pismo tekstowe, szwabacha zromanizowana
      M⁸¹. Stopień 20 ww. = = 80/81 mm. — Tabl. 244.}
    \label{fig:Ungler2_pismo09_tab244.png}
%    32+49+52+10=143\\ total 861
  \end{figure}

   Glyphs: 143.


\item[Font 10] 
  \begin{figure}[H]
    \centering
      \PTfont{40}{0.9}{Ungler2-10_PT05_245.png}
   % \includegraphics[width=0.9\textwidth]{img/Ungler2_pismo10_tab245.png}
    \caption{\Junicode Ungler (2. drukarnia): 10. Pismo tekstowe, rotunda M¹⁸. Stopień 20 ww. = 66
      mm. — Tabl. 245.}
    \label{fig:Ungler2_pismo10_tab245.png}
%    25+36+32+33=126\\ total 987
  \end{figure}

   Glyphs: 126.


\item[Font 11] 
  \begin{figure}[H]
    \centering
      \PTfont{41}{0.9}{Ungler2-11_PT05_245.png}
   % \includegraphics[width=0.9\textwidth]{img/Ungler2_pismo11_tab245.png}
    \caption{\Junicode Ungler (2. drukarnia): 11. Pismo tekstowe, antykwa. Stopień 1 w. = 3 mm. —
      Tabl. 245.}
    \label{fig:Ungler2_pismo11_tab245.png}
%    19\\ total 1006
  \end{figure}

   Glyphs: 19.


\item[Font 12] 
  \begin{figure}[H]
    \centering
      \PTfont{42}{0.9}{Ungler2-12_PT05_243.png}
  %  \includegraphics[width=0.9\textwidth]{img/Ungler2_pismo12_tab243.png}
    \caption{\Junicode Ungler (2. drukarnia): 12. Wersaliki tytułowe, antykwa. Wysokość 8 mm. —
      Tabl. 243.}
    \label{fig:Ungler2_pismo12_tab243.png}
%    18\\ total 1024
  \end{figure}

   Glyphs: 18.


\item[Font 13] 
  \begin{figure}[H]
    \centering
      \PTfont{43}{0.9}{Ungler2-13_PT05_243.png}
   % \includegraphics[width=0.9\textwidth]{img/Ungler2_pismo13_tab243.png}
    \caption{\Junicode Ungler (2. drukarnia): 13. Wersaliki tytułowe, antykwa. Wysokość 8 mm. —
      Tabl. 243.}
    \label{fig:Ungler2_pismo13_tab243.png}
%    15=8=23\\ total 1047
  \end{figure}

 Glyphs: 23.

\end{description}

\section{Zeszyt VI: Ungler --- druga drukarnia}
\label{sec:zeszyt-vi}

No font tables.

%tylko drzeworyty (brak skanów)?


\section{Zeszyt VII: Ungler --- druga drukarnia} 
\label{sec:zeszyt-vii}

Font tables: 10. Glyphs: 930.

\begin{description}
\item[14] \begin{figure}[H]
    \centering
    \PTfont{44}{0.9}{Ungler2-14_PT07_357.png}
%    \includegraphics[width=0.9\textwidth]{img/Ungler2_pismo14_tab357}
    \caption{\Junicode Ungler (2. drukarnia):  14.  Pismo tytułowe i nagłówkowe,
      fraktura Neudörffer-Andreae. Stopień 5 ww. = 65 mm. — Tabl. 357.}
    \label{fig:Ungler2_pismo14_tab357}
%    27+38+36+28=129
  \end{figure}

   Glyphs: 129.


\item[15] 
  \begin{figure}[H]
    \centering
    \PTfont{45}{0.9}{Ungler2-15_PT07_358.png}
 %   \includegraphics[width=0.9\textwidth]{img/Ungler2_pismo15a-b_tab358}
    \caption{\Junicode Ungler (2. drukarnia):  15 a-b.Pismo tekstowe, antykwa
      Qu|(G). Stopień 20 ww. = 91 mm. — Tabl. 358.}
    \label{fig:Ungler2_pismo15a-btab358}
%    22+35+29+23=109\\ razem 238
  \end{figure}

   Glyphs: 109..


\item[16] 
  \begin{figure}[H]
    \centering
    \PTfont{46}{0.9}{Ungler2-16_PT07_359.png}
  %  \includegraphics[width=0.9\textwidth]{img/Ungler2_pismo16_tab359}
    \caption{\Junicode Ungler (2. drukarnia) 16. Pismo tekstowe, rotunda zbliżona do M⁵⁰ (druga
      forma). Stopień 20 ww. = 65 mm. — Tabl. 359.}
    \label{fig:Ungler2_pismo16tab359}
%    29+44+44+12+5=134\\ razem 372
  \end{figure}

 Glyphs: 134.

 % total 372
  
\item[17] 
  \begin{figure}[H]
    \centering
    \PTfont{47}{0.9}{Ungler2-17_PT07_357.png}
   % \includegraphics[width=0.9\textwidth]{img/Ungler2_pismo17_tab357}
    \caption{\Junicode Ungler (2. drukarnia) 17. Pismo tekstowe, rotunda M⁸⁷. Stopień 20 ww. = 75 mm. —
      Tabl. 357.}
    \label{fig:Ungler2_pismo17tab357}
%    18+5+32+9=64\\ razem 436
  \end{figure}

 Glyphs: 64.
 

\item[18] 
  \begin{figure}[H]
    \centering
    \PTfont{48}{0.9}{Ungler2-18_PT07_360.png}
    %\includegraphics[width=0.9\textwidth]{img/Ungler2_pismo18_tab360}
    \caption{\Junicode Ungler (2. drukarnia) 18. Pismo nagłówkowe i tekstowe, fraktura
      Neudörffer-Andreae. Stopień 10 ww. == 73 mm. — Tabl. 360.}
    \label{fig:Ungler2_pismo18tab360}
%    25+41+37+10=113\\ razem 549
  \end{figure}

   Glyphs: 113.


\item[19] 
  \begin{figure}[H]
    \centering
    \PTfont{49}{0.9}{Ungler2-19_PT07_360.png}
%    \includegraphics[width=0.9\textwidth]{img/Ungler2_pismo19_tab360}
    \caption{\Junicode Ungler (2. drukarnia) 19. Pismo tekstowe, fraktura Neudörffer-Andreae. Stopień 20
      ww. = 90—91 mm. — Tabl. 360.}
    \label{fig:Ungler2_pismo19tab360}

    Glyphs: 83.
%    25+42+6+10=83\\razem 632
  \end{figure}
\item[20] 
  \begin{figure}[H]
    \centering
    \PTfont{50}{0.9}{Ungler2-20_PT07_361.png}
   % \includegraphics[width=0.9\textwidth]{img/Ungler2_pismo20_tab361}
    \caption{\Junicode Ungler (2. drukarnia) 20. Pismo tekstowe, antykwa Qu|(G). Stopień 20 ww. ==
      90-—91 mm. — Tabl. 361.}
    \label{fig:Ungler2_pismo20tab361}
%    26+44+34+26=130\\ razem 762
  \end{figure}

    Glyphs: 130.


\item[21] 
  \begin{figure}[H]
    \centering
    \PTfont{51}{0.9}{Ungler2-21_PT07_362.png}
   % \includegraphics[width=0.9\textwidth]{img/Ungler2_pismo21_tab362}
    \caption{\Junicode Ungler (2. drukarnia) 21. Pismo tekstowe, antykwa Qu|(H). Stopień 20 ww. = 113
      mm. — Tabl. 362.}
    \label{fig:Ungler2_pismo21tab120}
%    19+30+29=78\\ razem 762
  \end{figure}

   Glyphs: 78.
  
% 840
\item[22] 
  \begin{figure}[H]
    \centering
    \PTfont{52}{0.9}{Ungler2-22_PT07_362.png}
    \caption{\Junicode Ungler (2. drukarnia): 22. Pismo tytułowe i nagłówkowe, antykwa Q|u(C). Wysokość 1 ww. = 9 mm. - Tabl. 362."}
    \label{fig:Ungler2-22_PT07_362.png}
  \end{figure}
  % 14+21+15+ 
   Glyphs: 50.


\item[23] 

  \begin{figure}[H]
    \centering
    \PTfont{53}{0.9}{Ungler2-23_PT07_363.png}
    \caption{\Junicode Ungler (2. drukarnia): 23. Wersaliki tytułowe,
      antykwa Qu|(K3. Wysokość 7—8 mm. — Tabl. 363.}

    \label{fig:Ungler2-23_PT07_363.png}
  \end{figure}
  % 21
  Glyphs: 21.

    \begin{figure}[H]
    \centering
    \PTfont{53}{0.9}{Ungler2-24_PT07_363.png}
    \caption{\Junicode Ungler (2. drukarnia): 24. Wersaliki tytułowe,
      antykwa. Wysokość 8—9 mm. —- Tabl. 363.  Pismo greckie,
      tekstowe. Stopień 5 22/23 Tabl. 363.}
    \label{fig:Ungler2-24_PT07_363.png}
  \end{figure}

     Glyphs: 19.


\end{description}


\section{IX: Wirzbięta}
\label{sec:ix}

Font tables 5. Glyphs: 448.



\begin{description}
\item[Font 1]

  \begin{figure}[H]
    \centering
    \PTfont{73}{0.9}{Wirzbięta-01_PT09_469.png}\\
    \PTfont{73}{0.17}{Wirzbięta-01u_PT11_570.png}
    \caption{\Junicode Wirzbięta: 1. Pismo nagłówkowe, fraktura
      H. Schönspergera. Stopień 1 w, = 14,5 mm. — Tabl. 417, 419, 420,
      457, 464, 467, \textbf{469}, 471. Uzupełnienia Tab. \textbf{570}}
    \label{fig:Wirzbięta-01_PT09_469.png}

  \end{figure}

    % 19+7+29+30+4=89
  Glyphs: 89.

\item[2]
  
\begin{figure}[H]
    \centering
    \PTfont{74}{0.9}{Wirzbięta-02_PT09_469.png}\\
    \PTfont{73}{0.4}{Wirzbięta-02u_PT11_570.png}
    \caption{\Junicode Wirzbięta: 2. Pismo nagłówkowe i tekstowe,
      fraktura H. Schönspergera. Stopień 20 ww. = ca 160 mm. —
      Tabl. 420, 464, \textbf{469}. Uzupełnienia Tab. \textbf{570}}
    \label{fig:Wirzbięta-02_PT09_469.png}
  \end{figure}

  % 22+38+12+12=84
  Glyphs: 84.
  
\item[3] 
  
\begin{figure}[H]
    \centering
    \PTfont{75}{0.9}{Wirzbięta-03_PT09_469.png}\\
    \PTfont{73}{0.045}{Wirzbięta-03u_PT11_570.png}
    \PTfont{73}{0.25}{Wirzbięta-03uu_PT11_570.png}
    \caption{\Junicode Wirzbięta: 3. Pismo tekstowe, szwabacha M⁸¹,
      Stopień 20 ww. = 113 mm. — Tabl. 417, 419, 420, 456—458, 464,
      467—470 [\textbf{469}]. Uzupełnienia Tab. \textbf{570}}
    \label{fig:Wirzbięta-03_PT09_469.png}
  \end{figure}

  % 25+38+32+1=96
  Glyphs: 96.

  
\item[4] 
  
\begin{figure}[H]
    \centering
    \PTfont{75}{0.9}{Wirzbięta-04_PT09_469.png}\\
    \PTfont{73}{0.15}{Wirzbięta-04u_PT11_570.png}
    \caption{\Junicode Wirzbięta: 4. Pismo tekstowe, szwabacha M⁸¹,
      Stopień 20 ww. == 88 mm. — Tabl. 457, 468—470
      [\textbf{469}].Uzupełnienia Tab. \textbf{570}}
    \label{fig:Wirzbięta-04_PT09_469.png}
  \end{figure}

    % 24+41+35=100
  Glyphs: 100.

\item[5] 
  
\begin{figure}[H]
    \centering
    \PTfont{75}{0.9}{Wirzbięta-05_PT09_469.png}\\
    \PTfont{73}{0.245}{Wirzbięta-05u_PT11_570.png}
    \caption{\Junicode Wirzbięta: 5. Pismo komentarzowe, szwabacha
      M⁸¹. Stopień 20 ww. = 71/2 mm. — Tabl. 456, 457, 469, 470.
    [\textbf{469}].Uzupełnienia Tab. \textbf{570}}
    \label{fig:Wirzbięta-05_PT09_469.png}
  \end{figure}

    % 23+35+21=79
  Glyphs: 79.

\end{description}

  


% broszura niekompletna + Koszykowa

% PISMA
% 1. Pismo nagłówkowe, fraktura H. Schönspergera. Stopień
% 1 w, = 14,5 mm. — Tabl. 417, 419, 420, 457, 464, 467,
% 469, 471.
% 2. Pismo nagłówkowe i tekstowe, fraktura H. Schönspergera. Stopień 20 ww. = ca 160 mm. — Tabl. 420, 464,
% 469.
% 3. Pismo tekstowe, szwabacha M⁸¹, Stopień 20 ww. =
% 113 mm. — Tabl. 417, 419, 420, 456—458, 464, 467—470.
% 4. Pismo tekstowe, szwabacha M⁸¹, Stopień 20 ww. ==
% 88 mm. — Tabl. 457, 468—470.
% 5. Pismo komentarzowe, szwabacha M⁸¹. Stopień 20 ww. =
% 71/2 mm. — Tabl. 456, 457, 469, 470.
% 6. Czcionki odsyłaczowe, szwabacha, użyte sporadycznie
% w cz. II Postylli polskiej (nr 4). Wysokość 1 mm. — Tabl.
% 470.
% 7. Czcionki hebrajskie. Wysokość 8 mm. — Tabl. 471.
% 8. Pismo tekstowe, antykwa polska ozdobna. Stopień
% 1 w. = 6 mm. — Tabl. 458.
% tabele niewskanowane

\section{X: Wirzbieta}
\label{sec:x}

No font tables.

% tylko drzeworyty i inicjały

% indeks do OCR

% zasób omówiony

% Erratum do zesztu IX !!!!

% s. [19]

% PISMA

% 9 Antykwa tekstowa. Stopień 20 ww. =
% 92 mm. — Tabl. 5rą.
% 10. Antykwa tekstowa. Stopień 20 ww. =
% 114 mm. — Tabl. 494, 514.
% 11 Wersaliki tytułowe, antykwa. Wysokość
% 7/8 mm. — Tabl. 514.
% 12 Antykwa nagłówkowa. Stopień 5 ww. =
% 70 mm, wersaliki wysokość 10 mm. — Nie re-
% produkowana.
% 13 Kursywa komentarzowa. Stopień 20 ww. =
% 76 mm. — Nie reprodukowana.
% 14 Kursywa tekstowa. Stopień 20 ww. =
% 88/89 mm. — Tabl. 487, 493, 494.
% 15 Kursywa tekstowa. Stopień 20 ww. =
% 113/114 mm. — Tabl. 494, 514.
% 16Pismo greckie komentarzowe. Wysokość czcion-
% ki 2 mm. — Nie reprodukowane.

% tabele do wycięcia?



\section{XI: Wirzbiętowie}
\label{sec:xi}

Font tables 7. Glyphs: 545.


% * W niniejszym wykazie podano pełną rejestrację
% tylko tych elementów, które występują na tablicach we
% wszystkich trzech Zeszytach (IX—XI), zawierających
% monografię drukarni Wirzbięty. Jeżeli natomiast jedno-
% rodne gatunkowo elementy (np. winietki, drzeworyty)
% reprodukowano głównie w Żeszycie IX lub X i tam za-
% mieszczono ich wykaz, później zaś pojawiały się tylko spo-
% radycznie, zestaw obejmuje grupy przynależne do Ze-
% szytu XI.

% PISMA
% 1. Pismo nagłówkowe, fraktura H. Schönspergera.
% Stopień: I w. = 14,5 mm. — Tabl. 417, 419,
% 420, 457, 464, 467, 469, 471, 479, 486, 491,
% 492, 523, 557 500, 570.

% 2. Pismo nagłówkowe i tekstowe, fraktura H.
% Schönspergera. Stopień 20 ww. = ca 160 mm.
% — Tabl. 420, 464, 469, 476, 479, 486, 491,
% 492, 523, 527, 560, 501, 570.

% 3. Pismo tekstowe szwabacha M⁸¹. Stopień 20 ww.
% = 139 mm. — Tabl. 417, 419, 420, 456-458,
% 464, 467-470, 476, 479, 487, 496, 515, 523, 527,
% 560, 561, 570.

% 4. Pismo tekstowe, szwabacha M⁸¹. Stopień 20
% ww. = 88 mm.— Tabl. 457, 468-470, 476, 479,
% 487, 491, 492, 515, 516, 521, 523, 501, 508, 570.

% 5. Pismo komentarzowe, szwabacha M⁸¹. Stopień
% 20 ww. = 71/2 mm. — Tabl. 456, 457, 469,
% 470, 487, 521, 523, 570.

% 6. Czcionki odsyłaczowe, szwabacha. Wysokość
% 1 mm. — Tabl. 470, 568.

% 7. Czcionki hebrajskie. Wysokość 8 mm. —
% Tabl. 471.

% 8. Antykwa polska dorobiona do pisma 10. —
% Tabl. 458. Zob. p. 10.

% 9. Pismo tekstowe, antykwa. Stopień 20 ww.
% 92 mm. — Tabl. 458, 514, 515, 562, 568.

% 10. Pismo tekstowe, antykwa. Stopień 20 ww. ==
% 113/114 mm. Zob. p. 8. — Tabl. 494, 514,
% 522, 529, 563, 568.

% 11. Wersaliki tytułowe, antykwa. Wysokość 7 mm.
% — Tabl. 514, 522, 529, 540, 563, 564, 568.

% 11a. Wersaliki tytułowe greckie, dorobione do p.
% 11. — Tabl. 564.

% 12. Pismo nagłówkowe, antykwa. Stopień 5 ww.
% = 70 mm, wersaliki ro mm. — Tabl. 522,
% 529-532, 540, 543, 556, 564.

% 13. Pismo komentarzowe, kursywa. Stopień 20 ww.
% = 76 mm. — Tabl. 565.

% 14. Pismo tekstowe, kursywa. Stopień 20 ww.
% = 88/89 mm. — Tabl. 487, 493, 494, 521-523,
% 540, 561, 565, 568.

% 15. Pismo tekstowe, kursywa. Stopień 20 ww.
% = 113/114. mm. — Tabl. 494, 514, 521, 522,
% 566, 567.

% 16. Pismo greckie, tekstowe. Wysokość czcionki
% 2 mm. — Tabl. 567.

% 17. Pismo hebrajskie. Wysokość czcionki 3 mm bez
% przedłużek. — Tabl. 567.

% zasób omówiony

% Errata do IX i X!

% s. 46

% broszura indeks do OCR!

\begin{description}
\item[9] 


\begin{figure}[H]
  \centering
    \PTfont{76}{0.9}{Wirzbięta-09_PT11_562.png}
%  \includegraphics[width=0.9\textwidth]{img/Wirzbieta_pismo09_tab562}
    \caption{\Junicode Wirzbięta: 9. Pismo tekstowe, antykwa. Stopień
      20 ww. 92 mm. — Tabl. 458, 514, 515, 562, 568.}
  \label{fig:Wirzbięta_pismo09tab562}
%  24+35+33+30=122
\end{figure}

Glyphs: 122.

\item[10+8] 
\begin{figure}[H]
  \centering
    \PTfont{77}{0.9}{Wirzbięta-10+8_PT11_563.png}
    % \includegraphics[width=0.9\textwidth]{img/Wirzbieta_pismo10+8_tab563}
    \caption{\Junicode Wirzbięta: 10. Pismo tekstowe, antykwa. Stopień
      20 ww. == 113/114 mm. Zob. p. 8. — Tabl. 494, 514, 522, 529,
      563, 568. 8. Antykwa polska dorobiona do pisma 10. —
      Tabl. 458. Zob. p. 10.}
  \label{fig:Wirzbięta_pismo10_8tab563}
%   24+33+30+18=105\\razem 227
\end{figure}

Glyphs: 105.

\item[11+11a] 
\begin{figure}[H]
  \centering
  \PTfont{78}{0.9}{Wirzbięta-11+11a_PT11_564.png}
  %\includegraphics[width=0.9\textwidth]{img/Wirzbieta_pismo11+11a_tab564}
  \caption{\Junicode Wirzbięta: 11. Wersaliki tytułowe,
    antykwa. Wysokość 7 mm. — Tabl. 514, 522, 529, 540, 563, 564,
    568. [564]. 11a. Wersaliki tytułowe greckie, dorobione do
    p. 11. — Tabl. 564.}
  \label{fig:Wirzbięta_pismo11+11atab564}
%  21+10=31\\razem 258
\end{figure}

Glyphs: 31.

\item[12] 
\begin{figure}[H]
  \centering
  \PTfont{79}{0.9}{Wirzbięta-12_PT11_564.png}
 % \includegraphics[width=0.9\textwidth]{img/Wirzbieta_pismo12_tab564}
  \caption{\Junicode Wirzbięta: 12. Pismo nagłówkowe, antykwa. Stopień
    5 ww. = 70 mm, wersaliki 10 mm. — Tabl. 522, 529-532, 540, 543,
    556, 564.}
  \label{fig:Wirzbięta_pismo12tab564}
%  14+16+25+18=55\\razem 313
\end{figure}

Glyphs: 55.

\item[13] 


\begin{figure}[H]
  \centering
  \PTfont{80}{0.9}{Wirzbięta-13_PT11_565.png}
%\includegraphics[width=0.9\textwidth]{img/Wirzbięta_pismo13_tab565.png}
  \caption{\Junicode Wirzbięta: 13. Pismo komentarzowe,
    kursywa. Stopień 20 ww. = 76 mm. — Tabl. 565.}
  \label{fig:Wirzbięta_pismo13_tab565.png}
%20+35+6+1=62\\razem 659
\end{figure}

Glyphs: 62

\item[14] 
\begin{figure}[H]
  \centering
  \PTfont{81}{0.9}{Wirzbięta-14_PT11_565.png}
 % \includegraphics[width=0.9\textwidth]{img/Wirzbięta_pismo14_tab565.png}
  \caption{\Junicode Wirzbięta: 14. Pismo tekstowe, kursywa. Stopień
    20 ww. = 88/89 mm. — Tabl. 487, 493, 494, 521-523, 540, 561, 565,
    568.}
  \label{fig:Wirzbięta_pismo14_tab565.png}
%  26+39+38+31=134\\razem 597
\end{figure}

Glyphs: 134.

\item[15] 

\begin{figure}[H]
  \centering
  \PTfont{82}{0.9}{Wirzbięta-15_PT11_566.png}
%\includegraphics[width=0.9\textwidth]{img/Wirzbięta_pismo15_tab566.png}
  \caption{\Junicode Wirzbięta: 15. Pismo tekstowe, kursywa. Stopień
    20 ww. = 113/114. mm. — Tabl. 494, 514, 521, 522, 566, 567.}
  \label{fig:Wirzbięta_pismo15_tab566.png}
%  25+34+31+28+10=128\\razem 463
\end{figure}

Glyphs: 128.

% \item[1-5] 
%   \begin{figure}[H]
%   \centering
% \includegraphics[width=0.9\textwidth]{img/Wirzbięta_pismo1-5u_tab570.png}
%   \caption{Wirzbięta_pismo1-5u_tab570.png}
%   \label{fig:Wirzbięta_pismo1-5u_tab570.png}
%   4+12+1+5+10+10=42\\razem  355
% \end{figure}


% 565: pisma 13, 14

% 566: pismo 15

% 567: 16 17 greka

\end{description}
\section{XII: Szarfenberg}
\label{sec:xii}

% \begin{description}
% \item[181] Szarfenberg
% \end{description}

No font tables.

\section{Statistics}
\label{sec:tatystyka}


\begin{itemize}
\item Hochfeder. Font tables: 9. Glyphs: 953.
\item Ungler ---  pierwsza drukarnia. Font tables: 10. Glyphs: 934.
\item Haller. Font tables: 14. Glyphs: 1287.
\item Ungler --- druga drukarnia. Font tables: 11. Glyphs: 1047.
\item Ungler --- druga drukarnia. Font tables: 10. Glyphs: 930.
\item Augezdecki. Font tables 19. Glyphs: 1483.
\item Wirzbięta. Font tables 5. Glyphs: 448.
\item Wirzbiętowie. Font tables 7. Glyphs: 545.
\end{itemize}

Total tables: 75.

Total glyphs: 7627.

% 65 tabel


%   934 Ungler 1 +106
%  1503 Haller
%  1047 Ungler2
%   762 Ungler2 (VII)
%  1475 Auzdecki
%   659 Wirzbięta
% razem 6380 + 106 = 6486



%\clearpage
%\printbibliography{}


%%% Local Variables: 
%%% coding: utf-8-unix
%%% mode: latex
%%% TeX-master: t
%%% TeX-PDF-mode: t
%%% TeX-engine: xetex
%%% End: 
