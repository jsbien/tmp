\documentclass{article}
\usepackage[a4paper,margin=1.5cm]{geometry}
\usepackage{fontspec}
\newfontfamily{\Junicode}{Junicode}
\newcommand{\J}[1]{{\Junicode #1}}
% \usepackage{polyglossia}
% \setmainlanguage{polish}
% \setotherlanguage{english}
%\usepackage{csquotes}

\usepackage{metalogo}
% \usepackage[polish]{varioref}
% % dla varioref!
% \def\eob{ę}
\usepackage{xcolor}


\usepackage{relsize}

\usepackage{float}

\usepackage[verbose]{hyperref}

\usepackage{graphicx}
% [hyphens]: options clash
\usepackage{url}
%\usepackage{natbib}

% program name
\newcommand{\pname}[1]{\textsf{#1}}


% file name
\newcommand{\fname}[1]{\texttt{#1}}

\newcommand{\uname}[1]{\texttt{'#1'}}
\newcommand{\ucode}[1]{\texttt{U+#1}}
\newcommand{\usi}[1]{\texttt{#1}}

% Aletheia
\newcommand{\aname}[1]{\texttt{#1}}
\newcommand{\acode}[1]{\texttt{#1}}

% MUFI
\newcommand{\mname}[1]{\texttt{'#1 \textsc{<mufi>'}}}
\newcommand{\mcode}[1]{\texttt{M+#1}}



%\usepackage{draftwatermark}
\usepackage[doublespacing]{setspace}

\usepackage[draft]{fixme}

% nie działa:?
%\renewcommand{\topfraction}{0.9}
\renewcommand{\floatpagefraction}{0.9}	% require fuller float pages
\renewcommand{\topfraction}{0.9}	% max fraction of floats at top
\setcounter{topnumber}{5}
\setcounter{totalnumber}{5}  

\renewcommand{\labelenumii}{\arabic{enumii}.}

% \vrefwarning

% https://tex.stackexchange.com/questions/54136/hyperref-link-spans-a-pagebreak-looks-ugly
% nie zawsze działa!!!

% retrieve absolute page numbers (physical pages, as opposed to the
% ‘logical’ page number that is normally typeset when a page number is
% requested;
% \usepackage{zref-abspage}

% 1 number
% 2 width fraction
% 3 image
% \newcommand{\PTfont}[3]{\includegraphics[width=#2\textwidth]{/home/jsbien/git/early_fonts_inventory/font_tables/PNG/#3}}
\newcommand{\PTfont}[3]{\includegraphics[width=#2\textwidth]{~/git/early_fonts_inventory/font_tables/PNG/#3}}


\begin{document}
%\gappto\captionslingua{\renewcommand{\chaptername}{Caput}}
%\gappto\captionspolish{\renewcommand{\figurename}{Ilustracja}}


\title{POLONIA TYPOGRAPHICA
  SAECULI SEDECIMI\\
  {\relsize{-2} TŁOCZNIE POLSKIE XVI STULECIA\\ MONOGRAFIE I PODOBIZNY
    ZASOBÓW DRUKARSKICH}\\Summary: Reconstructed font tables\\
  (draft)}

\author{Janusz S. Bień (editor)}

\date{\today}

\maketitle

\catcode`\&=11
\catcode`\|=11
\catcode`\_=11

\def\apostrof{`}


% dodac indeks!:
\catcode`\`=\active
\def`#1{\fbox{{\znak#1}}}

% Font tables: 9. Glyphs: 953.

\begin{enumerate}
\item
  % 01,Augezdecki-01_PT08_402.djvu,Augezdecki,01,08,402

% Author "Paulina Buchwald-Pelcowa"
% Title "Aleksander Augezdecki 1549-1561"
% Editor "Alodia Kawecka Gryczowa"
% Series "Polonia Typographica Saeculi Sedecimi: zbiór podobizn zasobu drukarskiego tłoczni polskich XVI stulecia"
% Fascicule "VIII"
% Publisher "Zakład Narodowy imienia Ossolińskich — Wydawnictwo"
% Addres "Kraków  Wrocław Warszawa"
% Year "1972"
% Note "1. Pisma tekstowe, szwabacha M⁸¹. Stopień 20 ww. = 102—103 mm (tercja). — Tabl. 402—404. [402]"
% Note1 "Character set table prepared by Paulina Buchwald-Pelcowa"
% Note2 "Scan (prepared by Biblioteka Uniwersytecka w Warszawie from their own copy) converted to DjVu with didjvu by Janusz S. Bień"
% URL "https://github.com/jsbien/early_fonts_inventory/"


  \begin{figure}[H]
    \centering
%    \PTfont{1}{0.9}{Hochfeder-03_PT01_021.png}
    \caption{\Junicode Hochfeder: 3. Gothic text script [Pismo
      tekstowe gotyckie]. Typeface [Krój] M⁸⁸. Type size [Stopień] 20
      lines = 76/77 mm - Plate 20, 26. (Occurs in Haller's inventory
      as font 7. Plate. 170)}
    \label{fig:Hochfeder-03_PT01_021.png}
  \end{figure}
  \textbf{Primary example:}

\textit{Testamentu Nowego Czesc Pierwsza Czterzei
    Euangelistowie swieći, Mattheusz, Marek, Lukasz I Ian, Z Greckiego
    ięzyka na Polski przelozeni i wykladem krotkiem obiasnieni},   1551, \url{https://www.dbc.wroc.pl/publication/33779}

Page ???
  
%   *[TESTAMENTUM NOVUM. Evangelium secundum Matthaeum. Trad. polon. Stanislaus Murzynowski]: Ewangelia św. Mateusza. 4�. K. 104.
% (Arkusze A—B A—B A—Y). Druk czarno-czerwony. — Estr. XIII 26. Wierzb. 135. Arkusze
% Z aa—kk zob. poz. 4.
% W egz. B. Nar. XVI Qu. 6471 w pierwszym marginalium na lewym
% marginesie w. 4, zamiast majuskuły A wydrukowano minuskułę a.
% Pisma 1—3, 5—7, 10—12. — Cyfry 1—3, 5. — Przerywnik I, 5. —
% Inicjały 1, 3—8, 28.
% Warsz. B. U. 28.2.4.4. z notatkami J. Maleckiego. [4�


\item
  % Author "Paulina Buchwald-Pelcowa"
% Title "Aleksander Augezdecki 1549-1561"
% Editor "Alodia Kawecka Gryczowa"
% Series "Polonia Typographica Saeculi Sedecimi: zbiór podobizn zasobu drukarskiego tłoczni polskich XVI stulecia"
% Fascicule "VIII"
% Publisher "Zakład Narodowy imienia Ossolińskich — Wydawnictwo"
% Addres "Kraków  Wrocław Warszawa"
% Year "1972"
% Note "2 Pismo tekstowe, szwabacha M⁸¹. Stopień 20 ww. = 86—87 mm (cycero). — Tabl. 405, 406. [405]"
% Note1 "Character set table prepared by Paulina Buchwald-Pelcowa"
% Note2 "Scan (prepared by Biblioteka Uniwersytecka w Warszawie from their own copy) converted to DjVu with didjvu by Janusz S. Bień"
% URL "https://github.com/jsbien/early_fonts_inventory/"
% .

B1a
  
\end{enumerate}

\end{document}

\begin{description}
\item[3]
  \begin{figure}[H]
    \centering
%    \PTfont{1}{0.9}{Hochfeder-03_PT01_021.png}
    \caption{\Junicode Hochfeder: 3. Gothic text script [Pismo
      tekstowe gotyckie]. Typeface [Krój] M⁸⁸. Type size [Stopień] 20
      lines = 76/77 mm - Plate 20, 26. (Occurs in Haller's inventory
      as font 7. Plate. 170)}
    \label{fig:Hochfeder-03_PT01_021.png}
  \end{figure}

Glyphs: 123.

\end{description}

\begin{figure}[ht]
        \centering
      \input brevigraphs_b.tex \lessabovecaptionspace
        \caption{Modified \textit{b}}
        \label{fig:b}
      \end{figure}

      %\begin{figure}
\exdisplay \bg \gla
% Entry: b! Augezdecki-03 001:00042 [E]
\PTglyph{1}{5}{snippet_Augezdecki-03_page001x00289y00324}
% Entry: b! Haller-02 001:00032 [D]
\PTglyph{2}{5}{snippet_Haller-02_page001x00341y00318}
% % Entry: b! Haller-04 001:00036 [b]
% \PTglyph{3}{5}{snippet_Haller-04_page001x00266y00299}
% % Entry: b!0 Haller-04 001:00037 [b]
% \PTglyph{4}{5}{snippet_Haller-04_page001x00325y00299}
% Entry: bstrok Haller-05 001:00040 [D]
\PTglyph{5}{5}{snippet_Haller-05_page1x187y215}
% Entry: bstrok Haller-06 001:00034 [b]
\PTglyph{6}{5}{snippet_Haller-06_page001x00152y00157}
% Entry: bstrok Haller-07 001:00040 [b]
\PTglyph{7}{5}{snippet_Haller-07_page001x00240y00424}
% Entry: bstrok Haller-09 001:00042 [D]
\PTglyph{8}{5}{snippet_Haller-09_page001x00208y00255}
% % Entry: bstrok Hochfeder-03 001:00034 [b]
% \PTglyph{9}{5}{snippet_Hochfeder-03_page001x00347y00378}
% % Entry: bstrok Hochfeder-05 001:00036 [5]
% \PTglyph{10}{5}{snippet_Hochfeder-05_page001x00380y00413}
% % Entry: bstrok Hochfeder-06 001:00039 [B]
% \PTglyph{11}{5}{snippet_Hochfeder-06_page001x00322y00238}
% Entry: bstrok Hochfeder-07 001:00033 [b]
\PTglyph{12}{5}{snippet_Hochfeder-07_page1x339y195}
% Entry: bstrok Hochfeder-08 001:00033 [b]
\PTglyph{13}{5}{snippet_Hochfeder-08_page001x00335y00516}
% Entry: bstrok Hochfeder-08 001:00033 [b]
\PTglyph{14}{5}{snippet_Hochfeder-08_page001x00335y00516}
% Entry: bstrok Hochfeder-09 001:00030 [D]
\PTglyph{15}{5}{snippet_Hochfeder-09_page001x00351y00348}
% Entry: bstrok Hochfeder-09 001:00030 [D]
\PTglyph{16}{5}{snippet_Hochfeder-09_page001x00351y00348}
% Entry: bstrok Hochfeder-10 001:00037 [B5]
\PTglyph{17}{5}{snippet_Hochfeder-10_page001x00451y00531}
% Entry: bstrok Hochfeder-11 001:00034 [b]
\PTglyph{18}{5}{snippet_Hochfeder-11_page001x00290y00319}
% % Entry: bstrok Ungler1-01 001:00026 [D]
% \PTglyph{19}{5}{snippet_Ungler1-01_page001x00213y00425}
% Entry: bstrok Ungler1-02 001:00040 [i]
\PTglyph{20}{5}{snippet_Ungler1-02_page1x158y172}
% Entry: bstrok Ungler1-03 001:00035 [l*]
\PTglyph{21}{5}{snippet_Ungler1-03_page001x00272y00354}
% Entry: bstrok Ungler2-10
\PTglyph{22}{5}{snippet_Ungler2-10_page1x135y252}
% Entry: bstrok Ungler2-14 001:00047 [b]
\PTglyph{23}{5}{snippet_Ungler2-14_page001x00255y00207}
% Entry: bstrok Ungler2-16 001:00038 [b]
\PTglyph{24}{5}{snippet_Ungler2-16_page001x00285y00400}
//
\glpismo
\PTpismo{1}{A-03}
\PTpismo{2}{H-02}
% \PTpismo{3}{H-04}
% \PTpismo{4}{H-04}
\PTpismo{5}{H-05}
\PTpismo{6}{H-06}
\PTpismo{7}{H-07}
\PTpismo{8}{H-09}
% \PTpismo{9}{Hf-03}
% \PTpismo{10}{Hf-05}
% \PTpismo{11}{Hf-06}
\PTpismo{12}{Hf-07}
\PTpismo{13}{Hf-08}
\PTpismo{14}{Hf-08}
\PTpismo{15}{Hf-09}
\PTpismo{16}{Hf-09}
\PTpismo{17}{Hf-10}
\PTpismo{18}{Hf-11}
% \PTpismo{19}{U1-01}
\PTpismo{20}{U1-02}
\PTpismo{21}{U1-03}
\PTpismo{22}{U2-10}
\PTpismo{23}{U2-14}
\PTpismo{24}{U2-16}
//
\glnr
{\tiny 1}
{\tiny 2}
{\tiny 3}
{\tiny 4}
{\tiny 5}
{\tiny 6}
{\tiny 7}
{\tiny 8}
{\tiny 9}
{\tiny 10}
{\tiny 11}
{\tiny 12}
{\tiny 13}
{\tiny 14}
{\tiny 15}
{\tiny 16}
{\tiny 17}
{\tiny 18}
{\tiny 19}
{\tiny 20}
{\tiny 21}
{\tiny 22}
{\tiny 23}
{\tiny 24}
//
\endgl \xe

\usepackage{expex}

% 1 number
% 2 height
% 3 image
% 4 font
% 5 comment
\newcommand{\PTglyph}[3]{\includegraphics[height=#2ex]{glyphs/#3}}
% 1 numer
% 2 font identifier
\newcommand{\PTpismo}[2]{{\tiny #2}}

\lingset{glhangstyle=none}
\defineglwlevels{pismo,nr}
\newcommand{\bg}{\begingl}


% 22+36+33+29+3= 123

\item[Font 4] 
  \begin{figure}[H]
    \centering
    \PTfont{2}{0.9}{Hochfeder-04_PT01_022.png}
    \caption{\Junicode Hochfeder: 4. Pismo tekstowe gotyckie. Krój M⁸⁸. Stopień
      20 ww. = 76/77 mm - Tabl. 20, 26. (Występuje u Hallera jako
      pismo 7. Tabl. 170)}
    \label{fig:Hochfeder-04_PT01_022.png}
  \end{figure}

 Glyphs: 75.

% 14+23+21+17= 75 (198)

\item[Font 5] 
  \begin{figure}[H]
    \centering
    \PTfont{3}{0.9}{Hochfeder-05_PT01_023.png}
    \caption{\Junicode Hochfeder: 5. Pismo tekstowe gotyckie. Krój M⁸⁸. Stopień
      20 ww. = 88/89 mm - Tabl. 23}
    \label{fig:Hochfeder-05_PT01_023.png}
  \end{figure}

  Glyphs: 107.

%  21+39+36+11= 107 (305)

  
\item[Font 6] 
  
  \begin{figure}[H]
    \centering
    \PTfont{4}{0.9}{Hochfeder-06_PT01_024.png}
    \caption{\Junicode Hochfeder: 6. Pismo tekstowe gotyckie. Krój M⁸⁸. Stopień
      20 ww. = 79/80 mm - Tabl. 24}
    \label{fig:Hochfeder-06_PT01_024.png}
  \end{figure}

  Glyphs: 108.

%  25+44+39= 108 (413)

  
\item[Font 7] 
  
  \begin{figure}[H]
    \centering
    \PTfont{5}{0.9}{Hochfeder-07_PT01_024.png}
    \caption{\Junicode Hochfeder: 7. Pismo mszalne gotyckie. Krój M¹⁸. Stopień
      20 ww. = 154/156 mm - Tabl. 24, 25. (Występuje u Hallera jako
      pismo 2. Tabl.165; u Unglera jako pismo13. Tabl. 72)}
    \label{fig:Hochfeder-07_PT01_024.png}
  \end{figure}

Glyphs: 114.
  
% 25+39+37+13= 114 (527)

\item[Font 8] 
  
  \begin{figure}[H]
    \centering
    \PTfont{6}{0.9}{Hochfeder-08_PT01_020.png}
    \caption{\Junicode Hochfeder: 8. Pismo kanonowe gotyckie. Krój M¹⁸. Stopień
      20 ww. = 128/129 mm - Tabl. 20, 26. (Występuje u Hallera jako
      pismo 1. Tabl. 164; u Unglera jako pismo 13. Tabl. 72)}
    \label{fig:Hochfeder-08_PT01_020.png}
  \end{figure}

Glyphs: 114.
  
%  23+39+39+13= 114 (641)

  
\item[Font 9] 
  
  \begin{figure}[H]
    \centering
    \PTfont{7}{0.9}{Hochfeder-09_PT01_027.png}
    \caption{\Junicode Hochfeder: 9. Pismo tekstowe gotyckie. Krój M⁴⁸. Stopień
      20 ww. = 81/82 mm - Tabl. 27. (Występuje u Hallera jako pismo
      5. Tabl. 168; u Unglera jako pismo4. Tabl. 115)}
    \label{fig:Hochfeder-09_PT01_027.png}
  \end{figure}

Glyphs: 111.

%  21+33+30+27= 111 (752)

  
\item[Font 10] 
  
  \begin{figure}[H]
    \centering
    \PTfont{8}{0.9}{Hochfeder-10_PT01_028.png}
    \caption{\Junicode Hochfeder: 10. Pismo komentarzowe gotyckie. Krój
      M¹⁶. Stopień 20 ww. = 69 mm - Tabl. 28}
    \label{fig:Hochfeder-10_PT01_028.png}
  \end{figure}

Glyphs: 107.

% 22+31+27+27= 107 (859)

  
\item[Font 11] 
  
  \begin{figure}[H]
    \centering
    \PTfont{9}{0.9}{Hochfeder-11_PT01_028.png}
    \caption{\Junicode Hochfeder: 11. Pismo mszalne gotyckie. Krój M²³. Stopień
      20 ww. = 154/156 mm - Tabl. 25, 28. (Występuje u Hallera jako
      pismo 3. Tabl. 166; u Unglera jako pismo 11)}
    \label{fig:Hochfeder-11_PT01_028.png}
  \end{figure}

Glyphs: 94.

  
% 23+38+33= 94 (953)
  
\end{description}

\section{Zeszyt II: Haller}
\label{sec:zeszyt-ii}

No font tables.

\section{Zeszyt III: Ungler ---  pierwsza drukarnia}
\label{sec:zeszyt-iii}

Font tables: 10. Glyphs: 934
  
\begin{description}
\item[Font 1]

  \begin{figure}[H]
    \centering
    \PTfont{11}{0.9}{Ungler1-01_PT03_112.png}
    % \includegraphics[width=0.9\textwidth]{img/Ungler1_pismo01_tab112.png}
    \caption{\Junicode Ungler (1. drukarnia): 1. Pismo tekstowe gotyckie. Krój M⁹¹. Stopień 20 ww. ==
      76/77 mm. — Tabl. 112.}
    \label{fig:Ungler1_pismo01_tab112.png}
  \end{figure}

Glyphs: 120.
  
%  20+29+27+27+5+12=\textbf{120} U1 1

\item[Font 2] 
  \begin{figure}[H]
    \centering
    \PTfont{10}{0.9}{Ungler1-02_PT03_113.png}
%    \includegraphics[width=0.9\textwidth]{img/Ungler1_pismo02_tab113.png}
    \caption{\Junicode Ungler (1. drukarnia): 2. Pismo tekstowe gotyckie. Krój M⁹¹. Stopień 20 ww. ==
      76/77 mm. — Tabl. 113.}
    \label{fig:Ungler1_pismo02_tab113.png}
  \end{figure}

  Glyphs: 122.
  

  % 30+49+37+9=\textbf{125} U1 2
  % total 245

  
\item[Font 3] 
  \begin{figure}[H]
    \centering
    \PTfont{11}{0.9}{Ungler1-03_PT03_114.png}
 %   \includegraphics[width=0.9\textwidth]{img/Ungler1_pismo03_tab114.png}
    \caption{\Junicode Ungler (1. drukarnia): 3. Pismo tekstowe gotyckie. Krój M⁹¹. Stopień 20
      ww. == 72/73 mm. — Tabl. 114.}
    \label{fig:Ungler1_pismo03_tab114.png}
  \end{figure}

Glyphs: 129
  

  
  % 23+35+31+30+10=129
  % total 374

  
\item[Font 4] 
  
  \begin{figure}[H]
    \centering
    \PTfont{12}{0.9}{Ungler1-04_PT03_115.png}
%    \includegraphics[width=0.9\textwidth]{img/Ungler1_pismo04_tab115.png}
    \caption{\Junicode Ungler (1. drukarnia): 4. Pismo tekstowe gotyckie. Krój M⁴⁸. Stopień 20
      ww. == 79/81 mm. — Tabl. 115.}
    \label{fig:Ungler1_pismo04_tab115.png}
  \end{figure}

Glyphs: 77.
  
  % 23+31+23=77
  % total 451

\item[Font 5] 
  
  \begin{figure}[H]
    \centering
    \PTfont{13}{0.9}{Ungler1-05_PT03_116.png}
 %   \includegraphics[width=0.9\textwidth]{img/Ungler1_pismo05_tab116.png}
    \caption{\Junicode Ungler (1. drukarnia): 5. Pismo tekstowe gotyckie. Krój M⁴⁸. Stopień 20
      ww. == 89/90 mm. — Tabl. 116.}
    \label{fig:Ungler1_pismo05_tab116.png}
  \end{figure}

Glyphs: 95.

  
  % 24+40+31=95
  % total 546

\item[Font 6] 
  
  \begin{figure}[H]
    \centering
    \PTfont{14}{0.9}{Ungler1-06_PT03_117.png}
  %  \includegraphics[width=0.9\textwidth]{img/Ungler1_pismo06_tab117.png}
    \caption{\Junicode Ungler (1. drukarnia): 6. Pismo tekstowe gotyckie (stany a i b). Krój M⁴⁸. Stopień 20 ww. ==
      85/86 mm. — Tabl. 117.}
    \label{fig:Ungler1_pismo06_tab117.png}
  \end{figure}

Glyphs: 131.
  
  % 30+38+37+26=131
  % total 677

\item[Font 7]
  
  \begin{figure}[H]
    \centering
    \PTfont{15}{0.9}{Ungler1-07_PT03_118.png}
   % \includegraphics[width=0.9\textwidth]{img/Ungler1_pismo07_tab118.png}
    \caption{\Junicode Ungler (1. drukarnia): 7. Pismo nagłówkowe i tekstowe gotyckie. Krój M¹⁸ raz przekreślone.
      % przekreślone we wzorze?
      Stopień 20 ww. == 104/106 mm. — Tabl. 118.}
    \label{fig:Ungler1_pismo07_tab118.png}
%    20+30+28+28=106
  \end{figure}

  Glyphs: 106.

\item[Font 8] 
  
  \begin{figure}[H]
    \centering
    \PTfont{16}{0.9}{Ungler1-08_PT03_120.png}
    %\includegraphics[width=0.9\textwidth]{img/Ungler1_pismo08_tab120.png}
    \caption{\Junicode Ungler (1. drukarnia): 8. Pismo nagłówkowe gotyckie. Krój M⁶⁹. Stopień
      10 ww. == ok. 80mm. — Tabl. 120.}
    \label{fig:Ungler1_pismo08_tab120.png}
  \end{figure}

 Glyphs: 42.

  
  % 15+27=42
  % total 719

\item[Font 9] 
  
  \begin{figure}[H]
    \centering
%    \includegraphics[width=0.9\textwidth]{img/Ungler1_pismo09_tab120.png}
    \PTfont{17}{0.9}{Ungler1-09_PT03_120.png}
    \caption{\Junicode Ungler (1. drukarnia): 9.  Pismo nagłówkowe gotyckie. Krój M²³. Stopień
      10 ww. == ok. 70 mm. — Tabl. 120.}
    \label{fig:Ungler1_pismo09_tab120.png}
  \end{figure}

   Glyphs: 107.

  % 22+28+27+20+10=107
  % total 826

  
\item[10]
  
  \begin{figure}[H]
    \centering
    \PTfont{18}{0.9}{Ungler1-10_PT03_119.png}
%    \includegraphics[width=0.9\textwidth]{img/Ungler1_pismo10_tab119.png}
    \caption{\Junicode Ungler (1. drukarnia): 10. Pismo tekstowe antykwowe. Krój Qu/ pośredni między G i
      K⁵. Stopień 20 ww. == 95/96 mm. — Tabl. 119.}
    \label{fig:Ungler1_pismo10_tab119.png}
  \end{figure}

     Glyphs: 107.

  % 11.
  % Pismo mszalne małe gotyckie. Krój M²³ Stopień 20 ww. — 135/136 mm.
  % (Występuje u Hochfedera jako pismo 11. Tabl. 24, u Hallera jako pismo 3.
  % Tabl. 53).

  % sprawdzić Hochfeder nie ta tablica???

  % 12
  % Pismo mszalne duże gotyckie. Krój M¹⁸. Stopień 20 ww. — 160 mm.
  % Tabl. 72. (Występuje u Hochfedera jako pismo 7. Tabl. 24, u Hallera
  % jako pismo 2. Tabl. 30).

  % u Hochfera jest tablica!

  % sprawdzic

  % Anna Wolińska

  % 13.
  % Pismo kanonowe gotyckie. Krój M¹⁹. Stopień  20 ww. — 135 mm. —
  % Tabl. 72. (Występuje u Hochfedera jako pismo 8. Tabl. 25, u Hallera
  % jako pismo 1. Tabl. 30).

  % sprawdzić
  % Tab.72 bez tabeli!

  % 23+28+27+20+10=108

  % Ungler1 total 934
\end{description}

\section{Zeszyt IV: Haller}
\label{sec:zeszyt-iv}

Font tables: 14. Glyphs: 1287.

% BRAK 154 inicjały?

% 159 inicjały?


\begin{description}
\item[Font 1]

  \begin{figure}[H]
    \centering
    \PTfont{19}{0.9}{Haller-01_PT04_164.png}
 %   \includegraphics[width=0.9\textwidth]{img/Haller_pismo01_tab164}
    \caption{\Junicode Haller: 1. Pismo kanonowe. Krój M¹⁹, Stopień 10 ww. =
      128/129 mm. — Tabl. 164.  (Występuje u Hochfedera jako pismo
      8. Tabl. 25, u Unglera jako pismo 13.  Tabl. 72).}
    \label{fig:Haller_pismo01tab164}
  \end{figure}
  Glyphs: 85.

%   16+26+27+16=85

\item[Font 2] 
  \begin{figure}[H]
    \centering
    \PTfont{20}{0.9}{Haller-02_PT04_165.png}
%    \includegraphics[width=0.9\textwidth]{img/Haller_pismo02_tab165}
    \caption{\Junicode Haller: 2. Pismo mszalne większe. Krój M¹⁸. Stopień 10 ww. = 77 mm. — Tabl. 165.
      (Występuje u Hochfedera jako pismo 7. Tabl. 24, u Unglera jako pismo 12.
      Tabl. 72).}
    \label{fig:Haller_pismo02tab165}
  \end{figure}

  Glyphs: 111.
  
  % 21+40+37+13=111
  % total 196

  
\item[Font 3] 
  \begin{figure}[H]
    \centering
    \PTfont{21}{0.9}{Haller-03_PT04_166.png}
  %  \includegraphics[width=0.9\textwidth]{img/Haller_pismo03_tab166}
    \caption{\Junicode Haller: 3. Pismo mszalne mniejsze. Krój M²³. Stopień 20 ww. = 136/137 mm. —
      Tabl. 166. (Występuje u Hochfedera jako pismo 11. Tabl. 24, u Unglera
      jako pismo 11).}
    \label{fig:Haller_pismo03tab166}
  \end{figure}

  Glyphs: 111.

  
  % 23+39+43+16=121
  % total 317
\item[Font 4] 
  \begin{figure}[H]
    \centering
    \PTfont{22}{0.9}{Haller-04_PT04_167.png}
 %   \includegraphics[width=0.9\textwidth]{img/Haller_pismo04_tab167}
    \caption{\Junicode Haller: 4. Pismo nagłówkowe. Krój M⁸³. Stopień 20 ww. =112/114 mm. —
      Tabl. 167. (Występuje u Hochfedera jako pismo 2. Tabl. 19).}
    \label{fig:Haller_pismo04tab167}
  \end{figure}

  Glyphs: 135.

  
  % 25+40?+39+31=135
  % total 452

\item[Font 5] 
  \begin{figure}[H]
    \centering
    \PTfont{23}{0.9}{Haller-05_PT04_168.png}
%    \includegraphics[width=0.9\textwidth]{img/Haller_pismo05_tab168}
    \caption{\Junicode Haller: 5. Pismo tekstowe. Krój M⁴⁸. Stopień 20 ww. =81/82 mm. —
      Tabl. 168.  (Występuje u Hochfedera jako pismo 9. Tabl. 26, u
      Unglera jako pismo 4.  Tabl. 115).}
    \label{fig:Haller_pismo05tab168}
  \end{figure}

 Glyphs: 135.

  % 27+41+39+28=135
  % total 587

\item[Font 6] 
  \begin{figure}[H]
    \centering
    \PTfont{24}{0.9}{Haller-06_PT04_169.png}
    %\includegraphics[width=0.9\textwidth]{img/Haller_pismo06_tab169}
    \caption{\Junicode Haller: 6. Pismo komentarzowe. Krój M”. Stopień 20 ww. =
      82 mm (interliniowane). — Tabl. 169. (Występuje u Hochfedera
      jako pismo 4. Tabl. 21).}
    \label{fig:Haller_pismo06tab169}
    % 19+45+29=93\\
    % total 680
  \end{figure}

   Glyphs: 93.

  
\item[Font 7]
  
  \begin{figure}[H]
    \centering
    \PTfont{25}{0.9}{Haller-07_PT04_170.png}
%    \includegraphics[width=0.9\textwidth]{img/Haller_pismo07_tab170}
    \caption{\Junicode Haller: 7. Pismo tekstowe. Krój M⁸⁸. Stopień 20 ww. =82
      mm. — Tabl. 170.}
    \label{fig:Haller_pismo07tab170}
%    22+49+47+1+10=129\\total 809
  \end{figure}

   Glyphs: 129.

   
 \item[Font 8] 

  \begin{figure}[H]
    \centering
    \PTfont{26}{0.9}{Haller-08_PT04_171.png}
    %\includegraphics[width=0.9\textwidth]{img/Haller_pismo08_tab171}
    \caption{\Junicode Haller: 8. Pismo tekstowe. Krój M⁹¹. Stopień 20 ww. = 72 mm. —
      Tabl. 171.}
    \label{fig:Haller_pismo08tab171}
%    24+44+43+2=113\\ total 922
  \end{figure}

 Glyphs: 113.

 
\item[Font 9] 
  
  \begin{figure}[H]
    \centering
    \PTfont{27}{0.9}{Haller-09_PT04_172.png}
   % \includegraphics[width=0.9\textwidth]{img/Haller_pismo09_tab172}
    \caption{\Junicode Haller: 9. Pismo tekstowe. Krój M⁴⁹. Stopień 20 ww. = 72
      mm. — Tabl. 172.}
    \label{fig:Haller_pismo09tab172}
%    22+38+36+20=116\\ total 1038
  \end{figure}

 Glyphs: 116.

\item[Font 10] 
  
  \begin{figure}[H]
    \centering
    \PTfont{28}{0.9}{Haller-10_PT04_173.png}
  %  \includegraphics[width=0.9\textwidth]{img/Haller_pismo10_tab173}
    \caption{\Junicode Haller: 10. Pismo komentarzowe. Krój M⁸⁷. Stopień 20 ww. = 72 mm. —
      Tabl. 173.}
    \label{fig:Haller_pismo10tab173}
%    22+40+40=102\\ total 1140
  \end{figure}

 Glyphs: 102.

 
\item[Fonts 11a,11b]  
  \begin{figure}[H]
    \centering
    \PTfont{29}{0.9}{Haller-11_PT04_174.png}
    \PTfont{29}{0.9}{Haller-11pl_PT04_174.png}
    % \includegraphics[width=0.9\textwidth]{img/Haller_pismo11a_tab174}
    % \includegraphics[width=0.9\textwidth]{img/Haller_pismo11b_tab174}
    \caption{\Junicode Haller: 11. Pismo tekstowe antykwowe. Krój Qlu (C). Stopień 20
      ww. = 92 mm. — - Tabl. 174.}
    \label{fig:Haller_pismo11tab174}
  %  25+40+36+10+1+46=158\\ total 1298
  \end{figure}

 Glyphs: 158.

\item[Font 12] 
  
  \begin{figure}[H]
    \centering
    \PTfont{30}{0.9}{Haller-12_PT04_175.png}
 %   \includegraphics[width=0.9\textwidth]{img/Haller_pismo12_tab175}
    \caption{\Junicode Haller: 12. Pismo mszalne. Krój M⁶⁰. Stopień 20 ww. = 176 mm. —
      Tabl. 175.}
    \label{fig:Haller_pismo12tab175}
 %   22+37+37+4=100\\ total 1398
  \end{figure}

 Glyphs: 100.

 
\item[Font 13] 
  
  \begin{figure}[H]
    \centering
    \PTfont{31}{0.9}{Haller-13_PT04_172.png}
%    \includegraphics[width=0.9\textwidth]{img/Haller_pismo13_tab172}
    \caption{\Junicode Haller: 13.  Pismo tekstowe. Krój M¹⁸. Stopień 20 ww. = 112 mm
      (interliniowane). — Tabl. 172. (Występuje u Wietora, w drukarni
      wiedeńskiej).}
    \label{fig:Haller_pismo13tab172}
%    22+36+37+10=105\\total 1503
  \end{figure}

   Glyphs: 105.
  
%  14 i 15 bez zestawów
\end{description}


\section{Zeszyt V: Ungler --- druga drukarnia}
\label{sec:zeszyt-v}

Font tables: 11. Glyphs: 1047.

\begin{description}
\item[Font 1] \begin{figure}[H]
    \centering
      \PTfont{33}{0.9}{Ungler2-01_PT05_239.png}
%    \includegraphics[width=0.9\textwidth]{img/Ungler2_pismo01_tab239.png}
    \caption{\Junicode Ungler (2. drukarnia): 1. Pismo nagłówkowe i tekstowe, rotunda M²³. Stopień
      20 ww. = 132 mm. — Tabl. 239.}
    \label{fig:Ungler2_pismo01_tab239.png}
%    19+7+32+31+10=99
  \end{figure}
Glyphs: 99.
  
\item[Font 2] 
  \begin{figure}[H]
    \centering
      \PTfont{34}{0.9}{Ungler2-02_PT05_240.png}
 %   \includegraphics[width=0.9\textwidth]{img/Ungler2_pismo02_tab240.png}
    \caption{\Junicode Ungler (2. drukarnia): 2. Pismo tekstowe, antykwa Q/u (zbliżone do
      F⁵). Stopień 20 ww. =92— 94 mm. — Tabl. 240.}
    \label{fig:Ungler2_pismo02_tab240.png}
%    20+25+37+35+33=150\\ total 249
  \end{figure}

  Glyphs: 150.


\item[Font 3] 
  \begin{figure}[H]
    \centering
      \PTfont{35}{0.9}{Ungler2-03_PT05_241.png}
  %  \includegraphics[width=0.9\textwidth]{img/Ungler2_pismo03_tab241.png}
    \caption{\Junicode Ungler (2. drukarnia): 3. Pismo tekstowe, rotunda M⁴⁹, Stopień 20 ww. =
      62/63 mm. — Tabl. 241.}
    \label{fig:Ungler2_pismo03_tab241.png}
%    27+42+39+39+10=157\\ total 406
  \end{figure}

 Glyphs: 157.


\item[Font 4] 
  \begin{figure}[H]
    \centering
      \PTfont{36}{0.9}{Ungler2-04_PT05_242.png}
   % \includegraphics[width=0.9\textwidth]{img/Ungler2_pismo04_tab242.png}
    \caption{\Junicode Ungler (2. drukarnia): 4. Pismo tekstowe i nagłówkowe, szwabacha
      M⁸¹. Stopień 20 ww. = = 103 mm. — Tabl. 242.}
    \label{fig:Ungler2_pismo04_tab242.png}
%    24+31+21+31+29=136\\ total 542
  \end{figure}

   Glyphs: 136.


\item[Font 5] 
  \begin{figure}[H]
    \centering
      \PTfont{37}{0.9}{Ungler2-05_PT05_243.png}
    %\includegraphics[width=0.9\textwidth]{img/Ungler2_pismo05_tab243.png}
    \caption{\Junicode Ungler (2. drukarnia): 5. Pismo tekstowe, antykwa Q/u (C). Stopień 20 ww. =94—95
      mm. — Tabl. 243.}
    \label{fig:Ungler2_pismo05_tab243.png}
%    21+31+29=81\\ total 623
  \end{figure}

   Glyphs: 81.


\item[Font 6] 
  \begin{figure}[H]
    \centering
      \PTfont{38}{0.9}{Ungler2-06_PT05_241.png}
    %\includegraphics[width=0.9\textwidth]{img/Ungler2_pismo06_tab241.png}
    \caption{\Junicode Ungler (2. drukarnia): 6. Pismo tekstowe, rotunda M¹⁸. Stopień 20 ww. = 76
      mm. — Tabl. 241.}
    \label{fig:Ungler2_pismo06_tab241.png}
%    20+33+34+8=95\\ total 718
  \end{figure}


   Glyphs: 95.

  % Tylko teksty przykładowe

  % 7. Pismo tytułowe, fraktura. Stopień 1 w. =11 mm. —
  % Tabl. 218, 219.}

  % 7. . Pismo tytułowe, fraktura. Stopień 1 w. =11 mm. — Tabl. 218, 219.
  % 8. Pismo nagłówkowe, rotunda M* (1). Stopień 1 w. =4/5 mm bez przedłużek. — Tabl. 218.

\item[Font 9] 
  \begin{figure}[H]
    \centering
      \PTfont{39}{0.9}{Ungler2-09_PT05_244.png}
   % \includegraphics[width=0.9\textwidth]{img/Ungler2_pismo09_tab244.png}
    \caption{\Junicode Ungler (2. drukarnia): 9. Pismo tekstowe, szwabacha zromanizowana
      M⁸¹. Stopień 20 ww. = = 80/81 mm. — Tabl. 244.}
    \label{fig:Ungler2_pismo09_tab244.png}
%    32+49+52+10=143\\ total 861
  \end{figure}

   Glyphs: 143.


\item[Font 10] 
  \begin{figure}[H]
    \centering
      \PTfont{40}{0.9}{Ungler2-10_PT05_245.png}
   % \includegraphics[width=0.9\textwidth]{img/Ungler2_pismo10_tab245.png}
    \caption{\Junicode Ungler (2. drukarnia): 10. Pismo tekstowe, rotunda M¹⁸. Stopień 20 ww. = 66
      mm. — Tabl. 245.}
    \label{fig:Ungler2_pismo10_tab245.png}
%    25+36+32+33=126\\ total 987
  \end{figure}

   Glyphs: 126.


\item[Font 11] 
  \begin{figure}[H]
    \centering
      \PTfont{41}{0.9}{Ungler2-11_PT05_245.png}
   % \includegraphics[width=0.9\textwidth]{img/Ungler2_pismo11_tab245.png}
    \caption{\Junicode Ungler (2. drukarnia): 11. Pismo tekstowe, antykwa. Stopień 1 w. = 3 mm. —
      Tabl. 245.}
    \label{fig:Ungler2_pismo11_tab245.png}
%    19\\ total 1006
  \end{figure}

   Glyphs: 19.


\item[Font 12] 
  \begin{figure}[H]
    \centering
      \PTfont{42}{0.9}{Ungler2-12_PT05_243.png}
  %  \includegraphics[width=0.9\textwidth]{img/Ungler2_pismo12_tab243.png}
    \caption{\Junicode Ungler (2. drukarnia): 12. Wersaliki tytułowe, antykwa. Wysokość 8 mm. —
      Tabl. 243.}
    \label{fig:Ungler2_pismo12_tab243.png}
%    18\\ total 1024
  \end{figure}

   Glyphs: 18.


\item[Font 13] 
  \begin{figure}[H]
    \centering
      \PTfont{43}{0.9}{Ungler2-13_PT05_243.png}
   % \includegraphics[width=0.9\textwidth]{img/Ungler2_pismo13_tab243.png}
    \caption{\Junicode Ungler (2. drukarnia): 13. Wersaliki tytułowe, antykwa. Wysokość 8 mm. —
      Tabl. 243.}
    \label{fig:Ungler2_pismo13_tab243.png}
%    15=8=23\\ total 1047
  \end{figure}

 Glyphs: 23.

\end{description}

\section{Zeszyt VI: Ungler --- druga drukarnia}
\label{sec:zeszyt-vi}

No font tables.

%tylko drzeworyty (brak skanów)?


\section{Zeszyt VII: Ungler --- druga drukarnia} 
\label{sec:zeszyt-vii}

Font tables: 10. Glyphs: 930.

\begin{description}
\item[14] \begin{figure}[H]
    \centering
    \PTfont{44}{0.9}{Ungler2-14_PT07_357.png}
%    \includegraphics[width=0.9\textwidth]{img/Ungler2_pismo14_tab357}
    \caption{\Junicode Ungler (2. drukarnia):  14.  Pismo tytułowe i nagłówkowe,
      fraktura Neudörffer-Andreae. Stopień 5 ww. = 65 mm. — Tabl. 357.}
    \label{fig:Ungler2_pismo14_tab357}
%    27+38+36+28=129
  \end{figure}

   Glyphs: 129.


\item[15] 
  \begin{figure}[H]
    \centering
    \PTfont{45}{0.9}{Ungler2-15_PT07_358.png}
 %   \includegraphics[width=0.9\textwidth]{img/Ungler2_pismo15a-b_tab358}
    \caption{\Junicode Ungler (2. drukarnia):  15 a-b.Pismo tekstowe, antykwa
      Qu|(G). Stopień 20 ww. = 91 mm. — Tabl. 358.}
    \label{fig:Ungler2_pismo15a-btab358}
%    22+35+29+23=109\\ razem 238
  \end{figure}

   Glyphs: 109..


\item[16] 
  \begin{figure}[H]
    \centering
    \PTfont{46}{0.9}{Ungler2-16_PT07_359.png}
  %  \includegraphics[width=0.9\textwidth]{img/Ungler2_pismo16_tab359}
    \caption{\Junicode Ungler (2. drukarnia) 16. Pismo tekstowe, rotunda zbliżona do M⁵⁰ (druga
      forma). Stopień 20 ww. = 65 mm. — Tabl. 359.}
    \label{fig:Ungler2_pismo16tab359}
%    29+44+44+12+5=134\\ razem 372
  \end{figure}

 Glyphs: 134.

 % total 372
  
\item[17] 
  \begin{figure}[H]
    \centering
    \PTfont{47}{0.9}{Ungler2-17_PT07_357.png}
   % \includegraphics[width=0.9\textwidth]{img/Ungler2_pismo17_tab357}
    \caption{\Junicode Ungler (2. drukarnia) 17. Pismo tekstowe, rotunda M⁸⁷. Stopień 20 ww. = 75 mm. —
      Tabl. 357.}
    \label{fig:Ungler2_pismo17tab357}
%    18+5+32+9=64\\ razem 436
  \end{figure}

 Glyphs: 64.
 

\item[18] 
  \begin{figure}[H]
    \centering
    \PTfont{48}{0.9}{Ungler2-18_PT07_360.png}
    %\includegraphics[width=0.9\textwidth]{img/Ungler2_pismo18_tab360}
    \caption{\Junicode Ungler (2. drukarnia) 18. Pismo nagłówkowe i tekstowe, fraktura
      Neudörffer-Andreae. Stopień 10 ww. == 73 mm. — Tabl. 360.}
    \label{fig:Ungler2_pismo18tab360}
%    25+41+37+10=113\\ razem 549
  \end{figure}

   Glyphs: 113.


\item[19] 
  \begin{figure}[H]
    \centering
    \PTfont{49}{0.9}{Ungler2-19_PT07_360.png}
%    \includegraphics[width=0.9\textwidth]{img/Ungler2_pismo19_tab360}
    \caption{\Junicode Ungler (2. drukarnia) 19. Pismo tekstowe, fraktura Neudörffer-Andreae. Stopień 20
      ww. = 90—91 mm. — Tabl. 360.}
    \label{fig:Ungler2_pismo19tab360}

    Glyphs: 83.
%    25+42+6+10=83\\razem 632
  \end{figure}
\item[20] 
  \begin{figure}[H]
    \centering
    \PTfont{50}{0.9}{Ungler2-20_PT07_361.png}
   % \includegraphics[width=0.9\textwidth]{img/Ungler2_pismo20_tab361}
    \caption{\Junicode Ungler (2. drukarnia) 20. Pismo tekstowe, antykwa Qu|(G). Stopień 20 ww. ==
      90-—91 mm. — Tabl. 361.}
    \label{fig:Ungler2_pismo20tab361}
%    26+44+34+26=130\\ razem 762
  \end{figure}

    Glyphs: 130.


\item[21] 
  \begin{figure}[H]
    \centering
    \PTfont{51}{0.9}{Ungler2-21_PT07_362.png}
   % \includegraphics[width=0.9\textwidth]{img/Ungler2_pismo21_tab362}
    \caption{\Junicode Ungler (2. drukarnia) 21. Pismo tekstowe, antykwa Qu|(H). Stopień 20 ww. = 113
      mm. — Tabl. 362.}
    \label{fig:Ungler2_pismo21tab120}
%    19+30+29=78\\ razem 762
  \end{figure}

   Glyphs: 78.
  
% 840
\item[22] 
  \begin{figure}[H]
    \centering
    \PTfont{52}{0.9}{Ungler2-22_PT07_362.png}
    \caption{\Junicode Ungler (2. drukarnia): 22. Pismo tytułowe i nagłówkowe, antykwa Q|u(C). Wysokość 1 ww. = 9 mm. - Tabl. 362."}
    \label{fig:Ungler2-22_PT07_362.png}
  \end{figure}
  % 14+21+15+ 
   Glyphs: 50.


\item[23] 

  \begin{figure}[H]
    \centering
    \PTfont{53}{0.9}{Ungler2-23_PT07_363.png}
    \caption{\Junicode Ungler (2. drukarnia): 23. Wersaliki tytułowe,
      antykwa Qu|(K3. Wysokość 7—8 mm. — Tabl. 363.}

    \label{fig:Ungler2-23_PT07_363.png}
  \end{figure}
  % 21
  Glyphs: 21.

    \begin{figure}[H]
    \centering
    \PTfont{53}{0.9}{Ungler2-24_PT07_363.png}
    \caption{\Junicode Ungler (2. drukarnia): 24. Wersaliki tytułowe,
      antykwa. Wysokość 8—9 mm. —- Tabl. 363.  Pismo greckie,
      tekstowe. Stopień 5 22/23 Tabl. 363.}
    \label{fig:Ungler2-24_PT07_363.png}
  \end{figure}

     Glyphs: 19.


\end{description}

\section{VIII: Augezdecki}
\label{sec:viii}

Font tables 19. Glyphs: 1483.

\begin{description}
\item[1] 
  \begin{figure}[H]
    \centering
    \PTfont{54}{0.9}{Augezdecki-01_PT08_402.png}
    % \includegraphics[width=0.9\textwidth]{img/Augezdecki_pismo01_tab402}
    % \includegraphics[width=0.9\textwidth]{img/Augezdecki_pismo01a_tab404}
    % \includegraphics[width=0.9\textwidth]{img/Augezdecki_pismo01b_tab403}
    \caption{\Junicode Augezdecki: 1 [1]. Pisma tekstowe, szwabacha
      M⁸¹. Stopień 20 ww. = 102—103 mm (tercja). — Tabl. 402—404.}
    \label{fig:Augezdecki_pismo01}
%    28+40+41+25+41+10+23+10+32=241 
  \end{figure}

  Glyphs: 241

  
\item[1a] 
  \begin{figure}[H]
    \centering
    \PTfont{55}{0.9}{Augezdecki-01a_PT08_403.png}
    % \includegraphics[width=0.9\textwidth]{img/Augezdecki_pismo01_tab402}
    % \includegraphics[width=0.9\textwidth]{img/Augezdecki_pismo01a_tab404}
    % \includegraphics[width=0.9\textwidth]{img/Augezdecki_pismo01b_tab403}
    \caption{\Junicode Augezdecki: 1 [1a]. Pisma tekstowe, szwabacha
      M⁸¹. Stopień 20 ww. = 102—103 mm (tercja). — Tabl. 402—404.}
    \label{fig:Augezdecki_pismo01a}
%    28+40+41+25+41+10+23+10+32=241 
  \end{figure}

%  10+32
  
   Glyphs: 42.

 \item[1b] 

  \begin{figure}[H]
    \centering
    \PTfont{56}{0.9}{Augezdecki-01b_PT08_404.png}
    % \includegraphics[width=0.9\textwidth]{img/Augezdecki_pismo01_tab402}
    % \includegraphics[width=0.9\textwidth]{img/Augezdecki_pismo01a_tab404}
    % \includegraphics[width=0.9\textwidth]{img/Augezdecki_pismo01b_tab403}
    \caption{\Junicode Augezdecki: 1 [1b]. Pisma tekstowe, szwabacha
      M⁸¹. Stopień 20 ww. = 102—103 mm (tercja). — Tabl. 402—404.}
    \label{fig:Augezdecki_pismo01b}
  \end{figure}

 Glyphs: 23.

\item[2] 
      
  
  \begin{figure}[H]
    \centering
    \PTfont{57}{0.9}{Augezdecki-02_PT08_405.png}
    % \includegraphics[width=0.9\textwidth]{img/Augezdecki_pismo02_tab405}
    % \includegraphics[width=0.9\textwidth]{img/Augezdecki_pismo02a_tab406}
    \caption{\Junicode Augezdecki: 2 [2]. Pismo tekstowe, szwabacha M⁸¹. Stopień 20 ww. = 86—87 mm
      (cycero). — Tabl. 405, 406.}
    \label{fig:Augezdecki_pismo02}
 %   28+42+22+30+36+15=173\\ razem 414
  \end{figure}

  Glyphs: 158.

  
\item[2a] 

  
  \begin{figure}[H]
    \centering
    \PTfont{58}{0.9}{Augezdecki-02a_PT08_406.png}
    % \includegraphics[width=0.9\textwidth]{img/Augezdecki_pismo02_tab405}
    % \includegraphics[width=0.9\textwidth]{img/Augezdecki_pismo02a_tab406}
    \caption{\Junicode Augezdecki: 2 [2a]. Pismo tekstowe, szwabacha M⁸¹. Stopień 20 ww. = 86—87 mm
      (cycero). — Tabl. 405, 406.}
    \label{fig:Augezdecki_pismo02a}
%    28+42+22+30+36+15=173\\ razem 414
  \end{figure}

 Glyphs: 15.

  
\item[3]
  
  \begin{figure}[H]
    \centering
    \PTfont{59}{0.9}{Augezdecki-03_PT08_407.png}
%    \includegraphics[width=0.9\textwidth]{img/Augezdecki_pismo03_tab407}
    \caption{\Junicode Augezdecki: 3. Pismo komentarzowe, szwabacha M⁸¹ + M⁸⁷. Stopień 20 ww. =
      67 mm (garmond). — Tabl. 407.}
    \label{fig:Augezdecki_pismo03tab407}
%    16+18+30+30+18=112\\ razem 526
  \end{figure}

   Glyphs: 112

  
% \item[4] 
%   \begin{figure}[H]
%     \centering
%     % \includegraphics[width=0.9\textwidth]{img/Augezdecki_pismo04_tab408}
%     \caption{4. Pismo tekstowe, szwabacha M⁸¹. Stopień 1 w. = 4,5 mm. —
%       Tabl. 413.}
%     brak zestawu!!!
%     \label{fig:Augezdecki_pismo04tab408}
%     23+4+42+8+20=97\\razem 623
%   \end{figure}
\item[5] 
  
  \begin{figure}[H]
    \centering
    \PTfont{60}{0.9}{Augezdecki-05_PT08_408.png}
  %  \includegraphics[width=0.9\textwidth]{img/Augezdecki_pismo05_tab408}
    \caption{\Junicode Augezdecki: 5. Pismo nagłówkowe, tekstura M³⁰. Stopień I w. = 9 mm. —
      Tabl. 408.}
    \label{fig:Augezdecki_pismo05tab408}
%    22+4+41+8+20=95\\razem 718
  \end{figure}

     Glyphs: 95


\item[6] 
  \begin{figure}[H]
    \centering
    \PTfont{61}{0.9}{Augezdecki-06_PT08_408.png}
 %   \includegraphics[width=0.9\textwidth]{img/Augezdecki_pismo06_tab408}
    \caption{\Junicode  Augezdecki: 6. Pismo nagłówkowe, tekstura M²⁹. Stopień I w. = 7 mm. —
      Tabl. 408.}
    \label{fig:Augezdecki_pismo06tab408}
%    21+5+37+31=94\\razem 812
  \end{figure}

   Glyphs: 94.

  
\item[7] 
  \begin{figure}[H]
    \centering
    \PTfont{62}{0.9}{Augezdecki-07_PT08_409.png}
%    \includegraphics[width=0.9\textwidth]{img/Augezdecki_pismo07_tab409}
    \caption{\Junicode  Augezdecki: 7. Pismo tekstowe, antykwa Qu/(G, I). Stopień 20 ww. =
      101/102 mm (tercja). — Tabl. 409.}
    \label{fig:Augezdecki_pismo07tab409}
%    25+42+16=83\\ razem 895
  \end{figure}

     Glyphs: 83.


\item[8] 
  \begin{figure}[H]
    \centering
    \PTfont{63}{0.9}{Augezdecki-08_PT08_410.png}
%    \includegraphics[width=0.9\textwidth]{img/Augezdecki_pismo08_tab410}
    \caption{\Junicode  Augezdecki: 8. Pismo tekstowe, antykwa Qu/(G). Stopień 20 ww. = 101 mm
      (tercja). — Tabl. 410.}
    \label{fig:Augezdecki_pismo08tab410}
%    27+45+32=104\\razem 999
  \end{figure}

       Glyphs: 104.


\item[9] 
  \begin{figure}[H]
    \centering
    \PTfont{64}{0.9}{Augezdecki-09_PT08_408.png}
 %   \includegraphics[width=0.9\textwidth]{img/Augezdecki_pismo09_tab408}
    \caption{\Junicode  Augezdecki: 9. Wersaliki tytułowe, antykwa. Wysokość 8 mm. —
      Tabl. 408.}
    \label{fig:Augezdecki_pismo09tab408}
%    17+12=29\\razem 1028
  \end{figure}

       Glyphs: 29.



\item[10] 
  \begin{figure}[H]
    \centering
    \PTfont{65}{0.9}{Augezdecki-10_PT08_408.png}
  %  \includegraphics[width=0.9\textwidth]{img/Augezdecki_pismo10_tab408}
    \caption{\Junicode  Augezdecki: 10. Wersaliki tytułowe, antykwa. Wysokość 6—7 mm. — Tabl. 408.}
    \label{fig:Augezdecki_pismo10tab408}
%    15\\razem 1043
  \end{figure}

     Glyphs: 15.

  
\item[11] 
  \begin{figure}[H]
    \centering
    \PTfont{66}{0.9}{Augezdecki-11_PT08_408.png}
 %   \includegraphics[width=0.9\textwidth]{img/Augezdecki_pismo11_tab408}
    \caption{\Junicode  Augezdecki: 11. Wersaliki tytułowe, antykwa. Wysokość 4,5—5 mm:— Tabl. 408.}
    \label{fig:Augezdecki_pismo11tab408}
%    27+5=32\\razem 1075
  \end{figure}

       Glyphs: 32.

  
\item[12] 
  \begin{figure}[H]
    \centering
    \PTfont{67}{0.9}{Augezdecki-12_PT08_408.png}
   % \includegraphics[width=0.9\textwidth]{img/Augezdecki_pismo12_tab408}
    \caption{\Junicode  Augezdecki: 12. Wersaliki, antykwa. Wysokość 2—2,5 mm. — Tabl. 408.}
    \label{fig:Augezdecki_pismo12tab408}
%    22\\razem 1097
  \end{figure}

       Glyphs: 22.


\item[13] 
  \begin{figure}[H]
    \centering
    \PTfont{68}{0.9}{Augezdecki-13_PT08_411.png}
  %  \includegraphics[width=0.9\textwidth]{img/Augezdecki_pismo13_tab412}
    \caption{\Junicode  Augezdecki: 13. Pismo nagłówkowe, fraktura H. Schönspergera. Stopień 1
      w. = 8 mm. — Tabl. 412.}
    \label{fig:Augezdecki_pismo13tab412}
%    13+33+9=55\\razem 1152
  \end{figure}

       Glyphs: 55.


\item[14] 
  \begin{figure}[H]
    \centering
    \PTfont{69}{0.9}{Augezdecki-14_PT08_410.png}
 %   \includegraphics[width=0.9\textwidth]{img/Augezdecki_pismo14_tab410}
    \caption{\Junicode  Augezdecki: 14. Pismo tekstowe i nagłówkowe, fraktura H. Schönspergera. Stopień
      20 ww. = 153 mm. — Tabl. 410, 411.}
%    !!! o co mi chodziło???
    \label{fig:Augezdecki_pismo14tab410}
%    18+8+31+32+32+19=108\\razem 1260
  \end{figure}

       Glyphs: 108.



\item[15] 
  \begin{figure}[H]
    \centering
    \PTfont{70}{0.9}{Augezdecki-15_PT08_411.png}
  %  \includegraphics[width=0.9\textwidth]{img/Augezdecki_pismo15_tab412}
    \caption{\Junicode  Augezdecki: 15. Pismo tytułowe i nagłówkowe, fraktura
      H. Schönspergera. Wysokość 1 w. = 11—12 mm bez przedłużek. —
      Tabl. 412.}
    \label{fig:Augezdecki_pismo15tab412}
%    16+14+30+34+6+17=117\\razem 1337 
  \end{figure}

       Glyphs: 117.


\item[16] 
  \begin{figure}[H]
    \centering
    \PTfont{71}{0.9}{Augezdecki-16_PT08_413.png}
  %  \includegraphics[width=0.9\textwidth]{img/Augezdecki_pismo16_tab413}
    \caption{\Junicode  Augezdecki: 16. Pismo tekstowe, szwabacha M⁸¹. Stopień 20 ww. = 88
      mm. — Tabl. 413.}
    \label{fig:Augezdecki_pismo16tab413}
%    19+38+18=75\\razem 1412
  \end{figure}

       Glyphs: 75.


\item[17] 
  \begin{figure}[H]
    \centering
    \PTfont{72}{0.9}{Augezdecki-17_PT08_411.png}
  %  \includegraphics[width=0.9\textwidth]{img/Augezdecki_pismo17_tab412}
    \caption{\Junicode  Augezdecki: 17. Pismo tytułowe i nagłówkowe, tekstura M⁶³. Wysokość I
      w.= 12— 13 mm. — Tabl. 412.}
    \label{fig:Augezdecki_pismo17tab412}
%    13+3+25+22=63\\razem 1475
  \end{figure}

       Glyphs: 63.

  
\end{description}

\section{IX: Wirzbięta}
\label{sec:ix}

Font tables 5. Glyphs: 448.



\begin{description}
\item[Font 1]

  \begin{figure}[H]
    \centering
    \PTfont{73}{0.9}{Wirzbięta-01_PT09_469.png}\\
    \PTfont{73}{0.17}{Wirzbięta-01u_PT11_570.png}
    \caption{\Junicode Wirzbięta: 1. Pismo nagłówkowe, fraktura
      H. Schönspergera. Stopień 1 w, = 14,5 mm. — Tabl. 417, 419, 420,
      457, 464, 467, \textbf{469}, 471. Uzupełnienia Tab. \textbf{570}}
    \label{fig:Wirzbięta-01_PT09_469.png}

  \end{figure}

    % 19+7+29+30+4=89
  Glyphs: 89.

\item[2]
  
\begin{figure}[H]
    \centering
    \PTfont{74}{0.9}{Wirzbięta-02_PT09_469.png}\\
    \PTfont{73}{0.4}{Wirzbięta-02u_PT11_570.png}
    \caption{\Junicode Wirzbięta: 2. Pismo nagłówkowe i tekstowe,
      fraktura H. Schönspergera. Stopień 20 ww. = ca 160 mm. —
      Tabl. 420, 464, \textbf{469}. Uzupełnienia Tab. \textbf{570}}
    \label{fig:Wirzbięta-02_PT09_469.png}
  \end{figure}

  % 22+38+12+12=84
  Glyphs: 84.
  
\item[3] 
  
\begin{figure}[H]
    \centering
    \PTfont{75}{0.9}{Wirzbięta-03_PT09_469.png}\\
    \PTfont{73}{0.045}{Wirzbięta-03u_PT11_570.png}
    \PTfont{73}{0.25}{Wirzbięta-03uu_PT11_570.png}
    \caption{\Junicode Wirzbięta: 3. Pismo tekstowe, szwabacha M⁸¹,
      Stopień 20 ww. = 113 mm. — Tabl. 417, 419, 420, 456—458, 464,
      467—470 [\textbf{469}]. Uzupełnienia Tab. \textbf{570}}
    \label{fig:Wirzbięta-03_PT09_469.png}
  \end{figure}

  % 25+38+32+1=96
  Glyphs: 96.

  
\item[4] 
  
\begin{figure}[H]
    \centering
    \PTfont{75}{0.9}{Wirzbięta-04_PT09_469.png}\\
    \PTfont{73}{0.15}{Wirzbięta-04u_PT11_570.png}
    \caption{\Junicode Wirzbięta: 4. Pismo tekstowe, szwabacha M⁸¹,
      Stopień 20 ww. == 88 mm. — Tabl. 457, 468—470
      [\textbf{469}].Uzupełnienia Tab. \textbf{570}}
    \label{fig:Wirzbięta-04_PT09_469.png}
  \end{figure}

    % 24+41+35=100
  Glyphs: 100.

\item[5] 
  
\begin{figure}[H]
    \centering
    \PTfont{75}{0.9}{Wirzbięta-05_PT09_469.png}\\
    \PTfont{73}{0.245}{Wirzbięta-05u_PT11_570.png}
    \caption{\Junicode Wirzbięta: 5. Pismo komentarzowe, szwabacha
      M⁸¹. Stopień 20 ww. = 71/2 mm. — Tabl. 456, 457, 469, 470.
    [\textbf{469}].Uzupełnienia Tab. \textbf{570}}
    \label{fig:Wirzbięta-05_PT09_469.png}
  \end{figure}

    % 23+35+21=79
  Glyphs: 79.

\end{description}

  


% broszura niekompletna + Koszykowa

% PISMA
% 1. Pismo nagłówkowe, fraktura H. Schönspergera. Stopień
% 1 w, = 14,5 mm. — Tabl. 417, 419, 420, 457, 464, 467,
% 469, 471.
% 2. Pismo nagłówkowe i tekstowe, fraktura H. Schönspergera. Stopień 20 ww. = ca 160 mm. — Tabl. 420, 464,
% 469.
% 3. Pismo tekstowe, szwabacha M⁸¹, Stopień 20 ww. =
% 113 mm. — Tabl. 417, 419, 420, 456—458, 464, 467—470.
% 4. Pismo tekstowe, szwabacha M⁸¹, Stopień 20 ww. ==
% 88 mm. — Tabl. 457, 468—470.
% 5. Pismo komentarzowe, szwabacha M⁸¹. Stopień 20 ww. =
% 71/2 mm. — Tabl. 456, 457, 469, 470.
% 6. Czcionki odsyłaczowe, szwabacha, użyte sporadycznie
% w cz. II Postylli polskiej (nr 4). Wysokość 1 mm. — Tabl.
% 470.
% 7. Czcionki hebrajskie. Wysokość 8 mm. — Tabl. 471.
% 8. Pismo tekstowe, antykwa polska ozdobna. Stopień
% 1 w. = 6 mm. — Tabl. 458.
% tabele niewskanowane

\section{X: Wirzbieta}
\label{sec:x}

No font tables.

% tylko drzeworyty i inicjały

% indeks do OCR

% zasób omówiony

% Erratum do zesztu IX !!!!

% s. [19]

% PISMA

% 9 Antykwa tekstowa. Stopień 20 ww. =
% 92 mm. — Tabl. 5rą.
% 10. Antykwa tekstowa. Stopień 20 ww. =
% 114 mm. — Tabl. 494, 514.
% 11 Wersaliki tytułowe, antykwa. Wysokość
% 7/8 mm. — Tabl. 514.
% 12 Antykwa nagłówkowa. Stopień 5 ww. =
% 70 mm, wersaliki wysokość 10 mm. — Nie re-
% produkowana.
% 13 Kursywa komentarzowa. Stopień 20 ww. =
% 76 mm. — Nie reprodukowana.
% 14 Kursywa tekstowa. Stopień 20 ww. =
% 88/89 mm. — Tabl. 487, 493, 494.
% 15 Kursywa tekstowa. Stopień 20 ww. =
% 113/114 mm. — Tabl. 494, 514.
% 16Pismo greckie komentarzowe. Wysokość czcion-
% ki 2 mm. — Nie reprodukowane.

% tabele do wycięcia?



\section{XI: Wirzbiętowie}
\label{sec:xi}

Font tables 7. Glyphs: 545.


% * W niniejszym wykazie podano pełną rejestrację
% tylko tych elementów, które występują na tablicach we
% wszystkich trzech Zeszytach (IX—XI), zawierających
% monografię drukarni Wirzbięty. Jeżeli natomiast jedno-
% rodne gatunkowo elementy (np. winietki, drzeworyty)
% reprodukowano głównie w Żeszycie IX lub X i tam za-
% mieszczono ich wykaz, później zaś pojawiały się tylko spo-
% radycznie, zestaw obejmuje grupy przynależne do Ze-
% szytu XI.

% PISMA
% 1. Pismo nagłówkowe, fraktura H. Schönspergera.
% Stopień: I w. = 14,5 mm. — Tabl. 417, 419,
% 420, 457, 464, 467, 469, 471, 479, 486, 491,
% 492, 523, 557 500, 570.

% 2. Pismo nagłówkowe i tekstowe, fraktura H.
% Schönspergera. Stopień 20 ww. = ca 160 mm.
% — Tabl. 420, 464, 469, 476, 479, 486, 491,
% 492, 523, 527, 560, 501, 570.

% 3. Pismo tekstowe szwabacha M⁸¹. Stopień 20 ww.
% = 139 mm. — Tabl. 417, 419, 420, 456-458,
% 464, 467-470, 476, 479, 487, 496, 515, 523, 527,
% 560, 561, 570.

% 4. Pismo tekstowe, szwabacha M⁸¹. Stopień 20
% ww. = 88 mm.— Tabl. 457, 468-470, 476, 479,
% 487, 491, 492, 515, 516, 521, 523, 501, 508, 570.

% 5. Pismo komentarzowe, szwabacha M⁸¹. Stopień
% 20 ww. = 71/2 mm. — Tabl. 456, 457, 469,
% 470, 487, 521, 523, 570.

% 6. Czcionki odsyłaczowe, szwabacha. Wysokość
% 1 mm. — Tabl. 470, 568.

% 7. Czcionki hebrajskie. Wysokość 8 mm. —
% Tabl. 471.

% 8. Antykwa polska dorobiona do pisma 10. —
% Tabl. 458. Zob. p. 10.

% 9. Pismo tekstowe, antykwa. Stopień 20 ww.
% 92 mm. — Tabl. 458, 514, 515, 562, 568.

% 10. Pismo tekstowe, antykwa. Stopień 20 ww. ==
% 113/114 mm. Zob. p. 8. — Tabl. 494, 514,
% 522, 529, 563, 568.

% 11. Wersaliki tytułowe, antykwa. Wysokość 7 mm.
% — Tabl. 514, 522, 529, 540, 563, 564, 568.

% 11a. Wersaliki tytułowe greckie, dorobione do p.
% 11. — Tabl. 564.

% 12. Pismo nagłówkowe, antykwa. Stopień 5 ww.
% = 70 mm, wersaliki ro mm. — Tabl. 522,
% 529-532, 540, 543, 556, 564.

% 13. Pismo komentarzowe, kursywa. Stopień 20 ww.
% = 76 mm. — Tabl. 565.

% 14. Pismo tekstowe, kursywa. Stopień 20 ww.
% = 88/89 mm. — Tabl. 487, 493, 494, 521-523,
% 540, 561, 565, 568.

% 15. Pismo tekstowe, kursywa. Stopień 20 ww.
% = 113/114. mm. — Tabl. 494, 514, 521, 522,
% 566, 567.

% 16. Pismo greckie, tekstowe. Wysokość czcionki
% 2 mm. — Tabl. 567.

% 17. Pismo hebrajskie. Wysokość czcionki 3 mm bez
% przedłużek. — Tabl. 567.

% zasób omówiony

% Errata do IX i X!

% s. 46

% broszura indeks do OCR!

\begin{description}
\item[9] 


\begin{figure}[H]
  \centering
    \PTfont{76}{0.9}{Wirzbięta-09_PT11_562.png}
%  \includegraphics[width=0.9\textwidth]{img/Wirzbieta_pismo09_tab562}
    \caption{\Junicode Wirzbięta: 9. Pismo tekstowe, antykwa. Stopień
      20 ww. 92 mm. — Tabl. 458, 514, 515, 562, 568.}
  \label{fig:Wirzbięta_pismo09tab562}
%  24+35+33+30=122
\end{figure}

Glyphs: 122.

\item[10+8] 
\begin{figure}[H]
  \centering
    \PTfont{77}{0.9}{Wirzbięta-10+8_PT11_563.png}
    % \includegraphics[width=0.9\textwidth]{img/Wirzbieta_pismo10+8_tab563}
    \caption{\Junicode Wirzbięta: 10. Pismo tekstowe, antykwa. Stopień
      20 ww. == 113/114 mm. Zob. p. 8. — Tabl. 494, 514, 522, 529,
      563, 568. 8. Antykwa polska dorobiona do pisma 10. —
      Tabl. 458. Zob. p. 10.}
  \label{fig:Wirzbięta_pismo10_8tab563}
%   24+33+30+18=105\\razem 227
\end{figure}

Glyphs: 105.

\item[11+11a] 
\begin{figure}[H]
  \centering
  \PTfont{78}{0.9}{Wirzbięta-11+11a_PT11_564.png}
  %\includegraphics[width=0.9\textwidth]{img/Wirzbieta_pismo11+11a_tab564}
  \caption{\Junicode Wirzbięta: 11. Wersaliki tytułowe,
    antykwa. Wysokość 7 mm. — Tabl. 514, 522, 529, 540, 563, 564,
    568. [564]. 11a. Wersaliki tytułowe greckie, dorobione do
    p. 11. — Tabl. 564.}
  \label{fig:Wirzbięta_pismo11+11atab564}
%  21+10=31\\razem 258
\end{figure}

Glyphs: 31.

\item[12] 
\begin{figure}[H]
  \centering
  \PTfont{79}{0.9}{Wirzbięta-12_PT11_564.png}
 % \includegraphics[width=0.9\textwidth]{img/Wirzbieta_pismo12_tab564}
  \caption{\Junicode Wirzbięta: 12. Pismo nagłówkowe, antykwa. Stopień
    5 ww. = 70 mm, wersaliki 10 mm. — Tabl. 522, 529-532, 540, 543,
    556, 564.}
  \label{fig:Wirzbięta_pismo12tab564}
%  14+16+25+18=55\\razem 313
\end{figure}

Glyphs: 55.

\item[13] 


\begin{figure}[H]
  \centering
  \PTfont{80}{0.9}{Wirzbięta-13_PT11_565.png}
%\includegraphics[width=0.9\textwidth]{img/Wirzbięta_pismo13_tab565.png}
  \caption{\Junicode Wirzbięta: 13. Pismo komentarzowe,
    kursywa. Stopień 20 ww. = 76 mm. — Tabl. 565.}
  \label{fig:Wirzbięta_pismo13_tab565.png}
%20+35+6+1=62\\razem 659
\end{figure}

Glyphs: 62

\item[14] 
\begin{figure}[H]
  \centering
  \PTfont{81}{0.9}{Wirzbięta-14_PT11_565.png}
 % \includegraphics[width=0.9\textwidth]{img/Wirzbięta_pismo14_tab565.png}
  \caption{\Junicode Wirzbięta: 14. Pismo tekstowe, kursywa. Stopień
    20 ww. = 88/89 mm. — Tabl. 487, 493, 494, 521-523, 540, 561, 565,
    568.}
  \label{fig:Wirzbięta_pismo14_tab565.png}
%  26+39+38+31=134\\razem 597
\end{figure}

Glyphs: 134.

\item[15] 

\begin{figure}[H]
  \centering
  \PTfont{82}{0.9}{Wirzbięta-15_PT11_566.png}
%\includegraphics[width=0.9\textwidth]{img/Wirzbięta_pismo15_tab566.png}
  \caption{\Junicode Wirzbięta: 15. Pismo tekstowe, kursywa. Stopień
    20 ww. = 113/114. mm. — Tabl. 494, 514, 521, 522, 566, 567.}
  \label{fig:Wirzbięta_pismo15_tab566.png}
%  25+34+31+28+10=128\\razem 463
\end{figure}

Glyphs: 128.

% \item[1-5] 
%   \begin{figure}[H]
%   \centering
% \includegraphics[width=0.9\textwidth]{img/Wirzbięta_pismo1-5u_tab570.png}
%   \caption{Wirzbięta_pismo1-5u_tab570.png}
%   \label{fig:Wirzbięta_pismo1-5u_tab570.png}
%   4+12+1+5+10+10=42\\razem  355
% \end{figure}


% 565: pisma 13, 14

% 566: pismo 15

% 567: 16 17 greka

\end{description}
\section{XII: Szarfenberg}
\label{sec:xii}

% \begin{description}
% \item[181] Szarfenberg
% \end{description}

No font tables.

\section{Statistics}
\label{sec:tatystyka}


\begin{itemize}
\item Hochfeder. Font tables: 9. Glyphs: 953.
\item Ungler ---  pierwsza drukarnia. Font tables: 10. Glyphs: 934.
\item Haller. Font tables: 14. Glyphs: 1287.
\item Ungler --- druga drukarnia. Font tables: 11. Glyphs: 1047.
\item Ungler --- druga drukarnia. Font tables: 10. Glyphs: 930.
\item Augezdecki. Font tables 19. Glyphs: 1483.
\item Wirzbięta. Font tables 5. Glyphs: 448.
\item Wirzbiętowie. Font tables 7. Glyphs: 545.
\end{itemize}

Total tables: 75.

Total glyphs: 7627.

% 65 tabel


%   934 Ungler 1 +106
%  1503 Haller
%  1047 Ungler2
%   762 Ungler2 (VII)
%  1475 Auzdecki
%   659 Wirzbięta
% razem 6380 + 106 = 6486



%\clearpage
%\printbibliography{}
\end{document}


%%% Local Variables: 
%%% coding: utf-8-unix
%%% mode: latex
%%% TeX-master: t
%%% TeX-PDF-mode: t
%%% TeX-engine: xetex
%%% End: 
