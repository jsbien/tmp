% 10.5281/zenodo.14992305
% https://zenodo.org/uploads/14992305
% TO DO:
% https://en.wikipedia.org/wiki/Latin_delta
% https://en.wikipedia.org/wiki/Insular_script
% number tabular also in superscripts ???
% line the baseline of images ???
\documentclass[12pt]{article}
\usepackage[a3paper,margin=1.5cm]{geometry}
\usepackage{fontspec}
\newfontfamily{\Junicode}{Junicode}[Numbers=Lining]
%\fontspec[Numbers=Lining]{Junicode}
\newcommand{\J}[1]{{\Junicode #1}}
% \usepackage{polyglossia}
% \setmainlanguage{polish}
% \setotherlanguage{english}
%\usepackage{csquotes}

\usepackage{metalogo}
% \usepackage[polish]{varioref}
% % dla varioref!
% \def\eob{ę}
\usepackage{xcolor}


\usepackage{relsize}

%\usepackage{float}

\usepackage{caption}

\usepackage[verbose]{hyperref}

\usepackage{graphicx}
% [hyphens]: options clash
\usepackage{url}
%\usepackage{natbib}

% program name
\newcommand{\pname}[1]{\textsf{#1}}


% file name
\newcommand{\fname}[1]{\texttt{#1}}

\newcommand{\uname}[1]{\texttt{'#1'}}
\newcommand{\ucode}[1]{\texttt{U+#1}}
\newcommand{\usi}[1]{\texttt{#1}}

% Aletheia
\newcommand{\aname}[1]{\texttt{#1}}
\newcommand{\acode}[1]{\texttt{#1}}

% MUFI
\newcommand{\mname}[1]{\texttt{'#1 \textsc{<mufi>'}}}
\newcommand{\mcode}[1]{\texttt{M+#1}}



%\usepackage{draftwatermark}
% \usepackage[doublespacing]{setspace}

\usepackage[draft]{fixme}

% nie działa:?
%\renewcommand{\topfraction}{0.9}
\renewcommand{\floatpagefraction}{0.9}	% require fuller float pages
\renewcommand{\topfraction}{0.9}	% max fraction of floats at top
\setcounter{topnumber}{5}
\setcounter{totalnumber}{5}  

\renewcommand{\labelenumii}{\arabic{enumii}.}

% \vrefwarning

% https://tex.stackexchange.com/questions/54136/hyperref-link-spans-a-pagebreak-looks-ugly
% nie zawsze działa!!!

% retrieve absolute page numbers (physical pages, as opposed to the
% ‘logical’ page number that is normally typeset when a page number is
% requested;
% \usepackage{zref-abspage}

\usepackage{expex}


\lingset{glhangstyle=none}
\defineglwlevels{pismo,nr}
\newcommand{\bg}{\begingl}

% 1 height
% 2 image

%\newcommand{\PTglyph}[2]{\includegraphics[height=#1ex]{glyphs/#2}}
% \newcommand{\PTglyph}[2]{\includegraphics[height=8ex]{glyphs/#2}}
\newcommand{\PTglyph}[2]{\includegraphics[height=7ex]{glyphs/#2}}
%\newcommand{\PTglyph}[2]{\includegraphics[height=6ex]{glyphs/#2}}
\newcommand{\PTglyphid}[1]{#1}

\parindent0pt

\begin{document}
%\gappto\captionslingua{\renewcommand{\chaptername}{Caput}}
%\gappto\captionspolish{\renewcommand{\figurename}{Ilustracja}}


\title{POLONIA TYPOGRAPHICA
  SAECULI SEDECIMI\\
  {\relsize{-2} TŁOCZNIE POLSKIE XVI STULECIA\\ MONOGRAFIE I PODOBIZNY
    ZASOBÓW DRUKARSKICH}\\Reconstructed font tables\\
  (draft)}

\author{Janusz S. Bień (editor)}

\date{\today}

\maketitle

\catcode`\&=11
\catcode`\|=11
\catcode`\_=11

\def\apostrof{`}


% dodac indeks!:
% \catcode`\`=\active
% \def`#1{\fbox{{\znak#1}}}

\def\Hb#1{{\fontspec{Junicode}#1}}

\newcommand{\alfa}{\textit{alpha} (\J{α})}
\renewcommand{\alpha}{\textit{alpha} (\J{α})}
\renewcommand{\beta}{\textit{beta} (\J{β})}
\renewcommand{\delta}{\textit{delta} (\J{δ})}
\renewcommand{\epsilon}{\textit{epsilon} (\J{ε})}
\renewcommand{\eta}{\textit{eta} (\J{η})}
\renewcommand{\zeta}{\textit{zeta} (\J{ζ})}
\renewcommand{\theta}{\textit{theta} (\J{θ})}
\renewcommand{\gamma}{\textit{gamma} (\J{γ})}
\renewcommand{\chi}{\textit{chi} (\J{χ})}
\renewcommand{\kappa}{\textit{kappa} (\J{κ})}
\renewcommand{\iota}{\textit{iota} (\J{ι})}
\renewcommand{\lambda}{\textit{lambda} (\J{λ})}

\newpage

\section{Introduction}
\label{sec:introduction}

% – teka I – Kasper Hochfeder (1503–1505), wyd. II 1968;
% – teka II – Jan Haller (1505–1525), wyd. II 1963;
% – teka III – Florian Ungler (1510–1516), wyd. 1959;
% – teka IV – Jan Haller (1505–1525), wyd. 1962;
% – teka V – Florian Ungler (1521–1536), wyd. 1964;
% – teka VI – Florian Ungler (1521–1536), wyd. 1966;
% – teka VII – Florian Ungler (1521–1536), wyd. 1970;
% – teka VIII – Aleksander Augezdecki (1549–1561?),
% wyd. 1972;
% – teka IX – Maciej Wirzbięta (1555/7–1605), wyd. 1974;
% – teka X – Maciej Wirzbięta (1555/7–1605), wyd. 1975;
% – teka XI – Maciej i Paweł Wirzbiętowie (1555/7–
% –1609), wyd. 1981;
% – teka XII – Maciej Szarfenberg (1527–1547), wyd. 1981.

        \begin{itemize}
        \item I u prawników
        \item II u prawników
      \item III. Ungler — pierwsza drukarnia\\
        {\tiny \url{https://polona.pl/preview/f021c751-e1bd-41ae-bb39-117399724bac}}
%        Bułhak, Henryk (1930- ) - Polonia typographica saeculi sedecimi  zbiór podobizn zasobu drukarskiego tłoczni polskich XVI stulecia. Z. 3, Pierwsza drukarnia Floriana Unglera 1510-1516 - f021c751-e1bd-41ae-bb39-117399724bac.pdf
\item IV u prawników
      \item V. Ungler — druga drukarnia\\[.5ex]
        {\tiny\url{https://crispa.uw.edu.pl/object/files/764299/display/Default}\\
          \url{https://polona.pl/preview/29ab4ad7-f0a0-46c6-9f90-3d65e6407119}}
%        _- Polonia typographica saeculi sedecimi  zbiór podobizn zasobu drukarskiego tłoczni polskich XVI stulecia. Z. 5, Druga drukarnia Floriana Unglera 1521-1536  tabl. 176-245 - 29ab4ad7-f0a0-46c6-9f90-3d65e6407119.pdf
%        brakuje tabel ????
        % Crispa
      \item VI. Ungler — druga drukarnia\\
        {\tiny \url{https://polona.pl/preview/6e3ec50b-4be1-4581-945c-5665f4917178}}
 %       _- Polonia typographica saeculi sedecimi  zbiór podobizn zasobu drukarskiego tłoczni polskich XVI stulecia. Z. 6, Druga drukarnia Floriana Unglera 1521-1536  tablice 246-310 - 6e3ec50b-4be1-4581-945c-5665f4917178.pdf
      \item VII. Ungler — druga drukarnia\\
        {\tiny\url{https://polona.pl/preview/4d111123-7add-4a11-8d32-93635145ef4b}}
%  _- Polonia typographica saeculi sedecimi  zbiór podobizn zasobu drukarskiego tłoczni polskich XVI stulecia. Z. 7, Druga drukarnia Floriana Unglera 1521-1536  tablice 311-365 - 4d111123-7add-4a11-8d32-93635145ef4b.pdf      
\item VIII \url{https://polona.pl/preview/a3a4b39c-9cfa-434b-ba10-c2d8c7d89f62}
%  _- Polonia typographica saeculi sedecimi  zbiór podobizn zasobu drukarskiego tłoczni polskich XVI stulecia. Z. 8, Aleksander Augezdecki  Królewiec - Szamotuły 1549-1561  tabl. 366-415 - a3a4b39c-9cfa-434b-ba10-c2d8c7d89f62.pdf
\item IX \url{https://polona.pl/preview/a27a1dd3-0994-486b-9b87-5a4187bbeec2}
%  _- Polonia typographica saeculi sedecimi  zbiór podobizn zasobu drukarskiego tłoczni polskich XVI stulecia. Z. 9, Maciej Wirzbięta, Kraków 15557-1605  tablice 416-475 - a27a1dd3-0994-486b-9b87-5a4187bbeec2.pdf
\item X \url{https://polona.pl/preview/3717ea21-52eb-4dab-af8d-b84d897c71a1}
%  _- Polonia typographica saeculi sedecimi  zbiór podobizn zasobu drukarskiego tłoczni polskich XVI stulecia. Z. 10, Maciej Wirzbięta, Kraków 15557-1605  tabl. 476-520 - 3717ea21-52eb-4dab-af8d-b84d897c71a1.pdf
\item XI ???
  http://polona.pl/preview/3be15c17-50a5-45f8-b4ff-6571cbd515a0
  Crispa!
      \item XII. Szarfenberg\\
        {\tiny \url{https://crispa.uw.edu.pl/object/files/754258/display/Default}}
        http://polona.pl/preview/0e5c0b64-9360-4d2f-824b-bb6e9d4bebfd
      \end{itemize}


For more information about \textsc{POLONIA TYPOGRAPHICA SAECULI
  SEDECIMI} please consult
e.g. \url{https://github.com/jsbien/early_fonts_inventory}.
The repository contains also the \LuaLaTeX source of a present paper.

\bigskip

THe PDF version is available as ??? on Zenodo ???

\bigskip \textbf{Please remember this is a draft!} Some details still
require verification and improvements.

\subsection{CATALOGUS LIBRORUM}
\label{sec:catalogus-librorum}

By \textit{CATALOGUS LIBRORUM} we mean the list of publications
included in the booklet accompanying the plate in question. We
reference it by the item number (sometimes supplemented by a letter)
preceeded by the fascicule Roman number, e.g. `VIII:4a'.


\subsection{Haebler's font classification}
\label{sec:haebl-font-class}


Symbols such as \Hb{M¹⁶}, \Hb{M¹⁸}, \Hb{M⁴⁸}, \Hb{M⁶⁰}, \Hb{M⁹¹},
\Hb{Q|u}, \Hb{C}, \Hb{F7}, \Hb{G}, \Hb{K5} are explained in the
publication:

\begin{quote}
  Konrad Haebler:\\ «Typenrepertorium der Wiegendrucke» (the series
  \textit{Sammlung Bibliothekswissenschaftlicher Arbeiten})
  \begin{enumerate}
  \item Abteilung I: Deutschland und seine Nachbarlaender (1905).
  \item Abteilung II: Italien, die Niederlande, Frankreich, Spanien und Portugal (1908).
  \item Abteilung III:
    \begin{enumerate}
    \item Tabellen I: Antiqua-Typen (1909).
    \item Tabellen II: Gotische Typen" (1910)
    \end{enumerate}
  \item Ergänzungsband I (suplement, 1922).
  \item Ergänzungsband II (suplement, 1924).
  \end{enumerate}
\end{quote}
Reprinted in 1968, digitized in 2020 by Kujawsko-Pomorska Biblioteka
Cyfrowa (\url{https://kpbc.umk.pl/publication/222829}). Some volumes
digitized earlier by Google Books and Hathi Trust Digital Library but,
as of today, they seem available only for searching.

\noindent
Cf. also \url{https://tw.staatsbibliothek-berlin.de/}.

\subsection{Wierzbowski's bibliography}
\label{sec:wierzb-bibl}

The abbreviations in the form ``Wierzb. 891,'' refer to items (not
page numbers) in the bibliography \textit{Bibliographia Polonica XV ac
  XVII ss. quae in bibliotheca Universitatis Caesareare Varsoviensis
  asservantur} by Teodor Wierzbowski, published in 1889 and available
in the Polona digital library as
\url{https://polona.pl/item/96996417}.

\subsection{Estreicher's bibliography}
\label{sec:estr-bibl}

The abbreviations ``Estr.,'' refer Estreichers' bibliography, called
"most outstanding bibliography of Polish books, and probably one of
the most famous bibliographies in the
world".\footnote{\url{https://en.wikipedia.org/wiki/Karol_Estreicher_(senior)}}. It
was started by Karol Estrecher (1827--1908), continued by his son
Stanisław Estreicher (1869--1939) and finished by his grandson Karol
Estrecher(1906--1984).For most of the volumes the copyright has
expired, so they were reprinted, they are also (original or reprints)
available in several digital librarries,
e.g. \url{https://kpbc.umk.pl/dlibra/publication/13947} .
Additionally there is also \textit{ELEKTRONICZNA BAZA BIBLIOGRAFII
  ESTREICHERA} (EBBE), an electronic version of the
bibliographies\footnote{\url{https://www.estreicher.uj.edu.pl/}}; the
database includes also the scan of all the volumes, but there are some
restrictions on their usage.

The bibliography by Karol Estreicher (senior) consists of several
parts. The volumes have two numbers: the number in the whole
bibliography and in the specific part.
  \begin{itemize}
  \item Bibliografia polska. Cz. 1, Stólecie [!] XIX., volumes 7
  \item 
    Bibliografia polska. Cz. 2, Stólecie [!] XV-XIX spis chronologiczny. : volumes 3, one in
    two volumes (global volume numbers 8--11)
      \item 
    Bibliografia polska. Cz. 3, Stólecie [!] XV-XVIII w układzie
    abecadłowym: volumes 22 (global volume numbers 12--33)
  \item 
    Bibliografia polska. Cz. 4, Bibliografia polska XIX. stólecia [!] :
    lata 1881-1900: volumes 4
  \end{itemize}
  There are  also two unnumbered volume
  \begin{itemize}
  \item Bibliografia polska XV.-XVI. stólecia : zestawienie chronologiczne
    7200 druków w kształcie rejestru do Bibliografii, tudzież spis
    abecadłowy tych dzieł, które dochowały się w bibliotekach polskich
  \item 
    Bibliografia polska XIX. stulecia. Zeszyt dodatkowy, 1871-1873
  \end{itemize}

  The volumes authored by Stanisław Estreicher (with some use of his
  father manuscripts) include volumes 34--36 suplementing the part 3.

  The references to the bibliography contains the global volume number
  and the page number, e.g.  the reference ``Estr. XV. 242;
  XXXIV. 55'' refer to the mentions of Zaborowski's \textit{Tractatus
    contra malos divites et usurarios} which can be seen on page 242
  of volume 15\footnote{See e.g.
    \url{https://www.estreicher.uj.edu.pl/_skany/Bibliografia_Staropolska/15_Tom_XV/0257_0242.jpg}}
  and page 55 of volume 34\footnote{See e.g.
    \url{https://www.estreicher.uj.edu.pl/_skany/Bibliografia_Staropolska/34_Tom_XXXIV/0057_0055.jpg}}.

\subsection{Other abbreviations}
\label{sec:other-abbreviations}


TO DO!

% Piekarski

% Knihopis

% Knih.
% Knihopis ćeskosłovenskych tiskń od doby nejstarsi aź do konce XVIH.
% stoleti. Red. Z. Tobolka. DilII. Tisky z let 1501—1800. Ć. 1 i nast. V Praze
% 1936 i nast.
% https://www.digitalniknihovna.cz/nkp/periodical/uuid:595d5a00-baf9-11e3-b74a-5ef3fc9ae867

% BM Germ.: Short-title catalogue
% Short-title catalogue of book printed in the German-speaking countries
% and German books printed in other countries from 1455 to 1600 now in
% the British Museum. London 1962.

% Bohonos Ossol.

% Drukarze IV 

% Boh. Ossol.
% Drukarze IV

% M. Bohonos: Katalog starych druków Biblioteki Zakładu Narodowego
% im. Ossolińskich. Polonica wieku XVI. Z materiałów rejestracyjnych
% zebranych zespołowo pod kierownictwem Kazimierza Zatheya opraco-
% wała... Wrocław 1965.

% Drukarze dawnej Polski od XV do XVIII wieku. T. 4: Pomorze. Oprac.
% A. Kawecka-Gryczowa oraz K. Korotajowa. Wrocław 1962.

% War.???

% The abbreviation ``K.'' occuring in some descriptions stands for
% \textit{Karty}, i.e. \textit{sheets}.

\newpage
\section{\Huge Font tables}
\label{sec:font-tables}

\newpage

%%%%%%%%%%%%%%%%%%%%%%%%%%%%%%%%%%%%%%%%%%%%%%%%%%%%%%%%%%%%%%%%%%%%%%%%%%%%%%
% Tab. 01 Aleksander Augezdecki pismo 1
%%%%%%%%%%%%%%%%%%%%%%%%%%%%%%%%%%%%%%%%%%%%%%%%%%%%%%%%%%%%%%%%%%%%%%%%%%%%%%

% 01,Augezdecki-01_PT08_402.djvu,Augezdecki,01,08,402

% Author "Paulina Buchwald-Pelcowa"
% Title "Aleksander Augezdecki 1549-1561"
% Editor "Alodia Kawecka Gryczowa"
% Series "Polonia Typographica Saeculi Sedecimi: zbiór podobizn zasobu drukarskiego tłoczni polskich XVI stulecia"
% Fascicule "VIII"
% Publisher "Zakład Narodowy imienia Ossolińskich — Wydawnictwo"
% Addres "Kraków  Wrocław Warszawa"
% Year "1972"
% Note "1. Pisma tekstowe, szwabacha M⁸¹. Stopień 20 ww. = 102—103 mm (tercja). — Tabl. 402—404. [402]"
% Note1 "Character set table prepared by Paulina Buchwald-Pelcowa"
% Note2 "Scan (prepared by Biblioteka Uniwersytecka w Warszawie from their own copy) converted to DjVu with didjvu by Janusz S. Bień"
% URL "https://github.com/jsbien/early_fonts_inventory/"

\newcommand{\pismoPL}[1]{{\relsize{2}\Junicode\textbf{#1}}}
%\textbf{1. Pisma tekstowe, szwabacha M⁸¹. Stopień 20 ww. = 102—103 mm (tercja). — Tabl. 402—404.}

\pismoPL{Aleksander Augezdecki 1. Pisma tekstowe, szwabacha M⁸¹. Stopień 20 ww. = 102—103 mm (tercja). — Tabl. 402—404.}

\newcommand{\pismoEN}[1]{{\relsize{1}\Junicode\begin{quote}#1\end{quote}}}

% \begin{quote}
%   1. Schwabacher text script. Typeface M⁸¹. Type size 20 lines = 102—103
%   mm (tertia) - Plate 402-404.
% \end{quote}

\pismoEN{Aleksander Augezdecki 1. Schwabacher text script. Typeface M⁸¹. Type size 20 lines = 102—103 mm (tertia) - Plate 402-404.}

  
    \medskip

    \newcommand{\plate}[3]{\textbf{Plate #1} (fasc. #2, #3)}
    
%    \textbf{Plate 402} 

\plate{402}{VIII}{1972}
    
%     Original font table
    Prepared by Paulina Buchwald-Pelcowa.

    \medskip
    
    % Biblioteka Czartoryskich. Krakéw. P.B. P.
% 4*, [TESTAMENTUM NOVUM. Evangelium secundum Matthaeum. Trad. polon. Stanislaus Murzynowski]:
% Ewangelia Sw. Mateusza. Krélewiec, [Aleksander Augezdeckij, 1551. 4°.
% Karta LXXXIIIb.
% Estreicher XIII 26. Wierzbowski 135.
% Pismo 1: tekst i zestaw wraz z zestawem liter ze znakami diakrytycznymi polskimi. — Pismo 2: szwabacha tekstowa w marginaliach. — Pismo 3: szwabacha
% komentarzowa w marginaliach. — Pismo 7: nagtowek. — Rubryki «, 8, y, 6, 7 w zestawie — Cyfry 1 z pismem 3. — Cyfry 3 z pismem 3. — Przerywniki
% 5 z pismem 1 i 3.


      \bigskip

%        \textsc{Primary example:}

      \newcommand{\exampleBib}[1]{{\relsize{2}\Junicode\textbf{The
            example:}\\[2ex] CATALOGUS LIBRORUM \textbf{#1}}}
      \newcommand{\exampleBibExtra}[1]{{\relsize{2}\Junicode\textit{The
            plate contains also an example without a font
            table:}\\[2ex] CATALOGUS LIBRORUM \textbf{#1}}}

      \exampleBib{VIII:4a}
      \bigskip

\newcommand{\exampleDesc}[1]{{\relsize{0}\Junicode#1}}
\newcommand{\exampleDig}[1]{{\relsize{0}\Junicode \textbf{Digitization(s) [JSB]:} #1}}


\exampleDesc{[TESTAMENTUM NOVUM. Evangelium secundum Matthaeum. Trad. polon. Stanislaus Murzynowski]:
Ewangelia Sw. Mateusza. Królewiec, [Aleksander Augezdecki], 1551. 4°.}

\newcommand{\examplePage}[1]{{\relsize{0}\Junicode#1}}
\newcommand{\examplePageEN}[1]{{\relsize{0}\Junicode#1}}

\medskip
\examplePage{\textit{Karta LXXXIIIb.}}

\newcommand{\exampleLib}[1]{{\relsize{0}\Junicode \textbf{Library:} #1}}

\bigskip
\exampleLib{Biblioteka Czartoryskich. Kraków.}

\bigskip
\newcommand{\exampleRef}[1]{{\relsize{0}\Junicode \textbf{References:} #1}}

\exampleRef{\textit{Estreicher XIII 26. Wierzbowski 135.}}

\bigskip
\exampleDig{\url{https://www.dbc.wroc.pl/publication/33779} page 202.}

\newcommand{\examplePL}[1]{{\relsize{0}\Junicode#1}}

%      Pismo 1: tekst i zestaw wraz z zestawem liter ze znakami diakrytycznymi polskimi.

\bigskip

      \examplePL{Pismo 1: tekst i zestaw wraz z zestawem liter ze znakami diakrytycznymi polskimi.}
      
      \medskip

      % \begin{quote}
      %   Font 1. The text and the table including letters with Polish diacritical marks.
      % \end{quote}

\newcommand{\exampleEN}[1]{{\relsize{0}\Junicode\begin{quote}#1\end{quote}}}

\exampleEN{Font 1. The text and the table including letters with Polish diacritical marks.}


  % \textit{Testamentu Nowego Czesc Pierwsza Czterzei Euangelistowie
  %   swieći, Mattheusz, Marek, Lukasz I Ian, Z Greckiego ięzyka na
  %   Polski przelozeni i wykladem krotkiem obiasnieni}, 1551, page
  % LXXXIIIb.  Digitization
  % \url{https://www.dbc.wroc.pl/publication/33779} page 202.

  
  \bigskip
  
\newcommand{\fontID}[2]{{\relsize{1}\Junicode\textbf{Font identifier} (JSB): #1 (table #2)}}

%    \textbf{Font identifier} (JSB): Au-01 (table 01)

    \fontID{Au-01}{01}

\newcommand{\fontstat}[1]{{\relsize{1}\Junicode\textbf{Statistics} (JSB): #1 glyphs.}}

\fontstat{241}

    % \exdisplay \bg \gla
\input {t01_glyphs}
%//
%\glpismo
\glpismo
% 1
{\PTglyphid{Au-010101}}
% 2
{\PTglyphid{Au-010102}}
% 3
{\PTglyphid{Au-010103}}
% 4
{\PTglyphid{Au-010104}}
% 5
{\PTglyphid{Au-010105}}
% 6
{\PTglyphid{Au-010106}}
% 7
{\PTglyphid{Au-010107}}
% 8
{\PTglyphid{Au-010108}}
% 9
{\PTglyphid{Au-010109}}
% 10
{\PTglyphid{Au-010110}}
% 11
{\PTglyphid{Au-010111}}
% 12
{\PTglyphid{Au-010112}}
% 13
{\PTglyphid{Au-010113}}
% 14
{\PTglyphid{Au-010114}}
% 15
{\PTglyphid{Au-010115}}
% 16
{\PTglyphid{Au-010116}}
% 17
{\PTglyphid{Au-010117}}
% 18
{\PTglyphid{Au-010118}}
% 19
{\PTglyphid{Au-010119}}
% 20
{\PTglyphid{Au-010120}}
% 21
{\PTglyphid{Au-010121}}
% 22
{\PTglyphid{Au-010122}}
% 23
{\PTglyphid{Au-010123}}
% 24
{\PTglyphid{Au-010124}}
% 25
{\PTglyphid{Au-010125}}
% 26
{\PTglyphid{Au-010126}}
% 27
{\PTglyphid{Au-010127}}
% 28
{\PTglyphid{Au-010128}}
% 29
{\PTglyphid{Au-010201}}
% 30
{\PTglyphid{Au-010202}}
% 31
{\PTglyphid{Au-010203}}
% 32
{\PTglyphid{Au-010204}}
% 33
{\PTglyphid{Au-010205}}
% 34
{\PTglyphid{Au-010206}}
% 35
{\PTglyphid{Au-010207}}
% 36
{\PTglyphid{Au-010208}}
% 37
{\PTglyphid{Au-010209}}
% 38
{\PTglyphid{Au-010210}}
% 39
{\PTglyphid{Au-010211}}
% 40
{\PTglyphid{Au-010212}}
% 41
{\PTglyphid{Au-010213}}
% 42
{\PTglyphid{Au-010214}}
% 43
{\PTglyphid{Au-010215}}
% 44
{\PTglyphid{Au-010216}}
% 45
{\PTglyphid{Au-010217}}
% 46
{\PTglyphid{Au-010218}}
% 47
{\PTglyphid{Au-010219}}
% 48
{\PTglyphid{Au-010220}}
% 49
{\PTglyphid{Au-010221}}
% 50
{\PTglyphid{Au-010222}}
% 51
{\PTglyphid{Au-010223}}
% 52
{\PTglyphid{Au-010224}}
% 53
{\PTglyphid{Au-010225}}
% 54
{\PTglyphid{Au-010226}}
% 55
{\PTglyphid{Au-010227}}
% 56
{\PTglyphid{Au-010228}}
% 57
{\PTglyphid{Au-010229}}
% 58
{\PTglyphid{Au-010230}}
% 59
{\PTglyphid{Au-010231}}
% 60
{\PTglyphid{Au-010232}}
% 61
{\PTglyphid{Au-010233}}
% 62
{\PTglyphid{Au-010234}}
% 63
{\PTglyphid{Au-010235}}
% 64
{\PTglyphid{Au-010236}}
% 65
{\PTglyphid{Au-010237}}
% 66
{\PTglyphid{Au-010238}}
% 67
{\PTglyphid{Au-010239}}
% 68
{\PTglyphid{Au-010240}}
% 69
{\PTglyphid{Au-010241}}
% 70
{\PTglyphid{Au-010301}}
% 71
{\PTglyphid{Au-010302}}
% 72
{\PTglyphid{Au-010303}}
% 73
{\PTglyphid{Au-010304}}
% 74
{\PTglyphid{Au-010305}}
% 75
{\PTglyphid{Au-010306}}
% 76
{\PTglyphid{Au-010307}}
% 77
{\PTglyphid{Au-010308}}
% 78
{\PTglyphid{Au-010309}}
% 79
{\PTglyphid{Au-010310}}
% 80
{\PTglyphid{Au-010311}}
% 81
{\PTglyphid{Au-010312}}
% 82
{\PTglyphid{Au-010313}}
% 83
{\PTglyphid{Au-010314}}
% 84
{\PTglyphid{Au-010315}}
% 85
{\PTglyphid{Au-010316}}
% 86
{\PTglyphid{Au-010317}}
% 87
{\PTglyphid{Au-010318}}
% 88
{\PTglyphid{Au-010319}}
% 89
{\PTglyphid{Au-010320}}
% 90
{\PTglyphid{Au-010321}}
% 91
{\PTglyphid{Au-010322}}
% 92
{\PTglyphid{Au-010323}}
% 93
{\PTglyphid{Au-010324}}
% 94
{\PTglyphid{Au-010325}}
% 95
{\PTglyphid{Au-010326}}
% 96
{\PTglyphid{Au-010327}}
% 97
{\PTglyphid{Au-010328}}
% 98
{\PTglyphid{Au-010329}}
% 99
{\PTglyphid{Au-010330}}
% 100
{\PTglyphid{Au-010331}}
% 101
{\PTglyphid{Au-010332}}
% 102
{\PTglyphid{Au-010333}}
% 103
{\PTglyphid{Au-010334}}
% 104
{\PTglyphid{Au-010335}}
% 105
{\PTglyphid{Au-010336}}
% 106
{\PTglyphid{Au-010337}}
% 107
{\PTglyphid{Au-010339}}
% 108
{\PTglyphid{Au-010340}}
% 109
{\PTglyphid{Au-010341}}
% 110
{\PTglyphid{Au-010342}}
% 111
{\PTglyphid{Au-010343}}
% 112
{\PTglyphid{Au-010344}}
% 113
{\PTglyphid{Au-010401}}
% 114
{\PTglyphid{Au-010402}}
% 115
{\PTglyphid{Au-010403}}
% 116
{\PTglyphid{Au-010404}}
% 117
{\PTglyphid{Au-010405}}
% 118
{\PTglyphid{Au-010406}}
% 119
{\PTglyphid{Au-010407}}
% 120
{\PTglyphid{Au-010408}}
% 121
{\PTglyphid{Au-010409}}
% 122
{\PTglyphid{Au-010410}}
% 123
{\PTglyphid{Au-010411}}
% 124
{\PTglyphid{Au-010412}}
% 125
{\PTglyphid{Au-010413}}
% 126
{\PTglyphid{Au-010414}}
% 127
{\PTglyphid{Au-010415}}
% 128
{\PTglyphid{Au-010416}}
% 129
{\PTglyphid{Au-010417}}
% 130
{\PTglyphid{Au-010418}}
% 131
{\PTglyphid{Au-010419}}
% 132
{\PTglyphid{Au-010420}}
% 133
{\PTglyphid{Au-010421}}
% 134
{\PTglyphid{Au-010422}}
% 135
{\PTglyphid{Au-010423}}
% 136
{\PTglyphid{Au-010424}}
% 137
{\PTglyphid{Au-010425}}
% 138
{\PTglyphid{Au-010501}}
% 139
{\PTglyphid{Au-010502}}
% 140
{\PTglyphid{Au-010503}}
% 141
{\PTglyphid{Au-010504}}
% 142
{\PTglyphid{Au-010505}}
% 143
{\PTglyphid{Au-010506}}
% 144
{\PTglyphid{Au-010507}}
% 145
{\PTglyphid{Au-010508}}
% 146
{\PTglyphid{Au-010509}}
% 147
{\PTglyphid{Au-010510}}
% 148
{\PTglyphid{Au-010511}}
% 149
{\PTglyphid{Au-010512}}
% 150
{\PTglyphid{Au-010513}}
% 151
{\PTglyphid{Au-010514}}
% 152
{\PTglyphid{Au-010515}}
% 153
{\PTglyphid{Au-010516}}
% 154
{\PTglyphid{Au-010517}}
% 155
{\PTglyphid{Au-010518}}
% 156
{\PTglyphid{Au-010519}}
% 157
{\PTglyphid{Au-010520}}
% 158
{\PTglyphid{Au-010522}}
% 159
{\PTglyphid{Au-010523}}
% 160
{\PTglyphid{Au-010524}}
% 161
{\PTglyphid{Au-010525}}
% 162
{\PTglyphid{Au-010526}}
% 163
{\PTglyphid{Au-010527}}
% 164
{\PTglyphid{Au-010528}}
% 165
{\PTglyphid{Au-010529}}
% 166
{\PTglyphid{Au-010530}}
% 167
{\PTglyphid{Au-010531}}
% 168
{\PTglyphid{Au-010532}}
% 169
{\PTglyphid{Au-010533}}
% 170
{\PTglyphid{Au-010534}}
% 171
{\PTglyphid{Au-010535}}
% 172
{\PTglyphid{Au-010536}}
% 173
{\PTglyphid{Au-010537}}
% 174
{\PTglyphid{Au-010538}}
% 175
{\PTglyphid{Au-010539}}
% 176
{\PTglyphid{Au-010540}}
% 177
{\PTglyphid{Au-010541}}
% 178
{\PTglyphid{Au-010542}}
% 179
{\PTglyphid{Au-010601}}
% 180
{\PTglyphid{Au-010602}}
% 181
{\PTglyphid{Au-010603}}
% 182
{\PTglyphid{Au-010604}}
% 183
{\PTglyphid{Au-010605}}
% 184
{\PTglyphid{Au-010606}}
% 185
{\PTglyphid{Au-010607}}
% 186
{\PTglyphid{Au-010608}}
% 187
{\PTglyphid{Au-010609}}
% 188
{\PTglyphid{Au-010610}}
//
%%% Local Variables:
%%% mode: latex
%%% TeX-engine: luatex
%%% TeX-master: shared
%%% End:

% //
%\endgl \xe

%\end{flushleft}

\newpage
%%%%%%%%%%%%%%%%%%%%%%%%%%%%%%%%%%%%%%%%%%%%%%%%%%%%%%%%%%%%%%%%%%%%%%%%%%%%%%
% Tab. 02 Aleksander Augezdecki pismo 1a
%%%%%%%%%%%%%%%%%%%%%%%%%%%%%%%%%%%%%%%%%%%%%%%%%%%%%%%%%%%%%%%%%%%%%%%%%%%%%%

% Author "Paulina Buchwald-Pelcowa"
% Title "Aleksander Augezdecki 1549-1561"
% Editor "Alodia Kawecka Gryczowa"
% Series "Polonia Typographica Saeculi Sedecimi: zbiór podobizn zasobu drukarskiego tłoczni polskich XVI stulecia"
% Fascicule "VIII"
% Publisher "Zakład Narodowy imienia Ossolińskich — Wydawnictwo"
% Addres "Kraków  Wrocław Warszawa"
% Year "1972"
% Note "1. Pisma tekstowe, szwabacha M⁸¹. Stopień 20 ww. = 102—103 mm (tercja). — Tabl. 402—404. [403]"
% Note1 "Character set table prepared by Paulina Buchwald-Pelcowa"
% Note2 "Scan (prepared by Biblioteka Uniwersytecka w Warszawie from their own copy) converted to DjVu with didjvu by Janusz S. Bień"
    
\pismoPL{Aleksander Augezdecki 1. Pisma tekstowe, szwabacha M⁸¹. Stopień 20 ww. = 102—103 mm (tercja). — Tabl. 402—404.}

\pismoEN{Aleksander Augezdecki 1. Schwabacher text script. Typeface M⁸¹. Type size 20 lines = 102—103 mm (tertia) - Plate 402-404.}

\medskip

\plate{403}{VIII}{1972}

Prepared by Paulina Buchwald-Pelcowa.

\bigskip

\exampleBib{VIII:25}

\medskip
\bigskip

\exampleDesc{PIESNE Chwal Bozskych. Szamotuly, Aleksander Augezdecki, [25 I 1560—] 7 VI 1561. 2°. War. A.}

\medskip
\examplePage{\textit{Karta *₂b}}

\bigskip
\exampleLib{Biblioteka Czartoryskich. Kraków.}

\bigskip
\exampleRef{\textit{Estreicher XIX 91. Knihopis 12860.}}

\bigskip
\exampleDig{\url{https://cyfrowe.mnk.pl/dlibra/publication/13639/}, page 8.}

    \examplePL{Pismo 1: tekst i zestaw liter ze znakami diakrytycznymi czeskimi.}

    \medskip

    \exampleEN{Font 1. The text and the table of letters with Czech diacritical marks}

\bigskip

    \fontID{Au-01a}{02}

\fontstat{42}

\bigskip

% \exdisplay \bg \gla
\input {t02_glyphs.tex}
%//
%\glpismo
\glpismo
% 1
{\PTglyphid{Au-020101}}
% 2
{\PTglyphid{Au-020102}}
% 3
{\PTglyphid{Au-020103}}
% 4
{\PTglyphid{Au-020104}}
% 5
{\PTglyphid{Au-020105}}
% 6
{\PTglyphid{Au-020106}}
% 7
{\PTglyphid{Au-020107}}
% 8
{\PTglyphid{Au-020108}}
% 9
{\PTglyphid{Au-020109}}
% 10
{\PTglyphid{Au-020110}}
% 11
{\PTglyphid{Au-020201}}
% 12
{\PTglyphid{Au-020202}}
% 13
{\PTglyphid{Au-020203}}
% 14
{\PTglyphid{Au-020204}}
% 15
{\PTglyphid{Au-020205}}
% 16
{\PTglyphid{Au-020206}}
% 17
{\PTglyphid{Au-020207}}
% 18
{\PTglyphid{Au-020208}}
% 19
{\PTglyphid{Au-020210}}
% 20
{\PTglyphid{Au-020211}}
% 21
{\PTglyphid{Au-020212}}
% 22
{\PTglyphid{Au-020213}}
% 23
{\PTglyphid{Au-020214}}
% 24
{\PTglyphid{Au-020215}}
% 25
{\PTglyphid{Au-020216}}
% 26
{\PTglyphid{Au-020217}}
% 27
{\PTglyphid{Au-020219}}
% 28
{\PTglyphid{Au-020220}}
% 29
{\PTglyphid{Au-020221}}
% 30
{\PTglyphid{Au-020222}}
% 31
{\PTglyphid{Au-020223}}
% 32
{\PTglyphid{Au-020224}}
% 33
{\PTglyphid{Au-020225}}
% 34
{\PTglyphid{Au-020226}}
% 35
{\PTglyphid{Au-020227}}
% 36
{\PTglyphid{Au-020228}}
% 37
{\PTglyphid{Au-020229}}
% 38
{\PTglyphid{Au-020230}}
% 39
{\PTglyphid{Au-020231}}
% 40
{\PTglyphid{Au-020232}}
% 41
{\PTglyphid{Au-020233}}
% 42
{\PTglyphid{Au-020234}}
% 43
{\PTglyphid{Au-020235}}
//
%%% Local Variables:
%%% mode: latex
%%% TeX-engine: luatex
%%% TeX-master: shared
%%% End:

% //
%\endgl \xe


\newpage
%%%%%%%%%%%%%%%%%%%%%%%%%%%%%%%%%%%%%%%%%%%%%%%%%%%%%%%%%%%%%%%%%%%%%%%%%%%%%%
% Tab. 03 Aleksander Augezdecki pismo 1b
%%%%%%%%%%%%%%%%%%%%%%%%%%%%%%%%%%%%%%%%%%%%%%%%%%%%%%%%%%%%%%%%%%%%%%%%%%%%%%

\pismoPL{Aleksander Augezdecki 1. Pisma tekstowe, szwabacha M⁸¹. Stopień 20 ww. = 102—103 mm (tercja). — Tabl. 402—404.}

\pismoEN{Aleksander Augezdecki 1. Schwabacher text script. Typeface M⁸¹. Type size 20 lines = 102—103 mm (tertia) - Plate 402-404.}

\medskip

\plate{404}{VIII}{1972}

Prepared by Paulina Buchwald-Pelcowa.

\bigskip

\exampleBib{VIII:12}

\bigskip
\exampleDesc{CHRISTOPHORUS RUDOLFF: Die Coss. 4°. Krélewiec, Aleksander Augezdecki, 1553. 4°.}

\medskip
\examplePage{\textit{Karta 63a.}}

  \bigskip
\exampleLib{Biblioteka Czartoryskich. Kraków.}

\bigskip
\exampleRef{\textit{BM Germ.: Short-title catalogue s. 759}}

\bigskip
\exampleDig{\url{https://ds.ub.uni-bielefeld.de/viewer/api/v1/records/2014414/sections/LOG_0000/pdf/}}

\medskip

    \examplePL{Pismo 1: tekst i zestaw liter ze znakami diakrytycznymi niemieckimi.}

    \medskip

    \exampleEN{Font 1. The text and the table of letters with German diacritical marks [and mathematical notation]}

      \bigskip

%https://www.unicode.org/L2/L2024/24141-n5277-leibniz.pdf
%      Biblioteka Czartoryskich. Krakéw.
% P.B.P. 
% 12. CHRISTOPHORUS RUDOLFF: Die Coss. 4°. Krélewiec, Aleksander Augezdecki, 1553. 4°.
% Pismo 1: tekst i zestaw liter ze znakami diakrytycznymi niemieckimi. — Cyfry 4 z pismem 1.
% Karta 63a.
% BM Germ.: Short-title catalogue s. 759,

% https://www.unicode.org/L2/L2024/24141-n5277-leibniz.pdf
% https://www.researchgate.net/publication/230735404_From_the_second_unknown_to_the_symbolic_equation

%       12 karta 63a

%       1553
% CHRISTOPHORUS RUDOLFF: Die Coss. Ed.
% Michael Stifel. 4*. K. 104. (Arkusze A—Z Aa— Cc) —
% BM Germ. 759. Arkusze Dd—Zz Aaa—Zzz
% Aaaa—Zzzz Aaaaa—Zzzzz Aaaaaa—LlIII zob.
% poz. 138. *
% W niektórych egzemplarzach różnice w foliacji i sygnacji. Egzem-
% plarz B. Nar. XVI. Qu 115 k. ... 90,9, 92, 93, 95, 95—212...; egzem-
% płarz B. Czart. 52811 II k. ... 89—93, 95, 95—212..., karta Ca
% błędnie sygnowana B;.
% Pisma 1, 4—6, 13, 15. — Rubryka a, $, y. — Cyfry 1—4. —
% Przerywnik 2. — Inicjały 3, 15, 16, 21, 22, 27. — Drzeworyty
% 20, 21—23. - Li2.
      
% Digitization %%%%%which variant????
% https://ds.ub.uni-bielefeld.de/viewer/image/2014414/1/LOG_0000/
% https://old.maa.org/press/periodicals/convergence/mathematical-treasures-rudolffs-arithmetic-and-algebra
% https://old.maa.org/sites/default/files/images/upload_library/46/Swetz_2012_Math_Treasures/ColumbiaU/1302100032.png

\bigskip

    \fontID{Au-01b}{03}

    \fontstat{23}

    \bigskip
% \exdisplay \bg \gla
\input {t03_glyphs.tex}
%//
%\glpismo
\input {t03_glyphids.tex}
% //
%\endgl \xe


\newpage
%%%%%%%%%%%%%%%%%%%%%%%%%%%%%%%%%%%%%%%%%%%%%%%%%%%%%%%%%%%%%%%%%%%%%%%%%%%%%%
% Tab. 04 Aleksander Augezdecki pismo 2
%%%%%%%%%%%%%%%%%%%%%%%%%%%%%%%%%%%%%%%%%%%%%%%%%%%%%%%%%%%%%%%%%%%%%%%%%%%%%%

% Author "Paulina Buchwald-Pelcowa"
% Title "Aleksander Augezdecki 1549-1561"
% Editor "Alodia Kawecka Gryczowa"
% Series "Polonia Typographica Saeculi Sedecimi: zbiór podobizn zasobu drukarskiego tłoczni polskich XVI stulecia"
% Fascicule "VIII"
% Publisher "Zakład Narodowy imienia Ossolińskich — Wydawnictwo"
% Addres "Kraków  Wrocław Warszawa"
% Year "1972"
% Note "2 Pismo tekstowe, szwabacha M⁸¹. Stopień 20 ww. = 86—87 mm (cycero). — Tabl. 405, 406. [406]"
% Note1 "Character set table prepared by Paulina Buchwald-Pelcowa"

\pismoPL{Aleksander Augezdecki 2. Pisma tekstowe, szwabacha M⁸¹. Stopień 20 ww. = 86—87 mm (cycero). — Tabl. 402—404.}

\pismoEN{Aleksander Augezdecki 2. Schwabacher text script. Typeface M⁸¹. Type size 20 lines = 86—87 mm (cicero) - Plate 402-404.}

\medskip

\plate{405}{VIII}{1972}

Prepared by Paulina Buchwald-Pelcowa.

\bigskip

\exampleBib{VIII:6}

\bigskip
\exampleDesc{[TESTAMENTUM NOVUM. Trad. polon. Stanislaus Murzynowski]: Testamentu Nowego część pierwsza.
Królewiec, Aleksander Augezdecki, X 1551. 4°.}

\medskip
\examplePage{\textit{Karta B₁a.}}

  \bigskip
\exampleLib{Biblioteka Czartoryskich. Kraków.}

\bigskip
\exampleRef{\textit{Estreicher XIII 26, Wierzbowski 1288.}}

\bigskip
\exampleDig{\url{https://www.dbc.wroc.pl/publication/33779} page 13.}


\medskip

    \examplePL{Pismo 2: tekst i zestaw wraz z zestawem liter ze znakami diakrytycznymi polskimi.}

    \medskip

    \exampleEN{Font 2. The text and the table including letters with Polish diacritical marks.}


\bigskip

    \fontID{Au-02}{04}

    \fontstat{158}

\bigskip

% \exdisplay \bg \gla
\input {t04_glyphs.tex}
%//
%\glpismo
\glpismo
% 1
{\PTglyphid{Au-020101}}
% 2
{\PTglyphid{Au-020102}}
% 3
{\PTglyphid{Au-020103}}
% 4
{\PTglyphid{Au-020104}}
% 5
{\PTglyphid{Au-020105}}
% 6
{\PTglyphid{Au-020106}}
% 7
{\PTglyphid{Au-020107}}
% 8
{\PTglyphid{Au-020108}}
% 9
{\PTglyphid{Au-020109}}
% 10
{\PTglyphid{Au-020110}}
% 11
{\PTglyphid{Au-020111}}
% 12
{\PTglyphid{Au-020112}}
% 13
{\PTglyphid{Au-020113}}
% 14
{\PTglyphid{Au-020114}}
% 15
{\PTglyphid{Au-020115}}
% 16
{\PTglyphid{Au-020116}}
% 17
{\PTglyphid{Au-020117}}
% 18
{\PTglyphid{Au-020118}}
% 19
{\PTglyphid{Au-020119}}
% 20
{\PTglyphid{Au-020120}}
% 21
{\PTglyphid{Au-020121}}
% 22
{\PTglyphid{Au-020122}}
% 23
{\PTglyphid{Au-020123}}
% 24
{\PTglyphid{Au-020124}}
% 25
{\PTglyphid{Au-020125}}
% 26
{\PTglyphid{Au-020126}}
% 27
{\PTglyphid{Au-020127}}
% 28
{\PTglyphid{Au-020128}}
% 29
{\PTglyphid{Au-020201}}
% 30
{\PTglyphid{Au-020202}}
% 31
{\PTglyphid{Au-020203}}
% 32
{\PTglyphid{Au-020204}}
% 33
{\PTglyphid{Au-020205}}
% 34
{\PTglyphid{Au-020206}}
% 35
{\PTglyphid{Au-020207}}
% 36
{\PTglyphid{Au-020208}}
% 37
{\PTglyphid{Au-020209}}
% 38
{\PTglyphid{Au-020210}}
% 39
{\PTglyphid{Au-020211}}
% 40
{\PTglyphid{Au-020212}}
% 41
{\PTglyphid{Au-020213}}
% 42
{\PTglyphid{Au-020214}}
% 43
{\PTglyphid{Au-020215}}
% 44
{\PTglyphid{Au-020216}}
% 45
{\PTglyphid{Au-020217}}
% 46
{\PTglyphid{Au-020218}}
% 47
{\PTglyphid{Au-020219}}
% 48
{\PTglyphid{Au-020220}}
% 49
{\PTglyphid{Au-020221}}
% 50
{\PTglyphid{Au-020222}}
% 51
{\PTglyphid{Au-020223}}
% 52
{\PTglyphid{Au-020224}}
% 53
{\PTglyphid{Au-020225}}
% 54
{\PTglyphid{Au-020226}}
% 55
{\PTglyphid{Au-020227}}
% 56
{\PTglyphid{Au-020228}}
% 57
{\PTglyphid{Au-020229}}
% 58
{\PTglyphid{Au-020230}}
% 59
{\PTglyphid{Au-020231}}
% 60
{\PTglyphid{Au-020232}}
% 61
{\PTglyphid{Au-020233}}
% 62
{\PTglyphid{Au-020234}}
% 63
{\PTglyphid{Au-020235}}
% 64
{\PTglyphid{Au-020236}}
% 65
{\PTglyphid{Au-020237}}
% 66
{\PTglyphid{Au-020238}}
% 67
{\PTglyphid{Au-020239}}
% 68
{\PTglyphid{Au-020240}}
% 69
{\PTglyphid{Au-020241}}
% 70
{\PTglyphid{Au-020242}}
% 71
{\PTglyphid{Au-020301}}
% 72
{\PTglyphid{Au-020302}}
% 73
{\PTglyphid{Au-020303}}
% 74
{\PTglyphid{Au-020304}}
% 75
{\PTglyphid{Au-020305}}
% 76
{\PTglyphid{Au-020306}}
% 77
{\PTglyphid{Au-020307}}
% 78
{\PTglyphid{Au-020308}}
% 79
{\PTglyphid{Au-020309}}
% 80
{\PTglyphid{Au-020310}}
% 81
{\PTglyphid{Au-020311}}
% 82
{\PTglyphid{Au-020312}}
% 83
{\PTglyphid{Au-020313}}
% 84
{\PTglyphid{Au-020314}}
% 85
{\PTglyphid{Au-020315}}
% 86
{\PTglyphid{Au-020316}}
% 87
{\PTglyphid{Au-020317}}
% 88
{\PTglyphid{Au-020318}}
% 89
{\PTglyphid{Au-020319}}
% 90
{\PTglyphid{Au-020320}}
% 91
{\PTglyphid{Au-020321}}
% 92
{\PTglyphid{Au-020322}}
% 93
{\PTglyphid{Au-020323}}
% 94
{\PTglyphid{Au-020401}}
% 95
{\PTglyphid{Au-020402}}
% 96
{\PTglyphid{Au-020403}}
% 97
{\PTglyphid{Au-020404}}
% 98
{\PTglyphid{Au-020405}}
% 99
{\PTglyphid{Au-020406}}
% 100
{\PTglyphid{Au-020407}}
% 101
{\PTglyphid{Au-020408}}
% 102
{\PTglyphid{Au-020409}}
% 103
{\PTglyphid{Au-020410}}
% 104
{\PTglyphid{Au-020411}}
% 105
{\PTglyphid{Au-020412}}
% 106
{\PTglyphid{Au-020413}}
% 107
{\PTglyphid{Au-020414}}
% 108
{\PTglyphid{Au-020415}}
% 109
{\PTglyphid{Au-020416}}
% 110
{\PTglyphid{Au-020417}}
% 111
{\PTglyphid{Au-020418}}
% 112
{\PTglyphid{Au-020419}}
% 113
{\PTglyphid{Au-020420}}
% 114
{\PTglyphid{Au-020421}}
% 115
{\PTglyphid{Au-020422}}
% 116
{\PTglyphid{Au-020423}}
% 117
{\PTglyphid{Au-020424}}
% 118
{\PTglyphid{Au-020425}}
% 119
{\PTglyphid{Au-020426}}
% 120
{\PTglyphid{Au-020427}}
% 121
{\PTglyphid{Au-020428}}
% 122
{\PTglyphid{Au-020429}}
% 123
{\PTglyphid{Au-020501}}
% 124
{\PTglyphid{Au-020502}}
% 125
{\PTglyphid{Au-020503}}
% 126
{\PTglyphid{Au-020504}}
% 127
{\PTglyphid{Au-020505}}
% 128
{\PTglyphid{Au-020506}}
% 129
{\PTglyphid{Au-020507}}
% 130
{\PTglyphid{Au-020508}}
% 131
{\PTglyphid{Au-020509}}
% 132
{\PTglyphid{Au-020510}}
% 133
{\PTglyphid{Au-020511}}
% 134
{\PTglyphid{Au-020512}}
% 135
{\PTglyphid{Au-020513}}
% 136
{\PTglyphid{Au-020514}}
% 137
{\PTglyphid{Au-020515}}
% 138
{\PTglyphid{Au-020516}}
% 139
{\PTglyphid{Au-020517}}
% 140
{\PTglyphid{Au-020518}}
% 141
{\PTglyphid{Au-020519}}
% 142
{\PTglyphid{Au-020520}}
% 143
{\PTglyphid{Au-020521}}
% 144
{\PTglyphid{Au-020522}}
% 145
{\PTglyphid{Au-020523}}
% 146
{\PTglyphid{Au-020524}}
% 147
{\PTglyphid{Au-020525}}
% 148
{\PTglyphid{Au-020526}}
% 149
{\PTglyphid{Au-020527}}
% 150
{\PTglyphid{Au-020528}}
% 151
{\PTglyphid{Au-020529}}
% 152
{\PTglyphid{Au-020530}}
% 153
{\PTglyphid{Au-020531}}
% 154
{\PTglyphid{Au-020532}}
% 155
{\PTglyphid{Au-020533}}
% 156
{\PTglyphid{Au-020534}}
% 157
{\PTglyphid{Au-020535}}
% 158
{\PTglyphid{Au-020536}}
//
\endgl \xe
%%% Local Variables:
%%% mode: latex
%%% TeX-engine: luatex
%%% TeX-master: shared
%%% End:

% //
%\endgl \xe

%   *[TESTAMENTUM NOVUM. Evangelium secundum Matthaeum. Trad. polon. Stanislaus Murzynowski]: Ewangelia św. Mateusza. 4�. K. 104.
% (Arkusze A—B A—B A—Y). Druk czarno-czerwony. — Estr. XIII 26. Wierzb. 135. Arkusze
% Z aa—kk zob. poz. 4.
% W egz. B. Nar. XVI Qu. 6471 w pierwszym marginalium na lewym
% marginesie w. 4, zamiast majuskuły A wydrukowano minuskułę a.
% Pisma 1—3, 5—7, 10—12. — Cyfry 1—3, 5. — Przerywnik I, 5. —
% Inicjały 1, 3—8, 28.
% Warsz. B. U. 28.2.4.4. z notatkami J. Maleckiego. [4�


\newpage
%%%%%%%%%%%%%%%%%%%%%%%%%%%%%%%%%%%%%%%%%%%%%%%%%%%%%%%%%%%%%%%%%%%%%%%%%%%%%%
% Tab. 05 Aleksander Augezdecki pismo 2a
%%%%%%%%%%%%%%%%%%%%%%%%%%%%%%%%%%%%%%%%%%%%%%%%%%%%%%%%%%%%%%%%%%%%%%%%%%%%%%

 % Author "Paulina Buchwald-Pelcowa"
% Title "Aleksander Augezdecki 1549-1561"
% Editor "Alodia Kawecka Gryczowa"
% Series "Polonia Typographica Saeculi Sedecimi: zbiór podobizn zasobu drukarskiego tłoczni polskich XVI stulecia"
% Fascicule "VIII"
% Publisher "Zakład Narodowy imienia Ossolińskich — Wydawnictwo"
% Addres "Kraków  Wrocław Warszawa"
% Year "1972"
% Note "2 Pismo tekstowe, szwabacha M⁸¹. Stopień 20 ww. = 86—87 mm (cycero). — Tabl. 405, 406. [405]"
% Note1 "Character set table prepared by Paulina Buchwald-Pelcowa"
% Note2 "Scan (prepared by Biblioteka Uniwersytecka w Warszawie from their own copy) converted to DjVu with didjvu by Janusz S. Bień"
% URL "https://github.com/jsbien/early_fonts_inventory/"
% .

\pismoPL{Aleksander Augezdecki 2. Pisma tekstowe, szwabacha M⁸¹. Stopień 20 ww. = 86—87 mm (cycero). — Tabl. 405, 406.}

\pismoEN{Aleksander Augezdecki 2. Schwabacher text script. Typeface M⁸¹. Type size 20 lines = 86—87 mm (cicero) - Plate 405, 406.}

\medskip

\plate{406}{VIII}{1972}

Prepared by Paulina Buchwald-Pelcowa.

\bigskip

\exampleBib{VIII:25}

\bigskip
\exampleDesc{PIESNĚ Chwal Bożských. Szamotuly, Aleksander Augezdecki, [25 I 1560—] 7 VI 1561. 2°. War. A.}

\medskip
\examplePage{\textit{Karta *₄a.}}

  \bigskip
\exampleLib{Biblioteka Czartoryskich. Kraków.}

\bigskip
\exampleRef{\textit{Estreicher XIX 91. Knihopis 12860.}}

\bigskip
\exampleDig{\url{https://cyfrowe.mnk.pl/dlibra/publication/13639/}, page 10.}

\medskip

    \examplePL{Pismo 2: tekst i zestaw liter ze znakami diakrytycznymi czeskimi.}

    \medskip

    \exampleEN{Font 2. The text and the table including letters with Czech diacritical marks.}


\bigskip

    \fontID{Au-02a}{05}

    \fontstat{15}

\bigskip


% \exdisplay \bg \gla
\input {t05_glyphs.tex}
%//
%\glpismo
\input {t05_glyphids.tex}
% //
%\endgl \xe

\newpage
%%%%%%%%%%%%%%%%%%%%%%%%%%%%%%%%%%%%%%%%%%%%%%%%%%%%%%%%%%%%%%%%%%%%%%%%%%%%%%
% Tab. 06 Aleksander Augezdecki pismo 3
%%%%%%%%%%%%%%%%%%%%%%%%%%%%%%%%%%%%%%%%%%%%%%%%%%%%%%%%%%%%%%%%%%%%%%%%%%%%%%

% Note "3. Pismo komentarzowe, szwabacha M⁸¹ + M⁸⁷. Stopień 20 ww. = 67 mm (garmond). — Tabl. 407."
% Note1 "Character set table prepared by Maria Błońska"

\pismoPL{Aleksander Augezdecki 3. Pismo komentarzowe, szwabacha M⁸¹ + M⁸⁷. Stopień 20 ww. = 67 mm (garmond). — Tabl. 407.}

\pismoEN{Aleksander Augezdecki 3. Schwabacher comment script. Typeface M⁸¹ + M⁸⁷. Type size 20 lines = 67 mm (garmond). - Plate 407.}

\medskip

\plate{407}{VIII}{1972}

Prepared by Paulina Buchwald-Pelcowa.\\
The font table prepared by Paulina Buchwald-Pelcowa and Maria Błońska.

\bigskip

 \exampleBib{VIII:4$^a$}

\bigskip
\exampleDesc{[TESTAMENTUM NOVUM. Evangelium secundum Matthaeum. Trad. polon. Stanislaus Murzynowski]:
Ewangelia Św. Mateusza. Królewiec, [Aleksander Augezdecki], 1551. 4°.}

\medskip
\examplePage{\textit{Karta B₂a.}}

  \bigskip
\exampleLib{Biblioteka Czartoryskich. Kraków.}

\bigskip
\exampleRef{\textit{Estreicher XIII 26. Wierzbowski 135.}}

\bigskip
\exampleDig{\url{https://crispa.uw.edu.pl/object/files/318618/display/Default} page 18}

\medskip

    \examplePL{Pismo 3: tekst i zestaw.}

    \medskip

    \exampleEN{Font 3. The text and the font table}


\bigskip

    \fontID{Au-03}{06}

    \fontstat{112}

\bigskip


% \exdisplay \bg \gla
 \input {t06_glyphs.tex}
%//
%\glpismo
 \input {t06_glyphids.tex}
% //
%\endgl \xe

\newpage
%%%%%%%%%%%%%%%%%%%%%%%%%%%%%%%%%%%%%%%%%%%%%%%%%%%%%%%%%%%%%%%%%%%%%%%%%%%%%%
% BRAK w PT tabeli pisma 4!
% Tab. 07 Aleksander Augezdecki pismo 5
%%%%%%%%%%%%%%%%%%%%%%%%%%%%%%%%%%%%%%%%%%%%%%%%%%%%%%%%%%%%%%%%%%%%%%%%%%%%%%

% Note "5. Pismo nagłówkowe, tekstura M³⁰. Stopień I w. = 9 mm. — Tabl. 408."
% Note1 "Character set table prepared by Paulina Buchwald-Pelcowa"

\pismoPL{Aleksander Augezdecki 5. Pismo nagłówkowe, tekstura M³⁰. Stopień 1 w. = 9 mm. — Tabl. 408.}

\pismoEN{Aleksander Augezdecki 5. Display font, Textura M³⁰. Type size 1 line = 9 mm. - Plate 408.}
% https://www.adfontes.uzh.ch/en/tutorium/schriften-lesen/schriftgeschichte/gotische-minuskeln-textura-und-textualis/
\medskip

\plate{408}{VIII}{1972}

Prepared by Paulina Buchwald-Pelcowa [layout confusing, misinterpretation possible --- JSB].\\

\bigskip

\fontID{Au-05}{07}

    \fontstat{96}
    % 95???
    
% \exdisplay \bg \gla
 \input {t07_glyphs.tex}
%//
%\glpismo
 \glpismo
% 1
{\PTglyphid{Au-05_0101}}
% 2
{\PTglyphid{Au-05_0102}}
% 3
{\PTglyphid{Au-05_0103}}
% 4
{\PTglyphid{Au-05_0104}}
% 5
{\PTglyphid{Au-05_0105}}
% 6
{\PTglyphid{Au-05_0106}}
% 7
{\PTglyphid{Au-05_0107}}
% 8
{\PTglyphid{Au-05_0108}}
% 9
{\PTglyphid{Au-05_0109}}
% 10
{\PTglyphid{Au-05_0110}}
% 11
{\PTglyphid{Au-05_0111}}
% 12
{\PTglyphid{Au-05_0112}}
% 13
{\PTglyphid{Au-05_0113}}
% 14
{\PTglyphid{Au-05_0114}}
% 15
{\PTglyphid{Au-05_0115}}
% 16
{\PTglyphid{Au-05_0116}}
% 17
{\PTglyphid{Au-05_0117}}
% 18
{\PTglyphid{Au-05_0118}}
% 19
{\PTglyphid{Au-05_0119}}
% 20
{\PTglyphid{Au-05_0120}}
% 21
{\PTglyphid{Au-05_0121}}
% 22
{\PTglyphid{Au-05_0122}}
% 23
{\PTglyphid{Au-05_0123}}
% 24
{\PTglyphid{Au-05_0124}}
% 25
{\PTglyphid{Au-05_0201}}
% 26
{\PTglyphid{Au-05_0202}}
% 27
{\PTglyphid{Au-05_0203}}
% 28
{\PTglyphid{Au-05_0204}}
% 29
{\PTglyphid{Au-05_0301}}
% 30
{\PTglyphid{Au-05_0302}}
% 31
{\PTglyphid{Au-05_0303}}
% 32
{\PTglyphid{Au-05_0304}}
% 33
{\PTglyphid{Au-05_0305}}
% 34
{\PTglyphid{Au-05_0306}}
% 35
{\PTglyphid{Au-05_0307}}
% 36
{\PTglyphid{Au-05_0308}}
% 37
{\PTglyphid{Au-05_0309}}
% 38
{\PTglyphid{Au-05_0310}}
% 39
{\PTglyphid{Au-05_0311}}
% 40
{\PTglyphid{Au-05_0312}}
% 41
{\PTglyphid{Au-05_0313}}
% 42
{\PTglyphid{Au-05_0314}}
% 43
{\PTglyphid{Au-05_0315}}
% 44
{\PTglyphid{Au-05_0316}}
% 45
{\PTglyphid{Au-05_0317}}
% 46
{\PTglyphid{Au-05_0318}}
% 47
{\PTglyphid{Au-05_0319}}
% 48
{\PTglyphid{Au-05_0320}}
% 49
{\PTglyphid{Au-05_0321}}
% 50
{\PTglyphid{Au-05_0322}}
% 51
{\PTglyphid{Au-05_0323}}
% 52
{\PTglyphid{Au-05_0324}}
% 53
{\PTglyphid{Au-05_0325}}
% 54
{\PTglyphid{Au-05_0326}}
% 55
{\PTglyphid{Au-05_0327}}
% 56
{\PTglyphid{Au-05_0328}}
% 57
{\PTglyphid{Au-05_0329}}
% 58
{\PTglyphid{Au-05_0330}}
% 59
{\PTglyphid{Au-05_0331}}
% 60
{\PTglyphid{Au-05_0332}}
% 61
{\PTglyphid{Au-05_0333}}
% 62
{\PTglyphid{Au-05_0334}}
% 63
{\PTglyphid{Au-05_0335}}
% 64
{\PTglyphid{Au-05_0336}}
% 65
{\PTglyphid{Au-05_0337}}
% 66
{\PTglyphid{Au-05_0338}}
% 67
{\PTglyphid{Au-05_0339}}
% 68
{\PTglyphid{Au-05_0340}}
% 69
{\PTglyphid{Au-05_0341}}
% 70
{\PTglyphid{Au-05_0401}}
% 71
{\PTglyphid{Au-05_0402}}
% 72
{\PTglyphid{Au-05_0403}}
% 73
{\PTglyphid{Au-05_0404}}
% 74
{\PTglyphid{Au-05_0405}}
% 75
{\PTglyphid{Au-05_0406}}
% 76
{\PTglyphid{Au-05_0407}}
% 77
{\PTglyphid{Au-05_0408}}
% 78
{\PTglyphid{Au-05_0501}}
% 79
{\PTglyphid{Au-05_0502}}
% 80
{\PTglyphid{Au-05_0503}}
% 81
{\PTglyphid{Au-05_0504}}
% 82
{\PTglyphid{Au-05_0505}}
% 83
{\PTglyphid{Au-05_0506}}
% 84
{\PTglyphid{Au-05_0507}}
% 85
{\PTglyphid{Au-05_0508}}
% 86
{\PTglyphid{Au-05_0509}}
% 87
{\PTglyphid{Au-05_0510}}
% 88
{\PTglyphid{Au-05_0511}}
% 89
{\PTglyphid{Au-05_0512}}
% 90
{\PTglyphid{Au-05_0513}}
% 91
{\PTglyphid{Au-05_0514}}
% 92
{\PTglyphid{Au-05_0515}}
% 93
{\PTglyphid{Au-05_0516}}
% 94
{\PTglyphid{Au-05_0517}}
% 95
{\PTglyphid{Au-05_0518}}
% 96
{\PTglyphid{Au-05_0519}}
% 97
{\PTglyphid{Au-05_0520}}
//
\endgl \xe
%%% Local Variables:
%%% mode: latex
%%% TeX-engine: luatex
%%% TeX-master: shared
%%% End:

% //
%\endgl \xe

\newpage
%%%%%%%%%%%%%%%%%%%%%%%%%%%%%%%%%%%%%%%%%%%%%%%%%%%%%%%%%%%%%%%%%%%%%%%%%%%%%%
% Tab. 08 Aleksander Augezdecki pismo 6
%%%%%%%%%%%%%%%%%%%%%%%%%%%%%%%%%%%%%%%%%%%%%%%%%%%%%%%%%%%%%%%%%%%%%%%%%%%%%%

% Note "6. Pismo nagłówkowe, tekstura M²⁹. Stopień I w. = 7 mm. — Tabl. 408."
% Note1 "Character set table prepared by Paulina Buchwald-Pelcowa"

\pismoPL{Aleksander Augezdecki 6. Pismo nagłówkowe, tekstura M²⁹. Stopień 1 w. = 7 mm. — Tabl. 408.}

\pismoEN{Aleksander Augezdecki 6. Display font, Textura M²⁹. Type size 1 line = 7 mm. - Plate 408.}
% https://www.adfontes.uzh.ch/en/tutorium/schriften-lesen/schriftgeschichte/gotische-minuskeln-textura-und-textualis/
\medskip

\plate{408}{VIII}{1972}

Prepared by Paulina Buchwald-Pelcowa [layout confusing, misinterpretation possible --- JSB].\\

\bigskip

\fontID{Au-06}{08}

    \fontstat{94}

% \exdisplay \bg \gla
 \input {t08_glyphs.tex}
%//
%\glpismo%
 \input {t08_glyphids.tex}
% //
%\endgl \xe

 
 \newpage

%%%%%%%%%%%%%%%%%%%%%%%%%%%%%%%%%%%%%%%%%%%%%%%%%%%%%%%%%%%%%%%%%%%%%%%%%%%%%% 
% Tab. 09 Aleksander Augezdecki pismo 7
%%%%%%%%%%%%%%%%%%%%%%%%%%%%%%%%%%%%%%%%%%%%%%%%%%%%%%%%%%%%%%%%%%%%%%%%%%%%%%

% Note "7. Pismo tekstowe, antykwa Qu/(G, I). Stopień 20 ww. = 101/102 mm (tercja). — Tabl. 409."
% Note1 "Character set table prepared by Paulina Buchwald-Pelcowa"


\pismoPL{Aleksander Augezdecki 7. Pismo tekstowe, antykwa Qu/(G, I). Stopień 20 ww. = 101/102 mm (tercja). — Tabl. 409.}

\pismoEN{Aleksander Augezdecki 7. Text Roman type Qu/(G, I). Type size 20 lines = 101/102 mm (tertia). — Tabl. 409.}
% https://www.adfontes.uzh.ch/en/tutorium/schriften-lesen/schriftgeschichte/gotische-minuskeln-textura-und-textualis/
\medskip

\plate{409}{VIII}{1972}

Prepared by Paulina Buchwald-Pelcowa.

\bigskip

 \exampleBib{VIII:4$^a$}

\bigskip
\exampleDesc{[TESTAMENTUM NOVUM. Evangelium secundum Matthaeum. Trad. polon. Stanislaus Murzynowski]:
Ewangelia Św. Mateusza. Królewiec, [Aleksander Augezdecki], 1551. 4°.}

\medskip
\examplePage{\textit{Karta A₂a.}}

  \bigskip
\exampleLib{Biblioteka Czartoryskich. Kraków.}

\bigskip
\exampleRef{\textit{Estreicher XIII 26. Wierzbowski 135.}}

\bigskip
\exampleDig{\url{https://crispa.uw.edu.pl/object/files/318618/display/Default} page 10}

\medskip

    \examplePL{Pismo 7: tekst i zestaw.}

    \medskip

    \exampleEN{Font 7. The text and the font table}


\bigskip


\fontID{Au-07}{09}

\fontstat{83}

% \exdisplay \bg \gla
 \input {t09_glyphs.tex}
%//
%\glpismo%
 \input {t09_glyphids.tex}
% //
%\endgl \xe

 \newpage

%%%%%%%%%%%%%%%%%%%%%%%%%%%%%%%%%%%%%%%%%%%%%%%%%%%%%%%%%%%%%%%%%%%%%%%%%%%%%% 
% Tab. 10 Aleksander Augezdecki pismo 8
%%%%%%%%%%%%%%%%%%%%%%%%%%%%%%%%%%%%%%%%%%%%%%%%%%%%%%%%%%%%%%%%%%%%%%%%%%%%%%

% Note "8. Pismo tekstowe, antykwa Qu/(G). Stopień 20 ww. = 101 mm (tercja). — Tabl. 410."
% Note1 "Character set table prepared by Paulina Buchwald-Pelcowa"


\pismoPL{Aleksander Augezdecki 8. Pismo tekstowe, antykwa Qu/(G, I). Stopień 20 ww. = 101/102 mm (tercja). — Tabl. 410.}

\pismoEN{Aleksander Augezdecki 8. Text Roman type Qu/(G, I). Type size 20 lines = 101/102 mm (tertia). — Tabl. 410.}
% https://www.adfontes.uzh.ch/en/tutorium/schriften-lesen/schriftgeschichte/gotische-minuskeln-textura-und-textualis/
\medskip

\plate{410}{VIII}{1972}

Prepared by Paulina Buchwald-Pelcowa.

\bigskip

 \exampleBib{VIII:26}

\bigskip
\exampleDesc{[CONFESSIO AUGUSTANA. Trad. polon. Martinus Kwiatkowski]: Confessio Augustanae fidei to jest wyznanie
wiary krześciańskiej. Szamotuły, Aleksander Augezdecki, [po 13 V] 1561. 4°.}

\medskip
\examplePage{\textit{Karta B₁a.}}

  \bigskip
\exampleLib{Biblioteka Zakł. Nar. im. Ossolinskich. Wroclaw.}

\bigskip
\exampleRef{\textit{Estreicher XIV 355. Wierzbowski 1393. Bohonos Ossol. 442. Drukarze IV 31 i 85.}}

\bigskip
\exampleDig{\url{https://dbc.wroc.pl/dlibra/publication/15990/edition/14101} page 13}

\medskip

    \examplePL{Pismo 8: kolumna [? -- JSB]  i pierwszy zestaw.}

    \medskip

    \exampleEN{Font 7. The text column [?] and the first font table.}


\bigskip


\fontID{Au-08}{10}

\fontstat{104}

% \exdisplay \bg \gla
 \input {t10_glyphs.tex}
%//
%\glpismo%
 \input {t10_glyphids.tex}
% //
%\endgl \xe

 \newpage

%%%%%%%%%%%%%%%%%%%%%%%%%%%%%%%%%%%%%%%%%%%%%%%%%%%%%%%%%%%%%%%%%%%%%%%%%%%%%% 
% Tab. 11 Aleksander Augezdecki pismo 9
%%%%%%%%%%%%%%%%%%%%%%%%%%%%%%%%%%%%%%%%%%%%%%%%%%%%%%%%%%%%%%%%%%%%%%%%%%%%%%

% Note "9. Wersaliki tytułowe, antykwa. Wysokość 8 mm. — Tabl. 408."
% Note1 "Character set table prepared by Paulina Buchwald-Pelcowa"

\pismoPL{Aleksander Augezdecki 9. Wersaliki tytułowe, antykwa. Wysokość 8 mm. — Tabl. 408.}

\pismoEN{Aleksander Augezdecki 9. Roman title capitals.  Type size 1 [line] = 8 mm. - Plate 408.}
% https://www.adfontes.uzh.ch/en/tutorium/schriften-lesen/schriftgeschichte/gotische-minuskeln-textura-und-textualis/
\medskip

\plate{408}{VIII}{1972}

Prepared by Paulina Buchwald-Pelcowa [layout confusing, misinterpretation possible --- JSB].\\

\bigskip

\fontID{Au-09}{11}

\fontstat{29}

% \exdisplay \bg \gla
 \input {t11_glyphs.tex}
%//
%\glpismo%
 \input {t11_glyphids.tex}
% //
%\endgl \xe


\newpage

%%%%%%%%%%%%%%%%%%%%%%%%%%%%%%%%%%%%%%%%%%%%%%%%%%%%%%%%%%%%%%%%%%%%%%%%%%%%%% 
% Tab. 12 Aleksander Augezdecki pismo 10
%%%%%%%%%%%%%%%%%%%%%%%%%%%%%%%%%%%%%%%%%%%%%%%%%%%%%%%%%%%%%%%%%%%%%%%%%%%%%%

% Note "10.Wersaliki tytułowe, antykwa. Wysokość 6—7 mm. — Tabl. 408."
% Note1 "Character set table prepared by Paulina Buchwald-Pelcowa"

\pismoPL{Aleksander Augezdecki 10. Wersaliki tytułowe, antykwa. Wysokość 6—7 mm. — Tabl. 408.}

\pismoEN{Aleksander Augezdecki 10. Roman title capitals.  Type size [1 line] = 6--7 mm. - Plate 408.}
% https://www.adfontes.uzh.ch/en/tutorium/schriften-lesen/schriftgeschichte/gotische-minuskeln-textura-und-textualis/
\medskip

\plate{408}{VIII}{1972}

Prepared by Paulina Buchwald-Pelcowa [layout confusing, misinterpretation possible --- JSB].\\

\bigskip

\fontID{Au-10}{12}

\fontstat{15}

    
% \exdisplay \bg \gla
 \input {t12_glyphs.tex}
%//
%\glpismo%
 \input {t12_glyphids.tex}
% //
%\endgl \xe

 \newpage
 
%%%%%%%%%%%%%%%%%%%%%%%%%%%%%%%%%%%%%%%%%%%%%%%%%%%%%%%%%%%%%%%%%%%%%%%%%%%%%% 
% Tab. 13 Aleksander Augezdecki pismo 11
%%%%%%%%%%%%%%%%%%%%%%%%%%%%%%%%%%%%%%%%%%%%%%%%%%%%%%%%%%%%%%%%%%%%%%%%%%%%%%

% Note "11. Wersaliki tytułowe, antykwa. Wysokość 4,5—5 mm:— Tabl. 408."
% Note1 "Character set table prepared by Paulina Buchwald-Pelcowa"

\pismoPL{Aleksander Augezdecki 11. Wersaliki tytułowe, antykwa. Wysokość 4,5—5 mm:— Tabl. 408.}

\pismoEN{Aleksander Augezdecki 11. Roman title capitals. Type size [1 line] = 4.5--5 mm. - Plate 408.}
% https://www.adfontes.uzh.ch/en/tutorium/schriften-lesen/schriftgeschichte/gotische-minuskeln-textura-und-textualis/
\medskip

\plate{408}{VIII}{1972}

Prepared by Paulina Buchwald-Pelcowa [layout confusing, misinterpretation possible --- JSB].\\

\bigskip

\fontID{Au-11}{13}

\fontstat{32}

% \exdisplay \bg \gla
 \input {t13_glyphs.tex}
%//
%\glpismo%
 \input {t13_glyphids.tex}
% //
%\endgl \xe

\newpage
 
%%%%%%%%%%%%%%%%%%%%%%%%%%%%%%%%%%%%%%%%%%%%%%%%%%%%%%%%%%%%%%%%%%%%%%%%%%%%%% 
% Tab. 14 Aleksander Augezdecki pismo 12
%%%%%%%%%%%%%%%%%%%%%%%%%%%%%%%%%%%%%%%%%%%%%%%%%%%%%%%%%%%%%%%%%%%%%%%%%%%%%%

% Note "12. Wersaliki, antykwa. Wysokość 2—2,5 mm. — Tabl. 408."
% Note1 "Character set table prepared by Paulina Buchwald-Pelcowa"

\pismoPL{Aleksander Augezdecki 12. Wersaliki, antykwa. Wysokość 2—2,5 mm. — Tabl. 408.}

\pismoEN{Aleksander Augezdecki 12. Roman capitals. Type size [1 line] = 2--2.5 mm. - Plate 408.}
% https://www.adfontes.uzh.ch/en/tutorium/schriften-lesen/schriftgeschichte/gotische-minuskeln-textura-und-textualis/
\medskip

\plate{408}{VIII}{1972}

Prepared by Paulina Buchwald-Pelcowa [layout confusing, misinterpretation possible --- JSB].\\

\bigskip

\fontID{Au-12}{14}

\fontstat{22}

% \exdisplay \bg \gla
 \input {t14_glyphs.tex}
%//
%\glpismo%
 \input {t14_glyphids.tex}
% //
%\endgl \xe

\newpage
 

%%%%%%%%%%%%%%%%%%%%%%%%%%%%%%%%%%%%%%%%%%%%%%%%%%%%%%%%%%%%%%%%%%%%%%%%%%%%%% 
% Tab. 15 Aleksander Augezdecki pismo 13
%%%%%%%%%%%%%%%%%%%%%%%%%%%%%%%%%%%%%%%%%%%%%%%%%%%%%%%%%%%%%%%%%%%%%%%%%%%%%%

% Note "13. Pismo nagłówkowe, fraktura H. Schönspergera. Stopień 1 w. = 8 mm. — Tabl. 412."

\pismoPL{Aleksander Augezdecki 13. Pismo  nagłówkowe,
  fraktura H. Schönspergera. Wysokość 1 w. = 8 mm. — Tabl. 412.}

\pismoEN{Aleksander Augezdecki 13. H. Schönsperger's Fracture, header font. Type size 1 line = 8 mm. - Plate 412.}
% https://www.adfontes.uzh.ch/en/tutorium/schriften-lesen/schriftgeschichte/gotische-minuskeln-textura-und-textualis/
\medskip

\plate{412}{VIII}{1972}

Prepared by Paulina Buchwald-Pelcowa [layout confusing, misinterpretation possible --- JSB].\\

\bigskip

\fontID{Au-13}{15}

\fontstat{55}

% \exdisplay \bg \gla
 \input {t15_glyphs.tex}
%//
%\glpismo%
 \input {t15_glyphids.tex}
% //
%\endgl \xe


\newpage
 

%%%%%%%%%%%%%%%%%%%%%%%%%%%%%%%%%%%%%%%%%%%%%%%%%%%%%%%%%%%%%%%%%%%%%%%%%%%%%% 
% Tab. 16 Aleksander Augezdecki pismo 14
%%%%%%%%%%%%%%%%%%%%%%%%%%%%%%%%%%%%%%%%%%%%%%%%%%%%%%%%%%%%%%%%%%%%%%%%%%%%%%

% join!

% Note "14. Pismo tekstowe i nagłówkowe, fraktura H. Schönspergera. Stopień 20 ww. = 153 mm. — Tabl. 410, 411.[410]"
% Note1 "Character set table prepared by Paulina Buchwald-Pelcowa"

\pismoPL{Aleksander Augezdecki 14. Pismo tekstowe i nagłówkowe, fraktura H. Schönspergera. Stopień 20 ww. = 153 mm. — Tabl. 410, 411.}

\pismoEN{Aleksander Augezdecki 14. H. Schönsperger's Fracture, text and header font. Type size 20 lines = 153 mm. - Plate 410.}


\plate{410}{VIII}{1972}

Prepared by Paulina Buchwald-Pelcowa [layout confusing, misinterpretation possible --- JSB].\\

\bigskip

\fontID{Au-14}{16}

    \fontstat{144}
    % 108 ????

% \exdisplay \bg \gla
 \input {t16_glyphs.tex}
%//
%\glpismo%
 \input {t16_glyphids.tex}
% //
%\endgl \xe


\newpage

%%%%%%%%%%%%%%%%%%%%%%%%%%%%%%%%%%%%%%%%%%%%%%%%%%%%%%%%%%%%%%%%%%%%%%%%%%%%%% 
% Tab. 17 Aleksander Augezdecki pismo 15
%%%%%%%%%%%%%%%%%%%%%%%%%%%%%%%%%%%%%%%%%%%%%%%%%%%%%%%%%%%%%%%%%%%%%%%%%%%%%%

% Note "15. Pismo tytułowe i nagłówkowe, fraktura H. Schönspergera. Wysokość 1 w. = 11—12 mm bez przedłużek. — Tabl. 412."
% Note1 "Character set table prepared by Paulina Buchwald-Pelcowa"

\pismoPL{Aleksander Augezdecki 15. Pismo tytułowe i nagłówkowe,
  fraktura H. Schönspergera. Wysokość 1 w. = 11—12 mm bez
  przedłużek. — Tabl. 412.}

\pismoEN{Aleksander Augezdecki 15. H. Schönsperger's Fracture, title
  and header font. Type size 1 line = 11--12 mm without ascenders and
  descenders. - Plate 412.}

\plate{412}{VIII}{1972}

Prepared by Paulina Buchwald-Pelcowa [layout confusing, misinterpretation possible --- JSB].\\

\bigskip

\fontID{Au-15}{17}

\fontstat{109}

% \exdisplay \bg \gla
 \input {t17_glyphs.tex}
%//
%\glpismo%
 \input {t17_glyphids.tex}
% //
%\endgl \xe

 \newpage

%%%%%%%%%%%%%%%%%%%%%%%%%%%%%%%%%%%%%%%%%%%%%%%%%%%%%%%%%%%%%%%%%%%%%%%%%%%%%% 
% Tab. 18 Aleksander Augezdecki pismo 16
%%%%%%%%%%%%%%%%%%%%%%%%%%%%%%%%%%%%%%%%%%%%%%%%%%%%%%%%%%%%%%%%%%%%%%%%%%%%%%

% Note "16. Pismo tekstowe, szwabacha M⁸¹. Stopień 20 ww. = 88 mm. — Tabl. 413."
% Note1 "Character set table prepared by Paulina Buchwald-Pelcowa"


\pismoPL{Aleksander Augezdecki 16. Pismo tekstowe, szwabacha M⁸¹. Stopień 20 ww. = 88 mm. — Tabl. 413.}

\pismoEN{Aleksander Augezdecki 16. Text type, Schwabacher M⁸¹. Type size 20 lines = 88 mm. — Tabl. 413.}
% https://www.adfontes.uzh.ch/en/tutorium/schriften-lesen/schriftgeschichte/gotische-minuskeln-textura-und-textualis/
\medskip

\plate{413}{VIII}{1972}

Prepared by Paulina Buchwald-Pelcowa.

\bigskip

 \exampleBib{VIII:20}

\bigskip
\exampleDesc{IACOBUS KUCHLER: Epithalamion de nuptiis Andreae comitis in Gorka.
Szamotuly, Aleksander Augezdecki, 20 X 1558. 4°.}

\medskip
\examplePage{\textit{Karta A₃a.}}

  \bigskip
\exampleLib{Książnica Miejska im. Kopernika. Torun}

\bigskip
\exampleRef{\textit{Drukarze IV 29.}}

% \bigskip
% \exampleDig{\url{https://dbc.wroc.pl/dlibra/publication/15990/edition/14101} page 13}

\medskip

    \examplePL{Pismo 16: tekst i zestaw.}

    \medskip

    \exampleEN{Font 16. A text and the font table.}


\bigskip


\fontID{Au-16}{18}

\fontstat{74}

% \exdisplay \bg \gla
 \input {t18_glyphs.tex}
%//
%\glpismo%
 \input {t18_glyphids.tex}
% //
%\endgl \xe

 \newpage

%%%%%%%%%%%%%%%%%%%%%%%%%%%%%%%%%%%%%%%%%%%%%%%%%%%%%%%%%%%%%%%%%%%%%%%%%%%%%%% 
% Tab. 19 Aleksander Augezdecki pismo 17
%%%%%%%%%%%%%%%%%%%%%%%%%%%%%%%%%%%%%%%%%%%%%%%%%%%%%%%%%%%%%%%%%%%%%%%%%%%%%%

% Note "17. Pismo tytułowe i nagłówkowe, tekstura M⁶³. Wysokość I w.= 12— 13 mm. — Tabl. 412."
% Note1 "Character set table prepared by Paulina Buchwald-Pelcowa"

\pismoPL{Aleksander Augezdecki 17. Pismo tytułowe i nagłówkowe, tekstura M⁶³. Wysokość 1 w.= 12—13 mm. — Tabl. 412.}

\pismoEN{Aleksander Augezdecki 17. Title
  and header font. Type size 1 line = 12-13 mm.- Plate 412.}

\plate{412}{VIII}{1972}

Prepared by Paulina Buchwald-Pelcowa [layout confusing, misinterpretation possible --- JSB].\\

\bigskip

\fontID{Au-17}{19}

\fontstat{67}

% \exdisplay \bg \gla
 \input {t19_glyphs.tex}
%//
%\glpismo%
 \input {t19_glyphids.tex}
% //
%\endgl \xe

  \newpage

%%%%%%%%%%%%%%%%%%%%%%%%%%%%%%%%%%%%%%%%%%%%%%%%%%%%%%%%%%%%%%%%%%%%%%%%%%%%%%% 
% Tab. 20 Jan Haller pismo 1
%%%%%%%%%%%%%%%%%%%%%%%%%%%%%%%%%%%%%%%%%%%%%%%%%%%%%%%%%%%%%%%%%%%%%%%%%%%%%%

% Note "1. Pismo kanonowe. Krój M¹⁹, Stopień 10 ww. = 128/129 mm. — Tabl. 164."
% Note1 "Character set table prepared by Maria Błońska"

  \pismoPL{Jan Haller 1. Pismo kanonowe. Krój M¹⁹. Stopień 10 ww. =
    128/129 mm. — Tabl. 164. (Występuje u Hochfedera jako pismo
    8. Tabl. 25, u Unglera jako pismo 13,  Tabl. 72). }


  
  \pismoEN{Jan Haller 1. Canon [?] font. Typeface M¹⁹. Type size 10
    ww. = 128/129 mm. — Plate 164. (Used by Hochfeder as font 8, plate
    24, and by Ungler as font 13, plate 72.)}

\plate{164}{IV}{1962}

Prepared by Helena Kapełuś.\\
The font table prepared by Maria Błońska.

\bigskip

\exampleBib{IV:121}

\bigskip
\exampleDesc{MISSALE Vladislaviense. Kraków, Jan Haller, 29. XI. 1515 — 1. II. 1516. 2⁰.}

\medskip
\examplePage{\textit{Karta 1 (Canon) niepełna: wiersze 1——15.}}

  \bigskip
\exampleLib{Biblioteka Czartoryskich. Kraków.}

\bigskip
\exampleRef{\textit{Estreicher XXII. 434}}

\bigskip
\exampleDig{\url{https://cyfrowe.mnk.pl/dlibra/publication/22776/}, page 403.}

% \medskip

%     \examplePL{Pismo 2: tekst i zestaw liter ze znakami diakrytycznymi czeskimi.}

%     \medskip

%     \exampleEN{Font 2. The text and the table including letters with Czech diacritical marks.}


\bigskip


\fontID{Ha-01}{20}

\fontstat{85}

% \exdisplay \bg \gla
 \input {t20_glyphs.tex}
%//
%\glpismo%
 \input {t20_glyphids.tex}
% //
%\endgl \xe

  \newpage

%%%%%%%%%%%%%%%%%%%%%%%%%%%%%%%%%%%%%%%%%%%%%%%%%%%%%%%%%%%%%%%%%%%%%%%%%%%%%%% 
% Tab. 21 Jan Haller pismo 2
%%%%%%%%%%%%%%%%%%%%%%%%%%%%%%%%%%%%%%%%%%%%%%%%%%%%%%%%%%%%%%%%%%%%%%%%%%%%%%

% BAD:???
% python3 PT_chunks.py renumbered-lines-m21/
% OK:
% python3 PT_chunks-test.py renumbered-lines-m21/

  
% Note "2. Pismo mszalne większe. Krój M¹⁸. Stopień 10 ww. = 77 mm. — Tabl. 165."
% Note1 "Character set table prepared by Maria Błońska"

  \pismoPL{Jan Haller 2. Pismo mszalne większe. Krój M¹⁸. Stopień 10
    ww. = 77 mm. — Tabl. 165. (Występuje u Hochfedera jako pismo
    7. Tabl. 24, u Unglera jako pismo 12. Tabl. 72).}


  
\pismoEN{Jan Haller 2. Larger mass [?] font. Typeface M¹⁸. Type size 10 ww. =
    77 mm. — Plate 165. (Used by Hochfeder as font 7, plate
    24, and by Ungler as font 12, plate 72.)}

\plate{165}{IV}{1962}

Prepared by Helena Kapełuś.\\
The font table prepared by Maria Błońska.

\bigskip

\exampleBib{IV:2}

\bigskip
\exampleDesc{IOANNES LASKI: Commune Poloniae Regni privilegium. Kraków, Jan Haller, [po 27. I. 1506]. 2⁰}

\medskip
\examplePage{\textit{Karta a₃b.}}

  \bigskip
\exampleLib{Biblioteka Zakł. Nar. im. Ossolińskich. Wrocław.}

\bigskip
\exampleRef{\textit{Estreicher XXI. 79. Wierzbowski 9.}}

\bigskip
%\exampleDig{\url{https://www.wbc.poznan.pl/dlibra/publication/493453/}, page ???}???
\exampleDig{\url{https://dlibra.biblioteka.tarnow.pl/publication/196}, page 54.}

\medskip

    \examplePL{Pismo 2: wiersz 1—4, 17.}

    \medskip

    \exampleEN{Font 2: lines 1--4, 17}


\bigskip


\fontID{Ha-02}{21}

\fontstat{111}

% \exdisplay \bg \gla
 \input {t21_glyphs.tex}
%//
%\glpismo%
 \input {t21_glyphids.tex}
% //
%\endgl \xe

  \newpage

%%%%%%%%%%%%%%%%%%%%%%%%%%%%%%%%%%%%%%%%%%%%%%%%%%%%%%%%%%%%%%%%%%%%%%%%%%%%%%% 
% Tab. 22 Jan Haller pismo 3
%%%%%%%%%%%%%%%%%%%%%%%%%%%%%%%%%%%%%%%%%%%%%%%%%%%%%%%%%%%%%%%%%%%%%%%%%%%%%%

% Note "3. Pismo mszalne mniejsze. Krój M²³. Stopień 20 ww. = 136/137 mm. — Tabl. 166."
% Note1 "Character set table prepared by Maria Błońska"


  \pismoPL{Jan Haller 3. Pismo mszalne większe. Krój M²³. Stopień 10
    ww. = 136/137 mm. — Tabl. 166. (Występuje u Hochfedera jako pismo
    11. Tabl. 24, u Unglera jako pismo 11).}


  
\pismoEN{Jan Haller 3. Smaller mass [?] font. Typeface M²³. Type size 10 ww. =
    136/137 mm. — Plate 166. (Used by Hochfeder as font 11, plate
    11, and by Ungler as font 11.)}

\plate{166}{IV}{1962}

Prepared by Helena Kapełuś.\\
The font table prepared by Maria Błońska.

\bigskip

\exampleBib{IV:151}

\bigskip
\exampleDesc{AGENDA Cracoviensis. Kraków, Jan Haller, [1517]. 4⁰.}

\medskip
\examplePage{\textit{Karta M₂b.}}

  \bigskip
\exampleLib{Biblioteka Zakł. Nar. im. Ossolińskich. Wrocław.}

\bigskip
\exampleRef{\textit{Estreicher XII. 70.}}

\bigskip
%\exampleDig{\url{https://www.wbc.poznan.pl/dlibra/publication/493453/}, page ???}???
\exampleDig{\url{https://dbc.wroc.pl/dlibra/publication/11760/}, page 242.}

% \medskip

%     \examplePL{Pismo 2: wiersz 1—4, 17.}

%     \medskip

%     \exampleEN{Font 2: lines 1--4, 17}


\bigskip


\fontID{Ha-03}{22}

\fontstat{115}

% \exdisplay \bg \gla
 \input {t22_glyphs.tex}
%//
%\glpismo%
 \input {t22_glyphids.tex}
% //
%\endgl \xe

  \newpage

%%%%%%%%%%%%%%%%%%%%%%%%%%%%%%%%%%%%%%%%%%%%%%%%%%%%%%%%%%%%%%%%%%%%%%%%%%%%%%% 
% Tab. 23 Jan Haller pismo 4
%%%%%%%%%%%%%%%%%%%%%%%%%%%%%%%%%%%%%%%%%%%%%%%%%%%%%%%%%%%%%%%%%%%%%%%%%%%%%%

% Note "4. Pismo nagłówkowe. Krój M⁸³. Stopień 20 ww. =112/114 mm. — Tabl. 167."
% Note1 "Character set table prepared by Maria Błońska"


  \pismoPL{Jan Haller 4. Pismo nagłówkowe. Krój M⁸³. Stopień 20 ww. =
    112/114 mm. — Tabl. 167. (Występuje u Hochfedera jako pismo
    2. Tabl. 19).}

  
  \pismoEN{Jan Haller 4. Header font. Typeface M⁸³. Type size 20 ww. =
    112/114 mm. — Plate 167. (Used by Hochfeder as font 2, plate 19.)}

\plate{167}{IV}{1962}

Prepared by Helena Kapełuś.\\
The font table prepared by Maria Błońska.

\bigskip

\exampleBib{IV:21}

\bigskip
\exampleDesc{STATUTA conventus generalis Cracoviensis. [Kraków, Jan Haller, po 2. IIT. 1507]. 2*. Wyd. A.}

\medskip
\examplePage{\textit{Karta [1] a.}}

  \bigskip
\exampleLib{Biblioteka Narodowa. Warszawa.}

\bigskip
\exampleRef{\textit{Estreicher XXIX. 249.}}

\bigskip
%\exampleDig{\url{https://www.wbc.poznan.pl/dlibra/publication/493453/}, page ???}???
\exampleDig{\url{https://wbc.poznan.pl/dlibra/publication/493454} planned (as of \today)}

% \medskip

%     \examplePL{Pismo 2: wiersz 1—4, 17.}

%     \medskip

%     \exampleEN{Font 2: lines 1--4, 17}


\bigskip


\fontID{Ha-04}{23}

\fontstat{137}

% \exdisplay \bg \gla
 \input {t23_glyphs.tex}
%//
%\glpismo%
 \input {t23_glyphids.tex}
% //
%\endgl \xe

  \newpage

%%%%%%%%%%%%%%%%%%%%%%%%%%%%%%%%%%%%%%%%%%%%%%%%%%%%%%%%%%%%%%%%%%%%%%%%%%%%%%% 
% Tab. 24 Jan Haller pismo 5
%%%%%%%%%%%%%%%%%%%%%%%%%%%%%%%%%%%%%%%%%%%%%%%%%%%%%%%%%%%%%%%%%%%%%%%%%%%%%%

% Note "5. Pismo tekstowe. Krój M⁴⁸. Stopień 20 ww. =81/82 mm. — Tabl. 168."
% Note1 "Character set table prepared by Maria Błońska"

  \pismoPL{Jan Haller 5. Pismo tekstowe. Krój M⁴⁸. Stopień 20
    ww. =81/82 mm. — Tabl. 168. (Występuje u Hochfedera jako pismo
    9. Tabl. 26, u Unglera jako pismo 4.  Tabl. 115).}


  
\pismoEN{Jan Haller 5. Text font. Typeface M⁴⁸. Type size 20 ww. =
    81/82 mm. — Plate 168. (Used by Hochfeder as font 9, plate
    26, and by Ungler as font 4, plate 115.)}

\plate{168}{IV}{1962}

Prepared by Helena Kapełuś.\\
The font table prepared by Maria Błońska.

\bigskip

\exampleBib{IV:17}

\bigskip
\exampleDesc{17. IOANNES SACRANUS: Hlucidarius errorum ritus Ruthenici. [Kraków, Jan Haller, ok. 1507]. 4⁰.}
  

\medskip
\examplePage{\textit{Karta 7 a.}}

  \bigskip
\exampleLib{Biblioteka Narodowa. Warszawa.}

\bigskip
\exampleRef{\textit{Estreicher XVII. 13. Wierzbowski 773.}}

\bigskip
\exampleDig{\url{http://old.mbc.malopolska.pl/dlibra/docmetadata?id=83069}
page 13,\\
\url{https://www.dbc.wroc.pl/dlibra/publication/3624/edition/3513}
page 7.}
  % https://www.dbc.wroc.pl/dlibra/publication/3624/edition/3513?language=en
  % http://old.mbc.malopolska.pl/dlibra/docmetadata?id=83069&from=publication
  
  % \medskip
\bigskip

    \examplePL{Pismo 4: naglówek.}

    \medskip

    \exampleEN{Font 4: the header}


\bigskip


\fontID{Ha-05}{24}

\fontstat{135}

% \exdisplay \bg \gla
 \input {t24_glyphs.tex}
%//
%\glpismo%
 \input {t24_glyphids.tex}
% //
%\endgl \xe

  \newpage

%%%%%%%%%%%%%%%%%%%%%%%%%%%%%%%%%%%%%%%%%%%%%%%%%%%%%%%%%%%%%%%%%%%%%%%%%%%%%%% 
% Tab. 25 Jan Haller pismo 6
%%%%%%%%%%%%%%%%%%%%%%%%%%%%%%%%%%%%%%%%%%%%%%%%%%%%%%%%%%%%%%%%%%%%%%%%%%%%%%

% Note "6. Pismo komentarzowe. Krój M”. Stopień 20 ww. = 82 mm (interliniowane). — Tabl. 169."
% Note1 "Character set table prepared by Maria Błońska"


  \pismoPL{Jan Haller 6. Pismo komentarzowe. Krój M⁹¹. Stopień 20
    ww. = 82 mm (interliniowane). — Tabl. 169. (Występuje u Hochfedera
    jako pismo 4. Tabl. 21).}



  
\pismoEN{Jan Haller 6. Text font. Typeface M⁹¹. Type size 20 ww. =
    82 mm (with extra leading). — Plate 169. (Used by Hochfeder as font 4, plate
    21.)}

\plate{169}{IV}{1962}

Prepared by Helena Kapełuś.\\
The font table prepared by Maria Błońska.

\bigskip

\exampleBib{IV:23}

\bigskip \exampleDesc{THEOREMATA Autoris causarum cum
  annotationibus ac expositione Iacobi de Gostynin. Kraków, Jan
  Haller, 23. III. 1507. 4⁰.}
  

\medskip
\examplePage{\textit{Karta b₃a.}}

  \bigskip
\exampleLib{Biblioteka Zakl. Nar. im. Ossolińskich. Wrocław.}

\bigskip
\exampleRef{\textit{Estreicher XV. 85. Wierzbowski 848.}}

\bigskip
\exampleDig{\url{https://www.wbc.poznan.pl/dlibra/publication/310579/} page 25.}
  
  % \medskip
\bigskip

    \examplePL{Pismo 4: naglówek. [Pismo 6: tekst - JSB]}

    \medskip

    \exampleEN{Font 4: the header [Font 6: the text - JSB]}


\bigskip


\fontID{Ha-06}{25}

\fontstat{94}

% \exdisplay \bg \gla
 \input {t25_glyphs.tex}
%//
%\glpismo%
 \input {t25_glyphids.tex}
% //
%\endgl \xe

  

  \newpage

%%%%%%%%%%%%%%%%%%%%%%%%%%%%%%%%%%%%%%%%%%%%%%%%%%%%%%%%%%%%%%%%%%%%%%%%%%%%%%% 
% Tab. 26 Jan Haller pismo 7
%%%%%%%%%%%%%%%%%%%%%%%%%%%%%%%%%%%%%%%%%%%%%%%%%%%%%%%%%%%%%%%%%%%%%%%%%%%%%%

% Note "6. Pismo komentarzowe. Krój M”. Stopień 20 ww. = 82 mm (interliniowane). — Tabl. 169."
% Note1 "Character set table prepared by Maria Błońska"


  \pismoPL{Jan Haller 6. Pismo komentarzowe. Krój M⁹¹. Stopień 20
    ww. = 82 mm (interliniowane). — Tabl. 169. (Występuje u Hochfedera
    jako pismo 4. Tabl. 21).}



  
\pismoEN{Jan Haller 6. Text font. Typeface M⁹¹. Type size 20 ww. =
    82 mm (with extra leading). — Plate 169. (Used by Hochfeder as font 4, plate
    21.)}

\plate{170}{IV}{1962}

Prepared by Helena Kapełuś.\\
The font table prepared by Maria Błońska.

\bigskip

\exampleBib{IV:83}

\bigskip \exampleDesc{PETRUS ROSELLI: Quaestiones in libros priorum
  Analyticorum et Elenchorum Aristotelis. Ed. Ioannes de Stobnica.
  Kraków, Jan Haller, 24. V. 1511. 4⁰.}
  

\medskip
\examplePage{\textit{Karta 89}}

  \bigskip
\exampleLib{Biblioteka Narodowa. Warszawa.}


\bigskip
\exampleRef{\textit{Estreicher XXVI. 363. Wierzbowski 18.}}

% \bigskip
% \exampleDig{\url{} page} 
% podobne ale nie identyczne:
% PETRUS ROSELLI: Quaestiones in libros priorum
%   Analyticorum et Elenchorum Aristotelis. 
%   https://www.dbc.wroc.pl/dlibra/publication/15987/edition/13965

  
  % \medskip
\bigskip

    \examplePL{Pismo 4: naglówek. [Pismo 7: tekst - JSB]}

    \medskip

    \exampleEN{Font 4: the header [Font 7: the text - JSB]}


\bigskip


\fontID{Ha-07}{26}

\fontstat{129}

% \exdisplay \bg \gla
 \exdisplay \bg \gla
% 1
{\PTglyph{5}{t26_l01g01.png}}
% 2
{\PTglyph{5}{t26_l01g02.png}}
//
%%% Local Variables:
%%% mode: latex
%%% TeX-engine: luatex
%%% TeX-master: shared
%%% End:

%//
%\glpismo%
 \input {t26_glyphids.tex}
% //
%\endgl \xe

 \newpage

%%%%%%%%%%%%%%%%%%%%%%%%%%%%%%%%%%%%%%%%%%%%%%%%%%%%%%%%%%%%%%%%%%%%%%%%%%%%%%% 
% Tab. 27 Jan Haller pismo 8
%%%%%%%%%%%%%%%%%%%%%%%%%%%%%%%%%%%%%%%%%%%%%%%%%%%%%%%%%%%%%%%%%%%%%%%%%%%%%%

% Note "8. Pismo tekstowe. Krój M⁹¹. Stopień 20 ww. = 72 mm. — Tabl. 171."
% Note1 "Character set table prepared by Maria Błońska"

  \pismoPL{Jan Haller 8. Pismo tekstowe. Krój M⁹¹. Stopień 20 ww. = 72 mm. — Tabl. 171.}
  
\pismoEN{Jan Haller 8. Text font. Typeface M⁹¹. Type size 20 ww. =
    72 mm . — Plate 171.}

\plate{171}{IV}{1962}

Prepared by Helena Kapełuś.\\
The font table prepared by Maria Błońska.

\bigskip

\exampleBib{IV:201}

\bigskip


\exampleDesc{MICHAEL VRATISLAVIENSIS: Epithoma conclusionum
  theologicalium pro introductione in IV libros Petri
  Lombardi. Kraków, Jan Haller, 1521. 4⁰.}
  

\medskip
\examplePage{\textit{Karta a²a}}

  \bigskip
\exampleLib{Biblioteka Narodowa. Warszawa.}


\bigskip
\exampleRef{\textit{Estreicher XXXIII. 357. Wierzbowski 57.}}

\bigskip
\exampleDig{\url{https://www.dbc.wroc.pl/dlibra/publication/8876/} page 7} 

% https://polona.pl/preview/1f9dd490-e381-429e-9784-174169b32fb7
% Epithoma conclusionum theologicalium pro introductione in quatuor libros Sententiarum magistri Petri Lombardi [...] in [...] studio Cracouien[si] elucubratum

% https://www.dbc.wroc.pl/dlibra/publication/8876/edition/8006
% Epithoma conclusionum theologicalium pro introductione in quatuor libros sententiarum magistri Petri Lombardi [...]
% page 7


% https://polona2.pl/item/epithoma-conclusionum-theologicalium-pro-introductione-in-quatuor-libros-sententiarum,OTIxNzA2MTg/2/#info:metadata

% https://platforma.bk.pan.pl/en/search_results/1283666


\bigskip

    \examplePL{Pismo 12: wiersz 1, 4. [Pismo 8: tekst - JSB]}

    \medskip

    \exampleEN{Font 12: the lines 1, 4. [Font 8: the text - JSB]}


\bigskip


\fontID{Ha-08}{27}

\fontstat{121}

% \exdisplay \bg \gla
 \input {t27_glyphs.tex}
%//
%\glpismo%
 \input {t27_glyphids.tex}
% //
%\endgl \xe

 \newpage

%%%%%%%%%%%%%%%%%%%%%%%%%%%%%%%%%%%%%%%%%%%%%%%%%%%%%%%%%%%%%%%%%%%%%%%%%%%%%%% 
% Tab. 28 Jan Haller pismo 9
%%%%%%%%%%%%%%%%%%%%%%%%%%%%%%%%%%%%%%%%%%%%%%%%%%%%%%%%%%%%%%%%%%%%%%%%%%%%%%

% Note "9. Pismo tekstowe. Krój M⁴⁹. Stopień 20 ww. = 72 mm. — Tabl. 172."
% Note1 "Character set table prepared by Maria Błońska"

 
  \pismoPL{Jan Haller 9. Pismo tekstowe. Krój M⁴⁹. Stopień 20 ww. = 72 mm. — Tabl. 172.}
  
\pismoEN{Jan Haller 9. Text font. Typeface M⁴¹. Type size 20 ww. =
    72 mm . — Plate 172.}

\plate{172}{IV}{1962}

Prepared by Helena Kapełuś.\\
The font table prepared by Maria Błońska.

\bigskip

\exampleBib{IV:135}

\bigskip


\exampleDesc{MICHAEL VRATISLAVIENSIS: Expositio hymnorumque interpretatio. Kraków, Jan Haller, 1516. 4⁰.}
  

\medskip
\examplePage{\textit{Karta 3 b}}

  \bigskip
\exampleLib{Biblioteka Zakł. Nar. im. Ossolińskich. Wrocław.}


\bigskip
\exampleRef{\textit{Estreicher XXXIII. 358.}}

% \bigskip
% \exampleDig{\url page 7} 

\bigskip

\examplePL{Pismo 3: naglówek. — Pismo 9: tekst i pierwszy zestaw. —
  Rubryka \beta{} z pismem 9. — Cyfry 7: z pismem 9.  z pismem 13. [\ldots]}

    \medskip

    \exampleEN{Font 3: the header. — Font 9: the text and the first character set. — Rubric \beta{} with digits from font 9}


\bigskip


\fontID{Ha-09}{28}

\fontstat{119}

% \exdisplay \bg \gla
 \input {t28_glyphs.tex}
%//
%\glpismo%
 \input {t28_glyphids.tex}
% //
%\endgl \xe


 \newpage

%%%%%%%%%%%%%%%%%%%%%%%%%%%%%%%%%%%%%%%%%%%%%%%%%%%%%%%%%%%%%%%%%%%%%%%%%%%%%%% 
 % Tab. 29 Jan Haller pismo 10
%%%%%%%%%%%%%%%%%%%%%%%%%%%%%%%%%%%%%%%%%%%%%%%%%%%%%%%%%%%%%%%%%%%%%%%%%%%%%%

% Note "10. Pismo komentarzowe. Krój M⁸⁷. Stopień 20 ww. = 72 mm. — Tabl. 173."
% Note1 "Character set table prepared by Maria Błońska"


  \pismoPL{Jan Haller 10. Pismo komentarzowe. Krój M⁸⁷. Stopień 20 ww. = 72 mm. — Tabl. 173.}
  
  \pismoEN{Jan Haller 10. Text font. Typeface M⁸⁷. Type size 20 ww. =
    72 mm . — Plate 173.}

\plate{173}{IV}{1962}

Prepared by Helena Kapełuś.\\
The font table prepared by Maria Błońska.

\bigskip

\exampleBib{IV:208}

\bigskip


\exampleDesc{IACOBUS FABER STAPULENSIS: Introductio in libros De Anima Aristotelis. Kraków, Jan Haller, 1522. 4⁰}
  

\medskip
\examplePage{\textit{Karta A₂b}}

  \bigskip
\exampleLib{Biblioteka Zakł. Nar. im. Ossolińskich. Wrocław.}


\bigskip
\exampleRef{\textit{Estreicher. XIV. 150. Wierzbowski 993.}}

% \bigskip
% \exampleDig{\url page 7} 

\bigskip

\examplePL{Rubryka \eta{}: z pismem 10.}

    \medskip

    \exampleEN{Rubric \eta{} with font 10.}


\bigskip


\fontID{Ha-10}{29}

\fontstat{102}

% \exdisplay \bg \gla
 \input {t29_glyphs.tex}
%//
%\glpismo%
 \input {t29_glyphids.tex}
% //
%\endgl \xe

 
 \newpage

%%%%%%%%%%%%%%%%%%%%%%%%%%%%%%%%%%%%%%%%%%%%%%%%%%%%%%%%%%%%%%%%%%%%%%%%%%%%%%% 
 % Tab. 30 Jan Haller pismo 11
%%%%%%%%%%%%%%%%%%%%%%%%%%%%%%%%%%%%%%%%%%%%%%%%%%%%%%%%%%%%%%%%%%%%%%%%%%%%%%

% Note "11. Pismo tekstowe antykwowe. Krój Qlu (C). Stopień 20 ww. = 92 mm. — - Tabl. 174."
% Note1 "Character set table prepared by Maria Błońska"


  \pismoPL{Jan Haller 11. Pismo tekstowe antykwowe. Krój Q|u (C). Stopień 20 ww. = 92 mm. — - Tabl. 174.}
  
  \pismoEN{Jan Haller 11. Roman text font. Typeface Q|u (C). Type size 20 ww. =
    92 mm . — Plate 174.}

\plate{174}{IV}{1962}

Prepared by Helena Kapełuś.\\
The font table prepared by Maria Błońska.

\bigskip

\exampleBib{IV:163}

\bigskip


\exampleDesc{STANISLAUS ZABOROWSKI: Grammatices rudimenta. Kraków, Jan Haller, 1518. 4⁰}
  
% https://cyfrowe.mnk.pl/dlibra/publication/24392/edition/24082?language=en
\medskip
\examplePage{\textit{Karta K₃a}}

  \bigskip
\exampleLib{Biblioteka Narodowa. Warszawa.}



\bigskip
\exampleRef{\textit{Estreicher.XXXIV. 45. Wierzbowski 957.}}

 \bigskip
\exampleDig{\url{https://polona.pl/preview/dfb921b2-a937-4ee8-ad01-0724aa52c76a}
  page 121}

\bigskip

\examplePL{Pismo 11 (tekst łaciński). — Rubryki \theta, \iota. — Cyfry 6.}

    \medskip

    \exampleEN{Font 11 (Latin text). — Rubric \theta{}, \iota. — Digits 6.}


\bigskip

\exampleBib{IV:16}

\bigskip


\exampleDesc{STANISLAUS ZABOROWSKI: Orthographia seu modus recte scribendi. Kraków, Jan Haller, IV. 1518. 4⁰.}




\medskip
\examplePage{\textit{Karta [5] a}}

  \bigskip
\exampleLib{Biblioteka Narodowa. Warszawa.}



\bigskip
\exampleRef{\textit{Estreicher.XXXIV. 52. Wierzbowski 48.}}

 \bigskip
\exampleDig{\url{https://polona.pl/preview/2db2e5da-ec87-426c-8e0d-e6b09093407a}
  page 17}
% https://zpe.gov.pl/kronika/759455

\bigskip

\examplePL{Pismo 11 (tekst polski). — Rubryka \theta{}. — Inicjał 60 (V).}

    \medskip

    \exampleEN{Font 11 (Polish text). — Rubric \theta{}. — Initial 60. (V)}


\bigskip


\fontID{Ha-11}{30}

\fontstat{112}

% \exdisplay \bg \gla
 \input {t30_glyphs.tex}
%//
%\glpismo%
 \input {t30_glyphids.tex}
% //
%\endgl \xe


 
 \newpage

%%%%%%%%%%%%%%%%%%%%%%%%%%%%%%%%%%%%%%%%%%%%%%%%%%%%%%%%%%%%%%%%%%%%%%%%%%%%%%% 
 % Tab. 31 Jan Haller pismo 12 
%%%%%%%%%%%%%%%%%%%%%%%%%%%%%%%%%%%%%%%%%%%%%%%%%%%%%%%%%%%%%%%%%%%%%%%%%%%%%%%

% Note "12. Pismo mszalne. Krój M⁶⁰. Stopień 20 ww. = 176 mm. — Tabl. 175."
% Note1 "Character set table prepared by Maria Błońska"

  \pismoPL{Jan Haller 12. Pismo mszalne. Krój M⁶⁰. Stopień 20 ww. = 176 mm. — Tabl. 175.}
  
  \pismoEN{Jan Haller 12. Missal font. Typeface M⁶⁰. Type size 20 ww. =
    176 mm . — Plate 175.}

\plate{175}{IV}{1962}

Prepared by Helena Kapełuś.\\
The font table prepared by Maria Błońska.

\bigskip

\exampleBib{IV:173}

\bigskip

\exampleDesc{PSALTERIUM secundum morem Ecclesiae Cracoviensis. Kraków, Jan Haller, 1518. 2⁰.}
  

\medskip
\examplePage{\textit{Karta ostatnia, kolofon.}}

\examplePageEN{\textit{Last page, colophon}}

  
  \bigskip
\exampleLib{Biblioteka Narodowa. Warszawa.}

\bigskip
\exampleRef{\textit{Estreicher XXV. 387.}}

% \bigskip
% \exampleDig{\url{https://polona.pl/preview/dfb921b2-a937-4ee8-ad01-0724aa52c76a}
%   page 121}

\bigskip

\examplePL{Pismo 12. — Rubryki \chi{}, \lambda{}. — Cyfry 8.}

    \medskip

    \exampleEN{Font 12. — Rubric \chi{}, \lambda. — Digits 8.}


\bigskip

\bigskip


\fontID{Ha-12}{31}

\fontstat{100}

% \exdisplay \bg \gla
 \input {t31_glyphs.tex}
%//
%\glpismo%
 \input {t31_glyphids.tex}
% //
%\endgl \xe

 {
 \relsize{-1}
 
\exampleBibExtra{IV:26}

\bigskip

\exampleDesc{MICHAEL VRATISLAVIENSIS: Introductorium astronomiae Cracoviense almanach.
Kraków, Jan Haller, 29. V. 1507. 4⁰.}




\medskip
\examplePage{\textit{Karta A₆b}}

  \bigskip
\exampleLib{Biblioteka Zakl. Nar. im. Ossolińskich. Wrocław}


\bigskip
\exampleRef{\textit{Estreicher XXXIII. 356.}}

% https://polona2.pl/item/introductoriu-m-astronomie-cracouiense-elucidans-almanach,NDQzMzg2MDM/0/#info:metadata
% 1517
% https://dbc.wroc.pl/dlibra/publication/8925/edition/8046?language=pl
% 1507!!!
% https://wbc.poznan.pl/dlibra/publication/402464/edition/314800
% 1506


% \bigskip
% \exampleDig{\url{https://dbc.wroc.pl/dlibra/publication/8925}
%   page brak?!}

\bigskip

\examplePL{Znaki Zodiaku. — Pismo 5. — Rubryka \beta{} z pismem 5. — Cyfry 2.}

    \medskip

    \exampleEN{Znaki Zodiaku. — Font 5. — Rubric \beta{} with font 5. — Digits 2}
  }
  
 \newpage

%%%%%%%%%%%%%%%%%%%%%%%%%%%%%%%%%%%%%%%%%%%%%%%%%%%%%%%%%%%%%%%%%%%%%%%%%%%%%%% 
 % Tab. 32 Jan Haller pismo 13
%%%%%%%%%%%%%%%%%%%%%%%%%%%%%%%%%%%%%%%%%%%%%%%%%%%%%%%%%%%%%%%%%%%%%%%%%%%%%%%

% Note "13.  Pismo tekstowe. Krój M¹⁸. Stopień 20 ww. = 112 mm (interliniowane). — Tabl. 172."
% Note1 "Character set table prepared by Maria Błońska"

  \pismoPL{Jan Haller 13.  Pismo tekstowe. Krój M¹⁸. Stopień 20 ww. = 112 mm (interliniowane). — Tabl. 172.}
  
  \pismoEN{Jan Haller 13. Text font. Typeface M¹⁸. Type size 20 ww. =
    112 mm (with extra leading). — Plate 172.}

\plate{172}{IV}{1962}

Prepared by Helena Kapełuś.\\
The font table prepared by Maria Błońska.


\examplePL{Pismo 13: drugi zestaw. — Cyfry 9: z pismem 13.}

    \medskip

    \exampleEN{Font 13. — second character set,  — Digits 9: with font 13.}


\bigskip


\fontID{Ha-13}{32}

\fontstat{96}

% \exdisplay \bg \gla
 \input {t32_glyphs.tex}
%//
%\glpismo%
 \input {t32_glyphids.tex}
% //
%\endgl \xe

\newpage

%%%%%%%%%%%%%%%%%%%%%%%%%%%%%%%%%%%%%%%%%%%%%%%%%%%%%%%%%%%%%%%%%%%%%%%%%%%%%%% 
 % Tab. 33,Hochfeder-02_PT01_020bis.djvu,Hochfeder,02,01,020
%%%%%%%%%%%%%%%%%%%%%%%%%%%%%%%%%%%%%%%%%%%%%%%%%%%%%%%%%%%%%%%%%%%%%%%%%%%%%%%

% Fascicule "I"
% Edition "Wydanie II przejrzane i uzupełnione przez Marię Błońską"
% Publisher "Instytut Badań Literackich Polskiej Akademii Nauk — Biblioteka Narodowa"
% Addres "Warszawa"
% Year "1968"
% Note "2. Pismo nagłówkowe gotyckie. Krój M⁸⁹. Stopień 20 ww. = 111/113 mm - Tabl. 20bis. (Występuje u Hallera jako pismo 4. Tabl. 167) [20bis]"
% Note1 "Character set table prepared by Maria Błońska and Anna Wolińska"

\pismoPL{Kasper Hochfeder 2. Pismo nagłówkowe gotyckie. Krój
  M⁸⁹. Stopień 20 ww. = 111/113 mm - Tabl. 20bis. (Występuje u Hallera
  jako pismo 4. Tabl. 167)}

  
\pismoEN{Kasper Hochfeder 2. Gothic header font. Typeface M⁸⁹. Type
  size 20 lines = 111/113 mm. (Used by Haller as font 4. Plate 167)}

\plate{21bis}{I}{1968}

Prepared by Kazimierz Piekarski and  Maria Błońska.\\
The font table prepared by Maria Błońska and Anna Wolińska.

\bigskip

\exampleBib{I:8}

\bigskip \exampleDesc{HESIODUS: Georgica. Kraków, Kasper Hochfeder [nakładem Jana Hallera], 1 VII 1505. 4⁰}

\medskip
\examplePage{\textit{Karta a₀a}}

  \bigskip
\exampleLib{Biblioteka Narodowa. Warszawa.}

\bigskip
\exampleRef{\textit{Estreicher XVIII 168. Wierzbowski 836.}}
  
  % \medskip
\bigskip

    \examplePL{Pismo 2: tekst i zestaw.}
    
    \medskip

    \exampleEN{Font 2: text and font repertoire}


\bigskip

% Scharfenberg:
% https://kpbc.umk.pl/dlibra/publication/258387/edition/268632?language=pl
% Lipsk:
% https://cyfrowe.mnk.pl/dlibra/publication/25371/edition/25054?language=pl

\fontID{Ho-02}{33}

\fontstat{112}

% \exdisplay \bg \gla
 \input {t33_glyphs.tex}
%//
%\glpismo%
 \input {t33_glyphids.tex}
% //
%\endgl \xe



\newpage

%%%%%%%%%%%%%%%%%%%%%%%%%%%%%%%%%%%%%%%%%%%%%%%%%%%%%%%%%%%%%%%%%%%%%%%%%%%%%%% 
 % Tab. 34,Hochfeder-03_PT01_021.djvu,Hochfeder,03,01,021
%%%%%%%%%%%%%%%%%%%%%%%%%%%%%%%%%%%%%%%%%%%%%%%%%%%%%%%%%%%%%%%%%%%%%%%%%%%%%%%


% Note "3. Pismo tekstowe gotyckie. Krój M⁸⁸. Stopień 20 ww. = 76/77 mm - Tabl. 21. (Występuje u Hallera jako pismo 7. Tabl. 170) [21]"
% Note1 "Character set table prepared by Maria Błońska and Anna Wolińska"

\pismoPL{Kasper Hochfeder 3. Pismo tekstowe gotyckie. Krój
  M⁸⁸. Stopień 20 ww. = 76/77 mm - Tabl. 21. (Występuje u Hallera
  jako pismo 7. Tabl. 170)}

  
\pismoEN{Kasper Hochfeder 3. Gothic text font. Typeface M⁸⁸. Type
  size 20 lines = 76/77 mm. (Used by Haller as font 7. Plate 170)}

\plate{21}{I}{1968}

Prepared by Kazimierz Piekarski and  Maria Błońska.\\
The font table prepared by Maria Błońska and Anna Wolińska.

\bigskip

\exampleBib{I:8}

\bigskip \exampleDesc{IOANNES GLOGOVIENSIS: Minoris Donati
  interpretatio. Kraków, [Kasper Hochfeder] nakładem Jana Hallera,
  1503. 4⁰.}


\medskip
\examplePage{\textit{Karta a₅b}}

  \bigskip
\exampleLib{Biblioteka Jagiellońska. Kraków.}


\bigskip
\exampleRef{\textit{Estreicher XVII 175. Wierzbowski 2040.}}

  
  % \medskip
\bigskip

\examplePL{Pismo 3.  Rubryki \beta{}, \alfa{}. — Cyfry 1, 2.}

    
    \medskip

    \exampleEN{Font  3.  Rubrics \beta{}, \alfa{}. — Digits 1, 2.}


\bigskip

% Scharfenberg:
% https://kpbc.umk.pl/dlibra/publication/258387/edition/268632?language=pl
% Lipsk:
% https://cyfrowe.mnk.pl/dlibra/publication/25371/edition/25054?language=pl

\fontID{Ho-03}{34}

\fontstat{124}

% \exdisplay \bg \gla
 \input {t34_glyphs.tex}
%//
%\glpismo%
 \glpismo
% 1
{\PTglyphid{Ho-03_0101}}
% 2
{\PTglyphid{Ho-03_0102}}
% 3
{\PTglyphid{Ho-03_0103}}
% 4
{\PTglyphid{Ho-03_0104}}
% 5
{\PTglyphid{Ho-03_0105}}
% 6
{\PTglyphid{Ho-03_0106}}
% 7
{\PTglyphid{Ho-03_0107}}
% 8
{\PTglyphid{Ho-03_0108}}
% 9
{\PTglyphid{Ho-03_0109}}
% 10
{\PTglyphid{Ho-03_0110}}
% 11
{\PTglyphid{Ho-03_0111}}
% 12
{\PTglyphid{Ho-03_0112}}
% 13
{\PTglyphid{Ho-03_0113}}
% 14
{\PTglyphid{Ho-03_0114}}
% 15
{\PTglyphid{Ho-03_0115}}
% 16
{\PTglyphid{Ho-03_0116}}
% 17
{\PTglyphid{Ho-03_0117}}
% 18
{\PTglyphid{Ho-03_0118}}
% 19
{\PTglyphid{Ho-03_0201}}
% 20
{\PTglyphid{Ho-03_0202}}
% 21
{\PTglyphid{Ho-03_0203}}
% 22
{\PTglyphid{Ho-03_0204}}
% 23
{\PTglyphid{Ho-03_0205}}
% 24
{\PTglyphid{Ho-03_0206}}
% 25
{\PTglyphid{Ho-03_0207}}
% 26
{\PTglyphid{Ho-03_0208}}
% 27
{\PTglyphid{Ho-03_0209}}
% 28
{\PTglyphid{Ho-03_0210}}
% 29
{\PTglyphid{Ho-03_0211}}
% 30
{\PTglyphid{Ho-03_0212}}
% 31
{\PTglyphid{Ho-03_0213}}
% 32
{\PTglyphid{Ho-03_0214}}
% 33
{\PTglyphid{Ho-03_0215}}
% 34
{\PTglyphid{Ho-03_0216}}
% 35
{\PTglyphid{Ho-03_0217}}
% 36
{\PTglyphid{Ho-03_0218}}
% 37
{\PTglyphid{Ho-03_0219}}
% 38
{\PTglyphid{Ho-03_0220}}
% 39
{\PTglyphid{Ho-03_0221}}
% 40
{\PTglyphid{Ho-03_0222}}
% 41
{\PTglyphid{Ho-03_0223}}
% 42
{\PTglyphid{Ho-03_0224}}
% 43
{\PTglyphid{Ho-03_0225}}
% 44
{\PTglyphid{Ho-03_0226}}
% 45
{\PTglyphid{Ho-03_0227}}
% 46
{\PTglyphid{Ho-03_0228}}
% 47
{\PTglyphid{Ho-03_0229}}
% 48
{\PTglyphid{Ho-03_0230}}
% 49
{\PTglyphid{Ho-03_0231}}
% 50
{\PTglyphid{Ho-03_0232}}
% 51
{\PTglyphid{Ho-03_0233}}
% 52
{\PTglyphid{Ho-03_0234}}
% 53
{\PTglyphid{Ho-03_0235}}
% 54
{\PTglyphid{Ho-03_0236}}
% 55
{\PTglyphid{Ho-03_0301}}
% 56
{\PTglyphid{Ho-03_0302}}
% 57
{\PTglyphid{Ho-03_0303}}
% 58
{\PTglyphid{Ho-03_0304}}
% 59
{\PTglyphid{Ho-03_0305}}
% 60
{\PTglyphid{Ho-03_0306}}
% 61
{\PTglyphid{Ho-03_0307}}
% 62
{\PTglyphid{Ho-03_0308}}
% 63
{\PTglyphid{Ho-03_0309}}
% 64
{\PTglyphid{Ho-03_0310}}
% 65
{\PTglyphid{Ho-03_0311}}
% 66
{\PTglyphid{Ho-03_0312}}
% 67
{\PTglyphid{Ho-03_0313}}
% 68
{\PTglyphid{Ho-03_0314}}
% 69
{\PTglyphid{Ho-03_0315}}
% 70
{\PTglyphid{Ho-03_0316}}
% 71
{\PTglyphid{Ho-03_0317}}
% 72
{\PTglyphid{Ho-03_0318}}
% 73
{\PTglyphid{Ho-03_0319}}
% 74
{\PTglyphid{Ho-03_0320}}
% 75
{\PTglyphid{Ho-03_0321}}
% 76
{\PTglyphid{Ho-03_0322}}
% 77
{\PTglyphid{Ho-03_0323}}
% 78
{\PTglyphid{Ho-03_0324}}
% 79
{\PTglyphid{Ho-03_0325}}
% 80
{\PTglyphid{Ho-03_0326}}
% 81
{\PTglyphid{Ho-03_0327}}
% 82
{\PTglyphid{Ho-03_0328}}
% 83
{\PTglyphid{Ho-03_0329}}
% 84
{\PTglyphid{Ho-03_0330}}
% 85
{\PTglyphid{Ho-03_0331}}
% 86
{\PTglyphid{Ho-03_0332}}
% 87
{\PTglyphid{Ho-03_0333}}
% 88
{\PTglyphid{Ho-03_0334}}
% 89
{\PTglyphid{Ho-03_0401}}
% 90
{\PTglyphid{Ho-03_0402}}
% 91
{\PTglyphid{Ho-03_0403}}
% 92
{\PTglyphid{Ho-03_0404}}
% 93
{\PTglyphid{Ho-03_0405}}
% 94
{\PTglyphid{Ho-03_0406}}
% 95
{\PTglyphid{Ho-03_0407}}
% 96
{\PTglyphid{Ho-03_0408}}
% 97
{\PTglyphid{Ho-03_0409}}
% 98
{\PTglyphid{Ho-03_0410}}
% 99
{\PTglyphid{Ho-03_0411}}
% 100
{\PTglyphid{Ho-03_0412}}
% 101
{\PTglyphid{Ho-03_0413}}
% 102
{\PTglyphid{Ho-03_0414}}
% 103
{\PTglyphid{Ho-03_0415}}
% 104
{\PTglyphid{Ho-03_0416}}
% 105
{\PTglyphid{Ho-03_0417}}
% 106
{\PTglyphid{Ho-03_0418}}
% 107
{\PTglyphid{Ho-03_0419}}
% 108
{\PTglyphid{Ho-03_0420}}
% 109
{\PTglyphid{Ho-03_0421}}
% 110
{\PTglyphid{Ho-03_0422}}
% 111
{\PTglyphid{Ho-03_0423}}
% 112
{\PTglyphid{Ho-03_0424}}
% 113
{\PTglyphid{Ho-03_0425}}
% 114
{\PTglyphid{Ho-03_0426}}
% 115
{\PTglyphid{Ho-03_0427}}
% 116
{\PTglyphid{Ho-03_0428}}
% 117
{\PTglyphid{Ho-03_0429}}
% 118
{\PTglyphid{Ho-03_0430}}
% 119
{\PTglyphid{Ho-03_0431}}
% 120
{\PTglyphid{Ho-03_0501}}
% 121
{\PTglyphid{Ho-03_0502}}
% 122
{\PTglyphid{Ho-03_0503}}
//
\endgl \xe
%%% Local Variables:
%%% mode: latex
%%% TeX-engine: luatex
%%% TeX-master: shared
%%% End:

% //
%\endgl \xe

\newpage

%%%%%%%%%%%%%%%%%%%%%%%%%%%%%%%%%%%%%%%%%%%%%%%%%%%%%%%%%%%%%%%%%%%%%%%%%%%%%%%
% from meta.csv
 % Tab. 35,Hochfeder-04_PT01_022.djvu,Hochfeder,04,01,022
%%%%%%%%%%%%%%%%%%%%%%%%%%%%%%%%%%%%%%%%%%%%%%%%%%%%%%%%%%%%%%%%%%%%%%%%%%%%%%%

 % from dsed4test:

% Note "4. Pismo marginalne gotyckie. Krój M⁹⁸. Stopień 20 ww. = 57/58 mm - Tabl. 22, (Występuje u Hallera jako pismo 6. Tabl. 169) [22]"
% Note1 "Character set table prepared by Maria Błońska and Anna Wolińska"

\pismoPL{Kasper Hochfeder 4. Pismo marginalne gotyckie. Krój M⁹⁸. Stopień 20 ww. = 57/58 mm - Tabl. 22, (Występuje u Hallera jako pismo 6. Tabl. 169)}
  
\pismoEN{Kasper Hochfeder 4. Gothic margin font. Typeface M⁹⁸. Type
  size 20 lines = 57/58 mm. (Used by Haller as font 6. Plate 169)}

\plate{22}{I}{1968}

Prepared by Kazimierz Piekarski and  Maria Błońska.\\
The font table prepared by Maria Błońska and Anna Wolińska.

\bigskip

\exampleBib{I:1}

\bigskip \exampleDesc{FRANCISCUS NIGER: Compendiosa ars de
  epistolis. Kraków, Kasper Hochfeder, 10 VI 1503. 4⁰.}

\medskip
\examplePage{\textit{Karta ostatnia verso.}}

  \bigskip
\exampleLib{Biblioteka Czartoryskich Kraków.}


\bigskip
\exampleRef{\textit{Estreicher XXXI 153.}}

  
  % \medskip
\bigskip

\examplePL{Pismo 2: wiersz 1-4. — Pismo 3: wiersz 29-30. — Pismo 4: wiersz
  5-28. — Rubryka \beta{}: z pismem 3. — Rubryka \gamma{}: z pismem 4.}

\medskip

    \exampleEN{Font 2: lines 1-4. — Font 3: lines 29-30. — font 4: lines
  5-28. — Rubric \beta{}: with font 3. — Rubric \gamma{}: with font 4.}



\bigskip

% Scharfenberg:
% https://kpbc.umk.pl/dlibra/publication/258387/edition/268632?language=pl
% Lipsk:
% https://cyfrowe.mnk.pl/dlibra/publication/25371/edition/25054?language=pl

\fontID{Ho-04}{35}

\fontstat{75}

% \exdisplay \bg \gla
 \input {t35_glyphs.tex}
%//
%\glpismo%
 \glpismo
% 1
{\PTglyphid{04-01_0101}}
% 2
{\PTglyphid{04-01_0102}}
% 3
{\PTglyphid{04-01_0103}}
% 4
{\PTglyphid{04-01_0104}}
% 5
{\PTglyphid{04-01_0105}}
% 6
{\PTglyphid{04-01_0106}}
% 7
{\PTglyphid{04-01_0107}}
% 8
{\PTglyphid{04-01_0108}}
% 9
{\PTglyphid{04-01_0109}}
% 10
{\PTglyphid{04-01_0110}}
% 11
{\PTglyphid{04-01_0111}}
% 12
{\PTglyphid{04-01_0112}}
% 13
{\PTglyphid{04-01_0113}}
% 14
{\PTglyphid{04-01_0114}}
% 15
{\PTglyphid{04-01_0201}}
% 16
{\PTglyphid{04-01_0202}}
% 17
{\PTglyphid{04-01_0203}}
% 18
{\PTglyphid{04-01_0204}}
% 19
{\PTglyphid{04-01_0205}}
% 20
{\PTglyphid{04-01_0206}}
% 21
{\PTglyphid{04-01_0207}}
% 22
{\PTglyphid{04-01_0208}}
% 23
{\PTglyphid{04-01_0209}}
% 24
{\PTglyphid{04-01_0210}}
% 25
{\PTglyphid{04-01_0211}}
% 26
{\PTglyphid{04-01_0212}}
% 27
{\PTglyphid{04-01_0213}}
% 28
{\PTglyphid{04-01_0214}}
% 29
{\PTglyphid{04-01_0215}}
% 30
{\PTglyphid{04-01_0216}}
% 31
{\PTglyphid{04-01_0217}}
% 32
{\PTglyphid{04-01_0218}}
% 33
{\PTglyphid{04-01_0219}}
% 34
{\PTglyphid{04-01_0220}}
% 35
{\PTglyphid{04-01_0221}}
% 36
{\PTglyphid{04-01_0222}}
% 37
{\PTglyphid{04-01_0223}}
% 38
{\PTglyphid{04-01_0301}}
% 39
{\PTglyphid{04-01_0302}}
% 40
{\PTglyphid{04-01_0303}}
% 41
{\PTglyphid{04-01_0304}}
% 42
{\PTglyphid{04-01_0305}}
% 43
{\PTglyphid{04-01_0306}}
% 44
{\PTglyphid{04-01_0307}}
% 45
{\PTglyphid{04-01_0308}}
% 46
{\PTglyphid{04-01_0309}}
% 47
{\PTglyphid{04-01_0310}}
% 48
{\PTglyphid{04-01_0311}}
% 49
{\PTglyphid{04-01_0312}}
% 50
{\PTglyphid{04-01_0313}}
% 51
{\PTglyphid{04-01_0314}}
% 52
{\PTglyphid{04-01_0315}}
% 53
{\PTglyphid{04-01_0316}}
% 54
{\PTglyphid{04-01_0317}}
% 55
{\PTglyphid{04-01_0318}}
% 56
{\PTglyphid{04-01_0319}}
% 57
{\PTglyphid{04-01_0320}}
% 58
{\PTglyphid{04-01_0321}}
% 59
{\PTglyphid{04-01_0401}}
% 60
{\PTglyphid{04-01_0402}}
% 61
{\PTglyphid{04-01_0403}}
% 62
{\PTglyphid{04-01_0404}}
% 63
{\PTglyphid{04-01_0405}}
% 64
{\PTglyphid{04-01_0406}}
% 65
{\PTglyphid{04-01_0407}}
% 66
{\PTglyphid{04-01_0408}}
% 67
{\PTglyphid{04-01_0409}}
% 68
{\PTglyphid{04-01_0410}}
% 69
{\PTglyphid{04-01_0411}}
% 70
{\PTglyphid{04-01_0412}}
% 71
{\PTglyphid{04-01_0413}}
% 72
{\PTglyphid{04-01_0414}}
% 73
{\PTglyphid{04-01_0415}}
% 74
{\PTglyphid{04-01_0416}}
% 75
{\PTglyphid{04-01_0417}}
//
\endgl \xe
%%% Local Variables:
%%% mode: latex
%%% TeX-engine: luatex
%%% TeX-master: shared
%%% End:

% //
%\endgl \xe

\newpage

%%%%%%%%%%%%%%%%%%%%%%%%%%%%%%%%%%%%%%%%%%%%%%%%%%%%%%%%%%%%%%%%%%%%%%%%%%%%%%%
% from meta.csv
 % Tab. 36,Hochfeder-05_PT01_023.djvu,Hochfeder,05,01,023
%%%%%%%%%%%%%%%%%%%%%%%%%%%%%%%%%%%%%%%%%%%%%%%%%%%%%%%%%%%%%%%%%%%%%%%%%%%%%%%

 % from dsed4test:

% Note "5. Pismo tekstowe gotyckie. Krój M⁸⁸. Stopień 20 ww. = 88/89 mm - Tabl. 23 [23]"
% Note1 "Character set table prepared by Maria Błońska and Anna Wolińska"


\pismoPL{Kasper Hochfeder 5. Pismo tekstowe gotyckie. Krój M⁸⁸. Stopień 20 ww. = 88/89 mm - Tabl. 23}
  
\pismoEN{Kasper Hochfeder 5. Gothic text font. Typeface M⁸⁸. Type size 20 lines = 88/89 mm - Plate 23}

\plate{23}{I}{1968}

Prepared by Kazimierz Piekarski and  Maria Błońska.\\
The font table prepared by Maria Błońska and Anna Wolińska.

\bigskip

\exampleBib{I:8}

\bigskip \exampleDesc{IOANNES GLOGOVIENSIS: Minoris Donati
  interpretatio. Kraków, [Kasper Hochfeder] nakładem Jana Hallera,
  1503. 4⁰.}


\medskip
\examplePage{\textit{F₇b}}

  \bigskip
\exampleLib{Biblioteka Jagiellońska. Kraków.}


\bigskip
\exampleRef{\textit{Estreicher XVII 175. Wierzbowski 2040.}}

  
  % \medskip
\bigskip

\examplePL{[Pismo 5.] Rubryka \chi{}}

\medskip

    \exampleEN{[Font 5.] Rubric \chi{}}



\bigskip

\fontID{Ho-05}{36}

\fontstat{107}

% \exdisplay \bg \gla
 \input {t36_glyphs.tex}
%//
%\glpismo%
 \glpismo
% 1
{\PTglyphid{05-01_0101}}
% 2
{\PTglyphid{05-01_0102}}
% 3
{\PTglyphid{05-01_0103}}
% 4
{\PTglyphid{05-01_0104}}
% 5
{\PTglyphid{05-01_0105}}
% 6
{\PTglyphid{05-01_0106}}
% 7
{\PTglyphid{05-01_0107}}
% 8
{\PTglyphid{05-01_0108}}
% 9
{\PTglyphid{05-01_0109}}
% 10
{\PTglyphid{05-01_0110}}
% 11
{\PTglyphid{05-01_0111}}
% 12
{\PTglyphid{05-01_0112}}
% 13
{\PTglyphid{05-01_0113}}
% 14
{\PTglyphid{05-01_0114}}
% 15
{\PTglyphid{05-01_0115}}
% 16
{\PTglyphid{05-01_0116}}
% 17
{\PTglyphid{05-01_0117}}
% 18
{\PTglyphid{05-01_0118}}
% 19
{\PTglyphid{05-01_0119}}
% 20
{\PTglyphid{05-01_0120}}
% 21
{\PTglyphid{05-01_0121}}
% 22
{\PTglyphid{05-01_0201}}
% 23
{\PTglyphid{05-01_0202}}
% 24
{\PTglyphid{05-01_0203}}
% 25
{\PTglyphid{05-01_0204}}
% 26
{\PTglyphid{05-01_0205}}
% 27
{\PTglyphid{05-01_0206}}
% 28
{\PTglyphid{05-01_0207}}
% 29
{\PTglyphid{05-01_0208}}
% 30
{\PTglyphid{05-01_0209}}
% 31
{\PTglyphid{05-01_0210}}
% 32
{\PTglyphid{05-01_0211}}
% 33
{\PTglyphid{05-01_0212}}
% 34
{\PTglyphid{05-01_0213}}
% 35
{\PTglyphid{05-01_0214}}
% 36
{\PTglyphid{05-01_0215}}
% 37
{\PTglyphid{05-01_0216}}
% 38
{\PTglyphid{05-01_0217}}
% 39
{\PTglyphid{05-01_0218}}
% 40
{\PTglyphid{05-01_0219}}
% 41
{\PTglyphid{05-01_0220}}
% 42
{\PTglyphid{05-01_0221}}
% 43
{\PTglyphid{05-01_0222}}
% 44
{\PTglyphid{05-01_0223}}
% 45
{\PTglyphid{05-01_0224}}
% 46
{\PTglyphid{05-01_0225}}
% 47
{\PTglyphid{05-01_0226}}
% 48
{\PTglyphid{05-01_0227}}
% 49
{\PTglyphid{05-01_0228}}
% 50
{\PTglyphid{05-01_0229}}
% 51
{\PTglyphid{05-01_0230}}
% 52
{\PTglyphid{05-01_0231}}
% 53
{\PTglyphid{05-01_0232}}
% 54
{\PTglyphid{05-01_0233}}
% 55
{\PTglyphid{05-01_0234}}
% 56
{\PTglyphid{05-01_0235}}
% 57
{\PTglyphid{05-01_0236}}
% 58
{\PTglyphid{05-01_0237}}
% 59
{\PTglyphid{05-01_0238}}
% 60
{\PTglyphid{05-01_0239}}
% 61
{\PTglyphid{05-01_0301}}
% 62
{\PTglyphid{05-01_0302}}
% 63
{\PTglyphid{05-01_0303}}
% 64
{\PTglyphid{05-01_0304}}
% 65
{\PTglyphid{05-01_0305}}
% 66
{\PTglyphid{05-01_0306}}
% 67
{\PTglyphid{05-01_0307}}
% 68
{\PTglyphid{05-01_0308}}
% 69
{\PTglyphid{05-01_0309}}
% 70
{\PTglyphid{05-01_0310}}
% 71
{\PTglyphid{05-01_0311}}
% 72
{\PTglyphid{05-01_0312}}
% 73
{\PTglyphid{05-01_0313}}
% 74
{\PTglyphid{05-01_0314}}
% 75
{\PTglyphid{05-01_0315}}
% 76
{\PTglyphid{05-01_0316}}
% 77
{\PTglyphid{05-01_0317}}
% 78
{\PTglyphid{05-01_0318}}
% 79
{\PTglyphid{05-01_0319}}
% 80
{\PTglyphid{05-01_0320}}
% 81
{\PTglyphid{05-01_0321}}
% 82
{\PTglyphid{05-01_0322}}
% 83
{\PTglyphid{05-01_0323}}
% 84
{\PTglyphid{05-01_0324}}
% 85
{\PTglyphid{05-01_0325}}
% 86
{\PTglyphid{05-01_0326}}
% 87
{\PTglyphid{05-01_0327}}
% 88
{\PTglyphid{05-01_0328}}
% 89
{\PTglyphid{05-01_0329}}
% 90
{\PTglyphid{05-01_0330}}
% 91
{\PTglyphid{05-01_0331}}
% 92
{\PTglyphid{05-01_0332}}
% 93
{\PTglyphid{05-01_0333}}
% 94
{\PTglyphid{05-01_0334}}
% 95
{\PTglyphid{05-01_0335}}
% 96
{\PTglyphid{05-01_0336}}
% 97
{\PTglyphid{05-01_0401}}
% 98
{\PTglyphid{05-01_0402}}
% 99
{\PTglyphid{05-01_0403}}
% 100
{\PTglyphid{05-01_0404}}
% 101
{\PTglyphid{05-01_0405}}
% 102
{\PTglyphid{05-01_0406}}
% 103
{\PTglyphid{05-01_0407}}
% 104
{\PTglyphid{05-01_0408}}
% 105
{\PTglyphid{05-01_0409}}
% 106
{\PTglyphid{05-01_0410}}
% 107
{\PTglyphid{05-01_0411}}
//
\endgl \xe
%%% Local Variables:
%%% mode: latex
%%% TeX-engine: luatex
%%% TeX-master: shared
%%% End:

% //
%\endgl \xe

\newpage

%%%%%%%%%%%%%%%%%%%%%%%%%%%%%%%%%%%%%%%%%%%%%%%%%%%%%%%%%%%%%%%%%%%%%%%%%%%%%%%
% from meta.csv
 % Tab. 37,Hochfeder-06_PT01_024.djvu,Hochfeder,06,01,024
%%%%%%%%%%%%%%%%%%%%%%%%%%%%%%%%%%%%%%%%%%%%%%%%%%%%%%%%%%%%%%%%%%%%%%%%%%%%%%%

 % from dsed4test:

% Note "6. Pismo tekstowe gotyckie. Krój M⁸⁸. Stopień 20 ww. = 79/80 mm - Tabl. 24 [24]"
% Note1 "Character set table prepared by Maria Błońska and Anna Wolińska"



\pismoPL{Kasper Hochfeder 6. Pismo tekstowe gotyckie. Krój M⁸⁸. Stopień 20 ww. = 79/80 mm - Tabl. 24}
  
\pismoEN{Kasper Hochfeder 6. Gothic text font. Typeface M⁸⁸. Type size 20 lines = 79/80 mm - Plate 24}

\plate{24}{I}{1968}

Prepared by Kazimierz Piekarski and  Maria Błońska.\\
The font table prepared by Maria Błońska and Anna Wolińska.

\bigskip

\exampleBib{I:24}

\bigskip \exampleDesc{IOANNES[sic] DE SACROBOSCO: Algorithmus. Kraków, [Kasper Hochfeder]. 1504. 4⁰.}
IOHANNES???

\medskip
\examplePage{\textit{Bb}}

  \bigskip
\exampleLib{Biblioteka Czartoryskich. Kraków.}


\bigskip
\exampleRef{\textit{Estreicher XXVII 14. Wierzbowski 832.}}

\bigskip
\exampleDig{\url{https://cyfrowe.mnk.pl/dlibra/publication/edition/30503/content}
  page 18}

%algorismus!

  
  % \medskip
\bigskip

\examplePL{Pismo 6: tekst i drugi zestaw. — Pismo 7: nagłówki i
  pierwszy zestaw. — Rubryka \alpha{}: z pismem 6. — Cyfry 2: z pismem
  6.}

\medskip

\exampleEN{Font 6: the text and the second font table. — font 7:
  headers and the first font table. — Rubric \alpha{}: with font 6. —
  Digits 2: with font 6.}



\bigskip

\fontID{Ho-06}{37}

\fontstat{113}

% \exdisplay \bg \gla
 \input {t37_glyphs.tex}
%//
%\glpismo%
 \glpismo
% 1
{\PTglyphid{06-01_0101}}
% 2
{\PTglyphid{06-01_0102}}
% 3
{\PTglyphid{06-01_0103}}
% 4
{\PTglyphid{06-01_0104}}
% 5
{\PTglyphid{06-01_0105}}
% 6
{\PTglyphid{06-01_0106}}
% 7
{\PTglyphid{06-01_0107}}
% 8
{\PTglyphid{06-01_0108}}
% 9
{\PTglyphid{06-01_0109}}
% 10
{\PTglyphid{06-01_0110}}
% 11
{\PTglyphid{06-01_0111}}
% 12
{\PTglyphid{06-01_0112}}
% 13
{\PTglyphid{06-01_0113}}
% 14
{\PTglyphid{06-01_0114}}
% 15
{\PTglyphid{06-01_0115}}
% 16
{\PTglyphid{06-01_0116}}
% 17
{\PTglyphid{06-01_0117}}
% 18
{\PTglyphid{06-01_0118}}
% 19
{\PTglyphid{06-01_0119}}
% 20
{\PTglyphid{06-01_0120}}
% 21
{\PTglyphid{06-01_0121}}
% 22
{\PTglyphid{06-01_0122}}
% 23
{\PTglyphid{06-01_0123}}
% 24
{\PTglyphid{06-01_0124}}
% 25
{\PTglyphid{06-01_0125}}
% 26
{\PTglyphid{06-01_0201}}
% 27
{\PTglyphid{06-01_0202}}
% 28
{\PTglyphid{06-01_0203}}
% 29
{\PTglyphid{06-01_0204}}
% 30
{\PTglyphid{06-01_0205}}
% 31
{\PTglyphid{06-01_0206}}
% 32
{\PTglyphid{06-01_0207}}
% 33
{\PTglyphid{06-01_0208}}
% 34
{\PTglyphid{06-01_0209}}
% 35
{\PTglyphid{06-01_0210}}
% 36
{\PTglyphid{06-01_0211}}
% 37
{\PTglyphid{06-01_0212}}
% 38
{\PTglyphid{06-01_0213}}
% 39
{\PTglyphid{06-01_0214}}
% 40
{\PTglyphid{06-01_0215}}
% 41
{\PTglyphid{06-01_0216}}
% 42
{\PTglyphid{06-01_0217}}
% 43
{\PTglyphid{06-01_0218}}
% 44
{\PTglyphid{06-01_0219}}
% 45
{\PTglyphid{06-01_0220}}
% 46
{\PTglyphid{06-01_0221}}
% 47
{\PTglyphid{06-01_0222}}
% 48
{\PTglyphid{06-01_0223}}
% 49
{\PTglyphid{06-01_0224}}
% 50
{\PTglyphid{06-01_0225}}
% 51
{\PTglyphid{06-01_0226}}
% 52
{\PTglyphid{06-01_0227}}
% 53
{\PTglyphid{06-01_0228}}
% 54
{\PTglyphid{06-01_0229}}
% 55
{\PTglyphid{06-01_0230}}
% 56
{\PTglyphid{06-01_0231}}
% 57
{\PTglyphid{06-01_0232}}
% 58
{\PTglyphid{06-01_0233}}
% 59
{\PTglyphid{06-01_0234}}
% 60
{\PTglyphid{06-01_0235}}
% 61
{\PTglyphid{06-01_0236}}
% 62
{\PTglyphid{06-01_0237}}
% 63
{\PTglyphid{06-01_0238}}
% 64
{\PTglyphid{06-01_0239}}
% 65
{\PTglyphid{06-01_0301}}
% 66
{\PTglyphid{06-01_0302}}
% 67
{\PTglyphid{06-01_0303}}
% 68
{\PTglyphid{06-01_0304}}
% 69
{\PTglyphid{06-01_0305}}
% 70
{\PTglyphid{06-01_0306}}
% 71
{\PTglyphid{06-01_0307}}
% 72
{\PTglyphid{06-01_0308}}
% 73
{\PTglyphid{06-01_0309}}
% 74
{\PTglyphid{06-01_0310}}
% 75
{\PTglyphid{06-01_0311}}
% 76
{\PTglyphid{06-01_0312}}
% 77
{\PTglyphid{06-01_0313}}
% 78
{\PTglyphid{06-01_0314}}
% 79
{\PTglyphid{06-01_0315}}
% 80
{\PTglyphid{06-01_0316}}
% 81
{\PTglyphid{06-01_0317}}
% 82
{\PTglyphid{06-01_0318}}
% 83
{\PTglyphid{06-01_0319}}
% 84
{\PTglyphid{06-01_0320}}
% 85
{\PTglyphid{06-01_0321}}
% 86
{\PTglyphid{06-01_0322}}
% 87
{\PTglyphid{06-01_0323}}
% 88
{\PTglyphid{06-01_0324}}
% 89
{\PTglyphid{06-01_0325}}
% 90
{\PTglyphid{06-01_0326}}
% 91
{\PTglyphid{06-01_0327}}
% 92
{\PTglyphid{06-01_0328}}
% 93
{\PTglyphid{06-01_0329}}
% 94
{\PTglyphid{06-01_0330}}
% 95
{\PTglyphid{06-01_0331}}
% 96
{\PTglyphid{06-01_0332}}
% 97
{\PTglyphid{06-01_0333}}
% 98
{\PTglyphid{06-01_0334}}
% 99
{\PTglyphid{06-01_0335}}
% 100
{\PTglyphid{06-01_0336}}
% 101
{\PTglyphid{06-01_0337}}
% 102
{\PTglyphid{06-01_0401}}
% 103
{\PTglyphid{06-01_0402}}
% 104
{\PTglyphid{06-01_0403}}
% 105
{\PTglyphid{06-01_0404}}
% 106
{\PTglyphid{06-01_0405}}
% 107
{\PTglyphid{06-01_0406}}
% 108
{\PTglyphid{06-01_0407}}
% 109
{\PTglyphid{06-01_0408}}
% 110
{\PTglyphid{06-01_0409}}
% 111
{\PTglyphid{06-01_0410}}
% 112
{\PTglyphid{06-01_0411}}
% 113
{\PTglyphid{06-01_0412}}
//
\endgl \xe
%%% Local Variables:
%%% mode: latex
%%% TeX-engine: luatex
%%% TeX-master: shared
%%% End:

% //
%\endgl \xe

\newpage

%%%%%%%%%%%%%%%%%%%%%%%%%%%%%%%%%%%%%%%%%%%%%%%%%%%%%%%%%%%%%%%%%%%%%%%%%%%%%%%
% from meta.csv
 % Tab. 38,Hochfeder-07_PT01_024.djvu,Hochfeder,07,01,024
%%%%%%%%%%%%%%%%%%%%%%%%%%%%%%%%%%%%%%%%%%%%%%%%%%%%%%%%%%%%%%%%%%%%%%%%%%%%%%%

 % from dsed4test:

% Note "7. Pismo mszalne gotyckie. Krój M¹⁸. Stopień 20 ww. = 154/156 mm - Tabl. 24, 25. (Występuje u Hallera jako pismo 2. Tabl.165; u Unglera jako pismo13. Tabl. 72).  [24]"
% Note1 "Character set table prepared by Maria Błońska and Anna Wolińska"


\pismoPL{Kasper Hochfeder 7. Pismo mszalne gotyckie. Krój M¹⁸. Stopień
  20 ww. = 154/156 mm - Tabl. 24, 25. (Występuje u Hallera jako pismo
  2. Tabl.165; u Unglera jako pismo13. Tabl. 72).}
  
\pismoEN{Kasper Hochfeder 7. Gothic missal font. Typeface M¹⁸. Type size 20 lines = 154/156 mm - Plate 24, 25. (used by Haller as font 2. Plate 165; by Ungler as font 13. Plate 72).}

\plate{24}{I}{1968}

Prepared by Kazimierz Piekarski and  Maria Błońska.\\
The font table prepared by Maria Błońska and Anna Wolińska.

\bigskip

\exampleBib{I:24}

\bigskip \exampleDesc{IOANNES[sic] DE SACROBOSCO: Algorithmus. Kraków, [Kasper Hochfeder]. 1504. 4⁰.}
IOHANNES???

\medskip
\examplePage{\textit{Bb}}

  \bigskip
\exampleLib{Biblioteka Czartoryskich. Kraków.}


\bigskip
\exampleRef{\textit{Estreicher XXVII 14. Wierzbowski 832.}}

\bigskip
\exampleDig{\url{https://cyfrowe.mnk.pl/dlibra/publication/edition/30503/content}
  page 18}

%algorismus!

  
  % \medskip
\bigskip

\examplePL{Pismo 6: tekst i drugi zestaw. — Pismo 7: nagłówki i
  pierwszy zestaw. — Rubryka \alpha{}: z pismem 6. — Cyfry 2: z pismem
  6.}

\medskip

\exampleEN{Font 6: the text and the second font table. — font 7:
  headers and the first font table. — Rubric \alpha{}: with font 6. —
  Digits 2: with font 6.}



\bigskip

\fontID{Ho-07}{38}

\fontstat{109}

% \exdisplay \bg \gla
 \input {t38_glyphs.tex}
%//
%\glpismo%
 \glpismo
% 1
{\PTglyphid{06-01_0101}}
% 2
{\PTglyphid{06-01_0102}}
% 3
{\PTglyphid{06-01_0103}}
% 4
{\PTglyphid{06-01_0104}}
% 5
{\PTglyphid{06-01_0105}}
% 6
{\PTglyphid{06-01_0106}}
% 7
{\PTglyphid{06-01_0107}}
% 8
{\PTglyphid{06-01_0108}}
% 9
{\PTglyphid{06-01_0109}}
% 10
{\PTglyphid{06-01_0110}}
% 11
{\PTglyphid{06-01_0111}}
% 12
{\PTglyphid{06-01_0112}}
% 13
{\PTglyphid{06-01_0113}}
% 14
{\PTglyphid{06-01_0114}}
% 15
{\PTglyphid{06-01_0115}}
% 16
{\PTglyphid{06-01_0116}}
% 17
{\PTglyphid{06-01_0117}}
% 18
{\PTglyphid{06-01_0118}}
% 19
{\PTglyphid{06-01_0119}}
% 20
{\PTglyphid{06-01_0120}}
% 21
{\PTglyphid{06-01_0121}}
% 22
{\PTglyphid{06-01_0122}}
% 23
{\PTglyphid{06-01_0123}}
% 24
{\PTglyphid{06-01_0124}}
% 25
{\PTglyphid{06-01_0125}}
% 26
{\PTglyphid{06-01_0201}}
% 27
{\PTglyphid{06-01_0202}}
% 28
{\PTglyphid{06-01_0203}}
% 29
{\PTglyphid{06-01_0204}}
% 30
{\PTglyphid{06-01_0205}}
% 31
{\PTglyphid{06-01_0206}}
% 32
{\PTglyphid{06-01_0207}}
% 33
{\PTglyphid{06-01_0208}}
% 34
{\PTglyphid{06-01_0209}}
% 35
{\PTglyphid{06-01_0210}}
% 36
{\PTglyphid{06-01_0211}}
% 37
{\PTglyphid{06-01_0212}}
% 38
{\PTglyphid{06-01_0213}}
% 39
{\PTglyphid{06-01_0214}}
% 40
{\PTglyphid{06-01_0215}}
% 41
{\PTglyphid{06-01_0216}}
% 42
{\PTglyphid{06-01_0217}}
% 43
{\PTglyphid{06-01_0218}}
% 44
{\PTglyphid{06-01_0219}}
% 45
{\PTglyphid{06-01_0220}}
% 46
{\PTglyphid{06-01_0221}}
% 47
{\PTglyphid{06-01_0222}}
% 48
{\PTglyphid{06-01_0223}}
% 49
{\PTglyphid{06-01_0224}}
% 50
{\PTglyphid{06-01_0225}}
% 51
{\PTglyphid{06-01_0226}}
% 52
{\PTglyphid{06-01_0227}}
% 53
{\PTglyphid{06-01_0228}}
% 54
{\PTglyphid{06-01_0229}}
% 55
{\PTglyphid{06-01_0230}}
% 56
{\PTglyphid{06-01_0231}}
% 57
{\PTglyphid{06-01_0232}}
% 58
{\PTglyphid{06-01_0233}}
% 59
{\PTglyphid{06-01_0234}}
% 60
{\PTglyphid{06-01_0235}}
% 61
{\PTglyphid{06-01_0236}}
% 62
{\PTglyphid{06-01_0237}}
% 63
{\PTglyphid{06-01_0238}}
% 64
{\PTglyphid{06-01_0239}}
% 65
{\PTglyphid{06-01_0240}}
% 66
{\PTglyphid{06-01_0241}}
% 67
{\PTglyphid{06-01_0242}}
% 68
{\PTglyphid{06-01_0243}}
% 69
{\PTglyphid{06-01_0244}}
% 70
{\PTglyphid{06-01_0301}}
% 71
{\PTglyphid{06-01_0302}}
% 72
{\PTglyphid{06-01_0303}}
% 73
{\PTglyphid{06-01_0304}}
% 74
{\PTglyphid{06-01_0305}}
% 75
{\PTglyphid{06-01_0306}}
% 76
{\PTglyphid{06-01_0307}}
% 77
{\PTglyphid{06-01_0308}}
% 78
{\PTglyphid{06-01_0309}}
% 79
{\PTglyphid{06-01_0310}}
% 80
{\PTglyphid{06-01_0311}}
% 81
{\PTglyphid{06-01_0312}}
% 82
{\PTglyphid{06-01_0313}}
% 83
{\PTglyphid{06-01_0314}}
% 84
{\PTglyphid{06-01_0315}}
% 85
{\PTglyphid{06-01_0316}}
% 86
{\PTglyphid{06-01_0317}}
% 87
{\PTglyphid{06-01_0318}}
% 88
{\PTglyphid{06-01_0319}}
% 89
{\PTglyphid{06-01_0320}}
% 90
{\PTglyphid{06-01_0321}}
% 91
{\PTglyphid{06-01_0322}}
% 92
{\PTglyphid{06-01_0323}}
% 93
{\PTglyphid{06-01_0324}}
% 94
{\PTglyphid{06-01_0325}}
% 95
{\PTglyphid{06-01_0326}}
% 96
{\PTglyphid{06-01_0327}}
% 97
{\PTglyphid{06-01_0328}}
% 98
{\PTglyphid{06-01_0329}}
% 99
{\PTglyphid{06-01_0330}}
% 100
{\PTglyphid{06-01_0331}}
% 101
{\PTglyphid{06-01_0332}}
% 102
{\PTglyphid{06-01_0333}}
% 103
{\PTglyphid{06-01_0334}}
% 104
{\PTglyphid{06-01_0335}}
% 105
{\PTglyphid{06-01_0336}}
% 106
{\PTglyphid{06-01_0337}}
% 107
{\PTglyphid{06-01_0338}}
% 108
{\PTglyphid{06-01_0339}}
% 109
{\PTglyphid{06-01_0340}}
//
\endgl \xe
%%% Local Variables:
%%% mode: latex
%%% TeX-engine: luatex
%%% TeX-master: shared
%%% End:

% //
%\endgl \xe

%%%%%%%%%%%%%%%%%%%%%%%%%%%%%%%%%%%%%%%%%%%%%%%%%%%%%%%%%%%%%%%%%%%%%%%%%%%%%%%

 split poprawione
 
%%%%%%%%%%%%%%%%%%%%%%%%%%%%%%%%%%%%%%%%%%%%%%%%%%%%%%%%%%%%%%%%%%%%%%%%%%%%%%
 
 \newpage
 
%%%%%%%%%%%%%%%%%%%%%%%%%%%%%%%%%%%%%%%%%%%%%%%%%%%%%%%%%%%%%%%%%%%%%%%%%%%%%%%
% from meta.csv
 % Tab. 39,Hochfeder,09,01,027
%%%%%%%%%%%%%%%%%%%%%%%%%%%%%%%%%%%%%%%%%%%%%%%%%%%%%%%%%%%%%%%%%%%%%%%%%%%%%%%

 % from dsed4test:

% Note "9. Pismo tekstowe gotyckie. Krój M⁴⁸. Stopień 20 ww. = 81/82 mm - Tabl. 27. (Występuje u Hallera jako pismo 5. Tabl. 168; u Unglera jako pismo4. Tabl. 115).  [27]"
% Note1 "Character set table prepared by Maria Błońska and Anna Wolińska"

 \pismoPL{Kasper Hochfeder 9. Pismo tekstowe gotyckie. Krój
   M⁴⁸. Stopień 20 ww. = 81/82 mm - Tabl. 27. (Występuje u Hallera
   jako pismo 5. Tabl. 168; u Unglera jako pismo 4. Tabl. 115).}
  
 \pismoEN{Kasper Hochfeder 9. Gothic text font. Typeface M⁴⁸. Type
   size 20 lines = 81/82 mm - Plate 27. (used by Haller as font
   5. Plate 168; by Ungler as font 4. Plate 115).}

\plate{27}{I}{1968}

Prepared by Kazimierz Piekarski and  Maria Błońska.\\
The font table prepared by Maria Błońska and Anna Wolińska.

\bigskip

\exampleBib{I:15}

\bigskip \exampleDesc{IOANNES DE DOBCZYCE: Opusculum de arte memorativa. Kraków, [Kasper Hochfeder], 13 IX 1504. 4⁰}

\medskip
\examplePage{\textit{[Karta] c₃b}}

  \bigskip
\exampleLib{Biblioteka Jagiellońska. Kraków.}


\bigskip
\exampleRef{\textit{Estreicher XV 255, Wierzbowski 831.}}

\bigskip
\exampleDig{\url{ https://polona.pl/preview/aa8a86f1-23fe-4ef8-b9f4-b36499c4a351}
  page 263,\\ \url{https://www.wbc.poznan.pl/dlibra/publication/310297}
  page 42}

%algorismus!

  
  % \medskip
\bigskip

\examplePL{[Pismo 9] Rubryka \epsilon{}}

\medskip

\exampleEN{[Font 9] Rubric \epsilon{}}



\bigskip

\fontID{Ho-09}{39}

\fontstat{111}

% \exdisplay \bg \gla
 \input {t39_glyphs.tex}
%//
%\glpismo%
 \glpismo
% 1
{\PTglyphid{09-01_0101}}
% 2
{\PTglyphid{09-01_0102}}
% 3
{\PTglyphid{09-01_0103}}
% 4
{\PTglyphid{09-01_0104}}
% 5
{\PTglyphid{09-01_0105}}
% 6
{\PTglyphid{09-01_0106}}
% 7
{\PTglyphid{09-01_0107}}
% 8
{\PTglyphid{09-01_0108}}
% 9
{\PTglyphid{09-01_0109}}
% 10
{\PTglyphid{09-01_0110}}
% 11
{\PTglyphid{09-01_0111}}
% 12
{\PTglyphid{09-01_0112}}
% 13
{\PTglyphid{09-01_0113}}
% 14
{\PTglyphid{09-01_0114}}
% 15
{\PTglyphid{09-01_0115}}
% 16
{\PTglyphid{09-01_0116}}
% 17
{\PTglyphid{09-01_0117}}
% 18
{\PTglyphid{09-01_0118}}
% 19
{\PTglyphid{09-01_0119}}
% 20
{\PTglyphid{09-01_0120}}
% 21
{\PTglyphid{09-01_0121}}
% 22
{\PTglyphid{09-01_0201}}
% 23
{\PTglyphid{09-01_0202}}
% 24
{\PTglyphid{09-01_0203}}
% 25
{\PTglyphid{09-01_0204}}
% 26
{\PTglyphid{09-01_0205}}
% 27
{\PTglyphid{09-01_0206}}
% 28
{\PTglyphid{09-01_0207}}
% 29
{\PTglyphid{09-01_0208}}
% 30
{\PTglyphid{09-01_0209}}
% 31
{\PTglyphid{09-01_0210}}
% 32
{\PTglyphid{09-01_0211}}
% 33
{\PTglyphid{09-01_0212}}
% 34
{\PTglyphid{09-01_0213}}
% 35
{\PTglyphid{09-01_0214}}
% 36
{\PTglyphid{09-01_0215}}
% 37
{\PTglyphid{09-01_0216}}
% 38
{\PTglyphid{09-01_0217}}
% 39
{\PTglyphid{09-01_0218}}
% 40
{\PTglyphid{09-01_0219}}
% 41
{\PTglyphid{09-01_0220}}
% 42
{\PTglyphid{09-01_0221}}
% 43
{\PTglyphid{09-01_0222}}
% 44
{\PTglyphid{09-01_0223}}
% 45
{\PTglyphid{09-01_0224}}
% 46
{\PTglyphid{09-01_0225}}
% 47
{\PTglyphid{09-01_0226}}
% 48
{\PTglyphid{09-01_0227}}
% 49
{\PTglyphid{09-01_0228}}
% 50
{\PTglyphid{09-01_0229}}
% 51
{\PTglyphid{09-01_0230}}
% 52
{\PTglyphid{09-01_0231}}
% 53
{\PTglyphid{09-01_0232}}
% 54
{\PTglyphid{09-01_0233}}
% 55
{\PTglyphid{09-01_0301}}
% 56
{\PTglyphid{09-01_0302}}
% 57
{\PTglyphid{09-01_0303}}
% 58
{\PTglyphid{09-01_0304}}
% 59
{\PTglyphid{09-01_0305}}
% 60
{\PTglyphid{09-01_0306}}
% 61
{\PTglyphid{09-01_0307}}
% 62
{\PTglyphid{09-01_0308}}
% 63
{\PTglyphid{09-01_0309}}
% 64
{\PTglyphid{09-01_0310}}
% 65
{\PTglyphid{09-01_0311}}
% 66
{\PTglyphid{09-01_0312}}
% 67
{\PTglyphid{09-01_0313}}
% 68
{\PTglyphid{09-01_0314}}
% 69
{\PTglyphid{09-01_0315}}
% 70
{\PTglyphid{09-01_0316}}
% 71
{\PTglyphid{09-01_0317}}
% 72
{\PTglyphid{09-01_0318}}
% 73
{\PTglyphid{09-01_0319}}
% 74
{\PTglyphid{09-01_0320}}
% 75
{\PTglyphid{09-01_0321}}
% 76
{\PTglyphid{09-01_0322}}
% 77
{\PTglyphid{09-01_0323}}
% 78
{\PTglyphid{09-01_0324}}
% 79
{\PTglyphid{09-01_0325}}
% 80
{\PTglyphid{09-01_0326}}
% 81
{\PTglyphid{09-01_0327}}
% 82
{\PTglyphid{09-01_0328}}
% 83
{\PTglyphid{09-01_0329}}
% 84
{\PTglyphid{09-01_0330}}
% 85
{\PTglyphid{09-01_0401}}
% 86
{\PTglyphid{09-01_0402}}
% 87
{\PTglyphid{09-01_0403}}
% 88
{\PTglyphid{09-01_0404}}
% 89
{\PTglyphid{09-01_0405}}
% 90
{\PTglyphid{09-01_0406}}
% 91
{\PTglyphid{09-01_0407}}
% 92
{\PTglyphid{09-01_0408}}
% 93
{\PTglyphid{09-01_0409}}
% 94
{\PTglyphid{09-01_0410}}
% 95
{\PTglyphid{09-01_0411}}
% 96
{\PTglyphid{09-01_0412}}
% 97
{\PTglyphid{09-01_0413}}
% 98
{\PTglyphid{09-01_0414}}
% 99
{\PTglyphid{09-01_0415}}
% 100
{\PTglyphid{09-01_0416}}
% 101
{\PTglyphid{09-01_0417}}
% 102
{\PTglyphid{09-01_0418}}
% 103
{\PTglyphid{09-01_0419}}
% 104
{\PTglyphid{09-01_0420}}
% 105
{\PTglyphid{09-01_0421}}
% 106
{\PTglyphid{09-01_0422}}
% 107
{\PTglyphid{09-01_0423}}
% 108
{\PTglyphid{09-01_0424}}
% 109
{\PTglyphid{09-01_0425}}
% 110
{\PTglyphid{09-01_0426}}
% 111
{\PTglyphid{09-01_0427}}
% 112
{\PTglyphid{09-01_0428}}
//
\endgl \xe
%%% Local Variables:
%%% mode: latex
%%% TeX-engine: luatex
%%% TeX-master: shared
%%% End:

% //
%\endgl \xe

 \newpage
 
%%%%%%%%%%%%%%%%%%%%%%%%%%%%%%%%%%%%%%%%%%%%%%%%%%%%%%%%%%%%%%%%%%%%%%%%%%%%%%%
% from meta.csv
 % Tab. 40,Hochfeder-10_PT01_028.djvu,Hochfeder,10,01,028
%%%%%%%%%%%%%%%%%%%%%%%%%%%%%%%%%%%%%%%%%%%%%%%%%%%%%%%%%%%%%%%%%%%%%%%%%%%%%%%

 % from dsed4test:

 % Note "10. Pismo komentarzowe gotyckie. Krój M¹⁶. Stopień 20 ww. = 69 mm - Tabl. 28 [28]"
% Note1 "Character set table prepared by Maria Błońska and Anna Wolińska"

 \pismoPL{Kasper Hochfeder 10. Pismo komentarzowe gotyckie. Krój M¹⁶. Stopień 20 ww. = 69 mm - Tabl. 28.}
  
 \pismoEN{Kasper Hochfeder 10. Gothic comment font. Typeface M¹⁶. Type
   size 20 lines = 69 mm - Plate 28.}

\plate{28}{I}{1968}

Prepared by Kazimierz Piekarski and  Maria Błońska.\\
The font table prepared by Maria Błońska and Anna Wolińska.

\bigskip

\exampleBib{I:16}

\bigskip \exampleDesc{ IOANNES GLOGOVIENSIS: Exercitium veteris artis. Kraków, [Kasper Hochfeder]. 16 X 1504. 4⁰,
War. B: nakładem Jana Hallera.}


\medskip
\examplePage{\textit{[Karta] L₇b}}

  \bigskip
\exampleLib{Biblioteka Jagiellońska. Kraków.}


\bigskip
\exampleRef{\textit{Estreicher XVII 173, 160. Wierzbowski 7.}}

\bigskip
\exampleDig{\url{https://www.wbc.poznan.pl/dlibra/publication/559203/}
  page 176}

%algorismus!

  
  % \medskip
\bigskip

\examplePL{Pismo 10: tekst i pierwszy zestaw. — Pismo 11: nagłówki i
  drugi zestaw, — Rubryka \delta{}: z pismem 11. — Cyfry 4: z pismem
  11}

\medskip

\exampleEN{Font 10: the text and the first font table. — Font 11: the
  headers and the second font table. — Rubric \delta{} with font 11. —
  Digits 4: with font 11}



\bigskip

\fontID{Ho-10}{40}

\fontstat{107}

% \exdisplay \bg \gla
 \input {t40_glyphs.tex}
%//
%\glpismo%
 \input {t40_glyphids.tex}
% //
%\endgl \xe


 \newpage
 
%%%%%%%%%%%%%%%%%%%%%%%%%%%%%%%%%%%%%%%%%%%%%%%%%%%%%%%%%%%%%%%%%%%%%%%%%%%%%%%
% from meta.csv
 % Tab. 41,Hochfeder-11_PT01_028.djvu,Hochfeder,11,01,028
%%%%%%%%%%%%%%%%%%%%%%%%%%%%%%%%%%%%%%%%%%%%%%%%%%%%%%%%%%%%%%%%%%%%%%%%%%%%%%%

 % from dsed4test:

 % Note "11. Pismo mszalne gotyckie. Krój M²³. Stopień 20 ww. = 154/156 mm - Tabl. 25, 28. (Występuje u Hallera jako pismo 3. Tabl. 166; u Unglera jako pismo 11) [28]"
% Note1 "Character set table prepared by Maria Błońska and Anna Wolińska"

 \pismoPL{Kasper Hochfeder 11. Pismo mszalne gotyckie. Krój
   M²³. Stopień 20 ww. = 154/156 mm - Tabl. 25, 28. (Występuje u
   Hallera jako pismo 3. Tabl. 166; u Unglera jako pismo 11)}
  
 \pismoEN{Kasper Hochfeder 11. Gothic missal font. Typeface M²³. Type
   size 20 lines = 154/156 mm - Plate 28.}

\plate{28}{I}{1968}

Prepared by Kazimierz Piekarski and  Maria Błońska.\\
The font table prepared by Maria Błońska and Anna Wolińska.

\bigskip

\exampleBib{I:16}

\bigskip \exampleDesc{ IOANNES GLOGOVIENSIS: Exercitium veteris artis. Kraków, [Kasper Hochfeder]. 16 X 1504. 4⁰,
War. B: nakładem Jana Hallera.}


\medskip
\examplePage{\textit{[Karta] L₇b}}

  \bigskip
\exampleLib{Biblioteka Jagiellońska. Kraków.}


\bigskip
\exampleRef{\textit{Estreicher XVII 173, 160. Wierzbowski 7.}}

\bigskip
\exampleDig{\url{https://www.wbc.poznan.pl/dlibra/publication/559203/}
  page 176}

%algorismus!

  
  % \medskip
\bigskip

\examplePL{Pismo 10: tekst i pierwszy zestaw. — Pismo 11: nagłówki i
  drugi zestaw, — Rubryka \delta{}: z pismem 11. — Cyfry 4: z pismem
  11}

\medskip

\exampleEN{Font 10: the text and the first font table. — Font 11: the
  headers and the second font table. — Rubric \delta{} with font 11. —
  Digits 4: with font 11}



\bigskip

\fontID{Ho-11}{41}

\fontstat{95}

% \exdisplay \bg \gla
 \input {t41_glyphs.tex}
%//
%\glpismo%
 \input {t41_glyphids.tex}
% //
%\endgl \xe

  \newpage
 
%%%%%%%%%%%%%%%%%%%%%%%%%%%%%%%%%%%%%%%%%%%%%%%%%%%%%%%%%%%%%%%%%%%%%%%%%%%%%%%
% from meta.csv
 % Tab. 42,Ungler1-01_PT03_112.djvu,Ungler1,01,03,112
%%%%%%%%%%%%%%%%%%%%%%%%%%%%%%%%%%%%%%%%%%%%%%%%%%%%%%%%%%%%%%%%%%%%%%%%%%%%%%%

 % from dsed4test:

% Fascicule "III"
% Publisher "Zakład Narodowy imienia Ossolińskich — Wydawnictwo"
% Addres "Kraków  Wrocław Warszawa"
% Year "1959"
% Note "1. Pismo tekstowe gotyckie. Krój M⁹¹. Stopień 20 ww. == 76/77 mm. — Tabl. 112."


 \pismoPL{Florian Ungler 1. Pismo tekstowe gotyckie. Krój M⁹¹. Stopień 20 ww. == 76/77 mm. — Tabl. 112.}
  
 \pismoEN{Florian Ungler 1. Gothic text font. Typeface M⁹¹. Type size 20 ww. == 76/77 mm. — Plate 112.}

\plate{112}{III}{1959}

The plate prepared by Henryk Bułhak.\\
The font table prepared by Maria Błońska.

\bigskip

\exampleBib{III:3}

\bigskip \exampleDesc{IOANNES DE SACROBOSCO: Algorithmus. Kraków, Florian Ungler, 31. I. 1511. 4⁰} 


\medskip
\examplePage{\textit{Karta b₅b}}

  \bigskip
\exampleLib{Biblioteka Zakł. Nar. im. Ossolińskich. Wrocław.}


\bigskip
\exampleRef{\textit{Estreicher XXVI. 15. Piekarski U. 2.}}

  \bigskip
  \exampleDig{%1523 \url{https://polona.pl/preview/ff92eeb7-0467-4359-bdd8-4be4de13358c}    page ???,
    % 1522?: \url{https://www.wbc.poznan.pl/dlibra/publication/314379/} page    ???,
    \url{https://dbc.wroc.pl/dlibra/publication/3586/} page 24}

%algorismus!

  
  % \medskip
\bigskip

\examplePL{[Pismo 1] Rubryka \alpha{}. — Cyfry 1. — Inicjal 1 (P).}

\medskip

\exampleEN{[Font 1] Rubric \alpha{}  —
  Digits 1. — Initial 1 (P)}


\bigskip

\fontID{Un1-1}{42}

\fontstat{94}

% \exdisplay \bg \gla
 \input {t42_glyphs.tex}
%//
%\glpismo%
 \input {t42_glyphids.tex}
% //
%\endgl \xe

 \newpage
 
%%%%%%%%%%%%%%%%%%%%%%%%%%%%%%%%%%%%%%%%%%%%%%%%%%%%%%%%%%%%%%%%%%%%%%%%%%%%%%%
% from meta.csv
 % Tab. 43,Ungler1-02_PT03_113.djvu,Ungler1,02,03,113
%%%%%%%%%%%%%%%%%%%%%%%%%%%%%%%%%%%%%%%%%%%%%%%%%%%%%%%%%%%%%%%%%%%%%%%%%%%%%%%

 % from dsed4test:

% Note "2. Pismo tekstowe gotyckie. Krój M⁹¹. Stopień 20 ww. == 76/77 mm. — Tabl. 113."
% Note1 "Character set table prepared by Maria Błońska"

\pismoPL{Florian Ungler 2. Pismo tekstowe gotyckie. Krój M⁹¹. Stopień 20 ww. == 76/77 mm. — Tabl. 113.}
  
 \pismoEN{Florian Ungler 2. Gothic text font. Typeface M⁹¹. Type size 20 ww. == 76/77 mm. — Plate 113.}

\plate{113}{III}{1959}

The plate prepared by Henryk Bułhak.\\
The font table prepared by  Henryk Bułhak and Maria Błońska.

\bigskip

\exampleBib{III:9}

\bigskip \exampleDesc{STANISLAUS ZABOROWSKI: Tractatus contra malos divites et usurarios.
Kraków, Florian Ungler, 10. IV. 1512. 4⁰.}

\medskip
\examplePage{\textit{Karta A₃a}}

  \bigskip
  \exampleLib{Biblioteka Narodowa. Warszawa.}
  

\bigskip \exampleRef{\textit{Estreicher XVI. 242, XXXIV. 55.
    Wierzbowski 882. Piekarski U. 8.}}

  \bigskip
  \exampleDig{\url{https://cyfrowe.mnk.pl/dlibra/publication/863/} page 9}
%\url{https://dbc.wroc.pl/dlibra/publication/11691/} page ???. nie to wydanie???

  % \medskip
\bigskip

\examplePL{[Pismo 2] Pismo 9: naglówek. — [JSB niejasne:] Cyfry 3:
  pierwszy zestaw. — Cyfry 4: drugi zestaw.}

\medskip

\exampleEN{[Font 2] Font 9: the header. — [JSB unclear:] Digits 3:
  first set. — Digits 4: second set.}



\bigskip

\fontID{U1-02}{43}

\fontstat{120}

% \exdisplay \bg \gla
 \input {t43_glyphs.tex}
%//
%\glpismo%
 \input {t43_glyphids.tex}
% //
%\endgl \xe


 \newpage
 
%%%%%%%%%%%%%%%%%%%%%%%%%%%%%%%%%%%%%%%%%%%%%%%%%%%%%%%%%%%%%%%%%%%%%%%%%%%%%%%
% from meta.csv
 % Tab. 44,Ungler1-03_PT03_114.djvu,Ungler1,03,03,114
%%%%%%%%%%%%%%%%%%%%%%%%%%%%%%%%%%%%%%%%%%%%%%%%%%%%%%%%%%%%%%%%%%%%%%%%%%%%%%%

 % from dsed4test:

 % Note "3. Pismo tekstowe gotyckie. Krój M⁹¹. Stopień 20 ww. == 72/73 mm. — Tabl. 114."
% Note1 "Character set table prepared by Maria Błońska"

\pismoPL{Florian Ungler 3. Pismo tekstowe gotyckie. Krój M⁹¹. Stopień 20 ww. == 72/73 mm. — Tabl. 114."}
  
 \pismoEN{Florian Ungler 3. Gothic text font. Typeface M⁹¹. Type size 20 ww. == 72/73 mm. — Plate 114.}

\plate{114}{III}{1959}

The plate prepared by Henryk Bułhak.\\
The font table prepared by Maria Błońska.

\bigskip

\exampleBib{III:60$^a$}

\bigskip \exampleDesc{IOANNES DE AUERBACH:
Processus iudiciarius. Arkusze K—O. Kraków, Florian Ungler [1514]. 4⁰.}


\medskip
\examplePage{\textit{Karta M₂a}}

  \bigskip
  \exampleLib{Biblioteka Narodowa. Warszawa.}
  

\bigskip \exampleRef{\textit{Estreicher XXXII. 57. Piekarski U. 57.}}

  \bigskip
  \exampleDig{\url{ https://cyfrowe.mnk.pl/dlibra/publication/865/} page 95}
  
% Lipsk  https://kpbc.umk.pl/dlibra/publication/edition/146354/content
% Paris   https://bibliotekacyfrowa.pl/en/dlibra/publication/53621/processus-iudiciarius-johannes-de-auerbach-1405-1469
%  Ungler 1516? https://wbc.poznan.pl/dlibra/publication/505385/edition/473732/content?ref=L3B1YmxpY2F0aW9uLzQ5NDE4NC9lZGl0aW9uLzQyNTk4OA
 
  
  % \medskip
\bigskip

\examplePL{[Pismo 3] Rubryka \beta{}. — Cyfry 5.}

\medskip

\exampleEN{[Font 3] Rubric \beta{}. — Digits 5.}



\bigskip

\fontID{U1-03}{43}

\fontstat{129}

% \exdisplay \bg \gla
 \input {t44_glyphs.tex}
%//
%\glpismo%
 \input {t44_glyphids.tex}
% //
%\endgl \xe



 \newpage
 
%%%%%%%%%%%%%%%%%%%%%%%%%%%%%%%%%%%%%%%%%%%%%%%%%%%%%%%%%%%%%%%%%%%%%%%%%%%%%%%
% from meta.csv
 % Tab. 45,Ungler1-04_PT03_115.djvu,Ungler1,04,03,115
%%%%%%%%%%%%%%%%%%%%%%%%%%%%%%%%%%%%%%%%%%%%%%%%%%%%%%%%%%%%%%%%%%%%%%%%%%%%%%%

 % from dsed4test:
% Note "4. Pismo tekstowe gotyckie. Krój M⁴⁸. Stopień 20 ww. == 79/81 mm. — Tabl. 115."
% Note1 "Character set table prepared by Maria Błońska"


\pismoPL{Florian Ungler 4. Pismo tekstowe gotyckie. Krój M⁴⁸. Stopień 20 ww. == 79/81 mm. — Tabl. 115.}
  
 \pismoEN{Florian Ungler 4. Gothic text font. Typeface M⁴⁸. Type size 20 ww. == 79/81 mm. — Plate 115.}

\plate{115}{III}{1959}

The plate prepared by Henryk Bułhak.\\
The font table prepared by Henryk Bułhak and Maria Błońska.

\bigskip

\exampleBib{III:7}

\bigskip \exampleDesc{ARISTOTELES: Oeconomicorum libri duo. Trad. Leon. Bruti Aretini. Kraków, Florian Ungler [1512]. 4⁰.}


\medskip
\examplePage{\textit{Karta A₂b}}

  \bigskip
  \exampleLib{Biblioteka Jagiellońska. Kraków.}
  

\bigskip \exampleRef{\textit{Estreicher XII. 214. Piekarski U. 6.}}

  \bigskip
  \exampleDig{\url{ https://cyfrowe.mnk.pl/dlibra/publication/865/} page 95}
% Lipsk  https://kpbc.umk.pl/dlibra/publication/edition/146354/content
% Paris   https://bibliotekacyfrowa.pl/en/dlibra/publication/53621/processus-iudiciarius-johannes-de-auerbach-1405-1469
%  Ungler 1516? https://wbc.poznan.pl/dlibra/publication/505385/edition/473732/content?ref=L3B1YmxpY2F0aW9uLzQ5NDE4NC9lZGl0aW9uLzQyNTk4OA
 
  
  % \medskip
\bigskip

\examplePL{[Pismo 4] Rubryka \beta{}.}

\medskip

\exampleEN{[Font 4] Rubric \beta{}.}



\bigskip

\fontID{U1-04}{45}

\fontstat{74}

% \exdisplay \bg \gla
 \input {t45_glyphs.tex}
%//
%\glpismo%
 \input {t45_glyphids.tex}
% //
%\endgl \xe


 \newpage
 
%%%%%%%%%%%%%%%%%%%%%%%%%%%%%%%%%%%%%%%%%%%%%%%%%%%%%%%%%%%%%%%%%%%%%%%%%%%%%%%
% from meta.csv
 % Tab. 46,Ungler1-05_PT03_116.djvu,Ungler1,05,03,116
%%%%%%%%%%%%%%%%%%%%%%%%%%%%%%%%%%%%%%%%%%%%%%%%%%%%%%%%%%%%%%%%%%%%%%%%%%%%%%%

 
 % from dsed4test:
% Note "5. Pismo tekstowe gotyckie. Krój M⁴⁸. Stopień 20 ww. == 89/90 mm. — Tabl. 116."
% Note1 "Character set table prepared by Maria Błońska"



\pismoPL{Florian Ungler 5. Pismo tekstowe gotyckie. Krój M⁴⁸. Stopień 20 ww. == 89/90 mm. — Tabl. 116.}
  
 \pismoEN{Florian Ungler 5. Gothic text font. Typeface M⁴⁸. Type size 20 ww. == 89/90 mm. — Plate 116.}

\plate{116}{III}{1959}

The plate prepared by Henryk Bułhak.\\
The font table prepared by Henryk Bułhak and Maria Błońska.

\bigskip

\exampleBib{III:22}

\bigskip \exampleDesc{IOANNES BURHARDUS: Ordo missae cum glossa Stanislai Zaborowski. Kraków,
Florian Ungler, 29. XI. 1512. 4⁰}


\medskip
\examplePage{\textit{Karta C₂b}}

  \bigskip
  \exampleLib{Biblioteka Jagiellońska. Kraków.}
  

\bigskip \exampleRef{\textit{Estreicher XXIII. 413, XX XIV. 48. Wierzbowski 2067. Piekarski U. 18.}}

  \bigskip
  \exampleDig{\url{ https://cyfrowe.mnk.pl/dlibra/publication/865/} page 95}
% Lipsk  https://kpbc.umk.pl/dlibra/publication/edition/146354/content
% Paris   https://bibliotekacyfrowa.pl/en/dlibra/publication/53621/processus-iudiciarius-johannes-de-auerbach-1405-1469
%  Ungler 1516? https://wbc.poznan.pl/dlibra/publication/505385/edition/473732/content?ref=L3B1YmxpY2F0aW9uLzQ5NDE4NC9lZGl0aW9uLzQyNTk4OA
 
  
  % \medskip
\bigskip

\examplePL{[Pismo 5]}

\medskip

\exampleEN{[Font 5]}



\bigskip

\fontID{U1-05}{46}

\fontstat{95}

% \exdisplay \bg \gla
 \input {t46_glyphs.tex}
%//
%\glpismo%
 \input {t46_glyphids.tex}
% //
%\endgl \xe


 \newpage
 
%%%%%%%%%%%%%%%%%%%%%%%%%%%%%%%%%%%%%%%%%%%%%%%%%%%%%%%%%%%%%%%%%%%%%%%%%%%%%%%
% from meta.csv
 % Tab. 47,Ungler1-06_PT03_117.djvu,Ungler1,06,03,117
%%%%%%%%%%%%%%%%%%%%%%%%%%%%%%%%%%%%%%%%%%%%%%%%%%%%%%%%%%%%%%%%%%%%%%%%%%%%%%%

 
 % from dsed4test:
%  Note "6. Pismo tekstowe gotyckie (stany a i b). Krój M⁴⁸. Stopień 20 ww. == 85/86 mm. — Tabl. 117."
% Note1 "Character set table prepared by Maria Błońska"


\pismoPL{Florian Ungler 6. Pismo tekstowe gotyckie (stany a i b). Krój M⁴⁸. Stopień 20 ww. == 85/86 mm. — Tabl. 117.}
  
 \pismoEN{Florian Ungler 5. Gothic text font (status a and b). Typeface M⁴⁸. Type size 20 ww. == 85/86 mm. — Plate 117.}

\plate{117}{III}{1959}

The plate prepared by Henryk Bułhak.\\
The font table prepared by by Henryk Bułhak and  Maria Błońska.

\bigskip

\exampleBib{III:61}

\bigskip \exampleDesc{STANISLAUS ZABOROWSKI: Orthographia polonica. Kraków [Florian Ungler, 1514—1515]. 8⁰.}


\medskip
\examplePage{\textit{Karty B₂a, C₁b}}

  \bigskip
  \exampleLib{Muzeum Narodowe. Zbiory Czapskich. Kraków.}
  

\bigskip \exampleRef{\textit{Estreicher XXXIV. 45. Piekarski U. 58.}}

\bigskip

\examplePL{Pismo 6 (alfabet polski): prawa kolumna wiersze 21—24. —
  Pismo 3: prawa kolumna wiersze 1—20, 25—30. — Rubryka \beta{} z pismem
  3. — Rubryki \epsilon{}, \zeta{}, \eta{}: z pismem 6.}

\medskip

\exampleEN{Font 6 (Polish alphabet): right column lines 21—24. —
  Font 3: right column lines 1—20, 25—30. — Rubric \beta{} with font
  3. — Rubrics \epsilon{}, \zeta{}, \eta{}: with font 6.}


\bigskip

\fontID{U1-06}{47}

\fontstat{128}

% \exdisplay \bg \gla
 \input {t47_glyphs.tex}
%//
%\glpismo%
 \input {t47_glyphids.tex}
% //
%\endgl \xe

 \newpage
 
%%%%%%%%%%%%%%%%%%%%%%%%%%%%%%%%%%%%%%%%%%%%%%%%%%%%%%%%%%%%%%%%%%%%%%%%%%%%%%%
% from meta.csv
 % Tab. 48,Ungler1-07_PT03_118.djvu,Ungler1,07,03,118
%%%%%%%%%%%%%%%%%%%%%%%%%%%%%%%%%%%%%%%%%%%%%%%%%%%%%%%%%%%%%%%%%%%%%%%%%%%%%%%

 
 % from dsed4test:
% Note "7. Pismo nagłówkowe i tekstowe gotyckie. Krój M¹⁸ raz przekreślone. Stopień 20 ww. == 104/106 mm. — Tabl. 118."
% Note1 "Character set table prepared by Maria Błońska"


\pismoPL{Florian Ungler 7. Pismo nagłówkowe i tekstowe gotyckie. Krój M¹⁸ raz przekreślone. Stopień 20 ww. == 104/106 mm. — Tabl. 118.}
  
 \pismoEN{Florian Ungler 7. Gothic header and text font. Typeface M¹⁸. Type size 20 ww. == 104/106 mm. — Plate 1187.}

\plate{118}{III}{1959}

The plate prepared by Henryk Bułhak.\\
The font table prepared by by Henryk Bułhak and  Maria Błońska.

\bigskip

\exampleBib{III:32}

\bigskip \exampleDesc{LAURENTIUS CORVINUS: Latinum ideoma. Kraków, Florian Ungler, 1513. 4⁰}

\medskip
\examplePage{\textit{Karta A₇a}}

  \bigskip
  \exampleLib{Biblioteka Narodowa. Warszawa.}
  

\bigskip \exampleRef{\textit{Estreicher XIV. 424. Wierzbowski 885. Piekarski U. 30.}}

\bigskip

\examplePL{[Pismo 7]   Rubryka \gamma{}. — Cyfry 4.}

\medskip

\exampleEN{[Font 6] Rubric \gamma{}. — Digits 4.}


\bigskip

\fontID{U1-07}{48}

\fontstat{109}

% \exdisplay \bg \gla
 \input {t48_glyphs.tex}
%//
%\glpismo%
 \input {t48_glyphids.tex}
% //
%\endgl \xe

\end{document}


 \newpage
 
%%%%%%%%%%%%%%%%%%%%%%%%%%%%%%%%%%%%%%%%%%%%%%%%%%%%%%%%%%%%%%%%%%%%%%%%%%%%%%%
% from meta.csv
 % Tab. 49
%%%%%%%%%%%%%%%%%%%%%%%%%%%%%%%%%%%%%%%%%%%%%%%%%%%%%%%%%%%%%%%%%%%%%%%%%%%%%%%

 
 % from dsed4test:
% Note "7. Pismo nagłówkowe i tekstowe gotyckie. Krój M¹⁸ raz przekreślone. Stopień 20 ww. == 104/106 mm. — Tabl. 118."
% Note1 "Character set table prepared by Maria Błońska"


\pismoPL{Florian Ungler 7. Pismo nagłówkowe i tekstowe gotyckie. Krój M¹⁸ raz przekreślone. Stopień 20 ww. == 104/106 mm. — Tabl. 118.}
  
 \pismoEN{Florian Ungler 7. Gothic header and text font. Typeface M¹⁸. Type size 20 ww. == 104/106 mm. — Plate 1187.}

\plate{118}{III}{1959}

The plate prepared by Henryk Bułhak.\\
The font table prepared by by Henryk Bułhak and  Maria Błońska.

\bigskip

\exampleBib{III:32}

\bigskip \exampleDesc{LAURENTIUS CORVINUS: Latinum ideoma. Kraków, Florian Ungler, 1513. 4⁰}

\medskip
\examplePage{\textit{Karta A₇a}}

  \bigskip
  \exampleLib{Biblioteka Narodowa. Warszawa.}
  

\bigskip \exampleRef{\textit{Estreicher XIV. 424. Wierzbowski 885. Piekarski U. 30.}}

\bigskip

\examplePL{[Pismo 7]   Rubryka \gamma{}. — Cyfry 4.}

\medskip

\exampleEN{[Font 6] Rubric \gamma{}. — Digits 4.}


\bigskip

\fontID{U1-07}{48}

\fontstat{128?}

% \exdisplay \bg \gla
 \input {t48_glyphs.tex}
%//
%\glpismo%
 \input {t48_glyphids.tex}
% //
%\endgl \xe



PTglyphs.py

The arguments:
the obligatory input directory referenced here as <dir>
an optional ``--dry-run'' argument 

Action:

 1.

 Check the contented of `glyph-test`. Abort the run with an error
 message if not empty.

2.


2.1

Check if <dir> contains the subdirectory `join`. If it exists but is
empty: abort the run.

2.2

Run

python3 batch_join_chunks.py <dir>/join


2.1 Check the consistency of the output

The `join` directory should contain at least one file with `+` in the file
name, e.g. `m44_R_lines_01_chunk_13+14.png`. Abort the run with an error
 message if there is no such file.

 2.3
 
 Ask the user if the output is acceptable and wait for an answer. 

 2.3. Copy the file(s) with `+` in the file name to the input
 directory with an informational message.

3.

Check if <dir> contains the subdirectory `split`. If it exists but is
empty: abort the run.

3.1 Check the consistency of the directory content.

The directory should contain at least one file. Abort the run
with an error message if the directory exists but is empty.

3.2

Run a script:

python3 PT_chunk_split.py <dir>/split/

3.3. Ask the user if the output is acceptable and wait for an answer.

3.4. Process the script output

The directory python3 PT_chunk_split.py <dir>/split/output will
contain two kinds of files. We *are not* interested in the files whose
base name ends with `contours`.

Copy the other files to the input directory with an informational
message.

4.

Run the scripts and commands

python3 renumber_glyphs.py <dir> glyphs-test
python3 glyphids2tex.py glyph-test glyphs4tex/ meta.csv
python3 glyph2tex.py glyph-test glyphs4tex/
cp glyph-test/*.png tables/glyphs/

From `glyphs4tex` copy to `tables` only those files which are not yet
present there (rsync?).

5.

Change to `tables` and run

xelatex  -file-line-error    -interaction=nonstopmode PTfonts.tex

6.

Return to the former directory.
======================================================================================================================

glyphs2git
1 argument <arg>


cp glyphs-test/*.png glyphs-final
git add glyphs-final/*
git add glyphs4tex/*
git commit -m "<arg>"
git push

==========================================================================================================================================

python3 PTglyphs.py glyphs-edited/45 --dry-run

python3 glyphids2tex.py glyph-test glyphs4tex/ meta.csv

python3 renumber_glyphs.py glyphs-edited/m10 glyph-test
python3 glyph2tex.py glyph-test glyphs4tex/


BAD! python3 glyphids2tex.py glyph-test/ glyphs4tex names.csv


cp glyph-test/*.png glyphs-final/
jsbien@JSBbookworm:~/git/tmp$ cp glyph-test/*.png tables/1/glyphs/
jsbien@JSBbookworm:~/git/tmp$ cp glyphs4tex/m40* tables/1/
cp: cannot stat 'glyphs4tex/m40*': No such file or directory
jsbien@JSBbookworm:~/git/tmp$ cp glyphs4tex/t40* tables/1/
jsbien@JSBbookworm:~/git/tmp$ 

git:
cd glyphs-final/
cd ../glyphs4tex/

???:
https://www.wbc.poznan.pl/dlibra/publication/504801/edition/473399?language=pl
Priorum analyticorum Aristotelis [...] libri duo castigate. Impressi secundum exemplar Jacobi Stapulensis
Haller, Jan
1510

https://en.wikipedia.org/wiki/Traditional_point-size_names
https://ewangelie.uw.edu.pl/przeklady




%%% Local Variables: 
%%% coding: utf-8-unix
%%% mode: latex
%%% TeX-master: t
%%% TeX-PDF-mode: t
%%% TeX-engine: xetex
%%% End: 
