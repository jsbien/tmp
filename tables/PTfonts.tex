% https://github.com/WGUNDERWOOD/tex-fmt
\documentclass[12pt]{article}
\usepackage[a3paper,margin=1.5cm]{geometry}
\usepackage{fontspec}
\newfontfamily{\Junicode}{Junicode}[Numbers=Lining]
\usepackage{polyglossia}
\setmainlanguage{english}
\setotherlanguage{polish}

% \usepackage{metalogo}
% \usepackage{xcolor}


\usepackage{relsize}

%\usepackage{float}

% \usepackage{caption}

\usepackage[verbose]{hyperref}

\usepackage{graphicx}
\usepackage{url}

\usepackage{expex}


\lingset{glhangstyle=none}
\defineglwlevels{pismo,nr}
\newcommand{\bg}{\begingl}

% 1 height
% 2 image
\newcommand{\PTglyph}[2]{\includegraphics[height=7ex]{glyphs/#2}}
\newcommand{\PTglyphid}[1]{#1}

\newcommand{\pismoPL}[1]{{\relsize{2}\Junicode\textbf{#1}}}
\newcommand{\pismoEN}[1]{{\relsize{1}\Junicode\begin{quote}#1\end{quote}}}
\newcommand{\plate}[3]{\textbf{Plate #1} (fasc. #2, #3)}
\newcommand{\exampleBib}[1]{{\relsize{2}\Junicode\textbf{The
      example:}\\[2ex] CATALOGUS LIBRORUM \textbf{#1}}}
\newcommand{\exampleBibExtra}[1]{{\relsize{2}\Junicode\textit{The
      plate contains also an example without a font table:}\\[2ex]
    CATALOGUS LIBRORUM \textbf{#1}}}
\newcommand{\exampleDesc}[1]{{\relsize{0}\Junicode#1}}
\newcommand{\exampleDig}[1]{{\relsize{0}\Junicode \textbf{Digitization(s) [JSB]:} #1}}
\newcommand{\exampleLib}[1]{{\relsize{0}\Junicode \textbf{Library:} #1}}
%\newcommand{\examplePL}[1]{{\relsize{0}\Junicode#1}}
\newcommand{\examplePL}[1]{}
%\newcommand{\exampleEN}[1]{{\relsize{0}\Junicode\begin{quote}#1\end{quote}}}
\newcommand{\exampleEN}[1]{}
\newcommand{\fontID}[2]{{\relsize{1}\Junicode\textbf{Font identifier} (JSB): #1 (table #2)}}
\newcommand{\fontstat}[1]{{\relsize{1}\Junicode\textbf{Statistics} (JSB): #1 glyphs.}}
\newcommand{\exampleRef}[1]{{\relsize{0}\Junicode \textbf{References:} #1}}
\newcommand{\examplePage}[1]{{Page reference: \relsize{0}\Junicode#1}}
\newcommand{\examplePageEN}[1]{{\relsize{0}\Junicode#1}}



\parindent0pt

\begin{document}

\title{POLONIA TYPOGRAPHICA
  SAECULI SEDECIMI\\
  {\relsize{-2} TŁOCZNIE POLSKIE XVI STULECIA\\ MONOGRAFIE I PODOBIZNY
    ZASOBÓW DRUKARSKICH}\\Reconstructed font tables\\
  (draft)}

\author{Janusz S. Bień (editor)}

\date{\today}

\maketitle

% \catcode`\&=11
% \catcode`\|=11

% glyphs ids:
 \catcode`\_=11

% \def\apostrof{`}


% dodac indeks!:
% \catcode`\`=\active
% \def`#1{\fbox{{\znak#1}}}

% Hoebler
\def\Hb#1{{\fontspec{Junicode}#1}}

\newcommand{\alfa}{\textit{alpha} (\J{α})}
\renewcommand{\alpha}{\textit{alpha} (\J{α})}
\renewcommand{\beta}{\textit{beta} (\J{β})}
\renewcommand{\delta}{\textit{delta} (\J{δ})}
\renewcommand{\epsilon}{\textit{epsilon} (\J{ε})}
\renewcommand{\eta}{\textit{eta} (\J{η})}
\renewcommand{\zeta}{\textit{zeta} (\J{ζ})}
\renewcommand{\theta}{\textit{theta} (\J{θ})}
\renewcommand{\gamma}{\textit{gamma} (\J{γ})}
\renewcommand{\chi}{\textit{chi} (\J{χ})}
\renewcommand{\kappa}{\textit{kappa} (\J{κ})}
\renewcommand{\iota}{\textit{iota} (\J{ι})}
\renewcommand{\lambda}{\textit{lambda} (\J{λ})}

\newpage

\section{Introduction}
\label{sec:introduction}

\subsection{Fascicules}

% – teka I – Kasper Hochfeder (1503–1505), wyd. II 1968;
% – teka II – Jan Haller (1505–1525), wyd. II 1963;
% – teka III – Florian Ungler (1510–1516), wyd. 1959;
% – teka IV – Jan Haller (1505–1525), wyd. 1962;
% – teka V – Florian Ungler (1521–1536), wyd. 1964;
% – teka VI – Florian Ungler (1521–1536), wyd. 1966;
% – teka VII – Florian Ungler (1521–1536), wyd. 1970;
% – teka VIII – Aleksander Augezdecki (1549–1561?),
% wyd. 1972;
% – teka IX – Maciej Wirzbięta (1555/7–1605), wyd. 1974;
% – teka X – Maciej Wirzbięta (1555/7–1605), wyd. 1975;
% – teka XI – Maciej i Paweł Wirzbiętowie (1555/7–
% –1609), wyd. 1981;
% – teka XII – Maciej Szarfenberg (1527–1547), wyd. 1981.

        \begin{itemize}
        \item I Kasper Hochfeder (1503–1505), wyd. II 1968;\\offline
        \item II Jan Haller (1505–1525), wyd. II 1963;\\offline
      \item III. Florian Ungler (1510–1516), wyd. 1959; pierwsza drukarnia\\
        {\url{https://polona.pl/preview/f021c751-e1bd-41ae-bb39-117399724bac}}
%        Bułhak, Henryk (1930- ) - Polonia typographica saeculi sedecimi  zbiór podobizn zasobu drukarskiego tłoczni polskich XVI stulecia. Z. 3, Pierwsza drukarnia Floriana Unglera 1510-1516 - f021c751-e1bd-41ae-bb39-117399724bac.pdf
\item IV Jan Haller (1505–1525), wyd. 1962;\\offline
      \item V. Florian Ungler (1521–1536), wyd. 1964; druga drukarnia\\
          \url{https://polona.pl/preview/29ab4ad7-f0a0-46c6-9f90-3d65e6407119}\\
%        _- Polonia typographica saeculi sedecimi  zbiór podobizn zasobu drukarskiego tłoczni polskich XVI stulecia. Z. 5, Druga drukarnia Floriana Unglera 1521-1536  tabl. 176-245 - 29ab4ad7-f0a0-46c6-9f90-3d65e6407119.pdf
%        brakuje tabel ????
        \url{https://crispa.uw.edu.pl/object/files/764299/display/Default}
        % Crispa
      \item VI. Florian Ungler (1521–1536), wyd. 1966; druga drukarnia\\
        {\url{https://polona.pl/preview/6e3ec50b-4be1-4581-945c-5665f4917178}}
 %       _- Polonia typographica saeculi sedecimi  zbiór podobizn zasobu drukarskiego tłoczni polskich XVI stulecia. Z. 6, Druga drukarnia Floriana Unglera 1521-1536  tablice 246-310 - 6e3ec50b-4be1-4581-945c-5665f4917178.pdf
      \item VII. Florian Ungler (1521–1536), wyd. 1970; druga drukarnia\\
        {\url{https://polona.pl/preview/4d111123-7add-4a11-8d32-93635145ef4b}}
%  _- Polonia typographica saeculi sedecimi  zbiór podobizn zasobu drukarskiego tłoczni polskich XVI stulecia. Z. 7, Druga drukarnia Floriana Unglera 1521-1536  tablice 311-365 - 4d111123-7add-4a11-8d32-93635145ef4b.pdf      
      \item VIII. Aleksander Augezdecki (1549–1561?), wyd. 1972;\\
        \url{https://polona.pl/preview/a3a4b39c-9cfa-434b-ba10-c2d8c7d89f62}
%  _- Polonia typographica saeculi sedecimi  zbiór podobizn zasobu drukarskiego tłoczni polskich XVI stulecia. Z. 8, Aleksander Augezdecki  Królewiec - Szamotuły 1549-1561  tabl. 366-415 - a3a4b39c-9cfa-434b-ba10-c2d8c7d89f62.pdf
\item IX. Maciej Wirzbięta (1555/7–1605), wyd. 1974;\\ \url{https://polona.pl/preview/a27a1dd3-0994-486b-9b87-5a4187bbeec2}
%  _- Polonia typographica saeculi sedecimi  zbiór podobizn zasobu drukarskiego tłoczni polskich XVI stulecia. Z. 9, Maciej Wirzbięta, Kraków 15557-1605  tablice 416-475 - a27a1dd3-0994-486b-9b87-5a4187bbeec2.pdf
\item X. Maciej Wirzbięta (1555/7–1605), wyd. 1975;\\ \url{https://polona.pl/preview/3717ea21-52eb-4dab-af8d-b84d897c71a1}
%  _- Polonia typographica saeculi sedecimi  zbiór podobizn zasobu drukarskiego tłoczni polskich XVI stulecia. Z. 10, Maciej Wirzbięta, Kraków 15557-1605  tabl. 476-520 - 3717ea21-52eb-4dab-af8d-b84d897c71a1.pdf
\item XI. Maciej i Paweł Wirzbiętowie (1555/7–1609), wyd. 1981; \\
  \url{http://polona.pl/preview/3be15c17-50a5-45f8-b4ff-6571cbd515a0}\\
% tylko uW:
  \url{https://crispa.uw.edu.pl/object/files/754259/display/Default}
      \item XII. Maciej Szarfenberg (1527–1547), wyd. 1981.\\
        \url{http://polona.pl/preview/0e5c0b64-9360-4d2f-824b-bb6e9d4bebfd}\\
        {\url{https://crispa.uw.edu.pl/object/files/754258/display/Default}}
      \end{itemize}


      For more information about \textsc{POLONIA TYPOGRAPHICA SAECULI
        SEDECIMI} please consult e.g. the GitHub repository mentioned
      below.

\bigskip

This PDF version of this paper is available as ??? on Zenodo
???. Please report noticed mistakes in the \textsf{Issues} tab in the
repository \url{https://github.com/jsbien/early_fonts_inventory}.


\subsection{CATALOGUS LIBRORUM}
\label{sec:catalogus-librorum}

By \textit{CATALOGUS LIBRORUM} we mean the list of publications
included in the booklet accompanying the plate in question. We
reference it by the item number (sometimes supplemented by a letter)
preceeded by the fascicule Roman number, e.g. `VIII:4a'.


\subsection{Haebler's font classification}
\label{sec:haebl-font-class}


Symbols such as \Hb{M¹⁶}, \Hb{M¹⁸}, \Hb{M⁴⁸}, \Hb{M⁶⁰}, \Hb{M⁹¹},
\Hb{Q|u}, \Hb{C}, \Hb{F7}, \Hb{G}, \Hb{K5} are explained in the
publication:

\begin{quote}
  Konrad Haebler:\\ «Typenrepertorium der Wiegendrucke» (the series
  \textit{Sammlung Bibliothekswissenschaftlicher Arbeiten})
  \begin{enumerate}
  \item Abteilung I: Deutschland und seine Nachbarlaender (1905).
  \item Abteilung II: Italien, die Niederlande, Frankreich, Spanien und Portugal (1908).
  \item Abteilung III:
    \begin{enumerate}
    \item Tabellen I: Antiqua-Typen (1909).
    \item Tabellen II: Gotische Typen" (1910)
    \end{enumerate}
  \item Ergänzungsband I (suplement, 1922).
  \item Ergänzungsband II (suplement, 1924).
  \end{enumerate}
\end{quote}
Reprinted in 1968, digitized in 2020 by Kujawsko-Pomorska Biblioteka
Cyfrowa (\url{https://kpbc.umk.pl/publication/222829}). Some volumes
digitized earlier by Google Books and Hathi Trust Digital Library but,
as of today, they seem available only for searching.

\noindent
Cf. also \url{https://tw.staatsbibliothek-berlin.de/}.

\subsection{Wierzbowski's bibliography}
\label{sec:wierzb-bibl}

The abbreviations in the form ``Wierzb. 891,'' refer to items (not
page numbers) in the bibliography \textit{Bibliographia Polonica XV ac
  XVII ss. quae in bibliotheca Universitatis Caesareare Varsoviensis
  asservantur} by Teodor Wierzbowski, published in 1889 and available
in the Polona digital library as
\url{https://polona.pl/item/96996417}.

\subsection{Estreicher's bibliography}
\label{sec:estr-bibl}

The abbreviations ``Estr.,'' refer Estreichers' bibliography, called
"most outstanding bibliography of Polish books, and probably one of
the most famous bibliographies in the
world".\footnote{\url{https://en.wikipedia.org/wiki/Karol_Estreicher_(senior)}}. It
was started by Karol Estrecher (1827--1908), continued by his son
Stanisław Estreicher (1869--1939) and finished by his grandson Karol
Estrecher(1906--1984).For most of the volumes the copyright has
expired, so they were reprinted, they are also (original or reprints)
available in several digital librarries,
e.g. \url{https://kpbc.umk.pl/dlibra/publication/13947} .
Additionally there is also \textit{ELEKTRONICZNA BAZA BIBLIOGRAFII
  ESTREICHERA} (EBBE), an electronic version of the
bibliographies\footnote{\url{https://www.estreicher.uj.edu.pl/}}; the
database includes also the scan of all the volumes, but there are some
restrictions on their usage.

The bibliography by Karol Estreicher (senior) consists of several
parts. The volumes have two numbers: the number in the whole
bibliography and in the specific part.
  \begin{itemize}
  \item Bibliografia polska. Cz. 1, Stólecie [!] XIX., volumes 7
  \item 
    Bibliografia polska. Cz. 2, Stólecie [!] XV-XIX spis chronologiczny. : volumes 3, one in
    two volumes (global volume numbers 8--11)
      \item 
    Bibliografia polska. Cz. 3, Stólecie [!] XV-XVIII w układzie
    abecadłowym: volumes 22 (global volume numbers 12--33)
  \item 
    Bibliografia polska. Cz. 4, Bibliografia polska XIX. stólecia [!] :
    lata 1881-1900: volumes 4
  \end{itemize}
  There are  also two unnumbered volume
  \begin{itemize}
  \item Bibliografia polska XV.-XVI. stólecia : zestawienie chronologiczne
    7200 druków w kształcie rejestru do Bibliografii, tudzież spis
    abecadłowy tych dzieł, które dochowały się w bibliotekach polskich
  \item 
    Bibliografia polska XIX. stulecia. Zeszyt dodatkowy, 1871-1873
  \end{itemize}

  The volumes authored by Stanisław Estreicher (with some use of his
  father manuscripts) include volumes 34--36 suplementing the part 3.

  The references to the bibliography contains the global volume number
  and the page number, e.g.  the reference ``Estr. XV. 242;
  XXXIV. 55'' refer to the mentions of Zaborowski's \textit{Tractatus
    contra malos divites et usurarios} which can be seen on page 242
  of volume 15\footnote{See e.g.
    \url{https://www.estreicher.uj.edu.pl/_skany/Bibliografia_Staropolska/15_Tom_XV/0257_0242.jpg}}
  and page 55 of volume 34\footnote{See e.g.
    \url{https://www.estreicher.uj.edu.pl/_skany/Bibliografia_Staropolska/34_Tom_XXXIV/0057_0055.jpg}}.

\subsection{Other abbreviations}
\label{sec:other-abbreviations}


For the time being they have to be lookep up in the appropriate
fascicules.

% Piekarski

% Knihopis

% Knih.
% Knihopis ćeskosłovenskych tiskń od doby nejstarsi aź do konce XVIH.
% stoleti. Red. Z. Tobolka. DilII. Tisky z let 1501—1800. Ć. 1 i nast. V Praze
% 1936 i nast.
% https://www.digitalniknihovna.cz/nkp/periodical/uuid:595d5a00-baf9-11e3-b74a-5ef3fc9ae867

% BM Germ.: Short-title catalogue
% Short-title catalogue of book printed in the German-speaking countries
% and German books printed in other countries from 1455 to 1600 now in
% the British Museum. London 1962.

% Bohonos Ossol.

% Drukarze IV 

% Boh. Ossol.
% Drukarze IV

% M. Bohonos: Katalog starych druków Biblioteki Zakładu Narodowego
% im. Ossolińskich. Polonica wieku XVI. Z materiałów rejestracyjnych
% zebranych zespołowo pod kierownictwem Kazimierza Zatheya opraco-
% wała... Wrocław 1965.

% Drukarze dawnej Polski od XV do XVIII wieku. T. 4: Pomorze. Oprac.
% A. Kawecka-Gryczowa oraz K. Korotajowa. Wrocław 1962.

% War.???

% The abbreviation ``K.'' occuring in some descriptions stands for
% \textit{Karty}, i.e. \textit{sheets}.

\newpage
\section{\Huge Font tables}
\label{sec:font-tables}

Font tables: 89. Typoglyphs: 7\,676???
% ls -l . | egrep -c '^-'
\newpage


%%%%%%%%%%%%%%%%%%%%%%%%%%%%%%%%%%%%%%%%%%%%%%%%%%%%%%%%%%%%%%%%%%%%%%%%%%%%%%
% Tab. 01 Aleksander Augezdecki pismo 1
%%%%%%%%%%%%%%%%%%%%%%%%%%%%%%%%%%%%%%%%%%%%%%%%%%%%%%%%%%%%%%%%%%%%%%%%%%%%%%

% 01,Augezdecki-01_PT08_402.djvu,Augezdecki,01,08,402

% Author "Paulina Buchwald-Pelcowa"
% Title "Aleksander Augezdecki 1549-1561"
% Editor "Alodia Kawecka Gryczowa"
% Series "Polonia Typographica Saeculi Sedecimi: zbiór podobizn zasobu drukarskiego tłoczni polskich XVI stulecia"
% Fascicule "VIII"
% Publisher "Zakład Narodowy imienia Ossolińskich — Wydawnictwo"
% Addres "Kraków  Wrocław Warszawa"
% Year "1972"
% Note "1. Pisma tekstowe, szwabacha M⁸¹. Stopień 20 ww. = 102—103 mm (tercja). — Tabl. 402—404. [402]"
% Note1 "Character set table prepared by Paulina Buchwald-Pelcowa"
% Note2 "Scan (prepared by Biblioteka Uniwersytecka w Warszawie from their own copy) converted to DjVu with didjvu by Janusz S. Bień"
% URL "https://github.com/jsbien/early_fonts_inventory/"

\pismoPL{Aleksander Augezdecki 1. Pisma tekstowe, szwabacha M⁸¹. Stopień 20 ww. = 102—103 mm (tercja). — Tabl. 402—404.}


\pismoEN{Aleksander Augezdecki 1. Schwabacher text script. Typeface M⁸¹. Type size 20 lines = 102—103 mm (tertia) - Plate 402-404.}

  
    \medskip

    
\plate{402}{VIII}{1972}
    
The plate    prepared by Paulina Buchwald-Pelcowa.\\
The font table    prepared by Paulina Buchwald-Pelcowa.\\

    \medskip
    
    % Biblioteka Czartoryskich. Krakéw. P.B. P.
% 4*, [TESTAMENTUM NOVUM. Evangelium secundum Matthaeum. Trad. polon. Stanislaus Murzynowski]:
% Ewangelia Sw. Mateusza. Krélewiec, [Aleksander Augezdeckij, 1551. 4°.
% Karta LXXXIIIb.
% Estreicher XIII 26. Wierzbowski 135.
% Pismo 1: tekst i zestaw wraz z zestawem liter ze znakami diakrytycznymi polskimi. — Pismo 2: szwabacha tekstowa w marginaliach. — Pismo 3: szwabacha
% komentarzowa w marginaliach. — Pismo 7: nagtowek. — Rubryki «, 8, y, 6, 7 w zestawie — Cyfry 1 z pismem 3. — Cyfry 3 z pismem 3. — Przerywniki
% 5 z pismem 1 i 3.


      \bigskip


      \exampleBib{VIII:4a}
      \bigskip


\exampleDesc{[TESTAMENTUM NOVUM. Evangelium secundum Matthaeum. Trad. polon. Stanislaus Murzynowski]:
Ewangelia Sw. Mateusza. Królewiec, [Aleksander Augezdecki], 1551. 4°.}

\medskip
\examplePage{\textit{Karta LXXXIIIb.}}


\bigskip
\exampleLib{Biblioteka Czartoryskich. Kraków.}

\bigskip

\exampleRef{\textit{Estreicher XIII 26. Wierzbowski 135.}}

\bigskip
\exampleDig{\url{https://www.dbc.wroc.pl/publication/33779} page 202.}

%      Pismo 1: tekst i zestaw wraz z zestawem liter ze znakami diakrytycznymi polskimi.

\bigskip

      \examplePL{Pismo 1: tekst i zestaw wraz z zestawem liter ze znakami diakrytycznymi polskimi.}
      
      \medskip

\exampleEN{Font 1. The text and the table including letters with Polish diacritical marks.}

  
  \bigskip
  

    \fontID{Au-01}{01}


\fontstat{188}

    % \exdisplay \bg \gla
\exdisplay \bg \gla
% 1
{\PTglyph{5}{t01_l01g01.png}}
% 2
{\PTglyph{5}{t01_l01g02.png}}
% 3
{\PTglyph{5}{t01_l01g03.png}}
% 4
{\PTglyph{5}{t01_l01g04.png}}
% 5
{\PTglyph{5}{t01_l01g05.png}}
% 6
{\PTglyph{5}{t01_l01g06.png}}
% 7
{\PTglyph{5}{t01_l01g07.png}}
% 8
{\PTglyph{5}{t01_l01g08.png}}
% 9
{\PTglyph{5}{t01_l01g09.png}}
% 10
{\PTglyph{5}{t01_l01g10.png}}
% 11
{\PTglyph{5}{t01_l01g11.png}}
% 12
{\PTglyph{5}{t01_l01g12.png}}
% 13
{\PTglyph{5}{t01_l01g13.png}}
% 14
{\PTglyph{5}{t01_l01g14.png}}
% 15
{\PTglyph{5}{t01_l01g15.png}}
% 16
{\PTglyph{5}{t01_l01g16.png}}
% 17
{\PTglyph{5}{t01_l01g17.png}}
% 18
{\PTglyph{5}{t01_l01g18.png}}
% 19
{\PTglyph{5}{t01_l01g19.png}}
% 20
{\PTglyph{5}{t01_l01g20.png}}
% 21
{\PTglyph{5}{t01_l01g21.png}}
% 22
{\PTglyph{5}{t01_l01g22.png}}
% 23
{\PTglyph{5}{t01_l01g23.png}}
% 24
{\PTglyph{5}{t01_l01g24.png}}
% 25
{\PTglyph{5}{t01_l01g25.png}}
% 26
{\PTglyph{5}{t01_l01g26.png}}
% 27
{\PTglyph{5}{t01_l01g27.png}}
% 28
{\PTglyph{5}{t01_l01g28.png}}
% 29
{\PTglyph{5}{t01_l02g01.png}}
% 30
{\PTglyph{5}{t01_l02g02.png}}
% 31
{\PTglyph{5}{t01_l02g03.png}}
% 32
{\PTglyph{5}{t01_l02g04.png}}
% 33
{\PTglyph{5}{t01_l02g05.png}}
% 34
{\PTglyph{5}{t01_l02g06.png}}
% 35
{\PTglyph{5}{t01_l02g07.png}}
% 36
{\PTglyph{5}{t01_l02g08.png}}
% 37
{\PTglyph{5}{t01_l02g09.png}}
% 38
{\PTglyph{5}{t01_l02g10.png}}
% 39
{\PTglyph{5}{t01_l02g11.png}}
% 40
{\PTglyph{5}{t01_l02g12.png}}
% 41
{\PTglyph{5}{t01_l02g13.png}}
% 42
{\PTglyph{5}{t01_l02g14.png}}
% 43
{\PTglyph{5}{t01_l02g15.png}}
% 44
{\PTglyph{5}{t01_l02g16.png}}
% 45
{\PTglyph{5}{t01_l02g17.png}}
% 46
{\PTglyph{5}{t01_l02g18.png}}
% 47
{\PTglyph{5}{t01_l02g19.png}}
% 48
{\PTglyph{5}{t01_l02g20.png}}
% 49
{\PTglyph{5}{t01_l02g21.png}}
% 50
{\PTglyph{5}{t01_l02g22.png}}
% 51
{\PTglyph{5}{t01_l02g23.png}}
% 52
{\PTglyph{5}{t01_l02g24.png}}
% 53
{\PTglyph{5}{t01_l02g25.png}}
% 54
{\PTglyph{5}{t01_l02g26.png}}
% 55
{\PTglyph{5}{t01_l02g27.png}}
% 56
{\PTglyph{5}{t01_l02g28.png}}
% 57
{\PTglyph{5}{t01_l02g29.png}}
% 58
{\PTglyph{5}{t01_l02g30.png}}
% 59
{\PTglyph{5}{t01_l02g31.png}}
% 60
{\PTglyph{5}{t01_l02g32.png}}
% 61
{\PTglyph{5}{t01_l02g33.png}}
% 62
{\PTglyph{5}{t01_l02g34.png}}
% 63
{\PTglyph{5}{t01_l02g35.png}}
% 64
{\PTglyph{5}{t01_l02g36.png}}
% 65
{\PTglyph{5}{t01_l02g37.png}}
% 66
{\PTglyph{5}{t01_l02g38.png}}
% 67
{\PTglyph{5}{t01_l02g39.png}}
% 68
{\PTglyph{5}{t01_l02g40.png}}
% 69
{\PTglyph{5}{t01_l02g41.png}}
% 70
{\PTglyph{5}{t01_l03g01.png}}
% 71
{\PTglyph{5}{t01_l03g02.png}}
% 72
{\PTglyph{5}{t01_l03g03.png}}
% 73
{\PTglyph{5}{t01_l03g04.png}}
% 74
{\PTglyph{5}{t01_l03g05.png}}
% 75
{\PTglyph{5}{t01_l03g06.png}}
% 76
{\PTglyph{5}{t01_l03g07.png}}
% 77
{\PTglyph{5}{t01_l03g08.png}}
% 78
{\PTglyph{5}{t01_l03g09.png}}
% 79
{\PTglyph{5}{t01_l03g10.png}}
% 80
{\PTglyph{5}{t01_l03g11.png}}
% 81
{\PTglyph{5}{t01_l03g12.png}}
% 82
{\PTglyph{5}{t01_l03g13.png}}
% 83
{\PTglyph{5}{t01_l03g14.png}}
% 84
{\PTglyph{5}{t01_l03g15.png}}
% 85
{\PTglyph{5}{t01_l03g16.png}}
% 86
{\PTglyph{5}{t01_l03g17.png}}
% 87
{\PTglyph{5}{t01_l03g18.png}}
% 88
{\PTglyph{5}{t01_l03g19.png}}
% 89
{\PTglyph{5}{t01_l03g20.png}}
% 90
{\PTglyph{5}{t01_l03g21.png}}
% 91
{\PTglyph{5}{t01_l03g22.png}}
% 92
{\PTglyph{5}{t01_l03g23.png}}
% 93
{\PTglyph{5}{t01_l03g24.png}}
% 94
{\PTglyph{5}{t01_l03g25.png}}
% 95
{\PTglyph{5}{t01_l03g26.png}}
% 96
{\PTglyph{5}{t01_l03g27.png}}
% 97
{\PTglyph{5}{t01_l03g28.png}}
% 98
{\PTglyph{5}{t01_l03g29.png}}
% 99
{\PTglyph{5}{t01_l03g30.png}}
% 100
{\PTglyph{5}{t01_l03g31.png}}
% 101
{\PTglyph{5}{t01_l03g32.png}}
% 102
{\PTglyph{5}{t01_l03g33.png}}
% 103
{\PTglyph{5}{t01_l03g34.png}}
% 104
{\PTglyph{5}{t01_l03g35.png}}
% 105
{\PTglyph{5}{t01_l03g36.png}}
% 106
{\PTglyph{5}{t01_l03g37.png}}
% 107
{\PTglyph{5}{t01_l03g39.png}}
% 108
{\PTglyph{5}{t01_l03g40.png}}
% 109
{\PTglyph{5}{t01_l03g41.png}}
% 110
{\PTglyph{5}{t01_l03g42.png}}
% 111
{\PTglyph{5}{t01_l03g43.png}}
% 112
{\PTglyph{5}{t01_l03g44.png}}
% 113
{\PTglyph{5}{t01_l04g01.png}}
% 114
{\PTglyph{5}{t01_l04g02.png}}
% 115
{\PTglyph{5}{t01_l04g03.png}}
% 116
{\PTglyph{5}{t01_l04g04.png}}
% 117
{\PTglyph{5}{t01_l04g05.png}}
% 118
{\PTglyph{5}{t01_l04g06.png}}
% 119
{\PTglyph{5}{t01_l04g07.png}}
% 120
{\PTglyph{5}{t01_l04g08.png}}
% 121
{\PTglyph{5}{t01_l04g09.png}}
% 122
{\PTglyph{5}{t01_l04g10.png}}
% 123
{\PTglyph{5}{t01_l04g11.png}}
% 124
{\PTglyph{5}{t01_l04g12.png}}
% 125
{\PTglyph{5}{t01_l04g13.png}}
% 126
{\PTglyph{5}{t01_l04g14.png}}
% 127
{\PTglyph{5}{t01_l04g15.png}}
% 128
{\PTglyph{5}{t01_l04g16.png}}
% 129
{\PTglyph{5}{t01_l04g17.png}}
% 130
{\PTglyph{5}{t01_l04g18.png}}
% 131
{\PTglyph{5}{t01_l04g19.png}}
% 132
{\PTglyph{5}{t01_l04g20.png}}
% 133
{\PTglyph{5}{t01_l04g21.png}}
% 134
{\PTglyph{5}{t01_l04g22.png}}
% 135
{\PTglyph{5}{t01_l04g23.png}}
% 136
{\PTglyph{5}{t01_l04g24.png}}
% 137
{\PTglyph{5}{t01_l04g25.png}}
% 138
{\PTglyph{5}{t01_l05g01.png}}
% 139
{\PTglyph{5}{t01_l05g02.png}}
% 140
{\PTglyph{5}{t01_l05g03.png}}
% 141
{\PTglyph{5}{t01_l05g04.png}}
% 142
{\PTglyph{5}{t01_l05g05.png}}
% 143
{\PTglyph{5}{t01_l05g06.png}}
% 144
{\PTglyph{5}{t01_l05g07.png}}
% 145
{\PTglyph{5}{t01_l05g08.png}}
% 146
{\PTglyph{5}{t01_l05g09.png}}
% 147
{\PTglyph{5}{t01_l05g10.png}}
% 148
{\PTglyph{5}{t01_l05g11.png}}
% 149
{\PTglyph{5}{t01_l05g12.png}}
% 150
{\PTglyph{5}{t01_l05g13.png}}
% 151
{\PTglyph{5}{t01_l05g14.png}}
% 152
{\PTglyph{5}{t01_l05g15.png}}
% 153
{\PTglyph{5}{t01_l05g16.png}}
% 154
{\PTglyph{5}{t01_l05g17.png}}
% 155
{\PTglyph{5}{t01_l05g18.png}}
% 156
{\PTglyph{5}{t01_l05g19.png}}
% 157
{\PTglyph{5}{t01_l05g20.png}}
% 158
{\PTglyph{5}{t01_l05g22.png}}
% 159
{\PTglyph{5}{t01_l05g23.png}}
% 160
{\PTglyph{5}{t01_l05g24.png}}
% 161
{\PTglyph{5}{t01_l05g25.png}}
% 162
{\PTglyph{5}{t01_l05g26.png}}
% 163
{\PTglyph{5}{t01_l05g27.png}}
% 164
{\PTglyph{5}{t01_l05g28.png}}
% 165
{\PTglyph{5}{t01_l05g29.png}}
% 166
{\PTglyph{5}{t01_l05g30.png}}
% 167
{\PTglyph{5}{t01_l05g31.png}}
% 168
{\PTglyph{5}{t01_l05g32.png}}
% 169
{\PTglyph{5}{t01_l05g33.png}}
% 170
{\PTglyph{5}{t01_l05g34.png}}
% 171
{\PTglyph{5}{t01_l05g35.png}}
% 172
{\PTglyph{5}{t01_l05g36.png}}
% 173
{\PTglyph{5}{t01_l05g37.png}}
% 174
{\PTglyph{5}{t01_l05g38.png}}
% 175
{\PTglyph{5}{t01_l05g39.png}}
% 176
{\PTglyph{5}{t01_l05g40.png}}
% 177
{\PTglyph{5}{t01_l05g41.png}}
% 178
{\PTglyph{5}{t01_l05g42.png}}
% 179
{\PTglyph{5}{t01_l06g01.png}}
% 180
{\PTglyph{5}{t01_l06g02.png}}
% 181
{\PTglyph{5}{t01_l06g03.png}}
% 182
{\PTglyph{5}{t01_l06g04.png}}
% 183
{\PTglyph{5}{t01_l06g05.png}}
% 184
{\PTglyph{5}{t01_l06g06.png}}
% 185
{\PTglyph{5}{t01_l06g07.png}}
% 186
{\PTglyph{5}{t01_l06g08.png}}
% 187
{\PTglyph{5}{t01_l06g09.png}}
% 188
{\PTglyph{5}{t01_l06g10.png}}
//
%%% Local Variables:
%%% mode: latex
%%% TeX-engine: luatex
%%% TeX-master: shared
%%% End:

%//
%\glpismo
\glpismo
% 1
{\PTglyphid{Au-01_0101}}
% 2
{\PTglyphid{Au-01_0102}}
% 3
{\PTglyphid{Au-01_0103}}
% 4
{\PTglyphid{Au-01_0104}}
% 5
{\PTglyphid{Au-01_0105}}
% 6
{\PTglyphid{Au-01_0106}}
% 7
{\PTglyphid{Au-01_0107}}
% 8
{\PTglyphid{Au-01_0108}}
% 9
{\PTglyphid{Au-01_0109}}
% 10
{\PTglyphid{Au-01_0110}}
% 11
{\PTglyphid{Au-01_0111}}
% 12
{\PTglyphid{Au-01_0112}}
% 13
{\PTglyphid{Au-01_0113}}
% 14
{\PTglyphid{Au-01_0114}}
% 15
{\PTglyphid{Au-01_0115}}
% 16
{\PTglyphid{Au-01_0116}}
% 17
{\PTglyphid{Au-01_0117}}
% 18
{\PTglyphid{Au-01_0118}}
% 19
{\PTglyphid{Au-01_0119}}
% 20
{\PTglyphid{Au-01_0120}}
% 21
{\PTglyphid{Au-01_0121}}
% 22
{\PTglyphid{Au-01_0122}}
% 23
{\PTglyphid{Au-01_0123}}
% 24
{\PTglyphid{Au-01_0124}}
% 25
{\PTglyphid{Au-01_0125}}
% 26
{\PTglyphid{Au-01_0126}}
% 27
{\PTglyphid{Au-01_0127}}
% 28
{\PTglyphid{Au-01_0128}}
% 29
{\PTglyphid{Au-01_0201}}
% 30
{\PTglyphid{Au-01_0202}}
% 31
{\PTglyphid{Au-01_0203}}
% 32
{\PTglyphid{Au-01_0204}}
% 33
{\PTglyphid{Au-01_0205}}
% 34
{\PTglyphid{Au-01_0206}}
% 35
{\PTglyphid{Au-01_0207}}
% 36
{\PTglyphid{Au-01_0208}}
% 37
{\PTglyphid{Au-01_0209}}
% 38
{\PTglyphid{Au-01_0210}}
% 39
{\PTglyphid{Au-01_0211}}
% 40
{\PTglyphid{Au-01_0212}}
% 41
{\PTglyphid{Au-01_0213}}
% 42
{\PTglyphid{Au-01_0214}}
% 43
{\PTglyphid{Au-01_0215}}
% 44
{\PTglyphid{Au-01_0216}}
% 45
{\PTglyphid{Au-01_0217}}
% 46
{\PTglyphid{Au-01_0218}}
% 47
{\PTglyphid{Au-01_0219}}
% 48
{\PTglyphid{Au-01_0220}}
% 49
{\PTglyphid{Au-01_0221}}
% 50
{\PTglyphid{Au-01_0222}}
% 51
{\PTglyphid{Au-01_0223}}
% 52
{\PTglyphid{Au-01_0224}}
% 53
{\PTglyphid{Au-01_0225}}
% 54
{\PTglyphid{Au-01_0226}}
% 55
{\PTglyphid{Au-01_0227}}
% 56
{\PTglyphid{Au-01_0228}}
% 57
{\PTglyphid{Au-01_0229}}
% 58
{\PTglyphid{Au-01_0230}}
% 59
{\PTglyphid{Au-01_0231}}
% 60
{\PTglyphid{Au-01_0232}}
% 61
{\PTglyphid{Au-01_0233}}
% 62
{\PTglyphid{Au-01_0234}}
% 63
{\PTglyphid{Au-01_0235}}
% 64
{\PTglyphid{Au-01_0236}}
% 65
{\PTglyphid{Au-01_0237}}
% 66
{\PTglyphid{Au-01_0238}}
% 67
{\PTglyphid{Au-01_0239}}
% 68
{\PTglyphid{Au-01_0240}}
% 69
{\PTglyphid{Au-01_0241}}
% 70
{\PTglyphid{Au-01_0301}}
% 71
{\PTglyphid{Au-01_0302}}
% 72
{\PTglyphid{Au-01_0303}}
% 73
{\PTglyphid{Au-01_0304}}
% 74
{\PTglyphid{Au-01_0305}}
% 75
{\PTglyphid{Au-01_0306}}
% 76
{\PTglyphid{Au-01_0307}}
% 77
{\PTglyphid{Au-01_0308}}
% 78
{\PTglyphid{Au-01_0309}}
% 79
{\PTglyphid{Au-01_0310}}
% 80
{\PTglyphid{Au-01_0311}}
% 81
{\PTglyphid{Au-01_0312}}
% 82
{\PTglyphid{Au-01_0313}}
% 83
{\PTglyphid{Au-01_0314}}
% 84
{\PTglyphid{Au-01_0315}}
% 85
{\PTglyphid{Au-01_0316}}
% 86
{\PTglyphid{Au-01_0317}}
% 87
{\PTglyphid{Au-01_0318}}
% 88
{\PTglyphid{Au-01_0319}}
% 89
{\PTglyphid{Au-01_0320}}
% 90
{\PTglyphid{Au-01_0321}}
% 91
{\PTglyphid{Au-01_0322}}
% 92
{\PTglyphid{Au-01_0323}}
% 93
{\PTglyphid{Au-01_0324}}
% 94
{\PTglyphid{Au-01_0325}}
% 95
{\PTglyphid{Au-01_0326}}
% 96
{\PTglyphid{Au-01_0327}}
% 97
{\PTglyphid{Au-01_0328}}
% 98
{\PTglyphid{Au-01_0329}}
% 99
{\PTglyphid{Au-01_0330}}
% 100
{\PTglyphid{Au-01_0331}}
% 101
{\PTglyphid{Au-01_0332}}
% 102
{\PTglyphid{Au-01_0333}}
% 103
{\PTglyphid{Au-01_0334}}
% 104
{\PTglyphid{Au-01_0335}}
% 105
{\PTglyphid{Au-01_0336}}
% 106
{\PTglyphid{Au-01_0337}}
% 107
{\PTglyphid{Au-01_0339}}
% 108
{\PTglyphid{Au-01_0340}}
% 109
{\PTglyphid{Au-01_0341}}
% 110
{\PTglyphid{Au-01_0342}}
% 111
{\PTglyphid{Au-01_0343}}
% 112
{\PTglyphid{Au-01_0344}}
% 113
{\PTglyphid{Au-01_0401}}
% 114
{\PTglyphid{Au-01_0402}}
% 115
{\PTglyphid{Au-01_0403}}
% 116
{\PTglyphid{Au-01_0404}}
% 117
{\PTglyphid{Au-01_0405}}
% 118
{\PTglyphid{Au-01_0406}}
% 119
{\PTglyphid{Au-01_0407}}
% 120
{\PTglyphid{Au-01_0408}}
% 121
{\PTglyphid{Au-01_0409}}
% 122
{\PTglyphid{Au-01_0410}}
% 123
{\PTglyphid{Au-01_0411}}
% 124
{\PTglyphid{Au-01_0412}}
% 125
{\PTglyphid{Au-01_0413}}
% 126
{\PTglyphid{Au-01_0414}}
% 127
{\PTglyphid{Au-01_0415}}
% 128
{\PTglyphid{Au-01_0416}}
% 129
{\PTglyphid{Au-01_0417}}
% 130
{\PTglyphid{Au-01_0418}}
% 131
{\PTglyphid{Au-01_0419}}
% 132
{\PTglyphid{Au-01_0420}}
% 133
{\PTglyphid{Au-01_0421}}
% 134
{\PTglyphid{Au-01_0422}}
% 135
{\PTglyphid{Au-01_0423}}
% 136
{\PTglyphid{Au-01_0424}}
% 137
{\PTglyphid{Au-01_0425}}
% 138
{\PTglyphid{Au-01_0501}}
% 139
{\PTglyphid{Au-01_0502}}
% 140
{\PTglyphid{Au-01_0503}}
% 141
{\PTglyphid{Au-01_0504}}
% 142
{\PTglyphid{Au-01_0505}}
% 143
{\PTglyphid{Au-01_0506}}
% 144
{\PTglyphid{Au-01_0507}}
% 145
{\PTglyphid{Au-01_0508}}
% 146
{\PTglyphid{Au-01_0509}}
% 147
{\PTglyphid{Au-01_0510}}
% 148
{\PTglyphid{Au-01_0511}}
% 149
{\PTglyphid{Au-01_0512}}
% 150
{\PTglyphid{Au-01_0513}}
% 151
{\PTglyphid{Au-01_0514}}
% 152
{\PTglyphid{Au-01_0515}}
% 153
{\PTglyphid{Au-01_0516}}
% 154
{\PTglyphid{Au-01_0517}}
% 155
{\PTglyphid{Au-01_0518}}
% 156
{\PTglyphid{Au-01_0519}}
% 157
{\PTglyphid{Au-01_0520}}
% 158
{\PTglyphid{Au-01_0522}}
% 159
{\PTglyphid{Au-01_0523}}
% 160
{\PTglyphid{Au-01_0524}}
% 161
{\PTglyphid{Au-01_0525}}
% 162
{\PTglyphid{Au-01_0526}}
% 163
{\PTglyphid{Au-01_0527}}
% 164
{\PTglyphid{Au-01_0528}}
% 165
{\PTglyphid{Au-01_0529}}
% 166
{\PTglyphid{Au-01_0530}}
% 167
{\PTglyphid{Au-01_0531}}
% 168
{\PTglyphid{Au-01_0532}}
% 169
{\PTglyphid{Au-01_0533}}
% 170
{\PTglyphid{Au-01_0534}}
% 171
{\PTglyphid{Au-01_0535}}
% 172
{\PTglyphid{Au-01_0536}}
% 173
{\PTglyphid{Au-01_0537}}
% 174
{\PTglyphid{Au-01_0538}}
% 175
{\PTglyphid{Au-01_0539}}
% 176
{\PTglyphid{Au-01_0540}}
% 177
{\PTglyphid{Au-01_0541}}
% 178
{\PTglyphid{Au-01_0542}}
% 179
{\PTglyphid{Au-01_0601}}
% 180
{\PTglyphid{Au-01_0602}}
% 181
{\PTglyphid{Au-01_0603}}
% 182
{\PTglyphid{Au-01_0604}}
% 183
{\PTglyphid{Au-01_0605}}
% 184
{\PTglyphid{Au-01_0606}}
% 185
{\PTglyphid{Au-01_0607}}
% 186
{\PTglyphid{Au-01_0608}}
% 187
{\PTglyphid{Au-01_0609}}
% 188
{\PTglyphid{Au-01_0610}}
//
\endgl \xe
%%% Local Variables:
%%% mode: latex
%%% TeX-engine: luatex
%%% TeX-master: shared
%%% End:

% //
%\endgl \xe

%\end{flushleft}


\newpage
%%%%%%%%%%%%%%%%%%%%%%%%%%%%%%%%%%%%%%%%%%%%%%%%%%%%%%%%%%%%%%%%%%%%%%%%%%%%%%
% Tab. 02 Aleksander Augezdecki pismo 1a
%%%%%%%%%%%%%%%%%%%%%%%%%%%%%%%%%%%%%%%%%%%%%%%%%%%%%%%%%%%%%%%%%%%%%%%%%%%%%%

% Author "Paulina Buchwald-Pelcowa"
% Title "Aleksander Augezdecki 1549-1561"
% Editor "Alodia Kawecka Gryczowa"
% Series "Polonia Typographica Saeculi Sedecimi: zbiór podobizn zasobu drukarskiego tłoczni polskich XVI stulecia"
% Fascicule "VIII"
% Publisher "Zakład Narodowy imienia Ossolińskich — Wydawnictwo"
% Addres "Kraków  Wrocław Warszawa"
% Year "1972"
% Note "1. Pisma tekstowe, szwabacha M⁸¹. Stopień 20 ww. = 102—103 mm (tercja). — Tabl. 402—404. [403]"
% Note1 "Character set table prepared by Paulina Buchwald-Pelcowa"
% Note2 "Scan (prepared by Biblioteka Uniwersytecka w Warszawie from their own copy) converted to DjVu with didjvu by Janusz S. Bień"
    
\pismoPL{Aleksander Augezdecki 1. Pisma tekstowe, szwabacha M⁸¹. Stopień 20 ww. = 102—103 mm (tercja). — Tabl. 402—404.
[zestaw liter ze znakami diakrytycznymi czeskimi]}

\pismoEN{Aleksander Augezdecki 1. Schwabacher text script. Typeface M⁸¹. Type size 20 lines = 102—103 mm (tertia) - Plate 402-404.
[the table of letters with Czech diacritical marks]}

\medskip

\plate{403}{VIII}{1972}

The plate    prepared by Paulina Buchwald-Pelcowa.\\
The font table    prepared by Paulina Buchwald-Pelcowa.\\

\bigskip

\exampleBib{VIII:25}

\medskip
\bigskip

\exampleDesc{PIESNĚ Chwal Bożských. Szamotuly, Aleksander Augezdecki, [25 I 1560—] 7 VI 1561. 2°. War. A.}

\medskip
\examplePage{\textit{Karta *₂b}}

\bigskip
\exampleLib{Biblioteka Czartoryskich. Kraków.}

\bigskip
\exampleRef{\textit{Estreicher XIX 91. Knihopis 12860.}}

\bigskip
\exampleDig{\url{https://cyfrowe.mnk.pl/dlibra/publication/13639/}, page 8.}

    \examplePL{Pismo 1: tekst i zestaw liter ze znakami diakrytycznymi czeskimi.}

    \medskip

    \exampleEN{Font 1. The text and the table of letters with Czech diacritical marks}

\bigskip

    \fontID{Au-01a}{02}

\fontstat{43}

\bigskip

% \exdisplay \bg \gla
\exdisplay \bg \gla
% 1
{\PTglyph{5}{t02_l01g01.png}}
% 2
{\PTglyph{5}{t02_l01g02.png}}
% 3
{\PTglyph{5}{t02_l01g03.png}}
% 4
{\PTglyph{5}{t02_l01g04.png}}
% 5
{\PTglyph{5}{t02_l01g05.png}}
% 6
{\PTglyph{5}{t02_l01g06.png}}
% 7
{\PTglyph{5}{t02_l01g07.png}}
% 8
{\PTglyph{5}{t02_l01g08.png}}
% 9
{\PTglyph{5}{t02_l01g09.png}}
% 10
{\PTglyph{5}{t02_l01g10.png}}
% 11
{\PTglyph{5}{t02_l02g01.png}}
% 12
{\PTglyph{5}{t02_l02g02.png}}
% 13
{\PTglyph{5}{t02_l02g03.png}}
% 14
{\PTglyph{5}{t02_l02g04.png}}
% 15
{\PTglyph{5}{t02_l02g05.png}}
% 16
{\PTglyph{5}{t02_l02g06.png}}
% 17
{\PTglyph{5}{t02_l02g07.png}}
% 18
{\PTglyph{5}{t02_l02g08.png}}
% 19
{\PTglyph{5}{t02_l02g10.png}}
% 20
{\PTglyph{5}{t02_l02g11.png}}
% 21
{\PTglyph{5}{t02_l02g12.png}}
% 22
{\PTglyph{5}{t02_l02g13.png}}
% 23
{\PTglyph{5}{t02_l02g14.png}}
% 24
{\PTglyph{5}{t02_l02g15.png}}
% 25
{\PTglyph{5}{t02_l02g16.png}}
% 26
{\PTglyph{5}{t02_l02g17.png}}
% 27
{\PTglyph{5}{t02_l02g19.png}}
% 28
{\PTglyph{5}{t02_l02g20.png}}
% 29
{\PTglyph{5}{t02_l02g21.png}}
% 30
{\PTglyph{5}{t02_l02g22.png}}
% 31
{\PTglyph{5}{t02_l02g23.png}}
% 32
{\PTglyph{5}{t02_l02g24.png}}
% 33
{\PTglyph{5}{t02_l02g25.png}}
% 34
{\PTglyph{5}{t02_l02g26.png}}
% 35
{\PTglyph{5}{t02_l02g27.png}}
% 36
{\PTglyph{5}{t02_l02g28.png}}
% 37
{\PTglyph{5}{t02_l02g29.png}}
% 38
{\PTglyph{5}{t02_l02g30.png}}
% 39
{\PTglyph{5}{t02_l02g31.png}}
% 40
{\PTglyph{5}{t02_l02g32.png}}
% 41
{\PTglyph{5}{t02_l02g33.png}}
% 42
{\PTglyph{5}{t02_l02g34.png}}
% 43
{\PTglyph{5}{t02_l02g35.png}}
//
%%% Local Variables:
%%% mode: latex
%%% TeX-engine: luatex
%%% TeX-master: shared
%%% End:

%//
%\glpismo
\glpismo
% 1
{\PTglyphid{Au-01a0101}}
% 2
{\PTglyphid{Au-01a0102}}
% 3
{\PTglyphid{Au-01a0103}}
% 4
{\PTglyphid{Au-01a0104}}
% 5
{\PTglyphid{Au-01a0105}}
% 6
{\PTglyphid{Au-01a0106}}
% 7
{\PTglyphid{Au-01a0107}}
% 8
{\PTglyphid{Au-01a0108}}
% 9
{\PTglyphid{Au-01a0109}}
% 10
{\PTglyphid{Au-01a0110}}
% 11
{\PTglyphid{Au-01a0201}}
% 12
{\PTglyphid{Au-01a0202}}
% 13
{\PTglyphid{Au-01a0203}}
% 14
{\PTglyphid{Au-01a0204}}
% 15
{\PTglyphid{Au-01a0205}}
% 16
{\PTglyphid{Au-01a0206}}
% 17
{\PTglyphid{Au-01a0207}}
% 18
{\PTglyphid{Au-01a0208}}
% 19
{\PTglyphid{Au-01a0210}}
% 20
{\PTglyphid{Au-01a0211}}
% 21
{\PTglyphid{Au-01a0212}}
% 22
{\PTglyphid{Au-01a0213}}
% 23
{\PTglyphid{Au-01a0214}}
% 24
{\PTglyphid{Au-01a0215}}
% 25
{\PTglyphid{Au-01a0216}}
% 26
{\PTglyphid{Au-01a0217}}
% 27
{\PTglyphid{Au-01a0219}}
% 28
{\PTglyphid{Au-01a0220}}
% 29
{\PTglyphid{Au-01a0221}}
% 30
{\PTglyphid{Au-01a0222}}
% 31
{\PTglyphid{Au-01a0223}}
% 32
{\PTglyphid{Au-01a0224}}
% 33
{\PTglyphid{Au-01a0225}}
% 34
{\PTglyphid{Au-01a0226}}
% 35
{\PTglyphid{Au-01a0227}}
% 36
{\PTglyphid{Au-01a0228}}
% 37
{\PTglyphid{Au-01a0229}}
% 38
{\PTglyphid{Au-01a0230}}
% 39
{\PTglyphid{Au-01a0231}}
% 40
{\PTglyphid{Au-01a0232}}
% 41
{\PTglyphid{Au-01a0233}}
% 42
{\PTglyphid{Au-01a0234}}
% 43
{\PTglyphid{Au-01a0235}}
//
\endgl \xe
%%% Local Variables:
%%% mode: latex
%%% TeX-engine: luatex
%%% TeX-master: shared
%%% End:

% //
%\endgl \xe


\newpage
%%%%%%%%%%%%%%%%%%%%%%%%%%%%%%%%%%%%%%%%%%%%%%%%%%%%%%%%%%%%%%%%%%%%%%%%%%%%%%
% Tab. 03 Aleksander Augezdecki pismo 1b
%%%%%%%%%%%%%%%%%%%%%%%%%%%%%%%%%%%%%%%%%%%%%%%%%%%%%%%%%%%%%%%%%%%%%%%%%%%%%%

\pismoPL{Aleksander Augezdecki 1. Pisma tekstowe, szwabacha
  M⁸¹. Stopień 20 ww. = 102—103 mm (tercja). — Tabl. 402—404. [znaki
    diakrytyczne niemieckie, notacja matematyczna]}

  \pismoEN{Aleksander Augezdecki 1. Schwabacher text script. Typeface
    M⁸¹. Type size 20 lines = 102—103 mm (tertia) - Plate
    402-404. [German diacritical marks, mathematical notation]}

\medskip

\plate{404}{VIII}{1972}

The plate    prepared by Paulina Buchwald-Pelcowa.\\
The font table    prepared by Paulina Buchwald-Pelcowa.\\

\bigskip

\exampleBib{VIII:12}

\bigskip
\exampleDesc{CHRISTOPHORUS RUDOLFF: Die Coss. 4°. Królewiec, Aleksander Augezdecki, 1553. 4°.}

\medskip
\examplePage{\textit{Karta 63a.}}

  \bigskip
\exampleLib{Biblioteka Czartoryskich. Kraków.}

\bigskip
\exampleRef{\textit{BM Germ.: Short-title catalogue s. 759}}

\bigskip
\exampleDig{\url{https://ds.ub.uni-bielefeld.de/viewer/api/v1/records/2014414/sections/LOG_0000/pdf/}}

\medskip

    \examplePL{Pismo 1: tekst i zestaw liter ze znakami diakrytycznymi niemieckimi.}

    \medskip

    \exampleEN{Font 1. The text and the table of letters with German diacritical marks [and mathematical notation]}

      \bigskip

%https://www.unicode.org/L2/L2024/24141-n5277-leibniz.pdf
%      Biblioteka Czartoryskich. Krakéw.
% P.B.P. 
% 12. CHRISTOPHORUS RUDOLFF: Die Coss. 4°. Krélewiec, Aleksander Augezdecki, 1553. 4°.
% Pismo 1: tekst i zestaw liter ze znakami diakrytycznymi niemieckimi. — Cyfry 4 z pismem 1.
% Karta 63a.
% BM Germ.: Short-title catalogue s. 759,

% https://www.unicode.org/L2/L2024/24141-n5277-leibniz.pdf
% https://www.researchgate.net/publication/230735404_From_the_second_unknown_to_the_symbolic_equation

%       12 karta 63a

%       1553
% CHRISTOPHORUS RUDOLFF: Die Coss. Ed.
% Michael Stifel. 4*. K. 104. (Arkusze A—Z Aa— Cc) —
% BM Germ. 759. Arkusze Dd—Zz Aaa—Zzz
% Aaaa—Zzzz Aaaaa—Zzzzz Aaaaaa—LlIII zob.
% poz. 138. *
% W niektórych egzemplarzach różnice w foliacji i sygnacji. Egzem-
% plarz B. Nar. XVI. Qu 115 k. ... 90,9, 92, 93, 95, 95—212...; egzem-
% płarz B. Czart. 52811 II k. ... 89—93, 95, 95—212..., karta Ca
% błędnie sygnowana B;.
% Pisma 1, 4—6, 13, 15. — Rubryka a, $, y. — Cyfry 1—4. —
% Przerywnik 2. — Inicjały 3, 15, 16, 21, 22, 27. — Drzeworyty
% 20, 21—23. - Li2.
      
% Digitization %%%%%which variant????
% https://ds.ub.uni-bielefeld.de/viewer/image/2014414/1/LOG_0000/
% https://old.maa.org/press/periodicals/convergence/mathematical-treasures-rudolffs-arithmetic-and-algebra
% https://old.maa.org/sites/default/files/images/upload_library/46/Swetz_2012_Math_Treasures/ColumbiaU/1302100032.png

\bigskip

    \fontID{Au-01b}{03}

    \fontstat{23}

    \bigskip
% \exdisplay \bg \gla
\exdisplay \bg \gla
% 1
{\PTglyph{5}{t03_l01g01.png}}
% 2
{\PTglyph{5}{t03_l01g02.png}}
% 3
{\PTglyph{5}{t03_l01g03.png}}
% 4
{\PTglyph{5}{t03_l01g04.png}}
% 5
{\PTglyph{5}{t03_l01g05.png}}
% 6
{\PTglyph{5}{t03_l01g06.png}}
% 7
{\PTglyph{5}{t03_l01g07.png}}
% 8
{\PTglyph{5}{t03_l01g08.png}}
% 9
{\PTglyph{5}{t03_l01g09.png}}
% 10
{\PTglyph{5}{t03_l01g10.png}}
% 11
{\PTglyph{5}{t03_l01g11.png}}
% 12
{\PTglyph{5}{t03_l01g12.png}}
% 13
{\PTglyph{5}{t03_l01g13.png}}
% 14
{\PTglyph{5}{t03_l01g14.png}}
% 15
{\PTglyph{5}{t03_l01g15.png}}
% 16
{\PTglyph{5}{t03_l01g16.png}}
% 17
{\PTglyph{5}{t03_l01g17.png}}
% 18
{\PTglyph{5}{t03_l01g18.png}}
% 19
{\PTglyph{5}{t03_l01g19.png}}
% 20
{\PTglyph{5}{t03_l01g20.png}}
% 21
{\PTglyph{5}{t03_l01g21.png}}
% 22
{\PTglyph{5}{t03_l01g22.png}}
% 23
{\PTglyph{5}{t03_l01g23.png}}
//
%%% Local Variables:
%%% mode: latex
%%% TeX-engine: luatex
%%% TeX-master: shared
%%% End:

%//
%\glpismo
\glpismo
% 1
{\PTglyphid{Au-01b0101}}
% 2
{\PTglyphid{Au-01b0102}}
% 3
{\PTglyphid{Au-01b0103}}
% 4
{\PTglyphid{Au-01b0104}}
% 5
{\PTglyphid{Au-01b0105}}
% 6
{\PTglyphid{Au-01b0106}}
% 7
{\PTglyphid{Au-01b0107}}
% 8
{\PTglyphid{Au-01b0108}}
% 9
{\PTglyphid{Au-01b0109}}
% 10
{\PTglyphid{Au-01b0110}}
% 11
{\PTglyphid{Au-01b0111}}
% 12
{\PTglyphid{Au-01b0112}}
% 13
{\PTglyphid{Au-01b0113}}
% 14
{\PTglyphid{Au-01b0114}}
% 15
{\PTglyphid{Au-01b0115}}
% 16
{\PTglyphid{Au-01b0116}}
% 17
{\PTglyphid{Au-01b0117}}
% 18
{\PTglyphid{Au-01b0118}}
% 19
{\PTglyphid{Au-01b0119}}
% 20
{\PTglyphid{Au-01b0120}}
% 21
{\PTglyphid{Au-01b0121}}
% 22
{\PTglyphid{Au-01b0122}}
% 23
{\PTglyphid{Au-01b0123}}
//
\endgl \xe
%%% Local Variables:
%%% mode: latex
%%% TeX-engine: luatex
%%% TeX-master: shared
%%% End:

% //
%\endgl \xe

\newpage
%%%%%%%%%%%%%%%%%%%%%%%%%%%%%%%%%%%%%%%%%%%%%%%%%%%%%%%%%%%%%%%%%%%%%%%%%%%%%%
% Tab. 04 Aleksander Augezdecki pismo 2
%%%%%%%%%%%%%%%%%%%%%%%%%%%%%%%%%%%%%%%%%%%%%%%%%%%%%%%%%%%%%%%%%%%%%%%%%%%%%%

% Author "Paulina Buchwald-Pelcowa"
% Title "Aleksander Augezdecki 1549-1561"
% Editor "Alodia Kawecka Gryczowa"
% Series "Polonia Typographica Saeculi Sedecimi: zbiór podobizn zasobu drukarskiego tłoczni polskich XVI stulecia"
% Fascicule "VIII"
% Publisher "Zakład Narodowy imienia Ossolińskich — Wydawnictwo"
% Addres "Kraków  Wrocław Warszawa"
% Year "1972"
% Note "2 Pismo tekstowe, szwabacha M⁸¹. Stopień 20 ww. = 86—87 mm (cycero). — Tabl. 405, 406. [406]"
% Note1 "Character set table prepared by Paulina Buchwald-Pelcowa"

\pismoPL{Aleksander Augezdecki 2. Pisma tekstowe, szwabacha M⁸¹. Stopień 20 ww. = 86—87 mm (cycero). — Tabl. 402—404.}

\pismoEN{Aleksander Augezdecki 2. Schwabacher text script. Typeface M⁸¹. Type size 20 lines = 86—87 mm (cicero) - Plate 402-404.}

\medskip

\plate{405}{VIII}{1972}

The plate    prepared by Paulina Buchwald-Pelcowa.\\
The font table    prepared by Paulina Buchwald-Pelcowa.\\


\bigskip

\exampleBib{VIII:6}

\bigskip
\exampleDesc{[TESTAMENTUM NOVUM. Trad. polon. Stanislaus Murzynowski]: Testamentu Nowego część pierwsza.
Królewiec, Aleksander Augezdecki, X 1551. 4°.}

\medskip
\examplePage{\textit{Karta B₁a.}}

  \bigskip
\exampleLib{Biblioteka Czartoryskich. Kraków.}

\bigskip
\exampleRef{\textit{Estreicher XIII 26, Wierzbowski 1288.}}

\bigskip
\exampleDig{\url{https://www.dbc.wroc.pl/publication/33779} page 13.}


\medskip

    \examplePL{Pismo 2: tekst i zestaw wraz z zestawem liter ze znakami diakrytycznymi polskimi.}

    \medskip

    \exampleEN{Font 2. The text and the table including letters with Polish diacritical marks.}


\bigskip

    \fontID{Au-02}{04}

    \fontstat{158}

\bigskip

% \exdisplay \bg \gla
\exdisplay \bg \gla
% 1
{\PTglyph{5}{t04_l01g01.png}}
% 2
{\PTglyph{5}{t04_l01g02.png}}
% 3
{\PTglyph{5}{t04_l01g03.png}}
% 4
{\PTglyph{5}{t04_l01g04.png}}
% 5
{\PTglyph{5}{t04_l01g05.png}}
% 6
{\PTglyph{5}{t04_l01g06.png}}
% 7
{\PTglyph{5}{t04_l01g07.png}}
% 8
{\PTglyph{5}{t04_l01g08.png}}
% 9
{\PTglyph{5}{t04_l01g09.png}}
% 10
{\PTglyph{5}{t04_l01g10.png}}
% 11
{\PTglyph{5}{t04_l01g11.png}}
% 12
{\PTglyph{5}{t04_l01g12.png}}
% 13
{\PTglyph{5}{t04_l01g13.png}}
% 14
{\PTglyph{5}{t04_l01g14.png}}
% 15
{\PTglyph{5}{t04_l01g15.png}}
% 16
{\PTglyph{5}{t04_l01g16.png}}
% 17
{\PTglyph{5}{t04_l01g17.png}}
% 18
{\PTglyph{5}{t04_l01g18.png}}
% 19
{\PTglyph{5}{t04_l01g19.png}}
% 20
{\PTglyph{5}{t04_l01g20.png}}
% 21
{\PTglyph{5}{t04_l01g21.png}}
% 22
{\PTglyph{5}{t04_l01g22.png}}
% 23
{\PTglyph{5}{t04_l01g23.png}}
% 24
{\PTglyph{5}{t04_l01g24.png}}
% 25
{\PTglyph{5}{t04_l01g25.png}}
% 26
{\PTglyph{5}{t04_l01g26.png}}
% 27
{\PTglyph{5}{t04_l01g27.png}}
% 28
{\PTglyph{5}{t04_l01g28.png}}
% 29
{\PTglyph{5}{t04_l02g01.png}}
% 30
{\PTglyph{5}{t04_l02g02.png}}
% 31
{\PTglyph{5}{t04_l02g03.png}}
% 32
{\PTglyph{5}{t04_l02g04.png}}
% 33
{\PTglyph{5}{t04_l02g05.png}}
% 34
{\PTglyph{5}{t04_l02g06.png}}
% 35
{\PTglyph{5}{t04_l02g07.png}}
% 36
{\PTglyph{5}{t04_l02g08.png}}
% 37
{\PTglyph{5}{t04_l02g09.png}}
% 38
{\PTglyph{5}{t04_l02g10.png}}
% 39
{\PTglyph{5}{t04_l02g11.png}}
% 40
{\PTglyph{5}{t04_l02g12.png}}
% 41
{\PTglyph{5}{t04_l02g13.png}}
% 42
{\PTglyph{5}{t04_l02g14.png}}
% 43
{\PTglyph{5}{t04_l02g15.png}}
% 44
{\PTglyph{5}{t04_l02g16.png}}
% 45
{\PTglyph{5}{t04_l02g17.png}}
% 46
{\PTglyph{5}{t04_l02g18.png}}
% 47
{\PTglyph{5}{t04_l02g19.png}}
% 48
{\PTglyph{5}{t04_l02g20.png}}
% 49
{\PTglyph{5}{t04_l02g21.png}}
% 50
{\PTglyph{5}{t04_l02g22.png}}
% 51
{\PTglyph{5}{t04_l02g23.png}}
% 52
{\PTglyph{5}{t04_l02g24.png}}
% 53
{\PTglyph{5}{t04_l02g25.png}}
% 54
{\PTglyph{5}{t04_l02g26.png}}
% 55
{\PTglyph{5}{t04_l02g27.png}}
% 56
{\PTglyph{5}{t04_l02g28.png}}
% 57
{\PTglyph{5}{t04_l02g29.png}}
% 58
{\PTglyph{5}{t04_l02g30.png}}
% 59
{\PTglyph{5}{t04_l02g31.png}}
% 60
{\PTglyph{5}{t04_l02g32.png}}
% 61
{\PTglyph{5}{t04_l02g33.png}}
% 62
{\PTglyph{5}{t04_l02g34.png}}
% 63
{\PTglyph{5}{t04_l02g35.png}}
% 64
{\PTglyph{5}{t04_l02g36.png}}
% 65
{\PTglyph{5}{t04_l02g37.png}}
% 66
{\PTglyph{5}{t04_l02g38.png}}
% 67
{\PTglyph{5}{t04_l02g39.png}}
% 68
{\PTglyph{5}{t04_l02g40.png}}
% 69
{\PTglyph{5}{t04_l02g41.png}}
% 70
{\PTglyph{5}{t04_l02g42.png}}
% 71
{\PTglyph{5}{t04_l03g01.png}}
% 72
{\PTglyph{5}{t04_l03g02.png}}
% 73
{\PTglyph{5}{t04_l03g03.png}}
% 74
{\PTglyph{5}{t04_l03g04.png}}
% 75
{\PTglyph{5}{t04_l03g05.png}}
% 76
{\PTglyph{5}{t04_l03g06.png}}
% 77
{\PTglyph{5}{t04_l03g07.png}}
% 78
{\PTglyph{5}{t04_l03g08.png}}
% 79
{\PTglyph{5}{t04_l03g09.png}}
% 80
{\PTglyph{5}{t04_l03g10.png}}
% 81
{\PTglyph{5}{t04_l03g11.png}}
% 82
{\PTglyph{5}{t04_l03g12.png}}
% 83
{\PTglyph{5}{t04_l03g13.png}}
% 84
{\PTglyph{5}{t04_l03g14.png}}
% 85
{\PTglyph{5}{t04_l03g15.png}}
% 86
{\PTglyph{5}{t04_l03g16.png}}
% 87
{\PTglyph{5}{t04_l03g17.png}}
% 88
{\PTglyph{5}{t04_l03g18.png}}
% 89
{\PTglyph{5}{t04_l03g19.png}}
% 90
{\PTglyph{5}{t04_l03g20.png}}
% 91
{\PTglyph{5}{t04_l03g21.png}}
% 92
{\PTglyph{5}{t04_l03g22.png}}
% 93
{\PTglyph{5}{t04_l03g23.png}}
% 94
{\PTglyph{5}{t04_l04g01.png}}
% 95
{\PTglyph{5}{t04_l04g02.png}}
% 96
{\PTglyph{5}{t04_l04g03.png}}
% 97
{\PTglyph{5}{t04_l04g04.png}}
% 98
{\PTglyph{5}{t04_l04g05.png}}
% 99
{\PTglyph{5}{t04_l04g06.png}}
% 100
{\PTglyph{5}{t04_l04g07.png}}
% 101
{\PTglyph{5}{t04_l04g08.png}}
% 102
{\PTglyph{5}{t04_l04g09.png}}
% 103
{\PTglyph{5}{t04_l04g10.png}}
% 104
{\PTglyph{5}{t04_l04g11.png}}
% 105
{\PTglyph{5}{t04_l04g12.png}}
% 106
{\PTglyph{5}{t04_l04g13.png}}
% 107
{\PTglyph{5}{t04_l04g14.png}}
% 108
{\PTglyph{5}{t04_l04g15.png}}
% 109
{\PTglyph{5}{t04_l04g16.png}}
% 110
{\PTglyph{5}{t04_l04g17.png}}
% 111
{\PTglyph{5}{t04_l04g18.png}}
% 112
{\PTglyph{5}{t04_l04g19.png}}
% 113
{\PTglyph{5}{t04_l04g20.png}}
% 114
{\PTglyph{5}{t04_l04g21.png}}
% 115
{\PTglyph{5}{t04_l04g22.png}}
% 116
{\PTglyph{5}{t04_l04g23.png}}
% 117
{\PTglyph{5}{t04_l04g24.png}}
% 118
{\PTglyph{5}{t04_l04g25.png}}
% 119
{\PTglyph{5}{t04_l04g26.png}}
% 120
{\PTglyph{5}{t04_l04g27.png}}
% 121
{\PTglyph{5}{t04_l04g28.png}}
% 122
{\PTglyph{5}{t04_l04g29.png}}
% 123
{\PTglyph{5}{t04_l05g01.png}}
% 124
{\PTglyph{5}{t04_l05g02.png}}
% 125
{\PTglyph{5}{t04_l05g03.png}}
% 126
{\PTglyph{5}{t04_l05g04.png}}
% 127
{\PTglyph{5}{t04_l05g05.png}}
% 128
{\PTglyph{5}{t04_l05g06.png}}
% 129
{\PTglyph{5}{t04_l05g07.png}}
% 130
{\PTglyph{5}{t04_l05g08.png}}
% 131
{\PTglyph{5}{t04_l05g09.png}}
% 132
{\PTglyph{5}{t04_l05g10.png}}
% 133
{\PTglyph{5}{t04_l05g11.png}}
% 134
{\PTglyph{5}{t04_l05g12.png}}
% 135
{\PTglyph{5}{t04_l05g13.png}}
% 136
{\PTglyph{5}{t04_l05g14.png}}
% 137
{\PTglyph{5}{t04_l05g15.png}}
% 138
{\PTglyph{5}{t04_l05g16.png}}
% 139
{\PTglyph{5}{t04_l05g17.png}}
% 140
{\PTglyph{5}{t04_l05g18.png}}
% 141
{\PTglyph{5}{t04_l05g19.png}}
% 142
{\PTglyph{5}{t04_l05g20.png}}
% 143
{\PTglyph{5}{t04_l05g21.png}}
% 144
{\PTglyph{5}{t04_l05g22.png}}
% 145
{\PTglyph{5}{t04_l05g23.png}}
% 146
{\PTglyph{5}{t04_l05g24.png}}
% 147
{\PTglyph{5}{t04_l05g25.png}}
% 148
{\PTglyph{5}{t04_l05g26.png}}
% 149
{\PTglyph{5}{t04_l05g27.png}}
% 150
{\PTglyph{5}{t04_l05g28.png}}
% 151
{\PTglyph{5}{t04_l05g29.png}}
% 152
{\PTglyph{5}{t04_l05g30.png}}
% 153
{\PTglyph{5}{t04_l05g31.png}}
% 154
{\PTglyph{5}{t04_l05g32.png}}
% 155
{\PTglyph{5}{t04_l05g33.png}}
% 156
{\PTglyph{5}{t04_l05g34.png}}
% 157
{\PTglyph{5}{t04_l05g35.png}}
% 158
{\PTglyph{5}{t04_l05g36.png}}
//
%%% Local Variables:
%%% mode: latex
%%% TeX-engine: luatex
%%% TeX-master: shared
%%% End:

%//
%\glpismo
\glpismo
% 1
{\PTglyphid{Au-02_0101}}
% 2
{\PTglyphid{Au-02_0102}}
% 3
{\PTglyphid{Au-02_0103}}
% 4
{\PTglyphid{Au-02_0104}}
% 5
{\PTglyphid{Au-02_0105}}
% 6
{\PTglyphid{Au-02_0106}}
% 7
{\PTglyphid{Au-02_0107}}
% 8
{\PTglyphid{Au-02_0108}}
% 9
{\PTglyphid{Au-02_0109}}
% 10
{\PTglyphid{Au-02_0110}}
% 11
{\PTglyphid{Au-02_0111}}
% 12
{\PTglyphid{Au-02_0112}}
% 13
{\PTglyphid{Au-02_0113}}
% 14
{\PTglyphid{Au-02_0114}}
% 15
{\PTglyphid{Au-02_0115}}
% 16
{\PTglyphid{Au-02_0116}}
% 17
{\PTglyphid{Au-02_0117}}
% 18
{\PTglyphid{Au-02_0118}}
% 19
{\PTglyphid{Au-02_0119}}
% 20
{\PTglyphid{Au-02_0120}}
% 21
{\PTglyphid{Au-02_0121}}
% 22
{\PTglyphid{Au-02_0122}}
% 23
{\PTglyphid{Au-02_0123}}
% 24
{\PTglyphid{Au-02_0124}}
% 25
{\PTglyphid{Au-02_0125}}
% 26
{\PTglyphid{Au-02_0126}}
% 27
{\PTglyphid{Au-02_0127}}
% 28
{\PTglyphid{Au-02_0128}}
% 29
{\PTglyphid{Au-02_0201}}
% 30
{\PTglyphid{Au-02_0202}}
% 31
{\PTglyphid{Au-02_0203}}
% 32
{\PTglyphid{Au-02_0204}}
% 33
{\PTglyphid{Au-02_0205}}
% 34
{\PTglyphid{Au-02_0206}}
% 35
{\PTglyphid{Au-02_0207}}
% 36
{\PTglyphid{Au-02_0208}}
% 37
{\PTglyphid{Au-02_0209}}
% 38
{\PTglyphid{Au-02_0210}}
% 39
{\PTglyphid{Au-02_0211}}
% 40
{\PTglyphid{Au-02_0212}}
% 41
{\PTglyphid{Au-02_0213}}
% 42
{\PTglyphid{Au-02_0214}}
% 43
{\PTglyphid{Au-02_0215}}
% 44
{\PTglyphid{Au-02_0216}}
% 45
{\PTglyphid{Au-02_0217}}
% 46
{\PTglyphid{Au-02_0218}}
% 47
{\PTglyphid{Au-02_0219}}
% 48
{\PTglyphid{Au-02_0220}}
% 49
{\PTglyphid{Au-02_0221}}
% 50
{\PTglyphid{Au-02_0222}}
% 51
{\PTglyphid{Au-02_0223}}
% 52
{\PTglyphid{Au-02_0224}}
% 53
{\PTglyphid{Au-02_0225}}
% 54
{\PTglyphid{Au-02_0226}}
% 55
{\PTglyphid{Au-02_0227}}
% 56
{\PTglyphid{Au-02_0228}}
% 57
{\PTglyphid{Au-02_0229}}
% 58
{\PTglyphid{Au-02_0230}}
% 59
{\PTglyphid{Au-02_0231}}
% 60
{\PTglyphid{Au-02_0232}}
% 61
{\PTglyphid{Au-02_0233}}
% 62
{\PTglyphid{Au-02_0234}}
% 63
{\PTglyphid{Au-02_0235}}
% 64
{\PTglyphid{Au-02_0236}}
% 65
{\PTglyphid{Au-02_0237}}
% 66
{\PTglyphid{Au-02_0238}}
% 67
{\PTglyphid{Au-02_0239}}
% 68
{\PTglyphid{Au-02_0240}}
% 69
{\PTglyphid{Au-02_0241}}
% 70
{\PTglyphid{Au-02_0242}}
% 71
{\PTglyphid{Au-02_0301}}
% 72
{\PTglyphid{Au-02_0302}}
% 73
{\PTglyphid{Au-02_0303}}
% 74
{\PTglyphid{Au-02_0304}}
% 75
{\PTglyphid{Au-02_0305}}
% 76
{\PTglyphid{Au-02_0306}}
% 77
{\PTglyphid{Au-02_0307}}
% 78
{\PTglyphid{Au-02_0308}}
% 79
{\PTglyphid{Au-02_0309}}
% 80
{\PTglyphid{Au-02_0310}}
% 81
{\PTglyphid{Au-02_0311}}
% 82
{\PTglyphid{Au-02_0312}}
% 83
{\PTglyphid{Au-02_0313}}
% 84
{\PTglyphid{Au-02_0314}}
% 85
{\PTglyphid{Au-02_0315}}
% 86
{\PTglyphid{Au-02_0316}}
% 87
{\PTglyphid{Au-02_0317}}
% 88
{\PTglyphid{Au-02_0318}}
% 89
{\PTglyphid{Au-02_0319}}
% 90
{\PTglyphid{Au-02_0320}}
% 91
{\PTglyphid{Au-02_0321}}
% 92
{\PTglyphid{Au-02_0322}}
% 93
{\PTglyphid{Au-02_0323}}
% 94
{\PTglyphid{Au-02_0401}}
% 95
{\PTglyphid{Au-02_0402}}
% 96
{\PTglyphid{Au-02_0403}}
% 97
{\PTglyphid{Au-02_0404}}
% 98
{\PTglyphid{Au-02_0405}}
% 99
{\PTglyphid{Au-02_0406}}
% 100
{\PTglyphid{Au-02_0407}}
% 101
{\PTglyphid{Au-02_0408}}
% 102
{\PTglyphid{Au-02_0409}}
% 103
{\PTglyphid{Au-02_0410}}
% 104
{\PTglyphid{Au-02_0411}}
% 105
{\PTglyphid{Au-02_0412}}
% 106
{\PTglyphid{Au-02_0413}}
% 107
{\PTglyphid{Au-02_0414}}
% 108
{\PTglyphid{Au-02_0415}}
% 109
{\PTglyphid{Au-02_0416}}
% 110
{\PTglyphid{Au-02_0417}}
% 111
{\PTglyphid{Au-02_0418}}
% 112
{\PTglyphid{Au-02_0419}}
% 113
{\PTglyphid{Au-02_0420}}
% 114
{\PTglyphid{Au-02_0421}}
% 115
{\PTglyphid{Au-02_0422}}
% 116
{\PTglyphid{Au-02_0423}}
% 117
{\PTglyphid{Au-02_0424}}
% 118
{\PTglyphid{Au-02_0425}}
% 119
{\PTglyphid{Au-02_0426}}
% 120
{\PTglyphid{Au-02_0427}}
% 121
{\PTglyphid{Au-02_0428}}
% 122
{\PTglyphid{Au-02_0429}}
% 123
{\PTglyphid{Au-02_0501}}
% 124
{\PTglyphid{Au-02_0502}}
% 125
{\PTglyphid{Au-02_0503}}
% 126
{\PTglyphid{Au-02_0504}}
% 127
{\PTglyphid{Au-02_0505}}
% 128
{\PTglyphid{Au-02_0506}}
% 129
{\PTglyphid{Au-02_0507}}
% 130
{\PTglyphid{Au-02_0508}}
% 131
{\PTglyphid{Au-02_0509}}
% 132
{\PTglyphid{Au-02_0510}}
% 133
{\PTglyphid{Au-02_0511}}
% 134
{\PTglyphid{Au-02_0512}}
% 135
{\PTglyphid{Au-02_0513}}
% 136
{\PTglyphid{Au-02_0514}}
% 137
{\PTglyphid{Au-02_0515}}
% 138
{\PTglyphid{Au-02_0516}}
% 139
{\PTglyphid{Au-02_0517}}
% 140
{\PTglyphid{Au-02_0518}}
% 141
{\PTglyphid{Au-02_0519}}
% 142
{\PTglyphid{Au-02_0520}}
% 143
{\PTglyphid{Au-02_0521}}
% 144
{\PTglyphid{Au-02_0522}}
% 145
{\PTglyphid{Au-02_0523}}
% 146
{\PTglyphid{Au-02_0524}}
% 147
{\PTglyphid{Au-02_0525}}
% 148
{\PTglyphid{Au-02_0526}}
% 149
{\PTglyphid{Au-02_0527}}
% 150
{\PTglyphid{Au-02_0528}}
% 151
{\PTglyphid{Au-02_0529}}
% 152
{\PTglyphid{Au-02_0530}}
% 153
{\PTglyphid{Au-02_0531}}
% 154
{\PTglyphid{Au-02_0532}}
% 155
{\PTglyphid{Au-02_0533}}
% 156
{\PTglyphid{Au-02_0534}}
% 157
{\PTglyphid{Au-02_0535}}
% 158
{\PTglyphid{Au-02_0536}}
//
\endgl \xe
%%% Local Variables:
%%% mode: latex
%%% TeX-engine: luatex
%%% TeX-master: shared
%%% End:

% //
%\endgl \xe

%   *[TESTAMENTUM NOVUM. Evangelium secundum Matthaeum. Trad. polon. Stanislaus Murzynowski]: Ewangelia św. Mateusza. 4�. K. 104.
% (Arkusze A—B A—B A—Y). Druk czarno-czerwony. — Estr. XIII 26. Wierzb. 135. Arkusze
% Z aa—kk zob. poz. 4.
% W egz. B. Nar. XVI Qu. 6471 w pierwszym marginalium na lewym
% marginesie w. 4, zamiast majuskuły A wydrukowano minuskułę a.
% Pisma 1—3, 5—7, 10—12. — Cyfry 1—3, 5. — Przerywnik I, 5. —
% Inicjały 1, 3—8, 28.
% Warsz. B. U. 28.2.4.4. z notatkami J. Maleckiego. [4�

\newpage
%%%%%%%%%%%%%%%%%%%%%%%%%%%%%%%%%%%%%%%%%%%%%%%%%%%%%%%%%%%%%%%%%%%%%%%%%%%%%%
% Tab. 05 Aleksander Augezdecki pismo 2a
%%%%%%%%%%%%%%%%%%%%%%%%%%%%%%%%%%%%%%%%%%%%%%%%%%%%%%%%%%%%%%%%%%%%%%%%%%%%%%

 % Author "Paulina Buchwald-Pelcowa"
% Title "Aleksander Augezdecki 1549-1561"
% Editor "Alodia Kawecka Gryczowa"
% Series "Polonia Typographica Saeculi Sedecimi: zbiór podobizn zasobu drukarskiego tłoczni polskich XVI stulecia"
% Fascicule "VIII"
% Publisher "Zakład Narodowy imienia Ossolińskich — Wydawnictwo"
% Addres "Kraków  Wrocław Warszawa"
% Year "1972"
% Note "2 Pismo tekstowe, szwabacha M⁸¹. Stopień 20 ww. = 86—87 mm (cycero). — Tabl. 405, 406. [405]"
% Note1 "Character set table prepared by Paulina Buchwald-Pelcowa"
% Note2 "Scan (prepared by Biblioteka Uniwersytecka w Warszawie from their own copy) converted to DjVu with didjvu by Janusz S. Bień"
% URL "https://github.com/jsbien/early_fonts_inventory/"
% .

\pismoPL{Aleksander Augezdecki 2. Pisma tekstowe, szwabacha
  M⁸¹. Stopień 20 ww. = 86—87 mm (cycero). — Tabl. 405, 406. [litery
  ze znakami diakrytycznymi czeskimi]}

\pismoEN{Aleksander Augezdecki 2. Schwabacher text script. Typeface
  M⁸¹. Type size 20 lines = 86—87 mm (cicero) - Plate 405,
  406. [letters with Czech diacritical marks]}

\medskip

\plate{406}{VIII}{1972}

The plate    prepared by Paulina Buchwald-Pelcowa.\\
The font table    prepared by Paulina Buchwald-Pelcowa.\\


\bigskip

\exampleBib{VIII:25}

\bigskip
\exampleDesc{PIESNĚ Chwal Bożských. Szamotuly, Aleksander Augezdecki, [25 I 1560—] 7 VI 1561. 2°. War. A.}

\medskip
\examplePage{\textit{Karta *₄a.}}

  \bigskip
\exampleLib{Biblioteka Czartoryskich. Kraków.}

\bigskip
\exampleRef{\textit{Estreicher XIX 91. Knihopis 12860.}}

\bigskip
\exampleDig{\url{https://cyfrowe.mnk.pl/dlibra/publication/13639/}, page 10.}

\medskip

    \examplePL{Pismo 2: tekst i zestaw liter ze znakami diakrytycznymi czeskimi.}

    \medskip

    \exampleEN{Font 2. The text and the table including letters with Czech diacritical marks.}


\bigskip

    \fontID{Au-02a}{05}

    \fontstat{15}

\bigskip


% \exdisplay \bg \gla
\exdisplay \bg \gla
% 1
{\PTglyph{5}{t05_l01g01.png}}
% 2
{\PTglyph{5}{t05_l01g02.png}}
% 3
{\PTglyph{5}{t05_l01g03.png}}
% 4
{\PTglyph{5}{t05_l01g04.png}}
% 5
{\PTglyph{5}{t05_l01g05.png}}
% 6
{\PTglyph{5}{t05_l01g06.png}}
% 7
{\PTglyph{5}{t05_l01g07.png}}
% 8
{\PTglyph{5}{t05_l01g08.png}}
% 9
{\PTglyph{5}{t05_l01g09.png}}
% 10
{\PTglyph{5}{t05_l01g10.png}}
% 11
{\PTglyph{5}{t05_l01g11.png}}
% 12
{\PTglyph{5}{t05_l01g12.png}}
% 13
{\PTglyph{5}{t05_l01g13.png}}
% 14
{\PTglyph{5}{t05_l01g14.png}}
% 15
{\PTglyph{5}{t05_l01g15.png}}
//
%%% Local Variables:
%%% mode: latex
%%% TeX-engine: luatex
%%% TeX-master: shared
%%% End:

%//
%\glpismo
\glpismo
% 1
{\PTglyphid{Au-02a0101}}
% 2
{\PTglyphid{Au-02a0102}}
% 3
{\PTglyphid{Au-02a0103}}
% 4
{\PTglyphid{Au-02a0104}}
% 5
{\PTglyphid{Au-02a0105}}
% 6
{\PTglyphid{Au-02a0106}}
% 7
{\PTglyphid{Au-02a0107}}
% 8
{\PTglyphid{Au-02a0108}}
% 9
{\PTglyphid{Au-02a0109}}
% 10
{\PTglyphid{Au-02a0110}}
% 11
{\PTglyphid{Au-02a0111}}
% 12
{\PTglyphid{Au-02a0112}}
% 13
{\PTglyphid{Au-02a0113}}
% 14
{\PTglyphid{Au-02a0114}}
% 15
{\PTglyphid{Au-02a0115}}
//
\endgl \xe
%%% Local Variables:
%%% mode: latex
%%% TeX-engine: luatex
%%% TeX-master: shared
%%% End:

% //
%\endgl \xe


\newpage
%%%%%%%%%%%%%%%%%%%%%%%%%%%%%%%%%%%%%%%%%%%%%%%%%%%%%%%%%%%%%%%%%%%%%%%%%%%%%%
% Tab. 06,Augezdecki-03_PT08_407.djvu,Augezdecki,03,08,407
%%%%%%%%%%%%%%%%%%%%%%%%%%%%%%%%%%%%%%%%%%%%%%%%%%%%%%%%%%%%%%%%%%%%%%%%%%%%%%

% Note "3. Pismo komentarzowe, szwabacha M⁸¹ + M⁸⁷. Stopień 20 ww. = 67 mm (garmond). — Tabl. 407."
% Note1 "Character set table prepared by Maria Błońska"

\pismoPL{Aleksander Augezdecki 3. Pismo komentarzowe, szwabacha M⁸¹ + M⁸⁷. Stopień 20 ww. = 67 mm (garmond). — Tabl. 407.}

\pismoEN{Aleksander Augezdecki 3. Schwabacher comment script. Typeface M⁸¹ + M⁸⁷. Type size 20 lines = 67 mm (garmond). - Plate 407.}

\medskip

\plate{407}{VIII}{1972}

The plate    prepared by Paulina Buchwald-Pelcowa.\\
The font table prepared by Paulina Buchwald-Pelcowa and Maria Błońska.

\bigskip

 \exampleBib{VIII:4$^a$}

\bigskip
\exampleDesc{[TESTAMENTUM NOVUM. Evangelium secundum Matthaeum. Trad. polon. Stanislaus Murzynowski]:
Ewangelia Św. Mateusza. Królewiec, [Aleksander Augezdecki], 1551. 4°.}

\medskip
\examplePage{\textit{Karta B₂a.}}

  \bigskip
\exampleLib{Biblioteka Czartoryskich. Kraków.}

\bigskip
\exampleRef{\textit{Estreicher XIII 26. Wierzbowski 135.}}

\bigskip
\exampleDig{\url{https://crispa.uw.edu.pl/object/files/318618/display/Default} page 18}

\medskip

    \examplePL{Pismo 3: tekst i zestaw.}

    \medskip

    \exampleEN{Font 3. The text and the font table}


\bigskip

    \fontID{Au-03}{06}

    \fontstat{112}

\bigskip


% \exdisplay \bg \gla
 \exdisplay \bg \gla
% 1
{\PTglyph{5}{t06_l01g01.png}}
% 2
{\PTglyph{5}{t06_l01g02.png}}
% 3
{\PTglyph{5}{t06_l01g03.png}}
% 4
{\PTglyph{5}{t06_l01g04.png}}
% 5
{\PTglyph{5}{t06_l01g05.png}}
% 6
{\PTglyph{5}{t06_l01g06.png}}
% 7
{\PTglyph{5}{t06_l01g07.png}}
% 8
{\PTglyph{5}{t06_l01g08.png}}
% 9
{\PTglyph{5}{t06_l01g09.png}}
% 10
{\PTglyph{5}{t06_l01g10.png}}
% 11
{\PTglyph{5}{t06_l01g11.png}}
% 12
{\PTglyph{5}{t06_l01g12.png}}
% 13
{\PTglyph{5}{t06_l01g13.png}}
% 14
{\PTglyph{5}{t06_l01g14.png}}
% 15
{\PTglyph{5}{t06_l01g15.png}}
% 16
{\PTglyph{5}{t06_l01g16.png}}
% 17
{\PTglyph{5}{t06_l02g01.png}}
% 18
{\PTglyph{5}{t06_l02g02.png}}
% 19
{\PTglyph{5}{t06_l02g03.png}}
% 20
{\PTglyph{5}{t06_l02g04.png}}
% 21
{\PTglyph{5}{t06_l02g05.png}}
% 22
{\PTglyph{5}{t06_l02g06.png}}
% 23
{\PTglyph{5}{t06_l02g07.png}}
% 24
{\PTglyph{5}{t06_l02g08.png}}
% 25
{\PTglyph{5}{t06_l02g09.png}}
% 26
{\PTglyph{5}{t06_l02g10.png}}
% 27
{\PTglyph{5}{t06_l02g11.png}}
% 28
{\PTglyph{5}{t06_l02g12.png}}
% 29
{\PTglyph{5}{t06_l02g13.png}}
% 30
{\PTglyph{5}{t06_l02g14.png}}
% 31
{\PTglyph{5}{t06_l02g15.png}}
% 32
{\PTglyph{5}{t06_l02g16.png}}
% 33
{\PTglyph{5}{t06_l02g17.png}}
% 34
{\PTglyph{5}{t06_l02g18.png}}
% 35
{\PTglyph{5}{t06_l03g01.png}}
% 36
{\PTglyph{5}{t06_l03g02.png}}
% 37
{\PTglyph{5}{t06_l03g03.png}}
% 38
{\PTglyph{5}{t06_l03g04.png}}
% 39
{\PTglyph{5}{t06_l03g05.png}}
% 40
{\PTglyph{5}{t06_l03g06.png}}
% 41
{\PTglyph{5}{t06_l03g07.png}}
% 42
{\PTglyph{5}{t06_l03g08.png}}
% 43
{\PTglyph{5}{t06_l03g09.png}}
% 44
{\PTglyph{5}{t06_l03g10.png}}
% 45
{\PTglyph{5}{t06_l03g11.png}}
% 46
{\PTglyph{5}{t06_l03g12.png}}
% 47
{\PTglyph{5}{t06_l03g13.png}}
% 48
{\PTglyph{5}{t06_l03g14.png}}
% 49
{\PTglyph{5}{t06_l03g15.png}}
% 50
{\PTglyph{5}{t06_l03g16.png}}
% 51
{\PTglyph{5}{t06_l03g17.png}}
% 52
{\PTglyph{5}{t06_l03g18.png}}
% 53
{\PTglyph{5}{t06_l03g19.png}}
% 54
{\PTglyph{5}{t06_l03g20.png}}
% 55
{\PTglyph{5}{t06_l03g21.png}}
% 56
{\PTglyph{5}{t06_l03g22.png}}
% 57
{\PTglyph{5}{t06_l03g23.png}}
% 58
{\PTglyph{5}{t06_l03g24.png}}
% 59
{\PTglyph{5}{t06_l03g25.png}}
% 60
{\PTglyph{5}{t06_l03g26.png}}
% 61
{\PTglyph{5}{t06_l03g27.png}}
% 62
{\PTglyph{5}{t06_l03g28.png}}
% 63
{\PTglyph{5}{t06_l03g29.png}}
% 64
{\PTglyph{5}{t06_l03g30.png}}
% 65
{\PTglyph{5}{t06_l04g01.png}}
% 66
{\PTglyph{5}{t06_l04g02.png}}
% 67
{\PTglyph{5}{t06_l04g03.png}}
% 68
{\PTglyph{5}{t06_l04g04.png}}
% 69
{\PTglyph{5}{t06_l04g05.png}}
% 70
{\PTglyph{5}{t06_l04g06.png}}
% 71
{\PTglyph{5}{t06_l04g07.png}}
% 72
{\PTglyph{5}{t06_l04g08.png}}
% 73
{\PTglyph{5}{t06_l04g09.png}}
% 74
{\PTglyph{5}{t06_l04g10.png}}
% 75
{\PTglyph{5}{t06_l04g11.png}}
% 76
{\PTglyph{5}{t06_l04g12.png}}
% 77
{\PTglyph{5}{t06_l04g13.png}}
% 78
{\PTglyph{5}{t06_l04g14.png}}
% 79
{\PTglyph{5}{t06_l04g15.png}}
% 80
{\PTglyph{5}{t06_l04g16.png}}
% 81
{\PTglyph{5}{t06_l04g17.png}}
% 82
{\PTglyph{5}{t06_l04g18.png}}
% 83
{\PTglyph{5}{t06_l04g19.png}}
% 84
{\PTglyph{5}{t06_l04g20.png}}
% 85
{\PTglyph{5}{t06_l04g21.png}}
% 86
{\PTglyph{5}{t06_l04g22.png}}
% 87
{\PTglyph{5}{t06_l04g23.png}}
% 88
{\PTglyph{5}{t06_l04g24.png}}
% 89
{\PTglyph{5}{t06_l04g25.png}}
% 90
{\PTglyph{5}{t06_l04g26.png}}
% 91
{\PTglyph{5}{t06_l04g27.png}}
% 92
{\PTglyph{5}{t06_l04g28.png}}
% 93
{\PTglyph{5}{t06_l04g29.png}}
% 94
{\PTglyph{5}{t06_l04g30.png}}
% 95
{\PTglyph{5}{t06_l05g01.png}}
% 96
{\PTglyph{5}{t06_l05g02.png}}
% 97
{\PTglyph{5}{t06_l05g03.png}}
% 98
{\PTglyph{5}{t06_l05g04.png}}
% 99
{\PTglyph{5}{t06_l05g05.png}}
% 100
{\PTglyph{5}{t06_l05g06.png}}
% 101
{\PTglyph{5}{t06_l05g07.png}}
% 102
{\PTglyph{5}{t06_l05g08.png}}
% 103
{\PTglyph{5}{t06_l05g09.png}}
% 104
{\PTglyph{5}{t06_l05g10.png}}
% 105
{\PTglyph{5}{t06_l05g11.png}}
% 106
{\PTglyph{5}{t06_l05g12.png}}
% 107
{\PTglyph{5}{t06_l05g13.png}}
% 108
{\PTglyph{5}{t06_l05g14.png}}
% 109
{\PTglyph{5}{t06_l05g15.png}}
% 110
{\PTglyph{5}{t06_l05g16.png}}
% 111
{\PTglyph{5}{t06_l05g17.png}}
% 112
{\PTglyph{5}{t06_l05g18.png}}
//
%%% Local Variables:
%%% mode: latex
%%% TeX-engine: luatex
%%% TeX-master: shared
%%% End:

%//
%\glpismo
 \glpismo
% 1
{\PTglyphid{Au-03_0101}}
% 2
{\PTglyphid{Au-03_0102}}
% 3
{\PTglyphid{Au-03_0103}}
% 4
{\PTglyphid{Au-03_0104}}
% 5
{\PTglyphid{Au-03_0105}}
% 6
{\PTglyphid{Au-03_0106}}
% 7
{\PTglyphid{Au-03_0107}}
% 8
{\PTglyphid{Au-03_0108}}
% 9
{\PTglyphid{Au-03_0109}}
% 10
{\PTglyphid{Au-03_0110}}
% 11
{\PTglyphid{Au-03_0111}}
% 12
{\PTglyphid{Au-03_0112}}
% 13
{\PTglyphid{Au-03_0113}}
% 14
{\PTglyphid{Au-03_0114}}
% 15
{\PTglyphid{Au-03_0115}}
% 16
{\PTglyphid{Au-03_0116}}
% 17
{\PTglyphid{Au-03_0201}}
% 18
{\PTglyphid{Au-03_0202}}
% 19
{\PTglyphid{Au-03_0203}}
% 20
{\PTglyphid{Au-03_0204}}
% 21
{\PTglyphid{Au-03_0205}}
% 22
{\PTglyphid{Au-03_0206}}
% 23
{\PTglyphid{Au-03_0207}}
% 24
{\PTglyphid{Au-03_0208}}
% 25
{\PTglyphid{Au-03_0209}}
% 26
{\PTglyphid{Au-03_0210}}
% 27
{\PTglyphid{Au-03_0211}}
% 28
{\PTglyphid{Au-03_0212}}
% 29
{\PTglyphid{Au-03_0213}}
% 30
{\PTglyphid{Au-03_0214}}
% 31
{\PTglyphid{Au-03_0215}}
% 32
{\PTglyphid{Au-03_0216}}
% 33
{\PTglyphid{Au-03_0217}}
% 34
{\PTglyphid{Au-03_0218}}
% 35
{\PTglyphid{Au-03_0301}}
% 36
{\PTglyphid{Au-03_0302}}
% 37
{\PTglyphid{Au-03_0303}}
% 38
{\PTglyphid{Au-03_0304}}
% 39
{\PTglyphid{Au-03_0305}}
% 40
{\PTglyphid{Au-03_0306}}
% 41
{\PTglyphid{Au-03_0307}}
% 42
{\PTglyphid{Au-03_0308}}
% 43
{\PTglyphid{Au-03_0309}}
% 44
{\PTglyphid{Au-03_0310}}
% 45
{\PTglyphid{Au-03_0311}}
% 46
{\PTglyphid{Au-03_0312}}
% 47
{\PTglyphid{Au-03_0313}}
% 48
{\PTglyphid{Au-03_0314}}
% 49
{\PTglyphid{Au-03_0315}}
% 50
{\PTglyphid{Au-03_0316}}
% 51
{\PTglyphid{Au-03_0317}}
% 52
{\PTglyphid{Au-03_0318}}
% 53
{\PTglyphid{Au-03_0319}}
% 54
{\PTglyphid{Au-03_0320}}
% 55
{\PTglyphid{Au-03_0321}}
% 56
{\PTglyphid{Au-03_0322}}
% 57
{\PTglyphid{Au-03_0323}}
% 58
{\PTglyphid{Au-03_0324}}
% 59
{\PTglyphid{Au-03_0325}}
% 60
{\PTglyphid{Au-03_0326}}
% 61
{\PTglyphid{Au-03_0327}}
% 62
{\PTglyphid{Au-03_0328}}
% 63
{\PTglyphid{Au-03_0329}}
% 64
{\PTglyphid{Au-03_0330}}
% 65
{\PTglyphid{Au-03_0401}}
% 66
{\PTglyphid{Au-03_0402}}
% 67
{\PTglyphid{Au-03_0403}}
% 68
{\PTglyphid{Au-03_0404}}
% 69
{\PTglyphid{Au-03_0405}}
% 70
{\PTglyphid{Au-03_0406}}
% 71
{\PTglyphid{Au-03_0407}}
% 72
{\PTglyphid{Au-03_0408}}
% 73
{\PTglyphid{Au-03_0409}}
% 74
{\PTglyphid{Au-03_0410}}
% 75
{\PTglyphid{Au-03_0411}}
% 76
{\PTglyphid{Au-03_0412}}
% 77
{\PTglyphid{Au-03_0413}}
% 78
{\PTglyphid{Au-03_0414}}
% 79
{\PTglyphid{Au-03_0415}}
% 80
{\PTglyphid{Au-03_0416}}
% 81
{\PTglyphid{Au-03_0417}}
% 82
{\PTglyphid{Au-03_0418}}
% 83
{\PTglyphid{Au-03_0419}}
% 84
{\PTglyphid{Au-03_0420}}
% 85
{\PTglyphid{Au-03_0421}}
% 86
{\PTglyphid{Au-03_0422}}
% 87
{\PTglyphid{Au-03_0423}}
% 88
{\PTglyphid{Au-03_0424}}
% 89
{\PTglyphid{Au-03_0425}}
% 90
{\PTglyphid{Au-03_0426}}
% 91
{\PTglyphid{Au-03_0427}}
% 92
{\PTglyphid{Au-03_0428}}
% 93
{\PTglyphid{Au-03_0429}}
% 94
{\PTglyphid{Au-03_0430}}
% 95
{\PTglyphid{Au-03_0501}}
% 96
{\PTglyphid{Au-03_0502}}
% 97
{\PTglyphid{Au-03_0503}}
% 98
{\PTglyphid{Au-03_0504}}
% 99
{\PTglyphid{Au-03_0505}}
% 100
{\PTglyphid{Au-03_0506}}
% 101
{\PTglyphid{Au-03_0507}}
% 102
{\PTglyphid{Au-03_0508}}
% 103
{\PTglyphid{Au-03_0509}}
% 104
{\PTglyphid{Au-03_0510}}
% 105
{\PTglyphid{Au-03_0511}}
% 106
{\PTglyphid{Au-03_0512}}
% 107
{\PTglyphid{Au-03_0513}}
% 108
{\PTglyphid{Au-03_0514}}
% 109
{\PTglyphid{Au-03_0515}}
% 110
{\PTglyphid{Au-03_0516}}
% 111
{\PTglyphid{Au-03_0517}}
% 112
{\PTglyphid{Au-03_0518}}
//
\endgl \xe
%%% Local Variables:
%%% mode: latex
%%% TeX-engine: luatex
%%% TeX-master: shared
%%% End:

% //
%\endgl \xe


 
\newpage
%%%%%%%%%%%%%%%%%%%%%%%%%%%%%%%%%%%%%%%%%%%%%%%%%%%%%%%%%%%%%%%%%%%%%%%%%%%%%%
% BRAK w PT tabeli pisma 4!
% Tab. 07,Augezdecki-05_PT08_408.djvu,Augezdecki,05,08,408
%%%%%%%%%%%%%%%%%%%%%%%%%%%%%%%%%%%%%%%%%%%%%%%%%%%%%%%%%%%%%%%%%%%%%%%%%%%%%%

% Note "5. Pismo nagłówkowe, tekstura M³⁰. Stopień I w. = 9 mm. — Tabl. 408."
% Note1 "Character set table prepared by Paulina Buchwald-Pelcowa"

\pismoPL{Aleksander Augezdecki 5. Pismo nagłówkowe, tekstura M³⁰. Stopień 1 w. = 9 mm. — Tabl. 408. [pierwszy zestaw]}

\pismoEN{Aleksander Augezdecki 5. Display font, Textura M³⁰. Type size 1 line = 9 mm. - Plate 408. [first set]}
% https://www.adfontes.uzh.ch/en/tutorium/schriften-lesen/schriftgeschichte/gotische-minuskeln-textura-und-textualis/
\medskip

\plate{408[1]}{VIII}{1972}

The plate    prepared by Paulina Buchwald-Pelcowa.\\
The font table    prepared by Paulina Buchwald-Pelcowa.\\
\relax[Layout confusing, misinterpretation possible --- JSB]

\bigskip

\fontID{Au-05}{07}

    \fontstat{96}
    % 95???
    
% \exdisplay \bg \gla
 \exdisplay \bg \gla
% 1
{\PTglyph{5}{t07_l01g01.png}}
% 2
{\PTglyph{5}{t07_l01g02.png}}
% 3
{\PTglyph{5}{t07_l01g03.png}}
% 4
{\PTglyph{5}{t07_l01g04.png}}
% 5
{\PTglyph{5}{t07_l01g05.png}}
% 6
{\PTglyph{5}{t07_l01g06.png}}
% 7
{\PTglyph{5}{t07_l01g07.png}}
% 8
{\PTglyph{5}{t07_l01g08.png}}
% 9
{\PTglyph{5}{t07_l01g09.png}}
% 10
{\PTglyph{5}{t07_l01g10.png}}
% 11
{\PTglyph{5}{t07_l01g11.png}}
% 12
{\PTglyph{5}{t07_l01g12.png}}
% 13
{\PTglyph{5}{t07_l01g13.png}}
% 14
{\PTglyph{5}{t07_l01g14.png}}
% 15
{\PTglyph{5}{t07_l01g15.png}}
% 16
{\PTglyph{5}{t07_l01g16.png}}
% 17
{\PTglyph{5}{t07_l01g17.png}}
% 18
{\PTglyph{5}{t07_l01g18.png}}
% 19
{\PTglyph{5}{t07_l01g19.png}}
% 20
{\PTglyph{5}{t07_l01g20.png}}
% 21
{\PTglyph{5}{t07_l01g21.png}}
% 22
{\PTglyph{5}{t07_l01g22.png}}
% 23
{\PTglyph{5}{t07_l01g23.png}}
% 24
{\PTglyph{5}{t07_l01g24.png}}
% 25
{\PTglyph{5}{t07_l01g25.png}}
% 26
{\PTglyph{5}{t07_l02g01.png}}
% 27
{\PTglyph{5}{t07_l02g02.png}}
% 28
{\PTglyph{5}{t07_l02g03.png}}
% 29
{\PTglyph{5}{t07_l02g04.png}}
% 30
{\PTglyph{5}{t07_l03g01.png}}
% 31
{\PTglyph{5}{t07_l03g02.png}}
% 32
{\PTglyph{5}{t07_l03g03.png}}
% 33
{\PTglyph{5}{t07_l03g04.png}}
% 34
{\PTglyph{5}{t07_l03g05.png}}
% 35
{\PTglyph{5}{t07_l03g06.png}}
% 36
{\PTglyph{5}{t07_l03g07.png}}
% 37
{\PTglyph{5}{t07_l03g08.png}}
% 38
{\PTglyph{5}{t07_l03g09.png}}
% 39
{\PTglyph{5}{t07_l03g10.png}}
% 40
{\PTglyph{5}{t07_l03g11.png}}
% 41
{\PTglyph{5}{t07_l03g12.png}}
% 42
{\PTglyph{5}{t07_l03g13.png}}
% 43
{\PTglyph{5}{t07_l03g14.png}}
% 44
{\PTglyph{5}{t07_l03g15.png}}
% 45
{\PTglyph{5}{t07_l03g16.png}}
% 46
{\PTglyph{5}{t07_l03g17.png}}
% 47
{\PTglyph{5}{t07_l03g18.png}}
% 48
{\PTglyph{5}{t07_l03g19.png}}
% 49
{\PTglyph{5}{t07_l03g20.png}}
% 50
{\PTglyph{5}{t07_l03g21.png}}
% 51
{\PTglyph{5}{t07_l03g22.png}}
% 52
{\PTglyph{5}{t07_l03g23.png}}
% 53
{\PTglyph{5}{t07_l03g24.png}}
% 54
{\PTglyph{5}{t07_l03g25.png}}
% 55
{\PTglyph{5}{t07_l03g26.png}}
% 56
{\PTglyph{5}{t07_l03g27.png}}
% 57
{\PTglyph{5}{t07_l03g28.png}}
% 58
{\PTglyph{5}{t07_l03g29.png}}
% 59
{\PTglyph{5}{t07_l03g30.png}}
% 60
{\PTglyph{5}{t07_l03g31.png}}
% 61
{\PTglyph{5}{t07_l03g32.png}}
% 62
{\PTglyph{5}{t07_l03g33.png}}
% 63
{\PTglyph{5}{t07_l03g34.png}}
% 64
{\PTglyph{5}{t07_l03g35.png}}
% 65
{\PTglyph{5}{t07_l03g36.png}}
% 66
{\PTglyph{5}{t07_l03g37.png}}
% 67
{\PTglyph{5}{t07_l03g38.png}}
% 68
{\PTglyph{5}{t07_l03g39.png}}
% 69
{\PTglyph{5}{t07_l03g40.png}}
% 70
{\PTglyph{5}{t07_l03g41.png}}
% 71
{\PTglyph{5}{t07_l04g01.png}}
% 72
{\PTglyph{5}{t07_l04g02.png}}
% 73
{\PTglyph{5}{t07_l04g03.png}}
% 74
{\PTglyph{5}{t07_l04g04.png}}
% 75
{\PTglyph{5}{t07_l04g05.png}}
% 76
{\PTglyph{5}{t07_l04g06.png}}
% 77
{\PTglyph{5}{t07_l04g07.png}}
% 78
{\PTglyph{5}{t07_l04g08.png}}
% 79
{\PTglyph{5}{t07_l05g01.png}}
% 80
{\PTglyph{5}{t07_l05g02.png}}
% 81
{\PTglyph{5}{t07_l05g03.png}}
% 82
{\PTglyph{5}{t07_l05g04.png}}
% 83
{\PTglyph{5}{t07_l05g05.png}}
% 84
{\PTglyph{5}{t07_l05g06.png}}
% 85
{\PTglyph{5}{t07_l05g07.png}}
% 86
{\PTglyph{5}{t07_l05g08.png}}
% 87
{\PTglyph{5}{t07_l05g09.png}}
% 88
{\PTglyph{5}{t07_l05g10.png}}
% 89
{\PTglyph{5}{t07_l05g11.png}}
% 90
{\PTglyph{5}{t07_l05g12.png}}
% 91
{\PTglyph{5}{t07_l05g13.png}}
% 92
{\PTglyph{5}{t07_l05g14.png}}
% 93
{\PTglyph{5}{t07_l05g15.png}}
% 94
{\PTglyph{5}{t07_l05g16.png}}
% 95
{\PTglyph{5}{t07_l05g17.png}}
% 96
{\PTglyph{5}{t07_l05g18.png}}
% 97
{\PTglyph{5}{t07_l05g19.png}}
% 98
{\PTglyph{5}{t07_l05g20.png}}
//
%%% Local Variables:
%%% mode: latex
%%% TeX-engine: luatex
%%% TeX-master: shared
%%% End:

%//
%\glpismo
 \glpismo
% 1
{\PTglyphid{Au-05_0101}}
% 2
{\PTglyphid{Au-05_0102}}
% 3
{\PTglyphid{Au-05_0103}}
% 4
{\PTglyphid{Au-05_0104}}
% 5
{\PTglyphid{Au-05_0105}}
% 6
{\PTglyphid{Au-05_0106}}
% 7
{\PTglyphid{Au-05_0107}}
% 8
{\PTglyphid{Au-05_0108}}
% 9
{\PTglyphid{Au-05_0109}}
% 10
{\PTglyphid{Au-05_0110}}
% 11
{\PTglyphid{Au-05_0111}}
% 12
{\PTglyphid{Au-05_0112}}
% 13
{\PTglyphid{Au-05_0113}}
% 14
{\PTglyphid{Au-05_0114}}
% 15
{\PTglyphid{Au-05_0115}}
% 16
{\PTglyphid{Au-05_0116}}
% 17
{\PTglyphid{Au-05_0117}}
% 18
{\PTglyphid{Au-05_0118}}
% 19
{\PTglyphid{Au-05_0119}}
% 20
{\PTglyphid{Au-05_0120}}
% 21
{\PTglyphid{Au-05_0121}}
% 22
{\PTglyphid{Au-05_0122}}
% 23
{\PTglyphid{Au-05_0123}}
% 24
{\PTglyphid{Au-05_0124}}
% 25
{\PTglyphid{Au-05_0125}}
% 26
{\PTglyphid{Au-05_0201}}
% 27
{\PTglyphid{Au-05_0202}}
% 28
{\PTglyphid{Au-05_0203}}
% 29
{\PTglyphid{Au-05_0204}}
% 30
{\PTglyphid{Au-05_0301}}
% 31
{\PTglyphid{Au-05_0302}}
% 32
{\PTglyphid{Au-05_0303}}
% 33
{\PTglyphid{Au-05_0304}}
% 34
{\PTglyphid{Au-05_0305}}
% 35
{\PTglyphid{Au-05_0306}}
% 36
{\PTglyphid{Au-05_0307}}
% 37
{\PTglyphid{Au-05_0308}}
% 38
{\PTglyphid{Au-05_0309}}
% 39
{\PTglyphid{Au-05_0310}}
% 40
{\PTglyphid{Au-05_0311}}
% 41
{\PTglyphid{Au-05_0312}}
% 42
{\PTglyphid{Au-05_0313}}
% 43
{\PTglyphid{Au-05_0314}}
% 44
{\PTglyphid{Au-05_0315}}
% 45
{\PTglyphid{Au-05_0316}}
% 46
{\PTglyphid{Au-05_0317}}
% 47
{\PTglyphid{Au-05_0318}}
% 48
{\PTglyphid{Au-05_0319}}
% 49
{\PTglyphid{Au-05_0320}}
% 50
{\PTglyphid{Au-05_0321}}
% 51
{\PTglyphid{Au-05_0322}}
% 52
{\PTglyphid{Au-05_0323}}
% 53
{\PTglyphid{Au-05_0324}}
% 54
{\PTglyphid{Au-05_0325}}
% 55
{\PTglyphid{Au-05_0326}}
% 56
{\PTglyphid{Au-05_0327}}
% 57
{\PTglyphid{Au-05_0328}}
% 58
{\PTglyphid{Au-05_0329}}
% 59
{\PTglyphid{Au-05_0330}}
% 60
{\PTglyphid{Au-05_0331}}
% 61
{\PTglyphid{Au-05_0332}}
% 62
{\PTglyphid{Au-05_0333}}
% 63
{\PTglyphid{Au-05_0334}}
% 64
{\PTglyphid{Au-05_0335}}
% 65
{\PTglyphid{Au-05_0336}}
% 66
{\PTglyphid{Au-05_0337}}
% 67
{\PTglyphid{Au-05_0338}}
% 68
{\PTglyphid{Au-05_0339}}
% 69
{\PTglyphid{Au-05_0340}}
% 70
{\PTglyphid{Au-05_0341}}
% 71
{\PTglyphid{Au-05_0401}}
% 72
{\PTglyphid{Au-05_0402}}
% 73
{\PTglyphid{Au-05_0403}}
% 74
{\PTglyphid{Au-05_0404}}
% 75
{\PTglyphid{Au-05_0405}}
% 76
{\PTglyphid{Au-05_0406}}
% 77
{\PTglyphid{Au-05_0407}}
% 78
{\PTglyphid{Au-05_0408}}
% 79
{\PTglyphid{Au-05_0501}}
% 80
{\PTglyphid{Au-05_0502}}
% 81
{\PTglyphid{Au-05_0503}}
% 82
{\PTglyphid{Au-05_0504}}
% 83
{\PTglyphid{Au-05_0505}}
% 84
{\PTglyphid{Au-05_0506}}
% 85
{\PTglyphid{Au-05_0507}}
% 86
{\PTglyphid{Au-05_0508}}
% 87
{\PTglyphid{Au-05_0509}}
% 88
{\PTglyphid{Au-05_0510}}
% 89
{\PTglyphid{Au-05_0511}}
% 90
{\PTglyphid{Au-05_0512}}
% 91
{\PTglyphid{Au-05_0513}}
% 92
{\PTglyphid{Au-05_0514}}
% 93
{\PTglyphid{Au-05_0515}}
% 94
{\PTglyphid{Au-05_0516}}
% 95
{\PTglyphid{Au-05_0517}}
% 96
{\PTglyphid{Au-05_0518}}
% 97
{\PTglyphid{Au-05_0519}}
% 98
{\PTglyphid{Au-05_0520}}
//
\endgl \xe
%%% Local Variables:
%%% mode: latex
%%% TeX-engine: luatex
%%% TeX-master: shared
%%% End:

% //
%\endgl \xe

\newpage
%%%%%%%%%%%%%%%%%%%%%%%%%%%%%%%%%%%%%%%%%%%%%%%%%%%%%%%%%%%%%%%%%%%%%%%%%%%%%%
% Tab. 08,Augezdecki-06_PT08_408
%%%%%%%%%%%%%%%%%%%%%%%%%%%%%%%%%%%%%%%%%%%%%%%%%%%%%%%%%%%%%%%%%%%%%%%%%%%%%%

% Note "6. Pismo nagłówkowe, tekstura M²⁹. Stopień I w. = 7 mm. — Tabl. 408."
% Note1 "Character set table prepared by Paulina Buchwald-Pelcowa"

\pismoPL{Aleksander Augezdecki 6. Pismo nagłówkowe, tekstura M²⁹. Stopień 1 w. = 7 mm. — Tabl. 408. [drugi zestaw]}

\pismoEN{Aleksander Augezdecki 6. Display font, Textura M²⁹. Type size 1 line = 7 mm. - Plate 408. [second set]}
% https://www.adfontes.uzh.ch/en/tutorium/schriften-lesen/schriftgeschichte/gotische-minuskeln-textura-und-textualis/
\medskip

\plate{408[2]}{VIII}{1972}

The plate    prepared by Paulina Buchwald-Pelcowa.\\
The font table    prepared by Paulina Buchwald-Pelcowa.\\
\relax[Layout confusing, misinterpretation possible --- JSB]


\bigskip

\fontID{Au-06}{08}

    \fontstat{93}

% \exdisplay \bg \gla
 \exdisplay \bg \gla
% 1
{\PTglyph{5}{t08_l01g01.png}}
% 2
{\PTglyph{5}{t08_l01g02.png}}
% 3
{\PTglyph{5}{t08_l01g03.png}}
% 4
{\PTglyph{5}{t08_l01g04.png}}
% 5
{\PTglyph{5}{t08_l01g05.png}}
% 6
{\PTglyph{5}{t08_l01g06.png}}
% 7
{\PTglyph{5}{t08_l01g07.png}}
% 8
{\PTglyph{5}{t08_l01g08.png}}
% 9
{\PTglyph{5}{t08_l01g09.png}}
% 10
{\PTglyph{5}{t08_l01g10.png}}
% 11
{\PTglyph{5}{t08_l01g11.png}}
% 12
{\PTglyph{5}{t08_l01g12.png}}
% 13
{\PTglyph{5}{t08_l01g13.png}}
% 14
{\PTglyph{5}{t08_l01g14.png}}
% 15
{\PTglyph{5}{t08_l01g15.png}}
% 16
{\PTglyph{5}{t08_l01g16.png}}
% 17
{\PTglyph{5}{t08_l01g17.png}}
% 18
{\PTglyph{5}{t08_l01g18.png}}
% 19
{\PTglyph{5}{t08_l01g19.png}}
% 20
{\PTglyph{5}{t08_l01g20.png}}
% 21
{\PTglyph{5}{t08_l01g21.png}}
% 22
{\PTglyph{5}{t08_l02g01.png}}
% 23
{\PTglyph{5}{t08_l02g02.png}}
% 24
{\PTglyph{5}{t08_l02g03.png}}
% 25
{\PTglyph{5}{t08_l02g04.png}}
% 26
{\PTglyph{5}{t08_l02g05.png}}
% 27
{\PTglyph{5}{t08_l03g01.png}}
% 28
{\PTglyph{5}{t08_l03g02.png}}
% 29
{\PTglyph{5}{t08_l03g03.png}}
% 30
{\PTglyph{5}{t08_l03g04.png}}
% 31
{\PTglyph{5}{t08_l03g05.png}}
% 32
{\PTglyph{5}{t08_l03g06.png}}
% 33
{\PTglyph{5}{t08_l03g07.png}}
% 34
{\PTglyph{5}{t08_l03g08.png}}
% 35
{\PTglyph{5}{t08_l03g09.png}}
% 36
{\PTglyph{5}{t08_l03g10.png}}
% 37
{\PTglyph{5}{t08_l03g11.png}}
% 38
{\PTglyph{5}{t08_l03g12.png}}
% 39
{\PTglyph{5}{t08_l03g13.png}}
% 40
{\PTglyph{5}{t08_l03g14.png}}
% 41
{\PTglyph{5}{t08_l03g15.png}}
% 42
{\PTglyph{5}{t08_l03g16.png}}
% 43
{\PTglyph{5}{t08_l03g17.png}}
% 44
{\PTglyph{5}{t08_l03g18.png}}
% 45
{\PTglyph{5}{t08_l03g19.png}}
% 46
{\PTglyph{5}{t08_l03g20.png}}
% 47
{\PTglyph{5}{t08_l03g21.png}}
% 48
{\PTglyph{5}{t08_l03g22.png}}
% 49
{\PTglyph{5}{t08_l03g23.png}}
% 50
{\PTglyph{5}{t08_l03g24.png}}
% 51
{\PTglyph{5}{t08_l03g25.png}}
% 52
{\PTglyph{5}{t08_l03g26.png}}
% 53
{\PTglyph{5}{t08_l03g27.png}}
% 54
{\PTglyph{5}{t08_l03g28.png}}
% 55
{\PTglyph{5}{t08_l03g29.png}}
% 56
{\PTglyph{5}{t08_l03g30.png}}
% 57
{\PTglyph{5}{t08_l03g31.png}}
% 58
{\PTglyph{5}{t08_l03g32.png}}
% 59
{\PTglyph{5}{t08_l03g33.png}}
% 60
{\PTglyph{5}{t08_l03g34.png}}
% 61
{\PTglyph{5}{t08_l03g35.png}}
% 62
{\PTglyph{5}{t08_l03g36.png}}
% 63
{\PTglyph{5}{t08_l03g37.png}}
% 64
{\PTglyph{5}{t08_l04g01.png}}
% 65
{\PTglyph{5}{t08_l04g02.png}}
% 66
{\PTglyph{5}{t08_l04g03.png}}
% 67
{\PTglyph{5}{t08_l04g04.png}}
% 68
{\PTglyph{5}{t08_l04g05.png}}
% 69
{\PTglyph{5}{t08_l04g06.png}}
% 70
{\PTglyph{5}{t08_l04g07.png}}
% 71
{\PTglyph{5}{t08_l04g08.png}}
% 72
{\PTglyph{5}{t08_l04g09.png}}
% 73
{\PTglyph{5}{t08_l04g10.png}}
% 74
{\PTglyph{5}{t08_l04g11.png}}
% 75
{\PTglyph{5}{t08_l04g12.png}}
% 76
{\PTglyph{5}{t08_l04g13.png}}
% 77
{\PTglyph{5}{t08_l04g14.png}}
% 78
{\PTglyph{5}{t08_l04g15.png}}
% 79
{\PTglyph{5}{t08_l04g16.png}}
% 80
{\PTglyph{5}{t08_l04g17.png}}
% 81
{\PTglyph{5}{t08_l04g18.png}}
% 82
{\PTglyph{5}{t08_l04g19.png}}
% 83
{\PTglyph{5}{t08_l04g20.png}}
% 84
{\PTglyph{5}{t08_l04g21.png}}
% 85
{\PTglyph{5}{t08_l04g22.png}}
% 86
{\PTglyph{5}{t08_l04g23.png}}
% 87
{\PTglyph{5}{t08_l04g24.png}}
% 88
{\PTglyph{5}{t08_l04g25.png}}
% 89
{\PTglyph{5}{t08_l04g26.png}}
% 90
{\PTglyph{5}{t08_l04g27.png}}
% 91
{\PTglyph{5}{t08_l04g28.png}}
% 92
{\PTglyph{5}{t08_l04g29.png}}
% 93
{\PTglyph{5}{t08_l04g30.png}}
//
%%% Local Variables:
%%% mode: latex
%%% TeX-engine: luatex
%%% TeX-master: shared
%%% End:

%//
%\glpismo%
 \glpismo
% 1
{\PTglyphid{Au-06_0101}}
% 2
{\PTglyphid{Au-06_0102}}
% 3
{\PTglyphid{Au-06_0103}}
% 4
{\PTglyphid{Au-06_0104}}
% 5
{\PTglyphid{Au-06_0105}}
% 6
{\PTglyphid{Au-06_0106}}
% 7
{\PTglyphid{Au-06_0107}}
% 8
{\PTglyphid{Au-06_0108}}
% 9
{\PTglyphid{Au-06_0109}}
% 10
{\PTglyphid{Au-06_0110}}
% 11
{\PTglyphid{Au-06_0111}}
% 12
{\PTglyphid{Au-06_0112}}
% 13
{\PTglyphid{Au-06_0113}}
% 14
{\PTglyphid{Au-06_0114}}
% 15
{\PTglyphid{Au-06_0115}}
% 16
{\PTglyphid{Au-06_0116}}
% 17
{\PTglyphid{Au-06_0117}}
% 18
{\PTglyphid{Au-06_0118}}
% 19
{\PTglyphid{Au-06_0119}}
% 20
{\PTglyphid{Au-06_0120}}
% 21
{\PTglyphid{Au-06_0121}}
% 22
{\PTglyphid{Au-06_0201}}
% 23
{\PTglyphid{Au-06_0202}}
% 24
{\PTglyphid{Au-06_0203}}
% 25
{\PTglyphid{Au-06_0204}}
% 26
{\PTglyphid{Au-06_0205}}
% 27
{\PTglyphid{Au-06_0301}}
% 28
{\PTglyphid{Au-06_0302}}
% 29
{\PTglyphid{Au-06_0303}}
% 30
{\PTglyphid{Au-06_0304}}
% 31
{\PTglyphid{Au-06_0305}}
% 32
{\PTglyphid{Au-06_0306}}
% 33
{\PTglyphid{Au-06_0307}}
% 34
{\PTglyphid{Au-06_0308}}
% 35
{\PTglyphid{Au-06_0309}}
% 36
{\PTglyphid{Au-06_0310}}
% 37
{\PTglyphid{Au-06_0311}}
% 38
{\PTglyphid{Au-06_0312}}
% 39
{\PTglyphid{Au-06_0313}}
% 40
{\PTglyphid{Au-06_0314}}
% 41
{\PTglyphid{Au-06_0315}}
% 42
{\PTglyphid{Au-06_0316}}
% 43
{\PTglyphid{Au-06_0317}}
% 44
{\PTglyphid{Au-06_0318}}
% 45
{\PTglyphid{Au-06_0319}}
% 46
{\PTglyphid{Au-06_0320}}
% 47
{\PTglyphid{Au-06_0321}}
% 48
{\PTglyphid{Au-06_0322}}
% 49
{\PTglyphid{Au-06_0323}}
% 50
{\PTglyphid{Au-06_0324}}
% 51
{\PTglyphid{Au-06_0325}}
% 52
{\PTglyphid{Au-06_0326}}
% 53
{\PTglyphid{Au-06_0327}}
% 54
{\PTglyphid{Au-06_0328}}
% 55
{\PTglyphid{Au-06_0329}}
% 56
{\PTglyphid{Au-06_0330}}
% 57
{\PTglyphid{Au-06_0331}}
% 58
{\PTglyphid{Au-06_0332}}
% 59
{\PTglyphid{Au-06_0333}}
% 60
{\PTglyphid{Au-06_0334}}
% 61
{\PTglyphid{Au-06_0335}}
% 62
{\PTglyphid{Au-06_0336}}
% 63
{\PTglyphid{Au-06_0337}}
% 64
{\PTglyphid{Au-06_0401}}
% 65
{\PTglyphid{Au-06_0402}}
% 66
{\PTglyphid{Au-06_0403}}
% 67
{\PTglyphid{Au-06_0404}}
% 68
{\PTglyphid{Au-06_0405}}
% 69
{\PTglyphid{Au-06_0406}}
% 70
{\PTglyphid{Au-06_0407}}
% 71
{\PTglyphid{Au-06_0408}}
% 72
{\PTglyphid{Au-06_0409}}
% 73
{\PTglyphid{Au-06_0410}}
% 74
{\PTglyphid{Au-06_0411}}
% 75
{\PTglyphid{Au-06_0412}}
% 76
{\PTglyphid{Au-06_0413}}
% 77
{\PTglyphid{Au-06_0414}}
% 78
{\PTglyphid{Au-06_0415}}
% 79
{\PTglyphid{Au-06_0416}}
% 80
{\PTglyphid{Au-06_0417}}
% 81
{\PTglyphid{Au-06_0418}}
% 82
{\PTglyphid{Au-06_0419}}
% 83
{\PTglyphid{Au-06_0420}}
% 84
{\PTglyphid{Au-06_0421}}
% 85
{\PTglyphid{Au-06_0422}}
% 86
{\PTglyphid{Au-06_0423}}
% 87
{\PTglyphid{Au-06_0424}}
% 88
{\PTglyphid{Au-06_0425}}
% 89
{\PTglyphid{Au-06_0426}}
% 90
{\PTglyphid{Au-06_0427}}
% 91
{\PTglyphid{Au-06_0428}}
% 92
{\PTglyphid{Au-06_0429}}
% 93
{\PTglyphid{Au-06_0430}}
//
\endgl \xe
%%% Local Variables:
%%% mode: latex
%%% TeX-engine: luatex
%%% TeX-master: shared
%%% End:

% //
%\endgl \xe


 
 \newpage

%%%%%%%%%%%%%%%%%%%%%%%%%%%%%%%%%%%%%%%%%%%%%%%%%%%%%%%%%%%%%%%%%%%%%%%%%%%%%% 
% Tab. 09,Augezdecki-07_PT08_409.djvu,Augezdecki,07,08,409
%%%%%%%%%%%%%%%%%%%%%%%%%%%%%%%%%%%%%%%%%%%%%%%%%%%%%%%%%%%%%%%%%%%%%%%%%%%%%%

% Note "7. Pismo tekstowe, antykwa Qu/(G, I). Stopień 20 ww. = 101/102 mm (tercja). — Tabl. 409."
% Note1 "Character set table prepared by Paulina Buchwald-Pelcowa"


\pismoPL{Aleksander Augezdecki 7. Pismo tekstowe, antykwa Qu/(G, I). Stopień 20 ww. = 101/102 mm (tercja). — Tabl. 409.}

\pismoEN{Aleksander Augezdecki 7. Text Roman type Qu/(G, I). Type size 20 lines = 101/102 mm (tertia). — Tabl. 409.}
% https://www.adfontes.uzh.ch/en/tutorium/schriften-lesen/schriftgeschichte/gotische-minuskeln-textura-und-textualis/
\medskip

\plate{409}{VIII}{1972}

The plate    prepared by Paulina Buchwald-Pelcowa.\\
The font table    prepared by Paulina Buchwald-Pelcowa.



\bigskip

 \exampleBib{VIII:4$^a$}

\bigskip
\exampleDesc{[TESTAMENTUM NOVUM. Evangelium secundum Matthaeum. Trad. polon. Stanislaus Murzynowski]:
Ewangelia Św. Mateusza. Królewiec, [Aleksander Augezdecki], 1551. 4°.}

\medskip
\examplePage{\textit{Karta A₂a.}}

  \bigskip
\exampleLib{Biblioteka Czartoryskich. Kraków.}

\bigskip
\exampleRef{\textit{Estreicher XIII 26. Wierzbowski 135.}}

\bigskip
\exampleDig{\url{https://crispa.uw.edu.pl/object/files/318618/display/Default} page 10}

\medskip

    \examplePL{Pismo 7: tekst i zestaw.}

    \medskip

    \exampleEN{Font 7. The text and the font table}


\bigskip


\fontID{Au-07}{09}

\fontstat{84}

% \exdisplay \bg \gla
 \exdisplay \bg \gla
% 1
{\PTglyph{5}{t09_l01g01.png}}
% 2
{\PTglyph{5}{t09_l01g02.png}}
% 3
{\PTglyph{5}{t09_l01g03.png}}
% 4
{\PTglyph{5}{t09_l01g04.png}}
% 5
{\PTglyph{5}{t09_l01g05.png}}
% 6
{\PTglyph{5}{t09_l01g06.png}}
% 7
{\PTglyph{5}{t09_l01g07.png}}
% 8
{\PTglyph{5}{t09_l01g08.png}}
% 9
{\PTglyph{5}{t09_l01g09.png}}
% 10
{\PTglyph{5}{t09_l01g10.png}}
% 11
{\PTglyph{5}{t09_l01g11.png}}
% 12
{\PTglyph{5}{t09_l01g12.png}}
% 13
{\PTglyph{5}{t09_l01g13.png}}
% 14
{\PTglyph{5}{t09_l01g14.png}}
% 15
{\PTglyph{5}{t09_l01g15.png}}
% 16
{\PTglyph{5}{t09_l01g16.png}}
% 17
{\PTglyph{5}{t09_l01g17.png}}
% 18
{\PTglyph{5}{t09_l01g18.png}}
% 19
{\PTglyph{5}{t09_l01g19.png}}
% 20
{\PTglyph{5}{t09_l01g20.png}}
% 21
{\PTglyph{5}{t09_l01g21.png}}
% 22
{\PTglyph{5}{t09_l01g22.png}}
% 23
{\PTglyph{5}{t09_l01g23.png}}
% 24
{\PTglyph{5}{t09_l01g24.png}}
% 25
{\PTglyph{5}{t09_l01g25.png}}
% 26
{\PTglyph{5}{t09_l02g01.png}}
% 27
{\PTglyph{5}{t09_l02g02.png}}
% 28
{\PTglyph{5}{t09_l02g03.png}}
% 29
{\PTglyph{5}{t09_l02g04.png}}
% 30
{\PTglyph{5}{t09_l02g05.png}}
% 31
{\PTglyph{5}{t09_l02g06.png}}
% 32
{\PTglyph{5}{t09_l02g07.png}}
% 33
{\PTglyph{5}{t09_l02g08.png}}
% 34
{\PTglyph{5}{t09_l02g09.png}}
% 35
{\PTglyph{5}{t09_l02g10.png}}
% 36
{\PTglyph{5}{t09_l02g11.png}}
% 37
{\PTglyph{5}{t09_l02g12.png}}
% 38
{\PTglyph{5}{t09_l02g13.png}}
% 39
{\PTglyph{5}{t09_l02g14.png}}
% 40
{\PTglyph{5}{t09_l02g15.png}}
% 41
{\PTglyph{5}{t09_l02g16.png}}
% 42
{\PTglyph{5}{t09_l02g17.png}}
% 43
{\PTglyph{5}{t09_l02g18.png}}
% 44
{\PTglyph{5}{t09_l02g19.png}}
% 45
{\PTglyph{5}{t09_l02g20.png}}
% 46
{\PTglyph{5}{t09_l02g21.png}}
% 47
{\PTglyph{5}{t09_l02g22.png}}
% 48
{\PTglyph{5}{t09_l02g23.png}}
% 49
{\PTglyph{5}{t09_l02g24.png}}
% 50
{\PTglyph{5}{t09_l02g25.png}}
% 51
{\PTglyph{5}{t09_l02g26.png}}
% 52
{\PTglyph{5}{t09_l02g27.png}}
% 53
{\PTglyph{5}{t09_l02g28.png}}
% 54
{\PTglyph{5}{t09_l02g29.png}}
% 55
{\PTglyph{5}{t09_l02g30.png}}
% 56
{\PTglyph{5}{t09_l02g31.png}}
% 57
{\PTglyph{5}{t09_l02g32.png}}
% 58
{\PTglyph{5}{t09_l02g33.png}}
% 59
{\PTglyph{5}{t09_l02g34.png}}
% 60
{\PTglyph{5}{t09_l02g35.png}}
% 61
{\PTglyph{5}{t09_l02g36.png}}
% 62
{\PTglyph{5}{t09_l02g37.png}}
% 63
{\PTglyph{5}{t09_l02g38.png}}
% 64
{\PTglyph{5}{t09_l02g39.png}}
% 65
{\PTglyph{5}{t09_l02g40.png}}
% 66
{\PTglyph{5}{t09_l02g41.png}}
% 67
{\PTglyph{5}{t09_l02g42.png}}
% 68
{\PTglyph{5}{t09_l02g43.png}}
% 69
{\PTglyph{5}{t09_l03g01.png}}
% 70
{\PTglyph{5}{t09_l03g02.png}}
% 71
{\PTglyph{5}{t09_l03g03.png}}
% 72
{\PTglyph{5}{t09_l03g04.png}}
% 73
{\PTglyph{5}{t09_l03g05.png}}
% 74
{\PTglyph{5}{t09_l03g06.png}}
% 75
{\PTglyph{5}{t09_l03g07.png}}
% 76
{\PTglyph{5}{t09_l03g08.png}}
% 77
{\PTglyph{5}{t09_l03g09.png}}
% 78
{\PTglyph{5}{t09_l03g10.png}}
% 79
{\PTglyph{5}{t09_l03g11.png}}
% 80
{\PTglyph{5}{t09_l03g12.png}}
% 81
{\PTglyph{5}{t09_l03g13.png}}
% 82
{\PTglyph{5}{t09_l03g14.png}}
% 83
{\PTglyph{5}{t09_l03g15.png}}
% 84
{\PTglyph{5}{t09_l03g16.png}}
//
%%% Local Variables:
%%% mode: latex
%%% TeX-engine: luatex
%%% TeX-master: shared
%%% End:

%//
%\glpismo%
 \glpismo
% 1
{\PTglyphid{Au-07_0101}}
% 2
{\PTglyphid{Au-07_0102}}
% 3
{\PTglyphid{Au-07_0103}}
% 4
{\PTglyphid{Au-07_0104}}
% 5
{\PTglyphid{Au-07_0105}}
% 6
{\PTglyphid{Au-07_0106}}
% 7
{\PTglyphid{Au-07_0107}}
% 8
{\PTglyphid{Au-07_0108}}
% 9
{\PTglyphid{Au-07_0109}}
% 10
{\PTglyphid{Au-07_0110}}
% 11
{\PTglyphid{Au-07_0111}}
% 12
{\PTglyphid{Au-07_0112}}
% 13
{\PTglyphid{Au-07_0113}}
% 14
{\PTglyphid{Au-07_0114}}
% 15
{\PTglyphid{Au-07_0115}}
% 16
{\PTglyphid{Au-07_0116}}
% 17
{\PTglyphid{Au-07_0117}}
% 18
{\PTglyphid{Au-07_0118}}
% 19
{\PTglyphid{Au-07_0119}}
% 20
{\PTglyphid{Au-07_0120}}
% 21
{\PTglyphid{Au-07_0121}}
% 22
{\PTglyphid{Au-07_0122}}
% 23
{\PTglyphid{Au-07_0123}}
% 24
{\PTglyphid{Au-07_0124}}
% 25
{\PTglyphid{Au-07_0125}}
% 26
{\PTglyphid{Au-07_0201}}
% 27
{\PTglyphid{Au-07_0202}}
% 28
{\PTglyphid{Au-07_0203}}
% 29
{\PTglyphid{Au-07_0204}}
% 30
{\PTglyphid{Au-07_0205}}
% 31
{\PTglyphid{Au-07_0206}}
% 32
{\PTglyphid{Au-07_0207}}
% 33
{\PTglyphid{Au-07_0208}}
% 34
{\PTglyphid{Au-07_0209}}
% 35
{\PTglyphid{Au-07_0210}}
% 36
{\PTglyphid{Au-07_0211}}
% 37
{\PTglyphid{Au-07_0212}}
% 38
{\PTglyphid{Au-07_0213}}
% 39
{\PTglyphid{Au-07_0214}}
% 40
{\PTglyphid{Au-07_0215}}
% 41
{\PTglyphid{Au-07_0216}}
% 42
{\PTglyphid{Au-07_0217}}
% 43
{\PTglyphid{Au-07_0218}}
% 44
{\PTglyphid{Au-07_0219}}
% 45
{\PTglyphid{Au-07_0220}}
% 46
{\PTglyphid{Au-07_0221}}
% 47
{\PTglyphid{Au-07_0222}}
% 48
{\PTglyphid{Au-07_0223}}
% 49
{\PTglyphid{Au-07_0224}}
% 50
{\PTglyphid{Au-07_0225}}
% 51
{\PTglyphid{Au-07_0226}}
% 52
{\PTglyphid{Au-07_0227}}
% 53
{\PTglyphid{Au-07_0228}}
% 54
{\PTglyphid{Au-07_0229}}
% 55
{\PTglyphid{Au-07_0230}}
% 56
{\PTglyphid{Au-07_0231}}
% 57
{\PTglyphid{Au-07_0232}}
% 58
{\PTglyphid{Au-07_0233}}
% 59
{\PTglyphid{Au-07_0234}}
% 60
{\PTglyphid{Au-07_0235}}
% 61
{\PTglyphid{Au-07_0236}}
% 62
{\PTglyphid{Au-07_0237}}
% 63
{\PTglyphid{Au-07_0238}}
% 64
{\PTglyphid{Au-07_0239}}
% 65
{\PTglyphid{Au-07_0240}}
% 66
{\PTglyphid{Au-07_0241}}
% 67
{\PTglyphid{Au-07_0242}}
% 68
{\PTglyphid{Au-07_0243}}
% 69
{\PTglyphid{Au-07_0301}}
% 70
{\PTglyphid{Au-07_0302}}
% 71
{\PTglyphid{Au-07_0303}}
% 72
{\PTglyphid{Au-07_0304}}
% 73
{\PTglyphid{Au-07_0305}}
% 74
{\PTglyphid{Au-07_0306}}
% 75
{\PTglyphid{Au-07_0307}}
% 76
{\PTglyphid{Au-07_0308}}
% 77
{\PTglyphid{Au-07_0309}}
% 78
{\PTglyphid{Au-07_0310}}
% 79
{\PTglyphid{Au-07_0311}}
% 80
{\PTglyphid{Au-07_0312}}
% 81
{\PTglyphid{Au-07_0313}}
% 82
{\PTglyphid{Au-07_0314}}
% 83
{\PTglyphid{Au-07_0315}}
% 84
{\PTglyphid{Au-07_0316}}
//
\endgl \xe
%%% Local Variables:
%%% mode: latex
%%% TeX-engine: luatex
%%% TeX-master: shared
%%% End:

% //
%\endgl \xe

 \newpage

%%%%%%%%%%%%%%%%%%%%%%%%%%%%%%%%%%%%%%%%%%%%%%%%%%%%%%%%%%%%%%%%%%%%%%%%%%%%%% 
% Tab. 10,Augezdecki-08_PT08_410.djvu,Augezdecki,08,08,410
%%%%%%%%%%%%%%%%%%%%%%%%%%%%%%%%%%%%%%%%%%%%%%%%%%%%%%%%%%%%%%%%%%%%%%%%%%%%%%

% Note "8. Pismo tekstowe, antykwa Qu/(G). Stopień 20 ww. = 101 mm (tercja). — Tabl. 410."
% Note1 "Character set table prepared by Paulina Buchwald-Pelcowa"


\pismoPL{Aleksander Augezdecki 8. Pismo tekstowe, antykwa Qu/(G, I). Stopień 20 ww. = 101/102 mm (tercja). — Tabl. 410 [pierwszy zestaw].}

\pismoEN{Aleksander Augezdecki 8. Text Roman type Qu/(G, I). Type size 20 lines = 101/102 mm (tertia). — Tabl. 410 [first set].}
% https://www.adfontes.uzh.ch/en/tutorium/schriften-lesen/schriftgeschichte/gotische-minuskeln-textura-und-textualis/
\medskip

\plate{410[1]}{VIII}{1972}

The plate    prepared by Paulina Buchwald-Pelcowa.\\
The font table    prepared by Paulina Buchwald-Pelcowa.



\bigskip

 \exampleBib{VIII:26}

\bigskip
\exampleDesc{[CONFESSIO AUGUSTANA. Trad. polon. Martinus Kwiatkowski]: Confessio Augustanae fidei to jest wyznanie
wiary krześciańskiej. Szamotuły, Aleksander Augezdecki, [po 13 V] 1561. 4°.}

\medskip
\examplePage{\textit{Karta B₁a.}}

  \bigskip
\exampleLib{Biblioteka Zakł. Nar. im. Ossolinskich. Wroclaw.}

\bigskip
\exampleRef{\textit{Estreicher XIV 355. Wierzbowski 1393. Bohonos Ossol. 442. Drukarze IV 31 i 85.}}

\bigskip
\exampleDig{\url{https://dbc.wroc.pl/dlibra/publication/15990/edition/14101} page 13}

\medskip

    \examplePL{Pismo 8: kolumna [? -- JSB]  i pierwszy zestaw.}

    \medskip

    \exampleEN{Font 7. The text column [?] and the first font table.}


\bigskip


\fontID{Au-08}{10}

\fontstat{104}

% \exdisplay \bg \gla
 \exdisplay \bg \gla
% 1
{\PTglyph{5}{t10_l01g01.png}}
% 2
{\PTglyph{5}{t10_l01g02.png}}
% 3
{\PTglyph{5}{t10_l01g03.png}}
% 4
{\PTglyph{5}{t10_l01g04.png}}
% 5
{\PTglyph{5}{t10_l01g05.png}}
% 6
{\PTglyph{5}{t10_l01g06.png}}
% 7
{\PTglyph{5}{t10_l01g07.png}}
% 8
{\PTglyph{5}{t10_l01g08.png}}
% 9
{\PTglyph{5}{t10_l01g09.png}}
% 10
{\PTglyph{5}{t10_l01g10.png}}
% 11
{\PTglyph{5}{t10_l01g11.png}}
% 12
{\PTglyph{5}{t10_l01g12.png}}
% 13
{\PTglyph{5}{t10_l01g13.png}}
% 14
{\PTglyph{5}{t10_l01g14.png}}
% 15
{\PTglyph{5}{t10_l01g15.png}}
% 16
{\PTglyph{5}{t10_l01g16.png}}
% 17
{\PTglyph{5}{t10_l01g17.png}}
% 18
{\PTglyph{5}{t10_l01g18.png}}
% 19
{\PTglyph{5}{t10_l01g19.png}}
% 20
{\PTglyph{5}{t10_l01g20.png}}
% 21
{\PTglyph{5}{t10_l01g21.png}}
% 22
{\PTglyph{5}{t10_l01g22.png}}
% 23
{\PTglyph{5}{t10_l01g23.png}}
% 24
{\PTglyph{5}{t10_l01g24.png}}
% 25
{\PTglyph{5}{t10_l01g25.png}}
% 26
{\PTglyph{5}{t10_l01g26.png}}
% 27
{\PTglyph{5}{t10_l02g01.png}}
% 28
{\PTglyph{5}{t10_l02g02.png}}
% 29
{\PTglyph{5}{t10_l02g03.png}}
% 30
{\PTglyph{5}{t10_l02g04.png}}
% 31
{\PTglyph{5}{t10_l02g05.png}}
% 32
{\PTglyph{5}{t10_l02g06.png}}
% 33
{\PTglyph{5}{t10_l02g07.png}}
% 34
{\PTglyph{5}{t10_l02g08.png}}
% 35
{\PTglyph{5}{t10_l02g09.png}}
% 36
{\PTglyph{5}{t10_l02g10.png}}
% 37
{\PTglyph{5}{t10_l02g11.png}}
% 38
{\PTglyph{5}{t10_l02g12.png}}
% 39
{\PTglyph{5}{t10_l02g13.png}}
% 40
{\PTglyph{5}{t10_l02g14.png}}
% 41
{\PTglyph{5}{t10_l02g15.png}}
% 42
{\PTglyph{5}{t10_l02g16.png}}
% 43
{\PTglyph{5}{t10_l02g17.png}}
% 44
{\PTglyph{5}{t10_l02g18.png}}
% 45
{\PTglyph{5}{t10_l02g19.png}}
% 46
{\PTglyph{5}{t10_l02g20.png}}
% 47
{\PTglyph{5}{t10_l02g21.png}}
% 48
{\PTglyph{5}{t10_l02g22.png}}
% 49
{\PTglyph{5}{t10_l02g23.png}}
% 50
{\PTglyph{5}{t10_l02g24.png}}
% 51
{\PTglyph{5}{t10_l02g25.png}}
% 52
{\PTglyph{5}{t10_l02g26.png}}
% 53
{\PTglyph{5}{t10_l02g27.png}}
% 54
{\PTglyph{5}{t10_l02g28.png}}
% 55
{\PTglyph{5}{t10_l02g29.png}}
% 56
{\PTglyph{5}{t10_l02g30.png}}
% 57
{\PTglyph{5}{t10_l02g31.png}}
% 58
{\PTglyph{5}{t10_l02g32.png}}
% 59
{\PTglyph{5}{t10_l02g33.png}}
% 60
{\PTglyph{5}{t10_l02g34.png}}
% 61
{\PTglyph{5}{t10_l02g35.png}}
% 62
{\PTglyph{5}{t10_l02g36.png}}
% 63
{\PTglyph{5}{t10_l02g37.png}}
% 64
{\PTglyph{5}{t10_l02g38.png}}
% 65
{\PTglyph{5}{t10_l02g39.png}}
% 66
{\PTglyph{5}{t10_l02g40.png}}
% 67
{\PTglyph{5}{t10_l02g41.png}}
% 68
{\PTglyph{5}{t10_l02g42.png}}
% 69
{\PTglyph{5}{t10_l02g43.png}}
% 70
{\PTglyph{5}{t10_l02g44.png}}
% 71
{\PTglyph{5}{t10_l03g01.png}}
% 72
{\PTglyph{5}{t10_l03g02.png}}
% 73
{\PTglyph{5}{t10_l03g03.png}}
% 74
{\PTglyph{5}{t10_l03g04.png}}
% 75
{\PTglyph{5}{t10_l03g05.png}}
% 76
{\PTglyph{5}{t10_l03g06.png}}
% 77
{\PTglyph{5}{t10_l03g07.png}}
% 78
{\PTglyph{5}{t10_l03g08.png}}
% 79
{\PTglyph{5}{t10_l03g09.png}}
% 80
{\PTglyph{5}{t10_l03g10.png}}
% 81
{\PTglyph{5}{t10_l03g11.png}}
% 82
{\PTglyph{5}{t10_l03g12.png}}
% 83
{\PTglyph{5}{t10_l03g13.png}}
% 84
{\PTglyph{5}{t10_l03g14.png}}
% 85
{\PTglyph{5}{t10_l03g15.png}}
% 86
{\PTglyph{5}{t10_l03g16.png}}
% 87
{\PTglyph{5}{t10_l03g17.png}}
% 88
{\PTglyph{5}{t10_l03g18.png}}
% 89
{\PTglyph{5}{t10_l03g19.png}}
% 90
{\PTglyph{5}{t10_l03g20.png}}
% 91
{\PTglyph{5}{t10_l03g21.png}}
% 92
{\PTglyph{5}{t10_l03g22.png}}
% 93
{\PTglyph{5}{t10_l03g23.png}}
% 94
{\PTglyph{5}{t10_l03g24.png}}
% 95
{\PTglyph{5}{t10_l03g25.png}}
% 96
{\PTglyph{5}{t10_l03g26.png}}
% 97
{\PTglyph{5}{t10_l03g27.png}}
% 98
{\PTglyph{5}{t10_l03g28.png}}
% 99
{\PTglyph{5}{t10_l03g29.png}}
% 100
{\PTglyph{5}{t10_l03g30.png}}
% 101
{\PTglyph{5}{t10_l03g31.png}}
% 102
{\PTglyph{5}{t10_l03g32.png}}
% 103
{\PTglyph{5}{t10_l03g33.png}}
% 104
{\PTglyph{5}{t10_l03g34.png}}
//
%%% Local Variables:
%%% mode: latex
%%% TeX-engine: luatex
%%% TeX-master: shared
%%% End:

%//
%\glpismo%
 \glpismo
% 1
{\PTglyphid{Au-08_0101}}
% 2
{\PTglyphid{Au-08_0102}}
% 3
{\PTglyphid{Au-08_0103}}
% 4
{\PTglyphid{Au-08_0104}}
% 5
{\PTglyphid{Au-08_0105}}
% 6
{\PTglyphid{Au-08_0106}}
% 7
{\PTglyphid{Au-08_0107}}
% 8
{\PTglyphid{Au-08_0108}}
% 9
{\PTglyphid{Au-08_0109}}
% 10
{\PTglyphid{Au-08_0110}}
% 11
{\PTglyphid{Au-08_0111}}
% 12
{\PTglyphid{Au-08_0112}}
% 13
{\PTglyphid{Au-08_0113}}
% 14
{\PTglyphid{Au-08_0114}}
% 15
{\PTglyphid{Au-08_0115}}
% 16
{\PTglyphid{Au-08_0116}}
% 17
{\PTglyphid{Au-08_0117}}
% 18
{\PTglyphid{Au-08_0118}}
% 19
{\PTglyphid{Au-08_0119}}
% 20
{\PTglyphid{Au-08_0120}}
% 21
{\PTglyphid{Au-08_0121}}
% 22
{\PTglyphid{Au-08_0122}}
% 23
{\PTglyphid{Au-08_0123}}
% 24
{\PTglyphid{Au-08_0124}}
% 25
{\PTglyphid{Au-08_0125}}
% 26
{\PTglyphid{Au-08_0126}}
% 27
{\PTglyphid{Au-08_0201}}
% 28
{\PTglyphid{Au-08_0202}}
% 29
{\PTglyphid{Au-08_0203}}
% 30
{\PTglyphid{Au-08_0204}}
% 31
{\PTglyphid{Au-08_0205}}
% 32
{\PTglyphid{Au-08_0206}}
% 33
{\PTglyphid{Au-08_0207}}
% 34
{\PTglyphid{Au-08_0208}}
% 35
{\PTglyphid{Au-08_0209}}
% 36
{\PTglyphid{Au-08_0210}}
% 37
{\PTglyphid{Au-08_0211}}
% 38
{\PTglyphid{Au-08_0212}}
% 39
{\PTglyphid{Au-08_0213}}
% 40
{\PTglyphid{Au-08_0214}}
% 41
{\PTglyphid{Au-08_0215}}
% 42
{\PTglyphid{Au-08_0216}}
% 43
{\PTglyphid{Au-08_0217}}
% 44
{\PTglyphid{Au-08_0218}}
% 45
{\PTglyphid{Au-08_0219}}
% 46
{\PTglyphid{Au-08_0220}}
% 47
{\PTglyphid{Au-08_0221}}
% 48
{\PTglyphid{Au-08_0222}}
% 49
{\PTglyphid{Au-08_0223}}
% 50
{\PTglyphid{Au-08_0224}}
% 51
{\PTglyphid{Au-08_0225}}
% 52
{\PTglyphid{Au-08_0226}}
% 53
{\PTglyphid{Au-08_0227}}
% 54
{\PTglyphid{Au-08_0228}}
% 55
{\PTglyphid{Au-08_0229}}
% 56
{\PTglyphid{Au-08_0230}}
% 57
{\PTglyphid{Au-08_0231}}
% 58
{\PTglyphid{Au-08_0232}}
% 59
{\PTglyphid{Au-08_0233}}
% 60
{\PTglyphid{Au-08_0234}}
% 61
{\PTglyphid{Au-08_0235}}
% 62
{\PTglyphid{Au-08_0236}}
% 63
{\PTglyphid{Au-08_0237}}
% 64
{\PTglyphid{Au-08_0238}}
% 65
{\PTglyphid{Au-08_0239}}
% 66
{\PTglyphid{Au-08_0240}}
% 67
{\PTglyphid{Au-08_0241}}
% 68
{\PTglyphid{Au-08_0242}}
% 69
{\PTglyphid{Au-08_0243}}
% 70
{\PTglyphid{Au-08_0244}}
% 71
{\PTglyphid{Au-08_0301}}
% 72
{\PTglyphid{Au-08_0302}}
% 73
{\PTglyphid{Au-08_0303}}
% 74
{\PTglyphid{Au-08_0304}}
% 75
{\PTglyphid{Au-08_0305}}
% 76
{\PTglyphid{Au-08_0306}}
% 77
{\PTglyphid{Au-08_0307}}
% 78
{\PTglyphid{Au-08_0308}}
% 79
{\PTglyphid{Au-08_0309}}
% 80
{\PTglyphid{Au-08_0310}}
% 81
{\PTglyphid{Au-08_0311}}
% 82
{\PTglyphid{Au-08_0312}}
% 83
{\PTglyphid{Au-08_0313}}
% 84
{\PTglyphid{Au-08_0314}}
% 85
{\PTglyphid{Au-08_0315}}
% 86
{\PTglyphid{Au-08_0316}}
% 87
{\PTglyphid{Au-08_0317}}
% 88
{\PTglyphid{Au-08_0318}}
% 89
{\PTglyphid{Au-08_0319}}
% 90
{\PTglyphid{Au-08_0320}}
% 91
{\PTglyphid{Au-08_0321}}
% 92
{\PTglyphid{Au-08_0322}}
% 93
{\PTglyphid{Au-08_0323}}
% 94
{\PTglyphid{Au-08_0324}}
% 95
{\PTglyphid{Au-08_0325}}
% 96
{\PTglyphid{Au-08_0326}}
% 97
{\PTglyphid{Au-08_0327}}
% 98
{\PTglyphid{Au-08_0328}}
% 99
{\PTglyphid{Au-08_0329}}
% 100
{\PTglyphid{Au-08_0330}}
% 101
{\PTglyphid{Au-08_0331}}
% 102
{\PTglyphid{Au-08_0332}}
% 103
{\PTglyphid{Au-08_0333}}
% 104
{\PTglyphid{Au-08_0334}}
//
\endgl \xe
%%% Local Variables:
%%% mode: latex
%%% TeX-engine: luatex
%%% TeX-master: shared
%%% End:

% //
%\endgl \xe


 
 \newpage

%%%%%%%%%%%%%%%%%%%%%%%%%%%%%%%%%%%%%%%%%%%%%%%%%%%%%%%%%%%%%%%%%%%%%%%%%%%%%% 
% Tab. 11,Augezdecki-09_PT08_408.djvu,Augezdecki,09,08,408
%%%%%%%%%%%%%%%%%%%%%%%%%%%%%%%%%%%%%%%%%%%%%%%%%%%%%%%%%%%%%%%%%%%%%%%%%%%%%%

% Note "9. Wersaliki tytułowe, antykwa. Wysokość 8 mm. — Tabl. 408."
% Note1 "Character set table prepared by Paulina Buchwald-Pelcowa"

\pismoPL{Aleksander Augezdecki 9. Wersaliki tytułowe, antykwa. Wysokość 8 mm. — Tabl. 408 [trzeci zestaw].}

\pismoEN{Aleksander Augezdecki 9. Roman title capitals.  Type size 1 [line] = 8 mm. - Plate 408 [third set].}
% https://www.adfontes.uzh.ch/en/tutorium/schriften-lesen/schriftgeschichte/gotische-minuskeln-textura-und-textualis/
\medskip

\plate{408[3]}{VIII}{1972}

The plate    prepared by Paulina Buchwald-Pelcowa.\\
The font table    prepared by Paulina Buchwald-Pelcowa.\\
\relax[Layout confusing, misinterpretation possible --- JSB]


\bigskip

\fontID{Au-09}{11}

\fontstat{29}

% \exdisplay \bg \gla
 \exdisplay \bg \gla
% 1
{\PTglyph{5}{t11_l01g01.png}}
% 2
{\PTglyph{5}{t11_l01g02.png}}
% 3
{\PTglyph{5}{t11_l01g03.png}}
% 4
{\PTglyph{5}{t11_l01g04.png}}
% 5
{\PTglyph{5}{t11_l01g05.png}}
% 6
{\PTglyph{5}{t11_l01g06.png}}
% 7
{\PTglyph{5}{t11_l01g07.png}}
% 8
{\PTglyph{5}{t11_l01g08.png}}
% 9
{\PTglyph{5}{t11_l01g09.png}}
% 10
{\PTglyph{5}{t11_l01g10.png}}
% 11
{\PTglyph{5}{t11_l01g11.png}}
% 12
{\PTglyph{5}{t11_l01g12.png}}
% 13
{\PTglyph{5}{t11_l01g13.png}}
% 14
{\PTglyph{5}{t11_l01g14.png}}
% 15
{\PTglyph{5}{t11_l01g15.png}}
% 16
{\PTglyph{5}{t11_l01g16.png}}
% 17
{\PTglyph{5}{t11_l01g17.png}}
% 18
{\PTglyph{5}{t11_l02g01.png}}
% 19
{\PTglyph{5}{t11_l02g02.png}}
% 20
{\PTglyph{5}{t11_l02g03.png}}
% 21
{\PTglyph{5}{t11_l02g04.png}}
% 22
{\PTglyph{5}{t11_l02g05.png}}
% 23
{\PTglyph{5}{t11_l02g06.png}}
% 24
{\PTglyph{5}{t11_l02g07.png}}
% 25
{\PTglyph{5}{t11_l02g08.png}}
% 26
{\PTglyph{5}{t11_l02g09.png}}
% 27
{\PTglyph{5}{t11_l02g10.png}}
% 28
{\PTglyph{5}{t11_l02g11.png}}
% 29
{\PTglyph{5}{t11_l02g12.png}}
//
%%% Local Variables:
%%% mode: latex
%%% TeX-engine: luatex
%%% TeX-master: shared
%%% End:

%//
%\glpismo%
 \glpismo
% 1
{\PTglyphid{Au-09_0101}}
% 2
{\PTglyphid{Au-09_0102}}
% 3
{\PTglyphid{Au-09_0103}}
% 4
{\PTglyphid{Au-09_0104}}
% 5
{\PTglyphid{Au-09_0105}}
% 6
{\PTglyphid{Au-09_0106}}
% 7
{\PTglyphid{Au-09_0107}}
% 8
{\PTglyphid{Au-09_0108}}
% 9
{\PTglyphid{Au-09_0109}}
% 10
{\PTglyphid{Au-09_0110}}
% 11
{\PTglyphid{Au-09_0111}}
% 12
{\PTglyphid{Au-09_0112}}
% 13
{\PTglyphid{Au-09_0113}}
% 14
{\PTglyphid{Au-09_0114}}
% 15
{\PTglyphid{Au-09_0115}}
% 16
{\PTglyphid{Au-09_0116}}
% 17
{\PTglyphid{Au-09_0117}}
% 18
{\PTglyphid{Au-09_0201}}
% 19
{\PTglyphid{Au-09_0202}}
% 20
{\PTglyphid{Au-09_0203}}
% 21
{\PTglyphid{Au-09_0204}}
% 22
{\PTglyphid{Au-09_0205}}
% 23
{\PTglyphid{Au-09_0206}}
% 24
{\PTglyphid{Au-09_0207}}
% 25
{\PTglyphid{Au-09_0208}}
% 26
{\PTglyphid{Au-09_0209}}
% 27
{\PTglyphid{Au-09_0210}}
% 28
{\PTglyphid{Au-09_0211}}
% 29
{\PTglyphid{Au-09_0212}}
//
\endgl \xe
%%% Local Variables:
%%% mode: latex
%%% TeX-engine: luatex
%%% TeX-master: shared
%%% End:

% //
%\endgl \xe


\newpage

%%%%%%%%%%%%%%%%%%%%%%%%%%%%%%%%%%%%%%%%%%%%%%%%%%%%%%%%%%%%%%%%%%%%%%%%%%%%%% 
% Tab. 12,Augezdecki-10_PT08_408.djvu,Augezdecki,10,08,408
%%%%%%%%%%%%%%%%%%%%%%%%%%%%%%%%%%%%%%%%%%%%%%%%%%%%%%%%%%%%%%%%%%%%%%%%%%%%%%

% Note "10.Wersaliki tytułowe, antykwa. Wysokość 6—7 mm. — Tabl. 408."
% Note1 "Character set table prepared by Paulina Buchwald-Pelcowa"

\pismoPL{Aleksander Augezdecki 10. Wersaliki tytułowe, antykwa. Wysokość 6—7 mm. — Tabl. 408 [czwarty zestaw].}

\pismoEN{Aleksander Augezdecki 10. Roman title capitals.  Type size [1 line] = 6--7 mm. - Plate 408 [forth set].}
% https://www.adfontes.uzh.ch/en/tutorium/schriften-lesen/schriftgeschichte/gotische-minuskeln-textura-und-textualis/
\medskip

\plate{408[4]}{VIII}{1972}

The plate    prepared by Paulina Buchwald-Pelcowa.\\
The font table    prepared by Paulina Buchwald-Pelcowa.\\
\relax[Layout confusing, misinterpretation possible --- JSB]


\bigskip

\fontID{Au-10}{12}

\fontstat{15}

    
% \exdisplay \bg \gla
 \exdisplay \bg \gla
% 1
{\PTglyph{5}{t12_l01g01.png}}
% 2
{\PTglyph{5}{t12_l01g02.png}}
% 3
{\PTglyph{5}{t12_l01g03.png}}
% 4
{\PTglyph{5}{t12_l01g04.png}}
% 5
{\PTglyph{5}{t12_l01g05.png}}
% 6
{\PTglyph{5}{t12_l01g06.png}}
% 7
{\PTglyph{5}{t12_l01g07.png}}
% 8
{\PTglyph{5}{t12_l01g08.png}}
% 9
{\PTglyph{5}{t12_l01g09.png}}
% 10
{\PTglyph{5}{t12_l01g10.png}}
% 11
{\PTglyph{5}{t12_l01g11.png}}
% 12
{\PTglyph{5}{t12_l01g12.png}}
% 13
{\PTglyph{5}{t12_l01g13.png}}
% 14
{\PTglyph{5}{t12_l01g14.png}}
% 15
{\PTglyph{5}{t12_l01g15.png}}
//
%%% Local Variables:
%%% mode: latex
%%% TeX-engine: luatex
%%% TeX-master: shared
%%% End:

%//
%\glpismo%
 \glpismo
% 1
{\PTglyphid{Au-10_0101}}
% 2
{\PTglyphid{Au-10_0102}}
% 3
{\PTglyphid{Au-10_0103}}
% 4
{\PTglyphid{Au-10_0104}}
% 5
{\PTglyphid{Au-10_0105}}
% 6
{\PTglyphid{Au-10_0106}}
% 7
{\PTglyphid{Au-10_0107}}
% 8
{\PTglyphid{Au-10_0108}}
% 9
{\PTglyphid{Au-10_0109}}
% 10
{\PTglyphid{Au-10_0110}}
% 11
{\PTglyphid{Au-10_0111}}
% 12
{\PTglyphid{Au-10_0112}}
% 13
{\PTglyphid{Au-10_0113}}
% 14
{\PTglyphid{Au-10_0114}}
% 15
{\PTglyphid{Au-10_0115}}
//
\endgl \xe
%%% Local Variables:
%%% mode: latex
%%% TeX-engine: luatex
%%% TeX-master: shared
%%% End:

% //
%\endgl \xe

 \newpage
 
%%%%%%%%%%%%%%%%%%%%%%%%%%%%%%%%%%%%%%%%%%%%%%%%%%%%%%%%%%%%%%%%%%%%%%%%%%%%%% 
% Tab. 13,Augezdecki-11_PT08_408.djvu,Augezdecki,11,08,408
%%%%%%%%%%%%%%%%%%%%%%%%%%%%%%%%%%%%%%%%%%%%%%%%%%%%%%%%%%%%%%%%%%%%%%%%%%%%%%

% Note "11. Wersaliki tytułowe, antykwa. Wysokość 4,5—5 mm:— Tabl. 408."
% Note1 "Character set table prepared by Paulina Buchwald-Pelcowa"

\pismoPL{Aleksander Augezdecki 11. Wersaliki tytułowe, antykwa. Wysokość 4,5—5 mm:— Tabl. 408 [czwarty zestaw].}

\pismoEN{Aleksander Augezdecki 11. Roman title capitals. Type size [1 line] = 4.5--5 mm. - Plate 408 [forth set].}
% https://www.adfontes.uzh.ch/en/tutorium/schriften-lesen/schriftgeschichte/gotische-minuskeln-textura-und-textualis/
\medskip

\plate{408[4]}{VIII}{1972}

The plate    prepared by Paulina Buchwald-Pelcowa.\\
The font table    prepared by Paulina Buchwald-Pelcowa.\\
\relax[Layout confusing, misinterpretation possible --- JSB]


\bigskip

\fontID{Au-11}{13}

\fontstat{32}

% \exdisplay \bg \gla
 \exdisplay \bg \gla
% 1
{\PTglyph{5}{t13_l01g01.png}}
% 2
{\PTglyph{5}{t13_l01g02.png}}
% 3
{\PTglyph{5}{t13_l01g03.png}}
% 4
{\PTglyph{5}{t13_l01g04.png}}
% 5
{\PTglyph{5}{t13_l01g05.png}}
% 6
{\PTglyph{5}{t13_l01g06.png}}
% 7
{\PTglyph{5}{t13_l01g07.png}}
% 8
{\PTglyph{5}{t13_l01g08.png}}
% 9
{\PTglyph{5}{t13_l01g09.png}}
% 10
{\PTglyph{5}{t13_l01g10.png}}
% 11
{\PTglyph{5}{t13_l01g11.png}}
% 12
{\PTglyph{5}{t13_l01g12.png}}
% 13
{\PTglyph{5}{t13_l01g13.png}}
% 14
{\PTglyph{5}{t13_l01g14.png}}
% 15
{\PTglyph{5}{t13_l01g15.png}}
% 16
{\PTglyph{5}{t13_l01g16.png}}
% 17
{\PTglyph{5}{t13_l01g17.png}}
% 18
{\PTglyph{5}{t13_l01g18.png}}
% 19
{\PTglyph{5}{t13_l01g19.png}}
% 20
{\PTglyph{5}{t13_l01g20.png}}
% 21
{\PTglyph{5}{t13_l01g21.png}}
% 22
{\PTglyph{5}{t13_l01g22.png}}
% 23
{\PTglyph{5}{t13_l01g23.png}}
% 24
{\PTglyph{5}{t13_l01g24.png}}
% 25
{\PTglyph{5}{t13_l01g25.png}}
% 26
{\PTglyph{5}{t13_l01g26.png}}
% 27
{\PTglyph{5}{t13_l01g27.png}}
% 28
{\PTglyph{5}{t13_l02g01.png}}
% 29
{\PTglyph{5}{t13_l02g02.png}}
% 30
{\PTglyph{5}{t13_l02g03.png}}
% 31
{\PTglyph{5}{t13_l02g04.png}}
% 32
{\PTglyph{5}{t13_l02g05.png}}
//
%%% Local Variables:
%%% mode: latex
%%% TeX-engine: luatex
%%% TeX-master: shared
%%% End:

%//
%\glpismo%
 \glpismo
% 1
{\PTglyphid{Au-11_0101}}
% 2
{\PTglyphid{Au-11_0102}}
% 3
{\PTglyphid{Au-11_0103}}
% 4
{\PTglyphid{Au-11_0104}}
% 5
{\PTglyphid{Au-11_0105}}
% 6
{\PTglyphid{Au-11_0106}}
% 7
{\PTglyphid{Au-11_0107}}
% 8
{\PTglyphid{Au-11_0108}}
% 9
{\PTglyphid{Au-11_0109}}
% 10
{\PTglyphid{Au-11_0110}}
% 11
{\PTglyphid{Au-11_0111}}
% 12
{\PTglyphid{Au-11_0112}}
% 13
{\PTglyphid{Au-11_0113}}
% 14
{\PTglyphid{Au-11_0114}}
% 15
{\PTglyphid{Au-11_0115}}
% 16
{\PTglyphid{Au-11_0116}}
% 17
{\PTglyphid{Au-11_0117}}
% 18
{\PTglyphid{Au-11_0118}}
% 19
{\PTglyphid{Au-11_0119}}
% 20
{\PTglyphid{Au-11_0120}}
% 21
{\PTglyphid{Au-11_0121}}
% 22
{\PTglyphid{Au-11_0122}}
% 23
{\PTglyphid{Au-11_0123}}
% 24
{\PTglyphid{Au-11_0124}}
% 25
{\PTglyphid{Au-11_0125}}
% 26
{\PTglyphid{Au-11_0126}}
% 27
{\PTglyphid{Au-11_0127}}
% 28
{\PTglyphid{Au-11_0201}}
% 29
{\PTglyphid{Au-11_0202}}
% 30
{\PTglyphid{Au-11_0203}}
% 31
{\PTglyphid{Au-11_0204}}
% 32
{\PTglyphid{Au-11_0205}}
//
\endgl \xe
%%% Local Variables:
%%% mode: latex
%%% TeX-engine: luatex
%%% TeX-master: shared
%%% End:

% //
%\endgl \xe

\newpage
 
%%%%%%%%%%%%%%%%%%%%%%%%%%%%%%%%%%%%%%%%%%%%%%%%%%%%%%%%%%%%%%%%%%%%%%%%%%%%%% 
% Tab. 14,Augezdecki-12_PT08_408.djvu,Augezdecki,12,08,408
%%%%%%%%%%%%%%%%%%%%%%%%%%%%%%%%%%%%%%%%%%%%%%%%%%%%%%%%%%%%%%%%%%%%%%%%%%%%%%

% Note "12. Wersaliki, antykwa. Wysokość 2—2,5 mm. — Tabl. 408."
% Note1 "Character set table prepared by Paulina Buchwald-Pelcowa"

\pismoPL{Aleksander Augezdecki 12. Wersaliki, antykwa. Wysokość 2—2,5 mm. — Tabl. 408 [piąty zestaw].}

\pismoEN{Aleksander Augezdecki 12. Roman capitals. Type size [1 line] = 2--2.5 mm. - Plate 408 [fifth set].}
% https://www.adfontes.uzh.ch/en/tutorium/schriften-lesen/schriftgeschichte/gotische-minuskeln-textura-und-textualis/
\medskip

\plate{408[5]}{VIII}{1972}

The plate    prepared by Paulina Buchwald-Pelcowa.\\
The font table    prepared by Paulina Buchwald-Pelcowa.\\
\relax[Layout confusing, misinterpretation possible --- JSB]


\bigskip

\fontID{Au-12}{14}

\fontstat{22}

% \exdisplay \bg \gla
 \exdisplay \bg \gla
% 1
{\PTglyph{5}{t14_l01g01.png}}
% 2
{\PTglyph{5}{t14_l01g02.png}}
% 3
{\PTglyph{5}{t14_l01g03.png}}
% 4
{\PTglyph{5}{t14_l01g04.png}}
% 5
{\PTglyph{5}{t14_l01g05.png}}
% 6
{\PTglyph{5}{t14_l01g06.png}}
% 7
{\PTglyph{5}{t14_l01g07.png}}
% 8
{\PTglyph{5}{t14_l01g08.png}}
% 9
{\PTglyph{5}{t14_l01g09.png}}
% 10
{\PTglyph{5}{t14_l01g10.png}}
% 11
{\PTglyph{5}{t14_l01g11.png}}
% 12
{\PTglyph{5}{t14_l01g12.png}}
% 13
{\PTglyph{5}{t14_l01g13.png}}
% 14
{\PTglyph{5}{t14_l01g14.png}}
% 15
{\PTglyph{5}{t14_l01g15.png}}
% 16
{\PTglyph{5}{t14_l01g16.png}}
% 17
{\PTglyph{5}{t14_l01g17.png}}
% 18
{\PTglyph{5}{t14_l01g18.png}}
% 19
{\PTglyph{5}{t14_l01g19.png}}
% 20
{\PTglyph{5}{t14_l01g20.png}}
% 21
{\PTglyph{5}{t14_l01g21.png}}
% 22
{\PTglyph{5}{t14_l01g22.png}}
//
%%% Local Variables:
%%% mode: latex
%%% TeX-engine: luatex
%%% TeX-master: shared
%%% End:

%//
%\glpismo%
 \glpismo
% 1
{\PTglyphid{Au-12_0101}}
% 2
{\PTglyphid{Au-12_0102}}
% 3
{\PTglyphid{Au-12_0103}}
% 4
{\PTglyphid{Au-12_0104}}
% 5
{\PTglyphid{Au-12_0105}}
% 6
{\PTglyphid{Au-12_0106}}
% 7
{\PTglyphid{Au-12_0107}}
% 8
{\PTglyphid{Au-12_0108}}
% 9
{\PTglyphid{Au-12_0109}}
% 10
{\PTglyphid{Au-12_0110}}
% 11
{\PTglyphid{Au-12_0111}}
% 12
{\PTglyphid{Au-12_0112}}
% 13
{\PTglyphid{Au-12_0113}}
% 14
{\PTglyphid{Au-12_0114}}
% 15
{\PTglyphid{Au-12_0115}}
% 16
{\PTglyphid{Au-12_0116}}
% 17
{\PTglyphid{Au-12_0117}}
% 18
{\PTglyphid{Au-12_0118}}
% 19
{\PTglyphid{Au-12_0119}}
% 20
{\PTglyphid{Au-12_0120}}
% 21
{\PTglyphid{Au-12_0121}}
% 22
{\PTglyphid{Au-12_0122}}
//
\endgl \xe
%%% Local Variables:
%%% mode: latex
%%% TeX-engine: luatex
%%% TeX-master: shared
%%% End:

% //
%\endgl \xe

\newpage
 

%%%%%%%%%%%%%%%%%%%%%%%%%%%%%%%%%%%%%%%%%%%%%%%%%%%%%%%%%%%%%%%%%%%%%%%%%%%%%% 
% Tab. 15,Augezdecki-13_PT08_411.djvu,Augezdecki,13,08,411 poprawić!!!
% Tab. 15,Augezdecki-13_PT08_412.djvu,Augezdecki,13,08,411
%%%%%%%%%%%%%%%%%%%%%%%%%%%%%%%%%%%%%%%%%%%%%%%%%%%%%%%%%%%%%%%%%%%%%%%%%%%%%%

% Note "13. Pismo nagłówkowe, fraktura H. Schönspergera. Stopień 1 w. = 8 mm. — Tabl. 412."

\pismoPL{Aleksander Augezdecki 13. Pismo  nagłówkowe,
  fraktura H. Schönspergera. Wysokość 1 w. = 8 mm. — Tabl. 412 [pierwszy zestaw].}

\pismoEN{Aleksander Augezdecki 13. H. Schönsperger's Fracture, header font. Type size 1 line = 8 mm. - Plate 412 [first set].}
% https://www.adfontes.uzh.ch/en/tutorium/schriften-lesen/schriftgeschichte/gotische-minuskeln-textura-und-textualis/
\medskip

\plate{412[1]}{VIII}{1972}

The plate    prepared by Paulina Buchwald-Pelcowa.\\
The font table    prepared by Paulina Buchwald-Pelcowa.\\
\relax[Layout confusing, misinterpretation possible --- JSB]

\bigskip

\fontID{Au-13}{15}

\fontstat{55}

% \exdisplay \bg \gla
 \exdisplay \bg \gla
% 1
{\PTglyph{5}{t15_l01g01.png}}
% 2
{\PTglyph{5}{t15_l01g02.png}}
% 3
{\PTglyph{5}{t15_l01g03.png}}
% 4
{\PTglyph{5}{t15_l01g04.png}}
% 5
{\PTglyph{5}{t15_l01g05.png}}
% 6
{\PTglyph{5}{t15_l01g06.png}}
% 7
{\PTglyph{5}{t15_l01g07.png}}
% 8
{\PTglyph{5}{t15_l01g08.png}}
% 9
{\PTglyph{5}{t15_l01g09.png}}
% 10
{\PTglyph{5}{t15_l01g10.png}}
% 11
{\PTglyph{5}{t15_l01g11.png}}
% 12
{\PTglyph{5}{t15_l01g12.png}}
% 13
{\PTglyph{5}{t15_l01g13.png}}
% 14
{\PTglyph{5}{t15_l02g01.png}}
% 15
{\PTglyph{5}{t15_l02g02.png}}
% 16
{\PTglyph{5}{t15_l02g03.png}}
% 17
{\PTglyph{5}{t15_l02g04.png}}
% 18
{\PTglyph{5}{t15_l02g05.png}}
% 19
{\PTglyph{5}{t15_l02g06.png}}
% 20
{\PTglyph{5}{t15_l02g07.png}}
% 21
{\PTglyph{5}{t15_l02g08.png}}
% 22
{\PTglyph{5}{t15_l02g09.png}}
% 23
{\PTglyph{5}{t15_l02g10.png}}
% 24
{\PTglyph{5}{t15_l02g11.png}}
% 25
{\PTglyph{5}{t15_l02g12.png}}
% 26
{\PTglyph{5}{t15_l02g13.png}}
% 27
{\PTglyph{5}{t15_l02g14.png}}
% 28
{\PTglyph{5}{t15_l02g15.png}}
% 29
{\PTglyph{5}{t15_l02g16.png}}
% 30
{\PTglyph{5}{t15_l02g17.png}}
% 31
{\PTglyph{5}{t15_l02g18.png}}
% 32
{\PTglyph{5}{t15_l02g19.png}}
% 33
{\PTglyph{5}{t15_l02g20.png}}
% 34
{\PTglyph{5}{t15_l02g21.png}}
% 35
{\PTglyph{5}{t15_l02g22.png}}
% 36
{\PTglyph{5}{t15_l02g23.png}}
% 37
{\PTglyph{5}{t15_l02g24.png}}
% 38
{\PTglyph{5}{t15_l02g25.png}}
% 39
{\PTglyph{5}{t15_l02g26.png}}
% 40
{\PTglyph{5}{t15_l02g27.png}}
% 41
{\PTglyph{5}{t15_l02g28.png}}
% 42
{\PTglyph{5}{t15_l02g29.png}}
% 43
{\PTglyph{5}{t15_l02g30.png}}
% 44
{\PTglyph{5}{t15_l02g31.png}}
% 45
{\PTglyph{5}{t15_l02g32.png}}
% 46
{\PTglyph{5}{t15_l02g33.png}}
% 47
{\PTglyph{5}{t15_l03g01.png}}
% 48
{\PTglyph{5}{t15_l03g02.png}}
% 49
{\PTglyph{5}{t15_l03g03.png}}
% 50
{\PTglyph{5}{t15_l03g04.png}}
% 51
{\PTglyph{5}{t15_l03g05.png}}
% 52
{\PTglyph{5}{t15_l03g06.png}}
% 53
{\PTglyph{5}{t15_l03g07.png}}
% 54
{\PTglyph{5}{t15_l03g08.png}}
% 55
{\PTglyph{5}{t15_l03g09.png}}
//
%%% Local Variables:
%%% mode: latex
%%% TeX-engine: luatex
%%% TeX-master: shared
%%% End:

%//
%\glpismo%
 \glpismo
% 1
{\PTglyphid{Au-13_0101}}
% 2
{\PTglyphid{Au-13_0102}}
% 3
{\PTglyphid{Au-13_0103}}
% 4
{\PTglyphid{Au-13_0104}}
% 5
{\PTglyphid{Au-13_0105}}
% 6
{\PTglyphid{Au-13_0106}}
% 7
{\PTglyphid{Au-13_0107}}
% 8
{\PTglyphid{Au-13_0108}}
% 9
{\PTglyphid{Au-13_0109}}
% 10
{\PTglyphid{Au-13_0110}}
% 11
{\PTglyphid{Au-13_0111}}
% 12
{\PTglyphid{Au-13_0112}}
% 13
{\PTglyphid{Au-13_0113}}
% 14
{\PTglyphid{Au-13_0201}}
% 15
{\PTglyphid{Au-13_0202}}
% 16
{\PTglyphid{Au-13_0203}}
% 17
{\PTglyphid{Au-13_0204}}
% 18
{\PTglyphid{Au-13_0205}}
% 19
{\PTglyphid{Au-13_0206}}
% 20
{\PTglyphid{Au-13_0207}}
% 21
{\PTglyphid{Au-13_0208}}
% 22
{\PTglyphid{Au-13_0209}}
% 23
{\PTglyphid{Au-13_0210}}
% 24
{\PTglyphid{Au-13_0211}}
% 25
{\PTglyphid{Au-13_0212}}
% 26
{\PTglyphid{Au-13_0213}}
% 27
{\PTglyphid{Au-13_0214}}
% 28
{\PTglyphid{Au-13_0215}}
% 29
{\PTglyphid{Au-13_0216}}
% 30
{\PTglyphid{Au-13_0217}}
% 31
{\PTglyphid{Au-13_0218}}
% 32
{\PTglyphid{Au-13_0219}}
% 33
{\PTglyphid{Au-13_0220}}
% 34
{\PTglyphid{Au-13_0221}}
% 35
{\PTglyphid{Au-13_0222}}
% 36
{\PTglyphid{Au-13_0223}}
% 37
{\PTglyphid{Au-13_0224}}
% 38
{\PTglyphid{Au-13_0225}}
% 39
{\PTglyphid{Au-13_0226}}
% 40
{\PTglyphid{Au-13_0227}}
% 41
{\PTglyphid{Au-13_0228}}
% 42
{\PTglyphid{Au-13_0229}}
% 43
{\PTglyphid{Au-13_0230}}
% 44
{\PTglyphid{Au-13_0231}}
% 45
{\PTglyphid{Au-13_0232}}
% 46
{\PTglyphid{Au-13_0233}}
% 47
{\PTglyphid{Au-13_0301}}
% 48
{\PTglyphid{Au-13_0302}}
% 49
{\PTglyphid{Au-13_0303}}
% 50
{\PTglyphid{Au-13_0304}}
% 51
{\PTglyphid{Au-13_0305}}
% 52
{\PTglyphid{Au-13_0306}}
% 53
{\PTglyphid{Au-13_0307}}
% 54
{\PTglyphid{Au-13_0308}}
% 55
{\PTglyphid{Au-13_0309}}
//
\endgl \xe
%%% Local Variables:
%%% mode: latex
%%% TeX-engine: luatex
%%% TeX-master: shared
%%% End:

% //
%\endgl \xe



\newpage
 

%%%%%%%%%%%%%%%%%%%%%%%%%%%%%%%%%%%%%%%%%%%%%%%%%%%%%%%%%%%%%%%%%%%%%%%%%%%%%% 
% Tab. 16,Augezdecki-14_PT08_410.djvu,Augezdecki,14,08,410
%%%%%%%%%%%%%%%%%%%%%%%%%%%%%%%%%%%%%%%%%%%%%%%%%%%%%%%%%%%%%%%%%%%%%%%%%%%%%%

% join!

% Note "14. Pismo tekstowe i nagłówkowe, fraktura H. Schönspergera. Stopień 20 ww. = 153 mm. — Tabl. 410, 411.[410]"
% Note1 "Character set table prepared by Paulina Buchwald-Pelcowa"

\pismoPL{Aleksander Augezdecki 14. Pismo tekstowe i nagłówkowe,
  fraktura H. Schönspergera. Stopień 20 ww. = 153 mm. — Tabl. 410
  [drugi zestaw], 411.}

\pismoEN{Aleksander Augezdecki 14. H. Schönsperger's Fracture, text
  and header font. Type size 20 lines = 153 mm. - Plate 410 [second
  set]m 411.}


\plate{410[2]}{VIII}{1972}

The plate    prepared by Paulina Buchwald-Pelcowa.\\
The font table    prepared by Paulina Buchwald-Pelcowa.\\

\bigskip

\fontID{Au-14}{16}

    \fontstat{144}
    % 108 ????

% \exdisplay \bg \gla
 \exdisplay \bg \gla
% 1
{\PTglyph{5}{t16_l01g01.png}}
% 2
{\PTglyph{5}{t16_l01g02.png}}
% 3
{\PTglyph{5}{t16_l01g03.png}}
% 4
{\PTglyph{5}{t16_l01g04.png}}
% 5
{\PTglyph{5}{t16_l01g05.png}}
% 6
{\PTglyph{5}{t16_l01g06.png}}
% 7
{\PTglyph{5}{t16_l01g07.png}}
% 8
{\PTglyph{5}{t16_l01g08.png}}
% 9
{\PTglyph{5}{t16_l01g09.png}}
% 10
{\PTglyph{5}{t16_l01g10.png}}
% 11
{\PTglyph{5}{t16_l01g11.png}}
% 12
{\PTglyph{5}{t16_l01g12.png}}
% 13
{\PTglyph{5}{t16_l01g13.png}}
% 14
{\PTglyph{5}{t16_l01g14.png}}
% 15
{\PTglyph{5}{t16_l01g15.png}}
% 16
{\PTglyph{5}{t16_l01g16.png}}
% 17
{\PTglyph{5}{t16_l01g17.png}}
% 18
{\PTglyph{5}{t16_l01g18.png}}
% 19
{\PTglyph{5}{t16_l02g01.png}}
% 20
{\PTglyph{5}{t16_l02g02.png}}
% 21
{\PTglyph{5}{t16_l02g03.png}}
% 22
{\PTglyph{5}{t16_l02g04.png}}
% 23
{\PTglyph{5}{t16_l02g05.png}}
% 24
{\PTglyph{5}{t16_l02g06.png}}
% 25
{\PTglyph{5}{t16_l02g07.png}}
% 26
{\PTglyph{5}{t16_l02g08.png}}
% 27
{\PTglyph{5}{t16_l03g01.png}}
% 28
{\PTglyph{5}{t16_l03g02.png}}
% 29
{\PTglyph{5}{t16_l03g03.png}}
% 30
{\PTglyph{5}{t16_l03g04.png}}
% 31
{\PTglyph{5}{t16_l03g05.png}}
% 32
{\PTglyph{5}{t16_l03g06.png}}
% 33
{\PTglyph{5}{t16_l03g07.png}}
% 34
{\PTglyph{5}{t16_l03g08.png}}
% 35
{\PTglyph{5}{t16_l03g09.png}}
% 36
{\PTglyph{5}{t16_l03g10.png}}
% 37
{\PTglyph{5}{t16_l03g11.png}}
% 38
{\PTglyph{5}{t16_l03g12.png}}
% 39
{\PTglyph{5}{t16_l03g13.png}}
% 40
{\PTglyph{5}{t16_l03g14.png}}
% 41
{\PTglyph{5}{t16_l03g15.png}}
% 42
{\PTglyph{5}{t16_l03g16.png}}
% 43
{\PTglyph{5}{t16_l03g17.png}}
% 44
{\PTglyph{5}{t16_l03g18.png}}
% 45
{\PTglyph{5}{t16_l03g19.png}}
% 46
{\PTglyph{5}{t16_l03g20.png}}
% 47
{\PTglyph{5}{t16_l03g21.png}}
% 48
{\PTglyph{5}{t16_l03g22.png}}
% 49
{\PTglyph{5}{t16_l03g23.png}}
% 50
{\PTglyph{5}{t16_l03g24.png}}
% 51
{\PTglyph{5}{t16_l03g25.png}}
% 52
{\PTglyph{5}{t16_l03g26.png}}
% 53
{\PTglyph{5}{t16_l03g27.png}}
% 54
{\PTglyph{5}{t16_l03g28.png}}
% 55
{\PTglyph{5}{t16_l03g29.png}}
% 56
{\PTglyph{5}{t16_l03g30.png}}
% 57
{\PTglyph{5}{t16_l03g31.png}}
% 58
{\PTglyph{5}{t16_l03g32.png}}
% 59
{\PTglyph{5}{t16_l03g33.png}}
% 60
{\PTglyph{5}{t16_l03g34.png}}
% 61
{\PTglyph{5}{t16_l04g01.png}}
% 62
{\PTglyph{5}{t16_l04g02.png}}
% 63
{\PTglyph{5}{t16_l04g03.png}}
% 64
{\PTglyph{5}{t16_l04g04.png}}
% 65
{\PTglyph{5}{t16_l04g05.png}}
% 66
{\PTglyph{5}{t16_l04g06.png}}
% 67
{\PTglyph{5}{t16_l04g07.png}}
% 68
{\PTglyph{5}{t16_l04g08.png}}
% 69
{\PTglyph{5}{t16_l04g09.png}}
% 70
{\PTglyph{5}{t16_l04g10.png}}
% 71
{\PTglyph{5}{t16_l04g11.png}}
% 72
{\PTglyph{5}{t16_l04g12.png}}
% 73
{\PTglyph{5}{t16_l04g13.png}}
% 74
{\PTglyph{5}{t16_l04g14.png}}
% 75
{\PTglyph{5}{t16_l04g15.png}}
% 76
{\PTglyph{5}{t16_l04g16.png}}
% 77
{\PTglyph{5}{t16_l04g17.png}}
% 78
{\PTglyph{5}{t16_l04g18.png}}
% 79
{\PTglyph{5}{t16_l04g19.png}}
% 80
{\PTglyph{5}{t16_l04g20.png}}
% 81
{\PTglyph{5}{t16_l04g21.png}}
% 82
{\PTglyph{5}{t16_l04g22.png}}
% 83
{\PTglyph{5}{t16_l04g23.png}}
% 84
{\PTglyph{5}{t16_l04g24.png}}
% 85
{\PTglyph{5}{t16_l04g25.png}}
% 86
{\PTglyph{5}{t16_l04g26.png}}
% 87
{\PTglyph{5}{t16_l04g27.png}}
% 88
{\PTglyph{5}{t16_l04g28.png}}
% 89
{\PTglyph{5}{t16_l04g29.png}}
% 90
{\PTglyph{5}{t16_l04g30.png}}
% 91
{\PTglyph{5}{t16_l04g31.png}}
% 92
{\PTglyph{5}{t16_l05g01.png}}
% 93
{\PTglyph{5}{t16_l05g02.png}}
% 94
{\PTglyph{5}{t16_l05g03.png}}
% 95
{\PTglyph{5}{t16_l05g04.png}}
% 96
{\PTglyph{5}{t16_l05g05.png}}
% 97
{\PTglyph{5}{t16_l05g06.png}}
% 98
{\PTglyph{5}{t16_l05g07.png}}
% 99
{\PTglyph{5}{t16_l05g08.png}}
% 100
{\PTglyph{5}{t16_l05g09.png}}
% 101
{\PTglyph{5}{t16_l05g10.png}}
% 102
{\PTglyph{5}{t16_l05g11.png}}
% 103
{\PTglyph{5}{t16_l05g12.png}}
% 104
{\PTglyph{5}{t16_l05g13.png}}
% 105
{\PTglyph{5}{t16_l05g14.png}}
% 106
{\PTglyph{5}{t16_l05g15.png}}
% 107
{\PTglyph{5}{t16_l05g16.png}}
% 108
{\PTglyph{5}{t16_l05g17.png}}
% 109
{\PTglyph{5}{t16_l05g18.png}}
% 110
{\PTglyph{5}{t16_l05g19.png}}
% 111
{\PTglyph{5}{t16_l05g20.png}}
% 112
{\PTglyph{5}{t16_l05g21.png}}
% 113
{\PTglyph{5}{t16_l05g22.png}}
% 114
{\PTglyph{5}{t16_l05g23.png}}
% 115
{\PTglyph{5}{t16_l05g24.png}}
% 116
{\PTglyph{5}{t16_l05g25.png}}
% 117
{\PTglyph{5}{t16_l05g26.png}}
% 118
{\PTglyph{5}{t16_l05g27.png}}
% 119
{\PTglyph{5}{t16_l05g28.png}}
% 120
{\PTglyph{5}{t16_l05g29.png}}
% 121
{\PTglyph{5}{t16_l05g30.png}}
% 122
{\PTglyph{5}{t16_l05g31.png}}
% 123
{\PTglyph{5}{t16_l05g32.png}}
% 124
{\PTglyph{5}{t16_l05g33.png}}
% 125
{\PTglyph{5}{t16_l06g01.png}}
% 126
{\PTglyph{5}{t16_l06g02.png}}
% 127
{\PTglyph{5}{t16_l06g03.png}}
% 128
{\PTglyph{5}{t16_l06g04.png}}
% 129
{\PTglyph{5}{t16_l06g05.png}}
% 130
{\PTglyph{5}{t16_l06g06.png}}
% 131
{\PTglyph{5}{t16_l06g07.png}}
% 132
{\PTglyph{5}{t16_l06g08.png}}
% 133
{\PTglyph{5}{t16_l06g09.png}}
% 134
{\PTglyph{5}{t16_l06g10.png}}
% 135
{\PTglyph{5}{t16_l06g11.png}}
% 136
{\PTglyph{5}{t16_l06g12.png}}
% 137
{\PTglyph{5}{t16_l06g13.png}}
% 138
{\PTglyph{5}{t16_l06g14.png}}
% 139
{\PTglyph{5}{t16_l06g15.png}}
% 140
{\PTglyph{5}{t16_l06g16.png}}
% 141
{\PTglyph{5}{t16_l06g17.png}}
% 142
{\PTglyph{5}{t16_l06g18.png}}
% 143
{\PTglyph{5}{t16_l06g19.png}}
% 144
{\PTglyph{5}{t16_l06g20.png}}
//
%%% Local Variables:
%%% mode: latex
%%% TeX-engine: luatex
%%% TeX-master: shared
%%% End:

%//
%\glpismo%
 \glpismo
% 1
{\PTglyphid{Au-14_0101}}
% 2
{\PTglyphid{Au-14_0102}}
% 3
{\PTglyphid{Au-14_0103}}
% 4
{\PTglyphid{Au-14_0104}}
% 5
{\PTglyphid{Au-14_0105}}
% 6
{\PTglyphid{Au-14_0106}}
% 7
{\PTglyphid{Au-14_0107}}
% 8
{\PTglyphid{Au-14_0108}}
% 9
{\PTglyphid{Au-14_0109}}
% 10
{\PTglyphid{Au-14_0110}}
% 11
{\PTglyphid{Au-14_0111}}
% 12
{\PTglyphid{Au-14_0112}}
% 13
{\PTglyphid{Au-14_0113}}
% 14
{\PTglyphid{Au-14_0114}}
% 15
{\PTglyphid{Au-14_0115}}
% 16
{\PTglyphid{Au-14_0116}}
% 17
{\PTglyphid{Au-14_0117}}
% 18
{\PTglyphid{Au-14_0118}}
% 19
{\PTglyphid{Au-14_0201}}
% 20
{\PTglyphid{Au-14_0202}}
% 21
{\PTglyphid{Au-14_0203}}
% 22
{\PTglyphid{Au-14_0204}}
% 23
{\PTglyphid{Au-14_0205}}
% 24
{\PTglyphid{Au-14_0206}}
% 25
{\PTglyphid{Au-14_0207}}
% 26
{\PTglyphid{Au-14_0208}}
% 27
{\PTglyphid{Au-14_0301}}
% 28
{\PTglyphid{Au-14_0302}}
% 29
{\PTglyphid{Au-14_0303}}
% 30
{\PTglyphid{Au-14_0304}}
% 31
{\PTglyphid{Au-14_0305}}
% 32
{\PTglyphid{Au-14_0306}}
% 33
{\PTglyphid{Au-14_0307}}
% 34
{\PTglyphid{Au-14_0308}}
% 35
{\PTglyphid{Au-14_0309}}
% 36
{\PTglyphid{Au-14_0310}}
% 37
{\PTglyphid{Au-14_0311}}
% 38
{\PTglyphid{Au-14_0312}}
% 39
{\PTglyphid{Au-14_0313}}
% 40
{\PTglyphid{Au-14_0314}}
% 41
{\PTglyphid{Au-14_0315}}
% 42
{\PTglyphid{Au-14_0316}}
% 43
{\PTglyphid{Au-14_0317}}
% 44
{\PTglyphid{Au-14_0318}}
% 45
{\PTglyphid{Au-14_0319}}
% 46
{\PTglyphid{Au-14_0320}}
% 47
{\PTglyphid{Au-14_0321}}
% 48
{\PTglyphid{Au-14_0322}}
% 49
{\PTglyphid{Au-14_0323}}
% 50
{\PTglyphid{Au-14_0324}}
% 51
{\PTglyphid{Au-14_0325}}
% 52
{\PTglyphid{Au-14_0326}}
% 53
{\PTglyphid{Au-14_0327}}
% 54
{\PTglyphid{Au-14_0328}}
% 55
{\PTglyphid{Au-14_0329}}
% 56
{\PTglyphid{Au-14_0330}}
% 57
{\PTglyphid{Au-14_0331}}
% 58
{\PTglyphid{Au-14_0332}}
% 59
{\PTglyphid{Au-14_0333}}
% 60
{\PTglyphid{Au-14_0334}}
% 61
{\PTglyphid{Au-14_0401}}
% 62
{\PTglyphid{Au-14_0402}}
% 63
{\PTglyphid{Au-14_0403}}
% 64
{\PTglyphid{Au-14_0404}}
% 65
{\PTglyphid{Au-14_0405}}
% 66
{\PTglyphid{Au-14_0406}}
% 67
{\PTglyphid{Au-14_0407}}
% 68
{\PTglyphid{Au-14_0408}}
% 69
{\PTglyphid{Au-14_0409}}
% 70
{\PTglyphid{Au-14_0410}}
% 71
{\PTglyphid{Au-14_0411}}
% 72
{\PTglyphid{Au-14_0412}}
% 73
{\PTglyphid{Au-14_0413}}
% 74
{\PTglyphid{Au-14_0414}}
% 75
{\PTglyphid{Au-14_0415}}
% 76
{\PTglyphid{Au-14_0416}}
% 77
{\PTglyphid{Au-14_0417}}
% 78
{\PTglyphid{Au-14_0418}}
% 79
{\PTglyphid{Au-14_0419}}
% 80
{\PTglyphid{Au-14_0420}}
% 81
{\PTglyphid{Au-14_0421}}
% 82
{\PTglyphid{Au-14_0422}}
% 83
{\PTglyphid{Au-14_0423}}
% 84
{\PTglyphid{Au-14_0424}}
% 85
{\PTglyphid{Au-14_0425}}
% 86
{\PTglyphid{Au-14_0426}}
% 87
{\PTglyphid{Au-14_0427}}
% 88
{\PTglyphid{Au-14_0428}}
% 89
{\PTglyphid{Au-14_0429}}
% 90
{\PTglyphid{Au-14_0430}}
% 91
{\PTglyphid{Au-14_0431}}
% 92
{\PTglyphid{Au-14_0501}}
% 93
{\PTglyphid{Au-14_0502}}
% 94
{\PTglyphid{Au-14_0503}}
% 95
{\PTglyphid{Au-14_0504}}
% 96
{\PTglyphid{Au-14_0505}}
% 97
{\PTglyphid{Au-14_0506}}
% 98
{\PTglyphid{Au-14_0507}}
% 99
{\PTglyphid{Au-14_0508}}
% 100
{\PTglyphid{Au-14_0509}}
% 101
{\PTglyphid{Au-14_0510}}
% 102
{\PTglyphid{Au-14_0511}}
% 103
{\PTglyphid{Au-14_0512}}
% 104
{\PTglyphid{Au-14_0513}}
% 105
{\PTglyphid{Au-14_0514}}
% 106
{\PTglyphid{Au-14_0515}}
% 107
{\PTglyphid{Au-14_0516}}
% 108
{\PTglyphid{Au-14_0517}}
% 109
{\PTglyphid{Au-14_0518}}
% 110
{\PTglyphid{Au-14_0519}}
% 111
{\PTglyphid{Au-14_0520}}
% 112
{\PTglyphid{Au-14_0521}}
% 113
{\PTglyphid{Au-14_0522}}
% 114
{\PTglyphid{Au-14_0523}}
% 115
{\PTglyphid{Au-14_0524}}
% 116
{\PTglyphid{Au-14_0525}}
% 117
{\PTglyphid{Au-14_0526}}
% 118
{\PTglyphid{Au-14_0527}}
% 119
{\PTglyphid{Au-14_0528}}
% 120
{\PTglyphid{Au-14_0529}}
% 121
{\PTglyphid{Au-14_0530}}
% 122
{\PTglyphid{Au-14_0531}}
% 123
{\PTglyphid{Au-14_0532}}
% 124
{\PTglyphid{Au-14_0533}}
% 125
{\PTglyphid{Au-14_0601}}
% 126
{\PTglyphid{Au-14_0602}}
% 127
{\PTglyphid{Au-14_0603}}
% 128
{\PTglyphid{Au-14_0604}}
% 129
{\PTglyphid{Au-14_0605}}
% 130
{\PTglyphid{Au-14_0606}}
% 131
{\PTglyphid{Au-14_0607}}
% 132
{\PTglyphid{Au-14_0608}}
% 133
{\PTglyphid{Au-14_0609}}
% 134
{\PTglyphid{Au-14_0610}}
% 135
{\PTglyphid{Au-14_0611}}
% 136
{\PTglyphid{Au-14_0612}}
% 137
{\PTglyphid{Au-14_0613}}
% 138
{\PTglyphid{Au-14_0614}}
% 139
{\PTglyphid{Au-14_0615}}
% 140
{\PTglyphid{Au-14_0616}}
% 141
{\PTglyphid{Au-14_0617}}
% 142
{\PTglyphid{Au-14_0618}}
% 143
{\PTglyphid{Au-14_0619}}
% 144
{\PTglyphid{Au-14_0620}}
//
\endgl \xe
%%% Local Variables:
%%% mode: latex
%%% TeX-engine: luatex
%%% TeX-master: shared
%%% End:

% //
%\endgl \xe


\newpage

%%%%%%%%%%%%%%%%%%%%%%%%%%%%%%%%%%%%%%%%%%%%%%%%%%%%%%%%%%%%%%%%%%%%%%%%%%%%%% 
% Tab. 17,Augezdecki-15_PT08_411.djvu,Augezdecki,15,08,411 poprawić!!!
% Tab. 17,Augezdecki-15_PT08_412.djvu,Augezdecki,15,08,411
%%%%%%%%%%%%%%%%%%%%%%%%%%%%%%%%%%%%%%%%%%%%%%%%%%%%%%%%%%%%%%%%%%%%%%%%%%%%%%

% Note "15. Pismo tytułowe i nagłówkowe, fraktura H. Schönspergera. Wysokość 1 w. = 11—12 mm bez przedłużek. — Tabl. 412."
% Note1 "Character set table prepared by Paulina Buchwald-Pelcowa"

\pismoPL{Aleksander Augezdecki 15. Pismo tytułowe i nagłówkowe,
  fraktura H. Schönspergera. Wysokość 1 w. = 11—12 mm bez
  przedłużek. — Tabl. 412 [drugi zestaw].}

\pismoEN{Aleksander Augezdecki 15. H. Schönsperger's Fracture, title
  and header font. Type size 1 line = 11--12 mm without ascenders and
  descenders. - Plate 412 [second set].}

\plate{412[2]}{VIII}{1972}

The plate    prepared by Paulina Buchwald-Pelcowa.\\
The font table    prepared by Paulina Buchwald-Pelcowa.\\
\relax[Layout confusing, misinterpretation possible --- JSB]


\bigskip

\fontID{Au-15}{17}

\fontstat{109}

% \exdisplay \bg \gla
 \exdisplay \bg \gla
% 1
{\PTglyph{5}{t17_l01g01.png}}
% 2
{\PTglyph{5}{t17_l01g02.png}}
% 3
{\PTglyph{5}{t17_l01g03.png}}
% 4
{\PTglyph{5}{t17_l01g04.png}}
% 5
{\PTglyph{5}{t17_l01g05.png}}
% 6
{\PTglyph{5}{t17_l01g06.png}}
% 7
{\PTglyph{5}{t17_l01g07.png}}
% 8
{\PTglyph{5}{t17_l01g08.png}}
% 9
{\PTglyph{5}{t17_l01g09.png}}
% 10
{\PTglyph{5}{t17_l01g10.png}}
% 11
{\PTglyph{5}{t17_l01g11.png}}
% 12
{\PTglyph{5}{t17_l01g12.png}}
% 13
{\PTglyph{5}{t17_l01g13.png}}
% 14
{\PTglyph{5}{t17_l01g14.png}}
% 15
{\PTglyph{5}{t17_l01g15.png}}
% 16
{\PTglyph{5}{t17_l01g16.png}}
% 17
{\PTglyph{5}{t17_l02g01.png}}
% 18
{\PTglyph{5}{t17_l02g02.png}}
% 19
{\PTglyph{5}{t17_l02g03.png}}
% 20
{\PTglyph{5}{t17_l02g04.png}}
% 21
{\PTglyph{5}{t17_l02g05.png}}
% 22
{\PTglyph{5}{t17_l02g06.png}}
% 23
{\PTglyph{5}{t17_l02g07.png}}
% 24
{\PTglyph{5}{t17_l02g08.png}}
% 25
{\PTglyph{5}{t17_l02g09.png}}
% 26
{\PTglyph{5}{t17_l02g10.png}}
% 27
{\PTglyph{5}{t17_l02g11.png}}
% 28
{\PTglyph{5}{t17_l02g12.png}}
% 29
{\PTglyph{5}{t17_l02g13.png}}
% 30
{\PTglyph{5}{t17_l02g14.png}}
% 31
{\PTglyph{5}{t17_l03g01.png}}
% 32
{\PTglyph{5}{t17_l03g02.png}}
% 33
{\PTglyph{5}{t17_l03g03.png}}
% 34
{\PTglyph{5}{t17_l03g04.png}}
% 35
{\PTglyph{5}{t17_l03g05.png}}
% 36
{\PTglyph{5}{t17_l03g06.png}}
% 37
{\PTglyph{5}{t17_l03g07.png}}
% 38
{\PTglyph{5}{t17_l03g08.png}}
% 39
{\PTglyph{5}{t17_l03g09.png}}
% 40
{\PTglyph{5}{t17_l03g10.png}}
% 41
{\PTglyph{5}{t17_l03g11.png}}
% 42
{\PTglyph{5}{t17_l03g12.png}}
% 43
{\PTglyph{5}{t17_l03g13.png}}
% 44
{\PTglyph{5}{t17_l03g14.png}}
% 45
{\PTglyph{5}{t17_l03g15.png}}
% 46
{\PTglyph{5}{t17_l03g16.png}}
% 47
{\PTglyph{5}{t17_l03g17.png}}
% 48
{\PTglyph{5}{t17_l03g18.png}}
% 49
{\PTglyph{5}{t17_l03g19.png}}
% 50
{\PTglyph{5}{t17_l03g20.png}}
% 51
{\PTglyph{5}{t17_l03g21.png}}
% 52
{\PTglyph{5}{t17_l03g22.png}}
% 53
{\PTglyph{5}{t17_l03g23.png}}
% 54
{\PTglyph{5}{t17_l03g24.png}}
% 55
{\PTglyph{5}{t17_l03g25.png}}
% 56
{\PTglyph{5}{t17_l03g26.png}}
% 57
{\PTglyph{5}{t17_l03g27.png}}
% 58
{\PTglyph{5}{t17_l03g28.png}}
% 59
{\PTglyph{5}{t17_l03g29.png}}
% 60
{\PTglyph{5}{t17_l03g30.png}}
% 61
{\PTglyph{5}{t17_l04g01.png}}
% 62
{\PTglyph{5}{t17_l04g02.png}}
% 63
{\PTglyph{5}{t17_l04g03.png}}
% 64
{\PTglyph{5}{t17_l04g04.png}}
% 65
{\PTglyph{5}{t17_l04g05.png}}
% 66
{\PTglyph{5}{t17_l04g06.png}}
% 67
{\PTglyph{5}{t17_l04g07.png}}
% 68
{\PTglyph{5}{t17_l04g08.png}}
% 69
{\PTglyph{5}{t17_l04g09.png}}
% 70
{\PTglyph{5}{t17_l04g10.png}}
% 71
{\PTglyph{5}{t17_l04g11.png}}
% 72
{\PTglyph{5}{t17_l04g12.png}}
% 73
{\PTglyph{5}{t17_l04g13.png}}
% 74
{\PTglyph{5}{t17_l04g14.png}}
% 75
{\PTglyph{5}{t17_l04g15.png}}
% 76
{\PTglyph{5}{t17_l04g16.png}}
% 77
{\PTglyph{5}{t17_l04g17.png}}
% 78
{\PTglyph{5}{t17_l04g18.png}}
% 79
{\PTglyph{5}{t17_l04g19.png}}
% 80
{\PTglyph{5}{t17_l04g20.png}}
% 81
{\PTglyph{5}{t17_l04g21.png}}
% 82
{\PTglyph{5}{t17_l04g22.png}}
% 83
{\PTglyph{5}{t17_l04g23.png}}
% 84
{\PTglyph{5}{t17_l04g24.png}}
% 85
{\PTglyph{5}{t17_l04g25.png}}
% 86
{\PTglyph{5}{t17_l04g26.png}}
% 87
{\PTglyph{5}{t17_l05g01.png}}
% 88
{\PTglyph{5}{t17_l05g02.png}}
% 89
{\PTglyph{5}{t17_l05g03.png}}
% 90
{\PTglyph{5}{t17_l05g04.png}}
% 91
{\PTglyph{5}{t17_l05g05.png}}
% 92
{\PTglyph{5}{t17_l05g06.png}}
% 93
{\PTglyph{5}{t17_l06g01.png}}
% 94
{\PTglyph{5}{t17_l06g02.png}}
% 95
{\PTglyph{5}{t17_l06g03.png}}
% 96
{\PTglyph{5}{t17_l06g04.png}}
% 97
{\PTglyph{5}{t17_l06g05.png}}
% 98
{\PTglyph{5}{t17_l06g06.png}}
% 99
{\PTglyph{5}{t17_l06g07.png}}
% 100
{\PTglyph{5}{t17_l06g08.png}}
% 101
{\PTglyph{5}{t17_l06g09.png}}
% 102
{\PTglyph{5}{t17_l06g10.png}}
% 103
{\PTglyph{5}{t17_l06g11.png}}
% 104
{\PTglyph{5}{t17_l06g12.png}}
% 105
{\PTglyph{5}{t17_l06g13.png}}
% 106
{\PTglyph{5}{t17_l06g14.png}}
% 107
{\PTglyph{5}{t17_l06g15.png}}
% 108
{\PTglyph{5}{t17_l06g16.png}}
% 109
{\PTglyph{5}{t17_l06g17.png}}
//
%%% Local Variables:
%%% mode: latex
%%% TeX-engine: luatex
%%% TeX-master: shared
%%% End:

%//
%\glpismo%
 \glpismo
% 1
{\PTglyphid{Au-15_0101}}
% 2
{\PTglyphid{Au-15_0102}}
% 3
{\PTglyphid{Au-15_0103}}
% 4
{\PTglyphid{Au-15_0104}}
% 5
{\PTglyphid{Au-15_0105}}
% 6
{\PTglyphid{Au-15_0106}}
% 7
{\PTglyphid{Au-15_0107}}
% 8
{\PTglyphid{Au-15_0108}}
% 9
{\PTglyphid{Au-15_0109}}
% 10
{\PTglyphid{Au-15_0110}}
% 11
{\PTglyphid{Au-15_0111}}
% 12
{\PTglyphid{Au-15_0112}}
% 13
{\PTglyphid{Au-15_0113}}
% 14
{\PTglyphid{Au-15_0114}}
% 15
{\PTglyphid{Au-15_0115}}
% 16
{\PTglyphid{Au-15_0116}}
% 17
{\PTglyphid{Au-15_0201}}
% 18
{\PTglyphid{Au-15_0202}}
% 19
{\PTglyphid{Au-15_0203}}
% 20
{\PTglyphid{Au-15_0204}}
% 21
{\PTglyphid{Au-15_0205}}
% 22
{\PTglyphid{Au-15_0206}}
% 23
{\PTglyphid{Au-15_0207}}
% 24
{\PTglyphid{Au-15_0208}}
% 25
{\PTglyphid{Au-15_0209}}
% 26
{\PTglyphid{Au-15_0210}}
% 27
{\PTglyphid{Au-15_0211}}
% 28
{\PTglyphid{Au-15_0212}}
% 29
{\PTglyphid{Au-15_0213}}
% 30
{\PTglyphid{Au-15_0214}}
% 31
{\PTglyphid{Au-15_0301}}
% 32
{\PTglyphid{Au-15_0302}}
% 33
{\PTglyphid{Au-15_0303}}
% 34
{\PTglyphid{Au-15_0304}}
% 35
{\PTglyphid{Au-15_0305}}
% 36
{\PTglyphid{Au-15_0306}}
% 37
{\PTglyphid{Au-15_0307}}
% 38
{\PTglyphid{Au-15_0308}}
% 39
{\PTglyphid{Au-15_0309}}
% 40
{\PTglyphid{Au-15_0310}}
% 41
{\PTglyphid{Au-15_0311}}
% 42
{\PTglyphid{Au-15_0312}}
% 43
{\PTglyphid{Au-15_0313}}
% 44
{\PTglyphid{Au-15_0314}}
% 45
{\PTglyphid{Au-15_0315}}
% 46
{\PTglyphid{Au-15_0316}}
% 47
{\PTglyphid{Au-15_0317}}
% 48
{\PTglyphid{Au-15_0318}}
% 49
{\PTglyphid{Au-15_0319}}
% 50
{\PTglyphid{Au-15_0320}}
% 51
{\PTglyphid{Au-15_0321}}
% 52
{\PTglyphid{Au-15_0322}}
% 53
{\PTglyphid{Au-15_0323}}
% 54
{\PTglyphid{Au-15_0324}}
% 55
{\PTglyphid{Au-15_0325}}
% 56
{\PTglyphid{Au-15_0326}}
% 57
{\PTglyphid{Au-15_0327}}
% 58
{\PTglyphid{Au-15_0328}}
% 59
{\PTglyphid{Au-15_0329}}
% 60
{\PTglyphid{Au-15_0330}}
% 61
{\PTglyphid{Au-15_0401}}
% 62
{\PTglyphid{Au-15_0402}}
% 63
{\PTglyphid{Au-15_0403}}
% 64
{\PTglyphid{Au-15_0404}}
% 65
{\PTglyphid{Au-15_0405}}
% 66
{\PTglyphid{Au-15_0406}}
% 67
{\PTglyphid{Au-15_0407}}
% 68
{\PTglyphid{Au-15_0408}}
% 69
{\PTglyphid{Au-15_0409}}
% 70
{\PTglyphid{Au-15_0410}}
% 71
{\PTglyphid{Au-15_0411}}
% 72
{\PTglyphid{Au-15_0412}}
% 73
{\PTglyphid{Au-15_0413}}
% 74
{\PTglyphid{Au-15_0414}}
% 75
{\PTglyphid{Au-15_0415}}
% 76
{\PTglyphid{Au-15_0416}}
% 77
{\PTglyphid{Au-15_0417}}
% 78
{\PTglyphid{Au-15_0418}}
% 79
{\PTglyphid{Au-15_0419}}
% 80
{\PTglyphid{Au-15_0420}}
% 81
{\PTglyphid{Au-15_0421}}
% 82
{\PTglyphid{Au-15_0422}}
% 83
{\PTglyphid{Au-15_0423}}
% 84
{\PTglyphid{Au-15_0424}}
% 85
{\PTglyphid{Au-15_0425}}
% 86
{\PTglyphid{Au-15_0426}}
% 87
{\PTglyphid{Au-15_0501}}
% 88
{\PTglyphid{Au-15_0502}}
% 89
{\PTglyphid{Au-15_0503}}
% 90
{\PTglyphid{Au-15_0504}}
% 91
{\PTglyphid{Au-15_0505}}
% 92
{\PTglyphid{Au-15_0506}}
% 93
{\PTglyphid{Au-15_0601}}
% 94
{\PTglyphid{Au-15_0602}}
% 95
{\PTglyphid{Au-15_0603}}
% 96
{\PTglyphid{Au-15_0604}}
% 97
{\PTglyphid{Au-15_0605}}
% 98
{\PTglyphid{Au-15_0606}}
% 99
{\PTglyphid{Au-15_0607}}
% 100
{\PTglyphid{Au-15_0608}}
% 101
{\PTglyphid{Au-15_0609}}
% 102
{\PTglyphid{Au-15_0610}}
% 103
{\PTglyphid{Au-15_0611}}
% 104
{\PTglyphid{Au-15_0612}}
% 105
{\PTglyphid{Au-15_0613}}
% 106
{\PTglyphid{Au-15_0614}}
% 107
{\PTglyphid{Au-15_0615}}
% 108
{\PTglyphid{Au-15_0616}}
% 109
{\PTglyphid{Au-15_0617}}
//
\endgl \xe
%%% Local Variables:
%%% mode: latex
%%% TeX-engine: luatex
%%% TeX-master: shared
%%% End:

% //
%\endgl \xe


 \newpage

%%%%%%%%%%%%%%%%%%%%%%%%%%%%%%%%%%%%%%%%%%%%%%%%%%%%%%%%%%%%%%%%%%%%%%%%%%%%%% 
% Tab. 18,Augezdecki-16_PT08_413.djvu,Augezdecki,16,08,413
%%%%%%%%%%%%%%%%%%%%%%%%%%%%%%%%%%%%%%%%%%%%%%%%%%%%%%%%%%%%%%%%%%%%%%%%%%%%%%

% Note "16. Pismo tekstowe, szwabacha M⁸¹. Stopień 20 ww. = 88 mm. — Tabl. 413."
% Note1 "Character set table prepared by Paulina Buchwald-Pelcowa"


\pismoPL{Aleksander Augezdecki 16. Pismo tekstowe, szwabacha M⁸¹. Stopień 20 ww. = 88 mm. — Tabl. 413.}

\pismoEN{Aleksander Augezdecki 16. Text type, Schwabacher M⁸¹. Type size 20 lines = 88 mm. — Tabl. 413.}
% https://www.adfontes.uzh.ch/en/tutorium/schriften-lesen/schriftgeschichte/gotische-minuskeln-textura-und-textualis/
\medskip

\plate{413}{VIII}{1972}

Prepared by Paulina Buchwald-Pelcowa.

\bigskip

 \exampleBib{VIII:20}

\bigskip
\exampleDesc{IACOBUS KUCHLER: Epithalamion de nuptiis Andreae comitis in Gorka.
Szamotuly, Aleksander Augezdecki, 20 X 1558. 4°.}

\medskip
\examplePage{\textit{Karta A₃a.}}

  \bigskip
\exampleLib{Książnica Miejska im. Kopernika. Toruń}

\bigskip
\exampleRef{\textit{Drukarze IV 29.}}

% \bigskip
% \exampleDig{\url{https://dbc.wroc.pl/dlibra/publication/15990/edition/14101} page 13}

\medskip

    \examplePL{Pismo 16: tekst i zestaw.}

    \medskip

    \exampleEN{Font 16. A text and the font table.}


\bigskip


\fontID{Au-16}{18}

\fontstat{74}

% \exdisplay \bg \gla
 \exdisplay \bg \gla
% 1
{\PTglyph{5}{t18_l01g01.png}}
% 2
{\PTglyph{5}{t18_l01g02.png}}
% 3
{\PTglyph{5}{t18_l01g03.png}}
% 4
{\PTglyph{5}{t18_l01g04.png}}
% 5
{\PTglyph{5}{t18_l01g05.png}}
% 6
{\PTglyph{5}{t18_l01g06.png}}
% 7
{\PTglyph{5}{t18_l01g07.png}}
% 8
{\PTglyph{5}{t18_l01g08.png}}
% 9
{\PTglyph{5}{t18_l01g09.png}}
% 10
{\PTglyph{5}{t18_l01g10.png}}
% 11
{\PTglyph{5}{t18_l01g11.png}}
% 12
{\PTglyph{5}{t18_l01g12.png}}
% 13
{\PTglyph{5}{t18_l01g13.png}}
% 14
{\PTglyph{5}{t18_l01g14.png}}
% 15
{\PTglyph{5}{t18_l01g15.png}}
% 16
{\PTglyph{5}{t18_l01g16.png}}
% 17
{\PTglyph{5}{t18_l01g17.png}}
% 18
{\PTglyph{5}{t18_l01g18.png}}
% 19
{\PTglyph{5}{t18_l01g19.png}}
% 20
{\PTglyph{5}{t18_l02g01.png}}
% 21
{\PTglyph{5}{t18_l02g02.png}}
% 22
{\PTglyph{5}{t18_l02g03.png}}
% 23
{\PTglyph{5}{t18_l02g04.png}}
% 24
{\PTglyph{5}{t18_l02g05.png}}
% 25
{\PTglyph{5}{t18_l02g06.png}}
% 26
{\PTglyph{5}{t18_l02g07.png}}
% 27
{\PTglyph{5}{t18_l02g08.png}}
% 28
{\PTglyph{5}{t18_l02g09.png}}
% 29
{\PTglyph{5}{t18_l02g10.png}}
% 30
{\PTglyph{5}{t18_l02g11.png}}
% 31
{\PTglyph{5}{t18_l02g12.png}}
% 32
{\PTglyph{5}{t18_l02g13.png}}
% 33
{\PTglyph{5}{t18_l02g14.png}}
% 34
{\PTglyph{5}{t18_l02g15.png}}
% 35
{\PTglyph{5}{t18_l02g16.png}}
% 36
{\PTglyph{5}{t18_l02g17.png}}
% 37
{\PTglyph{5}{t18_l02g18.png}}
% 38
{\PTglyph{5}{t18_l02g19.png}}
% 39
{\PTglyph{5}{t18_l02g20.png}}
% 40
{\PTglyph{5}{t18_l02g21.png}}
% 41
{\PTglyph{5}{t18_l02g22.png}}
% 42
{\PTglyph{5}{t18_l02g23.png}}
% 43
{\PTglyph{5}{t18_l02g24.png}}
% 44
{\PTglyph{5}{t18_l02g25.png}}
% 45
{\PTglyph{5}{t18_l02g26.png}}
% 46
{\PTglyph{5}{t18_l02g27.png}}
% 47
{\PTglyph{5}{t18_l02g28.png}}
% 48
{\PTglyph{5}{t18_l02g29.png}}
% 49
{\PTglyph{5}{t18_l02g30.png}}
% 50
{\PTglyph{5}{t18_l02g31.png}}
% 51
{\PTglyph{5}{t18_l02g32.png}}
% 52
{\PTglyph{5}{t18_l02g33.png}}
% 53
{\PTglyph{5}{t18_l02g34.png}}
% 54
{\PTglyph{5}{t18_l02g35.png}}
% 55
{\PTglyph{5}{t18_l02g36.png}}
% 56
{\PTglyph{5}{t18_l02g37.png}}
% 57
{\PTglyph{5}{t18_l02g38.png}}
% 58
{\PTglyph{5}{t18_l03g01.png}}
% 59
{\PTglyph{5}{t18_l03g02.png}}
% 60
{\PTglyph{5}{t18_l03g03.png}}
% 61
{\PTglyph{5}{t18_l03g04.png}}
% 62
{\PTglyph{5}{t18_l03g05.png}}
% 63
{\PTglyph{5}{t18_l03g06.png}}
% 64
{\PTglyph{5}{t18_l03g07.png}}
% 65
{\PTglyph{5}{t18_l03g08.png}}
% 66
{\PTglyph{5}{t18_l03g09.png}}
% 67
{\PTglyph{5}{t18_l03g10.png}}
% 68
{\PTglyph{5}{t18_l03g11.png}}
% 69
{\PTglyph{5}{t18_l03g12.png}}
% 70
{\PTglyph{5}{t18_l03g13.png}}
% 71
{\PTglyph{5}{t18_l03g14.png}}
% 72
{\PTglyph{5}{t18_l03g15.png}}
% 73
{\PTglyph{5}{t18_l03g16.png}}
% 74
{\PTglyph{5}{t18_l03g17.png}}
//
%%% Local Variables:
%%% mode: latex
%%% TeX-engine: luatex
%%% TeX-master: shared
%%% End:

%//
%\glpismo%
 \glpismo
% 1
{\PTglyphid{Au-16_0101}}
% 2
{\PTglyphid{Au-16_0102}}
% 3
{\PTglyphid{Au-16_0103}}
% 4
{\PTglyphid{Au-16_0104}}
% 5
{\PTglyphid{Au-16_0105}}
% 6
{\PTglyphid{Au-16_0106}}
% 7
{\PTglyphid{Au-16_0107}}
% 8
{\PTglyphid{Au-16_0108}}
% 9
{\PTglyphid{Au-16_0109}}
% 10
{\PTglyphid{Au-16_0110}}
% 11
{\PTglyphid{Au-16_0111}}
% 12
{\PTglyphid{Au-16_0112}}
% 13
{\PTglyphid{Au-16_0113}}
% 14
{\PTglyphid{Au-16_0114}}
% 15
{\PTglyphid{Au-16_0115}}
% 16
{\PTglyphid{Au-16_0116}}
% 17
{\PTglyphid{Au-16_0117}}
% 18
{\PTglyphid{Au-16_0118}}
% 19
{\PTglyphid{Au-16_0119}}
% 20
{\PTglyphid{Au-16_0201}}
% 21
{\PTglyphid{Au-16_0202}}
% 22
{\PTglyphid{Au-16_0203}}
% 23
{\PTglyphid{Au-16_0204}}
% 24
{\PTglyphid{Au-16_0205}}
% 25
{\PTglyphid{Au-16_0206}}
% 26
{\PTglyphid{Au-16_0207}}
% 27
{\PTglyphid{Au-16_0208}}
% 28
{\PTglyphid{Au-16_0209}}
% 29
{\PTglyphid{Au-16_0210}}
% 30
{\PTglyphid{Au-16_0211}}
% 31
{\PTglyphid{Au-16_0212}}
% 32
{\PTglyphid{Au-16_0213}}
% 33
{\PTglyphid{Au-16_0214}}
% 34
{\PTglyphid{Au-16_0215}}
% 35
{\PTglyphid{Au-16_0216}}
% 36
{\PTglyphid{Au-16_0217}}
% 37
{\PTglyphid{Au-16_0218}}
% 38
{\PTglyphid{Au-16_0219}}
% 39
{\PTglyphid{Au-16_0220}}
% 40
{\PTglyphid{Au-16_0221}}
% 41
{\PTglyphid{Au-16_0222}}
% 42
{\PTglyphid{Au-16_0223}}
% 43
{\PTglyphid{Au-16_0224}}
% 44
{\PTglyphid{Au-16_0225}}
% 45
{\PTglyphid{Au-16_0226}}
% 46
{\PTglyphid{Au-16_0227}}
% 47
{\PTglyphid{Au-16_0228}}
% 48
{\PTglyphid{Au-16_0229}}
% 49
{\PTglyphid{Au-16_0230}}
% 50
{\PTglyphid{Au-16_0231}}
% 51
{\PTglyphid{Au-16_0232}}
% 52
{\PTglyphid{Au-16_0233}}
% 53
{\PTglyphid{Au-16_0234}}
% 54
{\PTglyphid{Au-16_0235}}
% 55
{\PTglyphid{Au-16_0236}}
% 56
{\PTglyphid{Au-16_0237}}
% 57
{\PTglyphid{Au-16_0238}}
% 58
{\PTglyphid{Au-16_0301}}
% 59
{\PTglyphid{Au-16_0302}}
% 60
{\PTglyphid{Au-16_0303}}
% 61
{\PTglyphid{Au-16_0304}}
% 62
{\PTglyphid{Au-16_0305}}
% 63
{\PTglyphid{Au-16_0306}}
% 64
{\PTglyphid{Au-16_0307}}
% 65
{\PTglyphid{Au-16_0308}}
% 66
{\PTglyphid{Au-16_0309}}
% 67
{\PTglyphid{Au-16_0310}}
% 68
{\PTglyphid{Au-16_0311}}
% 69
{\PTglyphid{Au-16_0312}}
% 70
{\PTglyphid{Au-16_0313}}
% 71
{\PTglyphid{Au-16_0314}}
% 72
{\PTglyphid{Au-16_0315}}
% 73
{\PTglyphid{Au-16_0316}}
% 74
{\PTglyphid{Au-16_0317}}
//
\endgl \xe
%%% Local Variables:
%%% mode: latex
%%% TeX-engine: luatex
%%% TeX-master: shared
%%% End:

% //
%\endgl \xe


 
 \newpage

%%%%%%%%%%%%%%%%%%%%%%%%%%%%%%%%%%%%%%%%%%%%%%%%%%%%%%%%%%%%%%%%%%%%%%%%%%%%%%% 
% Tab. 19,Augezdecki-17_PT08_411.djvu,Augezdecki,17,08,411 poprawić
% Tab. 19,Augezdecki-17_PT08_412.djvu,Augezdecki,17,08,411
%%%%%%%%%%%%%%%%%%%%%%%%%%%%%%%%%%%%%%%%%%%%%%%%%%%%%%%%%%%%%%%%%%%%%%%%%%%%%%

% Note "17. Pismo tytułowe i nagłówkowe, tekstura M⁶³. Wysokość I w.= 12— 13 mm. — Tabl. 412."
% Note1 "Character set table prepared by Paulina Buchwald-Pelcowa"

\pismoPL{Aleksander Augezdecki 17. Pismo tytułowe i nagłówkowe, tekstura M⁶³. Wysokość 1 w.= 12—13 mm. — Tabl. 412 [trzeci zestaw].}

\pismoEN{Aleksander Augezdecki 17. Title
  and header font. Type size 1 line = 12-13 mm.- Plate 412 [third set].}

\plate{412[3]}{VIII}{1972}

The plate    prepared by Paulina Buchwald-Pelcowa.\\
The font table    prepared by Paulina Buchwald-Pelcowa.\\
\relax[Layout confusing, misinterpretation possible --- JSB]

\bigskip

\fontID{Au-17}{19}

\fontstat{67}

% \exdisplay \bg \gla
 \exdisplay \bg \gla
% 1
{\PTglyph{5}{t19_l01g01.png}}
% 2
{\PTglyph{5}{t19_l01g02.png}}
% 3
{\PTglyph{5}{t19_l01g03.png}}
% 4
{\PTglyph{5}{t19_l01g04.png}}
% 5
{\PTglyph{5}{t19_l01g05.png}}
% 6
{\PTglyph{5}{t19_l01g06.png}}
% 7
{\PTglyph{5}{t19_l01g07.png}}
% 8
{\PTglyph{5}{t19_l01g08.png}}
% 9
{\PTglyph{5}{t19_l01g09.png}}
% 10
{\PTglyph{5}{t19_l01g10.png}}
% 11
{\PTglyph{5}{t19_l01g11.png}}
% 12
{\PTglyph{5}{t19_l01g12.png}}
% 13
{\PTglyph{5}{t19_l01g13.png}}
% 14
{\PTglyph{5}{t19_l01g14.png}}
% 15
{\PTglyph{5}{t19_l02g01.png}}
% 16
{\PTglyph{5}{t19_l02g02.png}}
% 17
{\PTglyph{5}{t19_l02g03.png}}
% 18
{\PTglyph{5}{t19_l03g01.png}}
% 19
{\PTglyph{5}{t19_l03g02.png}}
% 20
{\PTglyph{5}{t19_l03g03.png}}
% 21
{\PTglyph{5}{t19_l03g04.png}}
% 22
{\PTglyph{5}{t19_l03g05.png}}
% 23
{\PTglyph{5}{t19_l03g06.png}}
% 24
{\PTglyph{5}{t19_l03g07.png}}
% 25
{\PTglyph{5}{t19_l03g08.png}}
% 26
{\PTglyph{5}{t19_l03g09.png}}
% 27
{\PTglyph{5}{t19_l03g10.png}}
% 28
{\PTglyph{5}{t19_l03g11.png}}
% 29
{\PTglyph{5}{t19_l03g12.png}}
% 30
{\PTglyph{5}{t19_l03g13.png}}
% 31
{\PTglyph{5}{t19_l03g14.png}}
% 32
{\PTglyph{5}{t19_l03g15.png}}
% 33
{\PTglyph{5}{t19_l03g16.png}}
% 34
{\PTglyph{5}{t19_l03g17.png}}
% 35
{\PTglyph{5}{t19_l03g18.png}}
% 36
{\PTglyph{5}{t19_l03g19.png}}
% 37
{\PTglyph{5}{t19_l03g20.png}}
% 38
{\PTglyph{5}{t19_l03g21.png}}
% 39
{\PTglyph{5}{t19_l03g22.png}}
% 40
{\PTglyph{5}{t19_l03g23.png}}
% 41
{\PTglyph{5}{t19_l03g24.png}}
% 42
{\PTglyph{5}{t19_l03g25.png}}
% 43
{\PTglyph{5}{t19_l03g26.png}}
% 44
{\PTglyph{5}{t19_l03g27.png}}
% 45
{\PTglyph{5}{t19_l04g01.png}}
% 46
{\PTglyph{5}{t19_l04g02.png}}
% 47
{\PTglyph{5}{t19_l04g03.png}}
% 48
{\PTglyph{5}{t19_l04g04.png}}
% 49
{\PTglyph{5}{t19_l04g05.png}}
% 50
{\PTglyph{5}{t19_l04g06.png}}
% 51
{\PTglyph{5}{t19_l04g07.png}}
% 52
{\PTglyph{5}{t19_l04g08.png}}
% 53
{\PTglyph{5}{t19_l04g09.png}}
% 54
{\PTglyph{5}{t19_l04g10.png}}
% 55
{\PTglyph{5}{t19_l04g11.png}}
% 56
{\PTglyph{5}{t19_l04g12.png}}
% 57
{\PTglyph{5}{t19_l04g13.png}}
% 58
{\PTglyph{5}{t19_l04g14.png}}
% 59
{\PTglyph{5}{t19_l04g15.png}}
% 60
{\PTglyph{5}{t19_l04g16.png}}
% 61
{\PTglyph{5}{t19_l04g17.png}}
% 62
{\PTglyph{5}{t19_l04g18.png}}
% 63
{\PTglyph{5}{t19_l04g19.png}}
% 64
{\PTglyph{5}{t19_l04g20.png}}
% 65
{\PTglyph{5}{t19_l04g21.png}}
% 66
{\PTglyph{5}{t19_l04g22.png}}
% 67
{\PTglyph{5}{t19_l04g23.png}}
//
%%% Local Variables:
%%% mode: latex
%%% TeX-engine: luatex
%%% TeX-master: shared
%%% End:

%//
%\glpismo%
 \glpismo
% 1
{\PTglyphid{Au-17_0101}}
% 2
{\PTglyphid{Au-17_0102}}
% 3
{\PTglyphid{Au-17_0103}}
% 4
{\PTglyphid{Au-17_0104}}
% 5
{\PTglyphid{Au-17_0105}}
% 6
{\PTglyphid{Au-17_0106}}
% 7
{\PTglyphid{Au-17_0107}}
% 8
{\PTglyphid{Au-17_0108}}
% 9
{\PTglyphid{Au-17_0109}}
% 10
{\PTglyphid{Au-17_0110}}
% 11
{\PTglyphid{Au-17_0111}}
% 12
{\PTglyphid{Au-17_0112}}
% 13
{\PTglyphid{Au-17_0113}}
% 14
{\PTglyphid{Au-17_0114}}
% 15
{\PTglyphid{Au-17_0201}}
% 16
{\PTglyphid{Au-17_0202}}
% 17
{\PTglyphid{Au-17_0203}}
% 18
{\PTglyphid{Au-17_0301}}
% 19
{\PTglyphid{Au-17_0302}}
% 20
{\PTglyphid{Au-17_0303}}
% 21
{\PTglyphid{Au-17_0304}}
% 22
{\PTglyphid{Au-17_0305}}
% 23
{\PTglyphid{Au-17_0306}}
% 24
{\PTglyphid{Au-17_0307}}
% 25
{\PTglyphid{Au-17_0308}}
% 26
{\PTglyphid{Au-17_0309}}
% 27
{\PTglyphid{Au-17_0310}}
% 28
{\PTglyphid{Au-17_0311}}
% 29
{\PTglyphid{Au-17_0312}}
% 30
{\PTglyphid{Au-17_0313}}
% 31
{\PTglyphid{Au-17_0314}}
% 32
{\PTglyphid{Au-17_0315}}
% 33
{\PTglyphid{Au-17_0316}}
% 34
{\PTglyphid{Au-17_0317}}
% 35
{\PTglyphid{Au-17_0318}}
% 36
{\PTglyphid{Au-17_0319}}
% 37
{\PTglyphid{Au-17_0320}}
% 38
{\PTglyphid{Au-17_0321}}
% 39
{\PTglyphid{Au-17_0322}}
% 40
{\PTglyphid{Au-17_0323}}
% 41
{\PTglyphid{Au-17_0324}}
% 42
{\PTglyphid{Au-17_0325}}
% 43
{\PTglyphid{Au-17_0326}}
% 44
{\PTglyphid{Au-17_0327}}
% 45
{\PTglyphid{Au-17_0401}}
% 46
{\PTglyphid{Au-17_0402}}
% 47
{\PTglyphid{Au-17_0403}}
% 48
{\PTglyphid{Au-17_0404}}
% 49
{\PTglyphid{Au-17_0405}}
% 50
{\PTglyphid{Au-17_0406}}
% 51
{\PTglyphid{Au-17_0407}}
% 52
{\PTglyphid{Au-17_0408}}
% 53
{\PTglyphid{Au-17_0409}}
% 54
{\PTglyphid{Au-17_0410}}
% 55
{\PTglyphid{Au-17_0411}}
% 56
{\PTglyphid{Au-17_0412}}
% 57
{\PTglyphid{Au-17_0413}}
% 58
{\PTglyphid{Au-17_0414}}
% 59
{\PTglyphid{Au-17_0415}}
% 60
{\PTglyphid{Au-17_0416}}
% 61
{\PTglyphid{Au-17_0417}}
% 62
{\PTglyphid{Au-17_0418}}
% 63
{\PTglyphid{Au-17_0419}}
% 64
{\PTglyphid{Au-17_0420}}
% 65
{\PTglyphid{Au-17_0421}}
% 66
{\PTglyphid{Au-17_0422}}
% 67
{\PTglyphid{Au-17_0423}}
//
\endgl \xe
%%% Local Variables:
%%% mode: latex
%%% TeX-engine: luatex
%%% TeX-master: shared
%%% End:

% //
%\endgl \xe


     \newpage

%%%%%%%%%%%%%%%%%%%%%%%%%%%%%%%%%%%%%%%%%%%%%%%%%%%%%%%%%%%%%%%%%%%%%%%%%%%%%%% 
% Tab. 20,Haller-01_PT04_164.djvu,Haller,01,04,164
%%%%%%%%%%%%%%%%%%%%%%%%%%%%%%%%%%%%%%%%%%%%%%%%%%%%%%%%%%%%%%%%%%%%%%%%%%%%%%

% Note "1. Pismo kanonowe. Krój M¹⁹, Stopień 10 ww. = 128/129 mm. — Tabl. 164."
% Note1 "Character set table prepared by Maria Błońska"

  \pismoPL{Jan Haller 1. Pismo kanonowe. Krój M¹⁹. Stopień 10 ww. =
    128/129 mm. — Tabl. 164. (Występuje u Hochfedera jako pismo
    8. Tabl. 25, u Unglera jako pismo 13,  Tabl. 72). }


  
  \pismoEN{Jan Haller 1. Canon [?] font. Typeface M¹⁹. Type size 10
    ww. = 128/129 mm. — Plate 164. (Used by Hochfeder as font 8, plate
    24, and by Ungler as font 13, plate 72.)}

\plate{164}{IV}{1962}

The plate prepared by Helena Kapełuś.\\
The font table prepared by Helena Kapełuś and Maria Błońska.

\bigskip

\exampleBib{IV:121}

\bigskip
\exampleDesc{MISSALE Vladislaviense. Kraków, Jan Haller, 29. XI. 1515 — 1. II. 1516. 2⁰.}

\medskip
\examplePage{\textit{Karta 1 (Canon) niepełna: wiersze 1——15.}}

  \bigskip
\exampleLib{Biblioteka Czartoryskich. Kraków.}

\bigskip
\exampleRef{\textit{Estreicher XXII. 434}}

\bigskip
\exampleDig{\url{https://cyfrowe.mnk.pl/dlibra/publication/22776/}, page 403.}

% \medskip

%     \examplePL{Pismo 2: tekst i zestaw liter ze znakami diakrytycznymi czeskimi.}

%     \medskip

%     \exampleEN{Font 2. The text and the table including letters with Czech diacritical marks.}


\bigskip


\fontID{Ha-01}{20}

\fontstat{85}

% \exdisplay \bg \gla
 \exdisplay \bg \gla
% 1
{\PTglyph{5}{t20_l01g01.png}}
% 2
{\PTglyph{5}{t20_l01g02.png}}
% 3
{\PTglyph{5}{t20_l01g03.png}}
% 4
{\PTglyph{5}{t20_l01g04.png}}
% 5
{\PTglyph{5}{t20_l01g05.png}}
% 6
{\PTglyph{5}{t20_l01g06.png}}
% 7
{\PTglyph{5}{t20_l01g07.png}}
% 8
{\PTglyph{5}{t20_l01g08.png}}
% 9
{\PTglyph{5}{t20_l01g09.png}}
% 10
{\PTglyph{5}{t20_l01g10.png}}
% 11
{\PTglyph{5}{t20_l01g11.png}}
% 12
{\PTglyph{5}{t20_l01g12.png}}
% 13
{\PTglyph{5}{t20_l01g13.png}}
% 14
{\PTglyph{5}{t20_l01g14.png}}
% 15
{\PTglyph{5}{t20_l01g15.png}}
% 16
{\PTglyph{5}{t20_l01g16.png}}
% 17
{\PTglyph{5}{t20_l02g01.png}}
% 18
{\PTglyph{5}{t20_l02g02.png}}
% 19
{\PTglyph{5}{t20_l02g03.png}}
% 20
{\PTglyph{5}{t20_l02g04.png}}
% 21
{\PTglyph{5}{t20_l02g05.png}}
% 22
{\PTglyph{5}{t20_l02g06.png}}
% 23
{\PTglyph{5}{t20_l02g07.png}}
% 24
{\PTglyph{5}{t20_l02g08.png}}
% 25
{\PTglyph{5}{t20_l02g09.png}}
% 26
{\PTglyph{5}{t20_l02g10.png}}
% 27
{\PTglyph{5}{t20_l02g11.png}}
% 28
{\PTglyph{5}{t20_l02g12.png}}
% 29
{\PTglyph{5}{t20_l02g13.png}}
% 30
{\PTglyph{5}{t20_l02g14.png}}
% 31
{\PTglyph{5}{t20_l02g15.png}}
% 32
{\PTglyph{5}{t20_l02g16.png}}
% 33
{\PTglyph{5}{t20_l02g17.png}}
% 34
{\PTglyph{5}{t20_l02g18.png}}
% 35
{\PTglyph{5}{t20_l02g19.png}}
% 36
{\PTglyph{5}{t20_l02g20.png}}
% 37
{\PTglyph{5}{t20_l02g21.png}}
% 38
{\PTglyph{5}{t20_l02g22.png}}
% 39
{\PTglyph{5}{t20_l02g23.png}}
% 40
{\PTglyph{5}{t20_l02g24.png}}
% 41
{\PTglyph{5}{t20_l02g25.png}}
% 42
{\PTglyph{5}{t20_l02g26.png}}
% 43
{\PTglyph{5}{t20_l03g01.png}}
% 44
{\PTglyph{5}{t20_l03g02.png}}
% 45
{\PTglyph{5}{t20_l03g03.png}}
% 46
{\PTglyph{5}{t20_l03g04.png}}
% 47
{\PTglyph{5}{t20_l03g05.png}}
% 48
{\PTglyph{5}{t20_l03g06.png}}
% 49
{\PTglyph{5}{t20_l03g07.png}}
% 50
{\PTglyph{5}{t20_l03g08.png}}
% 51
{\PTglyph{5}{t20_l03g09.png}}
% 52
{\PTglyph{5}{t20_l03g10.png}}
% 53
{\PTglyph{5}{t20_l03g11.png}}
% 54
{\PTglyph{5}{t20_l03g12.png}}
% 55
{\PTglyph{5}{t20_l03g13.png}}
% 56
{\PTglyph{5}{t20_l03g14.png}}
% 57
{\PTglyph{5}{t20_l03g15.png}}
% 58
{\PTglyph{5}{t20_l03g16.png}}
% 59
{\PTglyph{5}{t20_l03g17.png}}
% 60
{\PTglyph{5}{t20_l03g18.png}}
% 61
{\PTglyph{5}{t20_l03g19.png}}
% 62
{\PTglyph{5}{t20_l03g20.png}}
% 63
{\PTglyph{5}{t20_l03g21.png}}
% 64
{\PTglyph{5}{t20_l03g22.png}}
% 65
{\PTglyph{5}{t20_l03g23.png}}
% 66
{\PTglyph{5}{t20_l03g24.png}}
% 67
{\PTglyph{5}{t20_l03g25.png}}
% 68
{\PTglyph{5}{t20_l03g26.png}}
% 69
{\PTglyph{5}{t20_l03g27.png}}
% 70
{\PTglyph{5}{t20_l04g01.png}}
% 71
{\PTglyph{5}{t20_l04g02.png}}
% 72
{\PTglyph{5}{t20_l04g03.png}}
% 73
{\PTglyph{5}{t20_l04g04.png}}
% 74
{\PTglyph{5}{t20_l04g05.png}}
% 75
{\PTglyph{5}{t20_l04g06.png}}
% 76
{\PTglyph{5}{t20_l04g07.png}}
% 77
{\PTglyph{5}{t20_l04g08.png}}
% 78
{\PTglyph{5}{t20_l04g09.png}}
% 79
{\PTglyph{5}{t20_l04g10.png}}
% 80
{\PTglyph{5}{t20_l04g11.png}}
% 81
{\PTglyph{5}{t20_l04g12.png}}
% 82
{\PTglyph{5}{t20_l04g13.png}}
% 83
{\PTglyph{5}{t20_l04g14.png}}
% 84
{\PTglyph{5}{t20_l04g15.png}}
% 85
{\PTglyph{5}{t20_l04g16.png}}
//
%%% Local Variables:
%%% mode: latex
%%% TeX-engine: luatex
%%% TeX-master: shared
%%% End:

%//
%\glpismo%
 \glpismo
% 1
{\PTglyphid{Ha-01_0101}}
% 2
{\PTglyphid{Ha-01_0102}}
% 3
{\PTglyphid{Ha-01_0103}}
% 4
{\PTglyphid{Ha-01_0104}}
% 5
{\PTglyphid{Ha-01_0105}}
% 6
{\PTglyphid{Ha-01_0106}}
% 7
{\PTglyphid{Ha-01_0107}}
% 8
{\PTglyphid{Ha-01_0108}}
% 9
{\PTglyphid{Ha-01_0109}}
% 10
{\PTglyphid{Ha-01_0110}}
% 11
{\PTglyphid{Ha-01_0111}}
% 12
{\PTglyphid{Ha-01_0112}}
% 13
{\PTglyphid{Ha-01_0113}}
% 14
{\PTglyphid{Ha-01_0114}}
% 15
{\PTglyphid{Ha-01_0115}}
% 16
{\PTglyphid{Ha-01_0116}}
% 17
{\PTglyphid{Ha-01_0201}}
% 18
{\PTglyphid{Ha-01_0202}}
% 19
{\PTglyphid{Ha-01_0203}}
% 20
{\PTglyphid{Ha-01_0204}}
% 21
{\PTglyphid{Ha-01_0205}}
% 22
{\PTglyphid{Ha-01_0206}}
% 23
{\PTglyphid{Ha-01_0207}}
% 24
{\PTglyphid{Ha-01_0208}}
% 25
{\PTglyphid{Ha-01_0209}}
% 26
{\PTglyphid{Ha-01_0210}}
% 27
{\PTglyphid{Ha-01_0211}}
% 28
{\PTglyphid{Ha-01_0212}}
% 29
{\PTglyphid{Ha-01_0213}}
% 30
{\PTglyphid{Ha-01_0214}}
% 31
{\PTglyphid{Ha-01_0215}}
% 32
{\PTglyphid{Ha-01_0216}}
% 33
{\PTglyphid{Ha-01_0217}}
% 34
{\PTglyphid{Ha-01_0218}}
% 35
{\PTglyphid{Ha-01_0219}}
% 36
{\PTglyphid{Ha-01_0220}}
% 37
{\PTglyphid{Ha-01_0221}}
% 38
{\PTglyphid{Ha-01_0222}}
% 39
{\PTglyphid{Ha-01_0223}}
% 40
{\PTglyphid{Ha-01_0224}}
% 41
{\PTglyphid{Ha-01_0225}}
% 42
{\PTglyphid{Ha-01_0226}}
% 43
{\PTglyphid{Ha-01_0301}}
% 44
{\PTglyphid{Ha-01_0302}}
% 45
{\PTglyphid{Ha-01_0303}}
% 46
{\PTglyphid{Ha-01_0304}}
% 47
{\PTglyphid{Ha-01_0305}}
% 48
{\PTglyphid{Ha-01_0306}}
% 49
{\PTglyphid{Ha-01_0307}}
% 50
{\PTglyphid{Ha-01_0308}}
% 51
{\PTglyphid{Ha-01_0309}}
% 52
{\PTglyphid{Ha-01_0310}}
% 53
{\PTglyphid{Ha-01_0311}}
% 54
{\PTglyphid{Ha-01_0312}}
% 55
{\PTglyphid{Ha-01_0313}}
% 56
{\PTglyphid{Ha-01_0314}}
% 57
{\PTglyphid{Ha-01_0315}}
% 58
{\PTglyphid{Ha-01_0316}}
% 59
{\PTglyphid{Ha-01_0317}}
% 60
{\PTglyphid{Ha-01_0318}}
% 61
{\PTglyphid{Ha-01_0319}}
% 62
{\PTglyphid{Ha-01_0320}}
% 63
{\PTglyphid{Ha-01_0321}}
% 64
{\PTglyphid{Ha-01_0322}}
% 65
{\PTglyphid{Ha-01_0323}}
% 66
{\PTglyphid{Ha-01_0324}}
% 67
{\PTglyphid{Ha-01_0325}}
% 68
{\PTglyphid{Ha-01_0326}}
% 69
{\PTglyphid{Ha-01_0327}}
% 70
{\PTglyphid{Ha-01_0401}}
% 71
{\PTglyphid{Ha-01_0402}}
% 72
{\PTglyphid{Ha-01_0403}}
% 73
{\PTglyphid{Ha-01_0404}}
% 74
{\PTglyphid{Ha-01_0405}}
% 75
{\PTglyphid{Ha-01_0406}}
% 76
{\PTglyphid{Ha-01_0407}}
% 77
{\PTglyphid{Ha-01_0408}}
% 78
{\PTglyphid{Ha-01_0409}}
% 79
{\PTglyphid{Ha-01_0410}}
% 80
{\PTglyphid{Ha-01_0411}}
% 81
{\PTglyphid{Ha-01_0412}}
% 82
{\PTglyphid{Ha-01_0413}}
% 83
{\PTglyphid{Ha-01_0414}}
% 84
{\PTglyphid{Ha-01_0415}}
% 85
{\PTglyphid{Ha-01_0416}}
//
\endgl \xe
%%% Local Variables:
%%% mode: latex
%%% TeX-engine: luatex
%%% TeX-master: shared
%%% End:

% //
%\endgl \xe


 
  \newpage

%%%%%%%%%%%%%%%%%%%%%%%%%%%%%%%%%%%%%%%%%%%%%%%%%%%%%%%%%%%%%%%%%%%%%%%%%%%%%%% 
% Tab. 22,Haller-03_PT04_166.djvu,Haller,03,04,166
%%%%%%%%%%%%%%%%%%%%%%%%%%%%%%%%%%%%%%%%%%%%%%%%%%%%%%%%%%%%%%%%%%%%%%%%%%%%%%

  
% Note "2. Pismo mszalne większe. Krój M¹⁸. Stopień 10 ww. = 77 mm. — Tabl. 165."
% Note1 "Character set table prepared by Maria Błońska"

  \pismoPL{Jan Haller 2. Pismo mszalne większe. Krój M¹⁸. Stopień 10
    ww. = 77 mm. — Tabl. 165. (Występuje u Hochfedera jako pismo
    7. Tabl. 24, u Unglera jako pismo 12. Tabl. 72).}


  
\pismoEN{Jan Haller 2. Larger mass [?] font. Typeface M¹⁸. Type size 10 ww. =
    77 mm. — Plate 165. (Used by Hochfeder as font 7, plate
    24, and by Ungler as font 12, plate 72.)}

\plate{165}{IV}{1962}

The plate prepared by Helena Kapełuś.\\
The font table prepared by Helena Kapełuś and Maria Błońska.

\bigskip

\exampleBib{IV:2}

\bigskip
\exampleDesc{IOANNES LASKI: Commune Poloniae Regni privilegium. Kraków, Jan Haller, [po 27. I. 1506]. 2⁰}

\medskip
\examplePage{\textit{Karta a₃b.}}

  \bigskip
\exampleLib{Biblioteka Zakł. Nar. im. Ossolińskich. Wrocław.}

\bigskip
\exampleRef{\textit{Estreicher XXI. 79. Wierzbowski 9.}}

\bigskip
%\exampleDig{\url{https://www.wbc.poznan.pl/dlibra/publication/493453/}, page ???}???
\exampleDig{\url{https://dlibra.biblioteka.tarnow.pl/publication/196}, page 54.}

\medskip

    \examplePL{Pismo 2: wiersz 1—4, 17.}

    \medskip

    \exampleEN{Font 2: lines 1--4, 17}


\bigskip


\fontID{Ha-02}{21}

\fontstat{111}

% \exdisplay \bg \gla
 \exdisplay \bg \gla
% 1
{\PTglyph{5}{t21_l01g01.png}}
% 2
{\PTglyph{5}{t21_l01g02.png}}
% 3
{\PTglyph{5}{t21_l01g03.png}}
% 4
{\PTglyph{5}{t21_l01g04.png}}
% 5
{\PTglyph{5}{t21_l01g05.png}}
% 6
{\PTglyph{5}{t21_l01g06.png}}
% 7
{\PTglyph{5}{t21_l01g07.png}}
% 8
{\PTglyph{5}{t21_l01g08.png}}
% 9
{\PTglyph{5}{t21_l01g09.png}}
% 10
{\PTglyph{5}{t21_l01g10.png}}
% 11
{\PTglyph{5}{t21_l01g11.png}}
% 12
{\PTglyph{5}{t21_l01g12.png}}
% 13
{\PTglyph{5}{t21_l01g13.png}}
% 14
{\PTglyph{5}{t21_l01g14.png}}
% 15
{\PTglyph{5}{t21_l01g15.png}}
% 16
{\PTglyph{5}{t21_l01g16.png}}
% 17
{\PTglyph{5}{t21_l01g17.png}}
% 18
{\PTglyph{5}{t21_l01g18.png}}
% 19
{\PTglyph{5}{t21_l01g19.png}}
% 20
{\PTglyph{5}{t21_l01g20.png}}
% 21
{\PTglyph{5}{t21_l01g21.png}}
% 22
{\PTglyph{5}{t21_l02g01.png}}
% 23
{\PTglyph{5}{t21_l02g02.png}}
% 24
{\PTglyph{5}{t21_l02g03.png}}
% 25
{\PTglyph{5}{t21_l02g04.png}}
% 26
{\PTglyph{5}{t21_l02g05.png}}
% 27
{\PTglyph{5}{t21_l02g06.png}}
% 28
{\PTglyph{5}{t21_l02g07.png}}
% 29
{\PTglyph{5}{t21_l02g08.png}}
% 30
{\PTglyph{5}{t21_l02g09.png}}
% 31
{\PTglyph{5}{t21_l02g10.png}}
% 32
{\PTglyph{5}{t21_l02g11.png}}
% 33
{\PTglyph{5}{t21_l02g12.png}}
% 34
{\PTglyph{5}{t21_l02g13.png}}
% 35
{\PTglyph{5}{t21_l02g14.png}}
% 36
{\PTglyph{5}{t21_l02g15.png}}
% 37
{\PTglyph{5}{t21_l02g16.png}}
% 38
{\PTglyph{5}{t21_l02g17.png}}
% 39
{\PTglyph{5}{t21_l02g18.png}}
% 40
{\PTglyph{5}{t21_l02g19.png}}
% 41
{\PTglyph{5}{t21_l02g20.png}}
% 42
{\PTglyph{5}{t21_l02g21.png}}
% 43
{\PTglyph{5}{t21_l02g22.png}}
% 44
{\PTglyph{5}{t21_l02g23.png}}
% 45
{\PTglyph{5}{t21_l02g24.png}}
% 46
{\PTglyph{5}{t21_l02g25.png}}
% 47
{\PTglyph{5}{t21_l02g26.png}}
% 48
{\PTglyph{5}{t21_l02g27.png}}
% 49
{\PTglyph{5}{t21_l02g28.png}}
% 50
{\PTglyph{5}{t21_l02g29.png}}
% 51
{\PTglyph{5}{t21_l02g30.png}}
% 52
{\PTglyph{5}{t21_l02g31.png}}
% 53
{\PTglyph{5}{t21_l02g32.png}}
% 54
{\PTglyph{5}{t21_l02g33.png}}
% 55
{\PTglyph{5}{t21_l02g34.png}}
% 56
{\PTglyph{5}{t21_l02g35.png}}
% 57
{\PTglyph{5}{t21_l02g36.png}}
% 58
{\PTglyph{5}{t21_l02g37.png}}
% 59
{\PTglyph{5}{t21_l02g38.png}}
% 60
{\PTglyph{5}{t21_l02g39.png}}
% 61
{\PTglyph{5}{t21_l02g40.png}}
% 62
{\PTglyph{5}{t21_l03g01.png}}
% 63
{\PTglyph{5}{t21_l03g02.png}}
% 64
{\PTglyph{5}{t21_l03g03.png}}
% 65
{\PTglyph{5}{t21_l03g04.png}}
% 66
{\PTglyph{5}{t21_l03g05.png}}
% 67
{\PTglyph{5}{t21_l03g06.png}}
% 68
{\PTglyph{5}{t21_l03g07.png}}
% 69
{\PTglyph{5}{t21_l03g08.png}}
% 70
{\PTglyph{5}{t21_l03g09.png}}
% 71
{\PTglyph{5}{t21_l03g10.png}}
% 72
{\PTglyph{5}{t21_l03g11.png}}
% 73
{\PTglyph{5}{t21_l03g12.png}}
% 74
{\PTglyph{5}{t21_l03g13.png}}
% 75
{\PTglyph{5}{t21_l03g14.png}}
% 76
{\PTglyph{5}{t21_l03g15.png}}
% 77
{\PTglyph{5}{t21_l03g16.png}}
% 78
{\PTglyph{5}{t21_l03g17.png}}
% 79
{\PTglyph{5}{t21_l03g18.png}}
% 80
{\PTglyph{5}{t21_l03g19.png}}
% 81
{\PTglyph{5}{t21_l03g20.png}}
% 82
{\PTglyph{5}{t21_l03g21.png}}
% 83
{\PTglyph{5}{t21_l03g22.png}}
% 84
{\PTglyph{5}{t21_l03g23.png}}
% 85
{\PTglyph{5}{t21_l03g24.png}}
% 86
{\PTglyph{5}{t21_l03g25.png}}
% 87
{\PTglyph{5}{t21_l03g26.png}}
% 88
{\PTglyph{5}{t21_l03g27.png}}
% 89
{\PTglyph{5}{t21_l03g28.png}}
% 90
{\PTglyph{5}{t21_l03g29.png}}
% 91
{\PTglyph{5}{t21_l03g30.png}}
% 92
{\PTglyph{5}{t21_l03g31.png}}
% 93
{\PTglyph{5}{t21_l03g32.png}}
% 94
{\PTglyph{5}{t21_l03g33.png}}
% 95
{\PTglyph{5}{t21_l03g34.png}}
% 96
{\PTglyph{5}{t21_l03g35.png}}
% 97
{\PTglyph{5}{t21_l03g36.png}}
% 98
{\PTglyph{5}{t21_l03g37.png}}
% 99
{\PTglyph{5}{t21_l04g01.png}}
% 100
{\PTglyph{5}{t21_l04g02.png}}
% 101
{\PTglyph{5}{t21_l04g03.png}}
% 102
{\PTglyph{5}{t21_l04g04.png}}
% 103
{\PTglyph{5}{t21_l04g05.png}}
% 104
{\PTglyph{5}{t21_l04g06.png}}
% 105
{\PTglyph{5}{t21_l04g07.png}}
% 106
{\PTglyph{5}{t21_l04g08.png}}
% 107
{\PTglyph{5}{t21_l04g09.png}}
% 108
{\PTglyph{5}{t21_l04g10.png}}
% 109
{\PTglyph{5}{t21_l04g11.png}}
% 110
{\PTglyph{5}{t21_l04g12.png}}
% 111
{\PTglyph{5}{t21_l04g13.png}}
//
%%% Local Variables:
%%% mode: latex
%%% TeX-engine: luatex
%%% TeX-master: shared
%%% End:

%//
%\glpismo%
 \glpismo
% 1
{\PTglyphid{Ha-02_0101}}
% 2
{\PTglyphid{Ha-02_0102}}
% 3
{\PTglyphid{Ha-02_0103}}
% 4
{\PTglyphid{Ha-02_0104}}
% 5
{\PTglyphid{Ha-02_0105}}
% 6
{\PTglyphid{Ha-02_0106}}
% 7
{\PTglyphid{Ha-02_0107}}
% 8
{\PTglyphid{Ha-02_0108}}
% 9
{\PTglyphid{Ha-02_0109}}
% 10
{\PTglyphid{Ha-02_0110}}
% 11
{\PTglyphid{Ha-02_0111}}
% 12
{\PTglyphid{Ha-02_0112}}
% 13
{\PTglyphid{Ha-02_0113}}
% 14
{\PTglyphid{Ha-02_0114}}
% 15
{\PTglyphid{Ha-02_0115}}
% 16
{\PTglyphid{Ha-02_0116}}
% 17
{\PTglyphid{Ha-02_0117}}
% 18
{\PTglyphid{Ha-02_0118}}
% 19
{\PTglyphid{Ha-02_0119}}
% 20
{\PTglyphid{Ha-02_0120}}
% 21
{\PTglyphid{Ha-02_0121}}
% 22
{\PTglyphid{Ha-02_0201}}
% 23
{\PTglyphid{Ha-02_0202}}
% 24
{\PTglyphid{Ha-02_0203}}
% 25
{\PTglyphid{Ha-02_0204}}
% 26
{\PTglyphid{Ha-02_0205}}
% 27
{\PTglyphid{Ha-02_0206}}
% 28
{\PTglyphid{Ha-02_0207}}
% 29
{\PTglyphid{Ha-02_0208}}
% 30
{\PTglyphid{Ha-02_0209}}
% 31
{\PTglyphid{Ha-02_0210}}
% 32
{\PTglyphid{Ha-02_0211}}
% 33
{\PTglyphid{Ha-02_0212}}
% 34
{\PTglyphid{Ha-02_0213}}
% 35
{\PTglyphid{Ha-02_0214}}
% 36
{\PTglyphid{Ha-02_0215}}
% 37
{\PTglyphid{Ha-02_0216}}
% 38
{\PTglyphid{Ha-02_0217}}
% 39
{\PTglyphid{Ha-02_0218}}
% 40
{\PTglyphid{Ha-02_0219}}
% 41
{\PTglyphid{Ha-02_0220}}
% 42
{\PTglyphid{Ha-02_0221}}
% 43
{\PTglyphid{Ha-02_0222}}
% 44
{\PTglyphid{Ha-02_0223}}
% 45
{\PTglyphid{Ha-02_0224}}
% 46
{\PTglyphid{Ha-02_0225}}
% 47
{\PTglyphid{Ha-02_0226}}
% 48
{\PTglyphid{Ha-02_0227}}
% 49
{\PTglyphid{Ha-02_0228}}
% 50
{\PTglyphid{Ha-02_0229}}
% 51
{\PTglyphid{Ha-02_0230}}
% 52
{\PTglyphid{Ha-02_0231}}
% 53
{\PTglyphid{Ha-02_0232}}
% 54
{\PTglyphid{Ha-02_0233}}
% 55
{\PTglyphid{Ha-02_0234}}
% 56
{\PTglyphid{Ha-02_0235}}
% 57
{\PTglyphid{Ha-02_0236}}
% 58
{\PTglyphid{Ha-02_0237}}
% 59
{\PTglyphid{Ha-02_0238}}
% 60
{\PTglyphid{Ha-02_0239}}
% 61
{\PTglyphid{Ha-02_0240}}
% 62
{\PTglyphid{Ha-02_0301}}
% 63
{\PTglyphid{Ha-02_0302}}
% 64
{\PTglyphid{Ha-02_0303}}
% 65
{\PTglyphid{Ha-02_0304}}
% 66
{\PTglyphid{Ha-02_0305}}
% 67
{\PTglyphid{Ha-02_0306}}
% 68
{\PTglyphid{Ha-02_0307}}
% 69
{\PTglyphid{Ha-02_0308}}
% 70
{\PTglyphid{Ha-02_0309}}
% 71
{\PTglyphid{Ha-02_0310}}
% 72
{\PTglyphid{Ha-02_0311}}
% 73
{\PTglyphid{Ha-02_0312}}
% 74
{\PTglyphid{Ha-02_0313}}
% 75
{\PTglyphid{Ha-02_0314}}
% 76
{\PTglyphid{Ha-02_0315}}
% 77
{\PTglyphid{Ha-02_0316}}
% 78
{\PTglyphid{Ha-02_0317}}
% 79
{\PTglyphid{Ha-02_0318}}
% 80
{\PTglyphid{Ha-02_0319}}
% 81
{\PTglyphid{Ha-02_0320}}
% 82
{\PTglyphid{Ha-02_0321}}
% 83
{\PTglyphid{Ha-02_0322}}
% 84
{\PTglyphid{Ha-02_0323}}
% 85
{\PTglyphid{Ha-02_0324}}
% 86
{\PTglyphid{Ha-02_0325}}
% 87
{\PTglyphid{Ha-02_0326}}
% 88
{\PTglyphid{Ha-02_0327}}
% 89
{\PTglyphid{Ha-02_0328}}
% 90
{\PTglyphid{Ha-02_0329}}
% 91
{\PTglyphid{Ha-02_0330}}
% 92
{\PTglyphid{Ha-02_0331}}
% 93
{\PTglyphid{Ha-02_0332}}
% 94
{\PTglyphid{Ha-02_0333}}
% 95
{\PTglyphid{Ha-02_0334}}
% 96
{\PTglyphid{Ha-02_0335}}
% 97
{\PTglyphid{Ha-02_0336}}
% 98
{\PTglyphid{Ha-02_0337}}
% 99
{\PTglyphid{Ha-02_0401}}
% 100
{\PTglyphid{Ha-02_0402}}
% 101
{\PTglyphid{Ha-02_0403}}
% 102
{\PTglyphid{Ha-02_0404}}
% 103
{\PTglyphid{Ha-02_0405}}
% 104
{\PTglyphid{Ha-02_0406}}
% 105
{\PTglyphid{Ha-02_0407}}
% 106
{\PTglyphid{Ha-02_0408}}
% 107
{\PTglyphid{Ha-02_0409}}
% 108
{\PTglyphid{Ha-02_0410}}
% 109
{\PTglyphid{Ha-02_0411}}
% 110
{\PTglyphid{Ha-02_0412}}
% 111
{\PTglyphid{Ha-02_0413}}
//
\endgl \xe
%%% Local Variables:
%%% mode: latex
%%% TeX-engine: luatex
%%% TeX-master: shared
%%% End:

% //
%\endgl \xe


  \newpage

%%%%%%%%%%%%%%%%%%%%%%%%%%%%%%%%%%%%%%%%%%%%%%%%%%%%%%%%%%%%%%%%%%%%%%%%%%%%%%% 
% Tab. 22,Haller-03_PT04_166.djvu,Haller,03,04,166
%%%%%%%%%%%%%%%%%%%%%%%%%%%%%%%%%%%%%%%%%%%%%%%%%%%%%%%%%%%%%%%%%%%%%%%%%%%%%%

% Note "3. Pismo mszalne mniejsze. Krój M²³. Stopień 20 ww. = 136/137 mm. — Tabl. 166."
% Note1 "Character set table prepared by Maria Błońska"


  \pismoPL{Jan Haller 3. Pismo mszalne większe. Krój M²³. Stopień 10
    ww. = 136/137 mm. — Tabl. 166. (Występuje u Hochfedera jako pismo
    11. Tabl. 24, u Unglera jako pismo 11).}


  
\pismoEN{Jan Haller 3. Smaller mass [?] font. Typeface M²³. Type size 10 ww. =
    136/137 mm. — Plate 166. (Used by Hochfeder as font 11, plate
    11, and by Ungler as font 11.)}

\plate{166}{IV}{1962}

Prepared by Helena Kapełuś.\\
The font table prepared by Helena Kapełuś and Maria Błońska.

\bigskip

\exampleBib{IV:151}

\bigskip
\exampleDesc{AGENDA Cracoviensis. Kraków, Jan Haller, [1517]. 4⁰.}

\medskip
\examplePage{\textit{Karta M₂b.}}

  \bigskip
\exampleLib{Biblioteka Zakł. Nar. im. Ossolińskich. Wrocław.}

\bigskip
\exampleRef{\textit{Estreicher XII. 70.}}

\bigskip
%\exampleDig{\url{https://www.wbc.poznan.pl/dlibra/publication/493453/}, page ???}???
\exampleDig{\url{https://dbc.wroc.pl/dlibra/publication/11760/}, page 242.}

% \medskip

%     \examplePL{Pismo 2: wiersz 1—4, 17.}

%     \medskip

%     \exampleEN{Font 2: lines 1--4, 17}


\bigskip


\fontID{Ha-03}{22}

\fontstat{115}

% \exdisplay \bg \gla
 \exdisplay \bg \gla
% 1
{\PTglyph{5}{t22_l01g01.png}}
% 2
{\PTglyph{5}{t22_l01g02.png}}
% 3
{\PTglyph{5}{t22_l01g03.png}}
% 4
{\PTglyph{5}{t22_l01g04.png}}
% 5
{\PTglyph{5}{t22_l01g05.png}}
% 6
{\PTglyph{5}{t22_l01g06.png}}
% 7
{\PTglyph{5}{t22_l01g07.png}}
% 8
{\PTglyph{5}{t22_l01g08.png}}
% 9
{\PTglyph{5}{t22_l01g09.png}}
% 10
{\PTglyph{5}{t22_l01g10.png}}
% 11
{\PTglyph{5}{t22_l01g11.png}}
% 12
{\PTglyph{5}{t22_l01g12.png}}
% 13
{\PTglyph{5}{t22_l01g13.png}}
% 14
{\PTglyph{5}{t22_l01g14.png}}
% 15
{\PTglyph{5}{t22_l01g15.png}}
% 16
{\PTglyph{5}{t22_l01g16.png}}
% 17
{\PTglyph{5}{t22_l01g17.png}}
% 18
{\PTglyph{5}{t22_l01g18.png}}
% 19
{\PTglyph{5}{t22_l01g19.png}}
% 20
{\PTglyph{5}{t22_l01g20.png}}
% 21
{\PTglyph{5}{t22_l01g21.png}}
% 22
{\PTglyph{5}{t22_l01g22.png}}
% 23
{\PTglyph{5}{t22_l02g01.png}}
% 24
{\PTglyph{5}{t22_l02g02.png}}
% 25
{\PTglyph{5}{t22_l02g03.png}}
% 26
{\PTglyph{5}{t22_l02g04.png}}
% 27
{\PTglyph{5}{t22_l02g05.png}}
% 28
{\PTglyph{5}{t22_l02g06.png}}
% 29
{\PTglyph{5}{t22_l02g07.png}}
% 30
{\PTglyph{5}{t22_l02g08.png}}
% 31
{\PTglyph{5}{t22_l02g09.png}}
% 32
{\PTglyph{5}{t22_l02g10.png}}
% 33
{\PTglyph{5}{t22_l02g11.png}}
% 34
{\PTglyph{5}{t22_l02g12.png}}
% 35
{\PTglyph{5}{t22_l02g13.png}}
% 36
{\PTglyph{5}{t22_l02g14.png}}
% 37
{\PTglyph{5}{t22_l02g15.png}}
% 38
{\PTglyph{5}{t22_l02g16.png}}
% 39
{\PTglyph{5}{t22_l02g17.png}}
% 40
{\PTglyph{5}{t22_l02g18.png}}
% 41
{\PTglyph{5}{t22_l02g19.png}}
% 42
{\PTglyph{5}{t22_l02g20.png}}
% 43
{\PTglyph{5}{t22_l02g21.png}}
% 44
{\PTglyph{5}{t22_l02g22.png}}
% 45
{\PTglyph{5}{t22_l02g23.png}}
% 46
{\PTglyph{5}{t22_l02g24.png}}
% 47
{\PTglyph{5}{t22_l02g25.png}}
% 48
{\PTglyph{5}{t22_l02g26.png}}
% 49
{\PTglyph{5}{t22_l02g27.png}}
% 50
{\PTglyph{5}{t22_l02g28.png}}
% 51
{\PTglyph{5}{t22_l02g29.png}}
% 52
{\PTglyph{5}{t22_l02g30.png}}
% 53
{\PTglyph{5}{t22_l02g31.png}}
% 54
{\PTglyph{5}{t22_l02g32.png}}
% 55
{\PTglyph{5}{t22_l02g33.png}}
% 56
{\PTglyph{5}{t22_l02g34.png}}
% 57
{\PTglyph{5}{t22_l02g35.png}}
% 58
{\PTglyph{5}{t22_l02g36.png}}
% 59
{\PTglyph{5}{t22_l02g37.png}}
% 60
{\PTglyph{5}{t22_l02g38.png}}
% 61
{\PTglyph{5}{t22_l03g01.png}}
% 62
{\PTglyph{5}{t22_l03g02.png}}
% 63
{\PTglyph{5}{t22_l03g03.png}}
% 64
{\PTglyph{5}{t22_l03g04.png}}
% 65
{\PTglyph{5}{t22_l03g05.png}}
% 66
{\PTglyph{5}{t22_l03g06.png}}
% 67
{\PTglyph{5}{t22_l03g07.png}}
% 68
{\PTglyph{5}{t22_l03g08.png}}
% 69
{\PTglyph{5}{t22_l03g09.png}}
% 70
{\PTglyph{5}{t22_l03g10.png}}
% 71
{\PTglyph{5}{t22_l03g11.png}}
% 72
{\PTglyph{5}{t22_l03g12.png}}
% 73
{\PTglyph{5}{t22_l03g13.png}}
% 74
{\PTglyph{5}{t22_l03g14.png}}
% 75
{\PTglyph{5}{t22_l03g15.png}}
% 76
{\PTglyph{5}{t22_l03g16.png}}
% 77
{\PTglyph{5}{t22_l03g17.png}}
% 78
{\PTglyph{5}{t22_l03g18.png}}
% 79
{\PTglyph{5}{t22_l03g19.png}}
% 80
{\PTglyph{5}{t22_l03g20.png}}
% 81
{\PTglyph{5}{t22_l03g21.png}}
% 82
{\PTglyph{5}{t22_l03g22.png}}
% 83
{\PTglyph{5}{t22_l03g23.png}}
% 84
{\PTglyph{5}{t22_l03g24.png}}
% 85
{\PTglyph{5}{t22_l03g25.png}}
% 86
{\PTglyph{5}{t22_l03g26.png}}
% 87
{\PTglyph{5}{t22_l03g27.png}}
% 88
{\PTglyph{5}{t22_l03g28.png}}
% 89
{\PTglyph{5}{t22_l03g29.png}}
% 90
{\PTglyph{5}{t22_l03g30.png}}
% 91
{\PTglyph{5}{t22_l03g31.png}}
% 92
{\PTglyph{5}{t22_l03g32.png}}
% 93
{\PTglyph{5}{t22_l03g33.png}}
% 94
{\PTglyph{5}{t22_l03g34.png}}
% 95
{\PTglyph{5}{t22_l03g35.png}}
% 96
{\PTglyph{5}{t22_l03g36.png}}
% 97
{\PTglyph{5}{t22_l03g37.png}}
% 98
{\PTglyph{5}{t22_l03g38.png}}
% 99
{\PTglyph{5}{t22_l03g39.png}}
% 100
{\PTglyph{5}{t22_l04g01.png}}
% 101
{\PTglyph{5}{t22_l04g02.png}}
% 102
{\PTglyph{5}{t22_l04g03.png}}
% 103
{\PTglyph{5}{t22_l04g04.png}}
% 104
{\PTglyph{5}{t22_l04g05.png}}
% 105
{\PTglyph{5}{t22_l04g06.png}}
% 106
{\PTglyph{5}{t22_l04g07.png}}
% 107
{\PTglyph{5}{t22_l04g08.png}}
% 108
{\PTglyph{5}{t22_l04g09.png}}
% 109
{\PTglyph{5}{t22_l04g10.png}}
% 110
{\PTglyph{5}{t22_l04g11.png}}
% 111
{\PTglyph{5}{t22_l04g12.png}}
% 112
{\PTglyph{5}{t22_l04g13.png}}
% 113
{\PTglyph{5}{t22_l04g14.png}}
% 114
{\PTglyph{5}{t22_l04g15.png}}
% 115
{\PTglyph{5}{t22_l04g16.png}}
//
%%% Local Variables:
%%% mode: latex
%%% TeX-engine: luatex
%%% TeX-master: shared
%%% End:

%//
%\glpismo%
 \glpismo
% 1
{\PTglyphid{Ha-03_0101}}
% 2
{\PTglyphid{Ha-03_0102}}
% 3
{\PTglyphid{Ha-03_0103}}
% 4
{\PTglyphid{Ha-03_0104}}
% 5
{\PTglyphid{Ha-03_0105}}
% 6
{\PTglyphid{Ha-03_0106}}
% 7
{\PTglyphid{Ha-03_0107}}
% 8
{\PTglyphid{Ha-03_0108}}
% 9
{\PTglyphid{Ha-03_0109}}
% 10
{\PTglyphid{Ha-03_0110}}
% 11
{\PTglyphid{Ha-03_0111}}
% 12
{\PTglyphid{Ha-03_0112}}
% 13
{\PTglyphid{Ha-03_0113}}
% 14
{\PTglyphid{Ha-03_0114}}
% 15
{\PTglyphid{Ha-03_0115}}
% 16
{\PTglyphid{Ha-03_0116}}
% 17
{\PTglyphid{Ha-03_0117}}
% 18
{\PTglyphid{Ha-03_0118}}
% 19
{\PTglyphid{Ha-03_0119}}
% 20
{\PTglyphid{Ha-03_0120}}
% 21
{\PTglyphid{Ha-03_0121}}
% 22
{\PTglyphid{Ha-03_0122}}
% 23
{\PTglyphid{Ha-03_0201}}
% 24
{\PTglyphid{Ha-03_0202}}
% 25
{\PTglyphid{Ha-03_0203}}
% 26
{\PTglyphid{Ha-03_0204}}
% 27
{\PTglyphid{Ha-03_0205}}
% 28
{\PTglyphid{Ha-03_0206}}
% 29
{\PTglyphid{Ha-03_0207}}
% 30
{\PTglyphid{Ha-03_0208}}
% 31
{\PTglyphid{Ha-03_0209}}
% 32
{\PTglyphid{Ha-03_0210}}
% 33
{\PTglyphid{Ha-03_0211}}
% 34
{\PTglyphid{Ha-03_0212}}
% 35
{\PTglyphid{Ha-03_0213}}
% 36
{\PTglyphid{Ha-03_0214}}
% 37
{\PTglyphid{Ha-03_0215}}
% 38
{\PTglyphid{Ha-03_0216}}
% 39
{\PTglyphid{Ha-03_0217}}
% 40
{\PTglyphid{Ha-03_0218}}
% 41
{\PTglyphid{Ha-03_0219}}
% 42
{\PTglyphid{Ha-03_0220}}
% 43
{\PTglyphid{Ha-03_0221}}
% 44
{\PTglyphid{Ha-03_0222}}
% 45
{\PTglyphid{Ha-03_0223}}
% 46
{\PTglyphid{Ha-03_0224}}
% 47
{\PTglyphid{Ha-03_0225}}
% 48
{\PTglyphid{Ha-03_0226}}
% 49
{\PTglyphid{Ha-03_0227}}
% 50
{\PTglyphid{Ha-03_0228}}
% 51
{\PTglyphid{Ha-03_0229}}
% 52
{\PTglyphid{Ha-03_0230}}
% 53
{\PTglyphid{Ha-03_0231}}
% 54
{\PTglyphid{Ha-03_0232}}
% 55
{\PTglyphid{Ha-03_0233}}
% 56
{\PTglyphid{Ha-03_0234}}
% 57
{\PTglyphid{Ha-03_0235}}
% 58
{\PTglyphid{Ha-03_0236}}
% 59
{\PTglyphid{Ha-03_0237}}
% 60
{\PTglyphid{Ha-03_0238}}
% 61
{\PTglyphid{Ha-03_0301}}
% 62
{\PTglyphid{Ha-03_0302}}
% 63
{\PTglyphid{Ha-03_0303}}
% 64
{\PTglyphid{Ha-03_0304}}
% 65
{\PTglyphid{Ha-03_0305}}
% 66
{\PTglyphid{Ha-03_0306}}
% 67
{\PTglyphid{Ha-03_0307}}
% 68
{\PTglyphid{Ha-03_0308}}
% 69
{\PTglyphid{Ha-03_0309}}
% 70
{\PTglyphid{Ha-03_0310}}
% 71
{\PTglyphid{Ha-03_0311}}
% 72
{\PTglyphid{Ha-03_0312}}
% 73
{\PTglyphid{Ha-03_0313}}
% 74
{\PTglyphid{Ha-03_0314}}
% 75
{\PTglyphid{Ha-03_0315}}
% 76
{\PTglyphid{Ha-03_0316}}
% 77
{\PTglyphid{Ha-03_0317}}
% 78
{\PTglyphid{Ha-03_0318}}
% 79
{\PTglyphid{Ha-03_0319}}
% 80
{\PTglyphid{Ha-03_0320}}
% 81
{\PTglyphid{Ha-03_0321}}
% 82
{\PTglyphid{Ha-03_0322}}
% 83
{\PTglyphid{Ha-03_0323}}
% 84
{\PTglyphid{Ha-03_0324}}
% 85
{\PTglyphid{Ha-03_0325}}
% 86
{\PTglyphid{Ha-03_0326}}
% 87
{\PTglyphid{Ha-03_0327}}
% 88
{\PTglyphid{Ha-03_0328}}
% 89
{\PTglyphid{Ha-03_0329}}
% 90
{\PTglyphid{Ha-03_0330}}
% 91
{\PTglyphid{Ha-03_0331}}
% 92
{\PTglyphid{Ha-03_0332}}
% 93
{\PTglyphid{Ha-03_0333}}
% 94
{\PTglyphid{Ha-03_0334}}
% 95
{\PTglyphid{Ha-03_0335}}
% 96
{\PTglyphid{Ha-03_0336}}
% 97
{\PTglyphid{Ha-03_0337}}
% 98
{\PTglyphid{Ha-03_0338}}
% 99
{\PTglyphid{Ha-03_0339}}
% 100
{\PTglyphid{Ha-03_0401}}
% 101
{\PTglyphid{Ha-03_0402}}
% 102
{\PTglyphid{Ha-03_0403}}
% 103
{\PTglyphid{Ha-03_0404}}
% 104
{\PTglyphid{Ha-03_0405}}
% 105
{\PTglyphid{Ha-03_0406}}
% 106
{\PTglyphid{Ha-03_0407}}
% 107
{\PTglyphid{Ha-03_0408}}
% 108
{\PTglyphid{Ha-03_0409}}
% 109
{\PTglyphid{Ha-03_0410}}
% 110
{\PTglyphid{Ha-03_0411}}
% 111
{\PTglyphid{Ha-03_0412}}
% 112
{\PTglyphid{Ha-03_0413}}
% 113
{\PTglyphid{Ha-03_0414}}
% 114
{\PTglyphid{Ha-03_0415}}
% 115
{\PTglyphid{Ha-03_0416}}
//
\endgl \xe
%%% Local Variables:
%%% mode: latex
%%% TeX-engine: luatex
%%% TeX-master: shared
%%% End:

% //
%\endgl \xe

 
  \newpage

%%%%%%%%%%%%%%%%%%%%%%%%%%%%%%%%%%%%%%%%%%%%%%%%%%%%%%%%%%%%%%%%%%%%%%%%%%%%%%% 
% Tab. 23,Haller-04_PT04_167.djvu,Haller,04,04,167
%%%%%%%%%%%%%%%%%%%%%%%%%%%%%%%%%%%%%%%%%%%%%%%%%%%%%%%%%%%%%%%%%%%%%%%%%%%%%%

% Note "4. Pismo nagłówkowe. Krój M⁸³. Stopień 20 ww. =112/114 mm. — Tabl. 167."
% Note1 "Character set table prepared by Maria Błońska"


  \pismoPL{Jan Haller 4. Pismo nagłówkowe. Krój M⁸³. Stopień 20 ww. =
    112/114 mm. — Tabl. 167. (Występuje u Hochfedera jako pismo
    2. Tabl. 19).}

  
  \pismoEN{Jan Haller 4. Header font. Typeface M⁸³. Type size 20 ww. =
    112/114 mm. — Plate 167. (Used by Hochfeder as font 2, plate 19.)}

\plate{167}{IV}{1962}

The plate prepared by Helena Kapełuś.\\
The font table prepared by Helena Kapełuś and Maria Błońska.

\bigskip

\exampleBib{IV:21}

\bigskip
\exampleDesc{STATUTA conventus generalis Cracoviensis. [Kraków, Jan Haller, po 2. IIT. 1507]. 2*. Wyd. A.}

\medskip
\examplePage{\textit{Karta [1] a.}}

  \bigskip
\exampleLib{Biblioteka Narodowa. Warszawa.}

\bigskip
\exampleRef{\textit{Estreicher XXIX. 249.}}

\bigskip
%\exampleDig{\url{https://www.wbc.poznan.pl/dlibra/publication/493453/}, page ???}???
\exampleDig{\url{https://wbc.poznan.pl/dlibra/publication/493454} planned (as of \today)}

% \medskip

%     \examplePL{Pismo 2: wiersz 1—4, 17.}

%     \medskip

%     \exampleEN{Font 2: lines 1--4, 17}


\bigskip


\fontID{Ha-04}{23}

\fontstat{137}

% \exdisplay \bg \gla
 \exdisplay \bg \gla
% 1
{\PTglyph{5}{t23_l01g01.png}}
% 2
{\PTglyph{5}{t23_l01g02.png}}
% 3
{\PTglyph{5}{t23_l01g03.png}}
% 4
{\PTglyph{5}{t23_l01g04.png}}
% 5
{\PTglyph{5}{t23_l01g05.png}}
% 6
{\PTglyph{5}{t23_l01g06.png}}
% 7
{\PTglyph{5}{t23_l01g07.png}}
% 8
{\PTglyph{5}{t23_l01g08.png}}
% 9
{\PTglyph{5}{t23_l01g09.png}}
% 10
{\PTglyph{5}{t23_l01g10.png}}
% 11
{\PTglyph{5}{t23_l01g11.png}}
% 12
{\PTglyph{5}{t23_l01g12.png}}
% 13
{\PTglyph{5}{t23_l01g13.png}}
% 14
{\PTglyph{5}{t23_l01g14.png}}
% 15
{\PTglyph{5}{t23_l01g15.png}}
% 16
{\PTglyph{5}{t23_l01g16.png}}
% 17
{\PTglyph{5}{t23_l01g17.png}}
% 18
{\PTglyph{5}{t23_l01g18.png}}
% 19
{\PTglyph{5}{t23_l01g19.png}}
% 20
{\PTglyph{5}{t23_l01g20.png}}
% 21
{\PTglyph{5}{t23_l01g21.png}}
% 22
{\PTglyph{5}{t23_l01g22.png}}
% 23
{\PTglyph{5}{t23_l01g23.png}}
% 24
{\PTglyph{5}{t23_l01g24.png}}
% 25
{\PTglyph{5}{t23_l01g25.png}}
% 26
{\PTglyph{5}{t23_l02g01.png}}
% 27
{\PTglyph{5}{t23_l02g02.png}}
% 28
{\PTglyph{5}{t23_l02g03.png}}
% 29
{\PTglyph{5}{t23_l02g04.png}}
% 30
{\PTglyph{5}{t23_l02g05.png}}
% 31
{\PTglyph{5}{t23_l02g06.png}}
% 32
{\PTglyph{5}{t23_l02g07.png}}
% 33
{\PTglyph{5}{t23_l02g08.png}}
% 34
{\PTglyph{5}{t23_l02g09.png}}
% 35
{\PTglyph{5}{t23_l02g10.png}}
% 36
{\PTglyph{5}{t23_l02g11.png}}
% 37
{\PTglyph{5}{t23_l02g12.png}}
% 38
{\PTglyph{5}{t23_l02g13.png}}
% 39
{\PTglyph{5}{t23_l02g14.png}}
% 40
{\PTglyph{5}{t23_l02g15.png}}
% 41
{\PTglyph{5}{t23_l02g16.png}}
% 42
{\PTglyph{5}{t23_l02g17.png}}
% 43
{\PTglyph{5}{t23_l02g18.png}}
% 44
{\PTglyph{5}{t23_l02g19.png}}
% 45
{\PTglyph{5}{t23_l02g20.png}}
% 46
{\PTglyph{5}{t23_l02g21.png}}
% 47
{\PTglyph{5}{t23_l02g22.png}}
% 48
{\PTglyph{5}{t23_l02g23.png}}
% 49
{\PTglyph{5}{t23_l02g24.png}}
% 50
{\PTglyph{5}{t23_l02g25.png}}
% 51
{\PTglyph{5}{t23_l02g26.png}}
% 52
{\PTglyph{5}{t23_l02g27.png}}
% 53
{\PTglyph{5}{t23_l02g28.png}}
% 54
{\PTglyph{5}{t23_l02g29.png}}
% 55
{\PTglyph{5}{t23_l02g30.png}}
% 56
{\PTglyph{5}{t23_l02g31.png}}
% 57
{\PTglyph{5}{t23_l02g32.png}}
% 58
{\PTglyph{5}{t23_l02g33.png}}
% 59
{\PTglyph{5}{t23_l02g34.png}}
% 60
{\PTglyph{5}{t23_l02g35.png}}
% 61
{\PTglyph{5}{t23_l02g36.png}}
% 62
{\PTglyph{5}{t23_l02g37.png}}
% 63
{\PTglyph{5}{t23_l02g38.png}}
% 64
{\PTglyph{5}{t23_l02g39.png}}
% 65
{\PTglyph{5}{t23_l02g40.png}}
% 66
{\PTglyph{5}{t23_l02g41.png}}
% 67
{\PTglyph{5}{t23_l02g42.png}}
% 68
{\PTglyph{5}{t23_l03g01.png}}
% 69
{\PTglyph{5}{t23_l03g02.png}}
% 70
{\PTglyph{5}{t23_l03g03.png}}
% 71
{\PTglyph{5}{t23_l03g04.png}}
% 72
{\PTglyph{5}{t23_l03g05.png}}
% 73
{\PTglyph{5}{t23_l03g06.png}}
% 74
{\PTglyph{5}{t23_l03g07.png}}
% 75
{\PTglyph{5}{t23_l03g08.png}}
% 76
{\PTglyph{5}{t23_l03g09.png}}
% 77
{\PTglyph{5}{t23_l03g10.png}}
% 78
{\PTglyph{5}{t23_l03g11.png}}
% 79
{\PTglyph{5}{t23_l03g12.png}}
% 80
{\PTglyph{5}{t23_l03g13.png}}
% 81
{\PTglyph{5}{t23_l03g14.png}}
% 82
{\PTglyph{5}{t23_l03g15.png}}
% 83
{\PTglyph{5}{t23_l03g16.png}}
% 84
{\PTglyph{5}{t23_l03g17.png}}
% 85
{\PTglyph{5}{t23_l03g18.png}}
% 86
{\PTglyph{5}{t23_l03g19.png}}
% 87
{\PTglyph{5}{t23_l03g20.png}}
% 88
{\PTglyph{5}{t23_l03g21.png}}
% 89
{\PTglyph{5}{t23_l03g22.png}}
% 90
{\PTglyph{5}{t23_l03g23.png}}
% 91
{\PTglyph{5}{t23_l03g24.png}}
% 92
{\PTglyph{5}{t23_l03g25.png}}
% 93
{\PTglyph{5}{t23_l03g26.png}}
% 94
{\PTglyph{5}{t23_l03g27.png}}
% 95
{\PTglyph{5}{t23_l03g28.png}}
% 96
{\PTglyph{5}{t23_l03g29.png}}
% 97
{\PTglyph{5}{t23_l03g30.png}}
% 98
{\PTglyph{5}{t23_l03g31.png}}
% 99
{\PTglyph{5}{t23_l03g32.png}}
% 100
{\PTglyph{5}{t23_l03g33.png}}
% 101
{\PTglyph{5}{t23_l03g34.png}}
% 102
{\PTglyph{5}{t23_l03g35.png}}
% 103
{\PTglyph{5}{t23_l03g36.png}}
% 104
{\PTglyph{5}{t23_l03g37.png}}
% 105
{\PTglyph{5}{t23_l03g38.png}}
% 106
{\PTglyph{5}{t23_l03g39.png}}
% 107
{\PTglyph{5}{t23_l04g01.png}}
% 108
{\PTglyph{5}{t23_l04g02.png}}
% 109
{\PTglyph{5}{t23_l04g03.png}}
% 110
{\PTglyph{5}{t23_l04g04.png}}
% 111
{\PTglyph{5}{t23_l04g05.png}}
% 112
{\PTglyph{5}{t23_l04g06.png}}
% 113
{\PTglyph{5}{t23_l04g07.png}}
% 114
{\PTglyph{5}{t23_l04g08.png}}
% 115
{\PTglyph{5}{t23_l04g09.png}}
% 116
{\PTglyph{5}{t23_l04g10.png}}
% 117
{\PTglyph{5}{t23_l04g11.png}}
% 118
{\PTglyph{5}{t23_l04g12.png}}
% 119
{\PTglyph{5}{t23_l04g13.png}}
% 120
{\PTglyph{5}{t23_l04g14.png}}
% 121
{\PTglyph{5}{t23_l04g15.png}}
% 122
{\PTglyph{5}{t23_l04g16.png}}
% 123
{\PTglyph{5}{t23_l04g17.png}}
% 124
{\PTglyph{5}{t23_l04g18.png}}
% 125
{\PTglyph{5}{t23_l04g19.png}}
% 126
{\PTglyph{5}{t23_l04g20.png}}
% 127
{\PTglyph{5}{t23_l04g21.png}}
% 128
{\PTglyph{5}{t23_l04g22.png}}
% 129
{\PTglyph{5}{t23_l04g23.png}}
% 130
{\PTglyph{5}{t23_l04g24.png}}
% 131
{\PTglyph{5}{t23_l04g25.png}}
% 132
{\PTglyph{5}{t23_l04g26.png}}
% 133
{\PTglyph{5}{t23_l04g27.png}}
% 134
{\PTglyph{5}{t23_l04g28.png}}
% 135
{\PTglyph{5}{t23_l04g29.png}}
% 136
{\PTglyph{5}{t23_l04g30.png}}
% 137
{\PTglyph{5}{t23_l04g31.png}}
//
%%% Local Variables:
%%% mode: latex
%%% TeX-engine: luatex
%%% TeX-master: shared
%%% End:

%//
%\glpismo%
 \glpismo
% 1
{\PTglyphid{Ha-04_0101}}
% 2
{\PTglyphid{Ha-04_0102}}
% 3
{\PTglyphid{Ha-04_0103}}
% 4
{\PTglyphid{Ha-04_0104}}
% 5
{\PTglyphid{Ha-04_0105}}
% 6
{\PTglyphid{Ha-04_0106}}
% 7
{\PTglyphid{Ha-04_0107}}
% 8
{\PTglyphid{Ha-04_0108}}
% 9
{\PTglyphid{Ha-04_0109}}
% 10
{\PTglyphid{Ha-04_0110}}
% 11
{\PTglyphid{Ha-04_0111}}
% 12
{\PTglyphid{Ha-04_0112}}
% 13
{\PTglyphid{Ha-04_0113}}
% 14
{\PTglyphid{Ha-04_0114}}
% 15
{\PTglyphid{Ha-04_0115}}
% 16
{\PTglyphid{Ha-04_0116}}
% 17
{\PTglyphid{Ha-04_0117}}
% 18
{\PTglyphid{Ha-04_0118}}
% 19
{\PTglyphid{Ha-04_0119}}
% 20
{\PTglyphid{Ha-04_0120}}
% 21
{\PTglyphid{Ha-04_0121}}
% 22
{\PTglyphid{Ha-04_0122}}
% 23
{\PTglyphid{Ha-04_0123}}
% 24
{\PTglyphid{Ha-04_0124}}
% 25
{\PTglyphid{Ha-04_0125}}
% 26
{\PTglyphid{Ha-04_0201}}
% 27
{\PTglyphid{Ha-04_0202}}
% 28
{\PTglyphid{Ha-04_0203}}
% 29
{\PTglyphid{Ha-04_0204}}
% 30
{\PTglyphid{Ha-04_0205}}
% 31
{\PTglyphid{Ha-04_0206}}
% 32
{\PTglyphid{Ha-04_0207}}
% 33
{\PTglyphid{Ha-04_0208}}
% 34
{\PTglyphid{Ha-04_0209}}
% 35
{\PTglyphid{Ha-04_0210}}
% 36
{\PTglyphid{Ha-04_0211}}
% 37
{\PTglyphid{Ha-04_0212}}
% 38
{\PTglyphid{Ha-04_0213}}
% 39
{\PTglyphid{Ha-04_0214}}
% 40
{\PTglyphid{Ha-04_0215}}
% 41
{\PTglyphid{Ha-04_0216}}
% 42
{\PTglyphid{Ha-04_0217}}
% 43
{\PTglyphid{Ha-04_0218}}
% 44
{\PTglyphid{Ha-04_0219}}
% 45
{\PTglyphid{Ha-04_0220}}
% 46
{\PTglyphid{Ha-04_0221}}
% 47
{\PTglyphid{Ha-04_0222}}
% 48
{\PTglyphid{Ha-04_0223}}
% 49
{\PTglyphid{Ha-04_0224}}
% 50
{\PTglyphid{Ha-04_0225}}
% 51
{\PTglyphid{Ha-04_0226}}
% 52
{\PTglyphid{Ha-04_0227}}
% 53
{\PTglyphid{Ha-04_0228}}
% 54
{\PTglyphid{Ha-04_0229}}
% 55
{\PTglyphid{Ha-04_0230}}
% 56
{\PTglyphid{Ha-04_0231}}
% 57
{\PTglyphid{Ha-04_0232}}
% 58
{\PTglyphid{Ha-04_0233}}
% 59
{\PTglyphid{Ha-04_0234}}
% 60
{\PTglyphid{Ha-04_0235}}
% 61
{\PTglyphid{Ha-04_0236}}
% 62
{\PTglyphid{Ha-04_0237}}
% 63
{\PTglyphid{Ha-04_0238}}
% 64
{\PTglyphid{Ha-04_0239}}
% 65
{\PTglyphid{Ha-04_0240}}
% 66
{\PTglyphid{Ha-04_0241}}
% 67
{\PTglyphid{Ha-04_0242}}
% 68
{\PTglyphid{Ha-04_0301}}
% 69
{\PTglyphid{Ha-04_0302}}
% 70
{\PTglyphid{Ha-04_0303}}
% 71
{\PTglyphid{Ha-04_0304}}
% 72
{\PTglyphid{Ha-04_0305}}
% 73
{\PTglyphid{Ha-04_0306}}
% 74
{\PTglyphid{Ha-04_0307}}
% 75
{\PTglyphid{Ha-04_0308}}
% 76
{\PTglyphid{Ha-04_0309}}
% 77
{\PTglyphid{Ha-04_0310}}
% 78
{\PTglyphid{Ha-04_0311}}
% 79
{\PTglyphid{Ha-04_0312}}
% 80
{\PTglyphid{Ha-04_0313}}
% 81
{\PTglyphid{Ha-04_0314}}
% 82
{\PTglyphid{Ha-04_0315}}
% 83
{\PTglyphid{Ha-04_0316}}
% 84
{\PTglyphid{Ha-04_0317}}
% 85
{\PTglyphid{Ha-04_0318}}
% 86
{\PTglyphid{Ha-04_0319}}
% 87
{\PTglyphid{Ha-04_0320}}
% 88
{\PTglyphid{Ha-04_0321}}
% 89
{\PTglyphid{Ha-04_0322}}
% 90
{\PTglyphid{Ha-04_0323}}
% 91
{\PTglyphid{Ha-04_0324}}
% 92
{\PTglyphid{Ha-04_0325}}
% 93
{\PTglyphid{Ha-04_0326}}
% 94
{\PTglyphid{Ha-04_0327}}
% 95
{\PTglyphid{Ha-04_0328}}
% 96
{\PTglyphid{Ha-04_0329}}
% 97
{\PTglyphid{Ha-04_0330}}
% 98
{\PTglyphid{Ha-04_0331}}
% 99
{\PTglyphid{Ha-04_0332}}
% 100
{\PTglyphid{Ha-04_0333}}
% 101
{\PTglyphid{Ha-04_0334}}
% 102
{\PTglyphid{Ha-04_0335}}
% 103
{\PTglyphid{Ha-04_0336}}
% 104
{\PTglyphid{Ha-04_0337}}
% 105
{\PTglyphid{Ha-04_0338}}
% 106
{\PTglyphid{Ha-04_0339}}
% 107
{\PTglyphid{Ha-04_0401}}
% 108
{\PTglyphid{Ha-04_0402}}
% 109
{\PTglyphid{Ha-04_0403}}
% 110
{\PTglyphid{Ha-04_0404}}
% 111
{\PTglyphid{Ha-04_0405}}
% 112
{\PTglyphid{Ha-04_0406}}
% 113
{\PTglyphid{Ha-04_0407}}
% 114
{\PTglyphid{Ha-04_0408}}
% 115
{\PTglyphid{Ha-04_0409}}
% 116
{\PTglyphid{Ha-04_0410}}
% 117
{\PTglyphid{Ha-04_0411}}
% 118
{\PTglyphid{Ha-04_0412}}
% 119
{\PTglyphid{Ha-04_0413}}
% 120
{\PTglyphid{Ha-04_0414}}
% 121
{\PTglyphid{Ha-04_0415}}
% 122
{\PTglyphid{Ha-04_0416}}
% 123
{\PTglyphid{Ha-04_0417}}
% 124
{\PTglyphid{Ha-04_0418}}
% 125
{\PTglyphid{Ha-04_0419}}
% 126
{\PTglyphid{Ha-04_0420}}
% 127
{\PTglyphid{Ha-04_0421}}
% 128
{\PTglyphid{Ha-04_0422}}
% 129
{\PTglyphid{Ha-04_0423}}
% 130
{\PTglyphid{Ha-04_0424}}
% 131
{\PTglyphid{Ha-04_0425}}
% 132
{\PTglyphid{Ha-04_0426}}
% 133
{\PTglyphid{Ha-04_0427}}
% 134
{\PTglyphid{Ha-04_0428}}
% 135
{\PTglyphid{Ha-04_0429}}
% 136
{\PTglyphid{Ha-04_0430}}
% 137
{\PTglyphid{Ha-04_0431}}
//
\endgl \xe
%%% Local Variables:
%%% mode: latex
%%% TeX-engine: luatex
%%% TeX-master: shared
%%% End:

% //
%\endgl \xe

\newpage

%%%%%%%%%%%%%%%%%%%%%%%%%%%%%%%%%%%%%%%%%%%%%%%%%%%%%%%%%%%%%%%%%%%%%%%%%%%%%%% 
% Tab. 24,Haller-05_PT04_168.djvu,Haller,05,04,168
%%%%%%%%%%%%%%%%%%%%%%%%%%%%%%%%%%%%%%%%%%%%%%%%%%%%%%%%%%%%%%%%%%%%%%%%%%%%%%

% Note "5. Pismo tekstowe. Krój M⁴⁸. Stopień 20 ww. =81/82 mm. — Tabl. 168."
% Note1 "Character set table prepared by Maria Błońska"

  \pismoPL{Jan Haller 5. Pismo tekstowe. Krój M⁴⁸. Stopień 20
    ww. =81/82 mm. — Tabl. 168. (Występuje u Hochfedera jako pismo
    9. Tabl. 26, u Unglera jako pismo 4.  Tabl. 115).}


  
\pismoEN{Jan Haller 5. Text font. Typeface M⁴⁸. Type size 20 ww. =
    81/82 mm. — Plate 168. (Used by Hochfeder as font 9, plate
    26, and by Ungler as font 4, plate 115.)}

\plate{168}{IV}{1962}

Prepared by Helena Kapełuś.\\
The font table prepared by Helena Kapełuś and Maria Błońska.

\bigskip

\exampleBib{IV:17}

\bigskip
\exampleDesc{17. IOANNES SACRANUS: Hlucidarius errorum ritus Ruthenici. [Kraków, Jan Haller, ok. 1507]. 4⁰.}
  

\medskip
\examplePage{\textit{Karta 7 a.}}

  \bigskip
\exampleLib{Biblioteka Narodowa. Warszawa.}

\bigskip
\exampleRef{\textit{Estreicher XVII. 13. Wierzbowski 773.}}

\bigskip
\exampleDig{\url{http://old.mbc.malopolska.pl/dlibra/docmetadata?id=83069}
page 13,\\
\url{https://www.dbc.wroc.pl/dlibra/publication/3624/edition/3513}
page 7.}
  % https://www.dbc.wroc.pl/dlibra/publication/3624/edition/3513?language=en
  % http://old.mbc.malopolska.pl/dlibra/docmetadata?id=83069&from=publication
  
  % \medskip
\bigskip

    \examplePL{Pismo 4: naglówek.}

    \medskip

    \exampleEN{Font 4: the header}


\bigskip


\fontID{Ha-05}{24}

\fontstat{135}

% \exdisplay \bg \gla
 \exdisplay \bg \gla
% 1
{\PTglyph{5}{t24_l01g01.png}}
% 2
{\PTglyph{5}{t24_l01g02.png}}
% 3
{\PTglyph{5}{t24_l01g03.png}}
% 4
{\PTglyph{5}{t24_l01g04.png}}
% 5
{\PTglyph{5}{t24_l01g05.png}}
% 6
{\PTglyph{5}{t24_l01g06.png}}
% 7
{\PTglyph{5}{t24_l01g07.png}}
% 8
{\PTglyph{5}{t24_l01g08.png}}
% 9
{\PTglyph{5}{t24_l01g09.png}}
% 10
{\PTglyph{5}{t24_l01g10.png}}
% 11
{\PTglyph{5}{t24_l01g11.png}}
% 12
{\PTglyph{5}{t24_l01g12.png}}
% 13
{\PTglyph{5}{t24_l01g13.png}}
% 14
{\PTglyph{5}{t24_l01g14.png}}
% 15
{\PTglyph{5}{t24_l01g15.png}}
% 16
{\PTglyph{5}{t24_l01g16.png}}
% 17
{\PTglyph{5}{t24_l01g17.png}}
% 18
{\PTglyph{5}{t24_l01g18.png}}
% 19
{\PTglyph{5}{t24_l01g19.png}}
% 20
{\PTglyph{5}{t24_l01g20.png}}
% 21
{\PTglyph{5}{t24_l01g21.png}}
% 22
{\PTglyph{5}{t24_l01g22.png}}
% 23
{\PTglyph{5}{t24_l01g23.png}}
% 24
{\PTglyph{5}{t24_l01g24.png}}
% 25
{\PTglyph{5}{t24_l01g25.png}}
% 26
{\PTglyph{5}{t24_l01g26.png}}
% 27
{\PTglyph{5}{t24_l01g27.png}}
% 28
{\PTglyph{5}{t24_l02g01.png}}
% 29
{\PTglyph{5}{t24_l02g02.png}}
% 30
{\PTglyph{5}{t24_l02g03.png}}
% 31
{\PTglyph{5}{t24_l02g04.png}}
% 32
{\PTglyph{5}{t24_l02g05.png}}
% 33
{\PTglyph{5}{t24_l02g06.png}}
% 34
{\PTglyph{5}{t24_l02g07.png}}
% 35
{\PTglyph{5}{t24_l02g08.png}}
% 36
{\PTglyph{5}{t24_l02g09.png}}
% 37
{\PTglyph{5}{t24_l02g10.png}}
% 38
{\PTglyph{5}{t24_l02g11.png}}
% 39
{\PTglyph{5}{t24_l02g12.png}}
% 40
{\PTglyph{5}{t24_l02g13.png}}
% 41
{\PTglyph{5}{t24_l02g14.png}}
% 42
{\PTglyph{5}{t24_l02g15.png}}
% 43
{\PTglyph{5}{t24_l02g16.png}}
% 44
{\PTglyph{5}{t24_l02g17.png}}
% 45
{\PTglyph{5}{t24_l02g18.png}}
% 46
{\PTglyph{5}{t24_l02g19.png}}
% 47
{\PTglyph{5}{t24_l02g20.png}}
% 48
{\PTglyph{5}{t24_l02g21.png}}
% 49
{\PTglyph{5}{t24_l02g22.png}}
% 50
{\PTglyph{5}{t24_l02g23.png}}
% 51
{\PTglyph{5}{t24_l02g24.png}}
% 52
{\PTglyph{5}{t24_l02g25.png}}
% 53
{\PTglyph{5}{t24_l02g26.png}}
% 54
{\PTglyph{5}{t24_l02g27.png}}
% 55
{\PTglyph{5}{t24_l02g28.png}}
% 56
{\PTglyph{5}{t24_l02g29.png}}
% 57
{\PTglyph{5}{t24_l02g30.png}}
% 58
{\PTglyph{5}{t24_l02g31.png}}
% 59
{\PTglyph{5}{t24_l02g32.png}}
% 60
{\PTglyph{5}{t24_l02g33.png}}
% 61
{\PTglyph{5}{t24_l02g34.png}}
% 62
{\PTglyph{5}{t24_l02g35.png}}
% 63
{\PTglyph{5}{t24_l02g36.png}}
% 64
{\PTglyph{5}{t24_l02g37.png}}
% 65
{\PTglyph{5}{t24_l02g38.png}}
% 66
{\PTglyph{5}{t24_l02g39.png}}
% 67
{\PTglyph{5}{t24_l02g40.png}}
% 68
{\PTglyph{5}{t24_l02g41.png}}
% 69
{\PTglyph{5}{t24_l03g01.png}}
% 70
{\PTglyph{5}{t24_l03g02.png}}
% 71
{\PTglyph{5}{t24_l03g03.png}}
% 72
{\PTglyph{5}{t24_l03g04.png}}
% 73
{\PTglyph{5}{t24_l03g05.png}}
% 74
{\PTglyph{5}{t24_l03g06.png}}
% 75
{\PTglyph{5}{t24_l03g07.png}}
% 76
{\PTglyph{5}{t24_l03g08.png}}
% 77
{\PTglyph{5}{t24_l03g09.png}}
% 78
{\PTglyph{5}{t24_l03g10.png}}
% 79
{\PTglyph{5}{t24_l03g11.png}}
% 80
{\PTglyph{5}{t24_l03g12.png}}
% 81
{\PTglyph{5}{t24_l03g13.png}}
% 82
{\PTglyph{5}{t24_l03g14.png}}
% 83
{\PTglyph{5}{t24_l03g15.png}}
% 84
{\PTglyph{5}{t24_l03g16.png}}
% 85
{\PTglyph{5}{t24_l03g17.png}}
% 86
{\PTglyph{5}{t24_l03g18.png}}
% 87
{\PTglyph{5}{t24_l03g19.png}}
% 88
{\PTglyph{5}{t24_l03g20.png}}
% 89
{\PTglyph{5}{t24_l03g21.png}}
% 90
{\PTglyph{5}{t24_l03g22.png}}
% 91
{\PTglyph{5}{t24_l03g23.png}}
% 92
{\PTglyph{5}{t24_l03g24.png}}
% 93
{\PTglyph{5}{t24_l03g25.png}}
% 94
{\PTglyph{5}{t24_l03g26.png}}
% 95
{\PTglyph{5}{t24_l03g27.png}}
% 96
{\PTglyph{5}{t24_l03g28.png}}
% 97
{\PTglyph{5}{t24_l03g29.png}}
% 98
{\PTglyph{5}{t24_l03g30.png}}
% 99
{\PTglyph{5}{t24_l03g31.png}}
% 100
{\PTglyph{5}{t24_l03g32.png}}
% 101
{\PTglyph{5}{t24_l03g33.png}}
% 102
{\PTglyph{5}{t24_l03g34.png}}
% 103
{\PTglyph{5}{t24_l03g35.png}}
% 104
{\PTglyph{5}{t24_l03g36.png}}
% 105
{\PTglyph{5}{t24_l03g37.png}}
% 106
{\PTglyph{5}{t24_l03g38.png}}
% 107
{\PTglyph{5}{t24_l03g39.png}}
% 108
{\PTglyph{5}{t24_l04g01.png}}
% 109
{\PTglyph{5}{t24_l04g02.png}}
% 110
{\PTglyph{5}{t24_l04g03.png}}
% 111
{\PTglyph{5}{t24_l04g04.png}}
% 112
{\PTglyph{5}{t24_l04g05.png}}
% 113
{\PTglyph{5}{t24_l04g06.png}}
% 114
{\PTglyph{5}{t24_l04g07.png}}
% 115
{\PTglyph{5}{t24_l04g08.png}}
% 116
{\PTglyph{5}{t24_l04g09.png}}
% 117
{\PTglyph{5}{t24_l04g10.png}}
% 118
{\PTglyph{5}{t24_l04g11.png}}
% 119
{\PTglyph{5}{t24_l04g12.png}}
% 120
{\PTglyph{5}{t24_l04g13.png}}
% 121
{\PTglyph{5}{t24_l04g14.png}}
% 122
{\PTglyph{5}{t24_l04g15.png}}
% 123
{\PTglyph{5}{t24_l04g16.png}}
% 124
{\PTglyph{5}{t24_l04g17.png}}
% 125
{\PTglyph{5}{t24_l04g18.png}}
% 126
{\PTglyph{5}{t24_l04g19.png}}
% 127
{\PTglyph{5}{t24_l04g20.png}}
% 128
{\PTglyph{5}{t24_l04g21.png}}
% 129
{\PTglyph{5}{t24_l04g22.png}}
% 130
{\PTglyph{5}{t24_l04g23.png}}
% 131
{\PTglyph{5}{t24_l04g24.png}}
% 132
{\PTglyph{5}{t24_l04g25.png}}
% 133
{\PTglyph{5}{t24_l04g26.png}}
% 134
{\PTglyph{5}{t24_l04g27.png}}
% 135
{\PTglyph{5}{t24_l04g28.png}}
//
%%% Local Variables:
%%% mode: latex
%%% TeX-engine: luatex
%%% TeX-master: shared
%%% End:

%//
%\glpismo%
 \glpismo
% 1
{\PTglyphid{Ha-05_0101}}
% 2
{\PTglyphid{Ha-05_0102}}
% 3
{\PTglyphid{Ha-05_0103}}
% 4
{\PTglyphid{Ha-05_0104}}
% 5
{\PTglyphid{Ha-05_0105}}
% 6
{\PTglyphid{Ha-05_0106}}
% 7
{\PTglyphid{Ha-05_0107}}
% 8
{\PTglyphid{Ha-05_0108}}
% 9
{\PTglyphid{Ha-05_0109}}
% 10
{\PTglyphid{Ha-05_0110}}
% 11
{\PTglyphid{Ha-05_0111}}
% 12
{\PTglyphid{Ha-05_0112}}
% 13
{\PTglyphid{Ha-05_0113}}
% 14
{\PTglyphid{Ha-05_0114}}
% 15
{\PTglyphid{Ha-05_0115}}
% 16
{\PTglyphid{Ha-05_0116}}
% 17
{\PTglyphid{Ha-05_0117}}
% 18
{\PTglyphid{Ha-05_0118}}
% 19
{\PTglyphid{Ha-05_0119}}
% 20
{\PTglyphid{Ha-05_0120}}
% 21
{\PTglyphid{Ha-05_0121}}
% 22
{\PTglyphid{Ha-05_0122}}
% 23
{\PTglyphid{Ha-05_0123}}
% 24
{\PTglyphid{Ha-05_0124}}
% 25
{\PTglyphid{Ha-05_0125}}
% 26
{\PTglyphid{Ha-05_0126}}
% 27
{\PTglyphid{Ha-05_0127}}
% 28
{\PTglyphid{Ha-05_0201}}
% 29
{\PTglyphid{Ha-05_0202}}
% 30
{\PTglyphid{Ha-05_0203}}
% 31
{\PTglyphid{Ha-05_0204}}
% 32
{\PTglyphid{Ha-05_0205}}
% 33
{\PTglyphid{Ha-05_0206}}
% 34
{\PTglyphid{Ha-05_0207}}
% 35
{\PTglyphid{Ha-05_0208}}
% 36
{\PTglyphid{Ha-05_0209}}
% 37
{\PTglyphid{Ha-05_0210}}
% 38
{\PTglyphid{Ha-05_0211}}
% 39
{\PTglyphid{Ha-05_0212}}
% 40
{\PTglyphid{Ha-05_0213}}
% 41
{\PTglyphid{Ha-05_0214}}
% 42
{\PTglyphid{Ha-05_0215}}
% 43
{\PTglyphid{Ha-05_0216}}
% 44
{\PTglyphid{Ha-05_0217}}
% 45
{\PTglyphid{Ha-05_0218}}
% 46
{\PTglyphid{Ha-05_0219}}
% 47
{\PTglyphid{Ha-05_0220}}
% 48
{\PTglyphid{Ha-05_0221}}
% 49
{\PTglyphid{Ha-05_0222}}
% 50
{\PTglyphid{Ha-05_0223}}
% 51
{\PTglyphid{Ha-05_0224}}
% 52
{\PTglyphid{Ha-05_0225}}
% 53
{\PTglyphid{Ha-05_0226}}
% 54
{\PTglyphid{Ha-05_0227}}
% 55
{\PTglyphid{Ha-05_0228}}
% 56
{\PTglyphid{Ha-05_0229}}
% 57
{\PTglyphid{Ha-05_0230}}
% 58
{\PTglyphid{Ha-05_0231}}
% 59
{\PTglyphid{Ha-05_0232}}
% 60
{\PTglyphid{Ha-05_0233}}
% 61
{\PTglyphid{Ha-05_0234}}
% 62
{\PTglyphid{Ha-05_0235}}
% 63
{\PTglyphid{Ha-05_0236}}
% 64
{\PTglyphid{Ha-05_0237}}
% 65
{\PTglyphid{Ha-05_0238}}
% 66
{\PTglyphid{Ha-05_0239}}
% 67
{\PTglyphid{Ha-05_0240}}
% 68
{\PTglyphid{Ha-05_0241}}
% 69
{\PTglyphid{Ha-05_0301}}
% 70
{\PTglyphid{Ha-05_0302}}
% 71
{\PTglyphid{Ha-05_0303}}
% 72
{\PTglyphid{Ha-05_0304}}
% 73
{\PTglyphid{Ha-05_0305}}
% 74
{\PTglyphid{Ha-05_0306}}
% 75
{\PTglyphid{Ha-05_0307}}
% 76
{\PTglyphid{Ha-05_0308}}
% 77
{\PTglyphid{Ha-05_0309}}
% 78
{\PTglyphid{Ha-05_0310}}
% 79
{\PTglyphid{Ha-05_0311}}
% 80
{\PTglyphid{Ha-05_0312}}
% 81
{\PTglyphid{Ha-05_0313}}
% 82
{\PTglyphid{Ha-05_0314}}
% 83
{\PTglyphid{Ha-05_0315}}
% 84
{\PTglyphid{Ha-05_0316}}
% 85
{\PTglyphid{Ha-05_0317}}
% 86
{\PTglyphid{Ha-05_0318}}
% 87
{\PTglyphid{Ha-05_0319}}
% 88
{\PTglyphid{Ha-05_0320}}
% 89
{\PTglyphid{Ha-05_0321}}
% 90
{\PTglyphid{Ha-05_0322}}
% 91
{\PTglyphid{Ha-05_0323}}
% 92
{\PTglyphid{Ha-05_0324}}
% 93
{\PTglyphid{Ha-05_0325}}
% 94
{\PTglyphid{Ha-05_0326}}
% 95
{\PTglyphid{Ha-05_0327}}
% 96
{\PTglyphid{Ha-05_0328}}
% 97
{\PTglyphid{Ha-05_0329}}
% 98
{\PTglyphid{Ha-05_0330}}
% 99
{\PTglyphid{Ha-05_0331}}
% 100
{\PTglyphid{Ha-05_0332}}
% 101
{\PTglyphid{Ha-05_0333}}
% 102
{\PTglyphid{Ha-05_0334}}
% 103
{\PTglyphid{Ha-05_0335}}
% 104
{\PTglyphid{Ha-05_0336}}
% 105
{\PTglyphid{Ha-05_0337}}
% 106
{\PTglyphid{Ha-05_0338}}
% 107
{\PTglyphid{Ha-05_0339}}
% 108
{\PTglyphid{Ha-05_0401}}
% 109
{\PTglyphid{Ha-05_0402}}
% 110
{\PTglyphid{Ha-05_0403}}
% 111
{\PTglyphid{Ha-05_0404}}
% 112
{\PTglyphid{Ha-05_0405}}
% 113
{\PTglyphid{Ha-05_0406}}
% 114
{\PTglyphid{Ha-05_0407}}
% 115
{\PTglyphid{Ha-05_0408}}
% 116
{\PTglyphid{Ha-05_0409}}
% 117
{\PTglyphid{Ha-05_0410}}
% 118
{\PTglyphid{Ha-05_0411}}
% 119
{\PTglyphid{Ha-05_0412}}
% 120
{\PTglyphid{Ha-05_0413}}
% 121
{\PTglyphid{Ha-05_0414}}
% 122
{\PTglyphid{Ha-05_0415}}
% 123
{\PTglyphid{Ha-05_0416}}
% 124
{\PTglyphid{Ha-05_0417}}
% 125
{\PTglyphid{Ha-05_0418}}
% 126
{\PTglyphid{Ha-05_0419}}
% 127
{\PTglyphid{Ha-05_0420}}
% 128
{\PTglyphid{Ha-05_0421}}
% 129
{\PTglyphid{Ha-05_0422}}
% 130
{\PTglyphid{Ha-05_0423}}
% 131
{\PTglyphid{Ha-05_0424}}
% 132
{\PTglyphid{Ha-05_0425}}
% 133
{\PTglyphid{Ha-05_0426}}
% 134
{\PTglyphid{Ha-05_0427}}
% 135
{\PTglyphid{Ha-05_0428}}
//
\endgl \xe
%%% Local Variables:
%%% mode: latex
%%% TeX-engine: luatex
%%% TeX-master: shared
%%% End:

% //
%\endgl \xe

 
  \newpage

%%%%%%%%%%%%%%%%%%%%%%%%%%%%%%%%%%%%%%%%%%%%%%%%%%%%%%%%%%%%%%%%%%%%%%%%%%%%%%% 
% Tab. 25,Haller-06_PT04_169.djvu,Haller,06,04,169
%%%%%%%%%%%%%%%%%%%%%%%%%%%%%%%%%%%%%%%%%%%%%%%%%%%%%%%%%%%%%%%%%%%%%%%%%%%%%%

% Note "6. Pismo komentarzowe. Krój M”. Stopień 20 ww. = 82 mm (interliniowane). — Tabl. 169."
% Note1 "Character set table prepared by Maria Błońska"


  \pismoPL{Jan Haller 6. Pismo komentarzowe. Krój M⁹¹. Stopień 20
    ww. = 82 mm (interliniowane). — Tabl. 169. (Występuje u Hochfedera
    jako pismo 4. Tabl. 21).}



  
\pismoEN{Jan Haller 6. Text font. Typeface M⁹¹. Type size 20 ww. =
    82 mm (with extra leading). — Plate 169. (Used by Hochfeder as font 4, plate
    21.)}

\plate{169}{IV}{1962}

The plate prepared by Helena Kapełuś.\\
The font table prepared by Helena Kapełuś and Maria Błońska.

\bigskip

\exampleBib{IV:23}

\bigskip \exampleDesc{THEOREMATA Autoris causarum cum
  annotationibus ac expositione Iacobi de Gostynin. Kraków, Jan
  Haller, 23. III. 1507. 4⁰.}
  

\medskip
\examplePage{\textit{Karta b₃a.}}

  \bigskip
\exampleLib{Biblioteka Zakl. Nar. im. Ossolińskich. Wrocław.}

\bigskip
\exampleRef{\textit{Estreicher XV. 85. Wierzbowski 848.}}

\bigskip
\exampleDig{\url{https://www.wbc.poznan.pl/dlibra/publication/310579/} page 25.}
  
  % \medskip
\bigskip

    \examplePL{Pismo 4: naglówek. [Pismo 6: tekst - JSB]}

    \medskip

    \exampleEN{Font 4: the header [Font 6: the text - JSB]}


\bigskip


\fontID{Ha-06}{25}

\fontstat{94}

% \exdisplay \bg \gla
 \exdisplay \bg \gla
% 1
{\PTglyph{5}{t25_l01g01.png}}
% 2
{\PTglyph{5}{t25_l01g02.png}}
% 3
{\PTglyph{5}{t25_l01g03.png}}
% 4
{\PTglyph{5}{t25_l01g04.png}}
% 5
{\PTglyph{5}{t25_l01g05.png}}
% 6
{\PTglyph{5}{t25_l01g06.png}}
% 7
{\PTglyph{5}{t25_l01g07.png}}
% 8
{\PTglyph{5}{t25_l01g08.png}}
% 9
{\PTglyph{5}{t25_l01g09.png}}
% 10
{\PTglyph{5}{t25_l01g10.png}}
% 11
{\PTglyph{5}{t25_l01g11.png}}
% 12
{\PTglyph{5}{t25_l01g12.png}}
% 13
{\PTglyph{5}{t25_l01g13.png}}
% 14
{\PTglyph{5}{t25_l01g14.png}}
% 15
{\PTglyph{5}{t25_l01g15.png}}
% 16
{\PTglyph{5}{t25_l01g16.png}}
% 17
{\PTglyph{5}{t25_l01g17.png}}
% 18
{\PTglyph{5}{t25_l01g18.png}}
% 19
{\PTglyph{5}{t25_l02g01.png}}
% 20
{\PTglyph{5}{t25_l02g02.png}}
% 21
{\PTglyph{5}{t25_l02g03.png}}
% 22
{\PTglyph{5}{t25_l02g04.png}}
% 23
{\PTglyph{5}{t25_l02g05.png}}
% 24
{\PTglyph{5}{t25_l02g06.png}}
% 25
{\PTglyph{5}{t25_l02g07.png}}
% 26
{\PTglyph{5}{t25_l02g08.png}}
% 27
{\PTglyph{5}{t25_l02g09.png}}
% 28
{\PTglyph{5}{t25_l02g10.png}}
% 29
{\PTglyph{5}{t25_l02g11.png}}
% 30
{\PTglyph{5}{t25_l02g12.png}}
% 31
{\PTglyph{5}{t25_l02g13.png}}
% 32
{\PTglyph{5}{t25_l02g14.png}}
% 33
{\PTglyph{5}{t25_l02g15.png}}
% 34
{\PTglyph{5}{t25_l02g16.png}}
% 35
{\PTglyph{5}{t25_l02g17.png}}
% 36
{\PTglyph{5}{t25_l02g18.png}}
% 37
{\PTglyph{5}{t25_l02g19.png}}
% 38
{\PTglyph{5}{t25_l02g20.png}}
% 39
{\PTglyph{5}{t25_l02g21.png}}
% 40
{\PTglyph{5}{t25_l02g22.png}}
% 41
{\PTglyph{5}{t25_l02g23.png}}
% 42
{\PTglyph{5}{t25_l02g24.png}}
% 43
{\PTglyph{5}{t25_l02g25.png}}
% 44
{\PTglyph{5}{t25_l02g26.png}}
% 45
{\PTglyph{5}{t25_l02g27.png}}
% 46
{\PTglyph{5}{t25_l02g28.png}}
% 47
{\PTglyph{5}{t25_l02g29.png}}
% 48
{\PTglyph{5}{t25_l02g30.png}}
% 49
{\PTglyph{5}{t25_l02g31.png}}
% 50
{\PTglyph{5}{t25_l02g32.png}}
% 51
{\PTglyph{5}{t25_l02g33.png}}
% 52
{\PTglyph{5}{t25_l02g34.png}}
% 53
{\PTglyph{5}{t25_l02g35.png}}
% 54
{\PTglyph{5}{t25_l02g36.png}}
% 55
{\PTglyph{5}{t25_l02g37.png}}
% 56
{\PTglyph{5}{t25_l02g38.png}}
% 57
{\PTglyph{5}{t25_l02g39.png}}
% 58
{\PTglyph{5}{t25_l02g40.png}}
% 59
{\PTglyph{5}{t25_l02g41.png}}
% 60
{\PTglyph{5}{t25_l02g42.png}}
% 61
{\PTglyph{5}{t25_l02g43.png}}
% 62
{\PTglyph{5}{t25_l02g44.png}}
% 63
{\PTglyph{5}{t25_l02g45.png}}
% 64
{\PTglyph{5}{t25_l02g46.png}}
% 65
{\PTglyph{5}{t25_l02g47.png}}
% 66
{\PTglyph{5}{t25_l03g01.png}}
% 67
{\PTglyph{5}{t25_l03g02.png}}
% 68
{\PTglyph{5}{t25_l03g03.png}}
% 69
{\PTglyph{5}{t25_l03g04.png}}
% 70
{\PTglyph{5}{t25_l03g05.png}}
% 71
{\PTglyph{5}{t25_l03g06.png}}
% 72
{\PTglyph{5}{t25_l03g07.png}}
% 73
{\PTglyph{5}{t25_l03g08.png}}
% 74
{\PTglyph{5}{t25_l03g09.png}}
% 75
{\PTglyph{5}{t25_l03g10.png}}
% 76
{\PTglyph{5}{t25_l03g11.png}}
% 77
{\PTglyph{5}{t25_l03g12.png}}
% 78
{\PTglyph{5}{t25_l03g13.png}}
% 79
{\PTglyph{5}{t25_l03g14.png}}
% 80
{\PTglyph{5}{t25_l03g15.png}}
% 81
{\PTglyph{5}{t25_l03g16.png}}
% 82
{\PTglyph{5}{t25_l03g17.png}}
% 83
{\PTglyph{5}{t25_l03g18.png}}
% 84
{\PTglyph{5}{t25_l03g19.png}}
% 85
{\PTglyph{5}{t25_l03g20.png}}
% 86
{\PTglyph{5}{t25_l03g21.png}}
% 87
{\PTglyph{5}{t25_l03g22.png}}
% 88
{\PTglyph{5}{t25_l03g23.png}}
% 89
{\PTglyph{5}{t25_l03g24.png}}
% 90
{\PTglyph{5}{t25_l03g25.png}}
% 91
{\PTglyph{5}{t25_l03g26.png}}
% 92
{\PTglyph{5}{t25_l03g27.png}}
% 93
{\PTglyph{5}{t25_l03g28.png}}
% 94
{\PTglyph{5}{t25_l03g29.png}}
//
%%% Local Variables:
%%% mode: latex
%%% TeX-engine: luatex
%%% TeX-master: shared
%%% End:

%//
%\glpismo%
 \glpismo
% 1
{\PTglyphid{Ha-06_0101}}
% 2
{\PTglyphid{Ha-06_0102}}
% 3
{\PTglyphid{Ha-06_0103}}
% 4
{\PTglyphid{Ha-06_0104}}
% 5
{\PTglyphid{Ha-06_0105}}
% 6
{\PTglyphid{Ha-06_0106}}
% 7
{\PTglyphid{Ha-06_0107}}
% 8
{\PTglyphid{Ha-06_0108}}
% 9
{\PTglyphid{Ha-06_0109}}
% 10
{\PTglyphid{Ha-06_0110}}
% 11
{\PTglyphid{Ha-06_0111}}
% 12
{\PTglyphid{Ha-06_0112}}
% 13
{\PTglyphid{Ha-06_0113}}
% 14
{\PTglyphid{Ha-06_0114}}
% 15
{\PTglyphid{Ha-06_0115}}
% 16
{\PTglyphid{Ha-06_0116}}
% 17
{\PTglyphid{Ha-06_0117}}
% 18
{\PTglyphid{Ha-06_0118}}
% 19
{\PTglyphid{Ha-06_0201}}
% 20
{\PTglyphid{Ha-06_0202}}
% 21
{\PTglyphid{Ha-06_0203}}
% 22
{\PTglyphid{Ha-06_0204}}
% 23
{\PTglyphid{Ha-06_0205}}
% 24
{\PTglyphid{Ha-06_0206}}
% 25
{\PTglyphid{Ha-06_0207}}
% 26
{\PTglyphid{Ha-06_0208}}
% 27
{\PTglyphid{Ha-06_0209}}
% 28
{\PTglyphid{Ha-06_0210}}
% 29
{\PTglyphid{Ha-06_0211}}
% 30
{\PTglyphid{Ha-06_0212}}
% 31
{\PTglyphid{Ha-06_0213}}
% 32
{\PTglyphid{Ha-06_0214}}
% 33
{\PTglyphid{Ha-06_0215}}
% 34
{\PTglyphid{Ha-06_0216}}
% 35
{\PTglyphid{Ha-06_0217}}
% 36
{\PTglyphid{Ha-06_0218}}
% 37
{\PTglyphid{Ha-06_0219}}
% 38
{\PTglyphid{Ha-06_0220}}
% 39
{\PTglyphid{Ha-06_0221}}
% 40
{\PTglyphid{Ha-06_0222}}
% 41
{\PTglyphid{Ha-06_0223}}
% 42
{\PTglyphid{Ha-06_0224}}
% 43
{\PTglyphid{Ha-06_0225}}
% 44
{\PTglyphid{Ha-06_0226}}
% 45
{\PTglyphid{Ha-06_0227}}
% 46
{\PTglyphid{Ha-06_0228}}
% 47
{\PTglyphid{Ha-06_0229}}
% 48
{\PTglyphid{Ha-06_0230}}
% 49
{\PTglyphid{Ha-06_0231}}
% 50
{\PTglyphid{Ha-06_0232}}
% 51
{\PTglyphid{Ha-06_0233}}
% 52
{\PTglyphid{Ha-06_0234}}
% 53
{\PTglyphid{Ha-06_0235}}
% 54
{\PTglyphid{Ha-06_0236}}
% 55
{\PTglyphid{Ha-06_0237}}
% 56
{\PTglyphid{Ha-06_0238}}
% 57
{\PTglyphid{Ha-06_0239}}
% 58
{\PTglyphid{Ha-06_0240}}
% 59
{\PTglyphid{Ha-06_0241}}
% 60
{\PTglyphid{Ha-06_0242}}
% 61
{\PTglyphid{Ha-06_0243}}
% 62
{\PTglyphid{Ha-06_0244}}
% 63
{\PTglyphid{Ha-06_0245}}
% 64
{\PTglyphid{Ha-06_0246}}
% 65
{\PTglyphid{Ha-06_0247}}
% 66
{\PTglyphid{Ha-06_0301}}
% 67
{\PTglyphid{Ha-06_0302}}
% 68
{\PTglyphid{Ha-06_0303}}
% 69
{\PTglyphid{Ha-06_0304}}
% 70
{\PTglyphid{Ha-06_0305}}
% 71
{\PTglyphid{Ha-06_0306}}
% 72
{\PTglyphid{Ha-06_0307}}
% 73
{\PTglyphid{Ha-06_0308}}
% 74
{\PTglyphid{Ha-06_0309}}
% 75
{\PTglyphid{Ha-06_0310}}
% 76
{\PTglyphid{Ha-06_0311}}
% 77
{\PTglyphid{Ha-06_0312}}
% 78
{\PTglyphid{Ha-06_0313}}
% 79
{\PTglyphid{Ha-06_0314}}
% 80
{\PTglyphid{Ha-06_0315}}
% 81
{\PTglyphid{Ha-06_0316}}
% 82
{\PTglyphid{Ha-06_0317}}
% 83
{\PTglyphid{Ha-06_0318}}
% 84
{\PTglyphid{Ha-06_0319}}
% 85
{\PTglyphid{Ha-06_0320}}
% 86
{\PTglyphid{Ha-06_0321}}
% 87
{\PTglyphid{Ha-06_0322}}
% 88
{\PTglyphid{Ha-06_0323}}
% 89
{\PTglyphid{Ha-06_0324}}
% 90
{\PTglyphid{Ha-06_0325}}
% 91
{\PTglyphid{Ha-06_0326}}
% 92
{\PTglyphid{Ha-06_0327}}
% 93
{\PTglyphid{Ha-06_0328}}
% 94
{\PTglyphid{Ha-06_0329}}
//
\endgl \xe
%%% Local Variables:
%%% mode: latex
%%% TeX-engine: luatex
%%% TeX-master: shared
%%% End:

% //
%\endgl \xe

  

  \newpage

%%%%%%%%%%%%%%%%%%%%%%%%%%%%%%%%%%%%%%%%%%%%%%%%%%%%%%%%%%%%%%%%%%%%%%%%%%%%%%% 
% Tab. 26,Haller-07_PT04_170.djvu,Haller,07,04,170
%%%%%%%%%%%%%%%%%%%%%%%%%%%%%%%%%%%%%%%%%%%%%%%%%%%%%%%%%%%%%%%%%%%%%%%%%%%%%%

% Note "6. Pismo komentarzowe. Krój M”. Stopień 20 ww. = 82 mm (interliniowane). — Tabl. 169."
% Note1 "Character set table prepared by Maria Błońska"


  \pismoPL{Jan Haller 6. Pismo komentarzowe. Krój M⁹¹. Stopień 20
    ww. = 82 mm (interliniowane). — Tabl. 169. (Występuje u Hochfedera
    jako pismo 4. Tabl. 21).}



  
\pismoEN{Jan Haller 6. Text font. Typeface M⁹¹. Type size 20 ww. =
    82 mm (with extra leading). — Plate 169. (Used by Hochfeder as font 4, plate
    21.)}

\plate{170}{IV}{1962}

The plate prepared by Helena Kapełuś.\\
The font table prepared by Helena Kapełuś and Maria Błońska.

\bigskip

\exampleBib{IV:83}

\bigskip \exampleDesc{PETRUS ROSELLI: Quaestiones in libros priorum
  Analyticorum et Elenchorum Aristotelis. Ed. Ioannes de Stobnica.
  Kraków, Jan Haller, 24. V. 1511. 4⁰.}
  

\medskip
\examplePage{\textit{Karta 89}}

  \bigskip
\exampleLib{Biblioteka Narodowa. Warszawa.}


\bigskip
\exampleRef{\textit{Estreicher XXVI. 363. Wierzbowski 18.}}

% \bigskip
% \exampleDig{\url{} page} 
% podobne ale nie identyczne:
% PETRUS ROSELLI: Quaestiones in libros priorum
%   Analyticorum et Elenchorum Aristotelis. 
%   https://www.dbc.wroc.pl/dlibra/publication/15987/edition/13965

  
  % \medskip
\bigskip

    \examplePL{Pismo 4: naglówek. [Pismo 7: tekst - JSB]}

    \medskip

    \exampleEN{Font 4: the header [Font 7: the text - JSB]}


\bigskip


\fontID{Ha-07}{26}

\fontstat{129}

% \exdisplay \bg \gla
 \exdisplay \bg \gla
% 1
{\PTglyph{5}{t26_l01g01.png}}
% 2
{\PTglyph{5}{t26_l01g02.png}}
//
%%% Local Variables:
%%% mode: latex
%%% TeX-engine: luatex
%%% TeX-master: shared
%%% End:

%//
%\glpismo%
 \glpismo
% 1
{\PTglyphid{Ha-07_0101}}
% 2
{\PTglyphid{Ha-07_0102}}
% 3
{\PTglyphid{Ha-07_0103}}
% 4
{\PTglyphid{Ha-07_0104}}
% 5
{\PTglyphid{Ha-07_0105}}
% 6
{\PTglyphid{Ha-07_0106}}
% 7
{\PTglyphid{Ha-07_0107}}
% 8
{\PTglyphid{Ha-07_0108}}
% 9
{\PTglyphid{Ha-07_0109}}
% 10
{\PTglyphid{Ha-07_0110}}
% 11
{\PTglyphid{Ha-07_0111}}
% 12
{\PTglyphid{Ha-07_0112}}
% 13
{\PTglyphid{Ha-07_0113}}
% 14
{\PTglyphid{Ha-07_0114}}
% 15
{\PTglyphid{Ha-07_0115}}
% 16
{\PTglyphid{Ha-07_0116}}
% 17
{\PTglyphid{Ha-07_0117}}
% 18
{\PTglyphid{Ha-07_0118}}
% 19
{\PTglyphid{Ha-07_0119}}
% 20
{\PTglyphid{Ha-07_0120}}
% 21
{\PTglyphid{Ha-07_0121}}
% 22
{\PTglyphid{Ha-07_0122}}
% 23
{\PTglyphid{Ha-07_0123}}
% 24
{\PTglyphid{Ha-07_0124}}
% 25
{\PTglyphid{Ha-07_0125}}
% 26
{\PTglyphid{Ha-07_0126}}
% 27
{\PTglyphid{Ha-07_0201}}
% 28
{\PTglyphid{Ha-07_0202}}
% 29
{\PTglyphid{Ha-07_0203}}
% 30
{\PTglyphid{Ha-07_0204}}
% 31
{\PTglyphid{Ha-07_0205}}
% 32
{\PTglyphid{Ha-07_0206}}
% 33
{\PTglyphid{Ha-07_0207}}
% 34
{\PTglyphid{Ha-07_0208}}
% 35
{\PTglyphid{Ha-07_0209}}
% 36
{\PTglyphid{Ha-07_0210}}
% 37
{\PTglyphid{Ha-07_0211}}
% 38
{\PTglyphid{Ha-07_0212}}
% 39
{\PTglyphid{Ha-07_0213}}
% 40
{\PTglyphid{Ha-07_0214}}
% 41
{\PTglyphid{Ha-07_0215}}
% 42
{\PTglyphid{Ha-07_0216}}
% 43
{\PTglyphid{Ha-07_0217}}
% 44
{\PTglyphid{Ha-07_0218}}
% 45
{\PTglyphid{Ha-07_0219}}
% 46
{\PTglyphid{Ha-07_0220}}
% 47
{\PTglyphid{Ha-07_0221}}
% 48
{\PTglyphid{Ha-07_0222}}
% 49
{\PTglyphid{Ha-07_0223}}
% 50
{\PTglyphid{Ha-07_0224}}
% 51
{\PTglyphid{Ha-07_0225}}
% 52
{\PTglyphid{Ha-07_0226}}
% 53
{\PTglyphid{Ha-07_0227}}
% 54
{\PTglyphid{Ha-07_0228}}
% 55
{\PTglyphid{Ha-07_0229}}
% 56
{\PTglyphid{Ha-07_0230}}
% 57
{\PTglyphid{Ha-07_0231}}
% 58
{\PTglyphid{Ha-07_0232}}
% 59
{\PTglyphid{Ha-07_0233}}
% 60
{\PTglyphid{Ha-07_0234}}
% 61
{\PTglyphid{Ha-07_0235}}
% 62
{\PTglyphid{Ha-07_0236}}
% 63
{\PTglyphid{Ha-07_0237}}
% 64
{\PTglyphid{Ha-07_0238}}
% 65
{\PTglyphid{Ha-07_0239}}
% 66
{\PTglyphid{Ha-07_0240}}
% 67
{\PTglyphid{Ha-07_0241}}
% 68
{\PTglyphid{Ha-07_0242}}
% 69
{\PTglyphid{Ha-07_0243}}
% 70
{\PTglyphid{Ha-07_0244}}
% 71
{\PTglyphid{Ha-07_0245}}
% 72
{\PTglyphid{Ha-07_0246}}
% 73
{\PTglyphid{Ha-07_0247}}
% 74
{\PTglyphid{Ha-07_0248}}
% 75
{\PTglyphid{Ha-07_0249}}
% 76
{\PTglyphid{Ha-07_0301}}
% 77
{\PTglyphid{Ha-07_0302}}
% 78
{\PTglyphid{Ha-07_0303}}
% 79
{\PTglyphid{Ha-07_0304}}
% 80
{\PTglyphid{Ha-07_0305}}
% 81
{\PTglyphid{Ha-07_0306}}
% 82
{\PTglyphid{Ha-07_0307}}
% 83
{\PTglyphid{Ha-07_0308}}
% 84
{\PTglyphid{Ha-07_0309}}
% 85
{\PTglyphid{Ha-07_0310}}
% 86
{\PTglyphid{Ha-07_0311}}
% 87
{\PTglyphid{Ha-07_0312}}
% 88
{\PTglyphid{Ha-07_0313}}
% 89
{\PTglyphid{Ha-07_0314}}
% 90
{\PTglyphid{Ha-07_0315}}
% 91
{\PTglyphid{Ha-07_0316}}
% 92
{\PTglyphid{Ha-07_0317}}
% 93
{\PTglyphid{Ha-07_0318}}
% 94
{\PTglyphid{Ha-07_0319}}
% 95
{\PTglyphid{Ha-07_0320}}
% 96
{\PTglyphid{Ha-07_0321}}
% 97
{\PTglyphid{Ha-07_0322}}
% 98
{\PTglyphid{Ha-07_0323}}
% 99
{\PTglyphid{Ha-07_0324}}
% 100
{\PTglyphid{Ha-07_0325}}
% 101
{\PTglyphid{Ha-07_0326}}
% 102
{\PTglyphid{Ha-07_0327}}
% 103
{\PTglyphid{Ha-07_0328}}
% 104
{\PTglyphid{Ha-07_0329}}
% 105
{\PTglyphid{Ha-07_0330}}
% 106
{\PTglyphid{Ha-07_0331}}
% 107
{\PTglyphid{Ha-07_0332}}
% 108
{\PTglyphid{Ha-07_0333}}
% 109
{\PTglyphid{Ha-07_0334}}
% 110
{\PTglyphid{Ha-07_0335}}
% 111
{\PTglyphid{Ha-07_0336}}
% 112
{\PTglyphid{Ha-07_0337}}
% 113
{\PTglyphid{Ha-07_0338}}
% 114
{\PTglyphid{Ha-07_0339}}
% 115
{\PTglyphid{Ha-07_0340}}
% 116
{\PTglyphid{Ha-07_0341}}
% 117
{\PTglyphid{Ha-07_0342}}
% 118
{\PTglyphid{Ha-07_0343}}
% 119
{\PTglyphid{Ha-07_0401}}
% 120
{\PTglyphid{Ha-07_0501}}
% 121
{\PTglyphid{Ha-07_0502}}
% 122
{\PTglyphid{Ha-07_0503}}
% 123
{\PTglyphid{Ha-07_0504}}
% 124
{\PTglyphid{Ha-07_0505}}
% 125
{\PTglyphid{Ha-07_0506}}
% 126
{\PTglyphid{Ha-07_0507}}
% 127
{\PTglyphid{Ha-07_0508}}
% 128
{\PTglyphid{Ha-07_0509}}
% 129
{\PTglyphid{Ha-07_0510}}
//
\endgl \xe
%%% Local Variables:
%%% mode: latex
%%% TeX-engine: luatex
%%% TeX-master: shared
%%% End:

% //
%\endgl \xe

 

 \newpage

%%%%%%%%%%%%%%%%%%%%%%%%%%%%%%%%%%%%%%%%%%%%%%%%%%%%%%%%%%%%%%%%%%%%%%%%%%%%%%% 
% Tab. 27,Haller-08_PT04_171.djvu,Haller,08,04,171
%%%%%%%%%%%%%%%%%%%%%%%%%%%%%%%%%%%%%%%%%%%%%%%%%%%%%%%%%%%%%%%%%%%%%%%%%%%%%%

% Note "8. Pismo tekstowe. Krój M⁹¹. Stopień 20 ww. = 72 mm. — Tabl. 171."
% Note1 "Character set table prepared by Maria Błońska"

  \pismoPL{Jan Haller 8. Pismo tekstowe. Krój M⁹¹. Stopień 20 ww. = 72 mm. — Tabl. 171.}
  
\pismoEN{Jan Haller 8. Text font. Typeface M⁹¹. Type size 20 ww. =
    72 mm . — Plate 171.}

\plate{171}{IV}{1962}

The plate prepared by Helena Kapełuś.\\
The font table prepared byHelena Kapełuś and Maria Błońska.

\bigskip

\exampleBib{IV:201}

\bigskip


\exampleDesc{MICHAEL VRATISLAVIENSIS: Epithoma conclusionum
  theologicalium pro introductione in IV libros Petri
  Lombardi. Kraków, Jan Haller, 1521. 4⁰.}
  

\medskip
\examplePage{\textit{Karta a²a}}

  \bigskip
\exampleLib{Biblioteka Narodowa. Warszawa.}


\bigskip
\exampleRef{\textit{Estreicher XXXIII. 357. Wierzbowski 57.}}

\bigskip
\exampleDig{\url{https://www.dbc.wroc.pl/dlibra/publication/8876/} page 7} 

% https://polona.pl/preview/1f9dd490-e381-429e-9784-174169b32fb7
% Epithoma conclusionum theologicalium pro introductione in quatuor libros Sententiarum magistri Petri Lombardi [...] in [...] studio Cracouien[si] elucubratum

% https://www.dbc.wroc.pl/dlibra/publication/8876/edition/8006
% Epithoma conclusionum theologicalium pro introductione in quatuor libros sententiarum magistri Petri Lombardi [...]
% page 7


% https://polona2.pl/item/epithoma-conclusionum-theologicalium-pro-introductione-in-quatuor-libros-sententiarum,OTIxNzA2MTg/2/#info:metadata

% https://platforma.bk.pan.pl/en/search_results/1283666


\bigskip

    \examplePL{Pismo 12: wiersz 1, 4. [Pismo 8: tekst - JSB]}

    \medskip

    \exampleEN{Font 12: the lines 1, 4. [Font 8: the text - JSB]}


\bigskip


\fontID{Ha-08}{27}

\fontstat{121}

% \exdisplay \bg \gla
 \exdisplay \bg \gla
% 1
{\PTglyph{5}{t27_l01g01.png}}
% 2
{\PTglyph{5}{t27_l01g02.png}}
% 3
{\PTglyph{5}{t27_l01g03.png}}
% 4
{\PTglyph{5}{t27_l01g04.png}}
% 5
{\PTglyph{5}{t27_l01g05.png}}
% 6
{\PTglyph{5}{t27_l01g06.png}}
% 7
{\PTglyph{5}{t27_l01g07.png}}
% 8
{\PTglyph{5}{t27_l01g08.png}}
% 9
{\PTglyph{5}{t27_l01g09.png}}
% 10
{\PTglyph{5}{t27_l01g10.png}}
% 11
{\PTglyph{5}{t27_l01g11.png}}
% 12
{\PTglyph{5}{t27_l01g12.png}}
% 13
{\PTglyph{5}{t27_l01g13.png}}
% 14
{\PTglyph{5}{t27_l01g14.png}}
% 15
{\PTglyph{5}{t27_l01g15.png}}
% 16
{\PTglyph{5}{t27_l01g16.png}}
% 17
{\PTglyph{5}{t27_l01g17.png}}
% 18
{\PTglyph{5}{t27_l01g18.png}}
% 19
{\PTglyph{5}{t27_l01g19.png}}
% 20
{\PTglyph{5}{t27_l01g20.png}}
% 21
{\PTglyph{5}{t27_l01g21.png}}
% 22
{\PTglyph{5}{t27_l01g22.png}}
% 23
{\PTglyph{5}{t27_l01g23.png}}
% 24
{\PTglyph{5}{t27_l01g24.png}}
% 25
{\PTglyph{5}{t27_l01g25.png}}
% 26
{\PTglyph{5}{t27_l01g26.png}}
% 27
{\PTglyph{5}{t27_l01g27.png}}
% 28
{\PTglyph{5}{t27_l01g28.png}}
% 29
{\PTglyph{5}{t27_l01g29.png}}
% 30
{\PTglyph{5}{t27_l01g30.png}}
% 31
{\PTglyph{5}{t27_l01g31.png}}
% 32
{\PTglyph{5}{t27_l01g32.png}}
% 33
{\PTglyph{5}{t27_l02g01.png}}
% 34
{\PTglyph{5}{t27_l02g02.png}}
% 35
{\PTglyph{5}{t27_l02g03.png}}
% 36
{\PTglyph{5}{t27_l02g04.png}}
% 37
{\PTglyph{5}{t27_l02g05.png}}
% 38
{\PTglyph{5}{t27_l02g06.png}}
% 39
{\PTglyph{5}{t27_l02g07.png}}
% 40
{\PTglyph{5}{t27_l02g08.png}}
% 41
{\PTglyph{5}{t27_l02g09.png}}
% 42
{\PTglyph{5}{t27_l02g10.png}}
% 43
{\PTglyph{5}{t27_l02g11.png}}
% 44
{\PTglyph{5}{t27_l02g12.png}}
% 45
{\PTglyph{5}{t27_l02g13.png}}
% 46
{\PTglyph{5}{t27_l02g14.png}}
% 47
{\PTglyph{5}{t27_l02g15.png}}
% 48
{\PTglyph{5}{t27_l02g16.png}}
% 49
{\PTglyph{5}{t27_l02g17.png}}
% 50
{\PTglyph{5}{t27_l02g18.png}}
% 51
{\PTglyph{5}{t27_l02g19.png}}
% 52
{\PTglyph{5}{t27_l02g20.png}}
% 53
{\PTglyph{5}{t27_l02g21.png}}
% 54
{\PTglyph{5}{t27_l02g22.png}}
% 55
{\PTglyph{5}{t27_l02g23.png}}
% 56
{\PTglyph{5}{t27_l02g24.png}}
% 57
{\PTglyph{5}{t27_l02g25.png}}
% 58
{\PTglyph{5}{t27_l02g26.png}}
% 59
{\PTglyph{5}{t27_l02g27.png}}
% 60
{\PTglyph{5}{t27_l02g28.png}}
% 61
{\PTglyph{5}{t27_l02g29.png}}
% 62
{\PTglyph{5}{t27_l02g30.png}}
% 63
{\PTglyph{5}{t27_l02g31.png}}
% 64
{\PTglyph{5}{t27_l02g32.png}}
% 65
{\PTglyph{5}{t27_l02g33.png}}
% 66
{\PTglyph{5}{t27_l02g34.png}}
% 67
{\PTglyph{5}{t27_l02g35.png}}
% 68
{\PTglyph{5}{t27_l02g36.png}}
% 69
{\PTglyph{5}{t27_l02g37.png}}
% 70
{\PTglyph{5}{t27_l02g38.png}}
% 71
{\PTglyph{5}{t27_l02g39.png}}
% 72
{\PTglyph{5}{t27_l02g40.png}}
% 73
{\PTglyph{5}{t27_l02g41.png}}
% 74
{\PTglyph{5}{t27_l02g42.png}}
% 75
{\PTglyph{5}{t27_l02g43.png}}
% 76
{\PTglyph{5}{t27_l02g44.png}}
% 77
{\PTglyph{5}{t27_l03g01.png}}
% 78
{\PTglyph{5}{t27_l03g02.png}}
% 79
{\PTglyph{5}{t27_l03g03.png}}
% 80
{\PTglyph{5}{t27_l03g04.png}}
% 81
{\PTglyph{5}{t27_l03g05.png}}
% 82
{\PTglyph{5}{t27_l03g06.png}}
% 83
{\PTglyph{5}{t27_l03g07.png}}
% 84
{\PTglyph{5}{t27_l03g08.png}}
% 85
{\PTglyph{5}{t27_l03g09.png}}
% 86
{\PTglyph{5}{t27_l03g10.png}}
% 87
{\PTglyph{5}{t27_l03g11.png}}
% 88
{\PTglyph{5}{t27_l03g12.png}}
% 89
{\PTglyph{5}{t27_l03g13.png}}
% 90
{\PTglyph{5}{t27_l03g14.png}}
% 91
{\PTglyph{5}{t27_l03g15.png}}
% 92
{\PTglyph{5}{t27_l03g16.png}}
% 93
{\PTglyph{5}{t27_l03g17.png}}
% 94
{\PTglyph{5}{t27_l03g18.png}}
% 95
{\PTglyph{5}{t27_l03g19.png}}
% 96
{\PTglyph{5}{t27_l03g20.png}}
% 97
{\PTglyph{5}{t27_l03g21.png}}
% 98
{\PTglyph{5}{t27_l03g22.png}}
% 99
{\PTglyph{5}{t27_l03g23.png}}
% 100
{\PTglyph{5}{t27_l03g24.png}}
% 101
{\PTglyph{5}{t27_l03g25.png}}
% 102
{\PTglyph{5}{t27_l03g26.png}}
% 103
{\PTglyph{5}{t27_l03g27.png}}
% 104
{\PTglyph{5}{t27_l03g28.png}}
% 105
{\PTglyph{5}{t27_l03g29.png}}
% 106
{\PTglyph{5}{t27_l03g30.png}}
% 107
{\PTglyph{5}{t27_l03g31.png}}
% 108
{\PTglyph{5}{t27_l03g32.png}}
% 109
{\PTglyph{5}{t27_l03g33.png}}
% 110
{\PTglyph{5}{t27_l03g34.png}}
% 111
{\PTglyph{5}{t27_l03g35.png}}
% 112
{\PTglyph{5}{t27_l03g36.png}}
% 113
{\PTglyph{5}{t27_l03g37.png}}
% 114
{\PTglyph{5}{t27_l03g38.png}}
% 115
{\PTglyph{5}{t27_l03g39.png}}
% 116
{\PTglyph{5}{t27_l03g40.png}}
% 117
{\PTglyph{5}{t27_l03g41.png}}
% 118
{\PTglyph{5}{t27_l03g42.png}}
% 119
{\PTglyph{5}{t27_l03g43.png}}
% 120
{\PTglyph{5}{t27_l04g01.png}}
% 121
{\PTglyph{5}{t27_l04g02.png}}
//
%%% Local Variables:
%%% mode: latex
%%% TeX-engine: luatex
%%% TeX-master: shared
%%% End:

%//
%\glpismo%
 \glpismo
% 1
{\PTglyphid{Ha-08_0101}}
% 2
{\PTglyphid{Ha-08_0102}}
% 3
{\PTglyphid{Ha-08_0103}}
% 4
{\PTglyphid{Ha-08_0104}}
% 5
{\PTglyphid{Ha-08_0105}}
% 6
{\PTglyphid{Ha-08_0106}}
% 7
{\PTglyphid{Ha-08_0107}}
% 8
{\PTglyphid{Ha-08_0108}}
% 9
{\PTglyphid{Ha-08_0109}}
% 10
{\PTglyphid{Ha-08_0110}}
% 11
{\PTglyphid{Ha-08_0111}}
% 12
{\PTglyphid{Ha-08_0112}}
% 13
{\PTglyphid{Ha-08_0113}}
% 14
{\PTglyphid{Ha-08_0114}}
% 15
{\PTglyphid{Ha-08_0115}}
% 16
{\PTglyphid{Ha-08_0116}}
% 17
{\PTglyphid{Ha-08_0117}}
% 18
{\PTglyphid{Ha-08_0118}}
% 19
{\PTglyphid{Ha-08_0119}}
% 20
{\PTglyphid{Ha-08_0120}}
% 21
{\PTglyphid{Ha-08_0121}}
% 22
{\PTglyphid{Ha-08_0122}}
% 23
{\PTglyphid{Ha-08_0123}}
% 24
{\PTglyphid{Ha-08_0124}}
% 25
{\PTglyphid{Ha-08_0125}}
% 26
{\PTglyphid{Ha-08_0126}}
% 27
{\PTglyphid{Ha-08_0127}}
% 28
{\PTglyphid{Ha-08_0128}}
% 29
{\PTglyphid{Ha-08_0129}}
% 30
{\PTglyphid{Ha-08_0130}}
% 31
{\PTglyphid{Ha-08_0131}}
% 32
{\PTglyphid{Ha-08_0132}}
% 33
{\PTglyphid{Ha-08_0201}}
% 34
{\PTglyphid{Ha-08_0202}}
% 35
{\PTglyphid{Ha-08_0203}}
% 36
{\PTglyphid{Ha-08_0204}}
% 37
{\PTglyphid{Ha-08_0205}}
% 38
{\PTglyphid{Ha-08_0206}}
% 39
{\PTglyphid{Ha-08_0207}}
% 40
{\PTglyphid{Ha-08_0208}}
% 41
{\PTglyphid{Ha-08_0209}}
% 42
{\PTglyphid{Ha-08_0210}}
% 43
{\PTglyphid{Ha-08_0211}}
% 44
{\PTglyphid{Ha-08_0212}}
% 45
{\PTglyphid{Ha-08_0213}}
% 46
{\PTglyphid{Ha-08_0214}}
% 47
{\PTglyphid{Ha-08_0215}}
% 48
{\PTglyphid{Ha-08_0216}}
% 49
{\PTglyphid{Ha-08_0217}}
% 50
{\PTglyphid{Ha-08_0218}}
% 51
{\PTglyphid{Ha-08_0219}}
% 52
{\PTglyphid{Ha-08_0220}}
% 53
{\PTglyphid{Ha-08_0221}}
% 54
{\PTglyphid{Ha-08_0222}}
% 55
{\PTglyphid{Ha-08_0223}}
% 56
{\PTglyphid{Ha-08_0224}}
% 57
{\PTglyphid{Ha-08_0225}}
% 58
{\PTglyphid{Ha-08_0226}}
% 59
{\PTglyphid{Ha-08_0227}}
% 60
{\PTglyphid{Ha-08_0228}}
% 61
{\PTglyphid{Ha-08_0229}}
% 62
{\PTglyphid{Ha-08_0230}}
% 63
{\PTglyphid{Ha-08_0231}}
% 64
{\PTglyphid{Ha-08_0232}}
% 65
{\PTglyphid{Ha-08_0233}}
% 66
{\PTglyphid{Ha-08_0234}}
% 67
{\PTglyphid{Ha-08_0235}}
% 68
{\PTglyphid{Ha-08_0236}}
% 69
{\PTglyphid{Ha-08_0237}}
% 70
{\PTglyphid{Ha-08_0238}}
% 71
{\PTglyphid{Ha-08_0239}}
% 72
{\PTglyphid{Ha-08_0240}}
% 73
{\PTglyphid{Ha-08_0241}}
% 74
{\PTglyphid{Ha-08_0242}}
% 75
{\PTglyphid{Ha-08_0243}}
% 76
{\PTglyphid{Ha-08_0244}}
% 77
{\PTglyphid{Ha-08_0301}}
% 78
{\PTglyphid{Ha-08_0302}}
% 79
{\PTglyphid{Ha-08_0303}}
% 80
{\PTglyphid{Ha-08_0304}}
% 81
{\PTglyphid{Ha-08_0305}}
% 82
{\PTglyphid{Ha-08_0306}}
% 83
{\PTglyphid{Ha-08_0307}}
% 84
{\PTglyphid{Ha-08_0308}}
% 85
{\PTglyphid{Ha-08_0309}}
% 86
{\PTglyphid{Ha-08_0310}}
% 87
{\PTglyphid{Ha-08_0311}}
% 88
{\PTglyphid{Ha-08_0312}}
% 89
{\PTglyphid{Ha-08_0313}}
% 90
{\PTglyphid{Ha-08_0314}}
% 91
{\PTglyphid{Ha-08_0315}}
% 92
{\PTglyphid{Ha-08_0316}}
% 93
{\PTglyphid{Ha-08_0317}}
% 94
{\PTglyphid{Ha-08_0318}}
% 95
{\PTglyphid{Ha-08_0319}}
% 96
{\PTglyphid{Ha-08_0320}}
% 97
{\PTglyphid{Ha-08_0321}}
% 98
{\PTglyphid{Ha-08_0322}}
% 99
{\PTglyphid{Ha-08_0323}}
% 100
{\PTglyphid{Ha-08_0324}}
% 101
{\PTglyphid{Ha-08_0325}}
% 102
{\PTglyphid{Ha-08_0326}}
% 103
{\PTglyphid{Ha-08_0327}}
% 104
{\PTglyphid{Ha-08_0328}}
% 105
{\PTglyphid{Ha-08_0329}}
% 106
{\PTglyphid{Ha-08_0330}}
% 107
{\PTglyphid{Ha-08_0331}}
% 108
{\PTglyphid{Ha-08_0332}}
% 109
{\PTglyphid{Ha-08_0333}}
% 110
{\PTglyphid{Ha-08_0334}}
% 111
{\PTglyphid{Ha-08_0335}}
% 112
{\PTglyphid{Ha-08_0336}}
% 113
{\PTglyphid{Ha-08_0337}}
% 114
{\PTglyphid{Ha-08_0338}}
% 115
{\PTglyphid{Ha-08_0339}}
% 116
{\PTglyphid{Ha-08_0340}}
% 117
{\PTglyphid{Ha-08_0341}}
% 118
{\PTglyphid{Ha-08_0342}}
% 119
{\PTglyphid{Ha-08_0343}}
% 120
{\PTglyphid{Ha-08_0401}}
% 121
{\PTglyphid{Ha-08_0402}}
//
\endgl \xe
%%% Local Variables:
%%% mode: latex
%%% TeX-engine: luatex
%%% TeX-master: shared
%%% End:

% //
%\endgl \xe

 \newpage

%%%%%%%%%%%%%%%%%%%%%%%%%%%%%%%%%%%%%%%%%%%%%%%%%%%%%%%%%%%%%%%%%%%%%%%%%%%%%%% 
% Tab. 28,Haller-09_PT04_172.djvu,Haller,09,04,172
%%%%%%%%%%%%%%%%%%%%%%%%%%%%%%%%%%%%%%%%%%%%%%%%%%%%%%%%%%%%%%%%%%%%%%%%%%%%%%

% Note "9. Pismo tekstowe. Krój M⁴⁹. Stopień 20 ww. = 72 mm. — Tabl. 172."
% Note1 "Character set table prepared by Maria Błońska"

 
  \pismoPL{Jan Haller 9. Pismo tekstowe. Krój M⁴⁹. Stopień 20 ww. = 72 mm. — Tabl. 172 [pierwszy zestaw].}
  
\pismoEN{Jan Haller 9. Text font. Typeface M⁴¹. Type size 20 ww. =
    72 mm . — Plate 172 [first set].}

\plate{172[1]}{IV}{1962}

The plate prepared by Helena Kapełuś.\\
The font table prepared by Helena Kapełuś and Maria Błońska.

\bigskip

\exampleBib{IV:135}

\bigskip


\exampleDesc{MICHAEL VRATISLAVIENSIS: Expositio hymnorumque interpretatio. Kraków, Jan Haller, 1516. 4⁰.}
  

\medskip
\examplePage{\textit{Karta 3 b}}

  \bigskip
\exampleLib{Biblioteka Zakł. Nar. im. Ossolińskich. Wrocław.}


\bigskip
\exampleRef{\textit{Estreicher XXXIII. 358.}}

% \bigskip
% \exampleDig{\url page 7} 

\bigskip

\examplePL{Pismo 3: naglówek. — Pismo 9: tekst i pierwszy zestaw. —
  Rubryka \beta{} z pismem 9. — Cyfry 7: z pismem 9.  z pismem 13. [\ldots]}

    \medskip

    \exampleEN{Font 3: the header. — Font 9: the text and the first character set. — Rubric \beta{} with digits from font 9}


\bigskip


\fontID{Ha-09}{28}

\fontstat{119}

% \exdisplay \bg \gla
 \exdisplay \bg \gla
% 1
{\PTglyph{5}{t28_l01g01.png}}
% 2
{\PTglyph{5}{t28_l01g02.png}}
% 3
{\PTglyph{5}{t28_l01g03.png}}
% 4
{\PTglyph{5}{t28_l01g04.png}}
% 5
{\PTglyph{5}{t28_l01g05.png}}
% 6
{\PTglyph{5}{t28_l01g06.png}}
% 7
{\PTglyph{5}{t28_l01g07.png}}
% 8
{\PTglyph{5}{t28_l01g08.png}}
% 9
{\PTglyph{5}{t28_l01g09.png}}
% 10
{\PTglyph{5}{t28_l01g10.png}}
% 11
{\PTglyph{5}{t28_l01g11.png}}
% 12
{\PTglyph{5}{t28_l01g12.png}}
% 13
{\PTglyph{5}{t28_l01g13.png}}
% 14
{\PTglyph{5}{t28_l01g14.png}}
% 15
{\PTglyph{5}{t28_l01g15.png}}
% 16
{\PTglyph{5}{t28_l01g16.png}}
% 17
{\PTglyph{5}{t28_l01g17.png}}
% 18
{\PTglyph{5}{t28_l01g18.png}}
% 19
{\PTglyph{5}{t28_l01g19.png}}
% 20
{\PTglyph{5}{t28_l01g20.png}}
% 21
{\PTglyph{5}{t28_l01g21.png}}
% 22
{\PTglyph{5}{t28_l01g22.png}}
% 23
{\PTglyph{5}{t28_l01g23.png}}
% 24
{\PTglyph{5}{t28_l02g01.png}}
% 25
{\PTglyph{5}{t28_l02g02.png}}
% 26
{\PTglyph{5}{t28_l02g03.png}}
% 27
{\PTglyph{5}{t28_l02g04.png}}
% 28
{\PTglyph{5}{t28_l02g05.png}}
% 29
{\PTglyph{5}{t28_l02g06.png}}
% 30
{\PTglyph{5}{t28_l02g07.png}}
% 31
{\PTglyph{5}{t28_l02g08.png}}
% 32
{\PTglyph{5}{t28_l02g09.png}}
% 33
{\PTglyph{5}{t28_l02g10.png}}
% 34
{\PTglyph{5}{t28_l02g11.png}}
% 35
{\PTglyph{5}{t28_l02g12.png}}
% 36
{\PTglyph{5}{t28_l02g13.png}}
% 37
{\PTglyph{5}{t28_l02g14.png}}
% 38
{\PTglyph{5}{t28_l02g15.png}}
% 39
{\PTglyph{5}{t28_l02g16.png}}
% 40
{\PTglyph{5}{t28_l02g17.png}}
% 41
{\PTglyph{5}{t28_l02g18.png}}
% 42
{\PTglyph{5}{t28_l02g19.png}}
% 43
{\PTglyph{5}{t28_l02g20.png}}
% 44
{\PTglyph{5}{t28_l02g21.png}}
% 45
{\PTglyph{5}{t28_l02g22.png}}
% 46
{\PTglyph{5}{t28_l02g23.png}}
% 47
{\PTglyph{5}{t28_l02g24.png}}
% 48
{\PTglyph{5}{t28_l02g25.png}}
% 49
{\PTglyph{5}{t28_l02g26.png}}
% 50
{\PTglyph{5}{t28_l02g27.png}}
% 51
{\PTglyph{5}{t28_l02g28.png}}
% 52
{\PTglyph{5}{t28_l02g29.png}}
% 53
{\PTglyph{5}{t28_l02g30.png}}
% 54
{\PTglyph{5}{t28_l02g31.png}}
% 55
{\PTglyph{5}{t28_l02g32.png}}
% 56
{\PTglyph{5}{t28_l02g33.png}}
% 57
{\PTglyph{5}{t28_l02g34.png}}
% 58
{\PTglyph{5}{t28_l02g35.png}}
% 59
{\PTglyph{5}{t28_l02g36.png}}
% 60
{\PTglyph{5}{t28_l02g37.png}}
% 61
{\PTglyph{5}{t28_l02g38.png}}
% 62
{\PTglyph{5}{t28_l02g39.png}}
% 63
{\PTglyph{5}{t28_l02g40.png}}
% 64
{\PTglyph{5}{t28_l03g01.png}}
% 65
{\PTglyph{5}{t28_l03g02.png}}
% 66
{\PTglyph{5}{t28_l03g03.png}}
% 67
{\PTglyph{5}{t28_l03g04.png}}
% 68
{\PTglyph{5}{t28_l03g05.png}}
% 69
{\PTglyph{5}{t28_l03g06.png}}
% 70
{\PTglyph{5}{t28_l03g07.png}}
% 71
{\PTglyph{5}{t28_l03g08.png}}
% 72
{\PTglyph{5}{t28_l03g09.png}}
% 73
{\PTglyph{5}{t28_l03g10.png}}
% 74
{\PTglyph{5}{t28_l03g11.png}}
% 75
{\PTglyph{5}{t28_l03g12.png}}
% 76
{\PTglyph{5}{t28_l03g13.png}}
% 77
{\PTglyph{5}{t28_l03g14.png}}
% 78
{\PTglyph{5}{t28_l03g15.png}}
% 79
{\PTglyph{5}{t28_l03g16.png}}
% 80
{\PTglyph{5}{t28_l03g17.png}}
% 81
{\PTglyph{5}{t28_l03g18.png}}
% 82
{\PTglyph{5}{t28_l03g19.png}}
% 83
{\PTglyph{5}{t28_l03g20.png}}
% 84
{\PTglyph{5}{t28_l03g21.png}}
% 85
{\PTglyph{5}{t28_l03g22.png}}
% 86
{\PTglyph{5}{t28_l03g23.png}}
% 87
{\PTglyph{5}{t28_l03g24.png}}
% 88
{\PTglyph{5}{t28_l03g25.png}}
% 89
{\PTglyph{5}{t28_l03g26.png}}
% 90
{\PTglyph{5}{t28_l03g27.png}}
% 91
{\PTglyph{5}{t28_l03g28.png}}
% 92
{\PTglyph{5}{t28_l03g29.png}}
% 93
{\PTglyph{5}{t28_l03g30.png}}
% 94
{\PTglyph{5}{t28_l03g31.png}}
% 95
{\PTglyph{5}{t28_l03g32.png}}
% 96
{\PTglyph{5}{t28_l03g33.png}}
% 97
{\PTglyph{5}{t28_l03g34.png}}
% 98
{\PTglyph{5}{t28_l03g35.png}}
% 99
{\PTglyph{5}{t28_l03g36.png}}
% 100
{\PTglyph{5}{t28_l04g01.png}}
% 101
{\PTglyph{5}{t28_l04g02.png}}
% 102
{\PTglyph{5}{t28_l04g03.png}}
% 103
{\PTglyph{5}{t28_l04g04.png}}
% 104
{\PTglyph{5}{t28_l04g05.png}}
% 105
{\PTglyph{5}{t28_l04g06.png}}
% 106
{\PTglyph{5}{t28_l04g07.png}}
% 107
{\PTglyph{5}{t28_l04g08.png}}
% 108
{\PTglyph{5}{t28_l04g09.png}}
% 109
{\PTglyph{5}{t28_l04g10.png}}
% 110
{\PTglyph{5}{t28_l04g11.png}}
% 111
{\PTglyph{5}{t28_l04g12.png}}
% 112
{\PTglyph{5}{t28_l04g13.png}}
% 113
{\PTglyph{5}{t28_l04g14.png}}
% 114
{\PTglyph{5}{t28_l04g15.png}}
% 115
{\PTglyph{5}{t28_l04g16.png}}
% 116
{\PTglyph{5}{t28_l04g17.png}}
% 117
{\PTglyph{5}{t28_l04g18.png}}
% 118
{\PTglyph{5}{t28_l04g19.png}}
% 119
{\PTglyph{5}{t28_l04g20.png}}
//
%%% Local Variables:
%%% mode: latex
%%% TeX-engine: luatex
%%% TeX-master: shared
%%% End:

%//
%\glpismo%
 \glpismo
% 1
{\PTglyphid{Ha-09_0101}}
% 2
{\PTglyphid{Ha-09_0102}}
% 3
{\PTglyphid{Ha-09_0103}}
% 4
{\PTglyphid{Ha-09_0104}}
% 5
{\PTglyphid{Ha-09_0105}}
% 6
{\PTglyphid{Ha-09_0106}}
% 7
{\PTglyphid{Ha-09_0107}}
% 8
{\PTglyphid{Ha-09_0108}}
% 9
{\PTglyphid{Ha-09_0109}}
% 10
{\PTglyphid{Ha-09_0110}}
% 11
{\PTglyphid{Ha-09_0111}}
% 12
{\PTglyphid{Ha-09_0112}}
% 13
{\PTglyphid{Ha-09_0113}}
% 14
{\PTglyphid{Ha-09_0114}}
% 15
{\PTglyphid{Ha-09_0115}}
% 16
{\PTglyphid{Ha-09_0116}}
% 17
{\PTglyphid{Ha-09_0117}}
% 18
{\PTglyphid{Ha-09_0118}}
% 19
{\PTglyphid{Ha-09_0119}}
% 20
{\PTglyphid{Ha-09_0120}}
% 21
{\PTglyphid{Ha-09_0121}}
% 22
{\PTglyphid{Ha-09_0122}}
% 23
{\PTglyphid{Ha-09_0123}}
% 24
{\PTglyphid{Ha-09_0201}}
% 25
{\PTglyphid{Ha-09_0202}}
% 26
{\PTglyphid{Ha-09_0203}}
% 27
{\PTglyphid{Ha-09_0204}}
% 28
{\PTglyphid{Ha-09_0205}}
% 29
{\PTglyphid{Ha-09_0206}}
% 30
{\PTglyphid{Ha-09_0207}}
% 31
{\PTglyphid{Ha-09_0208}}
% 32
{\PTglyphid{Ha-09_0209}}
% 33
{\PTglyphid{Ha-09_0210}}
% 34
{\PTglyphid{Ha-09_0211}}
% 35
{\PTglyphid{Ha-09_0212}}
% 36
{\PTglyphid{Ha-09_0213}}
% 37
{\PTglyphid{Ha-09_0214}}
% 38
{\PTglyphid{Ha-09_0215}}
% 39
{\PTglyphid{Ha-09_0216}}
% 40
{\PTglyphid{Ha-09_0217}}
% 41
{\PTglyphid{Ha-09_0218}}
% 42
{\PTglyphid{Ha-09_0219}}
% 43
{\PTglyphid{Ha-09_0220}}
% 44
{\PTglyphid{Ha-09_0221}}
% 45
{\PTglyphid{Ha-09_0222}}
% 46
{\PTglyphid{Ha-09_0223}}
% 47
{\PTglyphid{Ha-09_0224}}
% 48
{\PTglyphid{Ha-09_0225}}
% 49
{\PTglyphid{Ha-09_0226}}
% 50
{\PTglyphid{Ha-09_0227}}
% 51
{\PTglyphid{Ha-09_0228}}
% 52
{\PTglyphid{Ha-09_0229}}
% 53
{\PTglyphid{Ha-09_0230}}
% 54
{\PTglyphid{Ha-09_0231}}
% 55
{\PTglyphid{Ha-09_0232}}
% 56
{\PTglyphid{Ha-09_0233}}
% 57
{\PTglyphid{Ha-09_0234}}
% 58
{\PTglyphid{Ha-09_0235}}
% 59
{\PTglyphid{Ha-09_0236}}
% 60
{\PTglyphid{Ha-09_0237}}
% 61
{\PTglyphid{Ha-09_0238}}
% 62
{\PTglyphid{Ha-09_0239}}
% 63
{\PTglyphid{Ha-09_0240}}
% 64
{\PTglyphid{Ha-09_0301}}
% 65
{\PTglyphid{Ha-09_0302}}
% 66
{\PTglyphid{Ha-09_0303}}
% 67
{\PTglyphid{Ha-09_0304}}
% 68
{\PTglyphid{Ha-09_0305}}
% 69
{\PTglyphid{Ha-09_0306}}
% 70
{\PTglyphid{Ha-09_0307}}
% 71
{\PTglyphid{Ha-09_0308}}
% 72
{\PTglyphid{Ha-09_0309}}
% 73
{\PTglyphid{Ha-09_0310}}
% 74
{\PTglyphid{Ha-09_0311}}
% 75
{\PTglyphid{Ha-09_0312}}
% 76
{\PTglyphid{Ha-09_0313}}
% 77
{\PTglyphid{Ha-09_0314}}
% 78
{\PTglyphid{Ha-09_0315}}
% 79
{\PTglyphid{Ha-09_0316}}
% 80
{\PTglyphid{Ha-09_0317}}
% 81
{\PTglyphid{Ha-09_0318}}
% 82
{\PTglyphid{Ha-09_0319}}
% 83
{\PTglyphid{Ha-09_0320}}
% 84
{\PTglyphid{Ha-09_0321}}
% 85
{\PTglyphid{Ha-09_0322}}
% 86
{\PTglyphid{Ha-09_0323}}
% 87
{\PTglyphid{Ha-09_0324}}
% 88
{\PTglyphid{Ha-09_0325}}
% 89
{\PTglyphid{Ha-09_0326}}
% 90
{\PTglyphid{Ha-09_0327}}
% 91
{\PTglyphid{Ha-09_0328}}
% 92
{\PTglyphid{Ha-09_0329}}
% 93
{\PTglyphid{Ha-09_0330}}
% 94
{\PTglyphid{Ha-09_0331}}
% 95
{\PTglyphid{Ha-09_0332}}
% 96
{\PTglyphid{Ha-09_0333}}
% 97
{\PTglyphid{Ha-09_0334}}
% 98
{\PTglyphid{Ha-09_0335}}
% 99
{\PTglyphid{Ha-09_0336}}
% 100
{\PTglyphid{Ha-09_0401}}
% 101
{\PTglyphid{Ha-09_0402}}
% 102
{\PTglyphid{Ha-09_0403}}
% 103
{\PTglyphid{Ha-09_0404}}
% 104
{\PTglyphid{Ha-09_0405}}
% 105
{\PTglyphid{Ha-09_0406}}
% 106
{\PTglyphid{Ha-09_0407}}
% 107
{\PTglyphid{Ha-09_0408}}
% 108
{\PTglyphid{Ha-09_0409}}
% 109
{\PTglyphid{Ha-09_0410}}
% 110
{\PTglyphid{Ha-09_0411}}
% 111
{\PTglyphid{Ha-09_0412}}
% 112
{\PTglyphid{Ha-09_0413}}
% 113
{\PTglyphid{Ha-09_0414}}
% 114
{\PTglyphid{Ha-09_0415}}
% 115
{\PTglyphid{Ha-09_0416}}
% 116
{\PTglyphid{Ha-09_0417}}
% 117
{\PTglyphid{Ha-09_0418}}
% 118
{\PTglyphid{Ha-09_0419}}
% 119
{\PTglyphid{Ha-09_0420}}
//
\endgl \xe
%%% Local Variables:
%%% mode: latex
%%% TeX-engine: luatex
%%% TeX-master: shared
%%% End:

% //
%\endgl \xe


 \newpage

%%%%%%%%%%%%%%%%%%%%%%%%%%%%%%%%%%%%%%%%%%%%%%%%%%%%%%%%%%%%%%%%%%%%%%%%%%%%%%% 
 % Tab. 29,Haller-10_PT04_173.djvu,Haller,10,04,173
%%%%%%%%%%%%%%%%%%%%%%%%%%%%%%%%%%%%%%%%%%%%%%%%%%%%%%%%%%%%%%%%%%%%%%%%%%%%%%

% Note "10. Pismo komentarzowe. Krój M⁸⁷. Stopień 20 ww. = 72 mm. — Tabl. 173."
% Note1 "Character set table prepared by Maria Błońska"


  \pismoPL{Jan Haller 10. Pismo komentarzowe. Krój M⁸⁷. Stopień 20 ww. = 72 mm. — Tabl. 173.}
  
  \pismoEN{Jan Haller 10. Text font. Typeface M⁸⁷. Type size 20 ww. =
    72 mm . — Plate 173.}

\plate{173}{IV}{1962}

The plate prepared by Helena Kapełuś.\\
The font table prepared by Helena Kapełuś and Maria Błońska.

\bigskip

\exampleBib{IV:208}

\bigskip


\exampleDesc{IACOBUS FABER STAPULENSIS: Introductio in libros De Anima Aristotelis. Kraków, Jan Haller, 1522. 4⁰}
  

\medskip
\examplePage{\textit{Karta A₂b}}

  \bigskip
\exampleLib{Biblioteka Zakł. Nar. im. Ossolińskich. Wrocław.}


\bigskip
\exampleRef{\textit{Estreicher. XIV. 150. Wierzbowski 993.}}

% \bigskip
% \exampleDig{\url page 7} 

\bigskip

\examplePL{Rubryka \eta{}: z pismem 10.}

    \medskip

    \exampleEN{Rubric \eta{} with font 10.}


\bigskip


\fontID{Ha-10}{29}

\fontstat{102}

% \exdisplay \bg \gla
 \exdisplay \bg \gla
% 1
{\PTglyph{5}{t29_l01g01.png}}
% 2
{\PTglyph{5}{t29_l01g02.png}}
% 3
{\PTglyph{5}{t29_l01g03.png}}
% 4
{\PTglyph{5}{t29_l01g04.png}}
% 5
{\PTglyph{5}{t29_l01g05.png}}
% 6
{\PTglyph{5}{t29_l01g06.png}}
% 7
{\PTglyph{5}{t29_l01g07.png}}
% 8
{\PTglyph{5}{t29_l01g08.png}}
% 9
{\PTglyph{5}{t29_l01g09.png}}
% 10
{\PTglyph{5}{t29_l01g10.png}}
% 11
{\PTglyph{5}{t29_l01g11.png}}
% 12
{\PTglyph{5}{t29_l01g12.png}}
% 13
{\PTglyph{5}{t29_l01g13.png}}
% 14
{\PTglyph{5}{t29_l01g14.png}}
% 15
{\PTglyph{5}{t29_l01g15.png}}
% 16
{\PTglyph{5}{t29_l01g16.png}}
% 17
{\PTglyph{5}{t29_l01g17.png}}
% 18
{\PTglyph{5}{t29_l01g18.png}}
% 19
{\PTglyph{5}{t29_l01g19.png}}
% 20
{\PTglyph{5}{t29_l01g20.png}}
% 21
{\PTglyph{5}{t29_l01g21.png}}
% 22
{\PTglyph{5}{t29_l01g22.png}}
% 23
{\PTglyph{5}{t29_l02g01.png}}
% 24
{\PTglyph{5}{t29_l02g02.png}}
% 25
{\PTglyph{5}{t29_l02g03.png}}
% 26
{\PTglyph{5}{t29_l02g04.png}}
% 27
{\PTglyph{5}{t29_l02g05.png}}
% 28
{\PTglyph{5}{t29_l02g06.png}}
% 29
{\PTglyph{5}{t29_l02g07.png}}
% 30
{\PTglyph{5}{t29_l02g08.png}}
% 31
{\PTglyph{5}{t29_l02g09.png}}
% 32
{\PTglyph{5}{t29_l02g10.png}}
% 33
{\PTglyph{5}{t29_l02g11.png}}
% 34
{\PTglyph{5}{t29_l02g12.png}}
% 35
{\PTglyph{5}{t29_l02g13.png}}
% 36
{\PTglyph{5}{t29_l02g14.png}}
% 37
{\PTglyph{5}{t29_l02g15.png}}
% 38
{\PTglyph{5}{t29_l02g16.png}}
% 39
{\PTglyph{5}{t29_l02g17.png}}
% 40
{\PTglyph{5}{t29_l02g18.png}}
% 41
{\PTglyph{5}{t29_l02g19.png}}
% 42
{\PTglyph{5}{t29_l02g20.png}}
% 43
{\PTglyph{5}{t29_l02g21.png}}
% 44
{\PTglyph{5}{t29_l02g22.png}}
% 45
{\PTglyph{5}{t29_l02g23.png}}
% 46
{\PTglyph{5}{t29_l02g24.png}}
% 47
{\PTglyph{5}{t29_l02g25.png}}
% 48
{\PTglyph{5}{t29_l02g26.png}}
% 49
{\PTglyph{5}{t29_l02g27.png}}
% 50
{\PTglyph{5}{t29_l02g28.png}}
% 51
{\PTglyph{5}{t29_l02g29.png}}
% 52
{\PTglyph{5}{t29_l02g30.png}}
% 53
{\PTglyph{5}{t29_l02g31.png}}
% 54
{\PTglyph{5}{t29_l02g32.png}}
% 55
{\PTglyph{5}{t29_l02g33.png}}
% 56
{\PTglyph{5}{t29_l02g34.png}}
% 57
{\PTglyph{5}{t29_l02g35.png}}
% 58
{\PTglyph{5}{t29_l02g36.png}}
% 59
{\PTglyph{5}{t29_l02g37.png}}
% 60
{\PTglyph{5}{t29_l02g38.png}}
% 61
{\PTglyph{5}{t29_l02g39.png}}
% 62
{\PTglyph{5}{t29_l02g40.png}}
% 63
{\PTglyph{5}{t29_l03g01.png}}
% 64
{\PTglyph{5}{t29_l03g02.png}}
% 65
{\PTglyph{5}{t29_l03g03.png}}
% 66
{\PTglyph{5}{t29_l03g04.png}}
% 67
{\PTglyph{5}{t29_l03g05.png}}
% 68
{\PTglyph{5}{t29_l03g06.png}}
% 69
{\PTglyph{5}{t29_l03g07.png}}
% 70
{\PTglyph{5}{t29_l03g08.png}}
% 71
{\PTglyph{5}{t29_l03g09.png}}
% 72
{\PTglyph{5}{t29_l03g10.png}}
% 73
{\PTglyph{5}{t29_l03g11.png}}
% 74
{\PTglyph{5}{t29_l03g12.png}}
% 75
{\PTglyph{5}{t29_l03g13.png}}
% 76
{\PTglyph{5}{t29_l03g14.png}}
% 77
{\PTglyph{5}{t29_l03g15.png}}
% 78
{\PTglyph{5}{t29_l03g16.png}}
% 79
{\PTglyph{5}{t29_l03g17.png}}
% 80
{\PTglyph{5}{t29_l03g18.png}}
% 81
{\PTglyph{5}{t29_l03g19.png}}
% 82
{\PTglyph{5}{t29_l03g20.png}}
% 83
{\PTglyph{5}{t29_l03g21.png}}
% 84
{\PTglyph{5}{t29_l03g22.png}}
% 85
{\PTglyph{5}{t29_l03g23.png}}
% 86
{\PTglyph{5}{t29_l03g24.png}}
% 87
{\PTglyph{5}{t29_l03g25.png}}
% 88
{\PTglyph{5}{t29_l03g26.png}}
% 89
{\PTglyph{5}{t29_l03g27.png}}
% 90
{\PTglyph{5}{t29_l03g28.png}}
% 91
{\PTglyph{5}{t29_l03g29.png}}
% 92
{\PTglyph{5}{t29_l03g30.png}}
% 93
{\PTglyph{5}{t29_l03g31.png}}
% 94
{\PTglyph{5}{t29_l03g32.png}}
% 95
{\PTglyph{5}{t29_l03g33.png}}
% 96
{\PTglyph{5}{t29_l03g34.png}}
% 97
{\PTglyph{5}{t29_l03g35.png}}
% 98
{\PTglyph{5}{t29_l03g36.png}}
% 99
{\PTglyph{5}{t29_l03g37.png}}
% 100
{\PTglyph{5}{t29_l03g38.png}}
% 101
{\PTglyph{5}{t29_l03g39.png}}
% 102
{\PTglyph{5}{t29_l03g40.png}}
//
%%% Local Variables:
%%% mode: latex
%%% TeX-engine: luatex
%%% TeX-master: shared
%%% End:

%//
%\glpismo%
 \glpismo
% 1
{\PTglyphid{Ha-10_0101}}
% 2
{\PTglyphid{Ha-10_0102}}
% 3
{\PTglyphid{Ha-10_0103}}
% 4
{\PTglyphid{Ha-10_0104}}
% 5
{\PTglyphid{Ha-10_0105}}
% 6
{\PTglyphid{Ha-10_0106}}
% 7
{\PTglyphid{Ha-10_0107}}
% 8
{\PTglyphid{Ha-10_0108}}
% 9
{\PTglyphid{Ha-10_0109}}
% 10
{\PTglyphid{Ha-10_0110}}
% 11
{\PTglyphid{Ha-10_0111}}
% 12
{\PTglyphid{Ha-10_0112}}
% 13
{\PTglyphid{Ha-10_0113}}
% 14
{\PTglyphid{Ha-10_0114}}
% 15
{\PTglyphid{Ha-10_0115}}
% 16
{\PTglyphid{Ha-10_0116}}
% 17
{\PTglyphid{Ha-10_0117}}
% 18
{\PTglyphid{Ha-10_0118}}
% 19
{\PTglyphid{Ha-10_0119}}
% 20
{\PTglyphid{Ha-10_0120}}
% 21
{\PTglyphid{Ha-10_0121}}
% 22
{\PTglyphid{Ha-10_0122}}
% 23
{\PTglyphid{Ha-10_0201}}
% 24
{\PTglyphid{Ha-10_0202}}
% 25
{\PTglyphid{Ha-10_0203}}
% 26
{\PTglyphid{Ha-10_0204}}
% 27
{\PTglyphid{Ha-10_0205}}
% 28
{\PTglyphid{Ha-10_0206}}
% 29
{\PTglyphid{Ha-10_0207}}
% 30
{\PTglyphid{Ha-10_0208}}
% 31
{\PTglyphid{Ha-10_0209}}
% 32
{\PTglyphid{Ha-10_0210}}
% 33
{\PTglyphid{Ha-10_0211}}
% 34
{\PTglyphid{Ha-10_0212}}
% 35
{\PTglyphid{Ha-10_0213}}
% 36
{\PTglyphid{Ha-10_0214}}
% 37
{\PTglyphid{Ha-10_0215}}
% 38
{\PTglyphid{Ha-10_0216}}
% 39
{\PTglyphid{Ha-10_0217}}
% 40
{\PTglyphid{Ha-10_0218}}
% 41
{\PTglyphid{Ha-10_0219}}
% 42
{\PTglyphid{Ha-10_0220}}
% 43
{\PTglyphid{Ha-10_0221}}
% 44
{\PTglyphid{Ha-10_0222}}
% 45
{\PTglyphid{Ha-10_0223}}
% 46
{\PTglyphid{Ha-10_0224}}
% 47
{\PTglyphid{Ha-10_0225}}
% 48
{\PTglyphid{Ha-10_0226}}
% 49
{\PTglyphid{Ha-10_0227}}
% 50
{\PTglyphid{Ha-10_0228}}
% 51
{\PTglyphid{Ha-10_0229}}
% 52
{\PTglyphid{Ha-10_0230}}
% 53
{\PTglyphid{Ha-10_0231}}
% 54
{\PTglyphid{Ha-10_0232}}
% 55
{\PTglyphid{Ha-10_0233}}
% 56
{\PTglyphid{Ha-10_0234}}
% 57
{\PTglyphid{Ha-10_0235}}
% 58
{\PTglyphid{Ha-10_0236}}
% 59
{\PTglyphid{Ha-10_0237}}
% 60
{\PTglyphid{Ha-10_0238}}
% 61
{\PTglyphid{Ha-10_0239}}
% 62
{\PTglyphid{Ha-10_0240}}
% 63
{\PTglyphid{Ha-10_0301}}
% 64
{\PTglyphid{Ha-10_0302}}
% 65
{\PTglyphid{Ha-10_0303}}
% 66
{\PTglyphid{Ha-10_0304}}
% 67
{\PTglyphid{Ha-10_0305}}
% 68
{\PTglyphid{Ha-10_0306}}
% 69
{\PTglyphid{Ha-10_0307}}
% 70
{\PTglyphid{Ha-10_0308}}
% 71
{\PTglyphid{Ha-10_0309}}
% 72
{\PTglyphid{Ha-10_0310}}
% 73
{\PTglyphid{Ha-10_0311}}
% 74
{\PTglyphid{Ha-10_0312}}
% 75
{\PTglyphid{Ha-10_0313}}
% 76
{\PTglyphid{Ha-10_0314}}
% 77
{\PTglyphid{Ha-10_0315}}
% 78
{\PTglyphid{Ha-10_0316}}
% 79
{\PTglyphid{Ha-10_0317}}
% 80
{\PTglyphid{Ha-10_0318}}
% 81
{\PTglyphid{Ha-10_0319}}
% 82
{\PTglyphid{Ha-10_0320}}
% 83
{\PTglyphid{Ha-10_0321}}
% 84
{\PTglyphid{Ha-10_0322}}
% 85
{\PTglyphid{Ha-10_0323}}
% 86
{\PTglyphid{Ha-10_0324}}
% 87
{\PTglyphid{Ha-10_0325}}
% 88
{\PTglyphid{Ha-10_0326}}
% 89
{\PTglyphid{Ha-10_0327}}
% 90
{\PTglyphid{Ha-10_0328}}
% 91
{\PTglyphid{Ha-10_0329}}
% 92
{\PTglyphid{Ha-10_0330}}
% 93
{\PTglyphid{Ha-10_0331}}
% 94
{\PTglyphid{Ha-10_0332}}
% 95
{\PTglyphid{Ha-10_0333}}
% 96
{\PTglyphid{Ha-10_0334}}
% 97
{\PTglyphid{Ha-10_0335}}
% 98
{\PTglyphid{Ha-10_0336}}
% 99
{\PTglyphid{Ha-10_0337}}
% 100
{\PTglyphid{Ha-10_0338}}
% 101
{\PTglyphid{Ha-10_0339}}
% 102
{\PTglyphid{Ha-10_0340}}
//
\endgl \xe
%%% Local Variables:
%%% mode: latex
%%% TeX-engine: luatex
%%% TeX-master: shared
%%% End:

% //
%\endgl \xe

 
 \newpage

%%%%%%%%%%%%%%%%%%%%%%%%%%%%%%%%%%%%%%%%%%%%%%%%%%%%%%%%%%%%%%%%%%%%%%%%%%%%%%% 
 % Tab. 30,Haller-11_PT04_174.djvu,Haller,11,04,174
%%%%%%%%%%%%%%%%%%%%%%%%%%%%%%%%%%%%%%%%%%%%%%%%%%%%%%%%%%%%%%%%%%%%%%%%%%%%%%

% Note "11. Pismo tekstowe antykwowe. Krój Qlu (C). Stopień 20 ww. = 92 mm. — - Tabl. 174."
% Note1 "Character set table prepared by Maria Błońska"


  \pismoPL{Jan Haller 11. Pismo tekstowe antykwowe. Krój Q|u (C). Stopień 20 ww. = 92 mm. — - Tabl. 174.}
  
  \pismoEN{Jan Haller 11. Roman text font. Typeface Q|u (C). Type size 20 ww. =
    92 mm . — Plate 174.}

\plate{174}{IV}{1962}

Prepared by Helena Kapełuś.\\
The font table prepared by Helena Kapełuś and Maria Błońska.

\bigskip

\exampleBib{IV:163}

\bigskip


\exampleDesc{STANISLAUS ZABOROWSKI: Grammatices rudimenta. Kraków, Jan Haller, 1518. 4⁰}
  
% https://cyfrowe.mnk.pl/dlibra/publication/24392/edition/24082?language=en
\medskip
\examplePage{\textit{Karta K₃a}}

  \bigskip
\exampleLib{Biblioteka Narodowa. Warszawa.}



\bigskip
\exampleRef{\textit{Estreicher.XXXIV. 45. Wierzbowski 957.}}

 \bigskip
\exampleDig{\url{https://polona.pl/preview/dfb921b2-a937-4ee8-ad01-0724aa52c76a}
  page 121}

\bigskip

\examplePL{Pismo 11 (tekst łaciński). — Rubryki \theta, \iota. — Cyfry 6.}

    \medskip

    \exampleEN{Font 11 (Latin text). — Rubric \theta{}, \iota. — Digits 6.}


\bigskip

\exampleBib{IV:16}

\bigskip


\exampleDesc{STANISLAUS ZABOROWSKI: Orthographia seu modus recte scribendi. Kraków, Jan Haller, IV. 1518. 4⁰.}




\medskip
\examplePage{\textit{Karta [5] a}}

  \bigskip
\exampleLib{Biblioteka Narodowa. Warszawa.}



\bigskip
\exampleRef{\textit{Estreicher.XXXIV. 52. Wierzbowski 48.}}

 \bigskip
\exampleDig{\url{https://polona.pl/preview/2db2e5da-ec87-426c-8e0d-e6b09093407a}
  page 17}
% https://zpe.gov.pl/kronika/759455

\bigskip

\examplePL{Pismo 11 (tekst polski). — Rubryka \theta{}. — Inicjał 60 (V).}

    \medskip

    \exampleEN{Font 11 (Polish text). — Rubric \theta{}. — Initial 60. (V)}


\bigskip


\fontID{Ha-11}{30}

\fontstat{112}

% \exdisplay \bg \gla
 \exdisplay \bg \gla
% 1
{\PTglyph{5}{t30_l01g01.png}}
% 2
{\PTglyph{5}{t30_l01g02.png}}
% 3
{\PTglyph{5}{t30_l01g03.png}}
% 4
{\PTglyph{5}{t30_l01g04.png}}
% 5
{\PTglyph{5}{t30_l01g05.png}}
% 6
{\PTglyph{5}{t30_l01g06.png}}
% 7
{\PTglyph{5}{t30_l01g07.png}}
% 8
{\PTglyph{5}{t30_l01g08.png}}
% 9
{\PTglyph{5}{t30_l01g09.png}}
% 10
{\PTglyph{5}{t30_l01g10.png}}
% 11
{\PTglyph{5}{t30_l01g11.png}}
% 12
{\PTglyph{5}{t30_l01g12.png}}
% 13
{\PTglyph{5}{t30_l01g13.png}}
% 14
{\PTglyph{5}{t30_l01g14.png}}
% 15
{\PTglyph{5}{t30_l01g15.png}}
% 16
{\PTglyph{5}{t30_l01g16.png}}
% 17
{\PTglyph{5}{t30_l01g17.png}}
% 18
{\PTglyph{5}{t30_l01g18.png}}
% 19
{\PTglyph{5}{t30_l01g19.png}}
% 20
{\PTglyph{5}{t30_l01g20.png}}
% 21
{\PTglyph{5}{t30_l01g21.png}}
% 22
{\PTglyph{5}{t30_l01g22.png}}
% 23
{\PTglyph{5}{t30_l01g23.png}}
% 24
{\PTglyph{5}{t30_l01g24.png}}
% 25
{\PTglyph{5}{t30_l01g25.png}}
% 26
{\PTglyph{5}{t30_l02g01.png}}
% 27
{\PTglyph{5}{t30_l02g02.png}}
% 28
{\PTglyph{5}{t30_l02g03.png}}
% 29
{\PTglyph{5}{t30_l02g04.png}}
% 30
{\PTglyph{5}{t30_l02g05.png}}
% 31
{\PTglyph{5}{t30_l02g06.png}}
% 32
{\PTglyph{5}{t30_l02g07.png}}
% 33
{\PTglyph{5}{t30_l02g08.png}}
% 34
{\PTglyph{5}{t30_l02g09.png}}
% 35
{\PTglyph{5}{t30_l02g10.png}}
% 36
{\PTglyph{5}{t30_l02g11.png}}
% 37
{\PTglyph{5}{t30_l02g12.png}}
% 38
{\PTglyph{5}{t30_l02g13.png}}
% 39
{\PTglyph{5}{t30_l02g14.png}}
% 40
{\PTglyph{5}{t30_l02g15.png}}
% 41
{\PTglyph{5}{t30_l02g16.png}}
% 42
{\PTglyph{5}{t30_l02g17.png}}
% 43
{\PTglyph{5}{t30_l02g18.png}}
% 44
{\PTglyph{5}{t30_l02g19.png}}
% 45
{\PTglyph{5}{t30_l02g20.png}}
% 46
{\PTglyph{5}{t30_l02g21.png}}
% 47
{\PTglyph{5}{t30_l02g22.png}}
% 48
{\PTglyph{5}{t30_l02g23.png}}
% 49
{\PTglyph{5}{t30_l02g24.png}}
% 50
{\PTglyph{5}{t30_l02g25.png}}
% 51
{\PTglyph{5}{t30_l02g26.png}}
% 52
{\PTglyph{5}{t30_l02g27.png}}
% 53
{\PTglyph{5}{t30_l02g28.png}}
% 54
{\PTglyph{5}{t30_l02g29.png}}
% 55
{\PTglyph{5}{t30_l02g30.png}}
% 56
{\PTglyph{5}{t30_l02g31.png}}
% 57
{\PTglyph{5}{t30_l02g32.png}}
% 58
{\PTglyph{5}{t30_l02g33.png}}
% 59
{\PTglyph{5}{t30_l02g34.png}}
% 60
{\PTglyph{5}{t30_l02g35.png}}
% 61
{\PTglyph{5}{t30_l02g36.png}}
% 62
{\PTglyph{5}{t30_l02g37.png}}
% 63
{\PTglyph{5}{t30_l02g38.png}}
% 64
{\PTglyph{5}{t30_l02g39.png}}
% 65
{\PTglyph{5}{t30_l02g40.png}}
% 66
{\PTglyph{5}{t30_l03g01.png}}
% 67
{\PTglyph{5}{t30_l03g02.png}}
% 68
{\PTglyph{5}{t30_l03g03.png}}
% 69
{\PTglyph{5}{t30_l03g04.png}}
% 70
{\PTglyph{5}{t30_l03g05.png}}
% 71
{\PTglyph{5}{t30_l03g06.png}}
% 72
{\PTglyph{5}{t30_l03g07.png}}
% 73
{\PTglyph{5}{t30_l03g08.png}}
% 74
{\PTglyph{5}{t30_l03g09.png}}
% 75
{\PTglyph{5}{t30_l03g10.png}}
% 76
{\PTglyph{5}{t30_l03g11.png}}
% 77
{\PTglyph{5}{t30_l03g12.png}}
% 78
{\PTglyph{5}{t30_l03g13.png}}
% 79
{\PTglyph{5}{t30_l03g14.png}}
% 80
{\PTglyph{5}{t30_l03g15.png}}
% 81
{\PTglyph{5}{t30_l03g16.png}}
% 82
{\PTglyph{5}{t30_l03g17.png}}
% 83
{\PTglyph{5}{t30_l03g18.png}}
% 84
{\PTglyph{5}{t30_l03g19.png}}
% 85
{\PTglyph{5}{t30_l03g20.png}}
% 86
{\PTglyph{5}{t30_l03g21.png}}
% 87
{\PTglyph{5}{t30_l03g22.png}}
% 88
{\PTglyph{5}{t30_l03g23.png}}
% 89
{\PTglyph{5}{t30_l03g24.png}}
% 90
{\PTglyph{5}{t30_l03g25.png}}
% 91
{\PTglyph{5}{t30_l03g26.png}}
% 92
{\PTglyph{5}{t30_l03g27.png}}
% 93
{\PTglyph{5}{t30_l03g28.png}}
% 94
{\PTglyph{5}{t30_l03g29.png}}
% 95
{\PTglyph{5}{t30_l03g30.png}}
% 96
{\PTglyph{5}{t30_l03g31.png}}
% 97
{\PTglyph{5}{t30_l03g32.png}}
% 98
{\PTglyph{5}{t30_l03g33.png}}
% 99
{\PTglyph{5}{t30_l03g34.png}}
% 100
{\PTglyph{5}{t30_l03g35.png}}
% 101
{\PTglyph{5}{t30_l03g36.png}}
% 102
{\PTglyph{5}{t30_l04g01.png}}
% 103
{\PTglyph{5}{t30_l04g02.png}}
% 104
{\PTglyph{5}{t30_l04g03.png}}
% 105
{\PTglyph{5}{t30_l04g04.png}}
% 106
{\PTglyph{5}{t30_l04g05.png}}
% 107
{\PTglyph{5}{t30_l04g06.png}}
% 108
{\PTglyph{5}{t30_l04g07.png}}
% 109
{\PTglyph{5}{t30_l04g08.png}}
% 110
{\PTglyph{5}{t30_l04g09.png}}
% 111
{\PTglyph{5}{t30_l04g10.png}}
% 112
{\PTglyph{5}{t30_l05g01.png}}
//
%%% Local Variables:
%%% mode: latex
%%% TeX-engine: luatex
%%% TeX-master: shared
%%% End:

%//
%\glpismo%
 \glpismo
% 1
{\PTglyphid{Ha-11_0101}}
% 2
{\PTglyphid{Ha-11_0102}}
% 3
{\PTglyphid{Ha-11_0103}}
% 4
{\PTglyphid{Ha-11_0104}}
% 5
{\PTglyphid{Ha-11_0105}}
% 6
{\PTglyphid{Ha-11_0106}}
% 7
{\PTglyphid{Ha-11_0107}}
% 8
{\PTglyphid{Ha-11_0108}}
% 9
{\PTglyphid{Ha-11_0109}}
% 10
{\PTglyphid{Ha-11_0110}}
% 11
{\PTglyphid{Ha-11_0111}}
% 12
{\PTglyphid{Ha-11_0112}}
% 13
{\PTglyphid{Ha-11_0113}}
% 14
{\PTglyphid{Ha-11_0114}}
% 15
{\PTglyphid{Ha-11_0115}}
% 16
{\PTglyphid{Ha-11_0116}}
% 17
{\PTglyphid{Ha-11_0117}}
% 18
{\PTglyphid{Ha-11_0118}}
% 19
{\PTglyphid{Ha-11_0119}}
% 20
{\PTglyphid{Ha-11_0120}}
% 21
{\PTglyphid{Ha-11_0121}}
% 22
{\PTglyphid{Ha-11_0122}}
% 23
{\PTglyphid{Ha-11_0123}}
% 24
{\PTglyphid{Ha-11_0124}}
% 25
{\PTglyphid{Ha-11_0125}}
% 26
{\PTglyphid{Ha-11_0201}}
% 27
{\PTglyphid{Ha-11_0202}}
% 28
{\PTglyphid{Ha-11_0203}}
% 29
{\PTglyphid{Ha-11_0204}}
% 30
{\PTglyphid{Ha-11_0205}}
% 31
{\PTglyphid{Ha-11_0206}}
% 32
{\PTglyphid{Ha-11_0207}}
% 33
{\PTglyphid{Ha-11_0208}}
% 34
{\PTglyphid{Ha-11_0209}}
% 35
{\PTglyphid{Ha-11_0210}}
% 36
{\PTglyphid{Ha-11_0211}}
% 37
{\PTglyphid{Ha-11_0212}}
% 38
{\PTglyphid{Ha-11_0213}}
% 39
{\PTglyphid{Ha-11_0214}}
% 40
{\PTglyphid{Ha-11_0215}}
% 41
{\PTglyphid{Ha-11_0216}}
% 42
{\PTglyphid{Ha-11_0217}}
% 43
{\PTglyphid{Ha-11_0218}}
% 44
{\PTglyphid{Ha-11_0219}}
% 45
{\PTglyphid{Ha-11_0220}}
% 46
{\PTglyphid{Ha-11_0221}}
% 47
{\PTglyphid{Ha-11_0222}}
% 48
{\PTglyphid{Ha-11_0223}}
% 49
{\PTglyphid{Ha-11_0224}}
% 50
{\PTglyphid{Ha-11_0225}}
% 51
{\PTglyphid{Ha-11_0226}}
% 52
{\PTglyphid{Ha-11_0227}}
% 53
{\PTglyphid{Ha-11_0228}}
% 54
{\PTglyphid{Ha-11_0229}}
% 55
{\PTglyphid{Ha-11_0230}}
% 56
{\PTglyphid{Ha-11_0231}}
% 57
{\PTglyphid{Ha-11_0232}}
% 58
{\PTglyphid{Ha-11_0233}}
% 59
{\PTglyphid{Ha-11_0234}}
% 60
{\PTglyphid{Ha-11_0235}}
% 61
{\PTglyphid{Ha-11_0236}}
% 62
{\PTglyphid{Ha-11_0237}}
% 63
{\PTglyphid{Ha-11_0238}}
% 64
{\PTglyphid{Ha-11_0239}}
% 65
{\PTglyphid{Ha-11_0240}}
% 66
{\PTglyphid{Ha-11_0301}}
% 67
{\PTglyphid{Ha-11_0302}}
% 68
{\PTglyphid{Ha-11_0303}}
% 69
{\PTglyphid{Ha-11_0304}}
% 70
{\PTglyphid{Ha-11_0305}}
% 71
{\PTglyphid{Ha-11_0306}}
% 72
{\PTglyphid{Ha-11_0307}}
% 73
{\PTglyphid{Ha-11_0308}}
% 74
{\PTglyphid{Ha-11_0309}}
% 75
{\PTglyphid{Ha-11_0310}}
% 76
{\PTglyphid{Ha-11_0311}}
% 77
{\PTglyphid{Ha-11_0312}}
% 78
{\PTglyphid{Ha-11_0313}}
% 79
{\PTglyphid{Ha-11_0314}}
% 80
{\PTglyphid{Ha-11_0315}}
% 81
{\PTglyphid{Ha-11_0316}}
% 82
{\PTglyphid{Ha-11_0317}}
% 83
{\PTglyphid{Ha-11_0318}}
% 84
{\PTglyphid{Ha-11_0319}}
% 85
{\PTglyphid{Ha-11_0320}}
% 86
{\PTglyphid{Ha-11_0321}}
% 87
{\PTglyphid{Ha-11_0322}}
% 88
{\PTglyphid{Ha-11_0323}}
% 89
{\PTglyphid{Ha-11_0324}}
% 90
{\PTglyphid{Ha-11_0325}}
% 91
{\PTglyphid{Ha-11_0326}}
% 92
{\PTglyphid{Ha-11_0327}}
% 93
{\PTglyphid{Ha-11_0328}}
% 94
{\PTglyphid{Ha-11_0329}}
% 95
{\PTglyphid{Ha-11_0330}}
% 96
{\PTglyphid{Ha-11_0331}}
% 97
{\PTglyphid{Ha-11_0332}}
% 98
{\PTglyphid{Ha-11_0333}}
% 99
{\PTglyphid{Ha-11_0334}}
% 100
{\PTglyphid{Ha-11_0335}}
% 101
{\PTglyphid{Ha-11_0336}}
% 102
{\PTglyphid{Ha-11_0401}}
% 103
{\PTglyphid{Ha-11_0402}}
% 104
{\PTglyphid{Ha-11_0403}}
% 105
{\PTglyphid{Ha-11_0404}}
% 106
{\PTglyphid{Ha-11_0405}}
% 107
{\PTglyphid{Ha-11_0406}}
% 108
{\PTglyphid{Ha-11_0407}}
% 109
{\PTglyphid{Ha-11_0408}}
% 110
{\PTglyphid{Ha-11_0409}}
% 111
{\PTglyphid{Ha-11_0410}}
% 112
{\PTglyphid{Ha-11_0501}}
//
\endgl \xe
%%% Local Variables:
%%% mode: latex
%%% TeX-engine: luatex
%%% TeX-master: shared
%%% End:

% //
%\endgl \xe

 
 \newpage

%%%%%%%%%%%%%%%%%%%%%%%%%%%%%%%%%%%%%%%%%%%%%%%%%%%%%%%%%%%%%%%%%%%%%%%%%%%%%%% 
 % Tab. 31,Haller-12_PT04_175.djvu,Haller,12,04,175
%%%%%%%%%%%%%%%%%%%%%%%%%%%%%%%%%%%%%%%%%%%%%%%%%%%%%%%%%%%%%%%%%%%%%%%%%%%%%%%

% Note "12. Pismo mszalne. Krój M⁶⁰. Stopień 20 ww. = 176 mm. — Tabl. 175."
% Note1 "Character set table prepared by Maria Błońska"

  \pismoPL{Jan Haller 12. Pismo mszalne. Krój M⁶⁰. Stopień 20 ww. = 176 mm. — Tabl. 175.}
  
  \pismoEN{Jan Haller 12. Missal font. Typeface M⁶⁰. Type size 20 ww. =
    176 mm . — Plate 175.}

\plate{175}{IV}{1962}

The plate prepared by Helena Kapełuś.\\
The font table prepared by Helena Kapełuś and Maria Błońska.

\bigskip

\exampleBib{IV:173}

\bigskip

\exampleDesc{PSALTERIUM secundum morem Ecclesiae Cracoviensis. Kraków, Jan Haller, 1518. 2⁰.}
  

\medskip
\examplePage{\textit{Karta ostatnia, kolofon.}}

\examplePageEN{\textit{Last page, colophon}}

  
  \bigskip
\exampleLib{Biblioteka Narodowa. Warszawa.}

\bigskip
\exampleRef{\textit{Estreicher XXV. 387.}}

% \bigskip
% \exampleDig{\url{https://polona.pl/preview/dfb921b2-a937-4ee8-ad01-0724aa52c76a}
%   page 121}

\bigskip

\examplePL{Pismo 12. — Rubryki \chi{}, \lambda{}. — Cyfry 8.}

    \medskip

    \exampleEN{Font 12. — Rubric \chi{}, \lambda. — Digits 8.}


\bigskip

\bigskip


\fontID{Ha-12}{31}

\fontstat{100}

% \exdisplay \bg \gla
 \exdisplay \bg \gla
% 1
{\PTglyph{5}{t31_l01g01.png}}
% 2
{\PTglyph{5}{t31_l01g02.png}}
% 3
{\PTglyph{5}{t31_l01g03.png}}
% 4
{\PTglyph{5}{t31_l01g04.png}}
% 5
{\PTglyph{5}{t31_l01g05.png}}
% 6
{\PTglyph{5}{t31_l01g06.png}}
% 7
{\PTglyph{5}{t31_l01g07.png}}
% 8
{\PTglyph{5}{t31_l01g08.png}}
% 9
{\PTglyph{5}{t31_l01g09.png}}
% 10
{\PTglyph{5}{t31_l01g10.png}}
% 11
{\PTglyph{5}{t31_l01g11.png}}
% 12
{\PTglyph{5}{t31_l01g12.png}}
% 13
{\PTglyph{5}{t31_l01g13.png}}
% 14
{\PTglyph{5}{t31_l01g14.png}}
% 15
{\PTglyph{5}{t31_l01g15.png}}
% 16
{\PTglyph{5}{t31_l01g16.png}}
% 17
{\PTglyph{5}{t31_l01g17.png}}
% 18
{\PTglyph{5}{t31_l01g18.png}}
% 19
{\PTglyph{5}{t31_l01g19.png}}
% 20
{\PTglyph{5}{t31_l01g20.png}}
% 21
{\PTglyph{5}{t31_l01g21.png}}
% 22
{\PTglyph{5}{t31_l01g22.png}}
% 23
{\PTglyph{5}{t31_l02g01.png}}
% 24
{\PTglyph{5}{t31_l02g02.png}}
% 25
{\PTglyph{5}{t31_l02g03.png}}
% 26
{\PTglyph{5}{t31_l02g04.png}}
% 27
{\PTglyph{5}{t31_l02g05.png}}
% 28
{\PTglyph{5}{t31_l02g06.png}}
% 29
{\PTglyph{5}{t31_l02g07.png}}
% 30
{\PTglyph{5}{t31_l02g08.png}}
% 31
{\PTglyph{5}{t31_l02g09.png}}
% 32
{\PTglyph{5}{t31_l02g10.png}}
% 33
{\PTglyph{5}{t31_l02g11.png}}
% 34
{\PTglyph{5}{t31_l02g12.png}}
% 35
{\PTglyph{5}{t31_l02g13.png}}
% 36
{\PTglyph{5}{t31_l02g14.png}}
% 37
{\PTglyph{5}{t31_l02g15.png}}
% 38
{\PTglyph{5}{t31_l02g16.png}}
% 39
{\PTglyph{5}{t31_l02g17.png}}
% 40
{\PTglyph{5}{t31_l02g18.png}}
% 41
{\PTglyph{5}{t31_l02g19.png}}
% 42
{\PTglyph{5}{t31_l02g20.png}}
% 43
{\PTglyph{5}{t31_l02g21.png}}
% 44
{\PTglyph{5}{t31_l02g22.png}}
% 45
{\PTglyph{5}{t31_l02g23.png}}
% 46
{\PTglyph{5}{t31_l02g24.png}}
% 47
{\PTglyph{5}{t31_l02g25.png}}
% 48
{\PTglyph{5}{t31_l02g26.png}}
% 49
{\PTglyph{5}{t31_l02g27.png}}
% 50
{\PTglyph{5}{t31_l02g28.png}}
% 51
{\PTglyph{5}{t31_l02g29.png}}
% 52
{\PTglyph{5}{t31_l02g30.png}}
% 53
{\PTglyph{5}{t31_l02g31.png}}
% 54
{\PTglyph{5}{t31_l02g32.png}}
% 55
{\PTglyph{5}{t31_l02g33.png}}
% 56
{\PTglyph{5}{t31_l02g34.png}}
% 57
{\PTglyph{5}{t31_l02g35.png}}
% 58
{\PTglyph{5}{t31_l02g36.png}}
% 59
{\PTglyph{5}{t31_l02g37.png}}
% 60
{\PTglyph{5}{t31_l03g01.png}}
% 61
{\PTglyph{5}{t31_l03g02.png}}
% 62
{\PTglyph{5}{t31_l03g03.png}}
% 63
{\PTglyph{5}{t31_l03g04.png}}
% 64
{\PTglyph{5}{t31_l03g05.png}}
% 65
{\PTglyph{5}{t31_l03g06.png}}
% 66
{\PTglyph{5}{t31_l03g07.png}}
% 67
{\PTglyph{5}{t31_l03g08.png}}
% 68
{\PTglyph{5}{t31_l03g09.png}}
% 69
{\PTglyph{5}{t31_l03g10.png}}
% 70
{\PTglyph{5}{t31_l03g11.png}}
% 71
{\PTglyph{5}{t31_l03g12.png}}
% 72
{\PTglyph{5}{t31_l03g13.png}}
% 73
{\PTglyph{5}{t31_l03g14.png}}
% 74
{\PTglyph{5}{t31_l03g15.png}}
% 75
{\PTglyph{5}{t31_l03g16.png}}
% 76
{\PTglyph{5}{t31_l03g17.png}}
% 77
{\PTglyph{5}{t31_l03g18.png}}
% 78
{\PTglyph{5}{t31_l03g19.png}}
% 79
{\PTglyph{5}{t31_l03g20.png}}
% 80
{\PTglyph{5}{t31_l03g21.png}}
% 81
{\PTglyph{5}{t31_l03g22.png}}
% 82
{\PTglyph{5}{t31_l03g23.png}}
% 83
{\PTglyph{5}{t31_l03g24.png}}
% 84
{\PTglyph{5}{t31_l03g25.png}}
% 85
{\PTglyph{5}{t31_l03g26.png}}
% 86
{\PTglyph{5}{t31_l03g27.png}}
% 87
{\PTglyph{5}{t31_l03g28.png}}
% 88
{\PTglyph{5}{t31_l03g29.png}}
% 89
{\PTglyph{5}{t31_l03g30.png}}
% 90
{\PTglyph{5}{t31_l03g31.png}}
% 91
{\PTglyph{5}{t31_l03g32.png}}
% 92
{\PTglyph{5}{t31_l03g33.png}}
% 93
{\PTglyph{5}{t31_l03g34.png}}
% 94
{\PTglyph{5}{t31_l03g35.png}}
% 95
{\PTglyph{5}{t31_l03g36.png}}
% 96
{\PTglyph{5}{t31_l03g37.png}}
% 97
{\PTglyph{5}{t31_l04g01.png}}
% 98
{\PTglyph{5}{t31_l04g02.png}}
% 99
{\PTglyph{5}{t31_l04g03.png}}
% 100
{\PTglyph{5}{t31_l04g04.png}}
//
%%% Local Variables:
%%% mode: latex
%%% TeX-engine: luatex
%%% TeX-master: shared
%%% End:

%//
%\glpismo%
 \glpismo
% 1
{\PTglyphid{Ha-12_0101}}
% 2
{\PTglyphid{Ha-12_0102}}
% 3
{\PTglyphid{Ha-12_0103}}
% 4
{\PTglyphid{Ha-12_0104}}
% 5
{\PTglyphid{Ha-12_0105}}
% 6
{\PTglyphid{Ha-12_0106}}
% 7
{\PTglyphid{Ha-12_0107}}
% 8
{\PTglyphid{Ha-12_0108}}
% 9
{\PTglyphid{Ha-12_0109}}
% 10
{\PTglyphid{Ha-12_0110}}
% 11
{\PTglyphid{Ha-12_0111}}
% 12
{\PTglyphid{Ha-12_0112}}
% 13
{\PTglyphid{Ha-12_0113}}
% 14
{\PTglyphid{Ha-12_0114}}
% 15
{\PTglyphid{Ha-12_0115}}
% 16
{\PTglyphid{Ha-12_0116}}
% 17
{\PTglyphid{Ha-12_0117}}
% 18
{\PTglyphid{Ha-12_0118}}
% 19
{\PTglyphid{Ha-12_0119}}
% 20
{\PTglyphid{Ha-12_0120}}
% 21
{\PTglyphid{Ha-12_0121}}
% 22
{\PTglyphid{Ha-12_0122}}
% 23
{\PTglyphid{Ha-12_0201}}
% 24
{\PTglyphid{Ha-12_0202}}
% 25
{\PTglyphid{Ha-12_0203}}
% 26
{\PTglyphid{Ha-12_0204}}
% 27
{\PTglyphid{Ha-12_0205}}
% 28
{\PTglyphid{Ha-12_0206}}
% 29
{\PTglyphid{Ha-12_0207}}
% 30
{\PTglyphid{Ha-12_0208}}
% 31
{\PTglyphid{Ha-12_0209}}
% 32
{\PTglyphid{Ha-12_0210}}
% 33
{\PTglyphid{Ha-12_0211}}
% 34
{\PTglyphid{Ha-12_0212}}
% 35
{\PTglyphid{Ha-12_0213}}
% 36
{\PTglyphid{Ha-12_0214}}
% 37
{\PTglyphid{Ha-12_0215}}
% 38
{\PTglyphid{Ha-12_0216}}
% 39
{\PTglyphid{Ha-12_0217}}
% 40
{\PTglyphid{Ha-12_0218}}
% 41
{\PTglyphid{Ha-12_0219}}
% 42
{\PTglyphid{Ha-12_0220}}
% 43
{\PTglyphid{Ha-12_0221}}
% 44
{\PTglyphid{Ha-12_0222}}
% 45
{\PTglyphid{Ha-12_0223}}
% 46
{\PTglyphid{Ha-12_0224}}
% 47
{\PTglyphid{Ha-12_0225}}
% 48
{\PTglyphid{Ha-12_0226}}
% 49
{\PTglyphid{Ha-12_0227}}
% 50
{\PTglyphid{Ha-12_0228}}
% 51
{\PTglyphid{Ha-12_0229}}
% 52
{\PTglyphid{Ha-12_0230}}
% 53
{\PTglyphid{Ha-12_0231}}
% 54
{\PTglyphid{Ha-12_0232}}
% 55
{\PTglyphid{Ha-12_0233}}
% 56
{\PTglyphid{Ha-12_0234}}
% 57
{\PTglyphid{Ha-12_0235}}
% 58
{\PTglyphid{Ha-12_0236}}
% 59
{\PTglyphid{Ha-12_0237}}
% 60
{\PTglyphid{Ha-12_0301}}
% 61
{\PTglyphid{Ha-12_0302}}
% 62
{\PTglyphid{Ha-12_0303}}
% 63
{\PTglyphid{Ha-12_0304}}
% 64
{\PTglyphid{Ha-12_0305}}
% 65
{\PTglyphid{Ha-12_0306}}
% 66
{\PTglyphid{Ha-12_0307}}
% 67
{\PTglyphid{Ha-12_0308}}
% 68
{\PTglyphid{Ha-12_0309}}
% 69
{\PTglyphid{Ha-12_0310}}
% 70
{\PTglyphid{Ha-12_0311}}
% 71
{\PTglyphid{Ha-12_0312}}
% 72
{\PTglyphid{Ha-12_0313}}
% 73
{\PTglyphid{Ha-12_0314}}
% 74
{\PTglyphid{Ha-12_0315}}
% 75
{\PTglyphid{Ha-12_0316}}
% 76
{\PTglyphid{Ha-12_0317}}
% 77
{\PTglyphid{Ha-12_0318}}
% 78
{\PTglyphid{Ha-12_0319}}
% 79
{\PTglyphid{Ha-12_0320}}
% 80
{\PTglyphid{Ha-12_0321}}
% 81
{\PTglyphid{Ha-12_0322}}
% 82
{\PTglyphid{Ha-12_0323}}
% 83
{\PTglyphid{Ha-12_0324}}
% 84
{\PTglyphid{Ha-12_0325}}
% 85
{\PTglyphid{Ha-12_0326}}
% 86
{\PTglyphid{Ha-12_0327}}
% 87
{\PTglyphid{Ha-12_0328}}
% 88
{\PTglyphid{Ha-12_0329}}
% 89
{\PTglyphid{Ha-12_0330}}
% 90
{\PTglyphid{Ha-12_0331}}
% 91
{\PTglyphid{Ha-12_0332}}
% 92
{\PTglyphid{Ha-12_0333}}
% 93
{\PTglyphid{Ha-12_0334}}
% 94
{\PTglyphid{Ha-12_0335}}
% 95
{\PTglyphid{Ha-12_0336}}
% 96
{\PTglyphid{Ha-12_0337}}
% 97
{\PTglyphid{Ha-12_0401}}
% 98
{\PTglyphid{Ha-12_0402}}
% 99
{\PTglyphid{Ha-12_0403}}
% 100
{\PTglyphid{Ha-12_0404}}
//
\endgl \xe
%%% Local Variables:
%%% mode: latex
%%% TeX-engine: luatex
%%% TeX-master: shared
%%% End:

% //
%\endgl \xe

%  {
%  \relsize{-1}
 
% \exampleBibExtra{IV:26}

% \bigskip

% \exampleDesc{MICHAEL VRATISLAVIENSIS: Introductorium astronomiae Cracoviense almanach.
% Kraków, Jan Haller, 29. V. 1507. 4⁰.}




% \medskip
% \examplePage{\textit{Karta A₆b}}

%   \bigskip
% \exampleLib{Biblioteka Zakl. Nar. im. Ossolińskich. Wrocław}


% \bigskip
% \exampleRef{\textit{Estreicher XXXIII. 356.}}

% % https://polona2.pl/item/introductoriu-m-astronomie-cracouiense-elucidans-almanach,NDQzMzg2MDM/0/#info:metadata
% % 1517
% % https://dbc.wroc.pl/dlibra/publication/8925/edition/8046?language=pl
% % 1507!!!
% % https://wbc.poznan.pl/dlibra/publication/402464/edition/314800
% % 1506


% % \bigskip
% % \exampleDig{\url{https://dbc.wroc.pl/dlibra/publication/8925}
% %   page brak?!}

% \bigskip

% \examplePL{Znaki Zodiaku. — Pismo 5. — Rubryka \beta{} z pismem 5. — Cyfry 2.}

%     \medskip

%     \exampleEN{Znaki Zodiaku. — Font 5. — Rubric \beta{} with font 5. — Digits 2}
%   }
  

 \newpage

%%%%%%%%%%%%%%%%%%%%%%%%%%%%%%%%%%%%%%%%%%%%%%%%%%%%%%%%%%%%%%%%%%%%%%%%%%%%%%% 
 % Tab. 32,Haller-13_PT04_172.djvu,Haller,13,04,172
%%%%%%%%%%%%%%%%%%%%%%%%%%%%%%%%%%%%%%%%%%%%%%%%%%%%%%%%%%%%%%%%%%%%%%%%%%%%%%%

% Note "13.  Pismo tekstowe. Krój M¹⁸. Stopień 20 ww. = 112 mm (interliniowane). — Tabl. 172."
% Note1 "Character set table prepared by Maria Błońska"

  \pismoPL{Jan Haller 13.  Pismo tekstowe. Krój M¹⁸. Stopień 20 ww. = 112 mm (interliniowane). — Tabl. 172 [drugi zestaw].}
  
  \pismoEN{Jan Haller 13. Text font. Typeface M¹⁸. Type size 20 ww. =
    112 mm (with extra leading). — Plate 172 [second set].}

\plate{172[2]}{IV}{1962}

The plate prepared by Helena Kapełuś.\\
The font table prepared by Helena Kapełuś and Maria Błońska.


\examplePL{Pismo 13: drugi zestaw. — Cyfry 9: z pismem 13.}

    \medskip

    \exampleEN{Font 13. — second character set,  — Digits 9: with font 13.}


\bigskip


\fontID{Ha-13}{32}

\fontstat{96}

% \exdisplay \bg \gla
 \exdisplay \bg \gla
% 1
{\PTglyph{5}{t32_l01g01.png}}
% 2
{\PTglyph{5}{t32_l01g02.png}}
% 3
{\PTglyph{5}{t32_l01g03.png}}
% 4
{\PTglyph{5}{t32_l01g04.png}}
% 5
{\PTglyph{5}{t32_l01g05.png}}
% 6
{\PTglyph{5}{t32_l01g06.png}}
% 7
{\PTglyph{5}{t32_l01g07.png}}
% 8
{\PTglyph{5}{t32_l01g08.png}}
% 9
{\PTglyph{5}{t32_l01g09.png}}
% 10
{\PTglyph{5}{t32_l01g10.png}}
% 11
{\PTglyph{5}{t32_l01g11.png}}
% 12
{\PTglyph{5}{t32_l01g12.png}}
% 13
{\PTglyph{5}{t32_l01g13.png}}
% 14
{\PTglyph{5}{t32_l02g01.png}}
% 15
{\PTglyph{5}{t32_l02g02.png}}
% 16
{\PTglyph{5}{t32_l02g03.png}}
% 17
{\PTglyph{5}{t32_l02g04.png}}
% 18
{\PTglyph{5}{t32_l02g05.png}}
% 19
{\PTglyph{5}{t32_l02g06.png}}
% 20
{\PTglyph{5}{t32_l02g07.png}}
% 21
{\PTglyph{5}{t32_l02g08.png}}
% 22
{\PTglyph{5}{t32_l02g09.png}}
% 23
{\PTglyph{5}{t32_l02g10.png}}
% 24
{\PTglyph{5}{t32_l02g11.png}}
% 25
{\PTglyph{5}{t32_l02g12.png}}
% 26
{\PTglyph{5}{t32_l02g13.png}}
% 27
{\PTglyph{5}{t32_l02g14.png}}
% 28
{\PTglyph{5}{t32_l02g15.png}}
% 29
{\PTglyph{5}{t32_l02g16.png}}
% 30
{\PTglyph{5}{t32_l02g17.png}}
% 31
{\PTglyph{5}{t32_l02g18.png}}
% 32
{\PTglyph{5}{t32_l02g19.png}}
% 33
{\PTglyph{5}{t32_l02g20.png}}
% 34
{\PTglyph{5}{t32_l02g21.png}}
% 35
{\PTglyph{5}{t32_l02g22.png}}
% 36
{\PTglyph{5}{t32_l02g23.png}}
% 37
{\PTglyph{5}{t32_l02g24.png}}
% 38
{\PTglyph{5}{t32_l02g25.png}}
% 39
{\PTglyph{5}{t32_l02g26.png}}
% 40
{\PTglyph{5}{t32_l02g27.png}}
% 41
{\PTglyph{5}{t32_l02g28.png}}
% 42
{\PTglyph{5}{t32_l02g29.png}}
% 43
{\PTglyph{5}{t32_l02g30.png}}
% 44
{\PTglyph{5}{t32_l02g31.png}}
% 45
{\PTglyph{5}{t32_l02g32.png}}
% 46
{\PTglyph{5}{t32_l02g33.png}}
% 47
{\PTglyph{5}{t32_l02g34.png}}
% 48
{\PTglyph{5}{t32_l02g35.png}}
% 49
{\PTglyph{5}{t32_l02g36.png}}
% 50
{\PTglyph{5}{t32_l03g01.png}}
% 51
{\PTglyph{5}{t32_l03g02.png}}
% 52
{\PTglyph{5}{t32_l03g03.png}}
% 53
{\PTglyph{5}{t32_l03g04.png}}
% 54
{\PTglyph{5}{t32_l03g05.png}}
% 55
{\PTglyph{5}{t32_l03g06.png}}
% 56
{\PTglyph{5}{t32_l03g07.png}}
% 57
{\PTglyph{5}{t32_l03g08.png}}
% 58
{\PTglyph{5}{t32_l03g09.png}}
% 59
{\PTglyph{5}{t32_l03g10.png}}
% 60
{\PTglyph{5}{t32_l03g11.png}}
% 61
{\PTglyph{5}{t32_l03g12.png}}
% 62
{\PTglyph{5}{t32_l03g13.png}}
% 63
{\PTglyph{5}{t32_l03g14.png}}
% 64
{\PTglyph{5}{t32_l03g15.png}}
% 65
{\PTglyph{5}{t32_l03g16.png}}
% 66
{\PTglyph{5}{t32_l03g17.png}}
% 67
{\PTglyph{5}{t32_l03g18.png}}
% 68
{\PTglyph{5}{t32_l03g19.png}}
% 69
{\PTglyph{5}{t32_l03g20.png}}
% 70
{\PTglyph{5}{t32_l03g21.png}}
% 71
{\PTglyph{5}{t32_l03g22.png}}
% 72
{\PTglyph{5}{t32_l03g23.png}}
% 73
{\PTglyph{5}{t32_l03g24.png}}
% 74
{\PTglyph{5}{t32_l03g25.png}}
% 75
{\PTglyph{5}{t32_l03g26.png}}
% 76
{\PTglyph{5}{t32_l03g27.png}}
% 77
{\PTglyph{5}{t32_l03g28.png}}
% 78
{\PTglyph{5}{t32_l03g29.png}}
% 79
{\PTglyph{5}{t32_l03g30.png}}
% 80
{\PTglyph{5}{t32_l03g31.png}}
% 81
{\PTglyph{5}{t32_l03g32.png}}
% 82
{\PTglyph{5}{t32_l03g33.png}}
% 83
{\PTglyph{5}{t32_l03g34.png}}
% 84
{\PTglyph{5}{t32_l03g35.png}}
% 85
{\PTglyph{5}{t32_l03g36.png}}
% 86
{\PTglyph{5}{t32_l03g37.png}}
% 87
{\PTglyph{5}{t32_l04g01.png}}
% 88
{\PTglyph{5}{t32_l04g02.png}}
% 89
{\PTglyph{5}{t32_l04g03.png}}
% 90
{\PTglyph{5}{t32_l04g04.png}}
% 91
{\PTglyph{5}{t32_l04g05.png}}
% 92
{\PTglyph{5}{t32_l04g06.png}}
% 93
{\PTglyph{5}{t32_l04g07.png}}
% 94
{\PTglyph{5}{t32_l04g08.png}}
% 95
{\PTglyph{5}{t32_l04g09.png}}
% 96
{\PTglyph{5}{t32_l04g10.png}}
//
%%% Local Variables:
%%% mode: latex
%%% TeX-engine: luatex
%%% TeX-master: shared
%%% End:

%//
%\glpismo%
 \glpismo
% 1
{\PTglyphid{Ha-13_0101}}
% 2
{\PTglyphid{Ha-13_0102}}
% 3
{\PTglyphid{Ha-13_0103}}
% 4
{\PTglyphid{Ha-13_0104}}
% 5
{\PTglyphid{Ha-13_0105}}
% 6
{\PTglyphid{Ha-13_0106}}
% 7
{\PTglyphid{Ha-13_0107}}
% 8
{\PTglyphid{Ha-13_0108}}
% 9
{\PTglyphid{Ha-13_0109}}
% 10
{\PTglyphid{Ha-13_0110}}
% 11
{\PTglyphid{Ha-13_0111}}
% 12
{\PTglyphid{Ha-13_0112}}
% 13
{\PTglyphid{Ha-13_0113}}
% 14
{\PTglyphid{Ha-13_0201}}
% 15
{\PTglyphid{Ha-13_0202}}
% 16
{\PTglyphid{Ha-13_0203}}
% 17
{\PTglyphid{Ha-13_0204}}
% 18
{\PTglyphid{Ha-13_0205}}
% 19
{\PTglyphid{Ha-13_0206}}
% 20
{\PTglyphid{Ha-13_0207}}
% 21
{\PTglyphid{Ha-13_0208}}
% 22
{\PTglyphid{Ha-13_0209}}
% 23
{\PTglyphid{Ha-13_0210}}
% 24
{\PTglyphid{Ha-13_0211}}
% 25
{\PTglyphid{Ha-13_0212}}
% 26
{\PTglyphid{Ha-13_0213}}
% 27
{\PTglyphid{Ha-13_0214}}
% 28
{\PTglyphid{Ha-13_0215}}
% 29
{\PTglyphid{Ha-13_0216}}
% 30
{\PTglyphid{Ha-13_0217}}
% 31
{\PTglyphid{Ha-13_0218}}
% 32
{\PTglyphid{Ha-13_0219}}
% 33
{\PTglyphid{Ha-13_0220}}
% 34
{\PTglyphid{Ha-13_0221}}
% 35
{\PTglyphid{Ha-13_0222}}
% 36
{\PTglyphid{Ha-13_0223}}
% 37
{\PTglyphid{Ha-13_0224}}
% 38
{\PTglyphid{Ha-13_0225}}
% 39
{\PTglyphid{Ha-13_0226}}
% 40
{\PTglyphid{Ha-13_0227}}
% 41
{\PTglyphid{Ha-13_0228}}
% 42
{\PTglyphid{Ha-13_0229}}
% 43
{\PTglyphid{Ha-13_0230}}
% 44
{\PTglyphid{Ha-13_0231}}
% 45
{\PTglyphid{Ha-13_0232}}
% 46
{\PTglyphid{Ha-13_0233}}
% 47
{\PTglyphid{Ha-13_0234}}
% 48
{\PTglyphid{Ha-13_0235}}
% 49
{\PTglyphid{Ha-13_0236}}
% 50
{\PTglyphid{Ha-13_0301}}
% 51
{\PTglyphid{Ha-13_0302}}
% 52
{\PTglyphid{Ha-13_0303}}
% 53
{\PTglyphid{Ha-13_0304}}
% 54
{\PTglyphid{Ha-13_0305}}
% 55
{\PTglyphid{Ha-13_0306}}
% 56
{\PTglyphid{Ha-13_0307}}
% 57
{\PTglyphid{Ha-13_0308}}
% 58
{\PTglyphid{Ha-13_0309}}
% 59
{\PTglyphid{Ha-13_0310}}
% 60
{\PTglyphid{Ha-13_0311}}
% 61
{\PTglyphid{Ha-13_0312}}
% 62
{\PTglyphid{Ha-13_0313}}
% 63
{\PTglyphid{Ha-13_0314}}
% 64
{\PTglyphid{Ha-13_0315}}
% 65
{\PTglyphid{Ha-13_0316}}
% 66
{\PTglyphid{Ha-13_0317}}
% 67
{\PTglyphid{Ha-13_0318}}
% 68
{\PTglyphid{Ha-13_0319}}
% 69
{\PTglyphid{Ha-13_0320}}
% 70
{\PTglyphid{Ha-13_0321}}
% 71
{\PTglyphid{Ha-13_0322}}
% 72
{\PTglyphid{Ha-13_0323}}
% 73
{\PTglyphid{Ha-13_0324}}
% 74
{\PTglyphid{Ha-13_0325}}
% 75
{\PTglyphid{Ha-13_0326}}
% 76
{\PTglyphid{Ha-13_0327}}
% 77
{\PTglyphid{Ha-13_0328}}
% 78
{\PTglyphid{Ha-13_0329}}
% 79
{\PTglyphid{Ha-13_0330}}
% 80
{\PTglyphid{Ha-13_0331}}
% 81
{\PTglyphid{Ha-13_0332}}
% 82
{\PTglyphid{Ha-13_0333}}
% 83
{\PTglyphid{Ha-13_0334}}
% 84
{\PTglyphid{Ha-13_0335}}
% 85
{\PTglyphid{Ha-13_0336}}
% 86
{\PTglyphid{Ha-13_0337}}
% 87
{\PTglyphid{Ha-13_0401}}
% 88
{\PTglyphid{Ha-13_0402}}
% 89
{\PTglyphid{Ha-13_0403}}
% 90
{\PTglyphid{Ha-13_0404}}
% 91
{\PTglyphid{Ha-13_0405}}
% 92
{\PTglyphid{Ha-13_0406}}
% 93
{\PTglyphid{Ha-13_0407}}
% 94
{\PTglyphid{Ha-13_0408}}
% 95
{\PTglyphid{Ha-13_0409}}
% 96
{\PTglyphid{Ha-13_0410}}
//
\endgl \xe
%%% Local Variables:
%%% mode: latex
%%% TeX-engine: luatex
%%% TeX-master: shared
%%% End:

% //
%\endgl \xe


 
\newpage

%%%%%%%%%%%%%%%%%%%%%%%%%%%%%%%%%%%%%%%%%%%%%%%%%%%%%%%%%%%%%%%%%%%%%%%%%%%%%%% 
 % Tab. 33,Hochfeder-02_PT01_020bis.djvu,Hochfeder,02,01,020
%%%%%%%%%%%%%%%%%%%%%%%%%%%%%%%%%%%%%%%%%%%%%%%%%%%%%%%%%%%%%%%%%%%%%%%%%%%%%%%

% Fascicule "I"
% Edition "Wydanie II przejrzane i uzupełnione przez Marię Błońską"
% Publisher "Instytut Badań Literackich Polskiej Akademii Nauk — Biblioteka Narodowa"
% Addres "Warszawa"
% Year "1968"
% Note "2. Pismo nagłówkowe gotyckie. Krój M⁸⁹. Stopień 20 ww. = 111/113 mm - Tabl. 20bis. (Występuje u Hallera jako pismo 4. Tabl. 167) [20bis]"
% Note1 "Character set table prepared by Maria Błońska and Anna Wolińska"

\pismoPL{Kasper Hochfeder 2. Pismo nagłówkowe gotyckie. Krój
  M⁸⁹. Stopień 20 ww. = 111/113 mm - Tabl. 20bis. (Występuje u Hallera
  jako pismo 4. Tabl. 167)}

  
\pismoEN{Kasper Hochfeder 2. Gothic header font. Typeface M⁸⁹. Type
  size 20 lines = 111/113 mm. (Used by Haller as font 4. Plate 167)}

\plate{20bis}{I}{1968}

Prepared by Kazimierz Piekarski and  Maria Błońska.\\
The font table prepared by Maria Błońska and Anna Wolińska.

\bigskip

\exampleBib{I:8}

\bigskip \exampleDesc{HESIODUS: Georgica. Kraków, Kasper Hochfeder [nakładem Jana Hallera], 1 VII 1505. 4⁰}

\medskip
\examplePage{\textit{Karta a₀a}}

  \bigskip
\exampleLib{Biblioteka Narodowa. Warszawa.}

\bigskip
\exampleRef{\textit{Estreicher XVIII 168. Wierzbowski 836.}}
  
  % \medskip
\bigskip

    \examplePL{Pismo 2: tekst i zestaw.}
    
    \medskip

    \exampleEN{Font 2: text and font repertoire}


\bigskip

% Scharfenberg:
% https://kpbc.umk.pl/dlibra/publication/258387/edition/268632?language=pl
% Lipsk:
% https://cyfrowe.mnk.pl/dlibra/publication/25371/edition/25054?language=pl

\fontID{Ho-02}{33}

\fontstat{112}

% \exdisplay \bg \gla
 \exdisplay \bg \gla
% 1
{\PTglyph{5}{t33_l01g01.png}}
% 2
{\PTglyph{5}{t33_l01g02.png}}
% 3
{\PTglyph{5}{t33_l01g03.png}}
% 4
{\PTglyph{5}{t33_l01g04.png}}
% 5
{\PTglyph{5}{t33_l01g05.png}}
% 6
{\PTglyph{5}{t33_l01g06.png}}
% 7
{\PTglyph{5}{t33_l01g07.png}}
% 8
{\PTglyph{5}{t33_l01g08.png}}
% 9
{\PTglyph{5}{t33_l01g09.png}}
% 10
{\PTglyph{5}{t33_l01g10.png}}
% 11
{\PTglyph{5}{t33_l01g11.png}}
% 12
{\PTglyph{5}{t33_l01g12.png}}
% 13
{\PTglyph{5}{t33_l01g13.png}}
% 14
{\PTglyph{5}{t33_l01g14.png}}
% 15
{\PTglyph{5}{t33_l01g15.png}}
% 16
{\PTglyph{5}{t33_l01g16.png}}
% 17
{\PTglyph{5}{t33_l01g17.png}}
% 18
{\PTglyph{5}{t33_l01g18.png}}
% 19
{\PTglyph{5}{t33_l01g19.png}}
% 20
{\PTglyph{5}{t33_l01g20.png}}
% 21
{\PTglyph{5}{t33_l01g21.png}}
% 22
{\PTglyph{5}{t33_l01g22.png}}
% 23
{\PTglyph{5}{t33_l01g23.png}}
% 24
{\PTglyph{5}{t33_l02g01.png}}
% 25
{\PTglyph{5}{t33_l02g02.png}}
% 26
{\PTglyph{5}{t33_l02g03.png}}
% 27
{\PTglyph{5}{t33_l02g04.png}}
% 28
{\PTglyph{5}{t33_l02g05.png}}
% 29
{\PTglyph{5}{t33_l02g06.png}}
% 30
{\PTglyph{5}{t33_l02g07.png}}
% 31
{\PTglyph{5}{t33_l02g08.png}}
% 32
{\PTglyph{5}{t33_l02g09.png}}
% 33
{\PTglyph{5}{t33_l02g10.png}}
% 34
{\PTglyph{5}{t33_l02g11.png}}
% 35
{\PTglyph{5}{t33_l02g12.png}}
% 36
{\PTglyph{5}{t33_l02g13.png}}
% 37
{\PTglyph{5}{t33_l02g14.png}}
% 38
{\PTglyph{5}{t33_l02g15.png}}
% 39
{\PTglyph{5}{t33_l02g16.png}}
% 40
{\PTglyph{5}{t33_l02g17.png}}
% 41
{\PTglyph{5}{t33_l02g18.png}}
% 42
{\PTglyph{5}{t33_l02g19.png}}
% 43
{\PTglyph{5}{t33_l02g20.png}}
% 44
{\PTglyph{5}{t33_l02g21.png}}
% 45
{\PTglyph{5}{t33_l02g22.png}}
% 46
{\PTglyph{5}{t33_l02g23.png}}
% 47
{\PTglyph{5}{t33_l02g24.png}}
% 48
{\PTglyph{5}{t33_l02g25.png}}
% 49
{\PTglyph{5}{t33_l02g26.png}}
% 50
{\PTglyph{5}{t33_l02g27.png}}
% 51
{\PTglyph{5}{t33_l02g28.png}}
% 52
{\PTglyph{5}{t33_l02g29.png}}
% 53
{\PTglyph{5}{t33_l02g30.png}}
% 54
{\PTglyph{5}{t33_l02g31.png}}
% 55
{\PTglyph{5}{t33_l02g32.png}}
% 56
{\PTglyph{5}{t33_l02g33.png}}
% 57
{\PTglyph{5}{t33_l02g34.png}}
% 58
{\PTglyph{5}{t33_l02g35.png}}
% 59
{\PTglyph{5}{t33_l02g36.png}}
% 60
{\PTglyph{5}{t33_l02g37.png}}
% 61
{\PTglyph{5}{t33_l02g38.png}}
% 62
{\PTglyph{5}{t33_l02g39.png}}
% 63
{\PTglyph{5}{t33_l03g01.png}}
% 64
{\PTglyph{5}{t33_l03g02.png}}
% 65
{\PTglyph{5}{t33_l03g03.png}}
% 66
{\PTglyph{5}{t33_l03g04.png}}
% 67
{\PTglyph{5}{t33_l03g05.png}}
% 68
{\PTglyph{5}{t33_l03g06.png}}
% 69
{\PTglyph{5}{t33_l03g07.png}}
% 70
{\PTglyph{5}{t33_l03g08.png}}
% 71
{\PTglyph{5}{t33_l03g09.png}}
% 72
{\PTglyph{5}{t33_l03g10.png}}
% 73
{\PTglyph{5}{t33_l03g11.png}}
% 74
{\PTglyph{5}{t33_l03g12.png}}
% 75
{\PTglyph{5}{t33_l03g13.png}}
% 76
{\PTglyph{5}{t33_l03g14.png}}
% 77
{\PTglyph{5}{t33_l03g15.png}}
% 78
{\PTglyph{5}{t33_l03g16.png}}
% 79
{\PTglyph{5}{t33_l03g17.png}}
% 80
{\PTglyph{5}{t33_l03g18.png}}
% 81
{\PTglyph{5}{t33_l03g19.png}}
% 82
{\PTglyph{5}{t33_l03g20.png}}
% 83
{\PTglyph{5}{t33_l03g21.png}}
% 84
{\PTglyph{5}{t33_l03g22.png}}
% 85
{\PTglyph{5}{t33_l03g23.png}}
% 86
{\PTglyph{5}{t33_l03g24.png}}
% 87
{\PTglyph{5}{t33_l03g25.png}}
% 88
{\PTglyph{5}{t33_l03g26.png}}
% 89
{\PTglyph{5}{t33_l03g27.png}}
% 90
{\PTglyph{5}{t33_l03g28.png}}
% 91
{\PTglyph{5}{t33_l03g29.png}}
% 92
{\PTglyph{5}{t33_l03g30.png}}
% 93
{\PTglyph{5}{t33_l03g31.png}}
% 94
{\PTglyph{5}{t33_l03g32.png}}
% 95
{\PTglyph{5}{t33_l03g33.png}}
% 96
{\PTglyph{5}{t33_l03g34.png}}
% 97
{\PTglyph{5}{t33_l03g35.png}}
% 98
{\PTglyph{5}{t33_l03g36.png}}
% 99
{\PTglyph{5}{t33_l03g37.png}}
% 100
{\PTglyph{5}{t33_l04g01.png}}
% 101
{\PTglyph{5}{t33_l04g02.png}}
% 102
{\PTglyph{5}{t33_l04g03.png}}
% 103
{\PTglyph{5}{t33_l04g04.png}}
% 104
{\PTglyph{5}{t33_l04g05.png}}
% 105
{\PTglyph{5}{t33_l04g06.png}}
% 106
{\PTglyph{5}{t33_l04g07.png}}
% 107
{\PTglyph{5}{t33_l04g08.png}}
% 108
{\PTglyph{5}{t33_l04g09.png}}
% 109
{\PTglyph{5}{t33_l04g10.png}}
% 110
{\PTglyph{5}{t33_l04g11.png}}
% 111
{\PTglyph{5}{t33_l04g12.png}}
% 112
{\PTglyph{5}{t33_l04g13.png}}
//
%%% Local Variables:
%%% mode: latex
%%% TeX-engine: luatex
%%% TeX-master: shared
%%% End:

%//
%\glpismo%
 \glpismo
% 1
{\PTglyphid{Ho-02_0101}}
% 2
{\PTglyphid{Ho-02_0102}}
% 3
{\PTglyphid{Ho-02_0103}}
% 4
{\PTglyphid{Ho-02_0104}}
% 5
{\PTglyphid{Ho-02_0105}}
% 6
{\PTglyphid{Ho-02_0106}}
% 7
{\PTglyphid{Ho-02_0107}}
% 8
{\PTglyphid{Ho-02_0108}}
% 9
{\PTglyphid{Ho-02_0109}}
% 10
{\PTglyphid{Ho-02_0110}}
% 11
{\PTglyphid{Ho-02_0111}}
% 12
{\PTglyphid{Ho-02_0112}}
% 13
{\PTglyphid{Ho-02_0113}}
% 14
{\PTglyphid{Ho-02_0114}}
% 15
{\PTglyphid{Ho-02_0115}}
% 16
{\PTglyphid{Ho-02_0116}}
% 17
{\PTglyphid{Ho-02_0117}}
% 18
{\PTglyphid{Ho-02_0118}}
% 19
{\PTglyphid{Ho-02_0119}}
% 20
{\PTglyphid{Ho-02_0120}}
% 21
{\PTglyphid{Ho-02_0121}}
% 22
{\PTglyphid{Ho-02_0122}}
% 23
{\PTglyphid{Ho-02_0123}}
% 24
{\PTglyphid{Ho-02_0201}}
% 25
{\PTglyphid{Ho-02_0202}}
% 26
{\PTglyphid{Ho-02_0203}}
% 27
{\PTglyphid{Ho-02_0204}}
% 28
{\PTglyphid{Ho-02_0205}}
% 29
{\PTglyphid{Ho-02_0206}}
% 30
{\PTglyphid{Ho-02_0207}}
% 31
{\PTglyphid{Ho-02_0208}}
% 32
{\PTglyphid{Ho-02_0209}}
% 33
{\PTglyphid{Ho-02_0210}}
% 34
{\PTglyphid{Ho-02_0211}}
% 35
{\PTglyphid{Ho-02_0212}}
% 36
{\PTglyphid{Ho-02_0213}}
% 37
{\PTglyphid{Ho-02_0214}}
% 38
{\PTglyphid{Ho-02_0215}}
% 39
{\PTglyphid{Ho-02_0216}}
% 40
{\PTglyphid{Ho-02_0217}}
% 41
{\PTglyphid{Ho-02_0218}}
% 42
{\PTglyphid{Ho-02_0219}}
% 43
{\PTglyphid{Ho-02_0220}}
% 44
{\PTglyphid{Ho-02_0221}}
% 45
{\PTglyphid{Ho-02_0222}}
% 46
{\PTglyphid{Ho-02_0223}}
% 47
{\PTglyphid{Ho-02_0224}}
% 48
{\PTglyphid{Ho-02_0225}}
% 49
{\PTglyphid{Ho-02_0226}}
% 50
{\PTglyphid{Ho-02_0227}}
% 51
{\PTglyphid{Ho-02_0228}}
% 52
{\PTglyphid{Ho-02_0229}}
% 53
{\PTglyphid{Ho-02_0230}}
% 54
{\PTglyphid{Ho-02_0231}}
% 55
{\PTglyphid{Ho-02_0232}}
% 56
{\PTglyphid{Ho-02_0233}}
% 57
{\PTglyphid{Ho-02_0234}}
% 58
{\PTglyphid{Ho-02_0235}}
% 59
{\PTglyphid{Ho-02_0236}}
% 60
{\PTglyphid{Ho-02_0237}}
% 61
{\PTglyphid{Ho-02_0238}}
% 62
{\PTglyphid{Ho-02_0239}}
% 63
{\PTglyphid{Ho-02_0301}}
% 64
{\PTglyphid{Ho-02_0302}}
% 65
{\PTglyphid{Ho-02_0303}}
% 66
{\PTglyphid{Ho-02_0304}}
% 67
{\PTglyphid{Ho-02_0305}}
% 68
{\PTglyphid{Ho-02_0306}}
% 69
{\PTglyphid{Ho-02_0307}}
% 70
{\PTglyphid{Ho-02_0308}}
% 71
{\PTglyphid{Ho-02_0309}}
% 72
{\PTglyphid{Ho-02_0310}}
% 73
{\PTglyphid{Ho-02_0311}}
% 74
{\PTglyphid{Ho-02_0312}}
% 75
{\PTglyphid{Ho-02_0313}}
% 76
{\PTglyphid{Ho-02_0314}}
% 77
{\PTglyphid{Ho-02_0315}}
% 78
{\PTglyphid{Ho-02_0316}}
% 79
{\PTglyphid{Ho-02_0317}}
% 80
{\PTglyphid{Ho-02_0318}}
% 81
{\PTglyphid{Ho-02_0319}}
% 82
{\PTglyphid{Ho-02_0320}}
% 83
{\PTglyphid{Ho-02_0321}}
% 84
{\PTglyphid{Ho-02_0322}}
% 85
{\PTglyphid{Ho-02_0323}}
% 86
{\PTglyphid{Ho-02_0324}}
% 87
{\PTglyphid{Ho-02_0325}}
% 88
{\PTglyphid{Ho-02_0326}}
% 89
{\PTglyphid{Ho-02_0327}}
% 90
{\PTglyphid{Ho-02_0328}}
% 91
{\PTglyphid{Ho-02_0329}}
% 92
{\PTglyphid{Ho-02_0330}}
% 93
{\PTglyphid{Ho-02_0331}}
% 94
{\PTglyphid{Ho-02_0332}}
% 95
{\PTglyphid{Ho-02_0333}}
% 96
{\PTglyphid{Ho-02_0334}}
% 97
{\PTglyphid{Ho-02_0335}}
% 98
{\PTglyphid{Ho-02_0336}}
% 99
{\PTglyphid{Ho-02_0337}}
% 100
{\PTglyphid{Ho-02_0401}}
% 101
{\PTglyphid{Ho-02_0402}}
% 102
{\PTglyphid{Ho-02_0403}}
% 103
{\PTglyphid{Ho-02_0404}}
% 104
{\PTglyphid{Ho-02_0405}}
% 105
{\PTglyphid{Ho-02_0406}}
% 106
{\PTglyphid{Ho-02_0407}}
% 107
{\PTglyphid{Ho-02_0408}}
% 108
{\PTglyphid{Ho-02_0409}}
% 109
{\PTglyphid{Ho-02_0410}}
% 110
{\PTglyphid{Ho-02_0411}}
% 111
{\PTglyphid{Ho-02_0412}}
% 112
{\PTglyphid{Ho-02_0413}}
//
\endgl \xe
%%% Local Variables:
%%% mode: latex
%%% TeX-engine: luatex
%%% TeX-master: shared
%%% End:

% //
%\endgl \xe

  \end{document}



\newpage

%%%%%%%%%%%%%%%%%%%%%%%%%%%%%%%%%%%%%%%%%%%%%%%%%%%%%%%%%%%%%%%%%%%%%%%%%%%%%%% 
 % Tab. 34,Hochfeder-03_PT01_021.djvu,Hochfeder,03,01,021
%%%%%%%%%%%%%%%%%%%%%%%%%%%%%%%%%%%%%%%%%%%%%%%%%%%%%%%%%%%%%%%%%%%%%%%%%%%%%%%


% Note "3. Pismo tekstowe gotyckie. Krój M⁸⁸. Stopień 20 ww. = 76/77 mm - Tabl. 21. (Występuje u Hallera jako pismo 7. Tabl. 170) [21]"
% Note1 "Character set table prepared by Maria Błońska and Anna Wolińska"

\pismoPL{Kasper Hochfeder 3. Pismo tekstowe gotyckie. Krój
  M⁸⁸. Stopień 20 ww. = 76/77 mm - Tabl. 21. (Występuje u Hallera
  jako pismo 7. Tabl. 170)}

  
\pismoEN{Kasper Hochfeder 3. Gothic text font. Typeface M⁸⁸. Type
  size 20 lines = 76/77 mm. (Used by Haller as font 7. Plate 170)}

\plate{21}{I}{1968}

Prepared by Kazimierz Piekarski and  Maria Błońska.\\
The font table prepared by Maria Błońska and Anna Wolińska.

\bigskip

\exampleBib{I:8}

\bigskip \exampleDesc{IOANNES GLOGOVIENSIS: Minoris Donati
  interpretatio. Kraków, [Kasper Hochfeder] nakładem Jana Hallera,
  1503. 4⁰.}


\medskip
\examplePage{\textit{Karta a₅b}}

  \bigskip
\exampleLib{Biblioteka Jagiellońska. Kraków.}


\bigskip
\exampleRef{\textit{Estreicher XVII 175. Wierzbowski 2040.}}

  
  % \medskip
\bigskip

\examplePL{Pismo 3.  Rubryki \beta{}, \alfa{}. — Cyfry 1, 2.}

    
    \medskip

    \exampleEN{Font  3.  Rubrics \beta{}, \alfa{}. — Digits 1, 2.}


\bigskip

% Scharfenberg:
% https://kpbc.umk.pl/dlibra/publication/258387/edition/268632?language=pl
% Lipsk:
% https://cyfrowe.mnk.pl/dlibra/publication/25371/edition/25054?language=pl

\fontID{Ho-03}{34}

\fontstat{124}

% \exdisplay \bg \gla
 \exdisplay \bg \gla
% 1
{\PTglyph{5}{t34_l01g01.png}}
% 2
{\PTglyph{5}{t34_l01g02.png}}
% 3
{\PTglyph{5}{t34_l01g03.png}}
% 4
{\PTglyph{5}{t34_l01g04.png}}
% 5
{\PTglyph{5}{t34_l01g05.png}}
% 6
{\PTglyph{5}{t34_l01g06.png}}
% 7
{\PTglyph{5}{t34_l01g07.png}}
% 8
{\PTglyph{5}{t34_l01g08.png}}
% 9
{\PTglyph{5}{t34_l01g09.png}}
% 10
{\PTglyph{5}{t34_l01g10.png}}
% 11
{\PTglyph{5}{t34_l01g11.png}}
% 12
{\PTglyph{5}{t34_l01g12.png}}
% 13
{\PTglyph{5}{t34_l01g13.png}}
% 14
{\PTglyph{5}{t34_l01g14.png}}
% 15
{\PTglyph{5}{t34_l01g15.png}}
% 16
{\PTglyph{5}{t34_l01g16.png}}
% 17
{\PTglyph{5}{t34_l01g17.png}}
% 18
{\PTglyph{5}{t34_l01g18.png}}
% 19
{\PTglyph{5}{t34_l01g19.png}}
% 20
{\PTglyph{5}{t34_l01g20.png}}
% 21
{\PTglyph{5}{t34_l02g01.png}}
% 22
{\PTglyph{5}{t34_l02g02.png}}
% 23
{\PTglyph{5}{t34_l02g03.png}}
% 24
{\PTglyph{5}{t34_l02g04.png}}
% 25
{\PTglyph{5}{t34_l02g05.png}}
% 26
{\PTglyph{5}{t34_l02g06.png}}
% 27
{\PTglyph{5}{t34_l02g07.png}}
% 28
{\PTglyph{5}{t34_l02g08.png}}
% 29
{\PTglyph{5}{t34_l02g09.png}}
% 30
{\PTglyph{5}{t34_l02g10.png}}
% 31
{\PTglyph{5}{t34_l02g11.png}}
% 32
{\PTglyph{5}{t34_l02g12.png}}
% 33
{\PTglyph{5}{t34_l02g13.png}}
% 34
{\PTglyph{5}{t34_l02g14.png}}
% 35
{\PTglyph{5}{t34_l02g15.png}}
% 36
{\PTglyph{5}{t34_l02g16.png}}
% 37
{\PTglyph{5}{t34_l02g17.png}}
% 38
{\PTglyph{5}{t34_l02g18.png}}
% 39
{\PTglyph{5}{t34_l02g19.png}}
% 40
{\PTglyph{5}{t34_l02g20.png}}
% 41
{\PTglyph{5}{t34_l02g21.png}}
% 42
{\PTglyph{5}{t34_l02g22.png}}
% 43
{\PTglyph{5}{t34_l02g23.png}}
% 44
{\PTglyph{5}{t34_l02g24.png}}
% 45
{\PTglyph{5}{t34_l02g25.png}}
% 46
{\PTglyph{5}{t34_l02g26.png}}
% 47
{\PTglyph{5}{t34_l02g27.png}}
% 48
{\PTglyph{5}{t34_l02g28.png}}
% 49
{\PTglyph{5}{t34_l02g29.png}}
% 50
{\PTglyph{5}{t34_l02g30.png}}
% 51
{\PTglyph{5}{t34_l02g31.png}}
% 52
{\PTglyph{5}{t34_l02g32.png}}
% 53
{\PTglyph{5}{t34_l02g33.png}}
% 54
{\PTglyph{5}{t34_l02g34.png}}
% 55
{\PTglyph{5}{t34_l02g35.png}}
% 56
{\PTglyph{5}{t34_l02g36.png}}
% 57
{\PTglyph{5}{t34_l03g01.png}}
% 58
{\PTglyph{5}{t34_l03g02.png}}
% 59
{\PTglyph{5}{t34_l03g03.png}}
% 60
{\PTglyph{5}{t34_l03g04.png}}
% 61
{\PTglyph{5}{t34_l03g05.png}}
% 62
{\PTglyph{5}{t34_l03g06.png}}
% 63
{\PTglyph{5}{t34_l03g07.png}}
% 64
{\PTglyph{5}{t34_l03g08.png}}
% 65
{\PTglyph{5}{t34_l03g09.png}}
% 66
{\PTglyph{5}{t34_l03g10.png}}
% 67
{\PTglyph{5}{t34_l03g11.png}}
% 68
{\PTglyph{5}{t34_l03g12.png}}
% 69
{\PTglyph{5}{t34_l03g13.png}}
% 70
{\PTglyph{5}{t34_l03g14.png}}
% 71
{\PTglyph{5}{t34_l03g15.png}}
% 72
{\PTglyph{5}{t34_l03g16.png}}
% 73
{\PTglyph{5}{t34_l03g17.png}}
% 74
{\PTglyph{5}{t34_l03g18.png}}
% 75
{\PTglyph{5}{t34_l03g19.png}}
% 76
{\PTglyph{5}{t34_l03g20.png}}
% 77
{\PTglyph{5}{t34_l03g21.png}}
% 78
{\PTglyph{5}{t34_l03g22.png}}
% 79
{\PTglyph{5}{t34_l03g23.png}}
% 80
{\PTglyph{5}{t34_l03g24.png}}
% 81
{\PTglyph{5}{t34_l03g25.png}}
% 82
{\PTglyph{5}{t34_l03g26.png}}
% 83
{\PTglyph{5}{t34_l03g27.png}}
% 84
{\PTglyph{5}{t34_l03g28.png}}
% 85
{\PTglyph{5}{t34_l03g29.png}}
% 86
{\PTglyph{5}{t34_l03g30.png}}
% 87
{\PTglyph{5}{t34_l03g31.png}}
% 88
{\PTglyph{5}{t34_l03g32.png}}
% 89
{\PTglyph{5}{t34_l03g33.png}}
% 90
{\PTglyph{5}{t34_l03g34.png}}
% 91
{\PTglyph{5}{t34_l04g01.png}}
% 92
{\PTglyph{5}{t34_l04g02.png}}
% 93
{\PTglyph{5}{t34_l04g03.png}}
% 94
{\PTglyph{5}{t34_l04g04.png}}
% 95
{\PTglyph{5}{t34_l04g05.png}}
% 96
{\PTglyph{5}{t34_l04g06.png}}
% 97
{\PTglyph{5}{t34_l04g07.png}}
% 98
{\PTglyph{5}{t34_l04g08.png}}
% 99
{\PTglyph{5}{t34_l04g09.png}}
% 100
{\PTglyph{5}{t34_l04g10.png}}
% 101
{\PTglyph{5}{t34_l04g11.png}}
% 102
{\PTglyph{5}{t34_l04g12.png}}
% 103
{\PTglyph{5}{t34_l04g13.png}}
% 104
{\PTglyph{5}{t34_l04g14.png}}
% 105
{\PTglyph{5}{t34_l04g15.png}}
% 106
{\PTglyph{5}{t34_l04g16.png}}
% 107
{\PTglyph{5}{t34_l04g17.png}}
% 108
{\PTglyph{5}{t34_l04g18.png}}
% 109
{\PTglyph{5}{t34_l04g19.png}}
% 110
{\PTglyph{5}{t34_l04g20.png}}
% 111
{\PTglyph{5}{t34_l04g21.png}}
% 112
{\PTglyph{5}{t34_l04g22.png}}
% 113
{\PTglyph{5}{t34_l04g23.png}}
% 114
{\PTglyph{5}{t34_l04g24.png}}
% 115
{\PTglyph{5}{t34_l04g25.png}}
% 116
{\PTglyph{5}{t34_l04g26.png}}
% 117
{\PTglyph{5}{t34_l04g27.png}}
% 118
{\PTglyph{5}{t34_l04g28.png}}
% 119
{\PTglyph{5}{t34_l04g29.png}}
% 120
{\PTglyph{5}{t34_l04g30.png}}
% 121
{\PTglyph{5}{t34_l04g31.png}}
% 122
{\PTglyph{5}{t34_l05g01.png}}
% 123
{\PTglyph{5}{t34_l05g02.png}}
% 124
{\PTglyph{5}{t34_l05g03.png}}
//
%%% Local Variables:
%%% mode: latex
%%% TeX-engine: luatex
%%% TeX-master: shared
%%% End:

%//
%\glpismo%
 \glpismo
% 1
{\PTglyphid{Ho-03_0101}}
% 2
{\PTglyphid{Ho-03_0102}}
% 3
{\PTglyphid{Ho-03_0103}}
% 4
{\PTglyphid{Ho-03_0104}}
% 5
{\PTglyphid{Ho-03_0105}}
% 6
{\PTglyphid{Ho-03_0106}}
% 7
{\PTglyphid{Ho-03_0107}}
% 8
{\PTglyphid{Ho-03_0108}}
% 9
{\PTglyphid{Ho-03_0109}}
% 10
{\PTglyphid{Ho-03_0110}}
% 11
{\PTglyphid{Ho-03_0111}}
% 12
{\PTglyphid{Ho-03_0112}}
% 13
{\PTglyphid{Ho-03_0113}}
% 14
{\PTglyphid{Ho-03_0114}}
% 15
{\PTglyphid{Ho-03_0115}}
% 16
{\PTglyphid{Ho-03_0116}}
% 17
{\PTglyphid{Ho-03_0117}}
% 18
{\PTglyphid{Ho-03_0118}}
% 19
{\PTglyphid{Ho-03_0201}}
% 20
{\PTglyphid{Ho-03_0202}}
% 21
{\PTglyphid{Ho-03_0203}}
% 22
{\PTglyphid{Ho-03_0204}}
% 23
{\PTglyphid{Ho-03_0205}}
% 24
{\PTglyphid{Ho-03_0206}}
% 25
{\PTglyphid{Ho-03_0207}}
% 26
{\PTglyphid{Ho-03_0208}}
% 27
{\PTglyphid{Ho-03_0209}}
% 28
{\PTglyphid{Ho-03_0210}}
% 29
{\PTglyphid{Ho-03_0211}}
% 30
{\PTglyphid{Ho-03_0212}}
% 31
{\PTglyphid{Ho-03_0213}}
% 32
{\PTglyphid{Ho-03_0214}}
% 33
{\PTglyphid{Ho-03_0215}}
% 34
{\PTglyphid{Ho-03_0216}}
% 35
{\PTglyphid{Ho-03_0217}}
% 36
{\PTglyphid{Ho-03_0218}}
% 37
{\PTglyphid{Ho-03_0219}}
% 38
{\PTglyphid{Ho-03_0220}}
% 39
{\PTglyphid{Ho-03_0221}}
% 40
{\PTglyphid{Ho-03_0222}}
% 41
{\PTglyphid{Ho-03_0223}}
% 42
{\PTglyphid{Ho-03_0224}}
% 43
{\PTglyphid{Ho-03_0225}}
% 44
{\PTglyphid{Ho-03_0226}}
% 45
{\PTglyphid{Ho-03_0227}}
% 46
{\PTglyphid{Ho-03_0228}}
% 47
{\PTglyphid{Ho-03_0229}}
% 48
{\PTglyphid{Ho-03_0230}}
% 49
{\PTglyphid{Ho-03_0231}}
% 50
{\PTglyphid{Ho-03_0232}}
% 51
{\PTglyphid{Ho-03_0233}}
% 52
{\PTglyphid{Ho-03_0234}}
% 53
{\PTglyphid{Ho-03_0235}}
% 54
{\PTglyphid{Ho-03_0236}}
% 55
{\PTglyphid{Ho-03_0301}}
% 56
{\PTglyphid{Ho-03_0302}}
% 57
{\PTglyphid{Ho-03_0303}}
% 58
{\PTglyphid{Ho-03_0304}}
% 59
{\PTglyphid{Ho-03_0305}}
% 60
{\PTglyphid{Ho-03_0306}}
% 61
{\PTglyphid{Ho-03_0307}}
% 62
{\PTglyphid{Ho-03_0308}}
% 63
{\PTglyphid{Ho-03_0309}}
% 64
{\PTglyphid{Ho-03_0310}}
% 65
{\PTglyphid{Ho-03_0311}}
% 66
{\PTglyphid{Ho-03_0312}}
% 67
{\PTglyphid{Ho-03_0313}}
% 68
{\PTglyphid{Ho-03_0314}}
% 69
{\PTglyphid{Ho-03_0315}}
% 70
{\PTglyphid{Ho-03_0316}}
% 71
{\PTglyphid{Ho-03_0317}}
% 72
{\PTglyphid{Ho-03_0318}}
% 73
{\PTglyphid{Ho-03_0319}}
% 74
{\PTglyphid{Ho-03_0320}}
% 75
{\PTglyphid{Ho-03_0321}}
% 76
{\PTglyphid{Ho-03_0322}}
% 77
{\PTglyphid{Ho-03_0323}}
% 78
{\PTglyphid{Ho-03_0324}}
% 79
{\PTglyphid{Ho-03_0325}}
% 80
{\PTglyphid{Ho-03_0326}}
% 81
{\PTglyphid{Ho-03_0327}}
% 82
{\PTglyphid{Ho-03_0328}}
% 83
{\PTglyphid{Ho-03_0329}}
% 84
{\PTglyphid{Ho-03_0330}}
% 85
{\PTglyphid{Ho-03_0331}}
% 86
{\PTglyphid{Ho-03_0332}}
% 87
{\PTglyphid{Ho-03_0333}}
% 88
{\PTglyphid{Ho-03_0334}}
% 89
{\PTglyphid{Ho-03_0401}}
% 90
{\PTglyphid{Ho-03_0402}}
% 91
{\PTglyphid{Ho-03_0403}}
% 92
{\PTglyphid{Ho-03_0404}}
% 93
{\PTglyphid{Ho-03_0405}}
% 94
{\PTglyphid{Ho-03_0406}}
% 95
{\PTglyphid{Ho-03_0407}}
% 96
{\PTglyphid{Ho-03_0408}}
% 97
{\PTglyphid{Ho-03_0409}}
% 98
{\PTglyphid{Ho-03_0410}}
% 99
{\PTglyphid{Ho-03_0411}}
% 100
{\PTglyphid{Ho-03_0412}}
% 101
{\PTglyphid{Ho-03_0413}}
% 102
{\PTglyphid{Ho-03_0414}}
% 103
{\PTglyphid{Ho-03_0415}}
% 104
{\PTglyphid{Ho-03_0416}}
% 105
{\PTglyphid{Ho-03_0417}}
% 106
{\PTglyphid{Ho-03_0418}}
% 107
{\PTglyphid{Ho-03_0419}}
% 108
{\PTglyphid{Ho-03_0420}}
% 109
{\PTglyphid{Ho-03_0421}}
% 110
{\PTglyphid{Ho-03_0422}}
% 111
{\PTglyphid{Ho-03_0423}}
% 112
{\PTglyphid{Ho-03_0424}}
% 113
{\PTglyphid{Ho-03_0425}}
% 114
{\PTglyphid{Ho-03_0426}}
% 115
{\PTglyphid{Ho-03_0427}}
% 116
{\PTglyphid{Ho-03_0428}}
% 117
{\PTglyphid{Ho-03_0429}}
% 118
{\PTglyphid{Ho-03_0430}}
% 119
{\PTglyphid{Ho-03_0431}}
% 120
{\PTglyphid{Ho-03_0501}}
% 121
{\PTglyphid{Ho-03_0502}}
% 122
{\PTglyphid{Ho-03_0503}}
//
\endgl \xe
%%% Local Variables:
%%% mode: latex
%%% TeX-engine: luatex
%%% TeX-master: shared
%%% End:

% //
%\endgl \xe

\newpage

%%%%%%%%%%%%%%%%%%%%%%%%%%%%%%%%%%%%%%%%%%%%%%%%%%%%%%%%%%%%%%%%%%%%%%%%%%%%%%%
% from meta.csv
 % Tab. 35,Hochfeder-04_PT01_022.djvu,Hochfeder,04,01,022
%%%%%%%%%%%%%%%%%%%%%%%%%%%%%%%%%%%%%%%%%%%%%%%%%%%%%%%%%%%%%%%%%%%%%%%%%%%%%%%

 % from dsed4test:

% Note "4. Pismo marginalne gotyckie. Krój M⁹⁸. Stopień 20 ww. = 57/58 mm - Tabl. 22, (Występuje u Hallera jako pismo 6. Tabl. 169) [22]"
% Note1 "Character set table prepared by Maria Błońska and Anna Wolińska"

\pismoPL{Kasper Hochfeder 4. Pismo marginalne gotyckie. Krój M⁹⁸. Stopień 20 ww. = 57/58 mm - Tabl. 22, (Występuje u Hallera jako pismo 6. Tabl. 169)}
  
\pismoEN{Kasper Hochfeder 4. Gothic margin font. Typeface M⁹⁸. Type
  size 20 lines = 57/58 mm. (Used by Haller as font 6. Plate 169)}

\plate{22}{I}{1968}

Prepared by Kazimierz Piekarski and  Maria Błońska.\\
The font table prepared by Maria Błońska and Anna Wolińska.

\bigskip

\exampleBib{I:1}

\bigskip \exampleDesc{FRANCISCUS NIGER: Compendiosa ars de
  epistolis. Kraków, Kasper Hochfeder, 10 VI 1503. 4⁰.}

\medskip
\examplePage{\textit{Karta ostatnia verso.}}

  \bigskip
\exampleLib{Biblioteka Czartoryskich Kraków.}


\bigskip
\exampleRef{\textit{Estreicher XXXI 153.}}

  
  % \medskip
\bigskip

\examplePL{Pismo 2: wiersz 1-4. — Pismo 3: wiersz 29-30. — Pismo 4: wiersz
  5-28. — Rubryka \beta{}: z pismem 3. — Rubryka \gamma{}: z pismem 4.}

\medskip

    \exampleEN{Font 2: lines 1-4. — Font 3: lines 29-30. — font 4: lines
  5-28. — Rubric \beta{}: with font 3. — Rubric \gamma{}: with font 4.}



\bigskip

% Scharfenberg:
% https://kpbc.umk.pl/dlibra/publication/258387/edition/268632?language=pl
% Lipsk:
% https://cyfrowe.mnk.pl/dlibra/publication/25371/edition/25054?language=pl

\fontID{Ho-04}{35}

\fontstat{75}

% \exdisplay \bg \gla
 \exdisplay \bg \gla
% 1
{\PTglyph{5}{t35_l01g01.png}}
% 2
{\PTglyph{5}{t35_l01g02.png}}
% 3
{\PTglyph{5}{t35_l01g03.png}}
% 4
{\PTglyph{5}{t35_l01g04.png}}
% 5
{\PTglyph{5}{t35_l01g05.png}}
% 6
{\PTglyph{5}{t35_l01g06.png}}
% 7
{\PTglyph{5}{t35_l01g07.png}}
% 8
{\PTglyph{5}{t35_l01g08.png}}
% 9
{\PTglyph{5}{t35_l01g09.png}}
% 10
{\PTglyph{5}{t35_l01g10.png}}
% 11
{\PTglyph{5}{t35_l01g11.png}}
% 12
{\PTglyph{5}{t35_l01g12.png}}
% 13
{\PTglyph{5}{t35_l01g13.png}}
% 14
{\PTglyph{5}{t35_l01g14.png}}
% 15
{\PTglyph{5}{t35_l02g01.png}}
% 16
{\PTglyph{5}{t35_l02g02.png}}
% 17
{\PTglyph{5}{t35_l02g03.png}}
% 18
{\PTglyph{5}{t35_l02g04.png}}
% 19
{\PTglyph{5}{t35_l02g05.png}}
% 20
{\PTglyph{5}{t35_l02g06.png}}
% 21
{\PTglyph{5}{t35_l02g07.png}}
% 22
{\PTglyph{5}{t35_l02g08.png}}
% 23
{\PTglyph{5}{t35_l02g09.png}}
% 24
{\PTglyph{5}{t35_l02g10.png}}
% 25
{\PTglyph{5}{t35_l02g11.png}}
% 26
{\PTglyph{5}{t35_l02g12.png}}
% 27
{\PTglyph{5}{t35_l02g13.png}}
% 28
{\PTglyph{5}{t35_l02g14.png}}
% 29
{\PTglyph{5}{t35_l02g15.png}}
% 30
{\PTglyph{5}{t35_l02g16.png}}
% 31
{\PTglyph{5}{t35_l02g17.png}}
% 32
{\PTglyph{5}{t35_l02g18.png}}
% 33
{\PTglyph{5}{t35_l02g19.png}}
% 34
{\PTglyph{5}{t35_l02g20.png}}
% 35
{\PTglyph{5}{t35_l02g21.png}}
% 36
{\PTglyph{5}{t35_l02g22.png}}
% 37
{\PTglyph{5}{t35_l02g23.png}}
% 38
{\PTglyph{5}{t35_l03g01.png}}
% 39
{\PTglyph{5}{t35_l03g02.png}}
% 40
{\PTglyph{5}{t35_l03g03.png}}
% 41
{\PTglyph{5}{t35_l03g04.png}}
% 42
{\PTglyph{5}{t35_l03g05.png}}
% 43
{\PTglyph{5}{t35_l03g06.png}}
% 44
{\PTglyph{5}{t35_l03g07.png}}
% 45
{\PTglyph{5}{t35_l03g08.png}}
% 46
{\PTglyph{5}{t35_l03g09.png}}
% 47
{\PTglyph{5}{t35_l03g10.png}}
% 48
{\PTglyph{5}{t35_l03g11.png}}
% 49
{\PTglyph{5}{t35_l03g12.png}}
% 50
{\PTglyph{5}{t35_l03g13.png}}
% 51
{\PTglyph{5}{t35_l03g14.png}}
% 52
{\PTglyph{5}{t35_l03g15.png}}
% 53
{\PTglyph{5}{t35_l03g16.png}}
% 54
{\PTglyph{5}{t35_l03g17.png}}
% 55
{\PTglyph{5}{t35_l03g18.png}}
% 56
{\PTglyph{5}{t35_l03g19.png}}
% 57
{\PTglyph{5}{t35_l03g20.png}}
% 58
{\PTglyph{5}{t35_l03g21.png}}
% 59
{\PTglyph{5}{t35_l04g01.png}}
% 60
{\PTglyph{5}{t35_l04g02.png}}
% 61
{\PTglyph{5}{t35_l04g03.png}}
% 62
{\PTglyph{5}{t35_l04g04.png}}
% 63
{\PTglyph{5}{t35_l04g05.png}}
% 64
{\PTglyph{5}{t35_l04g06.png}}
% 65
{\PTglyph{5}{t35_l04g07.png}}
% 66
{\PTglyph{5}{t35_l04g08.png}}
% 67
{\PTglyph{5}{t35_l04g09.png}}
% 68
{\PTglyph{5}{t35_l04g10.png}}
% 69
{\PTglyph{5}{t35_l04g11.png}}
% 70
{\PTglyph{5}{t35_l04g12.png}}
% 71
{\PTglyph{5}{t35_l04g13.png}}
% 72
{\PTglyph{5}{t35_l04g14.png}}
% 73
{\PTglyph{5}{t35_l04g15.png}}
% 74
{\PTglyph{5}{t35_l04g16.png}}
% 75
{\PTglyph{5}{t35_l04g17.png}}
//
%%% Local Variables:
%%% mode: latex
%%% TeX-engine: luatex
%%% TeX-master: shared
%%% End:

%//
%\glpismo%
 \glpismo
% 1
{\PTglyphid{04-01_0101}}
% 2
{\PTglyphid{04-01_0102}}
% 3
{\PTglyphid{04-01_0103}}
% 4
{\PTglyphid{04-01_0104}}
% 5
{\PTglyphid{04-01_0105}}
% 6
{\PTglyphid{04-01_0106}}
% 7
{\PTglyphid{04-01_0107}}
% 8
{\PTglyphid{04-01_0108}}
% 9
{\PTglyphid{04-01_0109}}
% 10
{\PTglyphid{04-01_0110}}
% 11
{\PTglyphid{04-01_0111}}
% 12
{\PTglyphid{04-01_0112}}
% 13
{\PTglyphid{04-01_0113}}
% 14
{\PTglyphid{04-01_0114}}
% 15
{\PTglyphid{04-01_0201}}
% 16
{\PTglyphid{04-01_0202}}
% 17
{\PTglyphid{04-01_0203}}
% 18
{\PTglyphid{04-01_0204}}
% 19
{\PTglyphid{04-01_0205}}
% 20
{\PTglyphid{04-01_0206}}
% 21
{\PTglyphid{04-01_0207}}
% 22
{\PTglyphid{04-01_0208}}
% 23
{\PTglyphid{04-01_0209}}
% 24
{\PTglyphid{04-01_0210}}
% 25
{\PTglyphid{04-01_0211}}
% 26
{\PTglyphid{04-01_0212}}
% 27
{\PTglyphid{04-01_0213}}
% 28
{\PTglyphid{04-01_0214}}
% 29
{\PTglyphid{04-01_0215}}
% 30
{\PTglyphid{04-01_0216}}
% 31
{\PTglyphid{04-01_0217}}
% 32
{\PTglyphid{04-01_0218}}
% 33
{\PTglyphid{04-01_0219}}
% 34
{\PTglyphid{04-01_0220}}
% 35
{\PTglyphid{04-01_0221}}
% 36
{\PTglyphid{04-01_0222}}
% 37
{\PTglyphid{04-01_0223}}
% 38
{\PTglyphid{04-01_0301}}
% 39
{\PTglyphid{04-01_0302}}
% 40
{\PTglyphid{04-01_0303}}
% 41
{\PTglyphid{04-01_0304}}
% 42
{\PTglyphid{04-01_0305}}
% 43
{\PTglyphid{04-01_0306}}
% 44
{\PTglyphid{04-01_0307}}
% 45
{\PTglyphid{04-01_0308}}
% 46
{\PTglyphid{04-01_0309}}
% 47
{\PTglyphid{04-01_0310}}
% 48
{\PTglyphid{04-01_0311}}
% 49
{\PTglyphid{04-01_0312}}
% 50
{\PTglyphid{04-01_0313}}
% 51
{\PTglyphid{04-01_0314}}
% 52
{\PTglyphid{04-01_0315}}
% 53
{\PTglyphid{04-01_0316}}
% 54
{\PTglyphid{04-01_0317}}
% 55
{\PTglyphid{04-01_0318}}
% 56
{\PTglyphid{04-01_0319}}
% 57
{\PTglyphid{04-01_0320}}
% 58
{\PTglyphid{04-01_0321}}
% 59
{\PTglyphid{04-01_0401}}
% 60
{\PTglyphid{04-01_0402}}
% 61
{\PTglyphid{04-01_0403}}
% 62
{\PTglyphid{04-01_0404}}
% 63
{\PTglyphid{04-01_0405}}
% 64
{\PTglyphid{04-01_0406}}
% 65
{\PTglyphid{04-01_0407}}
% 66
{\PTglyphid{04-01_0408}}
% 67
{\PTglyphid{04-01_0409}}
% 68
{\PTglyphid{04-01_0410}}
% 69
{\PTglyphid{04-01_0411}}
% 70
{\PTglyphid{04-01_0412}}
% 71
{\PTglyphid{04-01_0413}}
% 72
{\PTglyphid{04-01_0414}}
% 73
{\PTglyphid{04-01_0415}}
% 74
{\PTglyphid{04-01_0416}}
% 75
{\PTglyphid{04-01_0417}}
//
\endgl \xe
%%% Local Variables:
%%% mode: latex
%%% TeX-engine: luatex
%%% TeX-master: shared
%%% End:

% //
%\endgl \xe

\newpage

%%%%%%%%%%%%%%%%%%%%%%%%%%%%%%%%%%%%%%%%%%%%%%%%%%%%%%%%%%%%%%%%%%%%%%%%%%%%%%%
% from meta.csv
 % Tab. 36,Hochfeder-05_PT01_023.djvu,Hochfeder,05,01,023
%%%%%%%%%%%%%%%%%%%%%%%%%%%%%%%%%%%%%%%%%%%%%%%%%%%%%%%%%%%%%%%%%%%%%%%%%%%%%%%

 % from dsed4test:

% Note "5. Pismo tekstowe gotyckie. Krój M⁸⁸. Stopień 20 ww. = 88/89 mm - Tabl. 23 [23]"
% Note1 "Character set table prepared by Maria Błońska and Anna Wolińska"


\pismoPL{Kasper Hochfeder 5. Pismo tekstowe gotyckie. Krój M⁸⁸. Stopień 20 ww. = 88/89 mm - Tabl. 23}
  
\pismoEN{Kasper Hochfeder 5. Gothic text font. Typeface M⁸⁸. Type size 20 lines = 88/89 mm - Plate 23}

\plate{23}{I}{1968}

Prepared by Kazimierz Piekarski and  Maria Błońska.\\
The font table prepared by Maria Błońska and Anna Wolińska.

\bigskip

\exampleBib{I:8}

\bigskip \exampleDesc{IOANNES GLOGOVIENSIS: Minoris Donati
  interpretatio. Kraków, [Kasper Hochfeder] nakładem Jana Hallera,
  1503. 4⁰.}


\medskip
\examplePage{\textit{F₇b}}

  \bigskip
\exampleLib{Biblioteka Jagiellońska. Kraków.}


\bigskip
\exampleRef{\textit{Estreicher XVII 175. Wierzbowski 2040.}}

  
  % \medskip
\bigskip

\examplePL{[Pismo 5.] Rubryka \chi{}}

\medskip

    \exampleEN{[Font 5.] Rubric \chi{}}



\bigskip

\fontID{Ho-05}{36}

\fontstat{107}

% \exdisplay \bg \gla
 \exdisplay \bg \gla
% 1
{\PTglyph{5}{t36_l01g01.png}}
% 2
{\PTglyph{5}{t36_l01g02.png}}
% 3
{\PTglyph{5}{t36_l01g03.png}}
% 4
{\PTglyph{5}{t36_l01g04.png}}
% 5
{\PTglyph{5}{t36_l01g05.png}}
% 6
{\PTglyph{5}{t36_l01g06.png}}
% 7
{\PTglyph{5}{t36_l01g07.png}}
% 8
{\PTglyph{5}{t36_l01g08.png}}
% 9
{\PTglyph{5}{t36_l01g09.png}}
% 10
{\PTglyph{5}{t36_l01g10.png}}
% 11
{\PTglyph{5}{t36_l01g11.png}}
% 12
{\PTglyph{5}{t36_l01g12.png}}
% 13
{\PTglyph{5}{t36_l01g13.png}}
% 14
{\PTglyph{5}{t36_l01g14.png}}
% 15
{\PTglyph{5}{t36_l01g15.png}}
% 16
{\PTglyph{5}{t36_l01g16.png}}
% 17
{\PTglyph{5}{t36_l01g17.png}}
% 18
{\PTglyph{5}{t36_l01g18.png}}
% 19
{\PTglyph{5}{t36_l01g19.png}}
% 20
{\PTglyph{5}{t36_l01g20.png}}
% 21
{\PTglyph{5}{t36_l01g21.png}}
% 22
{\PTglyph{5}{t36_l02g01.png}}
% 23
{\PTglyph{5}{t36_l02g02.png}}
% 24
{\PTglyph{5}{t36_l02g03.png}}
% 25
{\PTglyph{5}{t36_l02g04.png}}
% 26
{\PTglyph{5}{t36_l02g05.png}}
% 27
{\PTglyph{5}{t36_l02g06.png}}
% 28
{\PTglyph{5}{t36_l02g07.png}}
% 29
{\PTglyph{5}{t36_l02g08.png}}
% 30
{\PTglyph{5}{t36_l02g09.png}}
% 31
{\PTglyph{5}{t36_l02g10.png}}
% 32
{\PTglyph{5}{t36_l02g11.png}}
% 33
{\PTglyph{5}{t36_l02g12.png}}
% 34
{\PTglyph{5}{t36_l02g13.png}}
% 35
{\PTglyph{5}{t36_l02g14.png}}
% 36
{\PTglyph{5}{t36_l02g15.png}}
% 37
{\PTglyph{5}{t36_l02g16.png}}
% 38
{\PTglyph{5}{t36_l02g17.png}}
% 39
{\PTglyph{5}{t36_l02g18.png}}
% 40
{\PTglyph{5}{t36_l02g19.png}}
% 41
{\PTglyph{5}{t36_l02g20.png}}
% 42
{\PTglyph{5}{t36_l02g21.png}}
% 43
{\PTglyph{5}{t36_l02g22.png}}
% 44
{\PTglyph{5}{t36_l02g23.png}}
% 45
{\PTglyph{5}{t36_l02g24.png}}
% 46
{\PTglyph{5}{t36_l02g25.png}}
% 47
{\PTglyph{5}{t36_l02g26.png}}
% 48
{\PTglyph{5}{t36_l02g27.png}}
% 49
{\PTglyph{5}{t36_l02g28.png}}
% 50
{\PTglyph{5}{t36_l02g29.png}}
% 51
{\PTglyph{5}{t36_l02g30.png}}
% 52
{\PTglyph{5}{t36_l02g31.png}}
% 53
{\PTglyph{5}{t36_l02g32.png}}
% 54
{\PTglyph{5}{t36_l02g33.png}}
% 55
{\PTglyph{5}{t36_l02g34.png}}
% 56
{\PTglyph{5}{t36_l02g35.png}}
% 57
{\PTglyph{5}{t36_l02g36.png}}
% 58
{\PTglyph{5}{t36_l02g37.png}}
% 59
{\PTglyph{5}{t36_l02g38.png}}
% 60
{\PTglyph{5}{t36_l02g39.png}}
% 61
{\PTglyph{5}{t36_l03g01.png}}
% 62
{\PTglyph{5}{t36_l03g02.png}}
% 63
{\PTglyph{5}{t36_l03g03.png}}
% 64
{\PTglyph{5}{t36_l03g04.png}}
% 65
{\PTglyph{5}{t36_l03g05.png}}
% 66
{\PTglyph{5}{t36_l03g06.png}}
% 67
{\PTglyph{5}{t36_l03g07.png}}
% 68
{\PTglyph{5}{t36_l03g08.png}}
% 69
{\PTglyph{5}{t36_l03g09.png}}
% 70
{\PTglyph{5}{t36_l03g10.png}}
% 71
{\PTglyph{5}{t36_l03g11.png}}
% 72
{\PTglyph{5}{t36_l03g12.png}}
% 73
{\PTglyph{5}{t36_l03g13.png}}
% 74
{\PTglyph{5}{t36_l03g14.png}}
% 75
{\PTglyph{5}{t36_l03g15.png}}
% 76
{\PTglyph{5}{t36_l03g16.png}}
% 77
{\PTglyph{5}{t36_l03g17.png}}
% 78
{\PTglyph{5}{t36_l03g18.png}}
% 79
{\PTglyph{5}{t36_l03g19.png}}
% 80
{\PTglyph{5}{t36_l03g20.png}}
% 81
{\PTglyph{5}{t36_l03g21.png}}
% 82
{\PTglyph{5}{t36_l03g22.png}}
% 83
{\PTglyph{5}{t36_l03g23.png}}
% 84
{\PTglyph{5}{t36_l03g24.png}}
% 85
{\PTglyph{5}{t36_l03g25.png}}
% 86
{\PTglyph{5}{t36_l03g26.png}}
% 87
{\PTglyph{5}{t36_l03g27.png}}
% 88
{\PTglyph{5}{t36_l03g28.png}}
% 89
{\PTglyph{5}{t36_l03g29.png}}
% 90
{\PTglyph{5}{t36_l03g30.png}}
% 91
{\PTglyph{5}{t36_l03g31.png}}
% 92
{\PTglyph{5}{t36_l03g32.png}}
% 93
{\PTglyph{5}{t36_l03g33.png}}
% 94
{\PTglyph{5}{t36_l03g34.png}}
% 95
{\PTglyph{5}{t36_l03g35.png}}
% 96
{\PTglyph{5}{t36_l03g36.png}}
% 97
{\PTglyph{5}{t36_l04g01.png}}
% 98
{\PTglyph{5}{t36_l04g02.png}}
% 99
{\PTglyph{5}{t36_l04g03.png}}
% 100
{\PTglyph{5}{t36_l04g04.png}}
% 101
{\PTglyph{5}{t36_l04g05.png}}
% 102
{\PTglyph{5}{t36_l04g06.png}}
% 103
{\PTglyph{5}{t36_l04g07.png}}
% 104
{\PTglyph{5}{t36_l04g08.png}}
% 105
{\PTglyph{5}{t36_l04g09.png}}
% 106
{\PTglyph{5}{t36_l04g10.png}}
% 107
{\PTglyph{5}{t36_l04g11.png}}
//
%%% Local Variables:
%%% mode: latex
%%% TeX-engine: luatex
%%% TeX-master: shared
%%% End:

%//
%\glpismo%
 \glpismo
% 1
{\PTglyphid{Ho-05_0101}}
% 2
{\PTglyphid{Ho-05_0102}}
% 3
{\PTglyphid{Ho-05_0103}}
% 4
{\PTglyphid{Ho-05_0104}}
% 5
{\PTglyphid{Ho-05_0105}}
% 6
{\PTglyphid{Ho-05_0106}}
% 7
{\PTglyphid{Ho-05_0107}}
% 8
{\PTglyphid{Ho-05_0108}}
% 9
{\PTglyphid{Ho-05_0109}}
% 10
{\PTglyphid{Ho-05_0110}}
% 11
{\PTglyphid{Ho-05_0111}}
% 12
{\PTglyphid{Ho-05_0112}}
% 13
{\PTglyphid{Ho-05_0113}}
% 14
{\PTglyphid{Ho-05_0114}}
% 15
{\PTglyphid{Ho-05_0115}}
% 16
{\PTglyphid{Ho-05_0116}}
% 17
{\PTglyphid{Ho-05_0117}}
% 18
{\PTglyphid{Ho-05_0118}}
% 19
{\PTglyphid{Ho-05_0119}}
% 20
{\PTglyphid{Ho-05_0120}}
% 21
{\PTglyphid{Ho-05_0121}}
% 22
{\PTglyphid{Ho-05_0201}}
% 23
{\PTglyphid{Ho-05_0202}}
% 24
{\PTglyphid{Ho-05_0203}}
% 25
{\PTglyphid{Ho-05_0204}}
% 26
{\PTglyphid{Ho-05_0205}}
% 27
{\PTglyphid{Ho-05_0206}}
% 28
{\PTglyphid{Ho-05_0207}}
% 29
{\PTglyphid{Ho-05_0208}}
% 30
{\PTglyphid{Ho-05_0209}}
% 31
{\PTglyphid{Ho-05_0210}}
% 32
{\PTglyphid{Ho-05_0211}}
% 33
{\PTglyphid{Ho-05_0212}}
% 34
{\PTglyphid{Ho-05_0213}}
% 35
{\PTglyphid{Ho-05_0214}}
% 36
{\PTglyphid{Ho-05_0215}}
% 37
{\PTglyphid{Ho-05_0216}}
% 38
{\PTglyphid{Ho-05_0217}}
% 39
{\PTglyphid{Ho-05_0218}}
% 40
{\PTglyphid{Ho-05_0219}}
% 41
{\PTglyphid{Ho-05_0220}}
% 42
{\PTglyphid{Ho-05_0221}}
% 43
{\PTglyphid{Ho-05_0222}}
% 44
{\PTglyphid{Ho-05_0223}}
% 45
{\PTglyphid{Ho-05_0224}}
% 46
{\PTglyphid{Ho-05_0225}}
% 47
{\PTglyphid{Ho-05_0226}}
% 48
{\PTglyphid{Ho-05_0227}}
% 49
{\PTglyphid{Ho-05_0228}}
% 50
{\PTglyphid{Ho-05_0229}}
% 51
{\PTglyphid{Ho-05_0230}}
% 52
{\PTglyphid{Ho-05_0231}}
% 53
{\PTglyphid{Ho-05_0232}}
% 54
{\PTglyphid{Ho-05_0233}}
% 55
{\PTglyphid{Ho-05_0234}}
% 56
{\PTglyphid{Ho-05_0235}}
% 57
{\PTglyphid{Ho-05_0236}}
% 58
{\PTglyphid{Ho-05_0237}}
% 59
{\PTglyphid{Ho-05_0238}}
% 60
{\PTglyphid{Ho-05_0239}}
% 61
{\PTglyphid{Ho-05_0301}}
% 62
{\PTglyphid{Ho-05_0302}}
% 63
{\PTglyphid{Ho-05_0303}}
% 64
{\PTglyphid{Ho-05_0304}}
% 65
{\PTglyphid{Ho-05_0305}}
% 66
{\PTglyphid{Ho-05_0306}}
% 67
{\PTglyphid{Ho-05_0307}}
% 68
{\PTglyphid{Ho-05_0308}}
% 69
{\PTglyphid{Ho-05_0309}}
% 70
{\PTglyphid{Ho-05_0310}}
% 71
{\PTglyphid{Ho-05_0311}}
% 72
{\PTglyphid{Ho-05_0312}}
% 73
{\PTglyphid{Ho-05_0313}}
% 74
{\PTglyphid{Ho-05_0314}}
% 75
{\PTglyphid{Ho-05_0315}}
% 76
{\PTglyphid{Ho-05_0316}}
% 77
{\PTglyphid{Ho-05_0317}}
% 78
{\PTglyphid{Ho-05_0318}}
% 79
{\PTglyphid{Ho-05_0319}}
% 80
{\PTglyphid{Ho-05_0320}}
% 81
{\PTglyphid{Ho-05_0321}}
% 82
{\PTglyphid{Ho-05_0322}}
% 83
{\PTglyphid{Ho-05_0323}}
% 84
{\PTglyphid{Ho-05_0324}}
% 85
{\PTglyphid{Ho-05_0325}}
% 86
{\PTglyphid{Ho-05_0326}}
% 87
{\PTglyphid{Ho-05_0327}}
% 88
{\PTglyphid{Ho-05_0328}}
% 89
{\PTglyphid{Ho-05_0329}}
% 90
{\PTglyphid{Ho-05_0330}}
% 91
{\PTglyphid{Ho-05_0331}}
% 92
{\PTglyphid{Ho-05_0332}}
% 93
{\PTglyphid{Ho-05_0333}}
% 94
{\PTglyphid{Ho-05_0334}}
% 95
{\PTglyphid{Ho-05_0335}}
% 96
{\PTglyphid{Ho-05_0336}}
% 97
{\PTglyphid{Ho-05_0401}}
% 98
{\PTglyphid{Ho-05_0402}}
% 99
{\PTglyphid{Ho-05_0403}}
% 100
{\PTglyphid{Ho-05_0404}}
% 101
{\PTglyphid{Ho-05_0405}}
% 102
{\PTglyphid{Ho-05_0406}}
% 103
{\PTglyphid{Ho-05_0407}}
% 104
{\PTglyphid{Ho-05_0408}}
% 105
{\PTglyphid{Ho-05_0409}}
% 106
{\PTglyphid{Ho-05_0410}}
% 107
{\PTglyphid{Ho-05_0411}}
//
\endgl \xe
%%% Local Variables:
%%% mode: latex
%%% TeX-engine: luatex
%%% TeX-master: shared
%%% End:

% //
%\endgl \xe

\newpage

%%%%%%%%%%%%%%%%%%%%%%%%%%%%%%%%%%%%%%%%%%%%%%%%%%%%%%%%%%%%%%%%%%%%%%%%%%%%%%%
% from meta.csv
 % Tab. 37,Hochfeder-06_PT01_024.djvu,Hochfeder,06,01,024
%%%%%%%%%%%%%%%%%%%%%%%%%%%%%%%%%%%%%%%%%%%%%%%%%%%%%%%%%%%%%%%%%%%%%%%%%%%%%%%

 % from dsed4test:

% Note "6. Pismo tekstowe gotyckie. Krój M⁸⁸. Stopień 20 ww. = 79/80 mm - Tabl. 24 [24]"
% Note1 "Character set table prepared by Maria Błońska and Anna Wolińska"



\pismoPL{Kasper Hochfeder 6. Pismo tekstowe gotyckie. Krój M⁸⁸. Stopień 20 ww. = 79/80 mm - Tabl. 24}
  
\pismoEN{Kasper Hochfeder 6. Gothic text font. Typeface M⁸⁸. Type size 20 lines = 79/80 mm - Plate 24}

\plate{24}{I}{1968}

Prepared by Kazimierz Piekarski and  Maria Błońska.\\
The font table prepared by Maria Błońska and Anna Wolińska.

\bigskip

\exampleBib{I:24}

\bigskip \exampleDesc{IOANNES[sic] DE SACROBOSCO: Algorithmus. Kraków, [Kasper Hochfeder]. 1504. 4⁰.}
IOHANNES???

\medskip
\examplePage{\textit{Bb}}

  \bigskip
\exampleLib{Biblioteka Czartoryskich. Kraków.}


\bigskip
\exampleRef{\textit{Estreicher XXVII 14. Wierzbowski 832.}}

\bigskip
\exampleDig{\url{https://cyfrowe.mnk.pl/dlibra/publication/edition/30503/content}
  page 18}

%algorismus!

  
  % \medskip
\bigskip

\examplePL{Pismo 6: tekst i drugi zestaw. — Pismo 7: nagłówki i
  pierwszy zestaw. — Rubryka \alpha{}: z pismem 6. — Cyfry 2: z pismem
  6.}

\medskip

\exampleEN{Font 6: the text and the second font table. — font 7:
  headers and the first font table. — Rubric \alpha{}: with font 6. —
  Digits 2: with font 6.}



\bigskip

\fontID{Ho-06}{37}

\fontstat{113}

% \exdisplay \bg \gla
 \exdisplay \bg \gla
% 1
{\PTglyph{5}{t37_l01g01.png}}
% 2
{\PTglyph{5}{t37_l01g02.png}}
% 3
{\PTglyph{5}{t37_l01g03.png}}
% 4
{\PTglyph{5}{t37_l01g04.png}}
% 5
{\PTglyph{5}{t37_l01g05.png}}
% 6
{\PTglyph{5}{t37_l01g06.png}}
% 7
{\PTglyph{5}{t37_l01g07.png}}
% 8
{\PTglyph{5}{t37_l01g08.png}}
% 9
{\PTglyph{5}{t37_l01g09.png}}
% 10
{\PTglyph{5}{t37_l01g10.png}}
% 11
{\PTglyph{5}{t37_l01g11.png}}
% 12
{\PTglyph{5}{t37_l01g12.png}}
% 13
{\PTglyph{5}{t37_l01g13.png}}
% 14
{\PTglyph{5}{t37_l01g14.png}}
% 15
{\PTglyph{5}{t37_l01g15.png}}
% 16
{\PTglyph{5}{t37_l01g16.png}}
% 17
{\PTglyph{5}{t37_l01g17.png}}
% 18
{\PTglyph{5}{t37_l01g18.png}}
% 19
{\PTglyph{5}{t37_l01g19.png}}
% 20
{\PTglyph{5}{t37_l01g20.png}}
% 21
{\PTglyph{5}{t37_l01g21.png}}
% 22
{\PTglyph{5}{t37_l01g22.png}}
% 23
{\PTglyph{5}{t37_l01g23.png}}
% 24
{\PTglyph{5}{t37_l01g24.png}}
% 25
{\PTglyph{5}{t37_l01g25.png}}
% 26
{\PTglyph{5}{t37_l02g01.png}}
% 27
{\PTglyph{5}{t37_l02g02.png}}
% 28
{\PTglyph{5}{t37_l02g03.png}}
% 29
{\PTglyph{5}{t37_l02g04.png}}
% 30
{\PTglyph{5}{t37_l02g05.png}}
% 31
{\PTglyph{5}{t37_l02g06.png}}
% 32
{\PTglyph{5}{t37_l02g07.png}}
% 33
{\PTglyph{5}{t37_l02g08.png}}
% 34
{\PTglyph{5}{t37_l02g09.png}}
% 35
{\PTglyph{5}{t37_l02g10.png}}
% 36
{\PTglyph{5}{t37_l02g11.png}}
% 37
{\PTglyph{5}{t37_l02g12.png}}
% 38
{\PTglyph{5}{t37_l02g13.png}}
% 39
{\PTglyph{5}{t37_l02g14.png}}
% 40
{\PTglyph{5}{t37_l02g15.png}}
% 41
{\PTglyph{5}{t37_l02g16.png}}
% 42
{\PTglyph{5}{t37_l02g17.png}}
% 43
{\PTglyph{5}{t37_l02g18.png}}
% 44
{\PTglyph{5}{t37_l02g19.png}}
% 45
{\PTglyph{5}{t37_l02g20.png}}
% 46
{\PTglyph{5}{t37_l02g21.png}}
% 47
{\PTglyph{5}{t37_l02g22.png}}
% 48
{\PTglyph{5}{t37_l02g23.png}}
% 49
{\PTglyph{5}{t37_l02g24.png}}
% 50
{\PTglyph{5}{t37_l02g25.png}}
% 51
{\PTglyph{5}{t37_l02g26.png}}
% 52
{\PTglyph{5}{t37_l02g27.png}}
% 53
{\PTglyph{5}{t37_l02g28.png}}
% 54
{\PTglyph{5}{t37_l02g29.png}}
% 55
{\PTglyph{5}{t37_l02g30.png}}
% 56
{\PTglyph{5}{t37_l02g31.png}}
% 57
{\PTglyph{5}{t37_l02g32.png}}
% 58
{\PTglyph{5}{t37_l02g33.png}}
% 59
{\PTglyph{5}{t37_l02g34.png}}
% 60
{\PTglyph{5}{t37_l02g35.png}}
% 61
{\PTglyph{5}{t37_l02g36.png}}
% 62
{\PTglyph{5}{t37_l02g37.png}}
% 63
{\PTglyph{5}{t37_l02g38.png}}
% 64
{\PTglyph{5}{t37_l02g39.png}}
% 65
{\PTglyph{5}{t37_l03g01.png}}
% 66
{\PTglyph{5}{t37_l03g02.png}}
% 67
{\PTglyph{5}{t37_l03g03.png}}
% 68
{\PTglyph{5}{t37_l03g04.png}}
% 69
{\PTglyph{5}{t37_l03g05.png}}
% 70
{\PTglyph{5}{t37_l03g06.png}}
% 71
{\PTglyph{5}{t37_l03g07.png}}
% 72
{\PTglyph{5}{t37_l03g08.png}}
% 73
{\PTglyph{5}{t37_l03g09.png}}
% 74
{\PTglyph{5}{t37_l03g10.png}}
% 75
{\PTglyph{5}{t37_l03g11.png}}
% 76
{\PTglyph{5}{t37_l03g12.png}}
% 77
{\PTglyph{5}{t37_l03g13.png}}
% 78
{\PTglyph{5}{t37_l03g14.png}}
% 79
{\PTglyph{5}{t37_l03g15.png}}
% 80
{\PTglyph{5}{t37_l03g16.png}}
% 81
{\PTglyph{5}{t37_l03g17.png}}
% 82
{\PTglyph{5}{t37_l03g18.png}}
% 83
{\PTglyph{5}{t37_l03g19.png}}
% 84
{\PTglyph{5}{t37_l03g20.png}}
% 85
{\PTglyph{5}{t37_l03g21.png}}
% 86
{\PTglyph{5}{t37_l03g22.png}}
% 87
{\PTglyph{5}{t37_l03g23.png}}
% 88
{\PTglyph{5}{t37_l03g24.png}}
% 89
{\PTglyph{5}{t37_l03g25.png}}
% 90
{\PTglyph{5}{t37_l03g26.png}}
% 91
{\PTglyph{5}{t37_l03g27.png}}
% 92
{\PTglyph{5}{t37_l03g28.png}}
% 93
{\PTglyph{5}{t37_l03g29.png}}
% 94
{\PTglyph{5}{t37_l03g30.png}}
% 95
{\PTglyph{5}{t37_l03g31.png}}
% 96
{\PTglyph{5}{t37_l03g32.png}}
% 97
{\PTglyph{5}{t37_l03g33.png}}
% 98
{\PTglyph{5}{t37_l03g34.png}}
% 99
{\PTglyph{5}{t37_l03g35.png}}
% 100
{\PTglyph{5}{t37_l03g36.png}}
% 101
{\PTglyph{5}{t37_l03g37.png}}
% 102
{\PTglyph{5}{t37_l04g01.png}}
% 103
{\PTglyph{5}{t37_l04g02.png}}
% 104
{\PTglyph{5}{t37_l04g03.png}}
% 105
{\PTglyph{5}{t37_l04g04.png}}
% 106
{\PTglyph{5}{t37_l04g05.png}}
% 107
{\PTglyph{5}{t37_l04g06.png}}
% 108
{\PTglyph{5}{t37_l04g07.png}}
% 109
{\PTglyph{5}{t37_l04g08.png}}
% 110
{\PTglyph{5}{t37_l04g09.png}}
% 111
{\PTglyph{5}{t37_l04g10.png}}
% 112
{\PTglyph{5}{t37_l04g11.png}}
% 113
{\PTglyph{5}{t37_l04g12.png}}
//
%%% Local Variables:
%%% mode: latex
%%% TeX-engine: luatex
%%% TeX-master: shared
%%% End:

%//
%\glpismo%
 \glpismo
% 1
{\PTglyphid{Ho-06_0101}}
% 2
{\PTglyphid{Ho-06_0102}}
% 3
{\PTglyphid{Ho-06_0103}}
% 4
{\PTglyphid{Ho-06_0104}}
% 5
{\PTglyphid{Ho-06_0105}}
% 6
{\PTglyphid{Ho-06_0106}}
% 7
{\PTglyphid{Ho-06_0107}}
% 8
{\PTglyphid{Ho-06_0108}}
% 9
{\PTglyphid{Ho-06_0109}}
% 10
{\PTglyphid{Ho-06_0110}}
% 11
{\PTglyphid{Ho-06_0111}}
% 12
{\PTglyphid{Ho-06_0112}}
% 13
{\PTglyphid{Ho-06_0113}}
% 14
{\PTglyphid{Ho-06_0114}}
% 15
{\PTglyphid{Ho-06_0115}}
% 16
{\PTglyphid{Ho-06_0116}}
% 17
{\PTglyphid{Ho-06_0117}}
% 18
{\PTglyphid{Ho-06_0118}}
% 19
{\PTglyphid{Ho-06_0119}}
% 20
{\PTglyphid{Ho-06_0120}}
% 21
{\PTglyphid{Ho-06_0121}}
% 22
{\PTglyphid{Ho-06_0122}}
% 23
{\PTglyphid{Ho-06_0123}}
% 24
{\PTglyphid{Ho-06_0124}}
% 25
{\PTglyphid{Ho-06_0125}}
% 26
{\PTglyphid{Ho-06_0201}}
% 27
{\PTglyphid{Ho-06_0202}}
% 28
{\PTglyphid{Ho-06_0203}}
% 29
{\PTglyphid{Ho-06_0204}}
% 30
{\PTglyphid{Ho-06_0205}}
% 31
{\PTglyphid{Ho-06_0206}}
% 32
{\PTglyphid{Ho-06_0207}}
% 33
{\PTglyphid{Ho-06_0208}}
% 34
{\PTglyphid{Ho-06_0209}}
% 35
{\PTglyphid{Ho-06_0210}}
% 36
{\PTglyphid{Ho-06_0211}}
% 37
{\PTglyphid{Ho-06_0212}}
% 38
{\PTglyphid{Ho-06_0213}}
% 39
{\PTglyphid{Ho-06_0214}}
% 40
{\PTglyphid{Ho-06_0215}}
% 41
{\PTglyphid{Ho-06_0216}}
% 42
{\PTglyphid{Ho-06_0217}}
% 43
{\PTglyphid{Ho-06_0218}}
% 44
{\PTglyphid{Ho-06_0219}}
% 45
{\PTglyphid{Ho-06_0220}}
% 46
{\PTglyphid{Ho-06_0221}}
% 47
{\PTglyphid{Ho-06_0222}}
% 48
{\PTglyphid{Ho-06_0223}}
% 49
{\PTglyphid{Ho-06_0224}}
% 50
{\PTglyphid{Ho-06_0225}}
% 51
{\PTglyphid{Ho-06_0226}}
% 52
{\PTglyphid{Ho-06_0227}}
% 53
{\PTglyphid{Ho-06_0228}}
% 54
{\PTglyphid{Ho-06_0229}}
% 55
{\PTglyphid{Ho-06_0230}}
% 56
{\PTglyphid{Ho-06_0231}}
% 57
{\PTglyphid{Ho-06_0232}}
% 58
{\PTglyphid{Ho-06_0233}}
% 59
{\PTglyphid{Ho-06_0234}}
% 60
{\PTglyphid{Ho-06_0235}}
% 61
{\PTglyphid{Ho-06_0236}}
% 62
{\PTglyphid{Ho-06_0237}}
% 63
{\PTglyphid{Ho-06_0238}}
% 64
{\PTglyphid{Ho-06_0239}}
% 65
{\PTglyphid{Ho-06_0301}}
% 66
{\PTglyphid{Ho-06_0302}}
% 67
{\PTglyphid{Ho-06_0303}}
% 68
{\PTglyphid{Ho-06_0304}}
% 69
{\PTglyphid{Ho-06_0305}}
% 70
{\PTglyphid{Ho-06_0306}}
% 71
{\PTglyphid{Ho-06_0307}}
% 72
{\PTglyphid{Ho-06_0308}}
% 73
{\PTglyphid{Ho-06_0309}}
% 74
{\PTglyphid{Ho-06_0310}}
% 75
{\PTglyphid{Ho-06_0311}}
% 76
{\PTglyphid{Ho-06_0312}}
% 77
{\PTglyphid{Ho-06_0313}}
% 78
{\PTglyphid{Ho-06_0314}}
% 79
{\PTglyphid{Ho-06_0315}}
% 80
{\PTglyphid{Ho-06_0316}}
% 81
{\PTglyphid{Ho-06_0317}}
% 82
{\PTglyphid{Ho-06_0318}}
% 83
{\PTglyphid{Ho-06_0319}}
% 84
{\PTglyphid{Ho-06_0320}}
% 85
{\PTglyphid{Ho-06_0321}}
% 86
{\PTglyphid{Ho-06_0322}}
% 87
{\PTglyphid{Ho-06_0323}}
% 88
{\PTglyphid{Ho-06_0324}}
% 89
{\PTglyphid{Ho-06_0325}}
% 90
{\PTglyphid{Ho-06_0326}}
% 91
{\PTglyphid{Ho-06_0327}}
% 92
{\PTglyphid{Ho-06_0328}}
% 93
{\PTglyphid{Ho-06_0329}}
% 94
{\PTglyphid{Ho-06_0330}}
% 95
{\PTglyphid{Ho-06_0331}}
% 96
{\PTglyphid{Ho-06_0332}}
% 97
{\PTglyphid{Ho-06_0333}}
% 98
{\PTglyphid{Ho-06_0334}}
% 99
{\PTglyphid{Ho-06_0335}}
% 100
{\PTglyphid{Ho-06_0336}}
% 101
{\PTglyphid{Ho-06_0337}}
% 102
{\PTglyphid{Ho-06_0401}}
% 103
{\PTglyphid{Ho-06_0402}}
% 104
{\PTglyphid{Ho-06_0403}}
% 105
{\PTglyphid{Ho-06_0404}}
% 106
{\PTglyphid{Ho-06_0405}}
% 107
{\PTglyphid{Ho-06_0406}}
% 108
{\PTglyphid{Ho-06_0407}}
% 109
{\PTglyphid{Ho-06_0408}}
% 110
{\PTglyphid{Ho-06_0409}}
% 111
{\PTglyphid{Ho-06_0410}}
% 112
{\PTglyphid{Ho-06_0411}}
% 113
{\PTglyphid{Ho-06_0412}}
//
\endgl \xe
%%% Local Variables:
%%% mode: latex
%%% TeX-engine: luatex
%%% TeX-master: shared
%%% End:

% //
%\endgl \xe

\newpage

%%%%%%%%%%%%%%%%%%%%%%%%%%%%%%%%%%%%%%%%%%%%%%%%%%%%%%%%%%%%%%%%%%%%%%%%%%%%%%%
% from meta.csv
 % Tab. 38,Hochfeder-07_PT01_024.djvu,Hochfeder,07,01,024
%%%%%%%%%%%%%%%%%%%%%%%%%%%%%%%%%%%%%%%%%%%%%%%%%%%%%%%%%%%%%%%%%%%%%%%%%%%%%%%

 % from dsed4test:

% Note "7. Pismo mszalne gotyckie. Krój M¹⁸. Stopień 20 ww. = 154/156 mm - Tabl. 24, 25. (Występuje u Hallera jako pismo 2. Tabl.165; u Unglera jako pismo13. Tabl. 72).  [24]"
% Note1 "Character set table prepared by Maria Błońska and Anna Wolińska"


\pismoPL{Kasper Hochfeder 7. Pismo mszalne gotyckie. Krój M¹⁸. Stopień
  20 ww. = 154/156 mm - Tabl. 24, 25. (Występuje u Hallera jako pismo
  2. Tabl.165; u Unglera jako pismo13. Tabl. 72).}
  
\pismoEN{Kasper Hochfeder 7. Gothic missal font. Typeface M¹⁸. Type size 20 lines = 154/156 mm - Plate 24, 25. (used by Haller as font 2. Plate 165; by Ungler as font 13. Plate 72).}

\plate{24}{I}{1968}

Prepared by Kazimierz Piekarski and  Maria Błońska.\\
The font table prepared by Maria Błońska and Anna Wolińska.

\bigskip

\exampleBib{I:24}

\bigskip \exampleDesc{IOANNES[sic] DE SACROBOSCO: Algorithmus. Kraków, [Kasper Hochfeder]. 1504. 4⁰.}
IOHANNES???

\medskip
\examplePage{\textit{Bb}}

  \bigskip
\exampleLib{Biblioteka Czartoryskich. Kraków.}


\bigskip
\exampleRef{\textit{Estreicher XXVII 14. Wierzbowski 832.}}

\bigskip
\exampleDig{\url{https://cyfrowe.mnk.pl/dlibra/publication/edition/30503/content}
  page 18}

%algorismus!

  
  % \medskip
\bigskip

\examplePL{Pismo 6: tekst i drugi zestaw. — Pismo 7: nagłówki i
  pierwszy zestaw. — Rubryka \alpha{}: z pismem 6. — Cyfry 2: z pismem
  6.}

\medskip

\exampleEN{Font 6: the text and the second font table. — font 7:
  headers and the first font table. — Rubric \alpha{}: with font 6. —
  Digits 2: with font 6.}



\bigskip

\fontID{Ho-07}{38}

\fontstat{109}

% \exdisplay \bg \gla
 \exdisplay \bg \gla
% 1
{\PTglyph{5}{t37_l01g01.png}}
% 2
{\PTglyph{5}{t37_l01g02.png}}
% 3
{\PTglyph{5}{t37_l01g03.png}}
% 4
{\PTglyph{5}{t37_l01g04.png}}
% 5
{\PTglyph{5}{t37_l01g05.png}}
% 6
{\PTglyph{5}{t37_l01g06.png}}
% 7
{\PTglyph{5}{t37_l01g07.png}}
% 8
{\PTglyph{5}{t37_l01g08.png}}
% 9
{\PTglyph{5}{t37_l01g09.png}}
% 10
{\PTglyph{5}{t37_l01g10.png}}
% 11
{\PTglyph{5}{t37_l01g11.png}}
% 12
{\PTglyph{5}{t37_l01g12.png}}
% 13
{\PTglyph{5}{t37_l01g13.png}}
% 14
{\PTglyph{5}{t37_l01g14.png}}
% 15
{\PTglyph{5}{t37_l01g15.png}}
% 16
{\PTglyph{5}{t37_l01g16.png}}
% 17
{\PTglyph{5}{t37_l01g17.png}}
% 18
{\PTglyph{5}{t37_l01g18.png}}
% 19
{\PTglyph{5}{t37_l01g19.png}}
% 20
{\PTglyph{5}{t37_l01g20.png}}
% 21
{\PTglyph{5}{t37_l01g21.png}}
% 22
{\PTglyph{5}{t37_l01g22.png}}
% 23
{\PTglyph{5}{t37_l01g23.png}}
% 24
{\PTglyph{5}{t37_l01g24.png}}
% 25
{\PTglyph{5}{t37_l01g25.png}}
% 26
{\PTglyph{5}{t37_l02g01.png}}
% 27
{\PTglyph{5}{t37_l02g02.png}}
% 28
{\PTglyph{5}{t37_l02g03.png}}
% 29
{\PTglyph{5}{t37_l02g04.png}}
% 30
{\PTglyph{5}{t37_l02g05.png}}
% 31
{\PTglyph{5}{t37_l02g06.png}}
% 32
{\PTglyph{5}{t37_l02g07.png}}
% 33
{\PTglyph{5}{t37_l02g08.png}}
% 34
{\PTglyph{5}{t37_l02g09.png}}
% 35
{\PTglyph{5}{t37_l02g10.png}}
% 36
{\PTglyph{5}{t37_l02g11.png}}
% 37
{\PTglyph{5}{t37_l02g12.png}}
% 38
{\PTglyph{5}{t37_l02g13.png}}
% 39
{\PTglyph{5}{t37_l02g14.png}}
% 40
{\PTglyph{5}{t37_l02g15.png}}
% 41
{\PTglyph{5}{t37_l02g16.png}}
% 42
{\PTglyph{5}{t37_l02g17.png}}
% 43
{\PTglyph{5}{t37_l02g18.png}}
% 44
{\PTglyph{5}{t37_l02g19.png}}
% 45
{\PTglyph{5}{t37_l02g20.png}}
% 46
{\PTglyph{5}{t37_l02g21.png}}
% 47
{\PTglyph{5}{t37_l02g22.png}}
% 48
{\PTglyph{5}{t37_l02g23.png}}
% 49
{\PTglyph{5}{t37_l02g24.png}}
% 50
{\PTglyph{5}{t37_l02g25.png}}
% 51
{\PTglyph{5}{t37_l02g26.png}}
% 52
{\PTglyph{5}{t37_l02g27.png}}
% 53
{\PTglyph{5}{t37_l02g28.png}}
% 54
{\PTglyph{5}{t37_l02g29.png}}
% 55
{\PTglyph{5}{t37_l02g30.png}}
% 56
{\PTglyph{5}{t37_l02g31.png}}
% 57
{\PTglyph{5}{t37_l02g32.png}}
% 58
{\PTglyph{5}{t37_l02g33.png}}
% 59
{\PTglyph{5}{t37_l02g34.png}}
% 60
{\PTglyph{5}{t37_l02g35.png}}
% 61
{\PTglyph{5}{t37_l02g36.png}}
% 62
{\PTglyph{5}{t37_l02g37.png}}
% 63
{\PTglyph{5}{t37_l02g38.png}}
% 64
{\PTglyph{5}{t37_l02g39.png}}
% 65
{\PTglyph{5}{t37_l02g40.png}}
% 66
{\PTglyph{5}{t37_l02g41.png}}
% 67
{\PTglyph{5}{t37_l02g42.png}}
% 68
{\PTglyph{5}{t37_l02g43.png}}
% 69
{\PTglyph{5}{t37_l02g44.png}}
% 70
{\PTglyph{5}{t37_l03g01.png}}
% 71
{\PTglyph{5}{t37_l03g02.png}}
% 72
{\PTglyph{5}{t37_l03g03.png}}
% 73
{\PTglyph{5}{t37_l03g04.png}}
% 74
{\PTglyph{5}{t37_l03g05.png}}
% 75
{\PTglyph{5}{t37_l03g06.png}}
% 76
{\PTglyph{5}{t37_l03g07.png}}
% 77
{\PTglyph{5}{t37_l03g08.png}}
% 78
{\PTglyph{5}{t37_l03g09.png}}
% 79
{\PTglyph{5}{t37_l03g10.png}}
% 80
{\PTglyph{5}{t37_l03g11.png}}
% 81
{\PTglyph{5}{t37_l03g12.png}}
% 82
{\PTglyph{5}{t37_l03g13.png}}
% 83
{\PTglyph{5}{t37_l03g14.png}}
% 84
{\PTglyph{5}{t37_l03g15.png}}
% 85
{\PTglyph{5}{t37_l03g16.png}}
% 86
{\PTglyph{5}{t37_l03g17.png}}
% 87
{\PTglyph{5}{t37_l03g18.png}}
% 88
{\PTglyph{5}{t37_l03g19.png}}
% 89
{\PTglyph{5}{t37_l03g20.png}}
% 90
{\PTglyph{5}{t37_l03g21.png}}
% 91
{\PTglyph{5}{t37_l03g22.png}}
% 92
{\PTglyph{5}{t37_l03g23.png}}
% 93
{\PTglyph{5}{t37_l03g24.png}}
% 94
{\PTglyph{5}{t37_l03g25.png}}
% 95
{\PTglyph{5}{t37_l03g26.png}}
% 96
{\PTglyph{5}{t37_l03g27.png}}
% 97
{\PTglyph{5}{t37_l03g28.png}}
% 98
{\PTglyph{5}{t37_l03g29.png}}
% 99
{\PTglyph{5}{t37_l03g30.png}}
% 100
{\PTglyph{5}{t37_l03g31.png}}
% 101
{\PTglyph{5}{t37_l03g32.png}}
% 102
{\PTglyph{5}{t37_l03g33.png}}
% 103
{\PTglyph{5}{t37_l03g34.png}}
% 104
{\PTglyph{5}{t37_l03g35.png}}
% 105
{\PTglyph{5}{t37_l03g36.png}}
% 106
{\PTglyph{5}{t37_l03g37.png}}
% 107
{\PTglyph{5}{t37_l03g38.png}}
% 108
{\PTglyph{5}{t37_l03g39.png}}
% 109
{\PTglyph{5}{t37_l03g40.png}}
//
%%% Local Variables:
%%% mode: latex
%%% TeX-engine: luatex
%%% TeX-master: shared
%%% End:

%//
%\glpismo%
 \glpismo
% 1
{\PTglyphid{Ho-07_0101}}
% 2
{\PTglyphid{Ho-07_0102}}
% 3
{\PTglyphid{Ho-07_0103}}
% 4
{\PTglyphid{Ho-07_0104}}
% 5
{\PTglyphid{Ho-07_0105}}
% 6
{\PTglyphid{Ho-07_0106}}
% 7
{\PTglyphid{Ho-07_0107}}
% 8
{\PTglyphid{Ho-07_0108}}
% 9
{\PTglyphid{Ho-07_0109}}
% 10
{\PTglyphid{Ho-07_0110}}
% 11
{\PTglyphid{Ho-07_0111}}
% 12
{\PTglyphid{Ho-07_0112}}
% 13
{\PTglyphid{Ho-07_0113}}
% 14
{\PTglyphid{Ho-07_0114}}
% 15
{\PTglyphid{Ho-07_0115}}
% 16
{\PTglyphid{Ho-07_0116}}
% 17
{\PTglyphid{Ho-07_0117}}
% 18
{\PTglyphid{Ho-07_0118}}
% 19
{\PTglyphid{Ho-07_0119}}
% 20
{\PTglyphid{Ho-07_0120}}
% 21
{\PTglyphid{Ho-07_0121}}
% 22
{\PTglyphid{Ho-07_0122}}
% 23
{\PTglyphid{Ho-07_0123}}
% 24
{\PTglyphid{Ho-07_0124}}
% 25
{\PTglyphid{Ho-07_0125}}
% 26
{\PTglyphid{Ho-07_0201}}
% 27
{\PTglyphid{Ho-07_0202}}
% 28
{\PTglyphid{Ho-07_0203}}
% 29
{\PTglyphid{Ho-07_0204}}
% 30
{\PTglyphid{Ho-07_0205}}
% 31
{\PTglyphid{Ho-07_0206}}
% 32
{\PTglyphid{Ho-07_0207}}
% 33
{\PTglyphid{Ho-07_0208}}
% 34
{\PTglyphid{Ho-07_0209}}
% 35
{\PTglyphid{Ho-07_0210}}
% 36
{\PTglyphid{Ho-07_0211}}
% 37
{\PTglyphid{Ho-07_0212}}
% 38
{\PTglyphid{Ho-07_0213}}
% 39
{\PTglyphid{Ho-07_0214}}
% 40
{\PTglyphid{Ho-07_0215}}
% 41
{\PTglyphid{Ho-07_0216}}
% 42
{\PTglyphid{Ho-07_0217}}
% 43
{\PTglyphid{Ho-07_0218}}
% 44
{\PTglyphid{Ho-07_0219}}
% 45
{\PTglyphid{Ho-07_0220}}
% 46
{\PTglyphid{Ho-07_0221}}
% 47
{\PTglyphid{Ho-07_0222}}
% 48
{\PTglyphid{Ho-07_0223}}
% 49
{\PTglyphid{Ho-07_0224}}
% 50
{\PTglyphid{Ho-07_0225}}
% 51
{\PTglyphid{Ho-07_0226}}
% 52
{\PTglyphid{Ho-07_0227}}
% 53
{\PTglyphid{Ho-07_0228}}
% 54
{\PTglyphid{Ho-07_0229}}
% 55
{\PTglyphid{Ho-07_0230}}
% 56
{\PTglyphid{Ho-07_0231}}
% 57
{\PTglyphid{Ho-07_0232}}
% 58
{\PTglyphid{Ho-07_0233}}
% 59
{\PTglyphid{Ho-07_0234}}
% 60
{\PTglyphid{Ho-07_0235}}
% 61
{\PTglyphid{Ho-07_0236}}
% 62
{\PTglyphid{Ho-07_0237}}
% 63
{\PTglyphid{Ho-07_0238}}
% 64
{\PTglyphid{Ho-07_0239}}
% 65
{\PTglyphid{Ho-07_0240}}
% 66
{\PTglyphid{Ho-07_0241}}
% 67
{\PTglyphid{Ho-07_0242}}
% 68
{\PTglyphid{Ho-07_0243}}
% 69
{\PTglyphid{Ho-07_0244}}
% 70
{\PTglyphid{Ho-07_0301}}
% 71
{\PTglyphid{Ho-07_0302}}
% 72
{\PTglyphid{Ho-07_0303}}
% 73
{\PTglyphid{Ho-07_0304}}
% 74
{\PTglyphid{Ho-07_0305}}
% 75
{\PTglyphid{Ho-07_0306}}
% 76
{\PTglyphid{Ho-07_0307}}
% 77
{\PTglyphid{Ho-07_0308}}
% 78
{\PTglyphid{Ho-07_0309}}
% 79
{\PTglyphid{Ho-07_0310}}
% 80
{\PTglyphid{Ho-07_0311}}
% 81
{\PTglyphid{Ho-07_0312}}
% 82
{\PTglyphid{Ho-07_0313}}
% 83
{\PTglyphid{Ho-07_0314}}
% 84
{\PTglyphid{Ho-07_0315}}
% 85
{\PTglyphid{Ho-07_0316}}
% 86
{\PTglyphid{Ho-07_0317}}
% 87
{\PTglyphid{Ho-07_0318}}
% 88
{\PTglyphid{Ho-07_0319}}
% 89
{\PTglyphid{Ho-07_0320}}
% 90
{\PTglyphid{Ho-07_0321}}
% 91
{\PTglyphid{Ho-07_0322}}
% 92
{\PTglyphid{Ho-07_0323}}
% 93
{\PTglyphid{Ho-07_0324}}
% 94
{\PTglyphid{Ho-07_0325}}
% 95
{\PTglyphid{Ho-07_0326}}
% 96
{\PTglyphid{Ho-07_0327}}
% 97
{\PTglyphid{Ho-07_0328}}
% 98
{\PTglyphid{Ho-07_0329}}
% 99
{\PTglyphid{Ho-07_0330}}
% 100
{\PTglyphid{Ho-07_0331}}
% 101
{\PTglyphid{Ho-07_0332}}
% 102
{\PTglyphid{Ho-07_0333}}
% 103
{\PTglyphid{Ho-07_0334}}
% 104
{\PTglyphid{Ho-07_0335}}
% 105
{\PTglyphid{Ho-07_0336}}
% 106
{\PTglyphid{Ho-07_0337}}
% 107
{\PTglyphid{Ho-07_0338}}
% 108
{\PTglyphid{Ho-07_0339}}
% 109
{\PTglyphid{Ho-07_0340}}
//
\endgl \xe
%%% Local Variables:
%%% mode: latex
%%% TeX-engine: luatex
%%% TeX-master: shared
%%% End:

% //
%\endgl \xe

%%%%%%%%%%%%%%%%%%%%%%%%%%%%%%%%%%%%%%%%%%%%%%%%%%%%%%%%%%%%%%%%%%%%%%%%%%%%%%%

 split poprawione
 
%%%%%%%%%%%%%%%%%%%%%%%%%%%%%%%%%%%%%%%%%%%%%%%%%%%%%%%%%%%%%%%%%%%%%%%%%%%%%%
 
 \newpage
 
%%%%%%%%%%%%%%%%%%%%%%%%%%%%%%%%%%%%%%%%%%%%%%%%%%%%%%%%%%%%%%%%%%%%%%%%%%%%%%%
% from meta.csv
 % Tab. 39,Hochfeder,09,01,027
%%%%%%%%%%%%%%%%%%%%%%%%%%%%%%%%%%%%%%%%%%%%%%%%%%%%%%%%%%%%%%%%%%%%%%%%%%%%%%%

 % from dsed4test:

% Note "9. Pismo tekstowe gotyckie. Krój M⁴⁸. Stopień 20 ww. = 81/82 mm - Tabl. 27. (Występuje u Hallera jako pismo 5. Tabl. 168; u Unglera jako pismo4. Tabl. 115).  [27]"
% Note1 "Character set table prepared by Maria Błońska and Anna Wolińska"

 \pismoPL{Kasper Hochfeder 9. Pismo tekstowe gotyckie. Krój
   M⁴⁸. Stopień 20 ww. = 81/82 mm - Tabl. 27. (Występuje u Hallera
   jako pismo 5. Tabl. 168; u Unglera jako pismo 4. Tabl. 115).}
  
 \pismoEN{Kasper Hochfeder 9. Gothic text font. Typeface M⁴⁸. Type
   size 20 lines = 81/82 mm - Plate 27. (used by Haller as font
   5. Plate 168; by Ungler as font 4. Plate 115).}

\plate{27}{I}{1968}

Prepared by Kazimierz Piekarski and  Maria Błońska.\\
The font table prepared by Maria Błońska and Anna Wolińska.

\bigskip

\exampleBib{I:15}

\bigskip \exampleDesc{IOANNES DE DOBCZYCE: Opusculum de arte memorativa. Kraków, [Kasper Hochfeder], 13 IX 1504. 4⁰}

\medskip
\examplePage{\textit{[Karta] c₃b}}

  \bigskip
\exampleLib{Biblioteka Jagiellońska. Kraków.}


\bigskip
\exampleRef{\textit{Estreicher XV 255, Wierzbowski 831.}}

\bigskip
\exampleDig{\url{ https://polona.pl/preview/aa8a86f1-23fe-4ef8-b9f4-b36499c4a351}
  page 263,\\ \url{https://www.wbc.poznan.pl/dlibra/publication/310297}
  page 42}

%algorismus!

  
  % \medskip
\bigskip

\examplePL{[Pismo 9] Rubryka \epsilon{}}

\medskip

\exampleEN{[Font 9] Rubric \epsilon{}}



\bigskip

\fontID{Ho-09}{39}

\fontstat{111}

% \exdisplay \bg \gla
 \exdisplay \bg \gla
% 1
{\PTglyph{5}{t39_l01g01.png}}
% 2
{\PTglyph{5}{t39_l01g02.png}}
% 3
{\PTglyph{5}{t39_l01g03.png}}
% 4
{\PTglyph{5}{t39_l01g04.png}}
% 5
{\PTglyph{5}{t39_l01g05.png}}
% 6
{\PTglyph{5}{t39_l01g06.png}}
% 7
{\PTglyph{5}{t39_l01g07.png}}
% 8
{\PTglyph{5}{t39_l01g08.png}}
% 9
{\PTglyph{5}{t39_l01g09.png}}
% 10
{\PTglyph{5}{t39_l01g10.png}}
% 11
{\PTglyph{5}{t39_l01g11.png}}
% 12
{\PTglyph{5}{t39_l01g12.png}}
% 13
{\PTglyph{5}{t39_l01g13.png}}
% 14
{\PTglyph{5}{t39_l01g14.png}}
% 15
{\PTglyph{5}{t39_l01g15.png}}
% 16
{\PTglyph{5}{t39_l01g16.png}}
% 17
{\PTglyph{5}{t39_l01g17.png}}
% 18
{\PTglyph{5}{t39_l01g18.png}}
% 19
{\PTglyph{5}{t39_l01g19.png}}
% 20
{\PTglyph{5}{t39_l01g20.png}}
% 21
{\PTglyph{5}{t39_l01g21.png}}
% 22
{\PTglyph{5}{t39_l02g01.png}}
% 23
{\PTglyph{5}{t39_l02g02.png}}
% 24
{\PTglyph{5}{t39_l02g03.png}}
% 25
{\PTglyph{5}{t39_l02g04.png}}
% 26
{\PTglyph{5}{t39_l02g05.png}}
% 27
{\PTglyph{5}{t39_l02g06.png}}
% 28
{\PTglyph{5}{t39_l02g07.png}}
% 29
{\PTglyph{5}{t39_l02g08.png}}
% 30
{\PTglyph{5}{t39_l02g09.png}}
% 31
{\PTglyph{5}{t39_l02g10.png}}
% 32
{\PTglyph{5}{t39_l02g11.png}}
% 33
{\PTglyph{5}{t39_l02g12.png}}
% 34
{\PTglyph{5}{t39_l02g13.png}}
% 35
{\PTglyph{5}{t39_l02g14.png}}
% 36
{\PTglyph{5}{t39_l02g15.png}}
% 37
{\PTglyph{5}{t39_l02g16.png}}
% 38
{\PTglyph{5}{t39_l02g17.png}}
% 39
{\PTglyph{5}{t39_l02g18.png}}
% 40
{\PTglyph{5}{t39_l02g19.png}}
% 41
{\PTglyph{5}{t39_l02g20.png}}
% 42
{\PTglyph{5}{t39_l02g21.png}}
% 43
{\PTglyph{5}{t39_l02g22.png}}
% 44
{\PTglyph{5}{t39_l02g23.png}}
% 45
{\PTglyph{5}{t39_l02g24.png}}
% 46
{\PTglyph{5}{t39_l02g25.png}}
% 47
{\PTglyph{5}{t39_l02g26.png}}
% 48
{\PTglyph{5}{t39_l02g27.png}}
% 49
{\PTglyph{5}{t39_l02g28.png}}
% 50
{\PTglyph{5}{t39_l02g29.png}}
% 51
{\PTglyph{5}{t39_l02g30.png}}
% 52
{\PTglyph{5}{t39_l02g31.png}}
% 53
{\PTglyph{5}{t39_l02g32.png}}
% 54
{\PTglyph{5}{t39_l02g33.png}}
% 55
{\PTglyph{5}{t39_l03g01.png}}
% 56
{\PTglyph{5}{t39_l03g02.png}}
% 57
{\PTglyph{5}{t39_l03g03.png}}
% 58
{\PTglyph{5}{t39_l03g04.png}}
% 59
{\PTglyph{5}{t39_l03g05.png}}
% 60
{\PTglyph{5}{t39_l03g06.png}}
% 61
{\PTglyph{5}{t39_l03g07.png}}
% 62
{\PTglyph{5}{t39_l03g08.png}}
% 63
{\PTglyph{5}{t39_l03g09.png}}
% 64
{\PTglyph{5}{t39_l03g10.png}}
% 65
{\PTglyph{5}{t39_l03g11.png}}
% 66
{\PTglyph{5}{t39_l03g12.png}}
% 67
{\PTglyph{5}{t39_l03g13.png}}
% 68
{\PTglyph{5}{t39_l03g14.png}}
% 69
{\PTglyph{5}{t39_l03g15.png}}
% 70
{\PTglyph{5}{t39_l03g16.png}}
% 71
{\PTglyph{5}{t39_l03g17.png}}
% 72
{\PTglyph{5}{t39_l03g18.png}}
% 73
{\PTglyph{5}{t39_l03g19.png}}
% 74
{\PTglyph{5}{t39_l03g20.png}}
% 75
{\PTglyph{5}{t39_l03g21.png}}
% 76
{\PTglyph{5}{t39_l03g22.png}}
% 77
{\PTglyph{5}{t39_l03g23.png}}
% 78
{\PTglyph{5}{t39_l03g24.png}}
% 79
{\PTglyph{5}{t39_l03g25.png}}
% 80
{\PTglyph{5}{t39_l03g26.png}}
% 81
{\PTglyph{5}{t39_l03g27.png}}
% 82
{\PTglyph{5}{t39_l03g28.png}}
% 83
{\PTglyph{5}{t39_l03g29.png}}
% 84
{\PTglyph{5}{t39_l03g30.png}}
% 85
{\PTglyph{5}{t39_l04g01.png}}
% 86
{\PTglyph{5}{t39_l04g02.png}}
% 87
{\PTglyph{5}{t39_l04g03.png}}
% 88
{\PTglyph{5}{t39_l04g04.png}}
% 89
{\PTglyph{5}{t39_l04g05.png}}
% 90
{\PTglyph{5}{t39_l04g06.png}}
% 91
{\PTglyph{5}{t39_l04g07.png}}
% 92
{\PTglyph{5}{t39_l04g08.png}}
% 93
{\PTglyph{5}{t39_l04g09.png}}
% 94
{\PTglyph{5}{t39_l04g10.png}}
% 95
{\PTglyph{5}{t39_l04g11.png}}
% 96
{\PTglyph{5}{t39_l04g12.png}}
% 97
{\PTglyph{5}{t39_l04g13.png}}
% 98
{\PTglyph{5}{t39_l04g14.png}}
% 99
{\PTglyph{5}{t39_l04g15.png}}
% 100
{\PTglyph{5}{t39_l04g16.png}}
% 101
{\PTglyph{5}{t39_l04g17.png}}
% 102
{\PTglyph{5}{t39_l04g18.png}}
% 103
{\PTglyph{5}{t39_l04g19.png}}
% 104
{\PTglyph{5}{t39_l04g20.png}}
% 105
{\PTglyph{5}{t39_l04g21.png}}
% 106
{\PTglyph{5}{t39_l04g22.png}}
% 107
{\PTglyph{5}{t39_l04g23.png}}
% 108
{\PTglyph{5}{t39_l04g24.png}}
% 109
{\PTglyph{5}{t39_l04g25.png}}
% 110
{\PTglyph{5}{t39_l04g26.png}}
% 111
{\PTglyph{5}{t39_l04g27.png}}
% 112
{\PTglyph{5}{t39_l04g28.png}}
//
%%% Local Variables:
%%% mode: latex
%%% TeX-engine: luatex
%%% TeX-master: shared
%%% End:

%//
%\glpismo%
 \glpismo
% 1
{\PTglyphid{09-01_0101}}
% 2
{\PTglyphid{09-01_0102}}
% 3
{\PTglyphid{09-01_0103}}
% 4
{\PTglyphid{09-01_0104}}
% 5
{\PTglyphid{09-01_0105}}
% 6
{\PTglyphid{09-01_0106}}
% 7
{\PTglyphid{09-01_0107}}
% 8
{\PTglyphid{09-01_0108}}
% 9
{\PTglyphid{09-01_0109}}
% 10
{\PTglyphid{09-01_0110}}
% 11
{\PTglyphid{09-01_0111}}
% 12
{\PTglyphid{09-01_0112}}
% 13
{\PTglyphid{09-01_0113}}
% 14
{\PTglyphid{09-01_0114}}
% 15
{\PTglyphid{09-01_0115}}
% 16
{\PTglyphid{09-01_0116}}
% 17
{\PTglyphid{09-01_0117}}
% 18
{\PTglyphid{09-01_0118}}
% 19
{\PTglyphid{09-01_0119}}
% 20
{\PTglyphid{09-01_0120}}
% 21
{\PTglyphid{09-01_0121}}
% 22
{\PTglyphid{09-01_0201}}
% 23
{\PTglyphid{09-01_0202}}
% 24
{\PTglyphid{09-01_0203}}
% 25
{\PTglyphid{09-01_0204}}
% 26
{\PTglyphid{09-01_0205}}
% 27
{\PTglyphid{09-01_0206}}
% 28
{\PTglyphid{09-01_0207}}
% 29
{\PTglyphid{09-01_0208}}
% 30
{\PTglyphid{09-01_0209}}
% 31
{\PTglyphid{09-01_0210}}
% 32
{\PTglyphid{09-01_0211}}
% 33
{\PTglyphid{09-01_0212}}
% 34
{\PTglyphid{09-01_0213}}
% 35
{\PTglyphid{09-01_0214}}
% 36
{\PTglyphid{09-01_0215}}
% 37
{\PTglyphid{09-01_0216}}
% 38
{\PTglyphid{09-01_0217}}
% 39
{\PTglyphid{09-01_0218}}
% 40
{\PTglyphid{09-01_0219}}
% 41
{\PTglyphid{09-01_0220}}
% 42
{\PTglyphid{09-01_0221}}
% 43
{\PTglyphid{09-01_0222}}
% 44
{\PTglyphid{09-01_0223}}
% 45
{\PTglyphid{09-01_0224}}
% 46
{\PTglyphid{09-01_0225}}
% 47
{\PTglyphid{09-01_0226}}
% 48
{\PTglyphid{09-01_0227}}
% 49
{\PTglyphid{09-01_0228}}
% 50
{\PTglyphid{09-01_0229}}
% 51
{\PTglyphid{09-01_0230}}
% 52
{\PTglyphid{09-01_0231}}
% 53
{\PTglyphid{09-01_0232}}
% 54
{\PTglyphid{09-01_0233}}
% 55
{\PTglyphid{09-01_0301}}
% 56
{\PTglyphid{09-01_0302}}
% 57
{\PTglyphid{09-01_0303}}
% 58
{\PTglyphid{09-01_0304}}
% 59
{\PTglyphid{09-01_0305}}
% 60
{\PTglyphid{09-01_0306}}
% 61
{\PTglyphid{09-01_0307}}
% 62
{\PTglyphid{09-01_0308}}
% 63
{\PTglyphid{09-01_0309}}
% 64
{\PTglyphid{09-01_0310}}
% 65
{\PTglyphid{09-01_0311}}
% 66
{\PTglyphid{09-01_0312}}
% 67
{\PTglyphid{09-01_0313}}
% 68
{\PTglyphid{09-01_0314}}
% 69
{\PTglyphid{09-01_0315}}
% 70
{\PTglyphid{09-01_0316}}
% 71
{\PTglyphid{09-01_0317}}
% 72
{\PTglyphid{09-01_0318}}
% 73
{\PTglyphid{09-01_0319}}
% 74
{\PTglyphid{09-01_0320}}
% 75
{\PTglyphid{09-01_0321}}
% 76
{\PTglyphid{09-01_0322}}
% 77
{\PTglyphid{09-01_0323}}
% 78
{\PTglyphid{09-01_0324}}
% 79
{\PTglyphid{09-01_0325}}
% 80
{\PTglyphid{09-01_0326}}
% 81
{\PTglyphid{09-01_0327}}
% 82
{\PTglyphid{09-01_0328}}
% 83
{\PTglyphid{09-01_0329}}
% 84
{\PTglyphid{09-01_0330}}
% 85
{\PTglyphid{09-01_0401}}
% 86
{\PTglyphid{09-01_0402}}
% 87
{\PTglyphid{09-01_0403}}
% 88
{\PTglyphid{09-01_0404}}
% 89
{\PTglyphid{09-01_0405}}
% 90
{\PTglyphid{09-01_0406}}
% 91
{\PTglyphid{09-01_0407}}
% 92
{\PTglyphid{09-01_0408}}
% 93
{\PTglyphid{09-01_0409}}
% 94
{\PTglyphid{09-01_0410}}
% 95
{\PTglyphid{09-01_0411}}
% 96
{\PTglyphid{09-01_0412}}
% 97
{\PTglyphid{09-01_0413}}
% 98
{\PTglyphid{09-01_0414}}
% 99
{\PTglyphid{09-01_0415}}
% 100
{\PTglyphid{09-01_0416}}
% 101
{\PTglyphid{09-01_0417}}
% 102
{\PTglyphid{09-01_0418}}
% 103
{\PTglyphid{09-01_0419}}
% 104
{\PTglyphid{09-01_0420}}
% 105
{\PTglyphid{09-01_0421}}
% 106
{\PTglyphid{09-01_0422}}
% 107
{\PTglyphid{09-01_0423}}
% 108
{\PTglyphid{09-01_0424}}
% 109
{\PTglyphid{09-01_0425}}
% 110
{\PTglyphid{09-01_0426}}
% 111
{\PTglyphid{09-01_0427}}
% 112
{\PTglyphid{09-01_0428}}
//
\endgl \xe
%%% Local Variables:
%%% mode: latex
%%% TeX-engine: luatex
%%% TeX-master: shared
%%% End:

% //
%\endgl \xe

 \newpage
 
%%%%%%%%%%%%%%%%%%%%%%%%%%%%%%%%%%%%%%%%%%%%%%%%%%%%%%%%%%%%%%%%%%%%%%%%%%%%%%%
% from meta.csv
 % Tab. 40,Hochfeder-10_PT01_028.djvu,Hochfeder,10,01,028
%%%%%%%%%%%%%%%%%%%%%%%%%%%%%%%%%%%%%%%%%%%%%%%%%%%%%%%%%%%%%%%%%%%%%%%%%%%%%%%

 % from dsed4test:

 % Note "10. Pismo komentarzowe gotyckie. Krój M¹⁶. Stopień 20 ww. = 69 mm - Tabl. 28 [28]"
% Note1 "Character set table prepared by Maria Błońska and Anna Wolińska"

 \pismoPL{Kasper Hochfeder 10. Pismo komentarzowe gotyckie. Krój M¹⁶. Stopień 20 ww. = 69 mm - Tabl. 28.}
  
 \pismoEN{Kasper Hochfeder 10. Gothic comment font. Typeface M¹⁶. Type
   size 20 lines = 69 mm - Plate 28.}

\plate{28}{I}{1968}

Prepared by Kazimierz Piekarski and  Maria Błońska.\\
The font table prepared by Maria Błońska and Anna Wolińska.

\bigskip

\exampleBib{I:16}

\bigskip \exampleDesc{ IOANNES GLOGOVIENSIS: Exercitium veteris artis. Kraków, [Kasper Hochfeder]. 16 X 1504. 4⁰,
War. B: nakładem Jana Hallera.}


\medskip
\examplePage{\textit{[Karta] L₇b}}

  \bigskip
\exampleLib{Biblioteka Jagiellońska. Kraków.}


\bigskip
\exampleRef{\textit{Estreicher XVII 173, 160. Wierzbowski 7.}}

\bigskip
\exampleDig{\url{https://www.wbc.poznan.pl/dlibra/publication/559203/}
  page 176}

%algorismus!

  
  % \medskip
\bigskip

\examplePL{Pismo 10: tekst i pierwszy zestaw. — Pismo 11: nagłówki i
  drugi zestaw, — Rubryka \delta{}: z pismem 11. — Cyfry 4: z pismem
  11}

\medskip

\exampleEN{Font 10: the text and the first font table. — Font 11: the
  headers and the second font table. — Rubric \delta{} with font 11. —
  Digits 4: with font 11}



\bigskip

\fontID{Ho-10}{40}

\fontstat{107}

% \exdisplay \bg \gla
 \exdisplay \bg \gla
% 1
{\PTglyph{5}{t40_l01g01.png}}
% 2
{\PTglyph{5}{t40_l01g02.png}}
% 3
{\PTglyph{5}{t40_l01g03.png}}
% 4
{\PTglyph{5}{t40_l01g04.png}}
% 5
{\PTglyph{5}{t40_l01g05.png}}
% 6
{\PTglyph{5}{t40_l01g06.png}}
% 7
{\PTglyph{5}{t40_l01g07.png}}
% 8
{\PTglyph{5}{t40_l01g08.png}}
% 9
{\PTglyph{5}{t40_l01g09.png}}
% 10
{\PTglyph{5}{t40_l01g10.png}}
% 11
{\PTglyph{5}{t40_l01g11.png}}
% 12
{\PTglyph{5}{t40_l01g12.png}}
% 13
{\PTglyph{5}{t40_l01g13.png}}
% 14
{\PTglyph{5}{t40_l01g14.png}}
% 15
{\PTglyph{5}{t40_l01g15.png}}
% 16
{\PTglyph{5}{t40_l01g16.png}}
% 17
{\PTglyph{5}{t40_l01g17.png}}
% 18
{\PTglyph{5}{t40_l01g18.png}}
% 19
{\PTglyph{5}{t40_l01g19.png}}
% 20
{\PTglyph{5}{t40_l01g20.png}}
% 21
{\PTglyph{5}{t40_l01g21.png}}
% 22
{\PTglyph{5}{t40_l01g22.png}}
% 23
{\PTglyph{5}{t40_l02g01.png}}
% 24
{\PTglyph{5}{t40_l02g02.png}}
% 25
{\PTglyph{5}{t40_l02g03.png}}
% 26
{\PTglyph{5}{t40_l02g04.png}}
% 27
{\PTglyph{5}{t40_l02g05.png}}
% 28
{\PTglyph{5}{t40_l02g06.png}}
% 29
{\PTglyph{5}{t40_l02g07.png}}
% 30
{\PTglyph{5}{t40_l02g08.png}}
% 31
{\PTglyph{5}{t40_l02g09.png}}
% 32
{\PTglyph{5}{t40_l02g10.png}}
% 33
{\PTglyph{5}{t40_l02g11.png}}
% 34
{\PTglyph{5}{t40_l02g12.png}}
% 35
{\PTglyph{5}{t40_l02g13.png}}
% 36
{\PTglyph{5}{t40_l02g14.png}}
% 37
{\PTglyph{5}{t40_l02g15.png}}
% 38
{\PTglyph{5}{t40_l02g16.png}}
% 39
{\PTglyph{5}{t40_l02g17.png}}
% 40
{\PTglyph{5}{t40_l02g18.png}}
% 41
{\PTglyph{5}{t40_l02g19.png}}
% 42
{\PTglyph{5}{t40_l02g20.png}}
% 43
{\PTglyph{5}{t40_l02g21.png}}
% 44
{\PTglyph{5}{t40_l02g22.png}}
% 45
{\PTglyph{5}{t40_l02g23.png}}
% 46
{\PTglyph{5}{t40_l02g24.png}}
% 47
{\PTglyph{5}{t40_l02g25.png}}
% 48
{\PTglyph{5}{t40_l02g26.png}}
% 49
{\PTglyph{5}{t40_l02g27.png}}
% 50
{\PTglyph{5}{t40_l02g28.png}}
% 51
{\PTglyph{5}{t40_l02g29.png}}
% 52
{\PTglyph{5}{t40_l02g30.png}}
% 53
{\PTglyph{5}{t40_l02g31.png}}
% 54
{\PTglyph{5}{t40_l03g01.png}}
% 55
{\PTglyph{5}{t40_l03g02.png}}
% 56
{\PTglyph{5}{t40_l03g03.png}}
% 57
{\PTglyph{5}{t40_l03g04.png}}
% 58
{\PTglyph{5}{t40_l03g05.png}}
% 59
{\PTglyph{5}{t40_l03g06.png}}
% 60
{\PTglyph{5}{t40_l03g07.png}}
% 61
{\PTglyph{5}{t40_l03g08.png}}
% 62
{\PTglyph{5}{t40_l03g09.png}}
% 63
{\PTglyph{5}{t40_l03g10.png}}
% 64
{\PTglyph{5}{t40_l03g11.png}}
% 65
{\PTglyph{5}{t40_l03g12.png}}
% 66
{\PTglyph{5}{t40_l03g13.png}}
% 67
{\PTglyph{5}{t40_l03g14.png}}
% 68
{\PTglyph{5}{t40_l03g15.png}}
% 69
{\PTglyph{5}{t40_l03g16.png}}
% 70
{\PTglyph{5}{t40_l03g17.png}}
% 71
{\PTglyph{5}{t40_l03g18.png}}
% 72
{\PTglyph{5}{t40_l03g19.png}}
% 73
{\PTglyph{5}{t40_l03g20.png}}
% 74
{\PTglyph{5}{t40_l03g21.png}}
% 75
{\PTglyph{5}{t40_l03g22.png}}
% 76
{\PTglyph{5}{t40_l03g23.png}}
% 77
{\PTglyph{5}{t40_l03g24.png}}
% 78
{\PTglyph{5}{t40_l03g25.png}}
% 79
{\PTglyph{5}{t40_l03g26.png}}
% 80
{\PTglyph{5}{t40_l03g27.png}}
% 81
{\PTglyph{5}{t40_l04g01.png}}
% 82
{\PTglyph{5}{t40_l04g02.png}}
% 83
{\PTglyph{5}{t40_l04g03.png}}
% 84
{\PTglyph{5}{t40_l04g04.png}}
% 85
{\PTglyph{5}{t40_l04g05.png}}
% 86
{\PTglyph{5}{t40_l04g06.png}}
% 87
{\PTglyph{5}{t40_l04g07.png}}
% 88
{\PTglyph{5}{t40_l04g08.png}}
% 89
{\PTglyph{5}{t40_l04g09.png}}
% 90
{\PTglyph{5}{t40_l04g10.png}}
% 91
{\PTglyph{5}{t40_l04g11.png}}
% 92
{\PTglyph{5}{t40_l04g12.png}}
% 93
{\PTglyph{5}{t40_l04g13.png}}
% 94
{\PTglyph{5}{t40_l04g14.png}}
% 95
{\PTglyph{5}{t40_l04g15.png}}
% 96
{\PTglyph{5}{t40_l04g16.png}}
% 97
{\PTglyph{5}{t40_l04g17.png}}
% 98
{\PTglyph{5}{t40_l04g18.png}}
% 99
{\PTglyph{5}{t40_l04g19.png}}
% 100
{\PTglyph{5}{t40_l04g20.png}}
% 101
{\PTglyph{5}{t40_l04g21.png}}
% 102
{\PTglyph{5}{t40_l04g22.png}}
% 103
{\PTglyph{5}{t40_l04g23.png}}
% 104
{\PTglyph{5}{t40_l04g24.png}}
% 105
{\PTglyph{5}{t40_l04g25.png}}
% 106
{\PTglyph{5}{t40_l04g26.png}}
% 107
{\PTglyph{5}{t40_l04g27.png}}
//
%%% Local Variables:
%%% mode: latex
%%% TeX-engine: luatex
%%% TeX-master: shared
%%% End:

%//
%\glpismo%
 \glpismo
% 1
{\PTglyphid{Ho-10_0101}}
% 2
{\PTglyphid{Ho-10_0102}}
% 3
{\PTglyphid{Ho-10_0103}}
% 4
{\PTglyphid{Ho-10_0104}}
% 5
{\PTglyphid{Ho-10_0105}}
% 6
{\PTglyphid{Ho-10_0106}}
% 7
{\PTglyphid{Ho-10_0107}}
% 8
{\PTglyphid{Ho-10_0108}}
% 9
{\PTglyphid{Ho-10_0109}}
% 10
{\PTglyphid{Ho-10_0110}}
% 11
{\PTglyphid{Ho-10_0111}}
% 12
{\PTglyphid{Ho-10_0112}}
% 13
{\PTglyphid{Ho-10_0113}}
% 14
{\PTglyphid{Ho-10_0114}}
% 15
{\PTglyphid{Ho-10_0115}}
% 16
{\PTglyphid{Ho-10_0116}}
% 17
{\PTglyphid{Ho-10_0117}}
% 18
{\PTglyphid{Ho-10_0118}}
% 19
{\PTglyphid{Ho-10_0119}}
% 20
{\PTglyphid{Ho-10_0120}}
% 21
{\PTglyphid{Ho-10_0121}}
% 22
{\PTglyphid{Ho-10_0122}}
% 23
{\PTglyphid{Ho-10_0201}}
% 24
{\PTglyphid{Ho-10_0202}}
% 25
{\PTglyphid{Ho-10_0203}}
% 26
{\PTglyphid{Ho-10_0204}}
% 27
{\PTglyphid{Ho-10_0205}}
% 28
{\PTglyphid{Ho-10_0206}}
% 29
{\PTglyphid{Ho-10_0207}}
% 30
{\PTglyphid{Ho-10_0208}}
% 31
{\PTglyphid{Ho-10_0209}}
% 32
{\PTglyphid{Ho-10_0210}}
% 33
{\PTglyphid{Ho-10_0211}}
% 34
{\PTglyphid{Ho-10_0212}}
% 35
{\PTglyphid{Ho-10_0213}}
% 36
{\PTglyphid{Ho-10_0214}}
% 37
{\PTglyphid{Ho-10_0215}}
% 38
{\PTglyphid{Ho-10_0216}}
% 39
{\PTglyphid{Ho-10_0217}}
% 40
{\PTglyphid{Ho-10_0218}}
% 41
{\PTglyphid{Ho-10_0219}}
% 42
{\PTglyphid{Ho-10_0220}}
% 43
{\PTglyphid{Ho-10_0221}}
% 44
{\PTglyphid{Ho-10_0222}}
% 45
{\PTglyphid{Ho-10_0223}}
% 46
{\PTglyphid{Ho-10_0224}}
% 47
{\PTglyphid{Ho-10_0225}}
% 48
{\PTglyphid{Ho-10_0226}}
% 49
{\PTglyphid{Ho-10_0227}}
% 50
{\PTglyphid{Ho-10_0228}}
% 51
{\PTglyphid{Ho-10_0229}}
% 52
{\PTglyphid{Ho-10_0230}}
% 53
{\PTglyphid{Ho-10_0231}}
% 54
{\PTglyphid{Ho-10_0301}}
% 55
{\PTglyphid{Ho-10_0302}}
% 56
{\PTglyphid{Ho-10_0303}}
% 57
{\PTglyphid{Ho-10_0304}}
% 58
{\PTglyphid{Ho-10_0305}}
% 59
{\PTglyphid{Ho-10_0306}}
% 60
{\PTglyphid{Ho-10_0307}}
% 61
{\PTglyphid{Ho-10_0308}}
% 62
{\PTglyphid{Ho-10_0309}}
% 63
{\PTglyphid{Ho-10_0310}}
% 64
{\PTglyphid{Ho-10_0311}}
% 65
{\PTglyphid{Ho-10_0312}}
% 66
{\PTglyphid{Ho-10_0313}}
% 67
{\PTglyphid{Ho-10_0314}}
% 68
{\PTglyphid{Ho-10_0315}}
% 69
{\PTglyphid{Ho-10_0316}}
% 70
{\PTglyphid{Ho-10_0317}}
% 71
{\PTglyphid{Ho-10_0318}}
% 72
{\PTglyphid{Ho-10_0319}}
% 73
{\PTglyphid{Ho-10_0320}}
% 74
{\PTglyphid{Ho-10_0321}}
% 75
{\PTglyphid{Ho-10_0322}}
% 76
{\PTglyphid{Ho-10_0323}}
% 77
{\PTglyphid{Ho-10_0324}}
% 78
{\PTglyphid{Ho-10_0325}}
% 79
{\PTglyphid{Ho-10_0326}}
% 80
{\PTglyphid{Ho-10_0327}}
% 81
{\PTglyphid{Ho-10_0401}}
% 82
{\PTglyphid{Ho-10_0402}}
% 83
{\PTglyphid{Ho-10_0403}}
% 84
{\PTglyphid{Ho-10_0404}}
% 85
{\PTglyphid{Ho-10_0405}}
% 86
{\PTglyphid{Ho-10_0406}}
% 87
{\PTglyphid{Ho-10_0407}}
% 88
{\PTglyphid{Ho-10_0408}}
% 89
{\PTglyphid{Ho-10_0409}}
% 90
{\PTglyphid{Ho-10_0410}}
% 91
{\PTglyphid{Ho-10_0411}}
% 92
{\PTglyphid{Ho-10_0412}}
% 93
{\PTglyphid{Ho-10_0413}}
% 94
{\PTglyphid{Ho-10_0414}}
% 95
{\PTglyphid{Ho-10_0415}}
% 96
{\PTglyphid{Ho-10_0416}}
% 97
{\PTglyphid{Ho-10_0417}}
% 98
{\PTglyphid{Ho-10_0418}}
% 99
{\PTglyphid{Ho-10_0419}}
% 100
{\PTglyphid{Ho-10_0420}}
% 101
{\PTglyphid{Ho-10_0421}}
% 102
{\PTglyphid{Ho-10_0422}}
% 103
{\PTglyphid{Ho-10_0423}}
% 104
{\PTglyphid{Ho-10_0424}}
% 105
{\PTglyphid{Ho-10_0425}}
% 106
{\PTglyphid{Ho-10_0426}}
% 107
{\PTglyphid{Ho-10_0427}}
//
\endgl \xe
%%% Local Variables:
%%% mode: latex
%%% TeX-engine: luatex
%%% TeX-master: shared
%%% End:

% //
%\endgl \xe


 \newpage
 
%%%%%%%%%%%%%%%%%%%%%%%%%%%%%%%%%%%%%%%%%%%%%%%%%%%%%%%%%%%%%%%%%%%%%%%%%%%%%%%
% from meta.csv
 % Tab. 41,Hochfeder-11_PT01_028.djvu,Hochfeder,11,01,028
%%%%%%%%%%%%%%%%%%%%%%%%%%%%%%%%%%%%%%%%%%%%%%%%%%%%%%%%%%%%%%%%%%%%%%%%%%%%%%%

 % from dsed4test:

 % Note "11. Pismo mszalne gotyckie. Krój M²³. Stopień 20 ww. = 154/156 mm - Tabl. 25, 28. (Występuje u Hallera jako pismo 3. Tabl. 166; u Unglera jako pismo 11) [28]"
% Note1 "Character set table prepared by Maria Błońska and Anna Wolińska"

 \pismoPL{Kasper Hochfeder 11. Pismo mszalne gotyckie. Krój
   M²³. Stopień 20 ww. = 154/156 mm - Tabl. 25, 28. (Występuje u
   Hallera jako pismo 3. Tabl. 166; u Unglera jako pismo 11)}
  
 \pismoEN{Kasper Hochfeder 11. Gothic missal font. Typeface M²³. Type
   size 20 lines = 154/156 mm - Plate 28.}

\plate{28}{I}{1968}

Prepared by Kazimierz Piekarski and  Maria Błońska.\\
The font table prepared by Maria Błońska and Anna Wolińska.

\bigskip

\exampleBib{I:16}

\bigskip \exampleDesc{ IOANNES GLOGOVIENSIS: Exercitium veteris artis. Kraków, [Kasper Hochfeder]. 16 X 1504. 4⁰,
War. B: nakładem Jana Hallera.}


\medskip
\examplePage{\textit{[Karta] L₇b}}

  \bigskip
\exampleLib{Biblioteka Jagiellońska. Kraków.}


\bigskip
\exampleRef{\textit{Estreicher XVII 173, 160. Wierzbowski 7.}}

\bigskip
\exampleDig{\url{https://www.wbc.poznan.pl/dlibra/publication/559203/}
  page 176}

%algorismus!

  
  % \medskip
\bigskip

\examplePL{Pismo 10: tekst i pierwszy zestaw. — Pismo 11: nagłówki i
  drugi zestaw, — Rubryka \delta{}: z pismem 11. — Cyfry 4: z pismem
  11}

\medskip

\exampleEN{Font 10: the text and the first font table. — Font 11: the
  headers and the second font table. — Rubric \delta{} with font 11. —
  Digits 4: with font 11}



\bigskip

\fontID{Ho-11}{41}

\fontstat{95}

% \exdisplay \bg \gla
 \exdisplay \bg \gla
% 1
{\PTglyph{5}{t41_l01g01.png}}
% 2
{\PTglyph{5}{t41_l01g02.png}}
% 3
{\PTglyph{5}{t41_l01g03.png}}
% 4
{\PTglyph{5}{t41_l01g04.png}}
% 5
{\PTglyph{5}{t41_l01g05.png}}
% 6
{\PTglyph{5}{t41_l01g06.png}}
% 7
{\PTglyph{5}{t41_l01g07.png}}
% 8
{\PTglyph{5}{t41_l01g08.png}}
% 9
{\PTglyph{5}{t41_l01g09.png}}
% 10
{\PTglyph{5}{t41_l01g10.png}}
% 11
{\PTglyph{5}{t41_l01g11.png}}
% 12
{\PTglyph{5}{t41_l01g12.png}}
% 13
{\PTglyph{5}{t41_l01g13.png}}
% 14
{\PTglyph{5}{t41_l01g14.png}}
% 15
{\PTglyph{5}{t41_l01g15.png}}
% 16
{\PTglyph{5}{t41_l01g16.png}}
% 17
{\PTglyph{5}{t41_l01g17.png}}
% 18
{\PTglyph{5}{t41_l01g18.png}}
% 19
{\PTglyph{5}{t41_l01g19.png}}
% 20
{\PTglyph{5}{t41_l01g20.png}}
% 21
{\PTglyph{5}{t41_l01g21.png}}
% 22
{\PTglyph{5}{t41_l01g22.png}}
% 23
{\PTglyph{5}{t41_l01g23.png}}
% 24
{\PTglyph{5}{t41_l02g01.png}}
% 25
{\PTglyph{5}{t41_l02g02.png}}
% 26
{\PTglyph{5}{t41_l02g03.png}}
% 27
{\PTglyph{5}{t41_l02g04.png}}
% 28
{\PTglyph{5}{t41_l02g05.png}}
% 29
{\PTglyph{5}{t41_l02g06.png}}
% 30
{\PTglyph{5}{t41_l02g07.png}}
% 31
{\PTglyph{5}{t41_l02g08.png}}
% 32
{\PTglyph{5}{t41_l02g09.png}}
% 33
{\PTglyph{5}{t41_l02g10.png}}
% 34
{\PTglyph{5}{t41_l02g11.png}}
% 35
{\PTglyph{5}{t41_l02g12.png}}
% 36
{\PTglyph{5}{t41_l02g13.png}}
% 37
{\PTglyph{5}{t41_l02g14.png}}
% 38
{\PTglyph{5}{t41_l02g15.png}}
% 39
{\PTglyph{5}{t41_l02g16.png}}
% 40
{\PTglyph{5}{t41_l02g17.png}}
% 41
{\PTglyph{5}{t41_l02g18.png}}
% 42
{\PTglyph{5}{t41_l02g19.png}}
% 43
{\PTglyph{5}{t41_l02g20.png}}
% 44
{\PTglyph{5}{t41_l02g21.png}}
% 45
{\PTglyph{5}{t41_l02g22.png}}
% 46
{\PTglyph{5}{t41_l02g23.png}}
% 47
{\PTglyph{5}{t41_l02g24.png}}
% 48
{\PTglyph{5}{t41_l02g25.png}}
% 49
{\PTglyph{5}{t41_l02g26.png}}
% 50
{\PTglyph{5}{t41_l02g27.png}}
% 51
{\PTglyph{5}{t41_l02g28.png}}
% 52
{\PTglyph{5}{t41_l02g29.png}}
% 53
{\PTglyph{5}{t41_l02g30.png}}
% 54
{\PTglyph{5}{t41_l02g31.png}}
% 55
{\PTglyph{5}{t41_l02g32.png}}
% 56
{\PTglyph{5}{t41_l02g33.png}}
% 57
{\PTglyph{5}{t41_l02g34.png}}
% 58
{\PTglyph{5}{t41_l02g35.png}}
% 59
{\PTglyph{5}{t41_l02g36.png}}
% 60
{\PTglyph{5}{t41_l02g37.png}}
% 61
{\PTglyph{5}{t41_l02g38.png}}
% 62
{\PTglyph{5}{t41_l03g01.png}}
% 63
{\PTglyph{5}{t41_l03g02.png}}
% 64
{\PTglyph{5}{t41_l03g03.png}}
% 65
{\PTglyph{5}{t41_l03g04.png}}
% 66
{\PTglyph{5}{t41_l03g05.png}}
% 67
{\PTglyph{5}{t41_l03g06.png}}
% 68
{\PTglyph{5}{t41_l03g07.png}}
% 69
{\PTglyph{5}{t41_l03g08.png}}
% 70
{\PTglyph{5}{t41_l03g09.png}}
% 71
{\PTglyph{5}{t41_l03g10.png}}
% 72
{\PTglyph{5}{t41_l03g11.png}}
% 73
{\PTglyph{5}{t41_l03g12.png}}
% 74
{\PTglyph{5}{t41_l03g13.png}}
% 75
{\PTglyph{5}{t41_l03g14.png}}
% 76
{\PTglyph{5}{t41_l03g15.png}}
% 77
{\PTglyph{5}{t41_l03g16.png}}
% 78
{\PTglyph{5}{t41_l03g17.png}}
% 79
{\PTglyph{5}{t41_l03g18.png}}
% 80
{\PTglyph{5}{t41_l03g19.png}}
% 81
{\PTglyph{5}{t41_l03g20.png}}
% 82
{\PTglyph{5}{t41_l03g21.png}}
% 83
{\PTglyph{5}{t41_l03g22.png}}
% 84
{\PTglyph{5}{t41_l03g23.png}}
% 85
{\PTglyph{5}{t41_l03g24.png}}
% 86
{\PTglyph{5}{t41_l03g25.png}}
% 87
{\PTglyph{5}{t41_l03g26.png}}
% 88
{\PTglyph{5}{t41_l03g27.png}}
% 89
{\PTglyph{5}{t41_l03g28.png}}
% 90
{\PTglyph{5}{t41_l03g29.png}}
% 91
{\PTglyph{5}{t41_l03g30.png}}
% 92
{\PTglyph{5}{t41_l03g31.png}}
% 93
{\PTglyph{5}{t41_l03g32.png}}
% 94
{\PTglyph{5}{t41_l03g33.png}}
% 95
{\PTglyph{5}{t41_l03g34.png}}
//
%%% Local Variables:
%%% mode: latex
%%% TeX-engine: luatex
%%% TeX-master: shared
%%% End:

%//
%\glpismo%
 \glpismo
% 1
{\PTglyphid{Ho-11_0101}}
% 2
{\PTglyphid{Ho-11_0102}}
% 3
{\PTglyphid{Ho-11_0103}}
% 4
{\PTglyphid{Ho-11_0104}}
% 5
{\PTglyphid{Ho-11_0105}}
% 6
{\PTglyphid{Ho-11_0106}}
% 7
{\PTglyphid{Ho-11_0107}}
% 8
{\PTglyphid{Ho-11_0108}}
% 9
{\PTglyphid{Ho-11_0109}}
% 10
{\PTglyphid{Ho-11_0110}}
% 11
{\PTglyphid{Ho-11_0111}}
% 12
{\PTglyphid{Ho-11_0112}}
% 13
{\PTglyphid{Ho-11_0113}}
% 14
{\PTglyphid{Ho-11_0114}}
% 15
{\PTglyphid{Ho-11_0115}}
% 16
{\PTglyphid{Ho-11_0116}}
% 17
{\PTglyphid{Ho-11_0117}}
% 18
{\PTglyphid{Ho-11_0118}}
% 19
{\PTglyphid{Ho-11_0119}}
% 20
{\PTglyphid{Ho-11_0120}}
% 21
{\PTglyphid{Ho-11_0121}}
% 22
{\PTglyphid{Ho-11_0122}}
% 23
{\PTglyphid{Ho-11_0123}}
% 24
{\PTglyphid{Ho-11_0201}}
% 25
{\PTglyphid{Ho-11_0202}}
% 26
{\PTglyphid{Ho-11_0203}}
% 27
{\PTglyphid{Ho-11_0204}}
% 28
{\PTglyphid{Ho-11_0205}}
% 29
{\PTglyphid{Ho-11_0206}}
% 30
{\PTglyphid{Ho-11_0207}}
% 31
{\PTglyphid{Ho-11_0208}}
% 32
{\PTglyphid{Ho-11_0209}}
% 33
{\PTglyphid{Ho-11_0210}}
% 34
{\PTglyphid{Ho-11_0211}}
% 35
{\PTglyphid{Ho-11_0212}}
% 36
{\PTglyphid{Ho-11_0213}}
% 37
{\PTglyphid{Ho-11_0214}}
% 38
{\PTglyphid{Ho-11_0215}}
% 39
{\PTglyphid{Ho-11_0216}}
% 40
{\PTglyphid{Ho-11_0217}}
% 41
{\PTglyphid{Ho-11_0218}}
% 42
{\PTglyphid{Ho-11_0219}}
% 43
{\PTglyphid{Ho-11_0220}}
% 44
{\PTglyphid{Ho-11_0221}}
% 45
{\PTglyphid{Ho-11_0222}}
% 46
{\PTglyphid{Ho-11_0223}}
% 47
{\PTglyphid{Ho-11_0224}}
% 48
{\PTglyphid{Ho-11_0225}}
% 49
{\PTglyphid{Ho-11_0226}}
% 50
{\PTglyphid{Ho-11_0227}}
% 51
{\PTglyphid{Ho-11_0228}}
% 52
{\PTglyphid{Ho-11_0229}}
% 53
{\PTglyphid{Ho-11_0230}}
% 54
{\PTglyphid{Ho-11_0231}}
% 55
{\PTglyphid{Ho-11_0232}}
% 56
{\PTglyphid{Ho-11_0233}}
% 57
{\PTglyphid{Ho-11_0234}}
% 58
{\PTglyphid{Ho-11_0235}}
% 59
{\PTglyphid{Ho-11_0236}}
% 60
{\PTglyphid{Ho-11_0237}}
% 61
{\PTglyphid{Ho-11_0238}}
% 62
{\PTglyphid{Ho-11_0301}}
% 63
{\PTglyphid{Ho-11_0302}}
% 64
{\PTglyphid{Ho-11_0303}}
% 65
{\PTglyphid{Ho-11_0304}}
% 66
{\PTglyphid{Ho-11_0305}}
% 67
{\PTglyphid{Ho-11_0306}}
% 68
{\PTglyphid{Ho-11_0307}}
% 69
{\PTglyphid{Ho-11_0308}}
% 70
{\PTglyphid{Ho-11_0309}}
% 71
{\PTglyphid{Ho-11_0310}}
% 72
{\PTglyphid{Ho-11_0311}}
% 73
{\PTglyphid{Ho-11_0312}}
% 74
{\PTglyphid{Ho-11_0313}}
% 75
{\PTglyphid{Ho-11_0314}}
% 76
{\PTglyphid{Ho-11_0315}}
% 77
{\PTglyphid{Ho-11_0316}}
% 78
{\PTglyphid{Ho-11_0317}}
% 79
{\PTglyphid{Ho-11_0318}}
% 80
{\PTglyphid{Ho-11_0319}}
% 81
{\PTglyphid{Ho-11_0320}}
% 82
{\PTglyphid{Ho-11_0321}}
% 83
{\PTglyphid{Ho-11_0322}}
% 84
{\PTglyphid{Ho-11_0323}}
% 85
{\PTglyphid{Ho-11_0324}}
% 86
{\PTglyphid{Ho-11_0325}}
% 87
{\PTglyphid{Ho-11_0326}}
% 88
{\PTglyphid{Ho-11_0327}}
% 89
{\PTglyphid{Ho-11_0328}}
% 90
{\PTglyphid{Ho-11_0329}}
% 91
{\PTglyphid{Ho-11_0330}}
% 92
{\PTglyphid{Ho-11_0331}}
% 93
{\PTglyphid{Ho-11_0332}}
% 94
{\PTglyphid{Ho-11_0333}}
% 95
{\PTglyphid{Ho-11_0334}}
//
\endgl \xe
%%% Local Variables:
%%% mode: latex
%%% TeX-engine: luatex
%%% TeX-master: shared
%%% End:

% //
%\endgl \xe

  \newpage
 
%%%%%%%%%%%%%%%%%%%%%%%%%%%%%%%%%%%%%%%%%%%%%%%%%%%%%%%%%%%%%%%%%%%%%%%%%%%%%%%
% from meta.csv
 % Tab. 42,Ungler1-01_PT03_112.djvu,Ungler1,01,03,112
%%%%%%%%%%%%%%%%%%%%%%%%%%%%%%%%%%%%%%%%%%%%%%%%%%%%%%%%%%%%%%%%%%%%%%%%%%%%%%%

 % from dsed4test:

% Fascicule "III"
% Publisher "Zakład Narodowy imienia Ossolińskich — Wydawnictwo"
% Addres "Kraków  Wrocław Warszawa"
% Year "1959"
% Note "1. Pismo tekstowe gotyckie. Krój M⁹¹. Stopień 20 ww. == 76/77 mm. — Tabl. 112."


 \pismoPL{Florian Ungler pierwsza drukarnia 1. Pismo tekstowe gotyckie. Krój M⁹¹. Stopień 20 ww. == 76/77 mm. — Tabl. 112.}
  
 \pismoEN{Florian Ungler first house 1. Gothic text font. Typeface M⁹¹. Type size 20 ww. == 76/77 mm. — Plate 112.}

\plate{112}{III}{1959}

The plate prepared by Henryk Bułhak.\\
The font table prepared by Maria Błońska.

\bigskip

\exampleBib{III:3}

\bigskip \exampleDesc{IOANNES DE SACROBOSCO: Algorithmus. Kraków, Florian Ungler, 31. I. 1511. 4⁰} 


\medskip
\examplePage{\textit{Karta b₅b}}

  \bigskip
\exampleLib{Biblioteka Zakł. Nar. im. Ossolińskich. Wrocław.}


\bigskip
\exampleRef{\textit{Estreicher XXVI. 15. Piekarski U. 2.}}

  \bigskip
  \exampleDig{%1523 \url{https://polona.pl/preview/ff92eeb7-0467-4359-bdd8-4be4de13358c}    page ???,
    % 1522?: \url{https://www.wbc.poznan.pl/dlibra/publication/314379/} page    ???,
    \url{https://dbc.wroc.pl/dlibra/publication/3586/} page 24}

%algorismus!

  
  % \medskip
\bigskip

\examplePL{[Pismo 1] Rubryka \alpha{}. — Cyfry 1. — Inicjal 1 (P).}

\medskip

\exampleEN{[Font 1] Rubric \alpha{}  —
  Digits 1. — Initial 1 (P)}


\bigskip

\fontID{Un1-1}{42}

\fontstat{94}

% \exdisplay \bg \gla
 \exdisplay \bg \gla
% 1
{\PTglyph{5}{t42_l01g01.png}}
% 2
{\PTglyph{5}{t42_l01g02.png}}
% 3
{\PTglyph{5}{t42_l01g03.png}}
% 4
{\PTglyph{5}{t42_l01g04.png}}
% 5
{\PTglyph{5}{t42_l01g05.png}}
% 6
{\PTglyph{5}{t42_l01g06.png}}
% 7
{\PTglyph{5}{t42_l01g07.png}}
% 8
{\PTglyph{5}{t42_l01g08.png}}
% 9
{\PTglyph{5}{t42_l01g09.png}}
% 10
{\PTglyph{5}{t42_l01g10.png}}
% 11
{\PTglyph{5}{t42_l01g11.png}}
% 12
{\PTglyph{5}{t42_l01g12.png}}
% 13
{\PTglyph{5}{t42_l01g13.png}}
% 14
{\PTglyph{5}{t42_l01g14.png}}
% 15
{\PTglyph{5}{t42_l01g15.png}}
% 16
{\PTglyph{5}{t42_l01g16.png}}
% 17
{\PTglyph{5}{t42_l01g17.png}}
% 18
{\PTglyph{5}{t42_l01g18.png}}
% 19
{\PTglyph{5}{t42_l01g19.png}}
% 20
{\PTglyph{5}{t42_l01g20.png}}
% 21
{\PTglyph{5}{t42_l02g01.png}}
% 22
{\PTglyph{5}{t42_l02g02.png}}
% 23
{\PTglyph{5}{t42_l02g03.png}}
% 24
{\PTglyph{5}{t42_l02g04.png}}
% 25
{\PTglyph{5}{t42_l02g05.png}}
% 26
{\PTglyph{5}{t42_l02g06.png}}
% 27
{\PTglyph{5}{t42_l02g07.png}}
% 28
{\PTglyph{5}{t42_l02g08.png}}
% 29
{\PTglyph{5}{t42_l02g09.png}}
% 30
{\PTglyph{5}{t42_l02g10.png}}
% 31
{\PTglyph{5}{t42_l02g11.png}}
% 32
{\PTglyph{5}{t42_l02g12.png}}
% 33
{\PTglyph{5}{t42_l02g13.png}}
% 34
{\PTglyph{5}{t42_l02g14.png}}
% 35
{\PTglyph{5}{t42_l02g15.png}}
% 36
{\PTglyph{5}{t42_l02g16.png}}
% 37
{\PTglyph{5}{t42_l02g17.png}}
% 38
{\PTglyph{5}{t42_l02g18.png}}
% 39
{\PTglyph{5}{t42_l02g19.png}}
% 40
{\PTglyph{5}{t42_l02g20.png}}
% 41
{\PTglyph{5}{t42_l02g21.png}}
% 42
{\PTglyph{5}{t42_l02g22.png}}
% 43
{\PTglyph{5}{t42_l02g23.png}}
% 44
{\PTglyph{5}{t42_l02g24.png}}
% 45
{\PTglyph{5}{t42_l02g25.png}}
% 46
{\PTglyph{5}{t42_l02g26.png}}
% 47
{\PTglyph{5}{t42_l02g27.png}}
% 48
{\PTglyph{5}{t42_l02g28.png}}
% 49
{\PTglyph{5}{t42_l02g29.png}}
% 50
{\PTglyph{5}{t42_l03g01.png}}
% 51
{\PTglyph{5}{t42_l03g02.png}}
% 52
{\PTglyph{5}{t42_l03g03.png}}
% 53
{\PTglyph{5}{t42_l03g04.png}}
% 54
{\PTglyph{5}{t42_l03g05.png}}
% 55
{\PTglyph{5}{t42_l03g06.png}}
% 56
{\PTglyph{5}{t42_l03g07.png}}
% 57
{\PTglyph{5}{t42_l03g08.png}}
% 58
{\PTglyph{5}{t42_l03g09.png}}
% 59
{\PTglyph{5}{t42_l03g10.png}}
% 60
{\PTglyph{5}{t42_l03g11.png}}
% 61
{\PTglyph{5}{t42_l03g12.png}}
% 62
{\PTglyph{5}{t42_l03g13.png}}
% 63
{\PTglyph{5}{t42_l03g14.png}}
% 64
{\PTglyph{5}{t42_l03g15.png}}
% 65
{\PTglyph{5}{t42_l03g16.png}}
% 66
{\PTglyph{5}{t42_l03g17.png}}
% 67
{\PTglyph{5}{t42_l03g18.png}}
% 68
{\PTglyph{5}{t42_l03g19.png}}
% 69
{\PTglyph{5}{t42_l03g20.png}}
% 70
{\PTglyph{5}{t42_l03g21.png}}
% 71
{\PTglyph{5}{t42_l03g22.png}}
% 72
{\PTglyph{5}{t42_l03g23.png}}
% 73
{\PTglyph{5}{t42_l03g24.png}}
% 74
{\PTglyph{5}{t42_l03g25.png}}
% 75
{\PTglyph{5}{t42_l03g26.png}}
% 76
{\PTglyph{5}{t42_l03g27.png}}
% 77
{\PTglyph{5}{t42_l04g01.png}}
% 78
{\PTglyph{5}{t42_l04g02.png}}
% 79
{\PTglyph{5}{t42_l04g03.png}}
% 80
{\PTglyph{5}{t42_l04g04.png}}
% 81
{\PTglyph{5}{t42_l04g05.png}}
% 82
{\PTglyph{5}{t42_l05g01.png}}
% 83
{\PTglyph{5}{t42_l05g02.png}}
% 84
{\PTglyph{5}{t42_l05g03.png}}
% 85
{\PTglyph{5}{t42_l05g04.png}}
% 86
{\PTglyph{5}{t42_l05g05.png}}
% 87
{\PTglyph{5}{t42_l05g06.png}}
% 88
{\PTglyph{5}{t42_l05g07.png}}
% 89
{\PTglyph{5}{t42_l05g08.png}}
% 90
{\PTglyph{5}{t42_l05g09.png}}
% 91
{\PTglyph{5}{t42_l05g10.png}}
% 92
{\PTglyph{5}{t42_l05g11.png}}
% 93
{\PTglyph{5}{t42_l05g12.png}}
% 94
{\PTglyph{5}{t42_l05g13.png}}
//
%%% Local Variables:
%%% mode: latex
%%% TeX-engine: luatex
%%% TeX-master: shared
%%% End:

%//
%\glpismo%
 \glpismo
% 1
{\PTglyphid{U1-01_0101}}
% 2
{\PTglyphid{U1-01_0102}}
% 3
{\PTglyphid{U1-01_0103}}
% 4
{\PTglyphid{U1-01_0104}}
% 5
{\PTglyphid{U1-01_0105}}
% 6
{\PTglyphid{U1-01_0106}}
% 7
{\PTglyphid{U1-01_0107}}
% 8
{\PTglyphid{U1-01_0108}}
% 9
{\PTglyphid{U1-01_0109}}
% 10
{\PTglyphid{U1-01_0110}}
% 11
{\PTglyphid{U1-01_0111}}
% 12
{\PTglyphid{U1-01_0112}}
% 13
{\PTglyphid{U1-01_0113}}
% 14
{\PTglyphid{U1-01_0114}}
% 15
{\PTglyphid{U1-01_0115}}
% 16
{\PTglyphid{U1-01_0116}}
% 17
{\PTglyphid{U1-01_0117}}
% 18
{\PTglyphid{U1-01_0118}}
% 19
{\PTglyphid{U1-01_0119}}
% 20
{\PTglyphid{U1-01_0120}}
% 21
{\PTglyphid{U1-01_0201}}
% 22
{\PTglyphid{U1-01_0202}}
% 23
{\PTglyphid{U1-01_0203}}
% 24
{\PTglyphid{U1-01_0204}}
% 25
{\PTglyphid{U1-01_0205}}
% 26
{\PTglyphid{U1-01_0206}}
% 27
{\PTglyphid{U1-01_0207}}
% 28
{\PTglyphid{U1-01_0208}}
% 29
{\PTglyphid{U1-01_0209}}
% 30
{\PTglyphid{U1-01_0210}}
% 31
{\PTglyphid{U1-01_0211}}
% 32
{\PTglyphid{U1-01_0212}}
% 33
{\PTglyphid{U1-01_0213}}
% 34
{\PTglyphid{U1-01_0214}}
% 35
{\PTglyphid{U1-01_0215}}
% 36
{\PTglyphid{U1-01_0216}}
% 37
{\PTglyphid{U1-01_0217}}
% 38
{\PTglyphid{U1-01_0218}}
% 39
{\PTglyphid{U1-01_0219}}
% 40
{\PTglyphid{U1-01_0220}}
% 41
{\PTglyphid{U1-01_0221}}
% 42
{\PTglyphid{U1-01_0222}}
% 43
{\PTglyphid{U1-01_0223}}
% 44
{\PTglyphid{U1-01_0224}}
% 45
{\PTglyphid{U1-01_0225}}
% 46
{\PTglyphid{U1-01_0226}}
% 47
{\PTglyphid{U1-01_0227}}
% 48
{\PTglyphid{U1-01_0228}}
% 49
{\PTglyphid{U1-01_0229}}
% 50
{\PTglyphid{U1-01_0301}}
% 51
{\PTglyphid{U1-01_0302}}
% 52
{\PTglyphid{U1-01_0303}}
% 53
{\PTglyphid{U1-01_0304}}
% 54
{\PTglyphid{U1-01_0305}}
% 55
{\PTglyphid{U1-01_0306}}
% 56
{\PTglyphid{U1-01_0307}}
% 57
{\PTglyphid{U1-01_0308}}
% 58
{\PTglyphid{U1-01_0309}}
% 59
{\PTglyphid{U1-01_0310}}
% 60
{\PTglyphid{U1-01_0311}}
% 61
{\PTglyphid{U1-01_0312}}
% 62
{\PTglyphid{U1-01_0313}}
% 63
{\PTglyphid{U1-01_0314}}
% 64
{\PTglyphid{U1-01_0315}}
% 65
{\PTglyphid{U1-01_0316}}
% 66
{\PTglyphid{U1-01_0317}}
% 67
{\PTglyphid{U1-01_0318}}
% 68
{\PTglyphid{U1-01_0319}}
% 69
{\PTglyphid{U1-01_0320}}
% 70
{\PTglyphid{U1-01_0321}}
% 71
{\PTglyphid{U1-01_0322}}
% 72
{\PTglyphid{U1-01_0323}}
% 73
{\PTglyphid{U1-01_0324}}
% 74
{\PTglyphid{U1-01_0325}}
% 75
{\PTglyphid{U1-01_0326}}
% 76
{\PTglyphid{U1-01_0327}}
% 77
{\PTglyphid{U1-01_0401}}
% 78
{\PTglyphid{U1-01_0402}}
% 79
{\PTglyphid{U1-01_0403}}
% 80
{\PTglyphid{U1-01_0404}}
% 81
{\PTglyphid{U1-01_0405}}
% 82
{\PTglyphid{U1-01_0501}}
% 83
{\PTglyphid{U1-01_0502}}
% 84
{\PTglyphid{U1-01_0503}}
% 85
{\PTglyphid{U1-01_0504}}
% 86
{\PTglyphid{U1-01_0505}}
% 87
{\PTglyphid{U1-01_0506}}
% 88
{\PTglyphid{U1-01_0507}}
% 89
{\PTglyphid{U1-01_0508}}
% 90
{\PTglyphid{U1-01_0509}}
% 91
{\PTglyphid{U1-01_0510}}
% 92
{\PTglyphid{U1-01_0511}}
% 93
{\PTglyphid{U1-01_0512}}
% 94
{\PTglyphid{U1-01_0513}}
//
\endgl \xe
%%% Local Variables:
%%% mode: latex
%%% TeX-engine: luatex
%%% TeX-master: shared
%%% End:

% //
%\endgl \xe

 \newpage
 
%%%%%%%%%%%%%%%%%%%%%%%%%%%%%%%%%%%%%%%%%%%%%%%%%%%%%%%%%%%%%%%%%%%%%%%%%%%%%%%
% from meta.csv
 % Tab. 43,Ungler1-02_PT03_113.djvu,Ungler1,02,03,113
%%%%%%%%%%%%%%%%%%%%%%%%%%%%%%%%%%%%%%%%%%%%%%%%%%%%%%%%%%%%%%%%%%%%%%%%%%%%%%%

 % from dsed4test:

% Note "2. Pismo tekstowe gotyckie. Krój M⁹¹. Stopień 20 ww. == 76/77 mm. — Tabl. 113."
% Note1 "Character set table prepared by Maria Błońska"

\pismoPL{Florian Ungler  pierwsza drukarnia 2. Pismo tekstowe gotyckie. Krój M⁹¹. Stopień 20 ww. == 76/77 mm. — Tabl. 113.}
  
 \pismoEN{Florian Ungler  first house 2. Gothic text font. Typeface M⁹¹. Type size 20 ww. == 76/77 mm. — Plate 113.}

\plate{113}{III}{1959}

The plate prepared by Henryk Bułhak.\\
The font table prepared by  Henryk Bułhak and Maria Błońska.

\bigskip

\exampleBib{III:9}

\bigskip \exampleDesc{STANISLAUS ZABOROWSKI: Tractatus contra malos divites et usurarios.
Kraków, Florian Ungler, 10. IV. 1512. 4⁰.}

\medskip
\examplePage{\textit{Karta A₃a}}

  \bigskip
  \exampleLib{Biblioteka Narodowa. Warszawa.}
  

\bigskip \exampleRef{\textit{Estreicher XVI. 242, XXXIV. 55.
    Wierzbowski 882. Piekarski U. 8.}}

  \bigskip
  \exampleDig{\url{https://cyfrowe.mnk.pl/dlibra/publication/863/} page 9}
%\url{https://dbc.wroc.pl/dlibra/publication/11691/} page ???. nie to wydanie???

  % \medskip
\bigskip

\examplePL{[Pismo 2] Pismo 9: naglówek. — [JSB niejasne:] Cyfry 3:
  pierwszy zestaw. — Cyfry 4: drugi zestaw.}

\medskip

\exampleEN{[Font 2] Font 9: the header. — [JSB unclear:] Digits 3:
  first set. — Digits 4: second set.}



\bigskip

\fontID{U1-02}{43}

\fontstat{120}

% \exdisplay \bg \gla
 \exdisplay \bg \gla
% 1
{\PTglyph{5}{t43_l01g01.png}}
% 2
{\PTglyph{5}{t43_l01g02.png}}
% 3
{\PTglyph{5}{t43_l01g03.png}}
% 4
{\PTglyph{5}{t43_l01g04.png}}
% 5
{\PTglyph{5}{t43_l01g05.png}}
% 6
{\PTglyph{5}{t43_l01g06.png}}
% 7
{\PTglyph{5}{t43_l01g07.png}}
% 8
{\PTglyph{5}{t43_l01g08.png}}
% 9
{\PTglyph{5}{t43_l01g09.png}}
% 10
{\PTglyph{5}{t43_l01g10.png}}
% 11
{\PTglyph{5}{t43_l01g11.png}}
% 12
{\PTglyph{5}{t43_l01g12.png}}
% 13
{\PTglyph{5}{t43_l01g13.png}}
% 14
{\PTglyph{5}{t43_l01g14.png}}
% 15
{\PTglyph{5}{t43_l01g15.png}}
% 16
{\PTglyph{5}{t43_l01g16.png}}
% 17
{\PTglyph{5}{t43_l01g17.png}}
% 18
{\PTglyph{5}{t43_l01g18.png}}
% 19
{\PTglyph{5}{t43_l01g19.png}}
% 20
{\PTglyph{5}{t43_l01g20.png}}
% 21
{\PTglyph{5}{t43_l01g21.png}}
% 22
{\PTglyph{5}{t43_l01g22.png}}
% 23
{\PTglyph{5}{t43_l01g23.png}}
% 24
{\PTglyph{5}{t43_l01g24.png}}
% 25
{\PTglyph{5}{t43_l01g25.png}}
% 26
{\PTglyph{5}{t43_l01g26.png}}
% 27
{\PTglyph{5}{t43_l01g27.png}}
% 28
{\PTglyph{5}{t43_l01g28.png}}
% 29
{\PTglyph{5}{t43_l01g29.png}}
% 30
{\PTglyph{5}{t43_l02g01.png}}
% 31
{\PTglyph{5}{t43_l02g02.png}}
% 32
{\PTglyph{5}{t43_l02g03.png}}
% 33
{\PTglyph{5}{t43_l02g04.png}}
% 34
{\PTglyph{5}{t43_l02g05.png}}
% 35
{\PTglyph{5}{t43_l02g06.png}}
% 36
{\PTglyph{5}{t43_l02g07.png}}
% 37
{\PTglyph{5}{t43_l02g08.png}}
% 38
{\PTglyph{5}{t43_l02g09.png}}
% 39
{\PTglyph{5}{t43_l02g10.png}}
% 40
{\PTglyph{5}{t43_l02g11.png}}
% 41
{\PTglyph{5}{t43_l02g12.png}}
% 42
{\PTglyph{5}{t43_l02g13.png}}
% 43
{\PTglyph{5}{t43_l02g14.png}}
% 44
{\PTglyph{5}{t43_l02g15.png}}
% 45
{\PTglyph{5}{t43_l02g16.png}}
% 46
{\PTglyph{5}{t43_l02g17.png}}
% 47
{\PTglyph{5}{t43_l02g18.png}}
% 48
{\PTglyph{5}{t43_l02g19.png}}
% 49
{\PTglyph{5}{t43_l02g20.png}}
% 50
{\PTglyph{5}{t43_l02g21.png}}
% 51
{\PTglyph{5}{t43_l02g22.png}}
% 52
{\PTglyph{5}{t43_l02g23.png}}
% 53
{\PTglyph{5}{t43_l02g24.png}}
% 54
{\PTglyph{5}{t43_l02g25.png}}
% 55
{\PTglyph{5}{t43_l02g26.png}}
% 56
{\PTglyph{5}{t43_l02g27.png}}
% 57
{\PTglyph{5}{t43_l02g28.png}}
% 58
{\PTglyph{5}{t43_l02g29.png}}
% 59
{\PTglyph{5}{t43_l02g30.png}}
% 60
{\PTglyph{5}{t43_l02g31.png}}
% 61
{\PTglyph{5}{t43_l02g32.png}}
% 62
{\PTglyph{5}{t43_l02g33.png}}
% 63
{\PTglyph{5}{t43_l02g34.png}}
% 64
{\PTglyph{5}{t43_l02g35.png}}
% 65
{\PTglyph{5}{t43_l02g36.png}}
% 66
{\PTglyph{5}{t43_l02g37.png}}
% 67
{\PTglyph{5}{t43_l02g38.png}}
% 68
{\PTglyph{5}{t43_l02g39.png}}
% 69
{\PTglyph{5}{t43_l02g40.png}}
% 70
{\PTglyph{5}{t43_l02g41.png}}
% 71
{\PTglyph{5}{t43_l02g42.png}}
% 72
{\PTglyph{5}{t43_l02g43.png}}
% 73
{\PTglyph{5}{t43_l02g44.png}}
% 74
{\PTglyph{5}{t43_l03g01.png}}
% 75
{\PTglyph{5}{t43_l03g02.png}}
% 76
{\PTglyph{5}{t43_l03g03.png}}
% 77
{\PTglyph{5}{t43_l03g04.png}}
% 78
{\PTglyph{5}{t43_l03g05.png}}
% 79
{\PTglyph{5}{t43_l03g06.png}}
% 80
{\PTglyph{5}{t43_l03g07.png}}
% 81
{\PTglyph{5}{t43_l03g08.png}}
% 82
{\PTglyph{5}{t43_l03g09.png}}
% 83
{\PTglyph{5}{t43_l03g10.png}}
% 84
{\PTglyph{5}{t43_l03g11.png}}
% 85
{\PTglyph{5}{t43_l03g12.png}}
% 86
{\PTglyph{5}{t43_l03g13.png}}
% 87
{\PTglyph{5}{t43_l03g14.png}}
% 88
{\PTglyph{5}{t43_l03g15.png}}
% 89
{\PTglyph{5}{t43_l03g16.png}}
% 90
{\PTglyph{5}{t43_l03g17.png}}
% 91
{\PTglyph{5}{t43_l03g18.png}}
% 92
{\PTglyph{5}{t43_l03g19.png}}
% 93
{\PTglyph{5}{t43_l03g20.png}}
% 94
{\PTglyph{5}{t43_l03g21.png}}
% 95
{\PTglyph{5}{t43_l03g22.png}}
% 96
{\PTglyph{5}{t43_l03g23.png}}
% 97
{\PTglyph{5}{t43_l03g24.png}}
% 98
{\PTglyph{5}{t43_l03g25.png}}
% 99
{\PTglyph{5}{t43_l03g26.png}}
% 100
{\PTglyph{5}{t43_l03g27.png}}
% 101
{\PTglyph{5}{t43_l03g28.png}}
% 102
{\PTglyph{5}{t43_l03g29.png}}
% 103
{\PTglyph{5}{t43_l03g30.png}}
% 104
{\PTglyph{5}{t43_l03g31.png}}
% 105
{\PTglyph{5}{t43_l03g32.png}}
% 106
{\PTglyph{5}{t43_l03g33.png}}
% 107
{\PTglyph{5}{t43_l03g34.png}}
% 108
{\PTglyph{5}{t43_l03g35.png}}
% 109
{\PTglyph{5}{t43_l03g36.png}}
% 110
{\PTglyph{5}{t43_l03g37.png}}
% 111
{\PTglyph{5}{t43_l03g38.png}}
% 112
{\PTglyph{5}{t43_l04g01.png}}
% 113
{\PTglyph{5}{t43_l04g02.png}}
% 114
{\PTglyph{5}{t43_l04g03.png}}
% 115
{\PTglyph{5}{t43_l04g04.png}}
% 116
{\PTglyph{5}{t43_l04g05.png}}
% 117
{\PTglyph{5}{t43_l04g06.png}}
% 118
{\PTglyph{5}{t43_l04g07.png}}
% 119
{\PTglyph{5}{t43_l04g08.png}}
% 120
{\PTglyph{5}{t43_l04g09.png}}
//
%%% Local Variables:
%%% mode: latex
%%% TeX-engine: luatex
%%% TeX-master: shared
%%% End:

%//
%\glpismo%
 \glpismo
% 1
{\PTglyphid{U1-02_0101}}
% 2
{\PTglyphid{U1-02_0102}}
% 3
{\PTglyphid{U1-02_0103}}
% 4
{\PTglyphid{U1-02_0104}}
% 5
{\PTglyphid{U1-02_0105}}
% 6
{\PTglyphid{U1-02_0106}}
% 7
{\PTglyphid{U1-02_0107}}
% 8
{\PTglyphid{U1-02_0108}}
% 9
{\PTglyphid{U1-02_0109}}
% 10
{\PTglyphid{U1-02_0110}}
% 11
{\PTglyphid{U1-02_0111}}
% 12
{\PTglyphid{U1-02_0112}}
% 13
{\PTglyphid{U1-02_0113}}
% 14
{\PTglyphid{U1-02_0114}}
% 15
{\PTglyphid{U1-02_0115}}
% 16
{\PTglyphid{U1-02_0116}}
% 17
{\PTglyphid{U1-02_0117}}
% 18
{\PTglyphid{U1-02_0118}}
% 19
{\PTglyphid{U1-02_0119}}
% 20
{\PTglyphid{U1-02_0120}}
% 21
{\PTglyphid{U1-02_0121}}
% 22
{\PTglyphid{U1-02_0122}}
% 23
{\PTglyphid{U1-02_0123}}
% 24
{\PTglyphid{U1-02_0124}}
% 25
{\PTglyphid{U1-02_0125}}
% 26
{\PTglyphid{U1-02_0126}}
% 27
{\PTglyphid{U1-02_0127}}
% 28
{\PTglyphid{U1-02_0128}}
% 29
{\PTglyphid{U1-02_0129}}
% 30
{\PTglyphid{U1-02_0201}}
% 31
{\PTglyphid{U1-02_0202}}
% 32
{\PTglyphid{U1-02_0203}}
% 33
{\PTglyphid{U1-02_0204}}
% 34
{\PTglyphid{U1-02_0205}}
% 35
{\PTglyphid{U1-02_0206}}
% 36
{\PTglyphid{U1-02_0207}}
% 37
{\PTglyphid{U1-02_0208}}
% 38
{\PTglyphid{U1-02_0209}}
% 39
{\PTglyphid{U1-02_0210}}
% 40
{\PTglyphid{U1-02_0211}}
% 41
{\PTglyphid{U1-02_0212}}
% 42
{\PTglyphid{U1-02_0213}}
% 43
{\PTglyphid{U1-02_0214}}
% 44
{\PTglyphid{U1-02_0215}}
% 45
{\PTglyphid{U1-02_0216}}
% 46
{\PTglyphid{U1-02_0217}}
% 47
{\PTglyphid{U1-02_0218}}
% 48
{\PTglyphid{U1-02_0219}}
% 49
{\PTglyphid{U1-02_0220}}
% 50
{\PTglyphid{U1-02_0221}}
% 51
{\PTglyphid{U1-02_0222}}
% 52
{\PTglyphid{U1-02_0223}}
% 53
{\PTglyphid{U1-02_0224}}
% 54
{\PTglyphid{U1-02_0225}}
% 55
{\PTglyphid{U1-02_0226}}
% 56
{\PTglyphid{U1-02_0227}}
% 57
{\PTglyphid{U1-02_0228}}
% 58
{\PTglyphid{U1-02_0229}}
% 59
{\PTglyphid{U1-02_0230}}
% 60
{\PTglyphid{U1-02_0231}}
% 61
{\PTglyphid{U1-02_0232}}
% 62
{\PTglyphid{U1-02_0233}}
% 63
{\PTglyphid{U1-02_0234}}
% 64
{\PTglyphid{U1-02_0235}}
% 65
{\PTglyphid{U1-02_0236}}
% 66
{\PTglyphid{U1-02_0237}}
% 67
{\PTglyphid{U1-02_0238}}
% 68
{\PTglyphid{U1-02_0239}}
% 69
{\PTglyphid{U1-02_0240}}
% 70
{\PTglyphid{U1-02_0241}}
% 71
{\PTglyphid{U1-02_0242}}
% 72
{\PTglyphid{U1-02_0243}}
% 73
{\PTglyphid{U1-02_0244}}
% 74
{\PTglyphid{U1-02_0301}}
% 75
{\PTglyphid{U1-02_0302}}
% 76
{\PTglyphid{U1-02_0303}}
% 77
{\PTglyphid{U1-02_0304}}
% 78
{\PTglyphid{U1-02_0305}}
% 79
{\PTglyphid{U1-02_0306}}
% 80
{\PTglyphid{U1-02_0307}}
% 81
{\PTglyphid{U1-02_0308}}
% 82
{\PTglyphid{U1-02_0309}}
% 83
{\PTglyphid{U1-02_0310}}
% 84
{\PTglyphid{U1-02_0311}}
% 85
{\PTglyphid{U1-02_0312}}
% 86
{\PTglyphid{U1-02_0313}}
% 87
{\PTglyphid{U1-02_0314}}
% 88
{\PTglyphid{U1-02_0315}}
% 89
{\PTglyphid{U1-02_0316}}
% 90
{\PTglyphid{U1-02_0317}}
% 91
{\PTglyphid{U1-02_0318}}
% 92
{\PTglyphid{U1-02_0319}}
% 93
{\PTglyphid{U1-02_0320}}
% 94
{\PTglyphid{U1-02_0321}}
% 95
{\PTglyphid{U1-02_0322}}
% 96
{\PTglyphid{U1-02_0323}}
% 97
{\PTglyphid{U1-02_0324}}
% 98
{\PTglyphid{U1-02_0325}}
% 99
{\PTglyphid{U1-02_0326}}
% 100
{\PTglyphid{U1-02_0327}}
% 101
{\PTglyphid{U1-02_0328}}
% 102
{\PTglyphid{U1-02_0329}}
% 103
{\PTglyphid{U1-02_0330}}
% 104
{\PTglyphid{U1-02_0331}}
% 105
{\PTglyphid{U1-02_0332}}
% 106
{\PTglyphid{U1-02_0333}}
% 107
{\PTglyphid{U1-02_0334}}
% 108
{\PTglyphid{U1-02_0335}}
% 109
{\PTglyphid{U1-02_0336}}
% 110
{\PTglyphid{U1-02_0337}}
% 111
{\PTglyphid{U1-02_0338}}
% 112
{\PTglyphid{U1-02_0401}}
% 113
{\PTglyphid{U1-02_0402}}
% 114
{\PTglyphid{U1-02_0403}}
% 115
{\PTglyphid{U1-02_0404}}
% 116
{\PTglyphid{U1-02_0405}}
% 117
{\PTglyphid{U1-02_0406}}
% 118
{\PTglyphid{U1-02_0407}}
% 119
{\PTglyphid{U1-02_0408}}
% 120
{\PTglyphid{U1-02_0409}}
//
\endgl \xe
%%% Local Variables:
%%% mode: latex
%%% TeX-engine: luatex
%%% TeX-master: shared
%%% End:

% //
%\endgl \xe


 \newpage
 
%%%%%%%%%%%%%%%%%%%%%%%%%%%%%%%%%%%%%%%%%%%%%%%%%%%%%%%%%%%%%%%%%%%%%%%%%%%%%%%
% from meta.csv
 % Tab. 44,Ungler1-03_PT03_114.djvu,Ungler1,03,03,114
%%%%%%%%%%%%%%%%%%%%%%%%%%%%%%%%%%%%%%%%%%%%%%%%%%%%%%%%%%%%%%%%%%%%%%%%%%%%%%%

 % from dsed4test:

 % Note "3. Pismo tekstowe gotyckie. Krój M⁹¹. Stopień 20 ww. == 72/73 mm. — Tabl. 114."
% Note1 "Character set table prepared by Maria Błońska"

\pismoPL{Florian Ungler  pierwsza drukarnia 3. Pismo tekstowe gotyckie. Krój M⁹¹. Stopień 20 ww. == 72/73 mm. — Tabl. 114."}
  
 \pismoEN{Florian Ungler  first house 3. Gothic text font. Typeface M⁹¹. Type size 20 ww. == 72/73 mm. — Plate 114.}

\plate{114}{III}{1959}

The plate prepared by Henryk Bułhak.\\
The font table prepared by Maria Błońska.

\bigskip

\exampleBib{III:60$^a$}

\bigskip \exampleDesc{IOANNES DE AUERBACH:
Processus iudiciarius. Arkusze K—O. Kraków, Florian Ungler [1514]. 4⁰.}


\medskip
\examplePage{\textit{Karta M₂a}}

  \bigskip
  \exampleLib{Biblioteka Narodowa. Warszawa.}
  

\bigskip \exampleRef{\textit{Estreicher XXXII. 57. Piekarski U. 57.}}

  \bigskip
  \exampleDig{\url{ https://cyfrowe.mnk.pl/dlibra/publication/865/} page 95}
  
% Lipsk  https://kpbc.umk.pl/dlibra/publication/edition/146354/content
% Paris   https://bibliotekacyfrowa.pl/en/dlibra/publication/53621/processus-iudiciarius-johannes-de-auerbach-1405-1469
%  Ungler 1516? https://wbc.poznan.pl/dlibra/publication/505385/edition/473732/content?ref=L3B1YmxpY2F0aW9uLzQ5NDE4NC9lZGl0aW9uLzQyNTk4OA
 
  
  % \medskip
\bigskip

\examplePL{[Pismo 3] Rubryka \beta{}. — Cyfry 5.}

\medskip

\exampleEN{[Font 3] Rubric \beta{}. — Digits 5.}



\bigskip

\fontID{U1-03}{43}

\fontstat{129}

% \exdisplay \bg \gla
 \exdisplay \bg \gla
% 1
{\PTglyph{5}{t44_l01g01.png}}
% 2
{\PTglyph{5}{t44_l01g02.png}}
% 3
{\PTglyph{5}{t44_l01g03.png}}
% 4
{\PTglyph{5}{t44_l01g04.png}}
% 5
{\PTglyph{5}{t44_l01g05.png}}
% 6
{\PTglyph{5}{t44_l01g06.png}}
% 7
{\PTglyph{5}{t44_l01g07.png}}
% 8
{\PTglyph{5}{t44_l01g08.png}}
% 9
{\PTglyph{5}{t44_l01g09.png}}
% 10
{\PTglyph{5}{t44_l01g10.png}}
% 11
{\PTglyph{5}{t44_l01g11.png}}
% 12
{\PTglyph{5}{t44_l01g12.png}}
% 13
{\PTglyph{5}{t44_l01g13.png}}
% 14
{\PTglyph{5}{t44_l01g14.png}}
% 15
{\PTglyph{5}{t44_l01g15.png}}
% 16
{\PTglyph{5}{t44_l01g16.png}}
% 17
{\PTglyph{5}{t44_l01g17.png}}
% 18
{\PTglyph{5}{t44_l01g18.png}}
% 19
{\PTglyph{5}{t44_l01g19.png}}
% 20
{\PTglyph{5}{t44_l01g20.png}}
% 21
{\PTglyph{5}{t44_l01g21.png}}
% 22
{\PTglyph{5}{t44_l01g22.png}}
% 23
{\PTglyph{5}{t44_l01g23.png}}
% 24
{\PTglyph{5}{t44_l02g01.png}}
% 25
{\PTglyph{5}{t44_l02g02.png}}
% 26
{\PTglyph{5}{t44_l02g03.png}}
% 27
{\PTglyph{5}{t44_l02g04.png}}
% 28
{\PTglyph{5}{t44_l02g05.png}}
% 29
{\PTglyph{5}{t44_l02g06.png}}
% 30
{\PTglyph{5}{t44_l02g07.png}}
% 31
{\PTglyph{5}{t44_l02g08.png}}
% 32
{\PTglyph{5}{t44_l02g09.png}}
% 33
{\PTglyph{5}{t44_l02g10.png}}
% 34
{\PTglyph{5}{t44_l02g11.png}}
% 35
{\PTglyph{5}{t44_l02g12.png}}
% 36
{\PTglyph{5}{t44_l02g13.png}}
% 37
{\PTglyph{5}{t44_l02g14.png}}
% 38
{\PTglyph{5}{t44_l02g15.png}}
% 39
{\PTglyph{5}{t44_l02g16.png}}
% 40
{\PTglyph{5}{t44_l02g17.png}}
% 41
{\PTglyph{5}{t44_l02g18.png}}
% 42
{\PTglyph{5}{t44_l02g19.png}}
% 43
{\PTglyph{5}{t44_l02g20.png}}
% 44
{\PTglyph{5}{t44_l02g21.png}}
% 45
{\PTglyph{5}{t44_l02g22.png}}
% 46
{\PTglyph{5}{t44_l02g23.png}}
% 47
{\PTglyph{5}{t44_l02g24.png}}
% 48
{\PTglyph{5}{t44_l02g25.png}}
% 49
{\PTglyph{5}{t44_l02g26.png}}
% 50
{\PTglyph{5}{t44_l02g27.png}}
% 51
{\PTglyph{5}{t44_l02g28.png}}
% 52
{\PTglyph{5}{t44_l02g29.png}}
% 53
{\PTglyph{5}{t44_l02g30.png}}
% 54
{\PTglyph{5}{t44_l02g31.png}}
% 55
{\PTglyph{5}{t44_l02g32.png}}
% 56
{\PTglyph{5}{t44_l02g33.png}}
% 57
{\PTglyph{5}{t44_l02g34.png}}
% 58
{\PTglyph{5}{t44_l02g35.png}}
% 59
{\PTglyph{5}{t44_l03g01.png}}
% 60
{\PTglyph{5}{t44_l03g02.png}}
% 61
{\PTglyph{5}{t44_l03g03.png}}
% 62
{\PTglyph{5}{t44_l03g04.png}}
% 63
{\PTglyph{5}{t44_l03g05.png}}
% 64
{\PTglyph{5}{t44_l03g06.png}}
% 65
{\PTglyph{5}{t44_l03g07.png}}
% 66
{\PTglyph{5}{t44_l03g08.png}}
% 67
{\PTglyph{5}{t44_l03g09.png}}
% 68
{\PTglyph{5}{t44_l03g10.png}}
% 69
{\PTglyph{5}{t44_l03g11.png}}
% 70
{\PTglyph{5}{t44_l03g12.png}}
% 71
{\PTglyph{5}{t44_l03g13.png}}
% 72
{\PTglyph{5}{t44_l03g14.png}}
% 73
{\PTglyph{5}{t44_l03g15.png}}
% 74
{\PTglyph{5}{t44_l03g16.png}}
% 75
{\PTglyph{5}{t44_l03g17.png}}
% 76
{\PTglyph{5}{t44_l03g18.png}}
% 77
{\PTglyph{5}{t44_l03g19.png}}
% 78
{\PTglyph{5}{t44_l03g20.png}}
% 79
{\PTglyph{5}{t44_l03g21.png}}
% 80
{\PTglyph{5}{t44_l03g22.png}}
% 81
{\PTglyph{5}{t44_l03g23.png}}
% 82
{\PTglyph{5}{t44_l03g24.png}}
% 83
{\PTglyph{5}{t44_l03g25.png}}
% 84
{\PTglyph{5}{t44_l03g26.png}}
% 85
{\PTglyph{5}{t44_l03g27.png}}
% 86
{\PTglyph{5}{t44_l03g28.png}}
% 87
{\PTglyph{5}{t44_l03g29.png}}
% 88
{\PTglyph{5}{t44_l03g30.png}}
% 89
{\PTglyph{5}{t44_l03g31.png}}
% 90
{\PTglyph{5}{t44_l04g01.png}}
% 91
{\PTglyph{5}{t44_l04g02.png}}
% 92
{\PTglyph{5}{t44_l04g03.png}}
% 93
{\PTglyph{5}{t44_l04g04.png}}
% 94
{\PTglyph{5}{t44_l04g05.png}}
% 95
{\PTglyph{5}{t44_l04g06.png}}
% 96
{\PTglyph{5}{t44_l04g07.png}}
% 97
{\PTglyph{5}{t44_l04g08.png}}
% 98
{\PTglyph{5}{t44_l04g09.png}}
% 99
{\PTglyph{5}{t44_l04g10.png}}
% 100
{\PTglyph{5}{t44_l04g11.png}}
% 101
{\PTglyph{5}{t44_l04g12.png}}
% 102
{\PTglyph{5}{t44_l04g13.png}}
% 103
{\PTglyph{5}{t44_l04g14.png}}
% 104
{\PTglyph{5}{t44_l04g15.png}}
% 105
{\PTglyph{5}{t44_l04g16.png}}
% 106
{\PTglyph{5}{t44_l04g17.png}}
% 107
{\PTglyph{5}{t44_l04g18.png}}
% 108
{\PTglyph{5}{t44_l04g19.png}}
% 109
{\PTglyph{5}{t44_l04g20.png}}
% 110
{\PTglyph{5}{t44_l04g21.png}}
% 111
{\PTglyph{5}{t44_l04g22.png}}
% 112
{\PTglyph{5}{t44_l04g23.png}}
% 113
{\PTglyph{5}{t44_l04g24.png}}
% 114
{\PTglyph{5}{t44_l04g25.png}}
% 115
{\PTglyph{5}{t44_l04g26.png}}
% 116
{\PTglyph{5}{t44_l04g27.png}}
% 117
{\PTglyph{5}{t44_l04g28.png}}
% 118
{\PTglyph{5}{t44_l04g29.png}}
% 119
{\PTglyph{5}{t44_l04g30.png}}
% 120
{\PTglyph{5}{t44_l05g01.png}}
% 121
{\PTglyph{5}{t44_l05g02.png}}
% 122
{\PTglyph{5}{t44_l05g03.png}}
% 123
{\PTglyph{5}{t44_l05g04.png}}
% 124
{\PTglyph{5}{t44_l05g05.png}}
% 125
{\PTglyph{5}{t44_l05g06.png}}
% 126
{\PTglyph{5}{t44_l05g07.png}}
% 127
{\PTglyph{5}{t44_l05g08.png}}
% 128
{\PTglyph{5}{t44_l05g09.png}}
% 129
{\PTglyph{5}{t44_l05g10.png}}
//
%%% Local Variables:
%%% mode: latex
%%% TeX-engine: luatex
%%% TeX-master: shared
%%% End:

%//
%\glpismo%
 \glpismo
% 1
{\PTglyphid{U1-03_0101}}
% 2
{\PTglyphid{U1-03_0102}}
% 3
{\PTglyphid{U1-03_0103}}
% 4
{\PTglyphid{U1-03_0104}}
% 5
{\PTglyphid{U1-03_0105}}
% 6
{\PTglyphid{U1-03_0106}}
% 7
{\PTglyphid{U1-03_0107}}
% 8
{\PTglyphid{U1-03_0108}}
% 9
{\PTglyphid{U1-03_0109}}
% 10
{\PTglyphid{U1-03_0110}}
% 11
{\PTglyphid{U1-03_0111}}
% 12
{\PTglyphid{U1-03_0112}}
% 13
{\PTglyphid{U1-03_0113}}
% 14
{\PTglyphid{U1-03_0114}}
% 15
{\PTglyphid{U1-03_0115}}
% 16
{\PTglyphid{U1-03_0116}}
% 17
{\PTglyphid{U1-03_0117}}
% 18
{\PTglyphid{U1-03_0118}}
% 19
{\PTglyphid{U1-03_0119}}
% 20
{\PTglyphid{U1-03_0120}}
% 21
{\PTglyphid{U1-03_0121}}
% 22
{\PTglyphid{U1-03_0122}}
% 23
{\PTglyphid{U1-03_0123}}
% 24
{\PTglyphid{U1-03_0201}}
% 25
{\PTglyphid{U1-03_0202}}
% 26
{\PTglyphid{U1-03_0203}}
% 27
{\PTglyphid{U1-03_0204}}
% 28
{\PTglyphid{U1-03_0205}}
% 29
{\PTglyphid{U1-03_0206}}
% 30
{\PTglyphid{U1-03_0207}}
% 31
{\PTglyphid{U1-03_0208}}
% 32
{\PTglyphid{U1-03_0209}}
% 33
{\PTglyphid{U1-03_0210}}
% 34
{\PTglyphid{U1-03_0211}}
% 35
{\PTglyphid{U1-03_0212}}
% 36
{\PTglyphid{U1-03_0213}}
% 37
{\PTglyphid{U1-03_0214}}
% 38
{\PTglyphid{U1-03_0215}}
% 39
{\PTglyphid{U1-03_0216}}
% 40
{\PTglyphid{U1-03_0217}}
% 41
{\PTglyphid{U1-03_0218}}
% 42
{\PTglyphid{U1-03_0219}}
% 43
{\PTglyphid{U1-03_0220}}
% 44
{\PTglyphid{U1-03_0221}}
% 45
{\PTglyphid{U1-03_0222}}
% 46
{\PTglyphid{U1-03_0223}}
% 47
{\PTglyphid{U1-03_0224}}
% 48
{\PTglyphid{U1-03_0225}}
% 49
{\PTglyphid{U1-03_0226}}
% 50
{\PTglyphid{U1-03_0227}}
% 51
{\PTglyphid{U1-03_0228}}
% 52
{\PTglyphid{U1-03_0229}}
% 53
{\PTglyphid{U1-03_0230}}
% 54
{\PTglyphid{U1-03_0231}}
% 55
{\PTglyphid{U1-03_0232}}
% 56
{\PTglyphid{U1-03_0233}}
% 57
{\PTglyphid{U1-03_0234}}
% 58
{\PTglyphid{U1-03_0235}}
% 59
{\PTglyphid{U1-03_0301}}
% 60
{\PTglyphid{U1-03_0302}}
% 61
{\PTglyphid{U1-03_0303}}
% 62
{\PTglyphid{U1-03_0304}}
% 63
{\PTglyphid{U1-03_0305}}
% 64
{\PTglyphid{U1-03_0306}}
% 65
{\PTglyphid{U1-03_0307}}
% 66
{\PTglyphid{U1-03_0308}}
% 67
{\PTglyphid{U1-03_0309}}
% 68
{\PTglyphid{U1-03_0310}}
% 69
{\PTglyphid{U1-03_0311}}
% 70
{\PTglyphid{U1-03_0312}}
% 71
{\PTglyphid{U1-03_0313}}
% 72
{\PTglyphid{U1-03_0314}}
% 73
{\PTglyphid{U1-03_0315}}
% 74
{\PTglyphid{U1-03_0316}}
% 75
{\PTglyphid{U1-03_0317}}
% 76
{\PTglyphid{U1-03_0318}}
% 77
{\PTglyphid{U1-03_0319}}
% 78
{\PTglyphid{U1-03_0320}}
% 79
{\PTglyphid{U1-03_0321}}
% 80
{\PTglyphid{U1-03_0322}}
% 81
{\PTglyphid{U1-03_0323}}
% 82
{\PTglyphid{U1-03_0324}}
% 83
{\PTglyphid{U1-03_0325}}
% 84
{\PTglyphid{U1-03_0326}}
% 85
{\PTglyphid{U1-03_0327}}
% 86
{\PTglyphid{U1-03_0328}}
% 87
{\PTglyphid{U1-03_0329}}
% 88
{\PTglyphid{U1-03_0330}}
% 89
{\PTglyphid{U1-03_0331}}
% 90
{\PTglyphid{U1-03_0401}}
% 91
{\PTglyphid{U1-03_0402}}
% 92
{\PTglyphid{U1-03_0403}}
% 93
{\PTglyphid{U1-03_0404}}
% 94
{\PTglyphid{U1-03_0405}}
% 95
{\PTglyphid{U1-03_0406}}
% 96
{\PTglyphid{U1-03_0407}}
% 97
{\PTglyphid{U1-03_0408}}
% 98
{\PTglyphid{U1-03_0409}}
% 99
{\PTglyphid{U1-03_0410}}
% 100
{\PTglyphid{U1-03_0411}}
% 101
{\PTglyphid{U1-03_0412}}
% 102
{\PTglyphid{U1-03_0413}}
% 103
{\PTglyphid{U1-03_0414}}
% 104
{\PTglyphid{U1-03_0415}}
% 105
{\PTglyphid{U1-03_0416}}
% 106
{\PTglyphid{U1-03_0417}}
% 107
{\PTglyphid{U1-03_0418}}
% 108
{\PTglyphid{U1-03_0419}}
% 109
{\PTglyphid{U1-03_0420}}
% 110
{\PTglyphid{U1-03_0421}}
% 111
{\PTglyphid{U1-03_0422}}
% 112
{\PTglyphid{U1-03_0423}}
% 113
{\PTglyphid{U1-03_0424}}
% 114
{\PTglyphid{U1-03_0425}}
% 115
{\PTglyphid{U1-03_0426}}
% 116
{\PTglyphid{U1-03_0427}}
% 117
{\PTglyphid{U1-03_0428}}
% 118
{\PTglyphid{U1-03_0429}}
% 119
{\PTglyphid{U1-03_0430}}
% 120
{\PTglyphid{U1-03_0501}}
% 121
{\PTglyphid{U1-03_0502}}
% 122
{\PTglyphid{U1-03_0503}}
% 123
{\PTglyphid{U1-03_0504}}
% 124
{\PTglyphid{U1-03_0505}}
% 125
{\PTglyphid{U1-03_0506}}
% 126
{\PTglyphid{U1-03_0507}}
% 127
{\PTglyphid{U1-03_0508}}
% 128
{\PTglyphid{U1-03_0509}}
% 129
{\PTglyphid{U1-03_0510}}
//
\endgl \xe
%%% Local Variables:
%%% mode: latex
%%% TeX-engine: luatex
%%% TeX-master: shared
%%% End:

% //
%\endgl \xe



 \newpage
 
%%%%%%%%%%%%%%%%%%%%%%%%%%%%%%%%%%%%%%%%%%%%%%%%%%%%%%%%%%%%%%%%%%%%%%%%%%%%%%%
% from meta.csv
 % Tab. 45,Ungler1-04_PT03_115.djvu,Ungler1,04,03,115
%%%%%%%%%%%%%%%%%%%%%%%%%%%%%%%%%%%%%%%%%%%%%%%%%%%%%%%%%%%%%%%%%%%%%%%%%%%%%%%

 % from dsed4test:
% Note "4. Pismo tekstowe gotyckie. Krój M⁴⁸. Stopień 20 ww. == 79/81 mm. — Tabl. 115."
% Note1 "Character set table prepared by Maria Błońska"


\pismoPL{Florian Ungler  pierwsza drukarnia 4. Pismo tekstowe gotyckie. Krój M⁴⁸. Stopień 20 ww. == 79/81 mm. — Tabl. 115.}
  
 \pismoEN{Florian Ungler  first house 4. Gothic text font. Typeface M⁴⁸. Type size 20 ww. == 79/81 mm. — Plate 115.}

\plate{115}{III}{1959}

The plate prepared by Henryk Bułhak.\\
The font table prepared by Henryk Bułhak and Maria Błońska.

\bigskip

\exampleBib{III:7}

\bigskip \exampleDesc{ARISTOTELES: Oeconomicorum libri duo. Trad. Leon. Bruti Aretini. Kraków, Florian Ungler [1512]. 4⁰.}


\medskip
\examplePage{\textit{Karta A₂b}}

  \bigskip
  \exampleLib{Biblioteka Jagiellońska. Kraków.}
  

\bigskip \exampleRef{\textit{Estreicher XII. 214. Piekarski U. 6.}}

  \bigskip
  \exampleDig{\url{ https://cyfrowe.mnk.pl/dlibra/publication/865/} page 95}
% Lipsk  https://kpbc.umk.pl/dlibra/publication/edition/146354/content
% Paris   https://bibliotekacyfrowa.pl/en/dlibra/publication/53621/processus-iudiciarius-johannes-de-auerbach-1405-1469
%  Ungler 1516? https://wbc.poznan.pl/dlibra/publication/505385/edition/473732/content?ref=L3B1YmxpY2F0aW9uLzQ5NDE4NC9lZGl0aW9uLzQyNTk4OA
 
  
  % \medskip
\bigskip

\examplePL{[Pismo 4] Rubryka \beta{}.}

\medskip

\exampleEN{[Font 4] Rubric \beta{}.}



\bigskip

\fontID{U1-04}{45}

\fontstat{74}

% \exdisplay \bg \gla
 \exdisplay \bg \gla
% 1
{\PTglyph{5}{t45_l01g01.png}}
% 2
{\PTglyph{5}{t45_l01g02.png}}
% 3
{\PTglyph{5}{t45_l01g03.png}}
% 4
{\PTglyph{5}{t45_l01g04.png}}
% 5
{\PTglyph{5}{t45_l01g05.png}}
% 6
{\PTglyph{5}{t45_l01g06.png}}
% 7
{\PTglyph{5}{t45_l01g07.png}}
% 8
{\PTglyph{5}{t45_l01g08.png}}
% 9
{\PTglyph{5}{t45_l01g09.png}}
% 10
{\PTglyph{5}{t45_l01g10.png}}
% 11
{\PTglyph{5}{t45_l01g11.png}}
% 12
{\PTglyph{5}{t45_l01g12.png}}
% 13
{\PTglyph{5}{t45_l01g13.png}}
% 14
{\PTglyph{5}{t45_l01g14.png}}
% 15
{\PTglyph{5}{t45_l01g15.png}}
% 16
{\PTglyph{5}{t45_l01g16.png}}
% 17
{\PTglyph{5}{t45_l01g17.png}}
% 18
{\PTglyph{5}{t45_l01g18.png}}
% 19
{\PTglyph{5}{t45_l01g19.png}}
% 20
{\PTglyph{5}{t45_l01g20.png}}
% 21
{\PTglyph{5}{t45_l01g21.png}}
% 22
{\PTglyph{5}{t45_l01g22.png}}
% 23
{\PTglyph{5}{t45_l01g23.png}}
% 24
{\PTglyph{5}{t45_l02g01.png}}
% 25
{\PTglyph{5}{t45_l02g02.png}}
% 26
{\PTglyph{5}{t45_l02g03.png}}
% 27
{\PTglyph{5}{t45_l02g04.png}}
% 28
{\PTglyph{5}{t45_l02g05.png}}
% 29
{\PTglyph{5}{t45_l02g06.png}}
% 30
{\PTglyph{5}{t45_l02g07.png}}
% 31
{\PTglyph{5}{t45_l02g08.png}}
% 32
{\PTglyph{5}{t45_l02g09.png}}
% 33
{\PTglyph{5}{t45_l02g10.png}}
% 34
{\PTglyph{5}{t45_l02g11.png}}
% 35
{\PTglyph{5}{t45_l02g12.png}}
% 36
{\PTglyph{5}{t45_l02g13.png}}
% 37
{\PTglyph{5}{t45_l02g14.png}}
% 38
{\PTglyph{5}{t45_l02g15.png}}
% 39
{\PTglyph{5}{t45_l02g16.png}}
% 40
{\PTglyph{5}{t45_l02g17.png}}
% 41
{\PTglyph{5}{t45_l02g18.png}}
% 42
{\PTglyph{5}{t45_l02g19.png}}
% 43
{\PTglyph{5}{t45_l02g20.png}}
% 44
{\PTglyph{5}{t45_l02g21.png}}
% 45
{\PTglyph{5}{t45_l02g22.png}}
% 46
{\PTglyph{5}{t45_l02g23.png}}
% 47
{\PTglyph{5}{t45_l02g24.png}}
% 48
{\PTglyph{5}{t45_l02g25.png}}
% 49
{\PTglyph{5}{t45_l02g26.png}}
% 50
{\PTglyph{5}{t45_l02g27.png}}
% 51
{\PTglyph{5}{t45_l02g28.png}}
% 52
{\PTglyph{5}{t45_l02g29.png}}
% 53
{\PTglyph{5}{t45_l02g30.png}}
% 54
{\PTglyph{5}{t45_l02g31.png}}
% 55
{\PTglyph{5}{t45_l03g01.png}}
% 56
{\PTglyph{5}{t45_l03g02.png}}
% 57
{\PTglyph{5}{t45_l03g03.png}}
% 58
{\PTglyph{5}{t45_l03g04.png}}
% 59
{\PTglyph{5}{t45_l03g05.png}}
% 60
{\PTglyph{5}{t45_l03g06.png}}
% 61
{\PTglyph{5}{t45_l03g07.png}}
% 62
{\PTglyph{5}{t45_l03g08.png}}
% 63
{\PTglyph{5}{t45_l03g09.png}}
% 64
{\PTglyph{5}{t45_l03g10.png}}
% 65
{\PTglyph{5}{t45_l03g11.png}}
% 66
{\PTglyph{5}{t45_l03g12.png}}
% 67
{\PTglyph{5}{t45_l03g13.png}}
% 68
{\PTglyph{5}{t45_l03g14.png}}
% 69
{\PTglyph{5}{t45_l03g15.png}}
% 70
{\PTglyph{5}{t45_l03g16.png}}
% 71
{\PTglyph{5}{t45_l03g17.png}}
% 72
{\PTglyph{5}{t45_l03g18.png}}
% 73
{\PTglyph{5}{t45_l03g19.png}}
% 74
{\PTglyph{5}{t45_l03g20.png}}
% 75
{\PTglyph{5}{t45_l03g21.png}}
% 76
{\PTglyph{5}{t45_l03g22.png}}
% 77
{\PTglyph{5}{t45_l03g23.png}}
//
%%% Local Variables:
%%% mode: latex
%%% TeX-engine: luatex
%%% TeX-master: shared
%%% End:

%//
%\glpismo%
 \glpismo
% 1
{\PTglyphid{U1-04_0101}}
% 2
{\PTglyphid{U1-04_0102}}
% 3
{\PTglyphid{U1-04_0103}}
% 4
{\PTglyphid{U1-04_0104}}
% 5
{\PTglyphid{U1-04_0105}}
% 6
{\PTglyphid{U1-04_0106}}
% 7
{\PTglyphid{U1-04_0107}}
% 8
{\PTglyphid{U1-04_0108}}
% 9
{\PTglyphid{U1-04_0109}}
% 10
{\PTglyphid{U1-04_0110}}
% 11
{\PTglyphid{U1-04_0111}}
% 12
{\PTglyphid{U1-04_0112}}
% 13
{\PTglyphid{U1-04_0113}}
% 14
{\PTglyphid{U1-04_0114}}
% 15
{\PTglyphid{U1-04_0115}}
% 16
{\PTglyphid{U1-04_0116}}
% 17
{\PTglyphid{U1-04_0117}}
% 18
{\PTglyphid{U1-04_0118}}
% 19
{\PTglyphid{U1-04_0119}}
% 20
{\PTglyphid{U1-04_0120}}
% 21
{\PTglyphid{U1-04_0121}}
% 22
{\PTglyphid{U1-04_0122}}
% 23
{\PTglyphid{U1-04_0123}}
% 24
{\PTglyphid{U1-04_0201}}
% 25
{\PTglyphid{U1-04_0202}}
% 26
{\PTglyphid{U1-04_0203}}
% 27
{\PTglyphid{U1-04_0204}}
% 28
{\PTglyphid{U1-04_0205}}
% 29
{\PTglyphid{U1-04_0206}}
% 30
{\PTglyphid{U1-04_0207}}
% 31
{\PTglyphid{U1-04_0208}}
% 32
{\PTglyphid{U1-04_0209}}
% 33
{\PTglyphid{U1-04_0210}}
% 34
{\PTglyphid{U1-04_0211}}
% 35
{\PTglyphid{U1-04_0212}}
% 36
{\PTglyphid{U1-04_0213}}
% 37
{\PTglyphid{U1-04_0214}}
% 38
{\PTglyphid{U1-04_0215}}
% 39
{\PTglyphid{U1-04_0216}}
% 40
{\PTglyphid{U1-04_0217}}
% 41
{\PTglyphid{U1-04_0218}}
% 42
{\PTglyphid{U1-04_0219}}
% 43
{\PTglyphid{U1-04_0220}}
% 44
{\PTglyphid{U1-04_0221}}
% 45
{\PTglyphid{U1-04_0222}}
% 46
{\PTglyphid{U1-04_0223}}
% 47
{\PTglyphid{U1-04_0224}}
% 48
{\PTglyphid{U1-04_0225}}
% 49
{\PTglyphid{U1-04_0226}}
% 50
{\PTglyphid{U1-04_0227}}
% 51
{\PTglyphid{U1-04_0228}}
% 52
{\PTglyphid{U1-04_0229}}
% 53
{\PTglyphid{U1-04_0230}}
% 54
{\PTglyphid{U1-04_0231}}
% 55
{\PTglyphid{U1-04_0301}}
% 56
{\PTglyphid{U1-04_0302}}
% 57
{\PTglyphid{U1-04_0303}}
% 58
{\PTglyphid{U1-04_0304}}
% 59
{\PTglyphid{U1-04_0305}}
% 60
{\PTglyphid{U1-04_0306}}
% 61
{\PTglyphid{U1-04_0307}}
% 62
{\PTglyphid{U1-04_0308}}
% 63
{\PTglyphid{U1-04_0309}}
% 64
{\PTglyphid{U1-04_0310}}
% 65
{\PTglyphid{U1-04_0311}}
% 66
{\PTglyphid{U1-04_0312}}
% 67
{\PTglyphid{U1-04_0313}}
% 68
{\PTglyphid{U1-04_0314}}
% 69
{\PTglyphid{U1-04_0315}}
% 70
{\PTglyphid{U1-04_0316}}
% 71
{\PTglyphid{U1-04_0317}}
% 72
{\PTglyphid{U1-04_0318}}
% 73
{\PTglyphid{U1-04_0319}}
% 74
{\PTglyphid{U1-04_0320}}
% 75
{\PTglyphid{U1-04_0321}}
% 76
{\PTglyphid{U1-04_0322}}
% 77
{\PTglyphid{U1-04_0323}}
//
\endgl \xe
%%% Local Variables:
%%% mode: latex
%%% TeX-engine: luatex
%%% TeX-master: shared
%%% End:

% //
%\endgl \xe


 \newpage
 
%%%%%%%%%%%%%%%%%%%%%%%%%%%%%%%%%%%%%%%%%%%%%%%%%%%%%%%%%%%%%%%%%%%%%%%%%%%%%%%
% from meta.csv
 % Tab. 46,Ungler1-05_PT03_116.djvu,Ungler1,05,03,116
%%%%%%%%%%%%%%%%%%%%%%%%%%%%%%%%%%%%%%%%%%%%%%%%%%%%%%%%%%%%%%%%%%%%%%%%%%%%%%%

 
 % from dsed4test:
% Note "5. Pismo tekstowe gotyckie. Krój M⁴⁸. Stopień 20 ww. == 89/90 mm. — Tabl. 116."
% Note1 "Character set table prepared by Maria Błońska"



\pismoPL{Florian Ungler  pierwsza drukarnia 5. Pismo tekstowe gotyckie. Krój M⁴⁸. Stopień 20 ww. == 89/90 mm. — Tabl. 116.}
  
 \pismoEN{Florian Ungler  first house 5. Gothic text font. Typeface M⁴⁸. Type size 20 ww. == 89/90 mm. — Plate 116.}

\plate{116}{III}{1959}

The plate prepared by Henryk Bułhak.\\
The font table prepared by Henryk Bułhak and Maria Błońska.

\bigskip

\exampleBib{III:22}

\bigskip \exampleDesc{IOANNES BURHARDUS: Ordo missae cum glossa Stanislai Zaborowski. Kraków,
Florian Ungler, 29. XI. 1512. 4⁰}


\medskip
\examplePage{\textit{Karta C₂b}}

  \bigskip
  \exampleLib{Biblioteka Jagiellońska. Kraków.}
  

\bigskip \exampleRef{\textit{Estreicher XXIII. 413, XX XIV. 48. Wierzbowski 2067. Piekarski U. 18.}}

  \bigskip
  \exampleDig{\url{https://cyfrowe.mnk.pl/dlibra/publication/865/} page 95}
% Lipsk  https://kpbc.umk.pl/dlibra/publication/edition/146354/content
% Paris   https://bibliotekacyfrowa.pl/en/dlibra/publication/53621/processus-iudiciarius-johannes-de-auerbach-1405-1469
%  Ungler 1516? https://wbc.poznan.pl/dlibra/publication/505385/edition/473732/content?ref=L3B1YmxpY2F0aW9uLzQ5NDE4NC9lZGl0aW9uLzQyNTk4OA
 
  
  % \medskip
\bigskip

\examplePL{[Pismo 5]}

\medskip

\exampleEN{[Font 5]}



\bigskip

\fontID{U1-05}{46}

\fontstat{95}

% \exdisplay \bg \gla
 \exdisplay \bg \gla
% 1
{\PTglyph{5}{t46_l01g01.png}}
% 2
{\PTglyph{5}{t46_l01g02.png}}
% 3
{\PTglyph{5}{t46_l01g03.png}}
% 4
{\PTglyph{5}{t46_l01g04.png}}
% 5
{\PTglyph{5}{t46_l01g05.png}}
% 6
{\PTglyph{5}{t46_l01g06.png}}
% 7
{\PTglyph{5}{t46_l01g07.png}}
% 8
{\PTglyph{5}{t46_l01g08.png}}
% 9
{\PTglyph{5}{t46_l01g09.png}}
% 10
{\PTglyph{5}{t46_l01g10.png}}
% 11
{\PTglyph{5}{t46_l01g11.png}}
% 12
{\PTglyph{5}{t46_l01g12.png}}
% 13
{\PTglyph{5}{t46_l01g13.png}}
% 14
{\PTglyph{5}{t46_l01g14.png}}
% 15
{\PTglyph{5}{t46_l01g15.png}}
% 16
{\PTglyph{5}{t46_l01g16.png}}
% 17
{\PTglyph{5}{t46_l01g17.png}}
% 18
{\PTglyph{5}{t46_l01g18.png}}
% 19
{\PTglyph{5}{t46_l01g19.png}}
% 20
{\PTglyph{5}{t46_l01g20.png}}
% 21
{\PTglyph{5}{t46_l01g21.png}}
% 22
{\PTglyph{5}{t46_l01g22.png}}
% 23
{\PTglyph{5}{t46_l01g23.png}}
% 24
{\PTglyph{5}{t46_l01g24.png}}
% 25
{\PTglyph{5}{t46_l02g01.png}}
% 26
{\PTglyph{5}{t46_l02g02.png}}
% 27
{\PTglyph{5}{t46_l02g03.png}}
% 28
{\PTglyph{5}{t46_l02g04.png}}
% 29
{\PTglyph{5}{t46_l02g05.png}}
% 30
{\PTglyph{5}{t46_l02g06.png}}
% 31
{\PTglyph{5}{t46_l02g07.png}}
% 32
{\PTglyph{5}{t46_l02g08.png}}
% 33
{\PTglyph{5}{t46_l02g09.png}}
% 34
{\PTglyph{5}{t46_l02g10.png}}
% 35
{\PTglyph{5}{t46_l02g11.png}}
% 36
{\PTglyph{5}{t46_l02g12.png}}
% 37
{\PTglyph{5}{t46_l02g13.png}}
% 38
{\PTglyph{5}{t46_l02g14.png}}
% 39
{\PTglyph{5}{t46_l02g15.png}}
% 40
{\PTglyph{5}{t46_l02g16.png}}
% 41
{\PTglyph{5}{t46_l02g17.png}}
% 42
{\PTglyph{5}{t46_l02g18.png}}
% 43
{\PTglyph{5}{t46_l02g19.png}}
% 44
{\PTglyph{5}{t46_l02g20.png}}
% 45
{\PTglyph{5}{t46_l02g21.png}}
% 46
{\PTglyph{5}{t46_l02g22.png}}
% 47
{\PTglyph{5}{t46_l02g23.png}}
% 48
{\PTglyph{5}{t46_l02g24.png}}
% 49
{\PTglyph{5}{t46_l02g25.png}}
% 50
{\PTglyph{5}{t46_l02g26.png}}
% 51
{\PTglyph{5}{t46_l02g27.png}}
% 52
{\PTglyph{5}{t46_l02g28.png}}
% 53
{\PTglyph{5}{t46_l02g29.png}}
% 54
{\PTglyph{5}{t46_l02g30.png}}
% 55
{\PTglyph{5}{t46_l02g31.png}}
% 56
{\PTglyph{5}{t46_l02g32.png}}
% 57
{\PTglyph{5}{t46_l02g33.png}}
% 58
{\PTglyph{5}{t46_l02g34.png}}
% 59
{\PTglyph{5}{t46_l02g35.png}}
% 60
{\PTglyph{5}{t46_l02g36.png}}
% 61
{\PTglyph{5}{t46_l02g37.png}}
% 62
{\PTglyph{5}{t46_l02g38.png}}
% 63
{\PTglyph{5}{t46_l02g39.png}}
% 64
{\PTglyph{5}{t46_l03g01.png}}
% 65
{\PTglyph{5}{t46_l03g02.png}}
% 66
{\PTglyph{5}{t46_l03g03.png}}
% 67
{\PTglyph{5}{t46_l03g04.png}}
% 68
{\PTglyph{5}{t46_l03g05.png}}
% 69
{\PTglyph{5}{t46_l03g06.png}}
% 70
{\PTglyph{5}{t46_l03g07.png}}
% 71
{\PTglyph{5}{t46_l03g08.png}}
% 72
{\PTglyph{5}{t46_l03g09.png}}
% 73
{\PTglyph{5}{t46_l03g10.png}}
% 74
{\PTglyph{5}{t46_l03g11.png}}
% 75
{\PTglyph{5}{t46_l03g12.png}}
% 76
{\PTglyph{5}{t46_l03g13.png}}
% 77
{\PTglyph{5}{t46_l03g14.png}}
% 78
{\PTglyph{5}{t46_l03g15.png}}
% 79
{\PTglyph{5}{t46_l03g16.png}}
% 80
{\PTglyph{5}{t46_l03g17.png}}
% 81
{\PTglyph{5}{t46_l03g18.png}}
% 82
{\PTglyph{5}{t46_l03g19.png}}
% 83
{\PTglyph{5}{t46_l03g20.png}}
% 84
{\PTglyph{5}{t46_l03g21.png}}
% 85
{\PTglyph{5}{t46_l03g22.png}}
% 86
{\PTglyph{5}{t46_l03g23.png}}
% 87
{\PTglyph{5}{t46_l03g24.png}}
% 88
{\PTglyph{5}{t46_l03g25.png}}
% 89
{\PTglyph{5}{t46_l03g26.png}}
% 90
{\PTglyph{5}{t46_l03g27.png}}
% 91
{\PTglyph{5}{t46_l03g28.png}}
% 92
{\PTglyph{5}{t46_l03g29.png}}
% 93
{\PTglyph{5}{t46_l03g30.png}}
% 94
{\PTglyph{5}{t46_l03g31.png}}
//
%%% Local Variables:
%%% mode: latex
%%% TeX-engine: luatex
%%% TeX-master: shared
%%% End:

%//
%\glpismo%
 \glpismo
% 1
{\PTglyphid{U1-05_0101}}
% 2
{\PTglyphid{U1-05_0102}}
% 3
{\PTglyphid{U1-05_0103}}
% 4
{\PTglyphid{U1-05_0104}}
% 5
{\PTglyphid{U1-05_0105}}
% 6
{\PTglyphid{U1-05_0106}}
% 7
{\PTglyphid{U1-05_0107}}
% 8
{\PTglyphid{U1-05_0108}}
% 9
{\PTglyphid{U1-05_0109}}
% 10
{\PTglyphid{U1-05_0110}}
% 11
{\PTglyphid{U1-05_0111}}
% 12
{\PTglyphid{U1-05_0112}}
% 13
{\PTglyphid{U1-05_0113}}
% 14
{\PTglyphid{U1-05_0114}}
% 15
{\PTglyphid{U1-05_0115}}
% 16
{\PTglyphid{U1-05_0116}}
% 17
{\PTglyphid{U1-05_0117}}
% 18
{\PTglyphid{U1-05_0118}}
% 19
{\PTglyphid{U1-05_0119}}
% 20
{\PTglyphid{U1-05_0120}}
% 21
{\PTglyphid{U1-05_0121}}
% 22
{\PTglyphid{U1-05_0122}}
% 23
{\PTglyphid{U1-05_0123}}
% 24
{\PTglyphid{U1-05_0124}}
% 25
{\PTglyphid{U1-05_0201}}
% 26
{\PTglyphid{U1-05_0202}}
% 27
{\PTglyphid{U1-05_0203}}
% 28
{\PTglyphid{U1-05_0204}}
% 29
{\PTglyphid{U1-05_0205}}
% 30
{\PTglyphid{U1-05_0206}}
% 31
{\PTglyphid{U1-05_0207}}
% 32
{\PTglyphid{U1-05_0208}}
% 33
{\PTglyphid{U1-05_0209}}
% 34
{\PTglyphid{U1-05_0210}}
% 35
{\PTglyphid{U1-05_0211}}
% 36
{\PTglyphid{U1-05_0212}}
% 37
{\PTglyphid{U1-05_0213}}
% 38
{\PTglyphid{U1-05_0214}}
% 39
{\PTglyphid{U1-05_0215}}
% 40
{\PTglyphid{U1-05_0216}}
% 41
{\PTglyphid{U1-05_0217}}
% 42
{\PTglyphid{U1-05_0218}}
% 43
{\PTglyphid{U1-05_0219}}
% 44
{\PTglyphid{U1-05_0220}}
% 45
{\PTglyphid{U1-05_0221}}
% 46
{\PTglyphid{U1-05_0222}}
% 47
{\PTglyphid{U1-05_0223}}
% 48
{\PTglyphid{U1-05_0224}}
% 49
{\PTglyphid{U1-05_0225}}
% 50
{\PTglyphid{U1-05_0226}}
% 51
{\PTglyphid{U1-05_0227}}
% 52
{\PTglyphid{U1-05_0228}}
% 53
{\PTglyphid{U1-05_0229}}
% 54
{\PTglyphid{U1-05_0230}}
% 55
{\PTglyphid{U1-05_0231}}
% 56
{\PTglyphid{U1-05_0232}}
% 57
{\PTglyphid{U1-05_0233}}
% 58
{\PTglyphid{U1-05_0234}}
% 59
{\PTglyphid{U1-05_0235}}
% 60
{\PTglyphid{U1-05_0236}}
% 61
{\PTglyphid{U1-05_0237}}
% 62
{\PTglyphid{U1-05_0238}}
% 63
{\PTglyphid{U1-05_0239}}
% 64
{\PTglyphid{U1-05_0301}}
% 65
{\PTglyphid{U1-05_0302}}
% 66
{\PTglyphid{U1-05_0303}}
% 67
{\PTglyphid{U1-05_0304}}
% 68
{\PTglyphid{U1-05_0305}}
% 69
{\PTglyphid{U1-05_0306}}
% 70
{\PTglyphid{U1-05_0307}}
% 71
{\PTglyphid{U1-05_0308}}
% 72
{\PTglyphid{U1-05_0309}}
% 73
{\PTglyphid{U1-05_0310}}
% 74
{\PTglyphid{U1-05_0311}}
% 75
{\PTglyphid{U1-05_0312}}
% 76
{\PTglyphid{U1-05_0313}}
% 77
{\PTglyphid{U1-05_0314}}
% 78
{\PTglyphid{U1-05_0315}}
% 79
{\PTglyphid{U1-05_0316}}
% 80
{\PTglyphid{U1-05_0317}}
% 81
{\PTglyphid{U1-05_0318}}
% 82
{\PTglyphid{U1-05_0319}}
% 83
{\PTglyphid{U1-05_0320}}
% 84
{\PTglyphid{U1-05_0321}}
% 85
{\PTglyphid{U1-05_0322}}
% 86
{\PTglyphid{U1-05_0323}}
% 87
{\PTglyphid{U1-05_0324}}
% 88
{\PTglyphid{U1-05_0325}}
% 89
{\PTglyphid{U1-05_0326}}
% 90
{\PTglyphid{U1-05_0327}}
% 91
{\PTglyphid{U1-05_0328}}
% 92
{\PTglyphid{U1-05_0329}}
% 93
{\PTglyphid{U1-05_0330}}
% 94
{\PTglyphid{U1-05_0331}}
//
\endgl \xe
%%% Local Variables:
%%% mode: latex
%%% TeX-engine: luatex
%%% TeX-master: shared
%%% End:

% //
%\endgl \xe


 \newpage
 
%%%%%%%%%%%%%%%%%%%%%%%%%%%%%%%%%%%%%%%%%%%%%%%%%%%%%%%%%%%%%%%%%%%%%%%%%%%%%%%
% from meta.csv
 % Tab. 47,Ungler1-06_PT03_117.djvu,Ungler1,06,03,117
%%%%%%%%%%%%%%%%%%%%%%%%%%%%%%%%%%%%%%%%%%%%%%%%%%%%%%%%%%%%%%%%%%%%%%%%%%%%%%%

 
 % from dsed4test:
%  Note "6. Pismo tekstowe gotyckie (stany a i b). Krój M⁴⁸. Stopień 20 ww. == 85/86 mm. — Tabl. 117."
% Note1 "Character set table prepared by Maria Błońska"


\pismoPL{Florian Ungler  pierwsza drukarnia 6. Pismo tekstowe gotyckie (stany a i b). Krój M⁴⁸. Stopień 20 ww. == 85/86 mm. — Tabl. 117.}
  
 \pismoEN{Florian Ungler  first house 5. Gothic text font (status a and b). Typeface M⁴⁸. Type size 20 ww. == 85/86 mm. — Plate 117.}

\plate{117}{III}{1959}

The plate prepared by Henryk Bułhak.\\
The font table prepared by Henryk Bułhak and  Maria Błońska.

\bigskip

\exampleBib{III:61}

\bigskip \exampleDesc{STANISLAUS ZABOROWSKI: Orthographia polonica. Kraków [Florian Ungler, 1514—1515]. 8⁰.}


\medskip
\examplePage{\textit{Karty B₂a, C₁b}}

  \bigskip
  \exampleLib{Muzeum Narodowe. Zbiory Czapskich. Kraków.}
  

\bigskip \exampleRef{\textit{Estreicher XXXIV. 45. Piekarski U. 58.}}

\bigskip

\examplePL{Pismo 6 (alfabet polski): prawa kolumna wiersze 21—24. —
  Pismo 3: prawa kolumna wiersze 1—20, 25—30. — Rubryka \beta{} z pismem
  3. — Rubryki \epsilon{}, \zeta{}, \eta{}: z pismem 6.}

\medskip

\exampleEN{Font 6 (Polish alphabet): right column lines 21—24. —
  Font 3: right column lines 1—20, 25—30. — Rubric \beta{} with font
  3. — Rubrics \epsilon{}, \zeta{}, \eta{}: with font 6.}


\bigskip

\fontID{U1-06}{47}

\fontstat{128}

% \exdisplay \bg \gla
 \exdisplay \bg \gla
% 1
{\PTglyph{5}{t47_l01g01.png}}
% 2
{\PTglyph{5}{t47_l01g02.png}}
% 3
{\PTglyph{5}{t47_l01g03.png}}
% 4
{\PTglyph{5}{t47_l01g04.png}}
% 5
{\PTglyph{5}{t47_l01g05.png}}
% 6
{\PTglyph{5}{t47_l01g06.png}}
% 7
{\PTglyph{5}{t47_l01g07.png}}
% 8
{\PTglyph{5}{t47_l01g08.png}}
% 9
{\PTglyph{5}{t47_l01g09.png}}
% 10
{\PTglyph{5}{t47_l01g10.png}}
% 11
{\PTglyph{5}{t47_l01g11.png}}
% 12
{\PTglyph{5}{t47_l01g12.png}}
% 13
{\PTglyph{5}{t47_l01g13.png}}
% 14
{\PTglyph{5}{t47_l01g14.png}}
% 15
{\PTglyph{5}{t47_l01g15.png}}
% 16
{\PTglyph{5}{t47_l01g16.png}}
% 17
{\PTglyph{5}{t47_l01g17.png}}
% 18
{\PTglyph{5}{t47_l01g18.png}}
% 19
{\PTglyph{5}{t47_l01g19.png}}
% 20
{\PTglyph{5}{t47_l01g20.png}}
% 21
{\PTglyph{5}{t47_l01g21.png}}
% 22
{\PTglyph{5}{t47_l01g22.png}}
% 23
{\PTglyph{5}{t47_l01g23.png}}
% 24
{\PTglyph{5}{t47_l01g24.png}}
% 25
{\PTglyph{5}{t47_l01g25.png}}
% 26
{\PTglyph{5}{t47_l01g26.png}}
% 27
{\PTglyph{5}{t47_l01g27.png}}
% 28
{\PTglyph{5}{t47_l02g01.png}}
% 29
{\PTglyph{5}{t47_l02g02.png}}
% 30
{\PTglyph{5}{t47_l02g03.png}}
% 31
{\PTglyph{5}{t47_l02g04.png}}
% 32
{\PTglyph{5}{t47_l02g05.png}}
% 33
{\PTglyph{5}{t47_l02g06.png}}
% 34
{\PTglyph{5}{t47_l02g07.png}}
% 35
{\PTglyph{5}{t47_l02g08.png}}
% 36
{\PTglyph{5}{t47_l02g09.png}}
% 37
{\PTglyph{5}{t47_l02g10.png}}
% 38
{\PTglyph{5}{t47_l02g11.png}}
% 39
{\PTglyph{5}{t47_l02g12.png}}
% 40
{\PTglyph{5}{t47_l02g13.png}}
% 41
{\PTglyph{5}{t47_l02g14.png}}
% 42
{\PTglyph{5}{t47_l02g15.png}}
% 43
{\PTglyph{5}{t47_l02g16.png}}
% 44
{\PTglyph{5}{t47_l02g17.png}}
% 45
{\PTglyph{5}{t47_l02g18.png}}
% 46
{\PTglyph{5}{t47_l02g19.png}}
% 47
{\PTglyph{5}{t47_l02g20.png}}
% 48
{\PTglyph{5}{t47_l02g21.png}}
% 49
{\PTglyph{5}{t47_l02g22.png}}
% 50
{\PTglyph{5}{t47_l02g23.png}}
% 51
{\PTglyph{5}{t47_l02g24.png}}
% 52
{\PTglyph{5}{t47_l02g25.png}}
% 53
{\PTglyph{5}{t47_l02g26.png}}
% 54
{\PTglyph{5}{t47_l02g27.png}}
% 55
{\PTglyph{5}{t47_l02g28.png}}
% 56
{\PTglyph{5}{t47_l02g29.png}}
% 57
{\PTglyph{5}{t47_l02g30.png}}
% 58
{\PTglyph{5}{t47_l02g31.png}}
% 59
{\PTglyph{5}{t47_l02g32.png}}
% 60
{\PTglyph{5}{t47_l02g33.png}}
% 61
{\PTglyph{5}{t47_l02g34.png}}
% 62
{\PTglyph{5}{t47_l02g35.png}}
% 63
{\PTglyph{5}{t47_l02g36.png}}
% 64
{\PTglyph{5}{t47_l02g37.png}}
% 65
{\PTglyph{5}{t47_l02g38.png}}
% 66
{\PTglyph{5}{t47_l03g01.png}}
% 67
{\PTglyph{5}{t47_l03g02.png}}
% 68
{\PTglyph{5}{t47_l03g03.png}}
% 69
{\PTglyph{5}{t47_l03g04.png}}
% 70
{\PTglyph{5}{t47_l03g05.png}}
% 71
{\PTglyph{5}{t47_l03g06.png}}
% 72
{\PTglyph{5}{t47_l03g07.png}}
% 73
{\PTglyph{5}{t47_l03g08.png}}
% 74
{\PTglyph{5}{t47_l03g09.png}}
% 75
{\PTglyph{5}{t47_l03g10.png}}
% 76
{\PTglyph{5}{t47_l03g11.png}}
% 77
{\PTglyph{5}{t47_l03g12.png}}
% 78
{\PTglyph{5}{t47_l03g13.png}}
% 79
{\PTglyph{5}{t47_l03g14.png}}
% 80
{\PTglyph{5}{t47_l03g15.png}}
% 81
{\PTglyph{5}{t47_l03g16.png}}
% 82
{\PTglyph{5}{t47_l03g17.png}}
% 83
{\PTglyph{5}{t47_l03g18.png}}
% 84
{\PTglyph{5}{t47_l03g19.png}}
% 85
{\PTglyph{5}{t47_l03g20.png}}
% 86
{\PTglyph{5}{t47_l03g21.png}}
% 87
{\PTglyph{5}{t47_l03g22.png}}
% 88
{\PTglyph{5}{t47_l03g23.png}}
% 89
{\PTglyph{5}{t47_l03g24.png}}
% 90
{\PTglyph{5}{t47_l03g25.png}}
% 91
{\PTglyph{5}{t47_l03g26.png}}
% 92
{\PTglyph{5}{t47_l03g27.png}}
% 93
{\PTglyph{5}{t47_l03g28.png}}
% 94
{\PTglyph{5}{t47_l03g29.png}}
% 95
{\PTglyph{5}{t47_l03g30.png}}
% 96
{\PTglyph{5}{t47_l03g31.png}}
% 97
{\PTglyph{5}{t47_l03g32.png}}
% 98
{\PTglyph{5}{t47_l03g33.png}}
% 99
{\PTglyph{5}{t47_l03g34.png}}
% 100
{\PTglyph{5}{t47_l03g35.png}}
% 101
{\PTglyph{5}{t47_l03g36.png}}
% 102
{\PTglyph{5}{t47_l03g37.png}}
% 103
{\PTglyph{5}{t47_l04g01.png}}
% 104
{\PTglyph{5}{t47_l04g02.png}}
% 105
{\PTglyph{5}{t47_l04g03.png}}
% 106
{\PTglyph{5}{t47_l04g04.png}}
% 107
{\PTglyph{5}{t47_l04g05.png}}
% 108
{\PTglyph{5}{t47_l04g06.png}}
% 109
{\PTglyph{5}{t47_l04g07.png}}
% 110
{\PTglyph{5}{t47_l04g08.png}}
% 111
{\PTglyph{5}{t47_l04g09.png}}
% 112
{\PTglyph{5}{t47_l04g10.png}}
% 113
{\PTglyph{5}{t47_l04g11.png}}
% 114
{\PTglyph{5}{t47_l04g12.png}}
% 115
{\PTglyph{5}{t47_l04g13.png}}
% 116
{\PTglyph{5}{t47_l04g14.png}}
% 117
{\PTglyph{5}{t47_l04g15.png}}
% 118
{\PTglyph{5}{t47_l04g16.png}}
% 119
{\PTglyph{5}{t47_l04g17.png}}
% 120
{\PTglyph{5}{t47_l04g18.png}}
% 121
{\PTglyph{5}{t47_l04g19.png}}
% 122
{\PTglyph{5}{t47_l04g20.png}}
% 123
{\PTglyph{5}{t47_l04g21.png}}
% 124
{\PTglyph{5}{t47_l04g22.png}}
% 125
{\PTglyph{5}{t47_l04g23.png}}
% 126
{\PTglyph{5}{t47_l04g24.png}}
% 127
{\PTglyph{5}{t47_l04g25.png}}
% 128
{\PTglyph{5}{t47_l04g26.png}}
//
%%% Local Variables:
%%% mode: latex
%%% TeX-engine: luatex
%%% TeX-master: shared
%%% End:

%//
%\glpismo%
 \glpismo
% 1
{\PTglyphid{U1-06_0101}}
% 2
{\PTglyphid{U1-06_0102}}
% 3
{\PTglyphid{U1-06_0103}}
% 4
{\PTglyphid{U1-06_0104}}
% 5
{\PTglyphid{U1-06_0105}}
% 6
{\PTglyphid{U1-06_0106}}
% 7
{\PTglyphid{U1-06_0107}}
% 8
{\PTglyphid{U1-06_0108}}
% 9
{\PTglyphid{U1-06_0109}}
% 10
{\PTglyphid{U1-06_0110}}
% 11
{\PTglyphid{U1-06_0111}}
% 12
{\PTglyphid{U1-06_0112}}
% 13
{\PTglyphid{U1-06_0113}}
% 14
{\PTglyphid{U1-06_0114}}
% 15
{\PTglyphid{U1-06_0115}}
% 16
{\PTglyphid{U1-06_0116}}
% 17
{\PTglyphid{U1-06_0117}}
% 18
{\PTglyphid{U1-06_0118}}
% 19
{\PTglyphid{U1-06_0119}}
% 20
{\PTglyphid{U1-06_0120}}
% 21
{\PTglyphid{U1-06_0121}}
% 22
{\PTglyphid{U1-06_0122}}
% 23
{\PTglyphid{U1-06_0123}}
% 24
{\PTglyphid{U1-06_0124}}
% 25
{\PTglyphid{U1-06_0125}}
% 26
{\PTglyphid{U1-06_0126}}
% 27
{\PTglyphid{U1-06_0127}}
% 28
{\PTglyphid{U1-06_0201}}
% 29
{\PTglyphid{U1-06_0202}}
% 30
{\PTglyphid{U1-06_0203}}
% 31
{\PTglyphid{U1-06_0204}}
% 32
{\PTglyphid{U1-06_0205}}
% 33
{\PTglyphid{U1-06_0206}}
% 34
{\PTglyphid{U1-06_0207}}
% 35
{\PTglyphid{U1-06_0208}}
% 36
{\PTglyphid{U1-06_0209}}
% 37
{\PTglyphid{U1-06_0210}}
% 38
{\PTglyphid{U1-06_0211}}
% 39
{\PTglyphid{U1-06_0212}}
% 40
{\PTglyphid{U1-06_0213}}
% 41
{\PTglyphid{U1-06_0214}}
% 42
{\PTglyphid{U1-06_0215}}
% 43
{\PTglyphid{U1-06_0216}}
% 44
{\PTglyphid{U1-06_0217}}
% 45
{\PTglyphid{U1-06_0218}}
% 46
{\PTglyphid{U1-06_0219}}
% 47
{\PTglyphid{U1-06_0220}}
% 48
{\PTglyphid{U1-06_0221}}
% 49
{\PTglyphid{U1-06_0222}}
% 50
{\PTglyphid{U1-06_0223}}
% 51
{\PTglyphid{U1-06_0224}}
% 52
{\PTglyphid{U1-06_0225}}
% 53
{\PTglyphid{U1-06_0226}}
% 54
{\PTglyphid{U1-06_0227}}
% 55
{\PTglyphid{U1-06_0228}}
% 56
{\PTglyphid{U1-06_0229}}
% 57
{\PTglyphid{U1-06_0230}}
% 58
{\PTglyphid{U1-06_0231}}
% 59
{\PTglyphid{U1-06_0232}}
% 60
{\PTglyphid{U1-06_0233}}
% 61
{\PTglyphid{U1-06_0234}}
% 62
{\PTglyphid{U1-06_0235}}
% 63
{\PTglyphid{U1-06_0236}}
% 64
{\PTglyphid{U1-06_0237}}
% 65
{\PTglyphid{U1-06_0238}}
% 66
{\PTglyphid{U1-06_0301}}
% 67
{\PTglyphid{U1-06_0302}}
% 68
{\PTglyphid{U1-06_0303}}
% 69
{\PTglyphid{U1-06_0304}}
% 70
{\PTglyphid{U1-06_0305}}
% 71
{\PTglyphid{U1-06_0306}}
% 72
{\PTglyphid{U1-06_0307}}
% 73
{\PTglyphid{U1-06_0308}}
% 74
{\PTglyphid{U1-06_0309}}
% 75
{\PTglyphid{U1-06_0310}}
% 76
{\PTglyphid{U1-06_0311}}
% 77
{\PTglyphid{U1-06_0312}}
% 78
{\PTglyphid{U1-06_0313}}
% 79
{\PTglyphid{U1-06_0314}}
% 80
{\PTglyphid{U1-06_0315}}
% 81
{\PTglyphid{U1-06_0316}}
% 82
{\PTglyphid{U1-06_0317}}
% 83
{\PTglyphid{U1-06_0318}}
% 84
{\PTglyphid{U1-06_0319}}
% 85
{\PTglyphid{U1-06_0320}}
% 86
{\PTglyphid{U1-06_0321}}
% 87
{\PTglyphid{U1-06_0322}}
% 88
{\PTglyphid{U1-06_0323}}
% 89
{\PTglyphid{U1-06_0324}}
% 90
{\PTglyphid{U1-06_0325}}
% 91
{\PTglyphid{U1-06_0326}}
% 92
{\PTglyphid{U1-06_0327}}
% 93
{\PTglyphid{U1-06_0328}}
% 94
{\PTglyphid{U1-06_0329}}
% 95
{\PTglyphid{U1-06_0330}}
% 96
{\PTglyphid{U1-06_0331}}
% 97
{\PTglyphid{U1-06_0332}}
% 98
{\PTglyphid{U1-06_0333}}
% 99
{\PTglyphid{U1-06_0334}}
% 100
{\PTglyphid{U1-06_0335}}
% 101
{\PTglyphid{U1-06_0336}}
% 102
{\PTglyphid{U1-06_0337}}
% 103
{\PTglyphid{U1-06_0401}}
% 104
{\PTglyphid{U1-06_0402}}
% 105
{\PTglyphid{U1-06_0403}}
% 106
{\PTglyphid{U1-06_0404}}
% 107
{\PTglyphid{U1-06_0405}}
% 108
{\PTglyphid{U1-06_0406}}
% 109
{\PTglyphid{U1-06_0407}}
% 110
{\PTglyphid{U1-06_0408}}
% 111
{\PTglyphid{U1-06_0409}}
% 112
{\PTglyphid{U1-06_0410}}
% 113
{\PTglyphid{U1-06_0411}}
% 114
{\PTglyphid{U1-06_0412}}
% 115
{\PTglyphid{U1-06_0413}}
% 116
{\PTglyphid{U1-06_0414}}
% 117
{\PTglyphid{U1-06_0415}}
% 118
{\PTglyphid{U1-06_0416}}
% 119
{\PTglyphid{U1-06_0417}}
% 120
{\PTglyphid{U1-06_0418}}
% 121
{\PTglyphid{U1-06_0419}}
% 122
{\PTglyphid{U1-06_0420}}
% 123
{\PTglyphid{U1-06_0421}}
% 124
{\PTglyphid{U1-06_0422}}
% 125
{\PTglyphid{U1-06_0423}}
% 126
{\PTglyphid{U1-06_0424}}
% 127
{\PTglyphid{U1-06_0425}}
% 128
{\PTglyphid{U1-06_0426}}
//
\endgl \xe
%%% Local Variables:
%%% mode: latex
%%% TeX-engine: luatex
%%% TeX-master: shared
%%% End:

% //
%\endgl \xe

 \newpage
 
%%%%%%%%%%%%%%%%%%%%%%%%%%%%%%%%%%%%%%%%%%%%%%%%%%%%%%%%%%%%%%%%%%%%%%%%%%%%%%%
% from meta.csv
 % Tab. 48,Ungler1-07_PT03_118.djvu,Ungler1,07,03,118
%%%%%%%%%%%%%%%%%%%%%%%%%%%%%%%%%%%%%%%%%%%%%%%%%%%%%%%%%%%%%%%%%%%%%%%%%%%%%%%

 
 % from dsed4test:
% Note "7. Pismo nagłówkowe i tekstowe gotyckie. Krój M¹⁸ raz przekreślone. Stopień 20 ww. == 104/106 mm. — Tabl. 118."
% Note1 "Character set table prepared by Maria Błońska"


\pismoPL{Florian Ungler  pierwsza drukarnia 7. Pismo nagłówkowe i tekstowe gotyckie. Krój M¹⁸ raz przekreślone. Stopień 20 ww. == 104/106 mm. — Tabl. 118.}
  
 \pismoEN{Florian Ungler  first house 7. Gothic header and text font. Typeface M¹⁸. Type size 20 ww. == 104/106 mm. — Plate 118.}

\plate{118}{III}{1959}

The plate prepared by Henryk Bułhak.\\
The font table prepared by by Henryk Bułhak and  Maria Błońska.

\bigskip

\exampleBib{III:32}

\bigskip \exampleDesc{LAURENTIUS CORVINUS: Latinum ideoma. Kraków, Florian Ungler, 1513. 4⁰}

\medskip
\examplePage{\textit{Karta A₇a}}

  \bigskip
  \exampleLib{Biblioteka Narodowa. Warszawa.}
  

\bigskip \exampleRef{\textit{Estreicher XIV. 424. Wierzbowski 885. Piekarski U. 30.}}

\bigskip

\examplePL{[Pismo 7]   Rubryka \gamma{}. — Cyfry 4.}

\medskip

\exampleEN{[Font 6] Rubric \gamma{}. — Digits 4.}


\bigskip

\fontID{U1-07}{48}

\fontstat{109}

% \exdisplay \bg \gla
 \exdisplay \bg \gla
% 1
{\PTglyph{5}{t48_l01g01.png}}
% 2
{\PTglyph{5}{t48_l01g02.png}}
% 3
{\PTglyph{5}{t48_l01g03.png}}
% 4
{\PTglyph{5}{t48_l01g04.png}}
% 5
{\PTglyph{5}{t48_l01g05.png}}
% 6
{\PTglyph{5}{t48_l01g06.png}}
% 7
{\PTglyph{5}{t48_l01g07.png}}
% 8
{\PTglyph{5}{t48_l01g08.png}}
% 9
{\PTglyph{5}{t48_l01g09.png}}
% 10
{\PTglyph{5}{t48_l01g10.png}}
% 11
{\PTglyph{5}{t48_l01g11.png}}
% 12
{\PTglyph{5}{t48_l01g12.png}}
% 13
{\PTglyph{5}{t48_l01g13.png}}
% 14
{\PTglyph{5}{t48_l01g14.png}}
% 15
{\PTglyph{5}{t48_l01g15.png}}
% 16
{\PTglyph{5}{t48_l01g16.png}}
% 17
{\PTglyph{5}{t48_l01g17.png}}
% 18
{\PTglyph{5}{t48_l01g18.png}}
% 19
{\PTglyph{5}{t48_l01g19.png}}
% 20
{\PTglyph{5}{t48_l01g20.png}}
% 21
{\PTglyph{5}{t48_l01g21.png}}
% 22
{\PTglyph{5}{t48_l02g01.png}}
% 23
{\PTglyph{5}{t48_l02g02.png}}
% 24
{\PTglyph{5}{t48_l02g03.png}}
% 25
{\PTglyph{5}{t48_l02g04.png}}
% 26
{\PTglyph{5}{t48_l02g05.png}}
% 27
{\PTglyph{5}{t48_l02g06.png}}
% 28
{\PTglyph{5}{t48_l02g07.png}}
% 29
{\PTglyph{5}{t48_l02g08.png}}
% 30
{\PTglyph{5}{t48_l02g09.png}}
% 31
{\PTglyph{5}{t48_l02g10.png}}
% 32
{\PTglyph{5}{t48_l02g11.png}}
% 33
{\PTglyph{5}{t48_l02g12.png}}
% 34
{\PTglyph{5}{t48_l02g13.png}}
% 35
{\PTglyph{5}{t48_l02g14.png}}
% 36
{\PTglyph{5}{t48_l02g15.png}}
% 37
{\PTglyph{5}{t48_l02g16.png}}
% 38
{\PTglyph{5}{t48_l02g17.png}}
% 39
{\PTglyph{5}{t48_l02g18.png}}
% 40
{\PTglyph{5}{t48_l02g19.png}}
% 41
{\PTglyph{5}{t48_l02g20.png}}
% 42
{\PTglyph{5}{t48_l02g21.png}}
% 43
{\PTglyph{5}{t48_l02g22.png}}
% 44
{\PTglyph{5}{t48_l02g23.png}}
% 45
{\PTglyph{5}{t48_l02g24.png}}
% 46
{\PTglyph{5}{t48_l02g25.png}}
% 47
{\PTglyph{5}{t48_l02g26.png}}
% 48
{\PTglyph{5}{t48_l02g27.png}}
% 49
{\PTglyph{5}{t48_l02g28.png}}
% 50
{\PTglyph{5}{t48_l02g29.png}}
% 51
{\PTglyph{5}{t48_l02g30.png}}
% 52
{\PTglyph{5}{t48_l02g31.png}}
% 53
{\PTglyph{5}{t48_l02g32.png}}
% 54
{\PTglyph{5}{t48_l03g01.png}}
% 55
{\PTglyph{5}{t48_l03g02.png}}
% 56
{\PTglyph{5}{t48_l03g03.png}}
% 57
{\PTglyph{5}{t48_l03g04.png}}
% 58
{\PTglyph{5}{t48_l03g05.png}}
% 59
{\PTglyph{5}{t48_l03g06.png}}
% 60
{\PTglyph{5}{t48_l03g07.png}}
% 61
{\PTglyph{5}{t48_l03g08.png}}
% 62
{\PTglyph{5}{t48_l03g09.png}}
% 63
{\PTglyph{5}{t48_l03g10.png}}
% 64
{\PTglyph{5}{t48_l03g11.png}}
% 65
{\PTglyph{5}{t48_l03g12.png}}
% 66
{\PTglyph{5}{t48_l03g13.png}}
% 67
{\PTglyph{5}{t48_l03g14.png}}
% 68
{\PTglyph{5}{t48_l03g15.png}}
% 69
{\PTglyph{5}{t48_l03g16.png}}
% 70
{\PTglyph{5}{t48_l03g17.png}}
% 71
{\PTglyph{5}{t48_l03g18.png}}
% 72
{\PTglyph{5}{t48_l03g19.png}}
% 73
{\PTglyph{5}{t48_l03g20.png}}
% 74
{\PTglyph{5}{t48_l03g21.png}}
% 75
{\PTglyph{5}{t48_l03g22.png}}
% 76
{\PTglyph{5}{t48_l03g23.png}}
% 77
{\PTglyph{5}{t48_l03g24.png}}
% 78
{\PTglyph{5}{t48_l03g25.png}}
% 79
{\PTglyph{5}{t48_l03g26.png}}
% 80
{\PTglyph{5}{t48_l03g27.png}}
% 81
{\PTglyph{5}{t48_l03g28.png}}
% 82
{\PTglyph{5}{t48_l04g01.png}}
% 83
{\PTglyph{5}{t48_l04g02.png}}
% 84
{\PTglyph{5}{t48_l04g03.png}}
% 85
{\PTglyph{5}{t48_l04g04.png}}
% 86
{\PTglyph{5}{t48_l04g05.png}}
% 87
{\PTglyph{5}{t48_l04g06.png}}
% 88
{\PTglyph{5}{t48_l04g07.png}}
% 89
{\PTglyph{5}{t48_l04g08.png}}
% 90
{\PTglyph{5}{t48_l04g09.png}}
% 91
{\PTglyph{5}{t48_l04g10.png}}
% 92
{\PTglyph{5}{t48_l04g11.png}}
% 93
{\PTglyph{5}{t48_l04g12.png}}
% 94
{\PTglyph{5}{t48_l04g13.png}}
% 95
{\PTglyph{5}{t48_l04g14.png}}
% 96
{\PTglyph{5}{t48_l04g15.png}}
% 97
{\PTglyph{5}{t48_l04g16.png}}
% 98
{\PTglyph{5}{t48_l04g17.png}}
% 99
{\PTglyph{5}{t48_l04g18.png}}
% 100
{\PTglyph{5}{t48_l04g19.png}}
% 101
{\PTglyph{5}{t48_l04g20.png}}
% 102
{\PTglyph{5}{t48_l04g21.png}}
% 103
{\PTglyph{5}{t48_l04g22.png}}
% 104
{\PTglyph{5}{t48_l04g23.png}}
% 105
{\PTglyph{5}{t48_l04g24.png}}
% 106
{\PTglyph{5}{t48_l04g25.png}}
% 107
{\PTglyph{5}{t48_l04g26.png}}
% 108
{\PTglyph{5}{t48_l04g27.png}}
% 109
{\PTglyph{5}{t48_l04g28.png}}
//
%%% Local Variables:
%%% mode: latex
%%% TeX-engine: luatex
%%% TeX-master: shared
%%% End:

%//
%\glpismo%
 \glpismo
% 1
{\PTglyphid{U1-07_0101}}
% 2
{\PTglyphid{U1-07_0102}}
% 3
{\PTglyphid{U1-07_0103}}
% 4
{\PTglyphid{U1-07_0104}}
% 5
{\PTglyphid{U1-07_0105}}
% 6
{\PTglyphid{U1-07_0106}}
% 7
{\PTglyphid{U1-07_0107}}
% 8
{\PTglyphid{U1-07_0108}}
% 9
{\PTglyphid{U1-07_0109}}
% 10
{\PTglyphid{U1-07_0110}}
% 11
{\PTglyphid{U1-07_0111}}
% 12
{\PTglyphid{U1-07_0112}}
% 13
{\PTglyphid{U1-07_0113}}
% 14
{\PTglyphid{U1-07_0114}}
% 15
{\PTglyphid{U1-07_0115}}
% 16
{\PTglyphid{U1-07_0116}}
% 17
{\PTglyphid{U1-07_0117}}
% 18
{\PTglyphid{U1-07_0118}}
% 19
{\PTglyphid{U1-07_0119}}
% 20
{\PTglyphid{U1-07_0120}}
% 21
{\PTglyphid{U1-07_0121}}
% 22
{\PTglyphid{U1-07_0201}}
% 23
{\PTglyphid{U1-07_0202}}
% 24
{\PTglyphid{U1-07_0203}}
% 25
{\PTglyphid{U1-07_0204}}
% 26
{\PTglyphid{U1-07_0205}}
% 27
{\PTglyphid{U1-07_0206}}
% 28
{\PTglyphid{U1-07_0207}}
% 29
{\PTglyphid{U1-07_0208}}
% 30
{\PTglyphid{U1-07_0209}}
% 31
{\PTglyphid{U1-07_0210}}
% 32
{\PTglyphid{U1-07_0211}}
% 33
{\PTglyphid{U1-07_0212}}
% 34
{\PTglyphid{U1-07_0213}}
% 35
{\PTglyphid{U1-07_0214}}
% 36
{\PTglyphid{U1-07_0215}}
% 37
{\PTglyphid{U1-07_0216}}
% 38
{\PTglyphid{U1-07_0217}}
% 39
{\PTglyphid{U1-07_0218}}
% 40
{\PTglyphid{U1-07_0219}}
% 41
{\PTglyphid{U1-07_0220}}
% 42
{\PTglyphid{U1-07_0221}}
% 43
{\PTglyphid{U1-07_0222}}
% 44
{\PTglyphid{U1-07_0223}}
% 45
{\PTglyphid{U1-07_0224}}
% 46
{\PTglyphid{U1-07_0225}}
% 47
{\PTglyphid{U1-07_0226}}
% 48
{\PTglyphid{U1-07_0227}}
% 49
{\PTglyphid{U1-07_0228}}
% 50
{\PTglyphid{U1-07_0229}}
% 51
{\PTglyphid{U1-07_0230}}
% 52
{\PTglyphid{U1-07_0231}}
% 53
{\PTglyphid{U1-07_0232}}
% 54
{\PTglyphid{U1-07_0301}}
% 55
{\PTglyphid{U1-07_0302}}
% 56
{\PTglyphid{U1-07_0303}}
% 57
{\PTglyphid{U1-07_0304}}
% 58
{\PTglyphid{U1-07_0305}}
% 59
{\PTglyphid{U1-07_0306}}
% 60
{\PTglyphid{U1-07_0307}}
% 61
{\PTglyphid{U1-07_0308}}
% 62
{\PTglyphid{U1-07_0309}}
% 63
{\PTglyphid{U1-07_0310}}
% 64
{\PTglyphid{U1-07_0311}}
% 65
{\PTglyphid{U1-07_0312}}
% 66
{\PTglyphid{U1-07_0313}}
% 67
{\PTglyphid{U1-07_0314}}
% 68
{\PTglyphid{U1-07_0315}}
% 69
{\PTglyphid{U1-07_0316}}
% 70
{\PTglyphid{U1-07_0317}}
% 71
{\PTglyphid{U1-07_0318}}
% 72
{\PTglyphid{U1-07_0319}}
% 73
{\PTglyphid{U1-07_0320}}
% 74
{\PTglyphid{U1-07_0321}}
% 75
{\PTglyphid{U1-07_0322}}
% 76
{\PTglyphid{U1-07_0323}}
% 77
{\PTglyphid{U1-07_0324}}
% 78
{\PTglyphid{U1-07_0325}}
% 79
{\PTglyphid{U1-07_0326}}
% 80
{\PTglyphid{U1-07_0327}}
% 81
{\PTglyphid{U1-07_0328}}
% 82
{\PTglyphid{U1-07_0401}}
% 83
{\PTglyphid{U1-07_0402}}
% 84
{\PTglyphid{U1-07_0403}}
% 85
{\PTglyphid{U1-07_0404}}
% 86
{\PTglyphid{U1-07_0405}}
% 87
{\PTglyphid{U1-07_0406}}
% 88
{\PTglyphid{U1-07_0407}}
% 89
{\PTglyphid{U1-07_0408}}
% 90
{\PTglyphid{U1-07_0409}}
% 91
{\PTglyphid{U1-07_0410}}
% 92
{\PTglyphid{U1-07_0411}}
% 93
{\PTglyphid{U1-07_0412}}
% 94
{\PTglyphid{U1-07_0413}}
% 95
{\PTglyphid{U1-07_0414}}
% 96
{\PTglyphid{U1-07_0415}}
% 97
{\PTglyphid{U1-07_0416}}
% 98
{\PTglyphid{U1-07_0417}}
% 99
{\PTglyphid{U1-07_0418}}
% 100
{\PTglyphid{U1-07_0419}}
% 101
{\PTglyphid{U1-07_0420}}
% 102
{\PTglyphid{U1-07_0421}}
% 103
{\PTglyphid{U1-07_0422}}
% 104
{\PTglyphid{U1-07_0423}}
% 105
{\PTglyphid{U1-07_0424}}
% 106
{\PTglyphid{U1-07_0425}}
% 107
{\PTglyphid{U1-07_0426}}
% 108
{\PTglyphid{U1-07_0427}}
% 109
{\PTglyphid{U1-07_0428}}
//
\endgl \xe
%%% Local Variables:
%%% mode: latex
%%% TeX-engine: luatex
%%% TeX-master: shared
%%% End:

% //
%\endgl \xe


 \newpage
 
%%%%%%%%%%%%%%%%%%%%%%%%%%%%%%%%%%%%%%%%%%%%%%%%%%%%%%%%%%%%%%%%%%%%%%%%%%%%%%%
% from meta.csv
 % Tab. 49
% 49,Ungler1-08_PT03_120.djvu,Ungler1,08,03,120
%%%%%%%%%%%%%%%%%%%%%%%%%%%%%%%%%%%%%%%%%%%%%%%%%%%%%%%%%%%%%%%%%%%%%%%%%%%%%%%

 
 % from dsed4test:
% Ungler1-08_PT03_120_4dsed.txt:Note "8. Pismo nagłówkowe gotyckie. Krój M⁶⁹. Stopień 10 ww. == ok. 80mm. — Tabl. 120."
% Ungler1-08_PT03_120_4dsed.txt:Note1 "Character set table prepared by Maria Błońska"

\pismoPL{Florian Ungler pierwsza drukarnia 8. Pismo nagłówkowe gotyckie. Krój M⁶⁹. Stopień 10 ww. == ok. 80 mm. — Tabl. 120.}
  
 \pismoEN{Florian Ungler  first house 8. Gothic header font. Typeface M⁶⁹. Type size 20 ww. == about 80 mm. — Plate 120.}

\plate{120}{III}{1959}

The plate prepared by Henryk Bułhak.\\
The font table prepared by by Henryk Bułhak and  Maria Błońska.

\bigskip

% \exampleBib{III:32}

% \bigskip \exampleDesc{LAURENTIUS CORVINUS: Latinum ideoma. Kraków, Florian Ungler, 1513. 4⁰}

% \medskip
% \examplePage{\textit{Karta A₇a}}

%   \bigskip
%   \exampleLib{Biblioteka Narodowa. Warszawa.}
  

% \bigskip \exampleRef{\textit{Estreicher XIV. 424. Wierzbowski 885. Piekarski U. 30.}}

% \bigskip

% \examplePL{[Pismo 7]   Rubryka \gamma{}. — Cyfry 4.}

% \medskip

% \exampleEN{[Font 6] Rubric \gamma{}. — Digits 4.}

% Biblioteka Jagielloóska. Kraków (22, 48).
% Biblioteka Narodowa. Warszawa (37, 52). H. B. i M. B.
% Inicjaly serii 22 (lombardy) reprodukowano z pozygi: 22(A4, B, C, D EG,H, L LL, M;,N, O,P,Q, R,S, V, X), 48 (T).
% Znaki zodiaku 1 reprodukowano z pozycji 37 karta B,a (Pismo 7. — Cyfry 1).
% Znaki zodiaku 2 reprodukowano z pozycji 52 karta C,b (Pismo 3. — Rubryka 8. — Cyfry 5).


\bigskip

\fontID{U1-08}{49}

\fontstat{40}

% \exdisplay \bg \gla
 \exdisplay \bg \gla
% 1
{\PTglyph{5}{t49_l01g01.png}}
% 2
{\PTglyph{5}{t49_l01g02.png}}
% 3
{\PTglyph{5}{t49_l01g03.png}}
% 4
{\PTglyph{5}{t49_l01g04.png}}
% 5
{\PTglyph{5}{t49_l01g05.png}}
% 6
{\PTglyph{5}{t49_l01g06.png}}
% 7
{\PTglyph{5}{t49_l01g07.png}}
% 8
{\PTglyph{5}{t49_l01g08.png}}
% 9
{\PTglyph{5}{t49_l01g09.png}}
% 10
{\PTglyph{5}{t49_l01g10.png}}
% 11
{\PTglyph{5}{t49_l01g11.png}}
% 12
{\PTglyph{5}{t49_l01g12.png}}
% 13
{\PTglyph{5}{t49_l01g13.png}}
% 14
{\PTglyph{5}{t49_l01g14.png}}
% 15
{\PTglyph{5}{t49_l02g01.png}}
% 16
{\PTglyph{5}{t49_l02g02.png}}
% 17
{\PTglyph{5}{t49_l02g03.png}}
% 18
{\PTglyph{5}{t49_l02g04.png}}
% 19
{\PTglyph{5}{t49_l02g05.png}}
% 20
{\PTglyph{5}{t49_l02g06.png}}
% 21
{\PTglyph{5}{t49_l02g07.png}}
% 22
{\PTglyph{5}{t49_l02g08.png}}
% 23
{\PTglyph{5}{t49_l02g09.png}}
% 24
{\PTglyph{5}{t49_l02g10.png}}
% 25
{\PTglyph{5}{t49_l02g11.png}}
% 26
{\PTglyph{5}{t49_l02g12.png}}
% 27
{\PTglyph{5}{t49_l02g13.png}}
% 28
{\PTglyph{5}{t49_l02g14.png}}
% 29
{\PTglyph{5}{t49_l02g15.png}}
% 30
{\PTglyph{5}{t49_l02g16.png}}
% 31
{\PTglyph{5}{t49_l02g17.png}}
% 32
{\PTglyph{5}{t49_l02g18.png}}
% 33
{\PTglyph{5}{t49_l02g19.png}}
% 34
{\PTglyph{5}{t49_l02g20.png}}
% 35
{\PTglyph{5}{t49_l02g21.png}}
% 36
{\PTglyph{5}{t49_l02g22.png}}
% 37
{\PTglyph{5}{t49_l02g23.png}}
% 38
{\PTglyph{5}{t49_l02g24.png}}
% 39
{\PTglyph{5}{t49_l02g25.png}}
% 40
{\PTglyph{5}{t49_l02g26.png}}
//
%%% Local Variables:
%%% mode: latex
%%% TeX-engine: luatex
%%% TeX-master: shared
%%% End:

%//
%\glpismo%
 \glpismo
% 1
{\PTglyphid{U1-08_0101}}
% 2
{\PTglyphid{U1-08_0102}}
% 3
{\PTglyphid{U1-08_0103}}
% 4
{\PTglyphid{U1-08_0104}}
% 5
{\PTglyphid{U1-08_0105}}
% 6
{\PTglyphid{U1-08_0106}}
% 7
{\PTglyphid{U1-08_0107}}
% 8
{\PTglyphid{U1-08_0108}}
% 9
{\PTglyphid{U1-08_0109}}
% 10
{\PTglyphid{U1-08_0110}}
% 11
{\PTglyphid{U1-08_0111}}
% 12
{\PTglyphid{U1-08_0112}}
% 13
{\PTglyphid{U1-08_0113}}
% 14
{\PTglyphid{U1-08_0114}}
% 15
{\PTglyphid{U1-08_0201}}
% 16
{\PTglyphid{U1-08_0202}}
% 17
{\PTglyphid{U1-08_0203}}
% 18
{\PTglyphid{U1-08_0204}}
% 19
{\PTglyphid{U1-08_0205}}
% 20
{\PTglyphid{U1-08_0206}}
% 21
{\PTglyphid{U1-08_0207}}
% 22
{\PTglyphid{U1-08_0208}}
% 23
{\PTglyphid{U1-08_0209}}
% 24
{\PTglyphid{U1-08_0210}}
% 25
{\PTglyphid{U1-08_0211}}
% 26
{\PTglyphid{U1-08_0212}}
% 27
{\PTglyphid{U1-08_0213}}
% 28
{\PTglyphid{U1-08_0214}}
% 29
{\PTglyphid{U1-08_0215}}
% 30
{\PTglyphid{U1-08_0216}}
% 31
{\PTglyphid{U1-08_0217}}
% 32
{\PTglyphid{U1-08_0218}}
% 33
{\PTglyphid{U1-08_0219}}
% 34
{\PTglyphid{U1-08_0220}}
% 35
{\PTglyphid{U1-08_0221}}
% 36
{\PTglyphid{U1-08_0222}}
% 37
{\PTglyphid{U1-08_0223}}
% 38
{\PTglyphid{U1-08_0224}}
% 39
{\PTglyphid{U1-08_0225}}
% 40
{\PTglyphid{U1-08_0226}}
//
\endgl \xe
%%% Local Variables:
%%% mode: latex
%%% TeX-engine: luatex
%%% TeX-master: shared
%%% End:

% //
%\endgl \xe

 \newpage
 
%%%%%%%%%%%%%%%%%%%%%%%%%%%%%%%%%%%%%%%%%%%%%%%%%%%%%%%%%%%%%%%%%%%%%%%%%%%%%%%
% from meta.csv
 % 50,Ungler1-09_PT03_120.djvu,Ungler1,09,03,120
%%%%%%%%%%%%%%%%%%%%%%%%%%%%%%%%%%%%%%%%%%%%%%%%%%%%%%%%%%%%%%%%%%%%%%%%%%%%%%%

 
 % from dsed4test:
% Ungler1-09_PT03_120_4dsed.txt:Note "9. Pismo nagłówkowe gotyckie. Krój M²³. Stopień 10 ww. == ok. 70 mm. — Tabl. 120."
% Ungler1-09_PT03_120_4dsed.txt:Note1 "Character set table prepared by Maria Błońska"

 \pismoPL{Florian Ungler pierwsza drukarnia 9. Pismo nagłówkowe gotyckie. Krój M²³. Stopień 10 ww. == ok. 70 mm. — Tabl. 120.}
 
 \pismoEN{Florian Ungler  first house 9. Gothic header font. Typeface M²³. Type size 10 ww. == about 70 mm. — Plate 120.}

\plate{120}{III}{1959}

The plate prepared by Henryk Bułhak.\\
The font table prepared by  Maria Błońska.

\bigskip

% \exampleBib{III:32}

% \bigskip \exampleDesc{LAURENTIUS CORVINUS: Latinum ideoma. Kraków, Florian Ungler, 1513. 4⁰}

% \medskip
% \examplePage{\textit{Karta A₇a}}

%   \bigskip
%   \exampleLib{Biblioteka Narodowa. Warszawa.}
  

% \bigskip \exampleRef{\textit{Estreicher XIV. 424. Wierzbowski 885. Piekarski U. 30.}}

% \bigskip

% \examplePL{[Pismo 7]   Rubryka \gamma{}. — Cyfry 4.}

% \medskip

% \exampleEN{[Font 6] Rubric \gamma{}. — Digits 4.}


\bigskip

\fontID{U1-09}{50}

\fontstat{38}

% \exdisplay \bg \gla
 \exdisplay \bg \gla
% 1
{\PTglyph{5}{t50_l01g01.png}}
% 2
{\PTglyph{5}{t50_l01g02.png}}
% 3
{\PTglyph{5}{t50_l01g03.png}}
% 4
{\PTglyph{5}{t50_l01g04.png}}
% 5
{\PTglyph{5}{t50_l01g05.png}}
% 6
{\PTglyph{5}{t50_l01g06.png}}
% 7
{\PTglyph{5}{t50_l01g07.png}}
% 8
{\PTglyph{5}{t50_l01g08.png}}
% 9
{\PTglyph{5}{t50_l02g01.png}}
% 10
{\PTglyph{5}{t50_l02g02.png}}
% 11
{\PTglyph{5}{t50_l02g03.png}}
% 12
{\PTglyph{5}{t50_l02g04.png}}
% 13
{\PTglyph{5}{t50_l02g05.png}}
% 14
{\PTglyph{5}{t50_l02g06.png}}
% 15
{\PTglyph{5}{t50_l02g07.png}}
% 16
{\PTglyph{5}{t50_l02g08.png}}
% 17
{\PTglyph{5}{t50_l02g09.png}}
% 18
{\PTglyph{5}{t50_l02g10.png}}
% 19
{\PTglyph{5}{t50_l02g11.png}}
% 20
{\PTglyph{5}{t50_l02g12.png}}
% 21
{\PTglyph{5}{t50_l02g13.png}}
% 22
{\PTglyph{5}{t50_l02g14.png}}
% 23
{\PTglyph{5}{t50_l02g15.png}}
% 24
{\PTglyph{5}{t50_l02g16.png}}
% 25
{\PTglyph{5}{t50_l02g17.png}}
% 26
{\PTglyph{5}{t50_l02g18.png}}
% 27
{\PTglyph{5}{t50_l02g19.png}}
% 28
{\PTglyph{5}{t50_l02g20.png}}
% 29
{\PTglyph{5}{t50_l02g21.png}}
% 30
{\PTglyph{5}{t50_l02g22.png}}
% 31
{\PTglyph{5}{t50_l02g23.png}}
% 32
{\PTglyph{5}{t50_l02g24.png}}
% 33
{\PTglyph{5}{t50_l02g25.png}}
% 34
{\PTglyph{5}{t50_l02g26.png}}
% 35
{\PTglyph{5}{t50_l02g27.png}}
% 36
{\PTglyph{5}{t50_l02g28.png}}
% 37
{\PTglyph{5}{t50_l02g29.png}}
% 38
{\PTglyph{5}{t50_l02g30.png}}
//
%%% Local Variables:
%%% mode: latex
%%% TeX-engine: luatex
%%% TeX-master: shared
%%% End:

%//
%\glpismo%
 \glpismo
% 1
{\PTglyphid{U1-09_0101}}
% 2
{\PTglyphid{U1-09_0102}}
% 3
{\PTglyphid{U1-09_0103}}
% 4
{\PTglyphid{U1-09_0104}}
% 5
{\PTglyphid{U1-09_0105}}
% 6
{\PTglyphid{U1-09_0106}}
% 7
{\PTglyphid{U1-09_0107}}
% 8
{\PTglyphid{U1-09_0108}}
% 9
{\PTglyphid{U1-09_0201}}
% 10
{\PTglyphid{U1-09_0202}}
% 11
{\PTglyphid{U1-09_0203}}
% 12
{\PTglyphid{U1-09_0204}}
% 13
{\PTglyphid{U1-09_0205}}
% 14
{\PTglyphid{U1-09_0206}}
% 15
{\PTglyphid{U1-09_0207}}
% 16
{\PTglyphid{U1-09_0208}}
% 17
{\PTglyphid{U1-09_0209}}
% 18
{\PTglyphid{U1-09_0210}}
% 19
{\PTglyphid{U1-09_0211}}
% 20
{\PTglyphid{U1-09_0212}}
% 21
{\PTglyphid{U1-09_0213}}
% 22
{\PTglyphid{U1-09_0214}}
% 23
{\PTglyphid{U1-09_0215}}
% 24
{\PTglyphid{U1-09_0216}}
% 25
{\PTglyphid{U1-09_0217}}
% 26
{\PTglyphid{U1-09_0218}}
% 27
{\PTglyphid{U1-09_0219}}
% 28
{\PTglyphid{U1-09_0220}}
% 29
{\PTglyphid{U1-09_0221}}
% 30
{\PTglyphid{U1-09_0222}}
% 31
{\PTglyphid{U1-09_0223}}
% 32
{\PTglyphid{U1-09_0224}}
% 33
{\PTglyphid{U1-09_0225}}
% 34
{\PTglyphid{U1-09_0226}}
% 35
{\PTglyphid{U1-09_0227}}
% 36
{\PTglyphid{U1-09_0228}}
% 37
{\PTglyphid{U1-09_0229}}
% 38
{\PTglyphid{U1-09_0230}}
//
\endgl \xe
%%% Local Variables:
%%% mode: latex
%%% TeX-engine: luatex
%%% TeX-master: shared
%%% End:

% //
%\endgl \xe



 \newpage
 
%%%%%%%%%%%%%%%%%%%%%%%%%%%%%%%%%%%%%%%%%%%%%%%%%%%%%%%%%%%%%%%%%%%%%%%%%%%%%%%
% from meta.csv
% 51,Ungler1-10_PT03_119.djvu,Ungler1,10,03,119
% 
%%%%%%%%%%%%%%%%%%%%%%%%%%%%%%%%%%%%%%%%%%%%%%%%%%%%%%%%%%%%%%%%%%%%%%%%%%%%%%%

 
 % from dsed4test:
% Ungler1-10_PT03_119_4dsed.txt:Note "10. Pismo tekstowe antykwowe. Krój Qu/ pośredni między G i K⁵. Stopień 20 ww. == 95/96 mm. — Tabl. 119."
% Ungler1-10_PT03_119_4dsed.txt:Note1 "Character set table prepared by Maria Błońska"

 \pismoPL{Florian Ungler pierwsza drukarnia 10. Pismo tekstowe
   antykwowe. Krój Qu/ pośredni między G i K⁵. Stopień 20 ww. == 95/96
   mm. — Tabl. 119.}
  
 \pismoEN{Florian Ungler  first house 10. Roman text font. Typeface Qu/ halfway between G and K⁵. Type size 20 lines == 95/96
   mm. — Plate 119.}

\plate{119}{III}{1959}

The plate prepared by Henryk Bułhak.\\
The font table prepared  by Henryk Bułhak and Maria Błońska.

\bigskip

\exampleBib{III:50}

\bigskip \exampleDesc{IOANNES LASKI: Oratio ad Leonem X pp. habita 13 Iunii 1513.
Kraków, Florian Ungler [1 Wolfgang Lern?, 1514]. 4⁰.}

\medskip
\examplePage{\textit{Karta A₃b}}

  \bigskip
  \exampleLib{Biblioteka Jagiellońska. Kraków.}


\bigskip \exampleRef{\textit{Estreicher. XXI. 80. Wierzbowski 2070. Piekarski U. 39. Piekarski w Sil. Rer. III 1927 s. 181.}}
% \bigskip

% \examplePL{[Pismo 10]   Rubryka \delta{}.}

% \medskip

% \exampleEN{


\bigskip

\fontID{U1-10}{51}

\fontstat{109}

% \exdisplay \bg \gla
 \exdisplay \bg \gla
% 1
{\PTglyph{5}{t51_l01g01.png}}
% 2
{\PTglyph{5}{t51_l01g02.png}}
% 3
{\PTglyph{5}{t51_l01g03.png}}
% 4
{\PTglyph{5}{t51_l01g04.png}}
% 5
{\PTglyph{5}{t51_l01g05.png}}
% 6
{\PTglyph{5}{t51_l01g06.png}}
% 7
{\PTglyph{5}{t51_l01g07.png}}
% 8
{\PTglyph{5}{t51_l01g08.png}}
% 9
{\PTglyph{5}{t51_l01g09.png}}
% 10
{\PTglyph{5}{t51_l01g10.png}}
% 11
{\PTglyph{5}{t51_l01g11.png}}
% 12
{\PTglyph{5}{t51_l01g12.png}}
% 13
{\PTglyph{5}{t51_l01g13.png}}
% 14
{\PTglyph{5}{t51_l01g14.png}}
% 15
{\PTglyph{5}{t51_l01g15.png}}
% 16
{\PTglyph{5}{t51_l01g16.png}}
% 17
{\PTglyph{5}{t51_l01g17.png}}
% 18
{\PTglyph{5}{t51_l01g18.png}}
% 19
{\PTglyph{5}{t51_l01g19.png}}
% 20
{\PTglyph{5}{t51_l01g20.png}}
% 21
{\PTglyph{5}{t51_l01g21.png}}
% 22
{\PTglyph{5}{t51_l01g22.png}}
% 23
{\PTglyph{5}{t51_l01g23.png}}
% 24
{\PTglyph{5}{t51_l02g01.png}}
% 25
{\PTglyph{5}{t51_l02g02.png}}
% 26
{\PTglyph{5}{t51_l02g03.png}}
% 27
{\PTglyph{5}{t51_l02g04.png}}
% 28
{\PTglyph{5}{t51_l02g05.png}}
% 29
{\PTglyph{5}{t51_l02g06.png}}
% 30
{\PTglyph{5}{t51_l02g07.png}}
% 31
{\PTglyph{5}{t51_l02g08.png}}
% 32
{\PTglyph{5}{t51_l02g09.png}}
% 33
{\PTglyph{5}{t51_l02g10.png}}
% 34
{\PTglyph{5}{t51_l02g11.png}}
% 35
{\PTglyph{5}{t51_l02g12.png}}
% 36
{\PTglyph{5}{t51_l02g13.png}}
% 37
{\PTglyph{5}{t51_l02g14.png}}
% 38
{\PTglyph{5}{t51_l02g15.png}}
% 39
{\PTglyph{5}{t51_l02g16.png}}
% 40
{\PTglyph{5}{t51_l02g17.png}}
% 41
{\PTglyph{5}{t51_l02g18.png}}
% 42
{\PTglyph{5}{t51_l02g19.png}}
% 43
{\PTglyph{5}{t51_l02g20.png}}
% 44
{\PTglyph{5}{t51_l02g21.png}}
% 45
{\PTglyph{5}{t51_l02g22.png}}
% 46
{\PTglyph{5}{t51_l02g23.png}}
% 47
{\PTglyph{5}{t51_l02g24.png}}
% 48
{\PTglyph{5}{t51_l02g25.png}}
% 49
{\PTglyph{5}{t51_l02g26.png}}
% 50
{\PTglyph{5}{t51_l02g27.png}}
% 51
{\PTglyph{5}{t51_l02g28.png}}
% 52
{\PTglyph{5}{t51_l02g29.png}}
% 53
{\PTglyph{5}{t51_l03g01.png}}
% 54
{\PTglyph{5}{t51_l03g02.png}}
% 55
{\PTglyph{5}{t51_l03g03.png}}
% 56
{\PTglyph{5}{t51_l03g04.png}}
% 57
{\PTglyph{5}{t51_l03g05.png}}
% 58
{\PTglyph{5}{t51_l03g06.png}}
% 59
{\PTglyph{5}{t51_l03g07.png}}
% 60
{\PTglyph{5}{t51_l03g08.png}}
% 61
{\PTglyph{5}{t51_l03g09.png}}
% 62
{\PTglyph{5}{t51_l03g10.png}}
% 63
{\PTglyph{5}{t51_l03g11.png}}
% 64
{\PTglyph{5}{t51_l03g12.png}}
% 65
{\PTglyph{5}{t51_l03g13.png}}
% 66
{\PTglyph{5}{t51_l03g14.png}}
% 67
{\PTglyph{5}{t51_l03g15.png}}
% 68
{\PTglyph{5}{t51_l03g16.png}}
% 69
{\PTglyph{5}{t51_l03g17.png}}
% 70
{\PTglyph{5}{t51_l03g18.png}}
% 71
{\PTglyph{5}{t51_l03g19.png}}
% 72
{\PTglyph{5}{t51_l03g20.png}}
% 73
{\PTglyph{5}{t51_l03g21.png}}
% 74
{\PTglyph{5}{t51_l03g22.png}}
% 75
{\PTglyph{5}{t51_l03g23.png}}
% 76
{\PTglyph{5}{t51_l03g24.png}}
% 77
{\PTglyph{5}{t51_l03g25.png}}
% 78
{\PTglyph{5}{t51_l03g26.png}}
% 79
{\PTglyph{5}{t51_l03g27.png}}
% 80
{\PTglyph{5}{t51_l04g01.png}}
% 81
{\PTglyph{5}{t51_l04g02.png}}
% 82
{\PTglyph{5}{t51_l04g03.png}}
% 83
{\PTglyph{5}{t51_l04g04.png}}
% 84
{\PTglyph{5}{t51_l04g05.png}}
% 85
{\PTglyph{5}{t51_l04g06.png}}
% 86
{\PTglyph{5}{t51_l04g07.png}}
% 87
{\PTglyph{5}{t51_l04g08.png}}
% 88
{\PTglyph{5}{t51_l04g09.png}}
% 89
{\PTglyph{5}{t51_l04g10.png}}
% 90
{\PTglyph{5}{t51_l04g11.png}}
% 91
{\PTglyph{5}{t51_l04g12.png}}
% 92
{\PTglyph{5}{t51_l04g13.png}}
% 93
{\PTglyph{5}{t51_l04g14.png}}
% 94
{\PTglyph{5}{t51_l04g15.png}}
% 95
{\PTglyph{5}{t51_l04g16.png}}
% 96
{\PTglyph{5}{t51_l04g17.png}}
% 97
{\PTglyph{5}{t51_l04g18.png}}
% 98
{\PTglyph{5}{t51_l04g19.png}}
% 99
{\PTglyph{5}{t51_l04g20.png}}
% 100
{\PTglyph{5}{t51_l05g01.png}}
% 101
{\PTglyph{5}{t51_l05g02.png}}
% 102
{\PTglyph{5}{t51_l05g03.png}}
% 103
{\PTglyph{5}{t51_l05g04.png}}
% 104
{\PTglyph{5}{t51_l05g05.png}}
% 105
{\PTglyph{5}{t51_l05g06.png}}
% 106
{\PTglyph{5}{t51_l05g07.png}}
% 107
{\PTglyph{5}{t51_l05g08.png}}
% 108
{\PTglyph{5}{t51_l05g09.png}}
% 109
{\PTglyph{5}{t51_l05g10.png}}
//
%%% Local Variables:
%%% mode: latex
%%% TeX-engine: luatex
%%% TeX-master: shared
%%% End:

%//
%\glpismo%
 \glpismo
% 1
{\PTglyphid{U1-10_0101}}
% 2
{\PTglyphid{U1-10_0102}}
% 3
{\PTglyphid{U1-10_0103}}
% 4
{\PTglyphid{U1-10_0104}}
% 5
{\PTglyphid{U1-10_0105}}
% 6
{\PTglyphid{U1-10_0106}}
% 7
{\PTglyphid{U1-10_0107}}
% 8
{\PTglyphid{U1-10_0108}}
% 9
{\PTglyphid{U1-10_0109}}
% 10
{\PTglyphid{U1-10_0110}}
% 11
{\PTglyphid{U1-10_0111}}
% 12
{\PTglyphid{U1-10_0112}}
% 13
{\PTglyphid{U1-10_0113}}
% 14
{\PTglyphid{U1-10_0114}}
% 15
{\PTglyphid{U1-10_0115}}
% 16
{\PTglyphid{U1-10_0116}}
% 17
{\PTglyphid{U1-10_0117}}
% 18
{\PTglyphid{U1-10_0118}}
% 19
{\PTglyphid{U1-10_0119}}
% 20
{\PTglyphid{U1-10_0120}}
% 21
{\PTglyphid{U1-10_0121}}
% 22
{\PTglyphid{U1-10_0122}}
% 23
{\PTglyphid{U1-10_0123}}
% 24
{\PTglyphid{U1-10_0201}}
% 25
{\PTglyphid{U1-10_0202}}
% 26
{\PTglyphid{U1-10_0203}}
% 27
{\PTglyphid{U1-10_0204}}
% 28
{\PTglyphid{U1-10_0205}}
% 29
{\PTglyphid{U1-10_0206}}
% 30
{\PTglyphid{U1-10_0207}}
% 31
{\PTglyphid{U1-10_0208}}
% 32
{\PTglyphid{U1-10_0209}}
% 33
{\PTglyphid{U1-10_0210}}
% 34
{\PTglyphid{U1-10_0211}}
% 35
{\PTglyphid{U1-10_0212}}
% 36
{\PTglyphid{U1-10_0213}}
% 37
{\PTglyphid{U1-10_0214}}
% 38
{\PTglyphid{U1-10_0215}}
% 39
{\PTglyphid{U1-10_0216}}
% 40
{\PTglyphid{U1-10_0217}}
% 41
{\PTglyphid{U1-10_0218}}
% 42
{\PTglyphid{U1-10_0219}}
% 43
{\PTglyphid{U1-10_0220}}
% 44
{\PTglyphid{U1-10_0221}}
% 45
{\PTglyphid{U1-10_0222}}
% 46
{\PTglyphid{U1-10_0223}}
% 47
{\PTglyphid{U1-10_0224}}
% 48
{\PTglyphid{U1-10_0225}}
% 49
{\PTglyphid{U1-10_0226}}
% 50
{\PTglyphid{U1-10_0227}}
% 51
{\PTglyphid{U1-10_0228}}
% 52
{\PTglyphid{U1-10_0229}}
% 53
{\PTglyphid{U1-10_0301}}
% 54
{\PTglyphid{U1-10_0302}}
% 55
{\PTglyphid{U1-10_0303}}
% 56
{\PTglyphid{U1-10_0304}}
% 57
{\PTglyphid{U1-10_0305}}
% 58
{\PTglyphid{U1-10_0306}}
% 59
{\PTglyphid{U1-10_0307}}
% 60
{\PTglyphid{U1-10_0308}}
% 61
{\PTglyphid{U1-10_0309}}
% 62
{\PTglyphid{U1-10_0310}}
% 63
{\PTglyphid{U1-10_0311}}
% 64
{\PTglyphid{U1-10_0312}}
% 65
{\PTglyphid{U1-10_0313}}
% 66
{\PTglyphid{U1-10_0314}}
% 67
{\PTglyphid{U1-10_0315}}
% 68
{\PTglyphid{U1-10_0316}}
% 69
{\PTglyphid{U1-10_0317}}
% 70
{\PTglyphid{U1-10_0318}}
% 71
{\PTglyphid{U1-10_0319}}
% 72
{\PTglyphid{U1-10_0320}}
% 73
{\PTglyphid{U1-10_0321}}
% 74
{\PTglyphid{U1-10_0322}}
% 75
{\PTglyphid{U1-10_0323}}
% 76
{\PTglyphid{U1-10_0324}}
% 77
{\PTglyphid{U1-10_0325}}
% 78
{\PTglyphid{U1-10_0326}}
% 79
{\PTglyphid{U1-10_0327}}
% 80
{\PTglyphid{U1-10_0401}}
% 81
{\PTglyphid{U1-10_0402}}
% 82
{\PTglyphid{U1-10_0403}}
% 83
{\PTglyphid{U1-10_0404}}
% 84
{\PTglyphid{U1-10_0405}}
% 85
{\PTglyphid{U1-10_0406}}
% 86
{\PTglyphid{U1-10_0407}}
% 87
{\PTglyphid{U1-10_0408}}
% 88
{\PTglyphid{U1-10_0409}}
% 89
{\PTglyphid{U1-10_0410}}
% 90
{\PTglyphid{U1-10_0411}}
% 91
{\PTglyphid{U1-10_0412}}
% 92
{\PTglyphid{U1-10_0413}}
% 93
{\PTglyphid{U1-10_0414}}
% 94
{\PTglyphid{U1-10_0415}}
% 95
{\PTglyphid{U1-10_0416}}
% 96
{\PTglyphid{U1-10_0417}}
% 97
{\PTglyphid{U1-10_0418}}
% 98
{\PTglyphid{U1-10_0419}}
% 99
{\PTglyphid{U1-10_0420}}
% 100
{\PTglyphid{U1-10_0501}}
% 101
{\PTglyphid{U1-10_0502}}
% 102
{\PTglyphid{U1-10_0503}}
% 103
{\PTglyphid{U1-10_0504}}
% 104
{\PTglyphid{U1-10_0505}}
% 105
{\PTglyphid{U1-10_0506}}
% 106
{\PTglyphid{U1-10_0507}}
% 107
{\PTglyphid{U1-10_0508}}
% 108
{\PTglyphid{U1-10_0509}}
% 109
{\PTglyphid{U1-10_0510}}
//
\endgl \xe
%%% Local Variables:
%%% mode: latex
%%% TeX-engine: luatex
%%% TeX-master: shared
%%% End:

% //
%\endgl \xe


 \newpage
 
%%%%%%%%%%%%%%%%%%%%%%%%%%%%%%%%%%%%%%%%%%%%%%%%%%%%%%%%%%%%%%%%%%%%%%%%%%%%%%%
% from meta.csv
% 52,Ungler2-01_PT05_239.djvu,Ungler2,01,05,239
% 
%%%%%%%%%%%%%%%%%%%%%%%%%%%%%%%%%%%%%%%%%%%%%%%%%%%%%%%%%%%%%%%%%%%%%%%%%%%%%%%

 
 % from dsed4test:
% Ungler2-01_PT05_239_4dsed.txt:Note "1. Pismo nagłówkowe i tekstowe, rotunda M²³. Stopień 20 ww. = 132 mm. — Tabl. 239."
% Ungler2-01_PT05_239_4dsed.txt:Note1 "Character set table prepared by Maria Błońska"

 \pismoPL{Florian Ungler druga drukarnia 1. Pismo nagłówkowe i tekstowe, rotunda M²³. Stopień 20 ww. = 132 mm. — Tabl. 239.}
  
 \pismoEN{Florian Ungler  second house 1. Rotunda header and text font, typeface M²³. Type size 20 lines = 132 mm. — Plate 239.}

\plate{239}{V}{1964}

The plate prepared by Henryk Bułhak.\\
The font table prepared by Henryk Bułhak and Maria Błońska.

\bigskip

\exampleBib{V:23}

\bigskip \exampleDesc{DONATUS: Dialogus iuvenibus scholaribus perutilissimus. Kraków, [Florian Ungler], 1523. 4⁰.}

\medskip
\examplePage{\textit{Karta 2 b (?)}}

  \bigskip
%  \exampleLib{Biblioteka Jagiellońska. Kraków.}
% Egzemplarz niedostępny.
  

\bigskip \exampleRef{\textit{L. Zalewski: Biblioteka Seminarium Duchownego w Lublinie i biblioteki klasztorne w diecezji lubelskiej i podlaskiej. Warszawa 1926. s. 203.}}
% \bigskip

% \examplePL{[Pismo 10]   Rubryka \delta{}.}
% Cyfry 1. — Inicjał 4 (P). — Drzeworyt 157: listwa dolna. — Drzeworyt 158: listwa boczna.
% \medskip

% \exampleEN{


\bigskip

\fontID{U2-01}{52}

\fontstat{99}

% \exdisplay \bg \gla
 \exdisplay \bg \gla
% 1
{\PTglyph{5}{t52_l01g01.png}}
% 2
{\PTglyph{5}{t52_l01g02.png}}
% 3
{\PTglyph{5}{t52_l01g03.png}}
% 4
{\PTglyph{5}{t52_l01g04.png}}
% 5
{\PTglyph{5}{t52_l01g05.png}}
% 6
{\PTglyph{5}{t52_l01g06.png}}
% 7
{\PTglyph{5}{t52_l01g07.png}}
% 8
{\PTglyph{5}{t52_l01g08.png}}
% 9
{\PTglyph{5}{t52_l01g09.png}}
% 10
{\PTglyph{5}{t52_l01g10.png}}
% 11
{\PTglyph{5}{t52_l01g11.png}}
% 12
{\PTglyph{5}{t52_l01g12.png}}
% 13
{\PTglyph{5}{t52_l01g13.png}}
% 14
{\PTglyph{5}{t52_l01g14.png}}
% 15
{\PTglyph{5}{t52_l01g15.png}}
% 16
{\PTglyph{5}{t52_l01g16.png}}
% 17
{\PTglyph{5}{t52_l01g17.png}}
% 18
{\PTglyph{5}{t52_l01g18.png}}
% 19
{\PTglyph{5}{t52_l01g19.png}}
% 20
{\PTglyph{5}{t52_l02g01.png}}
% 21
{\PTglyph{5}{t52_l02g02.png}}
% 22
{\PTglyph{5}{t52_l02g03.png}}
% 23
{\PTglyph{5}{t52_l02g04.png}}
% 24
{\PTglyph{5}{t52_l02g05.png}}
% 25
{\PTglyph{5}{t52_l02g06.png}}
% 26
{\PTglyph{5}{t52_l02g07.png}}
% 27
{\PTglyph{5}{t52_l03g01.png}}
% 28
{\PTglyph{5}{t52_l03g02.png}}
% 29
{\PTglyph{5}{t52_l03g03.png}}
% 30
{\PTglyph{5}{t52_l03g04.png}}
% 31
{\PTglyph{5}{t52_l03g05.png}}
% 32
{\PTglyph{5}{t52_l03g06.png}}
% 33
{\PTglyph{5}{t52_l03g07.png}}
% 34
{\PTglyph{5}{t52_l03g08.png}}
% 35
{\PTglyph{5}{t52_l03g09.png}}
% 36
{\PTglyph{5}{t52_l03g10.png}}
% 37
{\PTglyph{5}{t52_l03g11.png}}
% 38
{\PTglyph{5}{t52_l03g12.png}}
% 39
{\PTglyph{5}{t52_l03g13.png}}
% 40
{\PTglyph{5}{t52_l03g14.png}}
% 41
{\PTglyph{5}{t52_l03g15.png}}
% 42
{\PTglyph{5}{t52_l03g16.png}}
% 43
{\PTglyph{5}{t52_l03g17.png}}
% 44
{\PTglyph{5}{t52_l03g18.png}}
% 45
{\PTglyph{5}{t52_l03g19.png}}
% 46
{\PTglyph{5}{t52_l03g20.png}}
% 47
{\PTglyph{5}{t52_l03g21.png}}
% 48
{\PTglyph{5}{t52_l03g22.png}}
% 49
{\PTglyph{5}{t52_l03g23.png}}
% 50
{\PTglyph{5}{t52_l03g24.png}}
% 51
{\PTglyph{5}{t52_l03g25.png}}
% 52
{\PTglyph{5}{t52_l03g26.png}}
% 53
{\PTglyph{5}{t52_l03g27.png}}
% 54
{\PTglyph{5}{t52_l03g28.png}}
% 55
{\PTglyph{5}{t52_l03g29.png}}
% 56
{\PTglyph{5}{t52_l03g30.png}}
% 57
{\PTglyph{5}{t52_l03g31.png}}
% 58
{\PTglyph{5}{t52_l03g32.png}}
% 59
{\PTglyph{5}{t52_l04g01.png}}
% 60
{\PTglyph{5}{t52_l04g02.png}}
% 61
{\PTglyph{5}{t52_l04g03.png}}
% 62
{\PTglyph{5}{t52_l04g04.png}}
% 63
{\PTglyph{5}{t52_l04g05.png}}
% 64
{\PTglyph{5}{t52_l04g06.png}}
% 65
{\PTglyph{5}{t52_l04g07.png}}
% 66
{\PTglyph{5}{t52_l04g08.png}}
% 67
{\PTglyph{5}{t52_l04g09.png}}
% 68
{\PTglyph{5}{t52_l04g10.png}}
% 69
{\PTglyph{5}{t52_l04g11.png}}
% 70
{\PTglyph{5}{t52_l04g12.png}}
% 71
{\PTglyph{5}{t52_l04g13.png}}
% 72
{\PTglyph{5}{t52_l04g14.png}}
% 73
{\PTglyph{5}{t52_l04g15.png}}
% 74
{\PTglyph{5}{t52_l04g16.png}}
% 75
{\PTglyph{5}{t52_l04g17.png}}
% 76
{\PTglyph{5}{t52_l04g18.png}}
% 77
{\PTglyph{5}{t52_l04g19.png}}
% 78
{\PTglyph{5}{t52_l04g20.png}}
% 79
{\PTglyph{5}{t52_l04g21.png}}
% 80
{\PTglyph{5}{t52_l04g22.png}}
% 81
{\PTglyph{5}{t52_l04g23.png}}
% 82
{\PTglyph{5}{t52_l04g24.png}}
% 83
{\PTglyph{5}{t52_l04g25.png}}
% 84
{\PTglyph{5}{t52_l04g26.png}}
% 85
{\PTglyph{5}{t52_l04g27.png}}
% 86
{\PTglyph{5}{t52_l04g28.png}}
% 87
{\PTglyph{5}{t52_l04g29.png}}
% 88
{\PTglyph{5}{t52_l04g30.png}}
% 89
{\PTglyph{5}{t52_l04g31.png}}
% 90
{\PTglyph{5}{t52_l05g01.png}}
% 91
{\PTglyph{5}{t52_l05g02.png}}
% 92
{\PTglyph{5}{t52_l05g03.png}}
% 93
{\PTglyph{5}{t52_l05g04.png}}
% 94
{\PTglyph{5}{t52_l05g05.png}}
% 95
{\PTglyph{5}{t52_l05g06.png}}
% 96
{\PTglyph{5}{t52_l05g07.png}}
% 97
{\PTglyph{5}{t52_l05g08.png}}
% 98
{\PTglyph{5}{t52_l05g09.png}}
% 99
{\PTglyph{5}{t52_l05g10.png}}
//
%%% Local Variables:
%%% mode: latex
%%% TeX-engine: luatex
%%% TeX-master: shared
%%% End:

%//
%\glpismo%
 \glpismo
% 1
{\PTglyphid{U2-01_0101}}
% 2
{\PTglyphid{U2-01_0102}}
% 3
{\PTglyphid{U2-01_0103}}
% 4
{\PTglyphid{U2-01_0104}}
% 5
{\PTglyphid{U2-01_0105}}
% 6
{\PTglyphid{U2-01_0106}}
% 7
{\PTglyphid{U2-01_0107}}
% 8
{\PTglyphid{U2-01_0108}}
% 9
{\PTglyphid{U2-01_0109}}
% 10
{\PTglyphid{U2-01_0110}}
% 11
{\PTglyphid{U2-01_0111}}
% 12
{\PTglyphid{U2-01_0112}}
% 13
{\PTglyphid{U2-01_0113}}
% 14
{\PTglyphid{U2-01_0114}}
% 15
{\PTglyphid{U2-01_0115}}
% 16
{\PTglyphid{U2-01_0116}}
% 17
{\PTglyphid{U2-01_0117}}
% 18
{\PTglyphid{U2-01_0118}}
% 19
{\PTglyphid{U2-01_0119}}
% 20
{\PTglyphid{U2-01_0201}}
% 21
{\PTglyphid{U2-01_0202}}
% 22
{\PTglyphid{U2-01_0203}}
% 23
{\PTglyphid{U2-01_0204}}
% 24
{\PTglyphid{U2-01_0205}}
% 25
{\PTglyphid{U2-01_0206}}
% 26
{\PTglyphid{U2-01_0207}}
% 27
{\PTglyphid{U2-01_0301}}
% 28
{\PTglyphid{U2-01_0302}}
% 29
{\PTglyphid{U2-01_0303}}
% 30
{\PTglyphid{U2-01_0304}}
% 31
{\PTglyphid{U2-01_0305}}
% 32
{\PTglyphid{U2-01_0306}}
% 33
{\PTglyphid{U2-01_0307}}
% 34
{\PTglyphid{U2-01_0308}}
% 35
{\PTglyphid{U2-01_0309}}
% 36
{\PTglyphid{U2-01_0310}}
% 37
{\PTglyphid{U2-01_0311}}
% 38
{\PTglyphid{U2-01_0312}}
% 39
{\PTglyphid{U2-01_0313}}
% 40
{\PTglyphid{U2-01_0314}}
% 41
{\PTglyphid{U2-01_0315}}
% 42
{\PTglyphid{U2-01_0316}}
% 43
{\PTglyphid{U2-01_0317}}
% 44
{\PTglyphid{U2-01_0318}}
% 45
{\PTglyphid{U2-01_0319}}
% 46
{\PTglyphid{U2-01_0320}}
% 47
{\PTglyphid{U2-01_0321}}
% 48
{\PTglyphid{U2-01_0322}}
% 49
{\PTglyphid{U2-01_0323}}
% 50
{\PTglyphid{U2-01_0324}}
% 51
{\PTglyphid{U2-01_0325}}
% 52
{\PTglyphid{U2-01_0326}}
% 53
{\PTglyphid{U2-01_0327}}
% 54
{\PTglyphid{U2-01_0328}}
% 55
{\PTglyphid{U2-01_0329}}
% 56
{\PTglyphid{U2-01_0330}}
% 57
{\PTglyphid{U2-01_0331}}
% 58
{\PTglyphid{U2-01_0332}}
% 59
{\PTglyphid{U2-01_0401}}
% 60
{\PTglyphid{U2-01_0402}}
% 61
{\PTglyphid{U2-01_0403}}
% 62
{\PTglyphid{U2-01_0404}}
% 63
{\PTglyphid{U2-01_0405}}
% 64
{\PTglyphid{U2-01_0406}}
% 65
{\PTglyphid{U2-01_0407}}
% 66
{\PTglyphid{U2-01_0408}}
% 67
{\PTglyphid{U2-01_0409}}
% 68
{\PTglyphid{U2-01_0410}}
% 69
{\PTglyphid{U2-01_0411}}
% 70
{\PTglyphid{U2-01_0412}}
% 71
{\PTglyphid{U2-01_0413}}
% 72
{\PTglyphid{U2-01_0414}}
% 73
{\PTglyphid{U2-01_0415}}
% 74
{\PTglyphid{U2-01_0416}}
% 75
{\PTglyphid{U2-01_0417}}
% 76
{\PTglyphid{U2-01_0418}}
% 77
{\PTglyphid{U2-01_0419}}
% 78
{\PTglyphid{U2-01_0420}}
% 79
{\PTglyphid{U2-01_0421}}
% 80
{\PTglyphid{U2-01_0422}}
% 81
{\PTglyphid{U2-01_0423}}
% 82
{\PTglyphid{U2-01_0424}}
% 83
{\PTglyphid{U2-01_0425}}
% 84
{\PTglyphid{U2-01_0426}}
% 85
{\PTglyphid{U2-01_0427}}
% 86
{\PTglyphid{U2-01_0428}}
% 87
{\PTglyphid{U2-01_0429}}
% 88
{\PTglyphid{U2-01_0430}}
% 89
{\PTglyphid{U2-01_0431}}
% 90
{\PTglyphid{U2-01_0501}}
% 91
{\PTglyphid{U2-01_0502}}
% 92
{\PTglyphid{U2-01_0503}}
% 93
{\PTglyphid{U2-01_0504}}
% 94
{\PTglyphid{U2-01_0505}}
% 95
{\PTglyphid{U2-01_0506}}
% 96
{\PTglyphid{U2-01_0507}}
% 97
{\PTglyphid{U2-01_0508}}
% 98
{\PTglyphid{U2-01_0509}}
% 99
{\PTglyphid{U2-01_0510}}
//
\endgl \xe
%%% Local Variables:
%%% mode: latex
%%% TeX-engine: luatex
%%% TeX-master: shared
%%% End:

% //
%\endgl \xe


 \newpage
 
%%%%%%%%%%%%%%%%%%%%%%%%%%%%%%%%%%%%%%%%%%%%%%%%%%%%%%%%%%%%%%%%%%%%%%%%%%%%%%%
% from meta.csv
% 53,Ungler2-02_PT05_240.djvu,Ungler2,02,05,240
% 
%%%%%%%%%%%%%%%%%%%%%%%%%%%%%%%%%%%%%%%%%%%%%%%%%%%%%%%%%%%%%%%%%%%%%%%%%%%%%%%

 
 % from dsed4test:
% Ungler2-02_PT05_240_4dsed.txt:Note "2. Pismo tekstowe, antykwa Q/u (zbliżone do F⁵). Stopień 20 ww. =92— 94 mm. — Tabl. 240."
% Ungler2-02_PT05_240_4dsed.txt:Note1 "Character set table prepared by Maria Błońska"

 \pismoPL{Florian Ungler druga drukarnia 2. Pismo tekstowe, antykwa Q/u (zbliżone do F⁵). Stopień 20 ww. = 92—94 mm. — Tabl. 240.}
  
 \pismoEN{Florian Ungler  second house 2. Roman text font, typeface Q/u (similar to F⁵). Type size 20 lines = 92-94  mm. — Plate 240.}

\plate{240}{V}{1964}

The plate prepared by Henryk Bułhak.\\
The font table prepared by Henryk Bułhak and Maria Błońska.

\bigskip

\exampleBib{V:32}

\bigskip \exampleDesc{PANDECTA successionum cum Alberti Costensis additionibus. Kraków, Florian Ungler, [po 30 III 1524]: 4⁰}

\medskip
\examplePage{\textit{Karta 23 b}}

  \bigskip
  \exampleLib{Biblioteka Jagiellońska. Kraków.}

  

\bigskip \exampleRef{\textit{Estreicher XIV 428.}}
% \bigskip

%Pismo 1: nagłówek. — Rubryki a, e: z pismem 2. — Cyfry 2: pierwszy zestaw. — Cyfry 4: drugi zestaw.

% \examplePL{[Pismo 10]   Rubryka \delta{}.}
% Cyfry 1. — Inicjał 4 (P). — Drzeworyt 157: listwa dolna. — Drzeworyt 158: listwa boczna.
% \medskip

% \exampleEN{


\bigskip

\fontID{U2-02}{53}

\fontstat{153}

% \exdisplay \bg \gla
 \exdisplay \bg \gla
% 1
{\PTglyph{5}{t53_l01g01.png}}
% 2
{\PTglyph{5}{t53_l01g02.png}}
% 3
{\PTglyph{5}{t53_l01g03.png}}
% 4
{\PTglyph{5}{t53_l01g04.png}}
% 5
{\PTglyph{5}{t53_l01g05.png}}
% 6
{\PTglyph{5}{t53_l01g06.png}}
% 7
{\PTglyph{5}{t53_l01g07.png}}
% 8
{\PTglyph{5}{t53_l01g08.png}}
% 9
{\PTglyph{5}{t53_l01g09.png}}
% 10
{\PTglyph{5}{t53_l01g10.png}}
% 11
{\PTglyph{5}{t53_l01g11.png}}
% 12
{\PTglyph{5}{t53_l01g12.png}}
% 13
{\PTglyph{5}{t53_l01g13.png}}
% 14
{\PTglyph{5}{t53_l01g14.png}}
% 15
{\PTglyph{5}{t53_l01g15.png}}
% 16
{\PTglyph{5}{t53_l01g16.png}}
% 17
{\PTglyph{5}{t53_l01g17.png}}
% 18
{\PTglyph{5}{t53_l01g18.png}}
% 19
{\PTglyph{5}{t53_l01g19.png}}
% 20
{\PTglyph{5}{t53_l01g20.png}}
% 21
{\PTglyph{5}{t53_l02g01.png}}
% 22
{\PTglyph{5}{t53_l02g02.png}}
% 23
{\PTglyph{5}{t53_l02g03.png}}
% 24
{\PTglyph{5}{t53_l02g04.png}}
% 25
{\PTglyph{5}{t53_l02g05.png}}
% 26
{\PTglyph{5}{t53_l02g06.png}}
% 27
{\PTglyph{5}{t53_l02g07.png}}
% 28
{\PTglyph{5}{t53_l02g08.png}}
% 29
{\PTglyph{5}{t53_l02g09.png}}
% 30
{\PTglyph{5}{t53_l02g10.png}}
% 31
{\PTglyph{5}{t53_l02g11.png}}
% 32
{\PTglyph{5}{t53_l02g12.png}}
% 33
{\PTglyph{5}{t53_l02g13.png}}
% 34
{\PTglyph{5}{t53_l02g14.png}}
% 35
{\PTglyph{5}{t53_l02g15.png}}
% 36
{\PTglyph{5}{t53_l02g16.png}}
% 37
{\PTglyph{5}{t53_l02g17.png}}
% 38
{\PTglyph{5}{t53_l02g18.png}}
% 39
{\PTglyph{5}{t53_l02g19.png}}
% 40
{\PTglyph{5}{t53_l02g20.png}}
% 41
{\PTglyph{5}{t53_l02g21.png}}
% 42
{\PTglyph{5}{t53_l02g22.png}}
% 43
{\PTglyph{5}{t53_l02g23.png}}
% 44
{\PTglyph{5}{t53_l02g24.png}}
% 45
{\PTglyph{5}{t53_l02g25.png}}
% 46
{\PTglyph{5}{t53_l02g26.png}}
% 47
{\PTglyph{5}{t53_l03g01.png}}
% 48
{\PTglyph{5}{t53_l03g02.png}}
% 49
{\PTglyph{5}{t53_l03g03.png}}
% 50
{\PTglyph{5}{t53_l03g04.png}}
% 51
{\PTglyph{5}{t53_l03g05.png}}
% 52
{\PTglyph{5}{t53_l03g06.png}}
% 53
{\PTglyph{5}{t53_l03g07.png}}
% 54
{\PTglyph{5}{t53_l03g08.png}}
% 55
{\PTglyph{5}{t53_l03g09.png}}
% 56
{\PTglyph{5}{t53_l03g10.png}}
% 57
{\PTglyph{5}{t53_l03g11.png}}
% 58
{\PTglyph{5}{t53_l03g12.png}}
% 59
{\PTglyph{5}{t53_l03g13.png}}
% 60
{\PTglyph{5}{t53_l03g14.png}}
% 61
{\PTglyph{5}{t53_l03g15.png}}
% 62
{\PTglyph{5}{t53_l03g16.png}}
% 63
{\PTglyph{5}{t53_l03g17.png}}
% 64
{\PTglyph{5}{t53_l03g18.png}}
% 65
{\PTglyph{5}{t53_l03g19.png}}
% 66
{\PTglyph{5}{t53_l03g20.png}}
% 67
{\PTglyph{5}{t53_l03g21.png}}
% 68
{\PTglyph{5}{t53_l03g22.png}}
% 69
{\PTglyph{5}{t53_l03g23.png}}
% 70
{\PTglyph{5}{t53_l03g24.png}}
% 71
{\PTglyph{5}{t53_l03g25.png}}
% 72
{\PTglyph{5}{t53_l03g26.png}}
% 73
{\PTglyph{5}{t53_l03g27.png}}
% 74
{\PTglyph{5}{t53_l03g28.png}}
% 75
{\PTglyph{5}{t53_l03g29.png}}
% 76
{\PTglyph{5}{t53_l03g30.png}}
% 77
{\PTglyph{5}{t53_l03g31.png}}
% 78
{\PTglyph{5}{t53_l03g32.png}}
% 79
{\PTglyph{5}{t53_l03g33.png}}
% 80
{\PTglyph{5}{t53_l03g34.png}}
% 81
{\PTglyph{5}{t53_l03g35.png}}
% 82
{\PTglyph{5}{t53_l03g36.png}}
% 83
{\PTglyph{5}{t53_l03g37.png}}
% 84
{\PTglyph{5}{t53_l03g38.png}}
% 85
{\PTglyph{5}{t53_l04g01.png}}
% 86
{\PTglyph{5}{t53_l04g02.png}}
% 87
{\PTglyph{5}{t53_l04g03.png}}
% 88
{\PTglyph{5}{t53_l04g04.png}}
% 89
{\PTglyph{5}{t53_l04g05.png}}
% 90
{\PTglyph{5}{t53_l04g06.png}}
% 91
{\PTglyph{5}{t53_l04g07.png}}
% 92
{\PTglyph{5}{t53_l04g08.png}}
% 93
{\PTglyph{5}{t53_l04g09.png}}
% 94
{\PTglyph{5}{t53_l04g10.png}}
% 95
{\PTglyph{5}{t53_l04g11.png}}
% 96
{\PTglyph{5}{t53_l04g12.png}}
% 97
{\PTglyph{5}{t53_l04g13.png}}
% 98
{\PTglyph{5}{t53_l04g14.png}}
% 99
{\PTglyph{5}{t53_l04g15.png}}
% 100
{\PTglyph{5}{t53_l04g16.png}}
% 101
{\PTglyph{5}{t53_l04g17.png}}
% 102
{\PTglyph{5}{t53_l04g18.png}}
% 103
{\PTglyph{5}{t53_l04g19.png}}
% 104
{\PTglyph{5}{t53_l04g20.png}}
% 105
{\PTglyph{5}{t53_l04g21.png}}
% 106
{\PTglyph{5}{t53_l04g22.png}}
% 107
{\PTglyph{5}{t53_l04g23.png}}
% 108
{\PTglyph{5}{t53_l04g24.png}}
% 109
{\PTglyph{5}{t53_l04g25.png}}
% 110
{\PTglyph{5}{t53_l04g26.png}}
% 111
{\PTglyph{5}{t53_l04g27.png}}
% 112
{\PTglyph{5}{t53_l04g28.png}}
% 113
{\PTglyph{5}{t53_l04g29.png}}
% 114
{\PTglyph{5}{t53_l04g30.png}}
% 115
{\PTglyph{5}{t53_l04g31.png}}
% 116
{\PTglyph{5}{t53_l04g32.png}}
% 117
{\PTglyph{5}{t53_l04g33.png}}
% 118
{\PTglyph{5}{t53_l04g34.png}}
% 119
{\PTglyph{5}{t53_l04g35.png}}
% 120
{\PTglyph{5}{t53_l04g36.png}}
% 121
{\PTglyph{5}{t53_l05g01.png}}
% 122
{\PTglyph{5}{t53_l05g02.png}}
% 123
{\PTglyph{5}{t53_l05g03.png}}
% 124
{\PTglyph{5}{t53_l05g04.png}}
% 125
{\PTglyph{5}{t53_l05g05.png}}
% 126
{\PTglyph{5}{t53_l05g06.png}}
% 127
{\PTglyph{5}{t53_l05g07.png}}
% 128
{\PTglyph{5}{t53_l05g08.png}}
% 129
{\PTglyph{5}{t53_l05g09.png}}
% 130
{\PTglyph{5}{t53_l05g10.png}}
% 131
{\PTglyph{5}{t53_l05g11.png}}
% 132
{\PTglyph{5}{t53_l05g12.png}}
% 133
{\PTglyph{5}{t53_l05g13.png}}
% 134
{\PTglyph{5}{t53_l05g14.png}}
% 135
{\PTglyph{5}{t53_l05g15.png}}
% 136
{\PTglyph{5}{t53_l05g16.png}}
% 137
{\PTglyph{5}{t53_l05g17.png}}
% 138
{\PTglyph{5}{t53_l05g18.png}}
% 139
{\PTglyph{5}{t53_l05g19.png}}
% 140
{\PTglyph{5}{t53_l05g20.png}}
% 141
{\PTglyph{5}{t53_l05g21.png}}
% 142
{\PTglyph{5}{t53_l05g22.png}}
% 143
{\PTglyph{5}{t53_l05g23.png}}
% 144
{\PTglyph{5}{t53_l05g24.png}}
% 145
{\PTglyph{5}{t53_l05g25.png}}
% 146
{\PTglyph{5}{t53_l05g26.png}}
% 147
{\PTglyph{5}{t53_l05g27.png}}
% 148
{\PTglyph{5}{t53_l05g28.png}}
% 149
{\PTglyph{5}{t53_l05g29.png}}
% 150
{\PTglyph{5}{t53_l05g30.png}}
% 151
{\PTglyph{5}{t53_l05g31.png}}
% 152
{\PTglyph{5}{t53_l05g32.png}}
% 153
{\PTglyph{5}{t53_l05g33.png}}
//
%%% Local Variables:
%%% mode: latex
%%% TeX-engine: luatex
%%% TeX-master: shared
%%% End:

%//
%\glpismo%
 \glpismo
% 1
{\PTglyphid{U2-02_0101}}
% 2
{\PTglyphid{U2-02_0102}}
% 3
{\PTglyphid{U2-02_0103}}
% 4
{\PTglyphid{U2-02_0104}}
% 5
{\PTglyphid{U2-02_0105}}
% 6
{\PTglyphid{U2-02_0106}}
% 7
{\PTglyphid{U2-02_0107}}
% 8
{\PTglyphid{U2-02_0108}}
% 9
{\PTglyphid{U2-02_0109}}
% 10
{\PTglyphid{U2-02_0110}}
% 11
{\PTglyphid{U2-02_0111}}
% 12
{\PTglyphid{U2-02_0112}}
% 13
{\PTglyphid{U2-02_0113}}
% 14
{\PTglyphid{U2-02_0114}}
% 15
{\PTglyphid{U2-02_0115}}
% 16
{\PTglyphid{U2-02_0116}}
% 17
{\PTglyphid{U2-02_0117}}
% 18
{\PTglyphid{U2-02_0118}}
% 19
{\PTglyphid{U2-02_0119}}
% 20
{\PTglyphid{U2-02_0120}}
% 21
{\PTglyphid{U2-02_0201}}
% 22
{\PTglyphid{U2-02_0202}}
% 23
{\PTglyphid{U2-02_0203}}
% 24
{\PTglyphid{U2-02_0204}}
% 25
{\PTglyphid{U2-02_0205}}
% 26
{\PTglyphid{U2-02_0206}}
% 27
{\PTglyphid{U2-02_0207}}
% 28
{\PTglyphid{U2-02_0208}}
% 29
{\PTglyphid{U2-02_0209}}
% 30
{\PTglyphid{U2-02_0210}}
% 31
{\PTglyphid{U2-02_0211}}
% 32
{\PTglyphid{U2-02_0212}}
% 33
{\PTglyphid{U2-02_0213}}
% 34
{\PTglyphid{U2-02_0214}}
% 35
{\PTglyphid{U2-02_0215}}
% 36
{\PTglyphid{U2-02_0216}}
% 37
{\PTglyphid{U2-02_0217}}
% 38
{\PTglyphid{U2-02_0218}}
% 39
{\PTglyphid{U2-02_0219}}
% 40
{\PTglyphid{U2-02_0220}}
% 41
{\PTglyphid{U2-02_0221}}
% 42
{\PTglyphid{U2-02_0222}}
% 43
{\PTglyphid{U2-02_0223}}
% 44
{\PTglyphid{U2-02_0224}}
% 45
{\PTglyphid{U2-02_0225}}
% 46
{\PTglyphid{U2-02_0226}}
% 47
{\PTglyphid{U2-02_0301}}
% 48
{\PTglyphid{U2-02_0302}}
% 49
{\PTglyphid{U2-02_0303}}
% 50
{\PTglyphid{U2-02_0304}}
% 51
{\PTglyphid{U2-02_0305}}
% 52
{\PTglyphid{U2-02_0306}}
% 53
{\PTglyphid{U2-02_0307}}
% 54
{\PTglyphid{U2-02_0308}}
% 55
{\PTglyphid{U2-02_0309}}
% 56
{\PTglyphid{U2-02_0310}}
% 57
{\PTglyphid{U2-02_0311}}
% 58
{\PTglyphid{U2-02_0312}}
% 59
{\PTglyphid{U2-02_0313}}
% 60
{\PTglyphid{U2-02_0314}}
% 61
{\PTglyphid{U2-02_0315}}
% 62
{\PTglyphid{U2-02_0316}}
% 63
{\PTglyphid{U2-02_0317}}
% 64
{\PTglyphid{U2-02_0318}}
% 65
{\PTglyphid{U2-02_0319}}
% 66
{\PTglyphid{U2-02_0320}}
% 67
{\PTglyphid{U2-02_0321}}
% 68
{\PTglyphid{U2-02_0322}}
% 69
{\PTglyphid{U2-02_0323}}
% 70
{\PTglyphid{U2-02_0324}}
% 71
{\PTglyphid{U2-02_0325}}
% 72
{\PTglyphid{U2-02_0326}}
% 73
{\PTglyphid{U2-02_0327}}
% 74
{\PTglyphid{U2-02_0328}}
% 75
{\PTglyphid{U2-02_0329}}
% 76
{\PTglyphid{U2-02_0330}}
% 77
{\PTglyphid{U2-02_0331}}
% 78
{\PTglyphid{U2-02_0332}}
% 79
{\PTglyphid{U2-02_0333}}
% 80
{\PTglyphid{U2-02_0334}}
% 81
{\PTglyphid{U2-02_0335}}
% 82
{\PTglyphid{U2-02_0336}}
% 83
{\PTglyphid{U2-02_0337}}
% 84
{\PTglyphid{U2-02_0338}}
% 85
{\PTglyphid{U2-02_0401}}
% 86
{\PTglyphid{U2-02_0402}}
% 87
{\PTglyphid{U2-02_0403}}
% 88
{\PTglyphid{U2-02_0404}}
% 89
{\PTglyphid{U2-02_0405}}
% 90
{\PTglyphid{U2-02_0406}}
% 91
{\PTglyphid{U2-02_0407}}
% 92
{\PTglyphid{U2-02_0408}}
% 93
{\PTglyphid{U2-02_0409}}
% 94
{\PTglyphid{U2-02_0410}}
% 95
{\PTglyphid{U2-02_0411}}
% 96
{\PTglyphid{U2-02_0412}}
% 97
{\PTglyphid{U2-02_0413}}
% 98
{\PTglyphid{U2-02_0414}}
% 99
{\PTglyphid{U2-02_0415}}
% 100
{\PTglyphid{U2-02_0416}}
% 101
{\PTglyphid{U2-02_0417}}
% 102
{\PTglyphid{U2-02_0418}}
% 103
{\PTglyphid{U2-02_0419}}
% 104
{\PTglyphid{U2-02_0420}}
% 105
{\PTglyphid{U2-02_0421}}
% 106
{\PTglyphid{U2-02_0422}}
% 107
{\PTglyphid{U2-02_0423}}
% 108
{\PTglyphid{U2-02_0424}}
% 109
{\PTglyphid{U2-02_0425}}
% 110
{\PTglyphid{U2-02_0426}}
% 111
{\PTglyphid{U2-02_0427}}
% 112
{\PTglyphid{U2-02_0428}}
% 113
{\PTglyphid{U2-02_0429}}
% 114
{\PTglyphid{U2-02_0430}}
% 115
{\PTglyphid{U2-02_0431}}
% 116
{\PTglyphid{U2-02_0432}}
% 117
{\PTglyphid{U2-02_0433}}
% 118
{\PTglyphid{U2-02_0434}}
% 119
{\PTglyphid{U2-02_0435}}
% 120
{\PTglyphid{U2-02_0436}}
% 121
{\PTglyphid{U2-02_0501}}
% 122
{\PTglyphid{U2-02_0502}}
% 123
{\PTglyphid{U2-02_0503}}
% 124
{\PTglyphid{U2-02_0504}}
% 125
{\PTglyphid{U2-02_0505}}
% 126
{\PTglyphid{U2-02_0506}}
% 127
{\PTglyphid{U2-02_0507}}
% 128
{\PTglyphid{U2-02_0508}}
% 129
{\PTglyphid{U2-02_0509}}
% 130
{\PTglyphid{U2-02_0510}}
% 131
{\PTglyphid{U2-02_0511}}
% 132
{\PTglyphid{U2-02_0512}}
% 133
{\PTglyphid{U2-02_0513}}
% 134
{\PTglyphid{U2-02_0514}}
% 135
{\PTglyphid{U2-02_0515}}
% 136
{\PTglyphid{U2-02_0516}}
% 137
{\PTglyphid{U2-02_0517}}
% 138
{\PTglyphid{U2-02_0518}}
% 139
{\PTglyphid{U2-02_0519}}
% 140
{\PTglyphid{U2-02_0520}}
% 141
{\PTglyphid{U2-02_0521}}
% 142
{\PTglyphid{U2-02_0522}}
% 143
{\PTglyphid{U2-02_0523}}
% 144
{\PTglyphid{U2-02_0524}}
% 145
{\PTglyphid{U2-02_0525}}
% 146
{\PTglyphid{U2-02_0526}}
% 147
{\PTglyphid{U2-02_0527}}
% 148
{\PTglyphid{U2-02_0528}}
% 149
{\PTglyphid{U2-02_0529}}
% 150
{\PTglyphid{U2-02_0530}}
% 151
{\PTglyphid{U2-02_0531}}
% 152
{\PTglyphid{U2-02_0532}}
% 153
{\PTglyphid{U2-02_0533}}
//
\endgl \xe
%%% Local Variables:
%%% mode: latex
%%% TeX-engine: luatex
%%% TeX-master: shared
%%% End:

% //
%\endgl \xe


 \newpage
 
%%%%%%%%%%%%%%%%%%%%%%%%%%%%%%%%%%%%%%%%%%%%%%%%%%%%%%%%%%%%%%%%%%%%%%%%%%%%%%%
% from meta.csv
% 54,Ungler2-03_PT05_241.djvu,Ungler2,03,05,241
% 
%%%%%%%%%%%%%%%%%%%%%%%%%%%%%%%%%%%%%%%%%%%%%%%%%%%%%%%%%%%%%%%%%%%%%%%%%%%%%%%

 
 % from dsed4test:
% Ungler2-03_PT05_241_4dsed.txt:Note "3. Pismo tekstowe, rotunda M⁴⁹, Stopień 20 ww. = 62/63 mm. — Tabl. 241."
% Ungler2-03_PT05_241_4dsed.txt:Note1 "Character set table prepared by Maria Błońska"

 \pismoPL{Florian Ungler druga drukarnia 3. Pismo tekstowe, rotunda M⁴⁹, Stopień 20 ww. = 62/63 mm. — Tabl. 241.}
  
 \pismoEN{Florian Ungler  second house 3. Rotunda text font, typeface M⁴⁹. Type size 20 lines = 62/63 mm. — Plate 241.}

\plate{241}{V}{1964}

The plate prepared by Henryk Bułhak.\\
The font table prepared by Henryk Bułhak and Maria Błońska.

\bigskip

\exampleBib{V:37}

\bigskip \exampleDesc{LECTURA super titulo de regulis iuris libro sexto. Kraków, Florian Ungłer, 1524. 8⁰.}


\medskip
\examplePage{\textit{Karta A₂b}}

  \bigskip
  \exampleLib{Biblioteka Jagiellońska. Kraków.}

\bigskip \exampleRef{\textit{Estreicher XXI 138, XXIII D. XXVI. Wierzbowski 1010. Piekarski Kórn. 831.}}
% \bigskip

% Pismo 2: nagłówek. — Pismo 3: tekst i zestaw. — Rubryki B, y: z pismem 3. — Cyfry 3: z pismem 3. 


% \examplePL{}


% \exampleEN{}


\bigskip

\fontID{U2-03}{54}

\fontstat{154}

% \exdisplay \bg \gla
 \exdisplay \bg \gla
% 1
{\PTglyph{5}{t54_l01g01.png}}
% 2
{\PTglyph{5}{t54_l01g02.png}}
% 3
{\PTglyph{5}{t54_l01g03.png}}
% 4
{\PTglyph{5}{t54_l01g04.png}}
% 5
{\PTglyph{5}{t54_l01g05.png}}
% 6
{\PTglyph{5}{t54_l01g06.png}}
% 7
{\PTglyph{5}{t54_l01g07.png}}
% 8
{\PTglyph{5}{t54_l01g08.png}}
% 9
{\PTglyph{5}{t54_l01g09.png}}
% 10
{\PTglyph{5}{t54_l01g10.png}}
% 11
{\PTglyph{5}{t54_l01g11.png}}
% 12
{\PTglyph{5}{t54_l01g12.png}}
% 13
{\PTglyph{5}{t54_l01g13.png}}
% 14
{\PTglyph{5}{t54_l01g14.png}}
% 15
{\PTglyph{5}{t54_l01g15.png}}
% 16
{\PTglyph{5}{t54_l01g16.png}}
% 17
{\PTglyph{5}{t54_l01g17.png}}
% 18
{\PTglyph{5}{t54_l01g18.png}}
% 19
{\PTglyph{5}{t54_l01g19.png}}
% 20
{\PTglyph{5}{t54_l01g20.png}}
% 21
{\PTglyph{5}{t54_l01g21.png}}
% 22
{\PTglyph{5}{t54_l01g22.png}}
% 23
{\PTglyph{5}{t54_l01g23.png}}
% 24
{\PTglyph{5}{t54_l01g24.png}}
% 25
{\PTglyph{5}{t54_l01g25.png}}
% 26
{\PTglyph{5}{t54_l01g26.png}}
% 27
{\PTglyph{5}{t54_l01g27.png}}
% 28
{\PTglyph{5}{t54_l02g01.png}}
% 29
{\PTglyph{5}{t54_l02g02.png}}
% 30
{\PTglyph{5}{t54_l02g03.png}}
% 31
{\PTglyph{5}{t54_l02g04.png}}
% 32
{\PTglyph{5}{t54_l02g05.png}}
% 33
{\PTglyph{5}{t54_l02g06.png}}
% 34
{\PTglyph{5}{t54_l02g07.png}}
% 35
{\PTglyph{5}{t54_l02g08.png}}
% 36
{\PTglyph{5}{t54_l02g09.png}}
% 37
{\PTglyph{5}{t54_l02g10.png}}
% 38
{\PTglyph{5}{t54_l02g11.png}}
% 39
{\PTglyph{5}{t54_l02g12.png}}
% 40
{\PTglyph{5}{t54_l02g13.png}}
% 41
{\PTglyph{5}{t54_l02g14.png}}
% 42
{\PTglyph{5}{t54_l02g15.png}}
% 43
{\PTglyph{5}{t54_l02g16.png}}
% 44
{\PTglyph{5}{t54_l02g17.png}}
% 45
{\PTglyph{5}{t54_l02g18.png}}
% 46
{\PTglyph{5}{t54_l02g19.png}}
% 47
{\PTglyph{5}{t54_l02g20.png}}
% 48
{\PTglyph{5}{t54_l02g21.png}}
% 49
{\PTglyph{5}{t54_l02g22.png}}
% 50
{\PTglyph{5}{t54_l02g23.png}}
% 51
{\PTglyph{5}{t54_l02g24.png}}
% 52
{\PTglyph{5}{t54_l02g25.png}}
% 53
{\PTglyph{5}{t54_l02g26.png}}
% 54
{\PTglyph{5}{t54_l02g27.png}}
% 55
{\PTglyph{5}{t54_l02g28.png}}
% 56
{\PTglyph{5}{t54_l02g29.png}}
% 57
{\PTglyph{5}{t54_l02g30.png}}
% 58
{\PTglyph{5}{t54_l02g31.png}}
% 59
{\PTglyph{5}{t54_l02g32.png}}
% 60
{\PTglyph{5}{t54_l02g33.png}}
% 61
{\PTglyph{5}{t54_l02g34.png}}
% 62
{\PTglyph{5}{t54_l02g35.png}}
% 63
{\PTglyph{5}{t54_l02g36.png}}
% 64
{\PTglyph{5}{t54_l02g37.png}}
% 65
{\PTglyph{5}{t54_l02g38.png}}
% 66
{\PTglyph{5}{t54_l02g39.png}}
% 67
{\PTglyph{5}{t54_l02g40.png}}
% 68
{\PTglyph{5}{t54_l03g01.png}}
% 69
{\PTglyph{5}{t54_l03g02.png}}
% 70
{\PTglyph{5}{t54_l03g03.png}}
% 71
{\PTglyph{5}{t54_l03g04.png}}
% 72
{\PTglyph{5}{t54_l03g05.png}}
% 73
{\PTglyph{5}{t54_l03g06.png}}
% 74
{\PTglyph{5}{t54_l03g07.png}}
% 75
{\PTglyph{5}{t54_l03g08.png}}
% 76
{\PTglyph{5}{t54_l03g09.png}}
% 77
{\PTglyph{5}{t54_l03g10.png}}
% 78
{\PTglyph{5}{t54_l03g11.png}}
% 79
{\PTglyph{5}{t54_l03g12.png}}
% 80
{\PTglyph{5}{t54_l03g13.png}}
% 81
{\PTglyph{5}{t54_l03g14.png}}
% 82
{\PTglyph{5}{t54_l03g15.png}}
% 83
{\PTglyph{5}{t54_l03g16.png}}
% 84
{\PTglyph{5}{t54_l03g17.png}}
% 85
{\PTglyph{5}{t54_l03g18.png}}
% 86
{\PTglyph{5}{t54_l03g19.png}}
% 87
{\PTglyph{5}{t54_l03g20.png}}
% 88
{\PTglyph{5}{t54_l03g21.png}}
% 89
{\PTglyph{5}{t54_l03g22.png}}
% 90
{\PTglyph{5}{t54_l03g23.png}}
% 91
{\PTglyph{5}{t54_l03g24.png}}
% 92
{\PTglyph{5}{t54_l03g25.png}}
% 93
{\PTglyph{5}{t54_l03g26.png}}
% 94
{\PTglyph{5}{t54_l03g27.png}}
% 95
{\PTglyph{5}{t54_l03g28.png}}
% 96
{\PTglyph{5}{t54_l03g29.png}}
% 97
{\PTglyph{5}{t54_l03g30.png}}
% 98
{\PTglyph{5}{t54_l03g31.png}}
% 99
{\PTglyph{5}{t54_l03g32.png}}
% 100
{\PTglyph{5}{t54_l03g33.png}}
% 101
{\PTglyph{5}{t54_l03g34.png}}
% 102
{\PTglyph{5}{t54_l03g35.png}}
% 103
{\PTglyph{5}{t54_l03g36.png}}
% 104
{\PTglyph{5}{t54_l03g37.png}}
% 105
{\PTglyph{5}{t54_l04g01.png}}
% 106
{\PTglyph{5}{t54_l04g02.png}}
% 107
{\PTglyph{5}{t54_l04g03.png}}
% 108
{\PTglyph{5}{t54_l04g04.png}}
% 109
{\PTglyph{5}{t54_l04g05.png}}
% 110
{\PTglyph{5}{t54_l04g06.png}}
% 111
{\PTglyph{5}{t54_l04g07.png}}
% 112
{\PTglyph{5}{t54_l04g08.png}}
% 113
{\PTglyph{5}{t54_l04g09.png}}
% 114
{\PTglyph{5}{t54_l04g10.png}}
% 115
{\PTglyph{5}{t54_l04g11.png}}
% 116
{\PTglyph{5}{t54_l04g12.png}}
% 117
{\PTglyph{5}{t54_l04g13.png}}
% 118
{\PTglyph{5}{t54_l04g14.png}}
% 119
{\PTglyph{5}{t54_l04g15.png}}
% 120
{\PTglyph{5}{t54_l04g16.png}}
% 121
{\PTglyph{5}{t54_l04g17.png}}
% 122
{\PTglyph{5}{t54_l04g18.png}}
% 123
{\PTglyph{5}{t54_l04g19.png}}
% 124
{\PTglyph{5}{t54_l04g20.png}}
% 125
{\PTglyph{5}{t54_l04g21.png}}
% 126
{\PTglyph{5}{t54_l04g22.png}}
% 127
{\PTglyph{5}{t54_l04g23.png}}
% 128
{\PTglyph{5}{t54_l04g24.png}}
% 129
{\PTglyph{5}{t54_l04g25.png}}
% 130
{\PTglyph{5}{t54_l04g26.png}}
% 131
{\PTglyph{5}{t54_l04g27.png}}
% 132
{\PTglyph{5}{t54_l04g28.png}}
% 133
{\PTglyph{5}{t54_l04g29.png}}
% 134
{\PTglyph{5}{t54_l04g30.png}}
% 135
{\PTglyph{5}{t54_l04g31.png}}
% 136
{\PTglyph{5}{t54_l04g32.png}}
% 137
{\PTglyph{5}{t54_l04g33.png}}
% 138
{\PTglyph{5}{t54_l04g34.png}}
% 139
{\PTglyph{5}{t54_l04g35.png}}
% 140
{\PTglyph{5}{t54_l04g36.png}}
% 141
{\PTglyph{5}{t54_l04g37.png}}
% 142
{\PTglyph{5}{t54_l04g38.png}}
% 143
{\PTglyph{5}{t54_l04g39.png}}
% 144
{\PTglyph{5}{t54_l05g01.png}}
% 145
{\PTglyph{5}{t54_l05g02.png}}
% 146
{\PTglyph{5}{t54_l05g03.png}}
% 147
{\PTglyph{5}{t54_l05g04.png}}
% 148
{\PTglyph{5}{t54_l05g05.png}}
% 149
{\PTglyph{5}{t54_l05g06.png}}
% 150
{\PTglyph{5}{t54_l05g07.png}}
% 151
{\PTglyph{5}{t54_l05g08.png}}
% 152
{\PTglyph{5}{t54_l05g09.png}}
% 153
{\PTglyph{5}{t54_l05g10.png}}
% 154
{\PTglyph{5}{t54_l05g11.png}}
//
%%% Local Variables:
%%% mode: latex
%%% TeX-engine: luatex
%%% TeX-master: shared
%%% End:

%//
%\glpismo%
 \glpismo
% 1
{\PTglyphid{U2-03_0101}}
% 2
{\PTglyphid{U2-03_0102}}
% 3
{\PTglyphid{U2-03_0103}}
% 4
{\PTglyphid{U2-03_0104}}
% 5
{\PTglyphid{U2-03_0105}}
% 6
{\PTglyphid{U2-03_0106}}
% 7
{\PTglyphid{U2-03_0107}}
% 8
{\PTglyphid{U2-03_0108}}
% 9
{\PTglyphid{U2-03_0109}}
% 10
{\PTglyphid{U2-03_0110}}
% 11
{\PTglyphid{U2-03_0111}}
% 12
{\PTglyphid{U2-03_0112}}
% 13
{\PTglyphid{U2-03_0113}}
% 14
{\PTglyphid{U2-03_0114}}
% 15
{\PTglyphid{U2-03_0115}}
% 16
{\PTglyphid{U2-03_0116}}
% 17
{\PTglyphid{U2-03_0117}}
% 18
{\PTglyphid{U2-03_0118}}
% 19
{\PTglyphid{U2-03_0119}}
% 20
{\PTglyphid{U2-03_0120}}
% 21
{\PTglyphid{U2-03_0121}}
% 22
{\PTglyphid{U2-03_0122}}
% 23
{\PTglyphid{U2-03_0123}}
% 24
{\PTglyphid{U2-03_0124}}
% 25
{\PTglyphid{U2-03_0125}}
% 26
{\PTglyphid{U2-03_0126}}
% 27
{\PTglyphid{U2-03_0127}}
% 28
{\PTglyphid{U2-03_0201}}
% 29
{\PTglyphid{U2-03_0202}}
% 30
{\PTglyphid{U2-03_0203}}
% 31
{\PTglyphid{U2-03_0204}}
% 32
{\PTglyphid{U2-03_0205}}
% 33
{\PTglyphid{U2-03_0206}}
% 34
{\PTglyphid{U2-03_0207}}
% 35
{\PTglyphid{U2-03_0208}}
% 36
{\PTglyphid{U2-03_0209}}
% 37
{\PTglyphid{U2-03_0210}}
% 38
{\PTglyphid{U2-03_0211}}
% 39
{\PTglyphid{U2-03_0212}}
% 40
{\PTglyphid{U2-03_0213}}
% 41
{\PTglyphid{U2-03_0214}}
% 42
{\PTglyphid{U2-03_0215}}
% 43
{\PTglyphid{U2-03_0216}}
% 44
{\PTglyphid{U2-03_0217}}
% 45
{\PTglyphid{U2-03_0218}}
% 46
{\PTglyphid{U2-03_0219}}
% 47
{\PTglyphid{U2-03_0220}}
% 48
{\PTglyphid{U2-03_0221}}
% 49
{\PTglyphid{U2-03_0222}}
% 50
{\PTglyphid{U2-03_0223}}
% 51
{\PTglyphid{U2-03_0224}}
% 52
{\PTglyphid{U2-03_0225}}
% 53
{\PTglyphid{U2-03_0226}}
% 54
{\PTglyphid{U2-03_0227}}
% 55
{\PTglyphid{U2-03_0228}}
% 56
{\PTglyphid{U2-03_0229}}
% 57
{\PTglyphid{U2-03_0230}}
% 58
{\PTglyphid{U2-03_0231}}
% 59
{\PTglyphid{U2-03_0232}}
% 60
{\PTglyphid{U2-03_0233}}
% 61
{\PTglyphid{U2-03_0234}}
% 62
{\PTglyphid{U2-03_0235}}
% 63
{\PTglyphid{U2-03_0236}}
% 64
{\PTglyphid{U2-03_0237}}
% 65
{\PTglyphid{U2-03_0238}}
% 66
{\PTglyphid{U2-03_0239}}
% 67
{\PTglyphid{U2-03_0240}}
% 68
{\PTglyphid{U2-03_0301}}
% 69
{\PTglyphid{U2-03_0302}}
% 70
{\PTglyphid{U2-03_0303}}
% 71
{\PTglyphid{U2-03_0304}}
% 72
{\PTglyphid{U2-03_0305}}
% 73
{\PTglyphid{U2-03_0306}}
% 74
{\PTglyphid{U2-03_0307}}
% 75
{\PTglyphid{U2-03_0308}}
% 76
{\PTglyphid{U2-03_0309}}
% 77
{\PTglyphid{U2-03_0310}}
% 78
{\PTglyphid{U2-03_0311}}
% 79
{\PTglyphid{U2-03_0312}}
% 80
{\PTglyphid{U2-03_0313}}
% 81
{\PTglyphid{U2-03_0314}}
% 82
{\PTglyphid{U2-03_0315}}
% 83
{\PTglyphid{U2-03_0316}}
% 84
{\PTglyphid{U2-03_0317}}
% 85
{\PTglyphid{U2-03_0318}}
% 86
{\PTglyphid{U2-03_0319}}
% 87
{\PTglyphid{U2-03_0320}}
% 88
{\PTglyphid{U2-03_0321}}
% 89
{\PTglyphid{U2-03_0322}}
% 90
{\PTglyphid{U2-03_0323}}
% 91
{\PTglyphid{U2-03_0324}}
% 92
{\PTglyphid{U2-03_0325}}
% 93
{\PTglyphid{U2-03_0326}}
% 94
{\PTglyphid{U2-03_0327}}
% 95
{\PTglyphid{U2-03_0328}}
% 96
{\PTglyphid{U2-03_0329}}
% 97
{\PTglyphid{U2-03_0330}}
% 98
{\PTglyphid{U2-03_0331}}
% 99
{\PTglyphid{U2-03_0332}}
% 100
{\PTglyphid{U2-03_0333}}
% 101
{\PTglyphid{U2-03_0334}}
% 102
{\PTglyphid{U2-03_0335}}
% 103
{\PTglyphid{U2-03_0336}}
% 104
{\PTglyphid{U2-03_0337}}
% 105
{\PTglyphid{U2-03_0401}}
% 106
{\PTglyphid{U2-03_0402}}
% 107
{\PTglyphid{U2-03_0403}}
% 108
{\PTglyphid{U2-03_0404}}
% 109
{\PTglyphid{U2-03_0405}}
% 110
{\PTglyphid{U2-03_0406}}
% 111
{\PTglyphid{U2-03_0407}}
% 112
{\PTglyphid{U2-03_0408}}
% 113
{\PTglyphid{U2-03_0409}}
% 114
{\PTglyphid{U2-03_0410}}
% 115
{\PTglyphid{U2-03_0411}}
% 116
{\PTglyphid{U2-03_0412}}
% 117
{\PTglyphid{U2-03_0413}}
% 118
{\PTglyphid{U2-03_0414}}
% 119
{\PTglyphid{U2-03_0415}}
% 120
{\PTglyphid{U2-03_0416}}
% 121
{\PTglyphid{U2-03_0417}}
% 122
{\PTglyphid{U2-03_0418}}
% 123
{\PTglyphid{U2-03_0419}}
% 124
{\PTglyphid{U2-03_0420}}
% 125
{\PTglyphid{U2-03_0421}}
% 126
{\PTglyphid{U2-03_0422}}
% 127
{\PTglyphid{U2-03_0423}}
% 128
{\PTglyphid{U2-03_0424}}
% 129
{\PTglyphid{U2-03_0425}}
% 130
{\PTglyphid{U2-03_0426}}
% 131
{\PTglyphid{U2-03_0427}}
% 132
{\PTglyphid{U2-03_0428}}
% 133
{\PTglyphid{U2-03_0429}}
% 134
{\PTglyphid{U2-03_0430}}
% 135
{\PTglyphid{U2-03_0431}}
% 136
{\PTglyphid{U2-03_0432}}
% 137
{\PTglyphid{U2-03_0433}}
% 138
{\PTglyphid{U2-03_0434}}
% 139
{\PTglyphid{U2-03_0435}}
% 140
{\PTglyphid{U2-03_0436}}
% 141
{\PTglyphid{U2-03_0437}}
% 142
{\PTglyphid{U2-03_0438}}
% 143
{\PTglyphid{U2-03_0439}}
% 144
{\PTglyphid{U2-03_0501}}
% 145
{\PTglyphid{U2-03_0502}}
% 146
{\PTglyphid{U2-03_0503}}
% 147
{\PTglyphid{U2-03_0504}}
% 148
{\PTglyphid{U2-03_0505}}
% 149
{\PTglyphid{U2-03_0506}}
% 150
{\PTglyphid{U2-03_0507}}
% 151
{\PTglyphid{U2-03_0508}}
% 152
{\PTglyphid{U2-03_0509}}
% 153
{\PTglyphid{U2-03_0510}}
% 154
{\PTglyphid{U2-03_0511}}
//
\endgl \xe
%%% Local Variables:
%%% mode: latex
%%% TeX-engine: luatex
%%% TeX-master: shared
%%% End:

% //
%\endgl \xe



 \newpage
 
%%%%%%%%%%%%%%%%%%%%%%%%%%%%%%%%%%%%%%%%%%%%%%%%%%%%%%%%%%%%%%%%%%%%%%%%%%%%%%%
% from meta.csv
% 55,Ungler2-04_PT05_242.djvu,Ungler2,04,05,242
% 
%%%%%%%%%%%%%%%%%%%%%%%%%%%%%%%%%%%%%%%%%%%%%%%%%%%%%%%%%%%%%%%%%%%%%%%%%%%%%%%

 
 % from dsed4test:
% Ungler2-04_PT05_242_4dsed.txt:Note "4. Pismo tekstowe i nagłówkowe, szwabacha M⁸¹. Stopień 20 ww. = = 103 mm. — Tabl. 242."
% Ungler2-04_PT05_242_4dsed.txt:Note1 "Character set table prepared by Maria Błońska"

 \pismoPL{Florian Ungler druga drukarnia 4. Pismo tekstowe i
   nagłówkowe, szwabacha M⁸¹. Stopień 20 ww. = = 103 mm. — Tabl. 242.}
  
 \pismoEN{Florian Ungler second house 4. Schwabacher header and text
   font, typeface M⁸¹. Type size 20 lines = 103 mm. — Plate 242.}

\plate{242}{V}{1964}

The plate prepared by Henryk Bułhak.\\
The font table prepared by Henryk Bułhak and Maria Błońska.

\bigskip

\exampleBib{V:11}

\bigskip \exampleDesc{PSEUDO-BONAVENTURA: [Meditationes vitae Christi. Trad. polon. Balthasar Opeć]. Żywot Pana Jezu Krysta.
Kraków, Florian Ungler i Jan Sandecki nakładem Jana Hallera, [po 3 V] 1522. 4⁰.}


\medskip
\examplePage{\textit{Karta 26 b}}

  \bigskip
  \exampleLib{Biblioteka Czartoryskich. Kraków.}

\bigskip \exampleRef{\textit{Estreicher XIII 252. Wierzbowski 984.}}
% \bigskip
% https://lib.amu.edu.pl/produkt/baltazar-opec-zywot-pana-jezu-krysta-stworzyciela-i-zbawiciela-rodzaju-ludskiego-wedle-ewanjelist-swietych-z-rozmyslanim-naboznym-doktorow-pisma-swietego-krotko-zebrany-krakow-florian-ungler-i-jan/
% https://psp.amu.edu.pl/?type=book&id_asset=986

% Drzeworyt 296: ozdobnik.


% \examplePL{}


% \exampleEN{}


\bigskip

\fontID{U2-04}{55}

\fontstat{102}

% \exdisplay \bg \gla
 \exdisplay \bg \gla
% 1
{\PTglyph{5}{t55_l01g01.png}}
% 2
{\PTglyph{5}{t55_l01g02.png}}
% 3
{\PTglyph{5}{t55_l01g03.png}}
% 4
{\PTglyph{5}{t55_l01g04.png}}
% 5
{\PTglyph{5}{t55_l01g05.png}}
% 6
{\PTglyph{5}{t55_l01g06.png}}
% 7
{\PTglyph{5}{t55_l01g07.png}}
% 8
{\PTglyph{5}{t55_l01g08.png}}
% 9
{\PTglyph{5}{t55_l01g09.png}}
% 10
{\PTglyph{5}{t55_l01g10.png}}
% 11
{\PTglyph{5}{t55_l01g11.png}}
% 12
{\PTglyph{5}{t55_l01g12.png}}
% 13
{\PTglyph{5}{t55_l01g13.png}}
% 14
{\PTglyph{5}{t55_l01g14.png}}
% 15
{\PTglyph{5}{t55_l01g15.png}}
% 16
{\PTglyph{5}{t55_l01g16.png}}
% 17
{\PTglyph{5}{t55_l01g17.png}}
% 18
{\PTglyph{5}{t55_l01g18.png}}
% 19
{\PTglyph{5}{t55_l01g19.png}}
% 20
{\PTglyph{5}{t55_l01g20.png}}
% 21
{\PTglyph{5}{t55_l01g21.png}}
% 22
{\PTglyph{5}{t55_l01g22.png}}
% 23
{\PTglyph{5}{t55_l01g23.png}}
% 24
{\PTglyph{5}{t55_l01g24.png}}
% 25
{\PTglyph{5}{t55_l02g01.png}}
% 26
{\PTglyph{5}{t55_l02g02.png}}
% 27
{\PTglyph{5}{t55_l02g03.png}}
% 28
{\PTglyph{5}{t55_l02g04.png}}
% 29
{\PTglyph{5}{t55_l02g05.png}}
% 30
{\PTglyph{5}{t55_l02g06.png}}
% 31
{\PTglyph{5}{t55_l02g07.png}}
% 32
{\PTglyph{5}{t55_l02g08.png}}
% 33
{\PTglyph{5}{t55_l02g09.png}}
% 34
{\PTglyph{5}{t55_l02g10.png}}
% 35
{\PTglyph{5}{t55_l02g11.png}}
% 36
{\PTglyph{5}{t55_l02g12.png}}
% 37
{\PTglyph{5}{t55_l02g13.png}}
% 38
{\PTglyph{5}{t55_l02g14.png}}
% 39
{\PTglyph{5}{t55_l02g15.png}}
% 40
{\PTglyph{5}{t55_l02g16.png}}
% 41
{\PTglyph{5}{t55_l02g17.png}}
% 42
{\PTglyph{5}{t55_l02g18.png}}
% 43
{\PTglyph{5}{t55_l02g19.png}}
% 44
{\PTglyph{5}{t55_l02g20.png}}
% 45
{\PTglyph{5}{t55_l02g21.png}}
% 46
{\PTglyph{5}{t55_l02g22.png}}
% 47
{\PTglyph{5}{t55_l02g23.png}}
% 48
{\PTglyph{5}{t55_l02g24.png}}
% 49
{\PTglyph{5}{t55_l02g25.png}}
% 50
{\PTglyph{5}{t55_l02g26.png}}
% 51
{\PTglyph{5}{t55_l02g27.png}}
% 52
{\PTglyph{5}{t55_l02g28.png}}
% 53
{\PTglyph{5}{t55_l02g29.png}}
% 54
{\PTglyph{5}{t55_l02g30.png}}
% 55
{\PTglyph{5}{t55_l02g31.png}}
% 56
{\PTglyph{5}{t55_l02g32.png}}
% 57
{\PTglyph{5}{t55_l02g33.png}}
% 58
{\PTglyph{5}{t55_l02g34.png}}
% 59
{\PTglyph{5}{t55_l02g35.png}}
% 60
{\PTglyph{5}{t55_l02g36.png}}
% 61
{\PTglyph{5}{t55_l02g37.png}}
% 62
{\PTglyph{5}{t55_l02g38.png}}
% 63
{\PTglyph{5}{t55_l02g39.png}}
% 64
{\PTglyph{5}{t55_l02g40.png}}
% 65
{\PTglyph{5}{t55_l03g01.png}}
% 66
{\PTglyph{5}{t55_l03g02.png}}
% 67
{\PTglyph{5}{t55_l03g03.png}}
% 68
{\PTglyph{5}{t55_l03g04.png}}
% 69
{\PTglyph{5}{t55_l03g05.png}}
% 70
{\PTglyph{5}{t55_l03g06.png}}
% 71
{\PTglyph{5}{t55_l03g07.png}}
% 72
{\PTglyph{5}{t55_l03g08.png}}
% 73
{\PTglyph{5}{t55_l03g09.png}}
% 74
{\PTglyph{5}{t55_l03g10.png}}
% 75
{\PTglyph{5}{t55_l03g11.png}}
% 76
{\PTglyph{5}{t55_l03g12.png}}
% 77
{\PTglyph{5}{t55_l03g13.png}}
% 78
{\PTglyph{5}{t55_l03g14.png}}
% 79
{\PTglyph{5}{t55_l03g15.png}}
% 80
{\PTglyph{5}{t55_l03g16.png}}
% 81
{\PTglyph{5}{t55_l03g17.png}}
% 82
{\PTglyph{5}{t55_l03g18.png}}
% 83
{\PTglyph{5}{t55_l03g19.png}}
% 84
{\PTglyph{5}{t55_l03g20.png}}
% 85
{\PTglyph{5}{t55_l03g21.png}}
% 86
{\PTglyph{5}{t55_l03g22.png}}
% 87
{\PTglyph{5}{t55_l03g23.png}}
% 88
{\PTglyph{5}{t55_l03g24.png}}
% 89
{\PTglyph{5}{t55_l03g25.png}}
% 90
{\PTglyph{5}{t55_l03g26.png}}
% 91
{\PTglyph{5}{t55_l03g27.png}}
% 92
{\PTglyph{5}{t55_l03g28.png}}
% 93
{\PTglyph{5}{t55_l04g01.png}}
% 94
{\PTglyph{5}{t55_l04g02.png}}
% 95
{\PTglyph{5}{t55_l04g03.png}}
% 96
{\PTglyph{5}{t55_l04g04.png}}
% 97
{\PTglyph{5}{t55_l04g05.png}}
% 98
{\PTglyph{5}{t55_l04g06.png}}
% 99
{\PTglyph{5}{t55_l04g07.png}}
% 100
{\PTglyph{5}{t55_l04g08.png}}
% 101
{\PTglyph{5}{t55_l04g09.png}}
% 102
{\PTglyph{5}{t55_l04g10.png}}
//
%%% Local Variables:
%%% mode: latex
%%% TeX-engine: luatex
%%% TeX-master: shared
%%% End:

%//
%\glpismo%
 \glpismo
% 1
{\PTglyphid{U2-04_0101}}
% 2
{\PTglyphid{U2-04_0102}}
% 3
{\PTglyphid{U2-04_0103}}
% 4
{\PTglyphid{U2-04_0104}}
% 5
{\PTglyphid{U2-04_0105}}
% 6
{\PTglyphid{U2-04_0106}}
% 7
{\PTglyphid{U2-04_0107}}
% 8
{\PTglyphid{U2-04_0108}}
% 9
{\PTglyphid{U2-04_0109}}
% 10
{\PTglyphid{U2-04_0110}}
% 11
{\PTglyphid{U2-04_0111}}
% 12
{\PTglyphid{U2-04_0112}}
% 13
{\PTglyphid{U2-04_0113}}
% 14
{\PTglyphid{U2-04_0114}}
% 15
{\PTglyphid{U2-04_0115}}
% 16
{\PTglyphid{U2-04_0116}}
% 17
{\PTglyphid{U2-04_0117}}
% 18
{\PTglyphid{U2-04_0118}}
% 19
{\PTglyphid{U2-04_0119}}
% 20
{\PTglyphid{U2-04_0120}}
% 21
{\PTglyphid{U2-04_0121}}
% 22
{\PTglyphid{U2-04_0122}}
% 23
{\PTglyphid{U2-04_0123}}
% 24
{\PTglyphid{U2-04_0124}}
% 25
{\PTglyphid{U2-04_0201}}
% 26
{\PTglyphid{U2-04_0202}}
% 27
{\PTglyphid{U2-04_0203}}
% 28
{\PTglyphid{U2-04_0204}}
% 29
{\PTglyphid{U2-04_0205}}
% 30
{\PTglyphid{U2-04_0206}}
% 31
{\PTglyphid{U2-04_0207}}
% 32
{\PTglyphid{U2-04_0208}}
% 33
{\PTglyphid{U2-04_0209}}
% 34
{\PTglyphid{U2-04_0210}}
% 35
{\PTglyphid{U2-04_0211}}
% 36
{\PTglyphid{U2-04_0212}}
% 37
{\PTglyphid{U2-04_0213}}
% 38
{\PTglyphid{U2-04_0214}}
% 39
{\PTglyphid{U2-04_0215}}
% 40
{\PTglyphid{U2-04_0216}}
% 41
{\PTglyphid{U2-04_0217}}
% 42
{\PTglyphid{U2-04_0218}}
% 43
{\PTglyphid{U2-04_0219}}
% 44
{\PTglyphid{U2-04_0220}}
% 45
{\PTglyphid{U2-04_0221}}
% 46
{\PTglyphid{U2-04_0222}}
% 47
{\PTglyphid{U2-04_0223}}
% 48
{\PTglyphid{U2-04_0224}}
% 49
{\PTglyphid{U2-04_0225}}
% 50
{\PTglyphid{U2-04_0226}}
% 51
{\PTglyphid{U2-04_0227}}
% 52
{\PTglyphid{U2-04_0228}}
% 53
{\PTglyphid{U2-04_0229}}
% 54
{\PTglyphid{U2-04_0230}}
% 55
{\PTglyphid{U2-04_0231}}
% 56
{\PTglyphid{U2-04_0232}}
% 57
{\PTglyphid{U2-04_0233}}
% 58
{\PTglyphid{U2-04_0234}}
% 59
{\PTglyphid{U2-04_0235}}
% 60
{\PTglyphid{U2-04_0236}}
% 61
{\PTglyphid{U2-04_0237}}
% 62
{\PTglyphid{U2-04_0238}}
% 63
{\PTglyphid{U2-04_0239}}
% 64
{\PTglyphid{U2-04_0240}}
% 65
{\PTglyphid{U2-04_0301}}
% 66
{\PTglyphid{U2-04_0302}}
% 67
{\PTglyphid{U2-04_0303}}
% 68
{\PTglyphid{U2-04_0304}}
% 69
{\PTglyphid{U2-04_0305}}
% 70
{\PTglyphid{U2-04_0306}}
% 71
{\PTglyphid{U2-04_0307}}
% 72
{\PTglyphid{U2-04_0308}}
% 73
{\PTglyphid{U2-04_0309}}
% 74
{\PTglyphid{U2-04_0310}}
% 75
{\PTglyphid{U2-04_0311}}
% 76
{\PTglyphid{U2-04_0312}}
% 77
{\PTglyphid{U2-04_0313}}
% 78
{\PTglyphid{U2-04_0314}}
% 79
{\PTglyphid{U2-04_0315}}
% 80
{\PTglyphid{U2-04_0316}}
% 81
{\PTglyphid{U2-04_0317}}
% 82
{\PTglyphid{U2-04_0318}}
% 83
{\PTglyphid{U2-04_0319}}
% 84
{\PTglyphid{U2-04_0320}}
% 85
{\PTglyphid{U2-04_0321}}
% 86
{\PTglyphid{U2-04_0322}}
% 87
{\PTglyphid{U2-04_0323}}
% 88
{\PTglyphid{U2-04_0324}}
% 89
{\PTglyphid{U2-04_0325}}
% 90
{\PTglyphid{U2-04_0326}}
% 91
{\PTglyphid{U2-04_0327}}
% 92
{\PTglyphid{U2-04_0328}}
% 93
{\PTglyphid{U2-04_0401}}
% 94
{\PTglyphid{U2-04_0402}}
% 95
{\PTglyphid{U2-04_0403}}
% 96
{\PTglyphid{U2-04_0404}}
% 97
{\PTglyphid{U2-04_0405}}
% 98
{\PTglyphid{U2-04_0406}}
% 99
{\PTglyphid{U2-04_0407}}
% 100
{\PTglyphid{U2-04_0408}}
% 101
{\PTglyphid{U2-04_0409}}
% 102
{\PTglyphid{U2-04_0410}}
//
\endgl \xe
%%% Local Variables:
%%% mode: latex
%%% TeX-engine: luatex
%%% TeX-master: shared
%%% End:

% //
%\endgl \xe



 \newpage
 
%%%%%%%%%%%%%%%%%%%%%%%%%%%%%%%%%%%%%%%%%%%%%%%%%%%%%%%%%%%%%%%%%%%%%%%%%%%%%%%
% from meta.csv
% 56,Ungler2-05_PT05_243.djvu,Ungler2,05,05,243
% 
%%%%%%%%%%%%%%%%%%%%%%%%%%%%%%%%%%%%%%%%%%%%%%%%%%%%%%%%%%%%%%%%%%%%%%%%%%%%%%%

 
% from dsed4test:
% Ungler2-05_PT05_243_4dsed.txt:Note "5. Pismo tekstowe, antykwa Q/u (C). Stopień 20 ww. =94—95 mm. — Tabl. 243."
% Ungler2-05_PT05_243_4dsed.txt:Note1 "Character set table prepared by Anna Wolińska"

 \pismoPL{Florian Ungler druga drukarnia 5. Pismo tekstowe, antykwa Q/u (C). Stopień 20 ww. = 94—95 mm. — Tabl. 243.}
  
 \pismoEN{Florian Ungler second house 3. Roman text font, typeface
   Q/u (C). Type size 20 lines = 94—95 mm. — Plate 243.}

\plate{243}{V}{1964}

The plate prepared by Henryk Bułhak.\\
The font table prepared by Henryk Bułhak and Anna Wolińska.

\bigskip

\exampleBib{V:52}

\bigskip \exampleDesc{MATTHIAS DE MIECHÓW: Contra saevam pestem regimen.
Kraków, Florian Ungler, [po 3 X] 1527. 8⁰.}


\medskip
\examplePage{\textit{Karta C₄b}}

  \bigskip
  \exampleLib{Archiwum Archidiecezjalne. Gniezno.}

\bigskip \exampleRef{\textit{Estreicher XXII 358. Wierzbowski 2125. Formanowicz 236.}}
 \bigskip

 \exampleDig{\url{https://cyfrowe.mnk.pl/dlibra/publication/943/} page 9}

  
 % Pismo 5: tekst i pierwszy zestaw. — Rubryka a: z pismem 5. — Pismo
 % 12: trzeci zestaw. — Pismo 13: drugi zestaw. — Przerywnik 1: z
 % pismem 12.


% \examplePL{}


% \exampleEN{}


\bigskip

\fontID{U2-05}{56}

\fontstat{80}

% \exdisplay \bg \gla
 \exdisplay \bg \gla
% 1
{\PTglyph{5}{t56_l01g01.png}}
% 2
{\PTglyph{5}{t56_l01g02.png}}
% 3
{\PTglyph{5}{t56_l01g03.png}}
% 4
{\PTglyph{5}{t56_l01g04.png}}
% 5
{\PTglyph{5}{t56_l01g05.png}}
% 6
{\PTglyph{5}{t56_l01g06.png}}
% 7
{\PTglyph{5}{t56_l01g07.png}}
% 8
{\PTglyph{5}{t56_l01g08.png}}
% 9
{\PTglyph{5}{t56_l01g09.png}}
% 10
{\PTglyph{5}{t56_l01g10.png}}
% 11
{\PTglyph{5}{t56_l01g11.png}}
% 12
{\PTglyph{5}{t56_l01g12.png}}
% 13
{\PTglyph{5}{t56_l01g13.png}}
% 14
{\PTglyph{5}{t56_l01g14.png}}
% 15
{\PTglyph{5}{t56_l01g15.png}}
% 16
{\PTglyph{5}{t56_l01g16.png}}
% 17
{\PTglyph{5}{t56_l01g17.png}}
% 18
{\PTglyph{5}{t56_l01g18.png}}
% 19
{\PTglyph{5}{t56_l01g19.png}}
% 20
{\PTglyph{5}{t56_l01g20.png}}
% 21
{\PTglyph{5}{t56_l01g21.png}}
% 22
{\PTglyph{5}{t56_l02g01.png}}
% 23
{\PTglyph{5}{t56_l02g02.png}}
% 24
{\PTglyph{5}{t56_l02g03.png}}
% 25
{\PTglyph{5}{t56_l02g04.png}}
% 26
{\PTglyph{5}{t56_l02g05.png}}
% 27
{\PTglyph{5}{t56_l02g06.png}}
% 28
{\PTglyph{5}{t56_l02g07.png}}
% 29
{\PTglyph{5}{t56_l02g08.png}}
% 30
{\PTglyph{5}{t56_l02g09.png}}
% 31
{\PTglyph{5}{t56_l02g10.png}}
% 32
{\PTglyph{5}{t56_l02g11.png}}
% 33
{\PTglyph{5}{t56_l02g12.png}}
% 34
{\PTglyph{5}{t56_l02g13.png}}
% 35
{\PTglyph{5}{t56_l02g14.png}}
% 36
{\PTglyph{5}{t56_l02g15.png}}
% 37
{\PTglyph{5}{t56_l02g16.png}}
% 38
{\PTglyph{5}{t56_l02g17.png}}
% 39
{\PTglyph{5}{t56_l02g18.png}}
% 40
{\PTglyph{5}{t56_l02g19.png}}
% 41
{\PTglyph{5}{t56_l02g20.png}}
% 42
{\PTglyph{5}{t56_l02g21.png}}
% 43
{\PTglyph{5}{t56_l02g22.png}}
% 44
{\PTglyph{5}{t56_l02g23.png}}
% 45
{\PTglyph{5}{t56_l02g24.png}}
% 46
{\PTglyph{5}{t56_l02g25.png}}
% 47
{\PTglyph{5}{t56_l02g26.png}}
% 48
{\PTglyph{5}{t56_l02g27.png}}
% 49
{\PTglyph{5}{t56_l02g28.png}}
% 50
{\PTglyph{5}{t56_l02g29.png}}
% 51
{\PTglyph{5}{t56_l02g30.png}}
% 52
{\PTglyph{5}{t56_l02g31.png}}
% 53
{\PTglyph{5}{t56_l02g32.png}}
% 54
{\PTglyph{5}{t56_l03g01.png}}
% 55
{\PTglyph{5}{t56_l03g02.png}}
% 56
{\PTglyph{5}{t56_l03g03.png}}
% 57
{\PTglyph{5}{t56_l03g04.png}}
% 58
{\PTglyph{5}{t56_l03g05.png}}
% 59
{\PTglyph{5}{t56_l03g06.png}}
% 60
{\PTglyph{5}{t56_l03g07.png}}
% 61
{\PTglyph{5}{t56_l03g08.png}}
% 62
{\PTglyph{5}{t56_l03g09.png}}
% 63
{\PTglyph{5}{t56_l03g10.png}}
% 64
{\PTglyph{5}{t56_l03g11.png}}
% 65
{\PTglyph{5}{t56_l03g12.png}}
% 66
{\PTglyph{5}{t56_l03g13.png}}
% 67
{\PTglyph{5}{t56_l03g14.png}}
% 68
{\PTglyph{5}{t56_l03g15.png}}
% 69
{\PTglyph{5}{t56_l03g16.png}}
% 70
{\PTglyph{5}{t56_l03g17.png}}
% 71
{\PTglyph{5}{t56_l03g18.png}}
% 72
{\PTglyph{5}{t56_l03g19.png}}
% 73
{\PTglyph{5}{t56_l03g20.png}}
% 74
{\PTglyph{5}{t56_l03g21.png}}
% 75
{\PTglyph{5}{t56_l03g22.png}}
% 76
{\PTglyph{5}{t56_l03g23.png}}
% 77
{\PTglyph{5}{t56_l03g24.png}}
% 78
{\PTglyph{5}{t56_l03g25.png}}
% 79
{\PTglyph{5}{t56_l03g26.png}}
% 80
{\PTglyph{5}{t56_l03g27.png}}
//
%%% Local Variables:
%%% mode: latex
%%% TeX-engine: luatex
%%% TeX-master: shared
%%% End:

%//
%\glpismo%
 \glpismo
% 1
{\PTglyphid{U2-05_0101}}
% 2
{\PTglyphid{U2-05_0102}}
% 3
{\PTglyphid{U2-05_0103}}
% 4
{\PTglyphid{U2-05_0104}}
% 5
{\PTglyphid{U2-05_0105}}
% 6
{\PTglyphid{U2-05_0106}}
% 7
{\PTglyphid{U2-05_0107}}
% 8
{\PTglyphid{U2-05_0108}}
% 9
{\PTglyphid{U2-05_0109}}
% 10
{\PTglyphid{U2-05_0110}}
% 11
{\PTglyphid{U2-05_0111}}
% 12
{\PTglyphid{U2-05_0112}}
% 13
{\PTglyphid{U2-05_0113}}
% 14
{\PTglyphid{U2-05_0114}}
% 15
{\PTglyphid{U2-05_0115}}
% 16
{\PTglyphid{U2-05_0116}}
% 17
{\PTglyphid{U2-05_0117}}
% 18
{\PTglyphid{U2-05_0118}}
% 19
{\PTglyphid{U2-05_0119}}
% 20
{\PTglyphid{U2-05_0120}}
% 21
{\PTglyphid{U2-05_0121}}
% 22
{\PTglyphid{U2-05_0201}}
% 23
{\PTglyphid{U2-05_0202}}
% 24
{\PTglyphid{U2-05_0203}}
% 25
{\PTglyphid{U2-05_0204}}
% 26
{\PTglyphid{U2-05_0205}}
% 27
{\PTglyphid{U2-05_0206}}
% 28
{\PTglyphid{U2-05_0207}}
% 29
{\PTglyphid{U2-05_0208}}
% 30
{\PTglyphid{U2-05_0209}}
% 31
{\PTglyphid{U2-05_0210}}
% 32
{\PTglyphid{U2-05_0211}}
% 33
{\PTglyphid{U2-05_0212}}
% 34
{\PTglyphid{U2-05_0213}}
% 35
{\PTglyphid{U2-05_0214}}
% 36
{\PTglyphid{U2-05_0215}}
% 37
{\PTglyphid{U2-05_0216}}
% 38
{\PTglyphid{U2-05_0217}}
% 39
{\PTglyphid{U2-05_0218}}
% 40
{\PTglyphid{U2-05_0219}}
% 41
{\PTglyphid{U2-05_0220}}
% 42
{\PTglyphid{U2-05_0221}}
% 43
{\PTglyphid{U2-05_0222}}
% 44
{\PTglyphid{U2-05_0223}}
% 45
{\PTglyphid{U2-05_0224}}
% 46
{\PTglyphid{U2-05_0225}}
% 47
{\PTglyphid{U2-05_0226}}
% 48
{\PTglyphid{U2-05_0227}}
% 49
{\PTglyphid{U2-05_0228}}
% 50
{\PTglyphid{U2-05_0229}}
% 51
{\PTglyphid{U2-05_0230}}
% 52
{\PTglyphid{U2-05_0231}}
% 53
{\PTglyphid{U2-05_0232}}
% 54
{\PTglyphid{U2-05_0301}}
% 55
{\PTglyphid{U2-05_0302}}
% 56
{\PTglyphid{U2-05_0303}}
% 57
{\PTglyphid{U2-05_0304}}
% 58
{\PTglyphid{U2-05_0305}}
% 59
{\PTglyphid{U2-05_0306}}
% 60
{\PTglyphid{U2-05_0307}}
% 61
{\PTglyphid{U2-05_0308}}
% 62
{\PTglyphid{U2-05_0309}}
% 63
{\PTglyphid{U2-05_0310}}
% 64
{\PTglyphid{U2-05_0311}}
% 65
{\PTglyphid{U2-05_0312}}
% 66
{\PTglyphid{U2-05_0313}}
% 67
{\PTglyphid{U2-05_0314}}
% 68
{\PTglyphid{U2-05_0315}}
% 69
{\PTglyphid{U2-05_0316}}
% 70
{\PTglyphid{U2-05_0317}}
% 71
{\PTglyphid{U2-05_0318}}
% 72
{\PTglyphid{U2-05_0319}}
% 73
{\PTglyphid{U2-05_0320}}
% 74
{\PTglyphid{U2-05_0321}}
% 75
{\PTglyphid{U2-05_0322}}
% 76
{\PTglyphid{U2-05_0323}}
% 77
{\PTglyphid{U2-05_0324}}
% 78
{\PTglyphid{U2-05_0325}}
% 79
{\PTglyphid{U2-05_0326}}
% 80
{\PTglyphid{U2-05_0327}}
//
\endgl \xe
%%% Local Variables:
%%% mode: latex
%%% TeX-engine: luatex
%%% TeX-master: shared
%%% End:

% //
%\endgl \xe


 \newpage
 
%%%%%%%%%%%%%%%%%%%%%%%%%%%%%%%%%%%%%%%%%%%%%%%%%%%%%%%%%%%%%%%%%%%%%%%%%%%%%%%
% from meta.csv
% 57,Ungler2-06_PT05_241.djvu,Ungler2,06,05,241
% 
%%%%%%%%%%%%%%%%%%%%%%%%%%%%%%%%%%%%%%%%%%%%%%%%%%%%%%%%%%%%%%%%%%%%%%%%%%%%%%%

 
% from dsed4test:
% Ungler2-06_PT05_241_4dsed.txt:Note "6. Pismo tekstowe, rotunda M¹⁸. Stopień 20 ww. = 76 mm. — Tabl. 241."
% Ungler2-06_PT05_241_4dsed.txt:Note1 "Character set table prepared by Maria Błońska"

 \pismoPL{Florian Ungler druga drukarnia 6. Pismo tekstowe, rotunda M¹⁸. Stopień 20 ww. = 76 mm. — Tabl. 241.}
  
 \pismoEN{Florian Ungler second house 6. Rotunda text font, typeface M¹⁸. Type size 20 lines = 76 mm. — Plate 241.}

\plate{241}{V}{1964}

The plate prepared by Henryk Bułhak.\\
The font table prepared by Henryk Bułhak and Maria Błońska.

\bigskip

\exampleBib{V:53}

\bigskip \exampleDesc{VINCENTIUS FERRERIUS: [Opusculum de fine mundi]. Prophetiae Danielis tres horribiles.
Kraków, Florian Ungler, 1527. 89. War. A.}


\medskip
\examplePage{\textit{Karta a₂b}}

  \bigskip
  \exampleLib{Biblioteka Czartoryskich. Kraków.}

\bigskip \exampleRef{\textit{Estreicher XV 33. Wierzbowski 1040. Piekarski Kórn. 1569.}}

 \bigskip

 % \exampleDig{\url{https://cyfrowe.mnk.pl/dlibra/publication/943/} page 9}

% Pismo 6: tekst i zestaw. — Pismo 7: nagłówek. — Rubryka \dzeta{}: z pismem 6. — Cyfry 6: z pismem 6.

% \examplePL{}


% \exampleEN{}


\bigskip

\fontID{U2-06}{57}

\fontstat{95}

% \exdisplay \bg \gla
 \exdisplay \bg \gla
% 1
{\PTglyph{5}{t57_l01g01.png}}
% 2
{\PTglyph{5}{t57_l01g02.png}}
% 3
{\PTglyph{5}{t57_l01g03.png}}
% 4
{\PTglyph{5}{t57_l01g04.png}}
% 5
{\PTglyph{5}{t57_l01g05.png}}
% 6
{\PTglyph{5}{t57_l01g06.png}}
% 7
{\PTglyph{5}{t57_l01g07.png}}
% 8
{\PTglyph{5}{t57_l01g08.png}}
% 9
{\PTglyph{5}{t57_l01g09.png}}
% 10
{\PTglyph{5}{t57_l01g10.png}}
% 11
{\PTglyph{5}{t57_l01g11.png}}
% 12
{\PTglyph{5}{t57_l01g12.png}}
% 13
{\PTglyph{5}{t57_l01g13.png}}
% 14
{\PTglyph{5}{t57_l01g14.png}}
% 15
{\PTglyph{5}{t57_l01g15.png}}
% 16
{\PTglyph{5}{t57_l01g16.png}}
% 17
{\PTglyph{5}{t57_l01g17.png}}
% 18
{\PTglyph{5}{t57_l01g18.png}}
% 19
{\PTglyph{5}{t57_l01g19.png}}
% 20
{\PTglyph{5}{t57_l01g20.png}}
% 21
{\PTglyph{5}{t57_l02g01.png}}
% 22
{\PTglyph{5}{t57_l02g02.png}}
% 23
{\PTglyph{5}{t57_l02g03.png}}
% 24
{\PTglyph{5}{t57_l02g04.png}}
% 25
{\PTglyph{5}{t57_l02g05.png}}
% 26
{\PTglyph{5}{t57_l02g06.png}}
% 27
{\PTglyph{5}{t57_l02g07.png}}
% 28
{\PTglyph{5}{t57_l02g08.png}}
% 29
{\PTglyph{5}{t57_l02g09.png}}
% 30
{\PTglyph{5}{t57_l02g10.png}}
% 31
{\PTglyph{5}{t57_l02g11.png}}
% 32
{\PTglyph{5}{t57_l02g12.png}}
% 33
{\PTglyph{5}{t57_l02g13.png}}
% 34
{\PTglyph{5}{t57_l02g14.png}}
% 35
{\PTglyph{5}{t57_l02g15.png}}
% 36
{\PTglyph{5}{t57_l02g16.png}}
% 37
{\PTglyph{5}{t57_l02g17.png}}
% 38
{\PTglyph{5}{t57_l02g18.png}}
% 39
{\PTglyph{5}{t57_l02g19.png}}
% 40
{\PTglyph{5}{t57_l02g20.png}}
% 41
{\PTglyph{5}{t57_l02g21.png}}
% 42
{\PTglyph{5}{t57_l02g22.png}}
% 43
{\PTglyph{5}{t57_l02g23.png}}
% 44
{\PTglyph{5}{t57_l02g24.png}}
% 45
{\PTglyph{5}{t57_l02g25.png}}
% 46
{\PTglyph{5}{t57_l02g26.png}}
% 47
{\PTglyph{5}{t57_l02g27.png}}
% 48
{\PTglyph{5}{t57_l02g28.png}}
% 49
{\PTglyph{5}{t57_l02g29.png}}
% 50
{\PTglyph{5}{t57_l02g30.png}}
% 51
{\PTglyph{5}{t57_l02g31.png}}
% 52
{\PTglyph{5}{t57_l02g32.png}}
% 53
{\PTglyph{5}{t57_l02g33.png}}
% 54
{\PTglyph{5}{t57_l03g01.png}}
% 55
{\PTglyph{5}{t57_l03g02.png}}
% 56
{\PTglyph{5}{t57_l03g03.png}}
% 57
{\PTglyph{5}{t57_l03g04.png}}
% 58
{\PTglyph{5}{t57_l03g05.png}}
% 59
{\PTglyph{5}{t57_l03g06.png}}
% 60
{\PTglyph{5}{t57_l03g07.png}}
% 61
{\PTglyph{5}{t57_l03g08.png}}
% 62
{\PTglyph{5}{t57_l03g09.png}}
% 63
{\PTglyph{5}{t57_l03g10.png}}
% 64
{\PTglyph{5}{t57_l03g11.png}}
% 65
{\PTglyph{5}{t57_l03g12.png}}
% 66
{\PTglyph{5}{t57_l03g13.png}}
% 67
{\PTglyph{5}{t57_l03g14.png}}
% 68
{\PTglyph{5}{t57_l03g15.png}}
% 69
{\PTglyph{5}{t57_l03g16.png}}
% 70
{\PTglyph{5}{t57_l03g17.png}}
% 71
{\PTglyph{5}{t57_l03g18.png}}
% 72
{\PTglyph{5}{t57_l03g19.png}}
% 73
{\PTglyph{5}{t57_l03g20.png}}
% 74
{\PTglyph{5}{t57_l03g21.png}}
% 75
{\PTglyph{5}{t57_l03g22.png}}
% 76
{\PTglyph{5}{t57_l03g23.png}}
% 77
{\PTglyph{5}{t57_l03g24.png}}
% 78
{\PTglyph{5}{t57_l03g25.png}}
% 79
{\PTglyph{5}{t57_l03g26.png}}
% 80
{\PTglyph{5}{t57_l03g27.png}}
% 81
{\PTglyph{5}{t57_l03g28.png}}
% 82
{\PTglyph{5}{t57_l03g29.png}}
% 83
{\PTglyph{5}{t57_l03g30.png}}
% 84
{\PTglyph{5}{t57_l03g31.png}}
% 85
{\PTglyph{5}{t57_l03g32.png}}
% 86
{\PTglyph{5}{t57_l03g33.png}}
% 87
{\PTglyph{5}{t57_l03g34.png}}
% 88
{\PTglyph{5}{t57_l04g01.png}}
% 89
{\PTglyph{5}{t57_l04g02.png}}
% 90
{\PTglyph{5}{t57_l04g03.png}}
% 91
{\PTglyph{5}{t57_l04g04.png}}
% 92
{\PTglyph{5}{t57_l04g05.png}}
% 93
{\PTglyph{5}{t57_l04g06.png}}
% 94
{\PTglyph{5}{t57_l04g07.png}}
% 95
{\PTglyph{5}{t57_l04g08.png}}
//
%%% Local Variables:
%%% mode: latex
%%% TeX-engine: luatex
%%% TeX-master: shared
%%% End:

%//
%\glpismo%
 \glpismo
% 1
{\PTglyphid{U2-06_0101}}
% 2
{\PTglyphid{U2-06_0102}}
% 3
{\PTglyphid{U2-06_0103}}
% 4
{\PTglyphid{U2-06_0104}}
% 5
{\PTglyphid{U2-06_0105}}
% 6
{\PTglyphid{U2-06_0106}}
% 7
{\PTglyphid{U2-06_0107}}
% 8
{\PTglyphid{U2-06_0108}}
% 9
{\PTglyphid{U2-06_0109}}
% 10
{\PTglyphid{U2-06_0110}}
% 11
{\PTglyphid{U2-06_0111}}
% 12
{\PTglyphid{U2-06_0112}}
% 13
{\PTglyphid{U2-06_0113}}
% 14
{\PTglyphid{U2-06_0114}}
% 15
{\PTglyphid{U2-06_0115}}
% 16
{\PTglyphid{U2-06_0116}}
% 17
{\PTglyphid{U2-06_0117}}
% 18
{\PTglyphid{U2-06_0118}}
% 19
{\PTglyphid{U2-06_0119}}
% 20
{\PTglyphid{U2-06_0120}}
% 21
{\PTglyphid{U2-06_0201}}
% 22
{\PTglyphid{U2-06_0202}}
% 23
{\PTglyphid{U2-06_0203}}
% 24
{\PTglyphid{U2-06_0204}}
% 25
{\PTglyphid{U2-06_0205}}
% 26
{\PTglyphid{U2-06_0206}}
% 27
{\PTglyphid{U2-06_0207}}
% 28
{\PTglyphid{U2-06_0208}}
% 29
{\PTglyphid{U2-06_0209}}
% 30
{\PTglyphid{U2-06_0210}}
% 31
{\PTglyphid{U2-06_0211}}
% 32
{\PTglyphid{U2-06_0212}}
% 33
{\PTglyphid{U2-06_0213}}
% 34
{\PTglyphid{U2-06_0214}}
% 35
{\PTglyphid{U2-06_0215}}
% 36
{\PTglyphid{U2-06_0216}}
% 37
{\PTglyphid{U2-06_0217}}
% 38
{\PTglyphid{U2-06_0218}}
% 39
{\PTglyphid{U2-06_0219}}
% 40
{\PTglyphid{U2-06_0220}}
% 41
{\PTglyphid{U2-06_0221}}
% 42
{\PTglyphid{U2-06_0222}}
% 43
{\PTglyphid{U2-06_0223}}
% 44
{\PTglyphid{U2-06_0224}}
% 45
{\PTglyphid{U2-06_0225}}
% 46
{\PTglyphid{U2-06_0226}}
% 47
{\PTglyphid{U2-06_0227}}
% 48
{\PTglyphid{U2-06_0228}}
% 49
{\PTglyphid{U2-06_0229}}
% 50
{\PTglyphid{U2-06_0230}}
% 51
{\PTglyphid{U2-06_0231}}
% 52
{\PTglyphid{U2-06_0232}}
% 53
{\PTglyphid{U2-06_0233}}
% 54
{\PTglyphid{U2-06_0301}}
% 55
{\PTglyphid{U2-06_0302}}
% 56
{\PTglyphid{U2-06_0303}}
% 57
{\PTglyphid{U2-06_0304}}
% 58
{\PTglyphid{U2-06_0305}}
% 59
{\PTglyphid{U2-06_0306}}
% 60
{\PTglyphid{U2-06_0307}}
% 61
{\PTglyphid{U2-06_0308}}
% 62
{\PTglyphid{U2-06_0309}}
% 63
{\PTglyphid{U2-06_0310}}
% 64
{\PTglyphid{U2-06_0311}}
% 65
{\PTglyphid{U2-06_0312}}
% 66
{\PTglyphid{U2-06_0313}}
% 67
{\PTglyphid{U2-06_0314}}
% 68
{\PTglyphid{U2-06_0315}}
% 69
{\PTglyphid{U2-06_0316}}
% 70
{\PTglyphid{U2-06_0317}}
% 71
{\PTglyphid{U2-06_0318}}
% 72
{\PTglyphid{U2-06_0319}}
% 73
{\PTglyphid{U2-06_0320}}
% 74
{\PTglyphid{U2-06_0321}}
% 75
{\PTglyphid{U2-06_0322}}
% 76
{\PTglyphid{U2-06_0323}}
% 77
{\PTglyphid{U2-06_0324}}
% 78
{\PTglyphid{U2-06_0325}}
% 79
{\PTglyphid{U2-06_0326}}
% 80
{\PTglyphid{U2-06_0327}}
% 81
{\PTglyphid{U2-06_0328}}
% 82
{\PTglyphid{U2-06_0329}}
% 83
{\PTglyphid{U2-06_0330}}
% 84
{\PTglyphid{U2-06_0331}}
% 85
{\PTglyphid{U2-06_0332}}
% 86
{\PTglyphid{U2-06_0333}}
% 87
{\PTglyphid{U2-06_0334}}
% 88
{\PTglyphid{U2-06_0401}}
% 89
{\PTglyphid{U2-06_0402}}
% 90
{\PTglyphid{U2-06_0403}}
% 91
{\PTglyphid{U2-06_0404}}
% 92
{\PTglyphid{U2-06_0405}}
% 93
{\PTglyphid{U2-06_0406}}
% 94
{\PTglyphid{U2-06_0407}}
% 95
{\PTglyphid{U2-06_0408}}
//
\endgl \xe
%%% Local Variables:
%%% mode: latex
%%% TeX-engine: luatex
%%% TeX-master: shared
%%% End:

% //
%\endgl \xe



 \newpage
 
%%%%%%%%%%%%%%%%%%%%%%%%%%%%%%%%%%%%%%%%%%%%%%%%%%%%%%%%%%%%%%%%%%%%%%%%%%%%%%%
% from meta.csv
% 58,Ungler2-09_PT05_244.djvu,Ungler2,09,05,244
% 
%%%%%%%%%%%%%%%%%%%%%%%%%%%%%%%%%%%%%%%%%%%%%%%%%%%%%%%%%%%%%%%%%%%%%%%%%%%%%%%

 
% from dsed4test:
% Ungler2-09_PT05_244_4dsed.txt:Note "9. Pismo tekstowe, szwabacha zromanizowana M⁸¹. Stopień 20 ww. = = 80/81 mm. — Tabl. 244."
% Ungler2-09_PT05_244_4dsed.txt:Note1 "Character set table prepared by Maria Błońska"

 \pismoPL{Florian Ungler druga drukarnia 9. Pismo tekstowe, szwabacha zromanizowana M⁸¹. Stopień 20 ww. =  80/81 mm. — Tabl. 244.}
  
 \pismoEN{Florian Ungler second house 9. Schwabacher romanized text font, typeface M⁸¹. Type size 20 lines = 80/81 mm. — Plate 244.}

\plate{244}{V}{1964}

The plate prepared by Henryk Bułhak.\\
The font table prepared by Henryk Bułhak and Maria Błońska.

\bigskip

\exampleBib{V:66}

\bigskip \exampleDesc{MICHAEL VRATISLAVIENSIS: Prosarum dilucidatio. Kraków, Florian Ungler, [po 1 III 1530]. 4⁰.}


\medskip
\examplePage{\textit{Karta tytułowa verso.}}

  \bigskip
  \exampleLib{Biblioteka Narodowa. Warszawa.}

\bigskip \exampleRef{\textit{Estreicher XXXII 359. Wierzbowski 750, 1071.}}

 \bigskip

 % \exampleDig{\url{https://cyfrowe.mnk.pl/dlibra/publication/943/} page 9}


%Pismo 4: wiersz I, 5. — Pismo 10: wiersz 2—4. — Pismo 9: tekst i zestaw. — Rubryka \eta{}?. — Cyfry 5. — lInicjał 18 (P).

% \examplePL{}


% \exampleEN{}


\bigskip

\fontID{U2-09}{58}

\fontstat{138}

% \exdisplay \bg \gla
 \exdisplay \bg \gla
% 1
{\PTglyph{5}{t58_l01g01.png}}
% 2
{\PTglyph{5}{t58_l01g02.png}}
% 3
{\PTglyph{5}{t58_l01g03.png}}
% 4
{\PTglyph{5}{t58_l01g04.png}}
% 5
{\PTglyph{5}{t58_l01g05.png}}
% 6
{\PTglyph{5}{t58_l01g06.png}}
% 7
{\PTglyph{5}{t58_l01g07.png}}
% 8
{\PTglyph{5}{t58_l01g08.png}}
% 9
{\PTglyph{5}{t58_l01g09.png}}
% 10
{\PTglyph{5}{t58_l01g10.png}}
% 11
{\PTglyph{5}{t58_l01g11.png}}
% 12
{\PTglyph{5}{t58_l01g12.png}}
% 13
{\PTglyph{5}{t58_l01g13.png}}
% 14
{\PTglyph{5}{t58_l01g14.png}}
% 15
{\PTglyph{5}{t58_l01g15.png}}
% 16
{\PTglyph{5}{t58_l01g16.png}}
% 17
{\PTglyph{5}{t58_l01g17.png}}
% 18
{\PTglyph{5}{t58_l01g18.png}}
% 19
{\PTglyph{5}{t58_l01g19.png}}
% 20
{\PTglyph{5}{t58_l01g20.png}}
% 21
{\PTglyph{5}{t58_l01g21.png}}
% 22
{\PTglyph{5}{t58_l01g22.png}}
% 23
{\PTglyph{5}{t58_l01g23.png}}
% 24
{\PTglyph{5}{t58_l01g24.png}}
% 25
{\PTglyph{5}{t58_l01g25.png}}
% 26
{\PTglyph{5}{t58_l01g26.png}}
% 27
{\PTglyph{5}{t58_l01g27.png}}
% 28
{\PTglyph{5}{t58_l01g28.png}}
% 29
{\PTglyph{5}{t58_l01g29.png}}
% 30
{\PTglyph{5}{t58_l01g30.png}}
% 31
{\PTglyph{5}{t58_l01g31.png}}
% 32
{\PTglyph{5}{t58_l01g32.png}}
% 33
{\PTglyph{5}{t58_l02g01.png}}
% 34
{\PTglyph{5}{t58_l02g02.png}}
% 35
{\PTglyph{5}{t58_l02g03.png}}
% 36
{\PTglyph{5}{t58_l02g04.png}}
% 37
{\PTglyph{5}{t58_l02g05.png}}
% 38
{\PTglyph{5}{t58_l02g06.png}}
% 39
{\PTglyph{5}{t58_l02g07.png}}
% 40
{\PTglyph{5}{t58_l02g08.png}}
% 41
{\PTglyph{5}{t58_l02g09.png}}
% 42
{\PTglyph{5}{t58_l02g10.png}}
% 43
{\PTglyph{5}{t58_l02g11.png}}
% 44
{\PTglyph{5}{t58_l02g12.png}}
% 45
{\PTglyph{5}{t58_l02g13.png}}
% 46
{\PTglyph{5}{t58_l02g14.png}}
% 47
{\PTglyph{5}{t58_l02g15.png}}
% 48
{\PTglyph{5}{t58_l02g16.png}}
% 49
{\PTglyph{5}{t58_l02g17.png}}
% 50
{\PTglyph{5}{t58_l02g18.png}}
% 51
{\PTglyph{5}{t58_l02g19.png}}
% 52
{\PTglyph{5}{t58_l02g20.png}}
% 53
{\PTglyph{5}{t58_l02g21.png}}
% 54
{\PTglyph{5}{t58_l02g22.png}}
% 55
{\PTglyph{5}{t58_l02g23.png}}
% 56
{\PTglyph{5}{t58_l02g24.png}}
% 57
{\PTglyph{5}{t58_l02g25.png}}
% 58
{\PTglyph{5}{t58_l02g26.png}}
% 59
{\PTglyph{5}{t58_l02g27.png}}
% 60
{\PTglyph{5}{t58_l02g28.png}}
% 61
{\PTglyph{5}{t58_l02g29.png}}
% 62
{\PTglyph{5}{t58_l02g30.png}}
% 63
{\PTglyph{5}{t58_l02g31.png}}
% 64
{\PTglyph{5}{t58_l02g32.png}}
% 65
{\PTglyph{5}{t58_l02g33.png}}
% 66
{\PTglyph{5}{t58_l02g34.png}}
% 67
{\PTglyph{5}{t58_l02g35.png}}
% 68
{\PTglyph{5}{t58_l02g36.png}}
% 69
{\PTglyph{5}{t58_l02g37.png}}
% 70
{\PTglyph{5}{t58_l02g38.png}}
% 71
{\PTglyph{5}{t58_l02g39.png}}
% 72
{\PTglyph{5}{t58_l02g40.png}}
% 73
{\PTglyph{5}{t58_l02g41.png}}
% 74
{\PTglyph{5}{t58_l02g42.png}}
% 75
{\PTglyph{5}{t58_l02g43.png}}
% 76
{\PTglyph{5}{t58_l02g44.png}}
% 77
{\PTglyph{5}{t58_l02g45.png}}
% 78
{\PTglyph{5}{t58_l02g46.png}}
% 79
{\PTglyph{5}{t58_l02g47.png}}
% 80
{\PTglyph{5}{t58_l02g48.png}}
% 81
{\PTglyph{5}{t58_l02g49.png}}
% 82
{\PTglyph{5}{t58_l03g01.png}}
% 83
{\PTglyph{5}{t58_l03g02.png}}
% 84
{\PTglyph{5}{t58_l03g03.png}}
% 85
{\PTglyph{5}{t58_l03g04.png}}
% 86
{\PTglyph{5}{t58_l03g05.png}}
% 87
{\PTglyph{5}{t58_l03g06.png}}
% 88
{\PTglyph{5}{t58_l03g07.png}}
% 89
{\PTglyph{5}{t58_l03g08.png}}
% 90
{\PTglyph{5}{t58_l03g09.png}}
% 91
{\PTglyph{5}{t58_l03g10.png}}
% 92
{\PTglyph{5}{t58_l03g11.png}}
% 93
{\PTglyph{5}{t58_l03g12.png}}
% 94
{\PTglyph{5}{t58_l03g13.png}}
% 95
{\PTglyph{5}{t58_l03g14.png}}
% 96
{\PTglyph{5}{t58_l03g15.png}}
% 97
{\PTglyph{5}{t58_l03g16.png}}
% 98
{\PTglyph{5}{t58_l03g17.png}}
% 99
{\PTglyph{5}{t58_l03g18.png}}
% 100
{\PTglyph{5}{t58_l03g19.png}}
% 101
{\PTglyph{5}{t58_l03g20.png}}
% 102
{\PTglyph{5}{t58_l03g21.png}}
% 103
{\PTglyph{5}{t58_l03g22.png}}
% 104
{\PTglyph{5}{t58_l03g23.png}}
% 105
{\PTglyph{5}{t58_l03g24.png}}
% 106
{\PTglyph{5}{t58_l03g25.png}}
% 107
{\PTglyph{5}{t58_l03g26.png}}
% 108
{\PTglyph{5}{t58_l03g27.png}}
% 109
{\PTglyph{5}{t58_l03g28.png}}
% 110
{\PTglyph{5}{t58_l03g29.png}}
% 111
{\PTglyph{5}{t58_l03g30.png}}
% 112
{\PTglyph{5}{t58_l03g31.png}}
% 113
{\PTglyph{5}{t58_l03g32.png}}
% 114
{\PTglyph{5}{t58_l03g33.png}}
% 115
{\PTglyph{5}{t58_l03g34.png}}
% 116
{\PTglyph{5}{t58_l03g35.png}}
% 117
{\PTglyph{5}{t58_l03g36.png}}
% 118
{\PTglyph{5}{t58_l03g37.png}}
% 119
{\PTglyph{5}{t58_l03g38.png}}
% 120
{\PTglyph{5}{t58_l03g39.png}}
% 121
{\PTglyph{5}{t58_l03g40.png}}
% 122
{\PTglyph{5}{t58_l03g41.png}}
% 123
{\PTglyph{5}{t58_l03g42.png}}
% 124
{\PTglyph{5}{t58_l03g43.png}}
% 125
{\PTglyph{5}{t58_l03g44.png}}
% 126
{\PTglyph{5}{t58_l03g45.png}}
% 127
{\PTglyph{5}{t58_l03g46.png}}
% 128
{\PTglyph{5}{t58_l03g47.png}}
% 129
{\PTglyph{5}{t58_l04g01.png}}
% 130
{\PTglyph{5}{t58_l04g02.png}}
% 131
{\PTglyph{5}{t58_l04g03.png}}
% 132
{\PTglyph{5}{t58_l04g04.png}}
% 133
{\PTglyph{5}{t58_l04g05.png}}
% 134
{\PTglyph{5}{t58_l04g06.png}}
% 135
{\PTglyph{5}{t58_l04g07.png}}
% 136
{\PTglyph{5}{t58_l04g08.png}}
% 137
{\PTglyph{5}{t58_l04g09.png}}
% 138
{\PTglyph{5}{t58_l04g10.png}}
//
%%% Local Variables:
%%% mode: latex
%%% TeX-engine: luatex
%%% TeX-master: shared
%%% End:

%//
%\glpismo%
 \glpismo
% 1
{\PTglyphid{U2-09_0101}}
% 2
{\PTglyphid{U2-09_0102}}
% 3
{\PTglyphid{U2-09_0103}}
% 4
{\PTglyphid{U2-09_0104}}
% 5
{\PTglyphid{U2-09_0105}}
% 6
{\PTglyphid{U2-09_0106}}
% 7
{\PTglyphid{U2-09_0107}}
% 8
{\PTglyphid{U2-09_0108}}
% 9
{\PTglyphid{U2-09_0109}}
% 10
{\PTglyphid{U2-09_0110}}
% 11
{\PTglyphid{U2-09_0111}}
% 12
{\PTglyphid{U2-09_0112}}
% 13
{\PTglyphid{U2-09_0113}}
% 14
{\PTglyphid{U2-09_0114}}
% 15
{\PTglyphid{U2-09_0115}}
% 16
{\PTglyphid{U2-09_0116}}
% 17
{\PTglyphid{U2-09_0117}}
% 18
{\PTglyphid{U2-09_0118}}
% 19
{\PTglyphid{U2-09_0119}}
% 20
{\PTglyphid{U2-09_0120}}
% 21
{\PTglyphid{U2-09_0121}}
% 22
{\PTglyphid{U2-09_0122}}
% 23
{\PTglyphid{U2-09_0123}}
% 24
{\PTglyphid{U2-09_0124}}
% 25
{\PTglyphid{U2-09_0125}}
% 26
{\PTglyphid{U2-09_0126}}
% 27
{\PTglyphid{U2-09_0127}}
% 28
{\PTglyphid{U2-09_0128}}
% 29
{\PTglyphid{U2-09_0129}}
% 30
{\PTglyphid{U2-09_0130}}
% 31
{\PTglyphid{U2-09_0131}}
% 32
{\PTglyphid{U2-09_0132}}
% 33
{\PTglyphid{U2-09_0201}}
% 34
{\PTglyphid{U2-09_0202}}
% 35
{\PTglyphid{U2-09_0203}}
% 36
{\PTglyphid{U2-09_0204}}
% 37
{\PTglyphid{U2-09_0205}}
% 38
{\PTglyphid{U2-09_0206}}
% 39
{\PTglyphid{U2-09_0207}}
% 40
{\PTglyphid{U2-09_0208}}
% 41
{\PTglyphid{U2-09_0209}}
% 42
{\PTglyphid{U2-09_0210}}
% 43
{\PTglyphid{U2-09_0211}}
% 44
{\PTglyphid{U2-09_0212}}
% 45
{\PTglyphid{U2-09_0213}}
% 46
{\PTglyphid{U2-09_0214}}
% 47
{\PTglyphid{U2-09_0215}}
% 48
{\PTglyphid{U2-09_0216}}
% 49
{\PTglyphid{U2-09_0217}}
% 50
{\PTglyphid{U2-09_0218}}
% 51
{\PTglyphid{U2-09_0219}}
% 52
{\PTglyphid{U2-09_0220}}
% 53
{\PTglyphid{U2-09_0221}}
% 54
{\PTglyphid{U2-09_0222}}
% 55
{\PTglyphid{U2-09_0223}}
% 56
{\PTglyphid{U2-09_0224}}
% 57
{\PTglyphid{U2-09_0225}}
% 58
{\PTglyphid{U2-09_0226}}
% 59
{\PTglyphid{U2-09_0227}}
% 60
{\PTglyphid{U2-09_0228}}
% 61
{\PTglyphid{U2-09_0229}}
% 62
{\PTglyphid{U2-09_0230}}
% 63
{\PTglyphid{U2-09_0231}}
% 64
{\PTglyphid{U2-09_0232}}
% 65
{\PTglyphid{U2-09_0233}}
% 66
{\PTglyphid{U2-09_0234}}
% 67
{\PTglyphid{U2-09_0235}}
% 68
{\PTglyphid{U2-09_0236}}
% 69
{\PTglyphid{U2-09_0237}}
% 70
{\PTglyphid{U2-09_0238}}
% 71
{\PTglyphid{U2-09_0239}}
% 72
{\PTglyphid{U2-09_0240}}
% 73
{\PTglyphid{U2-09_0241}}
% 74
{\PTglyphid{U2-09_0242}}
% 75
{\PTglyphid{U2-09_0243}}
% 76
{\PTglyphid{U2-09_0244}}
% 77
{\PTglyphid{U2-09_0245}}
% 78
{\PTglyphid{U2-09_0246}}
% 79
{\PTglyphid{U2-09_0247}}
% 80
{\PTglyphid{U2-09_0248}}
% 81
{\PTglyphid{U2-09_0249}}
% 82
{\PTglyphid{U2-09_0301}}
% 83
{\PTglyphid{U2-09_0302}}
% 84
{\PTglyphid{U2-09_0303}}
% 85
{\PTglyphid{U2-09_0304}}
% 86
{\PTglyphid{U2-09_0305}}
% 87
{\PTglyphid{U2-09_0306}}
% 88
{\PTglyphid{U2-09_0307}}
% 89
{\PTglyphid{U2-09_0308}}
% 90
{\PTglyphid{U2-09_0309}}
% 91
{\PTglyphid{U2-09_0310}}
% 92
{\PTglyphid{U2-09_0311}}
% 93
{\PTglyphid{U2-09_0312}}
% 94
{\PTglyphid{U2-09_0313}}
% 95
{\PTglyphid{U2-09_0314}}
% 96
{\PTglyphid{U2-09_0315}}
% 97
{\PTglyphid{U2-09_0316}}
% 98
{\PTglyphid{U2-09_0317}}
% 99
{\PTglyphid{U2-09_0318}}
% 100
{\PTglyphid{U2-09_0319}}
% 101
{\PTglyphid{U2-09_0320}}
% 102
{\PTglyphid{U2-09_0321}}
% 103
{\PTglyphid{U2-09_0322}}
% 104
{\PTglyphid{U2-09_0323}}
% 105
{\PTglyphid{U2-09_0324}}
% 106
{\PTglyphid{U2-09_0325}}
% 107
{\PTglyphid{U2-09_0326}}
% 108
{\PTglyphid{U2-09_0327}}
% 109
{\PTglyphid{U2-09_0328}}
% 110
{\PTglyphid{U2-09_0329}}
% 111
{\PTglyphid{U2-09_0330}}
% 112
{\PTglyphid{U2-09_0331}}
% 113
{\PTglyphid{U2-09_0332}}
% 114
{\PTglyphid{U2-09_0333}}
% 115
{\PTglyphid{U2-09_0334}}
% 116
{\PTglyphid{U2-09_0335}}
% 117
{\PTglyphid{U2-09_0336}}
% 118
{\PTglyphid{U2-09_0337}}
% 119
{\PTglyphid{U2-09_0338}}
% 120
{\PTglyphid{U2-09_0339}}
% 121
{\PTglyphid{U2-09_0340}}
% 122
{\PTglyphid{U2-09_0341}}
% 123
{\PTglyphid{U2-09_0342}}
% 124
{\PTglyphid{U2-09_0343}}
% 125
{\PTglyphid{U2-09_0344}}
% 126
{\PTglyphid{U2-09_0345}}
% 127
{\PTglyphid{U2-09_0346}}
% 128
{\PTglyphid{U2-09_0347}}
% 129
{\PTglyphid{U2-09_0401}}
% 130
{\PTglyphid{U2-09_0402}}
% 131
{\PTglyphid{U2-09_0403}}
% 132
{\PTglyphid{U2-09_0404}}
% 133
{\PTglyphid{U2-09_0405}}
% 134
{\PTglyphid{U2-09_0406}}
% 135
{\PTglyphid{U2-09_0407}}
% 136
{\PTglyphid{U2-09_0408}}
% 137
{\PTglyphid{U2-09_0409}}
% 138
{\PTglyphid{U2-09_0410}}
//
\endgl \xe
%%% Local Variables:
%%% mode: latex
%%% TeX-engine: luatex
%%% TeX-master: shared
%%% End:

% //
%\endgl \xe


 \newpage
 
%%%%%%%%%%%%%%%%%%%%%%%%%%%%%%%%%%%%%%%%%%%%%%%%%%%%%%%%%%%%%%%%%%%%%%%%%%%%%%%
% from meta.csv
% 59,Ungler2-10_PT05_245.djvu,Ungler2,10,05,245
% 
%%%%%%%%%%%%%%%%%%%%%%%%%%%%%%%%%%%%%%%%%%%%%%%%%%%%%%%%%%%%%%%%%%%%%%%%%%%%%%%

 
% from dsed4test:
% Ungler2-10_PT05_245_4dsed.txt:Note "10. Pismo tekstowe, rotunda M¹⁸. Stopień 20 ww. = 66 mm. — Tabl. 245."
% Ungler2-10_PT05_245_4dsed.txt:Note1 "Character set table prepared by Maria Błońska and Anna Wolińska"

 \pismoPL{Florian Ungler druga drukarnia 10. Pismo tekstowe, rotunda M¹⁸. Stopień 20 ww. = 66 mm. — Tabl. 245.}
  
 \pismoEN{Florian Ungler second house 10. Rotunda text font, typeface M¹⁸. Type size 20 lines = 66 mm. — Plate 245.}

\plate{245}{V}{1964}

The plate prepared by Henryk Bułhak.\\
The font table prepared by Henryk Bułhak,  Maria Błońska and Anna Wolińska.

\bigskip

\exampleBib{V:66}

\bigskip \exampleDesc{MICHAEL VRATISLAVIENSIS: Prosarum dilucidatio. Kraków, Florian Ungler, [po 1 III 1530]. 4⁰.}


\medskip
\examplePage{\textit{Karta 5 b.}}

  \bigskip
  \exampleLib{Archiwum Kapituły Metropolitalnej. Kraków.}

\bigskip \exampleRef{\textit{Estreicher XXX 359. Wierzbowski 750, 1071.}}

 \bigskip

 % \exampleDig{\url{https://cyfrowe.mnk.pl/dlibra/publication/943/} page 9}

% Pismo 4: nagłówek. — Pismo 10: tekst i pierwszy zestaw. — Pismo 11: drugi zestaw. — Rubryka ?: z pismem 10. — Cyfry 7: z pismem 10.

% \examplePL{}


% \exampleEN{}


\bigskip

\fontID{U2-10}{59}

\fontstat{123}

% \exdisplay \bg \gla
 \exdisplay \bg \gla
% 1
{\PTglyph{5}{t59_l01g01.png}}
% 2
{\PTglyph{5}{t59_l01g02.png}}
% 3
{\PTglyph{5}{t59_l01g03.png}}
% 4
{\PTglyph{5}{t59_l01g04.png}}
% 5
{\PTglyph{5}{t59_l01g05.png}}
% 6
{\PTglyph{5}{t59_l01g06.png}}
% 7
{\PTglyph{5}{t59_l01g07.png}}
% 8
{\PTglyph{5}{t59_l01g08.png}}
% 9
{\PTglyph{5}{t59_l01g09.png}}
% 10
{\PTglyph{5}{t59_l01g10.png}}
% 11
{\PTglyph{5}{t59_l01g11.png}}
% 12
{\PTglyph{5}{t59_l01g12.png}}
% 13
{\PTglyph{5}{t59_l01g13.png}}
% 14
{\PTglyph{5}{t59_l01g14.png}}
% 15
{\PTglyph{5}{t59_l01g15.png}}
% 16
{\PTglyph{5}{t59_l01g16.png}}
% 17
{\PTglyph{5}{t59_l01g17.png}}
% 18
{\PTglyph{5}{t59_l01g18.png}}
% 19
{\PTglyph{5}{t59_l01g19.png}}
% 20
{\PTglyph{5}{t59_l01g20.png}}
% 21
{\PTglyph{5}{t59_l01g21.png}}
% 22
{\PTglyph{5}{t59_l01g22.png}}
% 23
{\PTglyph{5}{t59_l02g01.png}}
% 24
{\PTglyph{5}{t59_l02g02.png}}
% 25
{\PTglyph{5}{t59_l02g03.png}}
% 26
{\PTglyph{5}{t59_l02g04.png}}
% 27
{\PTglyph{5}{t59_l02g05.png}}
% 28
{\PTglyph{5}{t59_l02g06.png}}
% 29
{\PTglyph{5}{t59_l02g07.png}}
% 30
{\PTglyph{5}{t59_l02g08.png}}
% 31
{\PTglyph{5}{t59_l02g09.png}}
% 32
{\PTglyph{5}{t59_l02g10.png}}
% 33
{\PTglyph{5}{t59_l02g11.png}}
% 34
{\PTglyph{5}{t59_l02g12.png}}
% 35
{\PTglyph{5}{t59_l02g13.png}}
% 36
{\PTglyph{5}{t59_l02g14.png}}
% 37
{\PTglyph{5}{t59_l02g15.png}}
% 38
{\PTglyph{5}{t59_l02g16.png}}
% 39
{\PTglyph{5}{t59_l02g17.png}}
% 40
{\PTglyph{5}{t59_l02g18.png}}
% 41
{\PTglyph{5}{t59_l02g19.png}}
% 42
{\PTglyph{5}{t59_l02g20.png}}
% 43
{\PTglyph{5}{t59_l02g21.png}}
% 44
{\PTglyph{5}{t59_l02g22.png}}
% 45
{\PTglyph{5}{t59_l02g23.png}}
% 46
{\PTglyph{5}{t59_l02g24.png}}
% 47
{\PTglyph{5}{t59_l02g25.png}}
% 48
{\PTglyph{5}{t59_l02g26.png}}
% 49
{\PTglyph{5}{t59_l02g27.png}}
% 50
{\PTglyph{5}{t59_l02g28.png}}
% 51
{\PTglyph{5}{t59_l02g29.png}}
% 52
{\PTglyph{5}{t59_l02g30.png}}
% 53
{\PTglyph{5}{t59_l02g31.png}}
% 54
{\PTglyph{5}{t59_l02g32.png}}
% 55
{\PTglyph{5}{t59_l02g33.png}}
% 56
{\PTglyph{5}{t59_l02g34.png}}
% 57
{\PTglyph{5}{t59_l02g35.png}}
% 58
{\PTglyph{5}{t59_l02g36.png}}
% 59
{\PTglyph{5}{t59_l03g01.png}}
% 60
{\PTglyph{5}{t59_l03g02.png}}
% 61
{\PTglyph{5}{t59_l03g03.png}}
% 62
{\PTglyph{5}{t59_l03g04.png}}
% 63
{\PTglyph{5}{t59_l03g05.png}}
% 64
{\PTglyph{5}{t59_l03g06.png}}
% 65
{\PTglyph{5}{t59_l03g07.png}}
% 66
{\PTglyph{5}{t59_l03g08.png}}
% 67
{\PTglyph{5}{t59_l03g09.png}}
% 68
{\PTglyph{5}{t59_l03g10.png}}
% 69
{\PTglyph{5}{t59_l03g11.png}}
% 70
{\PTglyph{5}{t59_l03g12.png}}
% 71
{\PTglyph{5}{t59_l03g13.png}}
% 72
{\PTglyph{5}{t59_l03g14.png}}
% 73
{\PTglyph{5}{t59_l03g15.png}}
% 74
{\PTglyph{5}{t59_l03g16.png}}
% 75
{\PTglyph{5}{t59_l03g17.png}}
% 76
{\PTglyph{5}{t59_l03g18.png}}
% 77
{\PTglyph{5}{t59_l03g19.png}}
% 78
{\PTglyph{5}{t59_l03g20.png}}
% 79
{\PTglyph{5}{t59_l03g21.png}}
% 80
{\PTglyph{5}{t59_l03g22.png}}
% 81
{\PTglyph{5}{t59_l03g23.png}}
% 82
{\PTglyph{5}{t59_l03g24.png}}
% 83
{\PTglyph{5}{t59_l03g25.png}}
% 84
{\PTglyph{5}{t59_l03g26.png}}
% 85
{\PTglyph{5}{t59_l03g27.png}}
% 86
{\PTglyph{5}{t59_l03g28.png}}
% 87
{\PTglyph{5}{t59_l03g29.png}}
% 88
{\PTglyph{5}{t59_l03g30.png}}
% 89
{\PTglyph{5}{t59_l03g31.png}}
% 90
{\PTglyph{5}{t59_l03g32.png}}
% 91
{\PTglyph{5}{t59_l04g01.png}}
% 92
{\PTglyph{5}{t59_l04g02.png}}
% 93
{\PTglyph{5}{t59_l04g03.png}}
% 94
{\PTglyph{5}{t59_l04g04.png}}
% 95
{\PTglyph{5}{t59_l04g05.png}}
% 96
{\PTglyph{5}{t59_l04g06.png}}
% 97
{\PTglyph{5}{t59_l04g07.png}}
% 98
{\PTglyph{5}{t59_l04g08.png}}
% 99
{\PTglyph{5}{t59_l04g09.png}}
% 100
{\PTglyph{5}{t59_l04g10.png}}
% 101
{\PTglyph{5}{t59_l04g11.png}}
% 102
{\PTglyph{5}{t59_l04g12.png}}
% 103
{\PTglyph{5}{t59_l04g13.png}}
% 104
{\PTglyph{5}{t59_l04g14.png}}
% 105
{\PTglyph{5}{t59_l04g15.png}}
% 106
{\PTglyph{5}{t59_l04g16.png}}
% 107
{\PTglyph{5}{t59_l04g17.png}}
% 108
{\PTglyph{5}{t59_l04g18.png}}
% 109
{\PTglyph{5}{t59_l04g19.png}}
% 110
{\PTglyph{5}{t59_l04g20.png}}
% 111
{\PTglyph{5}{t59_l04g21.png}}
% 112
{\PTglyph{5}{t59_l04g22.png}}
% 113
{\PTglyph{5}{t59_l04g23.png}}
% 114
{\PTglyph{5}{t59_l04g24.png}}
% 115
{\PTglyph{5}{t59_l04g25.png}}
% 116
{\PTglyph{5}{t59_l04g26.png}}
% 117
{\PTglyph{5}{t59_l04g27.png}}
% 118
{\PTglyph{5}{t59_l04g28.png}}
% 119
{\PTglyph{5}{t59_l04g29.png}}
% 120
{\PTglyph{5}{t59_l04g30.png}}
% 121
{\PTglyph{5}{t59_l04g31.png}}
% 122
{\PTglyph{5}{t59_l04g32.png}}
% 123
{\PTglyph{5}{t59_l04g33.png}}
//
%%% Local Variables:
%%% mode: latex
%%% TeX-engine: luatex
%%% TeX-master: shared
%%% End:

%//
%\glpismo%
 \glpismo
% 1
{\PTglyphid{U2-10_0101}}
% 2
{\PTglyphid{U2-10_0102}}
% 3
{\PTglyphid{U2-10_0103}}
% 4
{\PTglyphid{U2-10_0104}}
% 5
{\PTglyphid{U2-10_0105}}
% 6
{\PTglyphid{U2-10_0106}}
% 7
{\PTglyphid{U2-10_0107}}
% 8
{\PTglyphid{U2-10_0108}}
% 9
{\PTglyphid{U2-10_0109}}
% 10
{\PTglyphid{U2-10_0110}}
% 11
{\PTglyphid{U2-10_0111}}
% 12
{\PTglyphid{U2-10_0112}}
% 13
{\PTglyphid{U2-10_0113}}
% 14
{\PTglyphid{U2-10_0114}}
% 15
{\PTglyphid{U2-10_0115}}
% 16
{\PTglyphid{U2-10_0116}}
% 17
{\PTglyphid{U2-10_0117}}
% 18
{\PTglyphid{U2-10_0118}}
% 19
{\PTglyphid{U2-10_0119}}
% 20
{\PTglyphid{U2-10_0120}}
% 21
{\PTglyphid{U2-10_0121}}
% 22
{\PTglyphid{U2-10_0122}}
% 23
{\PTglyphid{U2-10_0201}}
% 24
{\PTglyphid{U2-10_0202}}
% 25
{\PTglyphid{U2-10_0203}}
% 26
{\PTglyphid{U2-10_0204}}
% 27
{\PTglyphid{U2-10_0205}}
% 28
{\PTglyphid{U2-10_0206}}
% 29
{\PTglyphid{U2-10_0207}}
% 30
{\PTglyphid{U2-10_0208}}
% 31
{\PTglyphid{U2-10_0209}}
% 32
{\PTglyphid{U2-10_0210}}
% 33
{\PTglyphid{U2-10_0211}}
% 34
{\PTglyphid{U2-10_0212}}
% 35
{\PTglyphid{U2-10_0213}}
% 36
{\PTglyphid{U2-10_0214}}
% 37
{\PTglyphid{U2-10_0215}}
% 38
{\PTglyphid{U2-10_0216}}
% 39
{\PTglyphid{U2-10_0217}}
% 40
{\PTglyphid{U2-10_0218}}
% 41
{\PTglyphid{U2-10_0219}}
% 42
{\PTglyphid{U2-10_0220}}
% 43
{\PTglyphid{U2-10_0221}}
% 44
{\PTglyphid{U2-10_0222}}
% 45
{\PTglyphid{U2-10_0223}}
% 46
{\PTglyphid{U2-10_0224}}
% 47
{\PTglyphid{U2-10_0225}}
% 48
{\PTglyphid{U2-10_0226}}
% 49
{\PTglyphid{U2-10_0227}}
% 50
{\PTglyphid{U2-10_0228}}
% 51
{\PTglyphid{U2-10_0229}}
% 52
{\PTglyphid{U2-10_0230}}
% 53
{\PTglyphid{U2-10_0231}}
% 54
{\PTglyphid{U2-10_0232}}
% 55
{\PTglyphid{U2-10_0233}}
% 56
{\PTglyphid{U2-10_0234}}
% 57
{\PTglyphid{U2-10_0235}}
% 58
{\PTglyphid{U2-10_0236}}
% 59
{\PTglyphid{U2-10_0301}}
% 60
{\PTglyphid{U2-10_0302}}
% 61
{\PTglyphid{U2-10_0303}}
% 62
{\PTglyphid{U2-10_0304}}
% 63
{\PTglyphid{U2-10_0305}}
% 64
{\PTglyphid{U2-10_0306}}
% 65
{\PTglyphid{U2-10_0307}}
% 66
{\PTglyphid{U2-10_0308}}
% 67
{\PTglyphid{U2-10_0309}}
% 68
{\PTglyphid{U2-10_0310}}
% 69
{\PTglyphid{U2-10_0311}}
% 70
{\PTglyphid{U2-10_0312}}
% 71
{\PTglyphid{U2-10_0313}}
% 72
{\PTglyphid{U2-10_0314}}
% 73
{\PTglyphid{U2-10_0315}}
% 74
{\PTglyphid{U2-10_0316}}
% 75
{\PTglyphid{U2-10_0317}}
% 76
{\PTglyphid{U2-10_0318}}
% 77
{\PTglyphid{U2-10_0319}}
% 78
{\PTglyphid{U2-10_0320}}
% 79
{\PTglyphid{U2-10_0321}}
% 80
{\PTglyphid{U2-10_0322}}
% 81
{\PTglyphid{U2-10_0323}}
% 82
{\PTglyphid{U2-10_0324}}
% 83
{\PTglyphid{U2-10_0325}}
% 84
{\PTglyphid{U2-10_0326}}
% 85
{\PTglyphid{U2-10_0327}}
% 86
{\PTglyphid{U2-10_0328}}
% 87
{\PTglyphid{U2-10_0329}}
% 88
{\PTglyphid{U2-10_0330}}
% 89
{\PTglyphid{U2-10_0331}}
% 90
{\PTglyphid{U2-10_0332}}
% 91
{\PTglyphid{U2-10_0401}}
% 92
{\PTglyphid{U2-10_0402}}
% 93
{\PTglyphid{U2-10_0403}}
% 94
{\PTglyphid{U2-10_0404}}
% 95
{\PTglyphid{U2-10_0405}}
% 96
{\PTglyphid{U2-10_0406}}
% 97
{\PTglyphid{U2-10_0407}}
% 98
{\PTglyphid{U2-10_0408}}
% 99
{\PTglyphid{U2-10_0409}}
% 100
{\PTglyphid{U2-10_0410}}
% 101
{\PTglyphid{U2-10_0411}}
% 102
{\PTglyphid{U2-10_0412}}
% 103
{\PTglyphid{U2-10_0413}}
% 104
{\PTglyphid{U2-10_0414}}
% 105
{\PTglyphid{U2-10_0415}}
% 106
{\PTglyphid{U2-10_0416}}
% 107
{\PTglyphid{U2-10_0417}}
% 108
{\PTglyphid{U2-10_0418}}
% 109
{\PTglyphid{U2-10_0419}}
% 110
{\PTglyphid{U2-10_0420}}
% 111
{\PTglyphid{U2-10_0421}}
% 112
{\PTglyphid{U2-10_0422}}
% 113
{\PTglyphid{U2-10_0423}}
% 114
{\PTglyphid{U2-10_0424}}
% 115
{\PTglyphid{U2-10_0425}}
% 116
{\PTglyphid{U2-10_0426}}
% 117
{\PTglyphid{U2-10_0427}}
% 118
{\PTglyphid{U2-10_0428}}
% 119
{\PTglyphid{U2-10_0429}}
% 120
{\PTglyphid{U2-10_0430}}
% 121
{\PTglyphid{U2-10_0431}}
% 122
{\PTglyphid{U2-10_0432}}
% 123
{\PTglyphid{U2-10_0433}}
//
\endgl \xe
%%% Local Variables:
%%% mode: latex
%%% TeX-engine: luatex
%%% TeX-master: shared
%%% End:

% //
%\endgl \xe



 \newpage
 
%%%%%%%%%%%%%%%%%%%%%%%%%%%%%%%%%%%%%%%%%%%%%%%%%%%%%%%%%%%%%%%%%%%%%%%%%%%%%%%
% from meta.csv
% 60,Ungler2-11_PT05_245.djvu,Ungler2,11,05,245
% 
%%%%%%%%%%%%%%%%%%%%%%%%%%%%%%%%%%%%%%%%%%%%%%%%%%%%%%%%%%%%%%%%%%%%%%%%%%%%%%%

 
% from dsed4test:
% Ungler2-11_PT05_245_4dsed.txt:Note "11. Pismo tekstowe, antykwa. Stopień 1 w. = 3 mm. — Tabl. 245.
% Ungler2-11_PT05_245_4dsed.txt:Note1 "Character set table prepared by Maria Błońska and Anna Wolińska"

 \pismoPL{Florian Ungler druga drukarnia 11. Pismo tekstowe, antykwa. Stopień 1 w. = 3 mm. — Tabl. 245.}
  
 \pismoEN{Florian Ungler second house 11. Roman text font. Type size 1 line = 3 mm. — Plate 245.}

\plate{245}{V}{1964}

The plate prepared by Henryk Bułhak.\\
The font table prepared by Henryk Bułhak,  Maria Błońska and Anna Wolińska.

\bigskip

% \exampleBib{V:66}

% \bigskip \exampleDesc{MICHAEL VRATISLAVIENSIS: Prosarum dilucidatio. Kraków, Florian Ungler, [po 1 III 1530]. 4⁰.}


% \medskip
% \examplePage{\textit{Karta 5 b.}

%   \bigskip
%   \exampleLib{Archiwum Kapituły Metropolitalnej. Kraków.}

% \bigskip \exampleRef{\textit{Estreicher XXX 359. Wierzbowski 750, 1071.}}

%  \bigskip

 % \exampleDig{\url{https://cyfrowe.mnk.pl/dlibra/publication/943/} page 9}

% Pismo 4: nagłówek. — Pismo 10: tekst i pierwszy zestaw. — Pismo 11: drugi zestaw. — Rubryka ?: z pismem 10. — Cyfry 7: z pismem 10.

% \examplePL{}


% \exampleEN{}


\bigskip

\fontID{U2-11}{60}

\fontstat{19}

% \exdisplay \bg \gla
 \exdisplay \bg \gla
% 1
{\PTglyph{5}{t60_l01g01.png}}
% 2
{\PTglyph{5}{t60_l01g02.png}}
% 3
{\PTglyph{5}{t60_l01g03.png}}
% 4
{\PTglyph{5}{t60_l01g04.png}}
% 5
{\PTglyph{5}{t60_l01g05.png}}
% 6
{\PTglyph{5}{t60_l01g06.png}}
% 7
{\PTglyph{5}{t60_l01g07.png}}
% 8
{\PTglyph{5}{t60_l01g08.png}}
% 9
{\PTglyph{5}{t60_l01g09.png}}
% 10
{\PTglyph{5}{t60_l01g10.png}}
% 11
{\PTglyph{5}{t60_l01g11.png}}
% 12
{\PTglyph{5}{t60_l01g12.png}}
% 13
{\PTglyph{5}{t60_l01g13.png}}
% 14
{\PTglyph{5}{t60_l01g14.png}}
% 15
{\PTglyph{5}{t60_l01g15.png}}
% 16
{\PTglyph{5}{t60_l01g16.png}}
% 17
{\PTglyph{5}{t60_l01g17.png}}
% 18
{\PTglyph{5}{t60_l01g18.png}}
% 19
{\PTglyph{5}{t60_l01g19.png}}
//
%%% Local Variables:
%%% mode: latex
%%% TeX-engine: luatex
%%% TeX-master: shared
%%% End:

%//
%\glpismo%
 \glpismo
% 1
{\PTglyphid{U2-11_0101}}
% 2
{\PTglyphid{U2-11_0102}}
% 3
{\PTglyphid{U2-11_0103}}
% 4
{\PTglyphid{U2-11_0104}}
% 5
{\PTglyphid{U2-11_0105}}
% 6
{\PTglyphid{U2-11_0106}}
% 7
{\PTglyphid{U2-11_0107}}
% 8
{\PTglyphid{U2-11_0108}}
% 9
{\PTglyphid{U2-11_0109}}
% 10
{\PTglyphid{U2-11_0110}}
% 11
{\PTglyphid{U2-11_0111}}
% 12
{\PTglyphid{U2-11_0112}}
% 13
{\PTglyphid{U2-11_0113}}
% 14
{\PTglyphid{U2-11_0114}}
% 15
{\PTglyphid{U2-11_0115}}
% 16
{\PTglyphid{U2-11_0116}}
% 17
{\PTglyphid{U2-11_0117}}
% 18
{\PTglyphid{U2-11_0118}}
% 19
{\PTglyphid{U2-11_0119}}
//
\endgl \xe
%%% Local Variables:
%%% mode: latex
%%% TeX-engine: luatex
%%% TeX-master: shared
%%% End:

% //
%\endgl \xe



 \newpage
 
%%%%%%%%%%%%%%%%%%%%%%%%%%%%%%%%%%%%%%%%%%%%%%%%%%%%%%%%%%%%%%%%%%%%%%%%%%%%%%%
% from meta.csv
% 61,Ungler2-12_PT05_243.djvu,Ungler2,12,05,243
% 
%%%%%%%%%%%%%%%%%%%%%%%%%%%%%%%%%%%%%%%%%%%%%%%%%%%%%%%%%%%%%%%%%%%%%%%%%%%%%%%

 
% from dsed4test:
% Ungler2-12_PT05_243_4dsed.txt:Note "12. Wersaliki tytułowe, antykwa. Wysokość 8 mm. — Tabl. 243."
% Ungler2-12_PT05_243_4dsed.txt:Note1 "Character set table prepared by Anna Wolińska." !!!

 \pismoPL{Florian Ungler druga drukarnia 12. Wersaliki tytułowe, antykwa. Wysokość 8 mm. — Tabl. 243.}
  
 \pismoEN{Florian Ungler second house 10. Roman title capitals font. Type size [1 line =] 8 mm. — Plate 243.}

\plate{243}{V}{1964}

The plate prepared by Henryk Bułhak.\\
The font table prepared by Henryk Bułhak and Anna Wolińska.

\bigskip

% \exampleBib{V:66}

% \bigskip \exampleDesc{MICHAEL VRATISLAVIENSIS: Prosarum dilucidatio. Kraków, Florian Ungler, [po 1 III 1530]. 4⁰.}


% \medskip
% \examplePage{\textit{Karta 5 b.}

%   \bigskip
%   \exampleLib{Archiwum Kapituły Metropolitalnej. Kraków.}

% \bigskip \exampleRef{\textit{Estreicher XXX 359. Wierzbowski 750, 1071.}}

%  \bigskip

 % \exampleDig{\url{https://cyfrowe.mnk.pl/dlibra/publication/943/} page 9}

%  Pismo 5: tekst i pierwszy zestaw. — Rubryka a: z pismem 5. — Pismo 12: trzeci zestaw. — Pismo 13: drugi zestaw. — Przerywnik 1: z pismem 12.

% \examplePL{}


% \exampleEN{}


\bigskip

\fontID{U2-13}{62}

\fontstat{?}

% \exdisplay \bg \gla
 \exdisplay \bg \gla
% 1
{\PTglyph{5}{t61_l01g01.png}}
% 2
{\PTglyph{5}{t61_l01g02.png}}
% 3
{\PTglyph{5}{t61_l01g03.png}}
% 4
{\PTglyph{5}{t61_l01g04.png}}
% 5
{\PTglyph{5}{t61_l01g05.png}}
% 6
{\PTglyph{5}{t61_l01g06.png}}
% 7
{\PTglyph{5}{t61_l01g07.png}}
% 8
{\PTglyph{5}{t61_l01g08.png}}
% 9
{\PTglyph{5}{t61_l01g09.png}}
% 10
{\PTglyph{5}{t61_l01g10.png}}
% 11
{\PTglyph{5}{t61_l01g11.png}}
% 12
{\PTglyph{5}{t61_l01g12.png}}
% 13
{\PTglyph{5}{t61_l01g13.png}}
% 14
{\PTglyph{5}{t61_l01g14.png}}
% 15
{\PTglyph{5}{t61_l01g15.png}}
% 16
{\PTglyph{5}{t61_l01g16.png}}
% 17
{\PTglyph{5}{t61_l01g17.png}}
% 18
{\PTglyph{5}{t61_l01g18.png}}
//
%%% Local Variables:
%%% mode: latex
%%% TeX-engine: luatex
%%% TeX-master: shared
%%% End:

%//
%\glpismo%
 \glpismo
% 1
{\PTglyphid{U2-12_0101}}
% 2
{\PTglyphid{U2-12_0102}}
% 3
{\PTglyphid{U2-12_0103}}
% 4
{\PTglyphid{U2-12_0104}}
% 5
{\PTglyphid{U2-12_0105}}
% 6
{\PTglyphid{U2-12_0106}}
% 7
{\PTglyphid{U2-12_0107}}
% 8
{\PTglyphid{U2-12_0108}}
% 9
{\PTglyphid{U2-12_0109}}
% 10
{\PTglyphid{U2-12_0110}}
% 11
{\PTglyphid{U2-12_0111}}
% 12
{\PTglyphid{U2-12_0112}}
% 13
{\PTglyphid{U2-12_0113}}
% 14
{\PTglyphid{U2-12_0114}}
% 15
{\PTglyphid{U2-12_0115}}
% 16
{\PTglyphid{U2-12_0116}}
% 17
{\PTglyphid{U2-12_0117}}
% 18
{\PTglyphid{U2-12_0118}}
//
\endgl \xe
%%% Local Variables:
%%% mode: latex
%%% TeX-engine: luatex
%%% TeX-master: shared
%%% End:

% //
%\endgl \xe


 \newpage
 
%%%%%%%%%%%%%%%%%%%%%%%%%%%%%%%%%%%%%%%%%%%%%%%%%%%%%%%%%%%%%%%%%%%%%%%%%%%%%%%
% from meta.csv
% 62,Ungler2-13_PT05_243.djvu,Ungler2,13,05,243
% 
%%%%%%%%%%%%%%%%%%%%%%%%%%%%%%%%%%%%%%%%%%%%%%%%%%%%%%%%%%%%%%%%%%%%%%%%%%%%%%%

 
% from dsed4test:
% Ungler2-13_PT05_243_4dsed.txt:Note "13. Wersaliki tytułowe, antykwa. Wysokość 8 mm. — Tabl. 243."
% Ungler2-13_PT05_243_4dsed.txt:Note1 "Character set table prepared by Maria Błońska"

 \pismoPL{Florian Ungler druga drukarnia 13. Wersaliki tytułowe, antykwa. Wysokość 8 mm. — Tabl. 243.}
  
 \pismoEN{Florian Ungler second house 13. Roman title capitals font. Type size [1 line =] 8 mm. — Plate 243.}

\plate{243}{V}{1964}

The plate prepared by Henryk Bułhak.\\
The font table prepared by Henryk Bułhak and Maria Błońska.

\bigskip

% \exampleBib{V:66}

% \bigskip \exampleDesc{MICHAEL VRATISLAVIENSIS: Prosarum dilucidatio. Kraków, Florian Ungler, [po 1 III 1530]. 4⁰.}


% \medskip
% \examplePage{\textit{Karta 5 b.}

%   \bigskip
%   \exampleLib{Archiwum Kapituły Metropolitalnej. Kraków.}

% \bigskip \exampleRef{\textit{Estreicher XXX 359. Wierzbowski 750, 1071.}}

%  \bigskip

 % \exampleDig{\url{https://cyfrowe.mnk.pl/dlibra/publication/943/} page 9}

%  Pismo 5: tekst i pierwszy zestaw. — Rubryka a: z pismem 5. — Pismo 12: trzeci zestaw. — Pismo 13: drugi zestaw. — Przerywnik 1: z pismem 12.

% \examplePL{}


% \exampleEN{}


\bigskip

\fontID{U2-13}{62}

\fontstat{23}

% \exdisplay \bg \gla
 \exdisplay \bg \gla
% 1
{\PTglyph{5}{t62_l01g01.png}}
% 2
{\PTglyph{5}{t62_l01g02.png}}
% 3
{\PTglyph{5}{t62_l01g03.png}}
% 4
{\PTglyph{5}{t62_l01g04.png}}
% 5
{\PTglyph{5}{t62_l01g05.png}}
% 6
{\PTglyph{5}{t62_l01g06.png}}
% 7
{\PTglyph{5}{t62_l01g07.png}}
% 8
{\PTglyph{5}{t62_l01g08.png}}
% 9
{\PTglyph{5}{t62_l01g09.png}}
% 10
{\PTglyph{5}{t62_l01g10.png}}
% 11
{\PTglyph{5}{t62_l01g11.png}}
% 12
{\PTglyph{5}{t62_l01g12.png}}
% 13
{\PTglyph{5}{t62_l01g13.png}}
% 14
{\PTglyph{5}{t62_l01g14.png}}
% 15
{\PTglyph{5}{t62_l01g15.png}}
% 16
{\PTglyph{5}{t62_l02g01.png}}
% 17
{\PTglyph{5}{t62_l02g02.png}}
% 18
{\PTglyph{5}{t62_l02g03.png}}
% 19
{\PTglyph{5}{t62_l02g04.png}}
% 20
{\PTglyph{5}{t62_l02g05.png}}
% 21
{\PTglyph{5}{t62_l02g06.png}}
% 22
{\PTglyph{5}{t62_l02g07.png}}
% 23
{\PTglyph{5}{t62_l02g08.png}}
//
%%% Local Variables:
%%% mode: latex
%%% TeX-engine: luatex
%%% TeX-master: shared
%%% End:

%//
%\glpismo%
 \glpismo
% 1
{\PTglyphid{U2-13_0101}}
% 2
{\PTglyphid{U2-13_0102}}
% 3
{\PTglyphid{U2-13_0103}}
% 4
{\PTglyphid{U2-13_0104}}
% 5
{\PTglyphid{U2-13_0105}}
% 6
{\PTglyphid{U2-13_0106}}
% 7
{\PTglyphid{U2-13_0107}}
% 8
{\PTglyphid{U2-13_0108}}
% 9
{\PTglyphid{U2-13_0109}}
% 10
{\PTglyphid{U2-13_0110}}
% 11
{\PTglyphid{U2-13_0111}}
% 12
{\PTglyphid{U2-13_0112}}
% 13
{\PTglyphid{U2-13_0113}}
% 14
{\PTglyphid{U2-13_0114}}
% 15
{\PTglyphid{U2-13_0115}}
% 16
{\PTglyphid{U2-13_0201}}
% 17
{\PTglyphid{U2-13_0202}}
% 18
{\PTglyphid{U2-13_0203}}
% 19
{\PTglyphid{U2-13_0204}}
% 20
{\PTglyphid{U2-13_0205}}
% 21
{\PTglyphid{U2-13_0206}}
% 22
{\PTglyphid{U2-13_0207}}
% 23
{\PTglyphid{U2-13_0208}}
//
\endgl \xe
%%% Local Variables:
%%% mode: latex
%%% TeX-engine: luatex
%%% TeX-master: shared
%%% End:

% //
%\endgl \xe


 \newpage
 
%%%%%%%%%%%%%%%%%%%%%%%%%%%%%%%%%%%%%%%%%%%%%%%%%%%%%%%%%%%%%%%%%%%%%%%%%%%%%%%
% from meta.csv
% 63,Ungler2-14_PT07_357.djvu,Ungler2,14,07,357
% 
%%%%%%%%%%%%%%%%%%%%%%%%%%%%%%%%%%%%%%%%%%%%%%%%%%%%%%%%%%%%%%%%%%%%%%%%%%%%%%%
Przemianować wiersze (i matryce?)
 
% from dsed4test:
%  Ungler2-14_PT07_357_4dsed.txt:Note "14.  Pismo tytułowe i nagłówkowe, fraktura Neudörffer-Andreae. Stopień 5 ww. = 65 mm. — Tabl. 357."
% Ungler2-14_PT07_357_4dsed.txt:Note1 "Character set table prepared by Anna Wolińska"

 \pismoPL{Florian Ungler druga drukarnia 14.  Pismo tytułowe i
   nagłówkowe, fraktura Neudörffer-Andreae. Stopień 5 ww. = 65 mm. —
   Tabl. 357.}
  
 \pismoEN{Florian Ungler second house 14. Title and header font, Neudörffer-Andreae Fraktur. Type size 5 lines = 65 mm. — Plate 357.}

\plate{357}{VII}{1970}

The plate prepared by Henryk Bułhak.\\
The font table prepared by Henryk Bułhak and Anna Wolińska.

\bigskip

% \exampleBib{VII:113}

% \bigskip \exampleDesc{ADAM TUSSINUS a Tarnów: ludicium et significatio cometae qui apparuit in fine dierum Iunii sub anno 1533.
% Kraków, Florian Ungler, [po 26 VI 1533]. 8⁰.}


%  \medskip
%  \examplePage{\textit{Karta B₂b.}}

%   \bigskip
%   \exampleLib{Biblioteka Narodowa. Warszawa.}

%  \bigskip \exampleRef{\textit{Estreicher XXXI 425. Wierzbowski 1103.}}

%  \bigskip

 % \exampleDig{\url{https://cyfrowe.mnk.pl/dlibra/publication/943/} page 9}

 % Pismo 14: drugi zestaw. — Pismo 17: kolumna tekstu i pierwszy
 % zestaw. — Pismo 18: nagłówek (wiersz 1). — Pismo 20: nagłówek
 % (wiersz 2). — Rubryka &:
% z pismem 17. — Cyfry 14: z pismem 14.

% \examplePL{}


% \exampleEN{}


\bigskip

\fontID{U2-14}{63}

\fontstat{129?}

% \exdisplay \bg \gla
 \exdisplay \bg \gla
% 1
{\PTglyph{5}{t63_l01g01.png}}
% 2
{\PTglyph{5}{t63_l01g02.png}}
% 3
{\PTglyph{5}{t63_l01g03.png}}
% 4
{\PTglyph{5}{t63_l01g04.png}}
% 5
{\PTglyph{5}{t63_l01g05.png}}
% 6
{\PTglyph{5}{t63_l01g06.png}}
% 7
{\PTglyph{5}{t63_l01g07.png}}
% 8
{\PTglyph{5}{t63_l01g08.png}}
% 9
{\PTglyph{5}{t63_l01g09.png}}
% 10
{\PTglyph{5}{t63_l01g10.png}}
% 11
{\PTglyph{5}{t63_l01g11.png}}
% 12
{\PTglyph{5}{t63_l01g12.png}}
% 13
{\PTglyph{5}{t63_l01g13.png}}
% 14
{\PTglyph{5}{t63_l01g14.png}}
% 15
{\PTglyph{5}{t63_l01g15.png}}
% 16
{\PTglyph{5}{t63_l01g16.png}}
% 17
{\PTglyph{5}{t63_l01g17.png}}
% 18
{\PTglyph{5}{t63_l01g18.png}}
% 19
{\PTglyph{5}{t63_l01g19.png}}
% 20
{\PTglyph{5}{t63_l01g20.png}}
% 21
{\PTglyph{5}{t63_l01g21.png}}
% 22
{\PTglyph{5}{t63_l01g22.png}}
% 23
{\PTglyph{5}{t63_l01g23.png}}
% 24
{\PTglyph{5}{t63_l01g24.png}}
% 25
{\PTglyph{5}{t63_l01g25.png}}
% 26
{\PTglyph{5}{t63_l01g26.png}}
% 27
{\PTglyph{5}{t63_l01g27.png}}
% 28
{\PTglyph{5}{t63_l02g01.png}}
% 29
{\PTglyph{5}{t63_l02g02.png}}
% 30
{\PTglyph{5}{t63_l02g03.png}}
% 31
{\PTglyph{5}{t63_l02g04.png}}
% 32
{\PTglyph{5}{t63_l02g05.png}}
% 33
{\PTglyph{5}{t63_l02g06.png}}
% 34
{\PTglyph{5}{t63_l02g07.png}}
% 35
{\PTglyph{5}{t63_l02g08.png}}
% 36
{\PTglyph{5}{t63_l02g09.png}}
% 37
{\PTglyph{5}{t63_l02g10.png}}
% 38
{\PTglyph{5}{t63_l02g11.png}}
% 39
{\PTglyph{5}{t63_l02g12.png}}
% 40
{\PTglyph{5}{t63_l02g13.png}}
% 41
{\PTglyph{5}{t63_l02g14.png}}
% 42
{\PTglyph{5}{t63_l02g15.png}}
% 43
{\PTglyph{5}{t63_l02g16.png}}
% 44
{\PTglyph{5}{t63_l02g17.png}}
% 45
{\PTglyph{5}{t63_l02g18.png}}
% 46
{\PTglyph{5}{t63_l02g19.png}}
% 47
{\PTglyph{5}{t63_l02g20.png}}
% 48
{\PTglyph{5}{t63_l02g21.png}}
% 49
{\PTglyph{5}{t63_l02g22.png}}
% 50
{\PTglyph{5}{t63_l02g23.png}}
% 51
{\PTglyph{5}{t63_l02g24.png}}
% 52
{\PTglyph{5}{t63_l02g25.png}}
% 53
{\PTglyph{5}{t63_l02g26.png}}
% 54
{\PTglyph{5}{t63_l02g27.png}}
% 55
{\PTglyph{5}{t63_l02g28.png}}
% 56
{\PTglyph{5}{t63_l02g29.png}}
% 57
{\PTglyph{5}{t63_l02g30.png}}
% 58
{\PTglyph{5}{t63_l02g31.png}}
% 59
{\PTglyph{5}{t63_l02g32.png}}
% 60
{\PTglyph{5}{t63_l02g33.png}}
% 61
{\PTglyph{5}{t63_l02g34.png}}
% 62
{\PTglyph{5}{t63_l02g35.png}}
% 63
{\PTglyph{5}{t63_l02g36.png}}
% 64
{\PTglyph{5}{t63_l02g37.png}}
% 65
{\PTglyph{5}{t63_l02g38.png}}
% 66
{\PTglyph{5}{t63_l03g01.png}}
% 67
{\PTglyph{5}{t63_l03g02.png}}
% 68
{\PTglyph{5}{t63_l03g03.png}}
% 69
{\PTglyph{5}{t63_l03g04.png}}
% 70
{\PTglyph{5}{t63_l03g05.png}}
% 71
{\PTglyph{5}{t63_l03g06.png}}
% 72
{\PTglyph{5}{t63_l03g07.png}}
% 73
{\PTglyph{5}{t63_l03g08.png}}
% 74
{\PTglyph{5}{t63_l03g09.png}}
% 75
{\PTglyph{5}{t63_l03g10.png}}
% 76
{\PTglyph{5}{t63_l03g11.png}}
% 77
{\PTglyph{5}{t63_l03g12.png}}
% 78
{\PTglyph{5}{t63_l03g13.png}}
% 79
{\PTglyph{5}{t63_l03g14.png}}
% 80
{\PTglyph{5}{t63_l03g15.png}}
% 81
{\PTglyph{5}{t63_l03g16.png}}
% 82
{\PTglyph{5}{t63_l03g17.png}}
% 83
{\PTglyph{5}{t63_l03g18.png}}
% 84
{\PTglyph{5}{t63_l03g19.png}}
% 85
{\PTglyph{5}{t63_l03g20.png}}
% 86
{\PTglyph{5}{t63_l03g21.png}}
% 87
{\PTglyph{5}{t63_l03g22.png}}
% 88
{\PTglyph{5}{t63_l03g23.png}}
% 89
{\PTglyph{5}{t63_l03g24.png}}
% 90
{\PTglyph{5}{t63_l03g25.png}}
% 91
{\PTglyph{5}{t63_l03g26.png}}
% 92
{\PTglyph{5}{t63_l03g27.png}}
% 93
{\PTglyph{5}{t63_l03g28.png}}
% 94
{\PTglyph{5}{t63_l03g29.png}}
% 95
{\PTglyph{5}{t63_l03g30.png}}
% 96
{\PTglyph{5}{t63_l03g31.png}}
% 97
{\PTglyph{5}{t63_l03g32.png}}
% 98
{\PTglyph{5}{t63_l03g33.png}}
% 99
{\PTglyph{5}{t63_l03g34.png}}
% 100
{\PTglyph{5}{t63_l03g35.png}}
% 101
{\PTglyph{5}{t63_l03g36.png}}
% 102
{\PTglyph{5}{t63_l04g01.png}}
% 103
{\PTglyph{5}{t63_l04g02.png}}
% 104
{\PTglyph{5}{t63_l04g03.png}}
% 105
{\PTglyph{5}{t63_l04g04.png}}
% 106
{\PTglyph{5}{t63_l04g05.png}}
% 107
{\PTglyph{5}{t63_l04g06.png}}
% 108
{\PTglyph{5}{t63_l04g07.png}}
% 109
{\PTglyph{5}{t63_l04g08.png}}
% 110
{\PTglyph{5}{t63_l04g09.png}}
% 111
{\PTglyph{5}{t63_l04g10.png}}
% 112
{\PTglyph{5}{t63_l04g11.png}}
% 113
{\PTglyph{5}{t63_l04g12.png}}
% 114
{\PTglyph{5}{t63_l04g13.png}}
% 115
{\PTglyph{5}{t63_l04g14.png}}
% 116
{\PTglyph{5}{t63_l04g15.png}}
% 117
{\PTglyph{5}{t63_l04g16.png}}
% 118
{\PTglyph{5}{t63_l04g17.png}}
% 119
{\PTglyph{5}{t63_l04g18.png}}
% 120
{\PTglyph{5}{t63_l04g19.png}}
% 121
{\PTglyph{5}{t63_l04g20.png}}
% 122
{\PTglyph{5}{t63_l04g21.png}}
% 123
{\PTglyph{5}{t63_l04g22.png}}
% 124
{\PTglyph{5}{t63_l04g23.png}}
% 125
{\PTglyph{5}{t63_l04g24.png}}
% 126
{\PTglyph{5}{t63_l04g25.png}}
% 127
{\PTglyph{5}{t63_l04g26.png}}
% 128
{\PTglyph{5}{t63_l04g27.png}}
% 129
{\PTglyph{5}{t63_l04g28.png}}
//
%%% Local Variables:
%%% mode: latex
%%% TeX-engine: luatex
%%% TeX-master: shared
%%% End:

%//
%\glpismo%
 \glpismo
% 1
{\PTglyphid{U2-14_0101}}
% 2
{\PTglyphid{U2-14_0102}}
% 3
{\PTglyphid{U2-14_0103}}
% 4
{\PTglyphid{U2-14_0104}}
% 5
{\PTglyphid{U2-14_0105}}
% 6
{\PTglyphid{U2-14_0106}}
% 7
{\PTglyphid{U2-14_0107}}
% 8
{\PTglyphid{U2-14_0108}}
% 9
{\PTglyphid{U2-14_0109}}
% 10
{\PTglyphid{U2-14_0110}}
% 11
{\PTglyphid{U2-14_0111}}
% 12
{\PTglyphid{U2-14_0112}}
% 13
{\PTglyphid{U2-14_0113}}
% 14
{\PTglyphid{U2-14_0114}}
% 15
{\PTglyphid{U2-14_0115}}
% 16
{\PTglyphid{U2-14_0116}}
% 17
{\PTglyphid{U2-14_0117}}
% 18
{\PTglyphid{U2-14_0118}}
% 19
{\PTglyphid{U2-14_0119}}
% 20
{\PTglyphid{U2-14_0120}}
% 21
{\PTglyphid{U2-14_0121}}
% 22
{\PTglyphid{U2-14_0122}}
% 23
{\PTglyphid{U2-14_0123}}
% 24
{\PTglyphid{U2-14_0124}}
% 25
{\PTglyphid{U2-14_0125}}
% 26
{\PTglyphid{U2-14_0126}}
% 27
{\PTglyphid{U2-14_0127}}
% 28
{\PTglyphid{U2-14_0201}}
% 29
{\PTglyphid{U2-14_0202}}
% 30
{\PTglyphid{U2-14_0203}}
% 31
{\PTglyphid{U2-14_0204}}
% 32
{\PTglyphid{U2-14_0205}}
% 33
{\PTglyphid{U2-14_0206}}
% 34
{\PTglyphid{U2-14_0207}}
% 35
{\PTglyphid{U2-14_0208}}
% 36
{\PTglyphid{U2-14_0209}}
% 37
{\PTglyphid{U2-14_0210}}
% 38
{\PTglyphid{U2-14_0211}}
% 39
{\PTglyphid{U2-14_0212}}
% 40
{\PTglyphid{U2-14_0213}}
% 41
{\PTglyphid{U2-14_0214}}
% 42
{\PTglyphid{U2-14_0215}}
% 43
{\PTglyphid{U2-14_0216}}
% 44
{\PTglyphid{U2-14_0217}}
% 45
{\PTglyphid{U2-14_0218}}
% 46
{\PTglyphid{U2-14_0219}}
% 47
{\PTglyphid{U2-14_0220}}
% 48
{\PTglyphid{U2-14_0221}}
% 49
{\PTglyphid{U2-14_0222}}
% 50
{\PTglyphid{U2-14_0223}}
% 51
{\PTglyphid{U2-14_0224}}
% 52
{\PTglyphid{U2-14_0225}}
% 53
{\PTglyphid{U2-14_0226}}
% 54
{\PTglyphid{U2-14_0227}}
% 55
{\PTglyphid{U2-14_0228}}
% 56
{\PTglyphid{U2-14_0229}}
% 57
{\PTglyphid{U2-14_0230}}
% 58
{\PTglyphid{U2-14_0231}}
% 59
{\PTglyphid{U2-14_0232}}
% 60
{\PTglyphid{U2-14_0233}}
% 61
{\PTglyphid{U2-14_0234}}
% 62
{\PTglyphid{U2-14_0235}}
% 63
{\PTglyphid{U2-14_0236}}
% 64
{\PTglyphid{U2-14_0237}}
% 65
{\PTglyphid{U2-14_0238}}
% 66
{\PTglyphid{U2-14_0301}}
% 67
{\PTglyphid{U2-14_0302}}
% 68
{\PTglyphid{U2-14_0303}}
% 69
{\PTglyphid{U2-14_0304}}
% 70
{\PTglyphid{U2-14_0305}}
% 71
{\PTglyphid{U2-14_0306}}
% 72
{\PTglyphid{U2-14_0307}}
% 73
{\PTglyphid{U2-14_0308}}
% 74
{\PTglyphid{U2-14_0309}}
% 75
{\PTglyphid{U2-14_0310}}
% 76
{\PTglyphid{U2-14_0311}}
% 77
{\PTglyphid{U2-14_0312}}
% 78
{\PTglyphid{U2-14_0313}}
% 79
{\PTglyphid{U2-14_0314}}
% 80
{\PTglyphid{U2-14_0315}}
% 81
{\PTglyphid{U2-14_0316}}
% 82
{\PTglyphid{U2-14_0317}}
% 83
{\PTglyphid{U2-14_0318}}
% 84
{\PTglyphid{U2-14_0319}}
% 85
{\PTglyphid{U2-14_0320}}
% 86
{\PTglyphid{U2-14_0321}}
% 87
{\PTglyphid{U2-14_0322}}
% 88
{\PTglyphid{U2-14_0323}}
% 89
{\PTglyphid{U2-14_0324}}
% 90
{\PTglyphid{U2-14_0325}}
% 91
{\PTglyphid{U2-14_0326}}
% 92
{\PTglyphid{U2-14_0327}}
% 93
{\PTglyphid{U2-14_0328}}
% 94
{\PTglyphid{U2-14_0329}}
% 95
{\PTglyphid{U2-14_0330}}
% 96
{\PTglyphid{U2-14_0331}}
% 97
{\PTglyphid{U2-14_0332}}
% 98
{\PTglyphid{U2-14_0333}}
% 99
{\PTglyphid{U2-14_0334}}
% 100
{\PTglyphid{U2-14_0335}}
% 101
{\PTglyphid{U2-14_0336}}
% 102
{\PTglyphid{U2-14_0401}}
% 103
{\PTglyphid{U2-14_0402}}
% 104
{\PTglyphid{U2-14_0403}}
% 105
{\PTglyphid{U2-14_0404}}
% 106
{\PTglyphid{U2-14_0405}}
% 107
{\PTglyphid{U2-14_0406}}
% 108
{\PTglyphid{U2-14_0407}}
% 109
{\PTglyphid{U2-14_0408}}
% 110
{\PTglyphid{U2-14_0409}}
% 111
{\PTglyphid{U2-14_0410}}
% 112
{\PTglyphid{U2-14_0411}}
% 113
{\PTglyphid{U2-14_0412}}
% 114
{\PTglyphid{U2-14_0413}}
% 115
{\PTglyphid{U2-14_0414}}
% 116
{\PTglyphid{U2-14_0415}}
% 117
{\PTglyphid{U2-14_0416}}
% 118
{\PTglyphid{U2-14_0417}}
% 119
{\PTglyphid{U2-14_0418}}
% 120
{\PTglyphid{U2-14_0419}}
% 121
{\PTglyphid{U2-14_0420}}
% 122
{\PTglyphid{U2-14_0421}}
% 123
{\PTglyphid{U2-14_0422}}
% 124
{\PTglyphid{U2-14_0423}}
% 125
{\PTglyphid{U2-14_0424}}
% 126
{\PTglyphid{U2-14_0425}}
% 127
{\PTglyphid{U2-14_0426}}
% 128
{\PTglyphid{U2-14_0427}}
% 129
{\PTglyphid{U2-14_0428}}
//
\endgl \xe
%%% Local Variables:
%%% mode: latex
%%% TeX-engine: luatex
%%% TeX-master: shared
%%% End:

% //
%\endgl \xe



 \newpage
 
%%%%%%%%%%%%%%%%%%%%%%%%%%%%%%%%%%%%%%%%%%%%%%%%%%%%%%%%%%%%%%%%%%%%%%%%%%%%%%%
% from meta.csv
% 64,Ungler2-15_PT07_358.djvu,Ungler2,15,07,358
% 
%%%%%%%%%%%%%%%%%%%%%%%%%%%%%%%%%%%%%%%%%%%%%%%%%%%%%%%%%%%%%%%%%%%%%%%%%%%%%%%

 
% from dsed4test:
% Ungler2-15_PT07_358_4dsed.txt:Note "15 a-b.Pismo tekstowe, antykwa Qu|(G). Stopień 20 ww. = 91 mm. — Tabl. 358."
% Ungler2-15_PT07_358_4dsed.txt:Note1 "Character set table prepared by Anna Wolińska"

 \pismoPL{Florian Ungler druga drukarnia 15${a-b}$. Pismo tekstowe, antykwa Qu|(G). Stopień 20 ww. = 91 mm. — Tabl. 358.}
  
 \pismoEN{Florian Ungler second house 15${a-b}$. Roman text font, [typeface] Qu|(G). Type size 20 lines = 91 mm. — Plate 358.}

\plate{358}{VII}{1970}

The plate prepared by Henryk Bułhak.\\
The font table prepared by Henryk Bułhak and Anna Wolińska.

\bigskip

15${a}$: \exampleBib{VII:69}

\bigskip \exampleDesc{MATHEOLUS PERUSINUS: De memoria augenda. [Kraków, Florian Ungler, po 22 X 1530]. 8⁰.}


 \medskip
 \examplePage{\textit{Karta A₃b.}}

  \bigskip
  \exampleLib{Biblioteka Narodowa. Warszawa.}

  \bigskip \exampleRef{\textit{Estreicher XXII 231.}}

  % Pismo 15%, — Rubryka t. — Cyfry 10% (w zestawie).

\bigskip\bigskip  
15${b}$: \exampleBib{VII:76}

\bigskip \exampleDesc{MODUS legendi abbreviaturas. Kraków, Florian Ungler, [1531]. 8⁰.}

% https://www.wbc.poznan.pl/dlibra/publication/496742/edition/434947 1524?
% https://www.ariadna.bs.katowice.pl/details/1f6fb9232dadf5725a77
%https://integro.bs.katowice.pl/site/recorddetail/0033105534713
 \medskip
 \examplePage{\textit{Karta tytułowa verso.}}

  \bigskip
  \exampleLib{Biblioteka PAN. Kraków (76).}

  \bigskip \exampleRef{\textit{Estreicher XXXI 271. Wierzbowski 782.}}

%Pismo 15%. — Inicjał 34.

  % 
  
%  \bigskip

 % \exampleDig{\url{https://cyfrowe.mnk.pl/dlibra/publication/943/} page 9}

 % Pismo 14: drugi zestaw. — Pismo 17: kolumna tekstu i pierwszy
 % zestaw. — Pismo 18: nagłówek (wiersz 1). — Pismo 20: nagłówek
 % (wiersz 2). — Rubryka &:
% z pismem 17. — Cyfry 14: z pismem 14.

% \examplePL{}


% \exampleEN{}


\bigskip

\fontID{U2-15}{64}

\fontstat{107}

% \exdisplay \bg \gla
 \exdisplay \bg \gla
% 1
{\PTglyph{5}{t64_l01g01.png}}
% 2
{\PTglyph{5}{t64_l01g02.png}}
% 3
{\PTglyph{5}{t64_l01g03.png}}
% 4
{\PTglyph{5}{t64_l01g04.png}}
% 5
{\PTglyph{5}{t64_l01g05.png}}
% 6
{\PTglyph{5}{t64_l01g06.png}}
% 7
{\PTglyph{5}{t64_l01g07.png}}
% 8
{\PTglyph{5}{t64_l01g08.png}}
% 9
{\PTglyph{5}{t64_l01g09.png}}
% 10
{\PTglyph{5}{t64_l01g10.png}}
% 11
{\PTglyph{5}{t64_l01g11.png}}
% 12
{\PTglyph{5}{t64_l01g12.png}}
% 13
{\PTglyph{5}{t64_l01g13.png}}
% 14
{\PTglyph{5}{t64_l01g14.png}}
% 15
{\PTglyph{5}{t64_l01g15.png}}
% 16
{\PTglyph{5}{t64_l01g16.png}}
% 17
{\PTglyph{5}{t64_l01g17.png}}
% 18
{\PTglyph{5}{t64_l01g18.png}}
% 19
{\PTglyph{5}{t64_l01g19.png}}
% 20
{\PTglyph{5}{t64_l01g20.png}}
% 21
{\PTglyph{5}{t64_l01g21.png}}
% 22
{\PTglyph{5}{t64_l01g22.png}}
% 23
{\PTglyph{5}{t64_l02g01.png}}
% 24
{\PTglyph{5}{t64_l02g02.png}}
% 25
{\PTglyph{5}{t64_l02g03.png}}
% 26
{\PTglyph{5}{t64_l02g04.png}}
% 27
{\PTglyph{5}{t64_l02g05.png}}
% 28
{\PTglyph{5}{t64_l02g06.png}}
% 29
{\PTglyph{5}{t64_l02g07.png}}
% 30
{\PTglyph{5}{t64_l02g08.png}}
% 31
{\PTglyph{5}{t64_l02g09.png}}
% 32
{\PTglyph{5}{t64_l02g10.png}}
% 33
{\PTglyph{5}{t64_l02g11.png}}
% 34
{\PTglyph{5}{t64_l02g12.png}}
% 35
{\PTglyph{5}{t64_l02g13.png}}
% 36
{\PTglyph{5}{t64_l02g14.png}}
% 37
{\PTglyph{5}{t64_l02g15.png}}
% 38
{\PTglyph{5}{t64_l02g16.png}}
% 39
{\PTglyph{5}{t64_l02g17.png}}
% 40
{\PTglyph{5}{t64_l02g18.png}}
% 41
{\PTglyph{5}{t64_l02g19.png}}
% 42
{\PTglyph{5}{t64_l02g20.png}}
% 43
{\PTglyph{5}{t64_l02g21.png}}
% 44
{\PTglyph{5}{t64_l02g22.png}}
% 45
{\PTglyph{5}{t64_l02g23.png}}
% 46
{\PTglyph{5}{t64_l02g24.png}}
% 47
{\PTglyph{5}{t64_l02g25.png}}
% 48
{\PTglyph{5}{t64_l02g26.png}}
% 49
{\PTglyph{5}{t64_l02g27.png}}
% 50
{\PTglyph{5}{t64_l02g28.png}}
% 51
{\PTglyph{5}{t64_l02g29.png}}
% 52
{\PTglyph{5}{t64_l02g30.png}}
% 53
{\PTglyph{5}{t64_l02g31.png}}
% 54
{\PTglyph{5}{t64_l02g32.png}}
% 55
{\PTglyph{5}{t64_l02g33.png}}
% 56
{\PTglyph{5}{t64_l03g01.png}}
% 57
{\PTglyph{5}{t64_l03g02.png}}
% 58
{\PTglyph{5}{t64_l03g03.png}}
% 59
{\PTglyph{5}{t64_l03g04.png}}
% 60
{\PTglyph{5}{t64_l03g05.png}}
% 61
{\PTglyph{5}{t64_l03g06.png}}
% 62
{\PTglyph{5}{t64_l03g07.png}}
% 63
{\PTglyph{5}{t64_l03g08.png}}
% 64
{\PTglyph{5}{t64_l03g09.png}}
% 65
{\PTglyph{5}{t64_l03g10.png}}
% 66
{\PTglyph{5}{t64_l03g11.png}}
% 67
{\PTglyph{5}{t64_l03g12.png}}
% 68
{\PTglyph{5}{t64_l03g13.png}}
% 69
{\PTglyph{5}{t64_l03g14.png}}
% 70
{\PTglyph{5}{t64_l03g15.png}}
% 71
{\PTglyph{5}{t64_l03g16.png}}
% 72
{\PTglyph{5}{t64_l03g17.png}}
% 73
{\PTglyph{5}{t64_l03g18.png}}
% 74
{\PTglyph{5}{t64_l03g19.png}}
% 75
{\PTglyph{5}{t64_l03g20.png}}
% 76
{\PTglyph{5}{t64_l03g21.png}}
% 77
{\PTglyph{5}{t64_l03g22.png}}
% 78
{\PTglyph{5}{t64_l03g23.png}}
% 79
{\PTglyph{5}{t64_l03g24.png}}
% 80
{\PTglyph{5}{t64_l03g25.png}}
% 81
{\PTglyph{5}{t64_l03g26.png}}
% 82
{\PTglyph{5}{t64_l03g27.png}}
% 83
{\PTglyph{5}{t64_l03g28.png}}
% 84
{\PTglyph{5}{t64_l03g29.png}}
% 85
{\PTglyph{5}{t64_l04g01.png}}
% 86
{\PTglyph{5}{t64_l04g02.png}}
% 87
{\PTglyph{5}{t64_l04g03.png}}
% 88
{\PTglyph{5}{t64_l04g04.png}}
% 89
{\PTglyph{5}{t64_l04g05.png}}
% 90
{\PTglyph{5}{t64_l04g06.png}}
% 91
{\PTglyph{5}{t64_l04g07.png}}
% 92
{\PTglyph{5}{t64_l04g08.png}}
% 93
{\PTglyph{5}{t64_l04g09.png}}
% 94
{\PTglyph{5}{t64_l04g10.png}}
% 95
{\PTglyph{5}{t64_l04g11.png}}
% 96
{\PTglyph{5}{t64_l04g12.png}}
% 97
{\PTglyph{5}{t64_l04g13.png}}
% 98
{\PTglyph{5}{t64_l04g14.png}}
% 99
{\PTglyph{5}{t64_l04g15.png}}
% 100
{\PTglyph{5}{t64_l04g16.png}}
% 101
{\PTglyph{5}{t64_l04g17.png}}
% 102
{\PTglyph{5}{t64_l04g18.png}}
% 103
{\PTglyph{5}{t64_l04g19.png}}
% 104
{\PTglyph{5}{t64_l04g20.png}}
% 105
{\PTglyph{5}{t64_l04g21.png}}
% 106
{\PTglyph{5}{t64_l04g22.png}}
% 107
{\PTglyph{5}{t64_l04g23.png}}
//
%%% Local Variables:
%%% mode: latex
%%% TeX-engine: luatex
%%% TeX-master: shared
%%% End:

%//
%\glpismo%
 \glpismo
% 1
{\PTglyphid{U2-15_0101}}
% 2
{\PTglyphid{U2-15_0102}}
% 3
{\PTglyphid{U2-15_0103}}
% 4
{\PTglyphid{U2-15_0104}}
% 5
{\PTglyphid{U2-15_0105}}
% 6
{\PTglyphid{U2-15_0106}}
% 7
{\PTglyphid{U2-15_0107}}
% 8
{\PTglyphid{U2-15_0108}}
% 9
{\PTglyphid{U2-15_0109}}
% 10
{\PTglyphid{U2-15_0110}}
% 11
{\PTglyphid{U2-15_0111}}
% 12
{\PTglyphid{U2-15_0112}}
% 13
{\PTglyphid{U2-15_0113}}
% 14
{\PTglyphid{U2-15_0114}}
% 15
{\PTglyphid{U2-15_0115}}
% 16
{\PTglyphid{U2-15_0116}}
% 17
{\PTglyphid{U2-15_0117}}
% 18
{\PTglyphid{U2-15_0118}}
% 19
{\PTglyphid{U2-15_0119}}
% 20
{\PTglyphid{U2-15_0120}}
% 21
{\PTglyphid{U2-15_0121}}
% 22
{\PTglyphid{U2-15_0122}}
% 23
{\PTglyphid{U2-15_0201}}
% 24
{\PTglyphid{U2-15_0202}}
% 25
{\PTglyphid{U2-15_0203}}
% 26
{\PTglyphid{U2-15_0204}}
% 27
{\PTglyphid{U2-15_0205}}
% 28
{\PTglyphid{U2-15_0206}}
% 29
{\PTglyphid{U2-15_0207}}
% 30
{\PTglyphid{U2-15_0208}}
% 31
{\PTglyphid{U2-15_0209}}
% 32
{\PTglyphid{U2-15_0210}}
% 33
{\PTglyphid{U2-15_0211}}
% 34
{\PTglyphid{U2-15_0212}}
% 35
{\PTglyphid{U2-15_0213}}
% 36
{\PTglyphid{U2-15_0214}}
% 37
{\PTglyphid{U2-15_0215}}
% 38
{\PTglyphid{U2-15_0216}}
% 39
{\PTglyphid{U2-15_0217}}
% 40
{\PTglyphid{U2-15_0218}}
% 41
{\PTglyphid{U2-15_0219}}
% 42
{\PTglyphid{U2-15_0220}}
% 43
{\PTglyphid{U2-15_0221}}
% 44
{\PTglyphid{U2-15_0222}}
% 45
{\PTglyphid{U2-15_0223}}
% 46
{\PTglyphid{U2-15_0224}}
% 47
{\PTglyphid{U2-15_0225}}
% 48
{\PTglyphid{U2-15_0226}}
% 49
{\PTglyphid{U2-15_0227}}
% 50
{\PTglyphid{U2-15_0228}}
% 51
{\PTglyphid{U2-15_0229}}
% 52
{\PTglyphid{U2-15_0230}}
% 53
{\PTglyphid{U2-15_0231}}
% 54
{\PTglyphid{U2-15_0232}}
% 55
{\PTglyphid{U2-15_0233}}
% 56
{\PTglyphid{U2-15_0301}}
% 57
{\PTglyphid{U2-15_0302}}
% 58
{\PTglyphid{U2-15_0303}}
% 59
{\PTglyphid{U2-15_0304}}
% 60
{\PTglyphid{U2-15_0305}}
% 61
{\PTglyphid{U2-15_0306}}
% 62
{\PTglyphid{U2-15_0307}}
% 63
{\PTglyphid{U2-15_0308}}
% 64
{\PTglyphid{U2-15_0309}}
% 65
{\PTglyphid{U2-15_0310}}
% 66
{\PTglyphid{U2-15_0311}}
% 67
{\PTglyphid{U2-15_0312}}
% 68
{\PTglyphid{U2-15_0313}}
% 69
{\PTglyphid{U2-15_0314}}
% 70
{\PTglyphid{U2-15_0315}}
% 71
{\PTglyphid{U2-15_0316}}
% 72
{\PTglyphid{U2-15_0317}}
% 73
{\PTglyphid{U2-15_0318}}
% 74
{\PTglyphid{U2-15_0319}}
% 75
{\PTglyphid{U2-15_0320}}
% 76
{\PTglyphid{U2-15_0321}}
% 77
{\PTglyphid{U2-15_0322}}
% 78
{\PTglyphid{U2-15_0323}}
% 79
{\PTglyphid{U2-15_0324}}
% 80
{\PTglyphid{U2-15_0325}}
% 81
{\PTglyphid{U2-15_0326}}
% 82
{\PTglyphid{U2-15_0327}}
% 83
{\PTglyphid{U2-15_0328}}
% 84
{\PTglyphid{U2-15_0329}}
% 85
{\PTglyphid{U2-15_0401}}
% 86
{\PTglyphid{U2-15_0402}}
% 87
{\PTglyphid{U2-15_0403}}
% 88
{\PTglyphid{U2-15_0404}}
% 89
{\PTglyphid{U2-15_0405}}
% 90
{\PTglyphid{U2-15_0406}}
% 91
{\PTglyphid{U2-15_0407}}
% 92
{\PTglyphid{U2-15_0408}}
% 93
{\PTglyphid{U2-15_0409}}
% 94
{\PTglyphid{U2-15_0410}}
% 95
{\PTglyphid{U2-15_0411}}
% 96
{\PTglyphid{U2-15_0412}}
% 97
{\PTglyphid{U2-15_0413}}
% 98
{\PTglyphid{U2-15_0414}}
% 99
{\PTglyphid{U2-15_0415}}
% 100
{\PTglyphid{U2-15_0416}}
% 101
{\PTglyphid{U2-15_0417}}
% 102
{\PTglyphid{U2-15_0418}}
% 103
{\PTglyphid{U2-15_0419}}
% 104
{\PTglyphid{U2-15_0420}}
% 105
{\PTglyphid{U2-15_0421}}
% 106
{\PTglyphid{U2-15_0422}}
% 107
{\PTglyphid{U2-15_0423}}
//
\endgl \xe
%%% Local Variables:
%%% mode: latex
%%% TeX-engine: luatex
%%% TeX-master: shared
%%% End:

% //
%\endgl \xe


 \newpage
 
%%%%%%%%%%%%%%%%%%%%%%%%%%%%%%%%%%%%%%%%%%%%%%%%%%%%%%%%%%%%%%%%%%%%%%%%%%%%%%%
% from meta.csv
% 65,Ungler2-16_PT07_359.djvu,Ungler2,16,07,359
% 
%%%%%%%%%%%%%%%%%%%%%%%%%%%%%%%%%%%%%%%%%%%%%%%%%%%%%%%%%%%%%%%%%%%%%%%%%%%%%%%

 
% from dsed4test:
% Ungler2-16_PT07_359_4dsed.txt:Note "16. Pismo tekstowe, rotunda zbliżona do M⁵⁰ (druga forma). Stopień 20 ww. = 65 mm. — Tabl. 359."
% Ungler2-16_PT07_359_4dsed.txt:Note1 "Character set table prepared by Anna Wolińska"

 \pismoPL{Florian Ungler druga drukarnia 16. Pismo tekstowe, rotunda
   zbliżona do M⁵⁰ (druga forma). Stopień 20 ww. = 65 mm. —
   Tabl. 359.}
  
 \pismoEN{Florian Ungler second house 16. Rotunda text font, typeface
   similar to M⁵⁰ (second form). Type size 20 lines = 65 mm. — Plate
   359.}

\plate{359}{VII}{1970}

The plate prepared by Henryk Bułhak.\\
The font table prepared by Henryk Bułhak and Anna Wolińska.

\bigskip

\exampleBib{VII:75}

\bigskip \exampleDesc{GREGORIUS DE SZAMOTUŁY: Processus iuris brevior Ioannis Andreae pro tyrunculis resolutus.
Kraków, Florian Ungler, 1531. 8⁰.}


 \medskip
 \examplePage{\textit{Karta K₁b.}}

  \bigskip
  \exampleLib{Biblioteka Narodowa. Warszawa.}

  \bigskip \exampleRef{\textit{Estreicher XXX 197. Wierzbowski 80.}}

  % Pismo 15%: nagłówek. — Pismo 16: tekst i zestaw. — Rubryki x, 8. — Cyfry 11. — Znaki kalendarzowe 6.

\bigskip

\fontID{U2-16}{65}

\fontstat{135}

% \exdisplay \bg \gla
 \exdisplay \bg \gla
% 1
{\PTglyph{5}{t65_l01g01.png}}
% 2
{\PTglyph{5}{t65_l01g02.png}}
% 3
{\PTglyph{5}{t65_l01g03.png}}
% 4
{\PTglyph{5}{t65_l01g04.png}}
% 5
{\PTglyph{5}{t65_l01g05.png}}
% 6
{\PTglyph{5}{t65_l01g06.png}}
% 7
{\PTglyph{5}{t65_l01g07.png}}
% 8
{\PTglyph{5}{t65_l01g08.png}}
% 9
{\PTglyph{5}{t65_l01g09.png}}
% 10
{\PTglyph{5}{t65_l01g10.png}}
% 11
{\PTglyph{5}{t65_l01g11.png}}
% 12
{\PTglyph{5}{t65_l01g12.png}}
% 13
{\PTglyph{5}{t65_l01g13.png}}
% 14
{\PTglyph{5}{t65_l01g14.png}}
% 15
{\PTglyph{5}{t65_l01g15.png}}
% 16
{\PTglyph{5}{t65_l01g16.png}}
% 17
{\PTglyph{5}{t65_l01g17.png}}
% 18
{\PTglyph{5}{t65_l01g18.png}}
% 19
{\PTglyph{5}{t65_l01g19.png}}
% 20
{\PTglyph{5}{t65_l01g20.png}}
% 21
{\PTglyph{5}{t65_l01g21.png}}
% 22
{\PTglyph{5}{t65_l01g22.png}}
% 23
{\PTglyph{5}{t65_l01g23.png}}
% 24
{\PTglyph{5}{t65_l01g24.png}}
% 25
{\PTglyph{5}{t65_l01g25.png}}
% 26
{\PTglyph{5}{t65_l01g26.png}}
% 27
{\PTglyph{5}{t65_l01g27.png}}
% 28
{\PTglyph{5}{t65_l01g28.png}}
% 29
{\PTglyph{5}{t65_l01g29.png}}
% 30
{\PTglyph{5}{t65_l02g01.png}}
% 31
{\PTglyph{5}{t65_l02g02.png}}
% 32
{\PTglyph{5}{t65_l02g03.png}}
% 33
{\PTglyph{5}{t65_l02g04.png}}
% 34
{\PTglyph{5}{t65_l02g05.png}}
% 35
{\PTglyph{5}{t65_l02g06.png}}
% 36
{\PTglyph{5}{t65_l02g07.png}}
% 37
{\PTglyph{5}{t65_l02g08.png}}
% 38
{\PTglyph{5}{t65_l02g09.png}}
% 39
{\PTglyph{5}{t65_l02g10.png}}
% 40
{\PTglyph{5}{t65_l02g11.png}}
% 41
{\PTglyph{5}{t65_l02g12.png}}
% 42
{\PTglyph{5}{t65_l02g13.png}}
% 43
{\PTglyph{5}{t65_l02g14.png}}
% 44
{\PTglyph{5}{t65_l02g15.png}}
% 45
{\PTglyph{5}{t65_l02g16.png}}
% 46
{\PTglyph{5}{t65_l02g17.png}}
% 47
{\PTglyph{5}{t65_l02g18.png}}
% 48
{\PTglyph{5}{t65_l02g19.png}}
% 49
{\PTglyph{5}{t65_l02g20.png}}
% 50
{\PTglyph{5}{t65_l02g21.png}}
% 51
{\PTglyph{5}{t65_l02g22.png}}
% 52
{\PTglyph{5}{t65_l02g23.png}}
% 53
{\PTglyph{5}{t65_l02g24.png}}
% 54
{\PTglyph{5}{t65_l02g25.png}}
% 55
{\PTglyph{5}{t65_l02g26.png}}
% 56
{\PTglyph{5}{t65_l02g27.png}}
% 57
{\PTglyph{5}{t65_l02g28.png}}
% 58
{\PTglyph{5}{t65_l02g29.png}}
% 59
{\PTglyph{5}{t65_l02g30.png}}
% 60
{\PTglyph{5}{t65_l02g31.png}}
% 61
{\PTglyph{5}{t65_l02g32.png}}
% 62
{\PTglyph{5}{t65_l02g33.png}}
% 63
{\PTglyph{5}{t65_l02g34.png}}
% 64
{\PTglyph{5}{t65_l02g35.png}}
% 65
{\PTglyph{5}{t65_l02g36.png}}
% 66
{\PTglyph{5}{t65_l02g37.png}}
% 67
{\PTglyph{5}{t65_l02g38.png}}
% 68
{\PTglyph{5}{t65_l02g39.png}}
% 69
{\PTglyph{5}{t65_l02g40.png}}
% 70
{\PTglyph{5}{t65_l02g41.png}}
% 71
{\PTglyph{5}{t65_l02g42.png}}
% 72
{\PTglyph{5}{t65_l02g43.png}}
% 73
{\PTglyph{5}{t65_l02g44.png}}
% 74
{\PTglyph{5}{t65_l03g01.png}}
% 75
{\PTglyph{5}{t65_l03g02.png}}
% 76
{\PTglyph{5}{t65_l03g03.png}}
% 77
{\PTglyph{5}{t65_l03g04.png}}
% 78
{\PTglyph{5}{t65_l03g05.png}}
% 79
{\PTglyph{5}{t65_l03g06.png}}
% 80
{\PTglyph{5}{t65_l03g07.png}}
% 81
{\PTglyph{5}{t65_l03g08.png}}
% 82
{\PTglyph{5}{t65_l03g09.png}}
% 83
{\PTglyph{5}{t65_l03g10.png}}
% 84
{\PTglyph{5}{t65_l03g11.png}}
% 85
{\PTglyph{5}{t65_l03g12.png}}
% 86
{\PTglyph{5}{t65_l03g13.png}}
% 87
{\PTglyph{5}{t65_l03g14.png}}
% 88
{\PTglyph{5}{t65_l03g15.png}}
% 89
{\PTglyph{5}{t65_l03g16.png}}
% 90
{\PTglyph{5}{t65_l03g17.png}}
% 91
{\PTglyph{5}{t65_l03g18.png}}
% 92
{\PTglyph{5}{t65_l03g19.png}}
% 93
{\PTglyph{5}{t65_l03g20.png}}
% 94
{\PTglyph{5}{t65_l03g21.png}}
% 95
{\PTglyph{5}{t65_l03g22.png}}
% 96
{\PTglyph{5}{t65_l03g23.png}}
% 97
{\PTglyph{5}{t65_l03g24.png}}
% 98
{\PTglyph{5}{t65_l03g25.png}}
% 99
{\PTglyph{5}{t65_l03g26.png}}
% 100
{\PTglyph{5}{t65_l03g27.png}}
% 101
{\PTglyph{5}{t65_l03g28.png}}
% 102
{\PTglyph{5}{t65_l03g29.png}}
% 103
{\PTglyph{5}{t65_l03g30.png}}
% 104
{\PTglyph{5}{t65_l03g31.png}}
% 105
{\PTglyph{5}{t65_l03g32.png}}
% 106
{\PTglyph{5}{t65_l03g33.png}}
% 107
{\PTglyph{5}{t65_l03g34.png}}
% 108
{\PTglyph{5}{t65_l03g35.png}}
% 109
{\PTglyph{5}{t65_l03g36.png}}
% 110
{\PTglyph{5}{t65_l03g37.png}}
% 111
{\PTglyph{5}{t65_l03g38.png}}
% 112
{\PTglyph{5}{t65_l03g39.png}}
% 113
{\PTglyph{5}{t65_l03g40.png}}
% 114
{\PTglyph{5}{t65_l03g41.png}}
% 115
{\PTglyph{5}{t65_l03g42.png}}
% 116
{\PTglyph{5}{t65_l03g43.png}}
% 117
{\PTglyph{5}{t65_l03g44.png}}
% 118
{\PTglyph{5}{t65_l03g45.png}}
% 119
{\PTglyph{5}{t65_l04g01.png}}
% 120
{\PTglyph{5}{t65_l04g02.png}}
% 121
{\PTglyph{5}{t65_l04g03.png}}
% 122
{\PTglyph{5}{t65_l04g04.png}}
% 123
{\PTglyph{5}{t65_l04g05.png}}
% 124
{\PTglyph{5}{t65_l04g06.png}}
% 125
{\PTglyph{5}{t65_l04g07.png}}
% 126
{\PTglyph{5}{t65_l04g08.png}}
% 127
{\PTglyph{5}{t65_l04g09.png}}
% 128
{\PTglyph{5}{t65_l04g10.png}}
% 129
{\PTglyph{5}{t65_l04g11.png}}
% 130
{\PTglyph{5}{t65_l04g12.png}}
% 131
{\PTglyph{5}{t65_l05g01.png}}
% 132
{\PTglyph{5}{t65_l06g01.png}}
% 133
{\PTglyph{5}{t65_l06g02.png}}
% 134
{\PTglyph{5}{t65_l06g03.png}}
% 135
{\PTglyph{5}{t65_l07g01.png}}
//
%%% Local Variables:
%%% mode: latex
%%% TeX-engine: luatex
%%% TeX-master: shared
%%% End:

%//
%\glpismo%
 \glpismo
% 1
{\PTglyphid{U2-16_0101}}
% 2
{\PTglyphid{U2-16_0102}}
% 3
{\PTglyphid{U2-16_0103}}
% 4
{\PTglyphid{U2-16_0104}}
% 5
{\PTglyphid{U2-16_0105}}
% 6
{\PTglyphid{U2-16_0106}}
% 7
{\PTglyphid{U2-16_0107}}
% 8
{\PTglyphid{U2-16_0108}}
% 9
{\PTglyphid{U2-16_0109}}
% 10
{\PTglyphid{U2-16_0110}}
% 11
{\PTglyphid{U2-16_0111}}
% 12
{\PTglyphid{U2-16_0112}}
% 13
{\PTglyphid{U2-16_0113}}
% 14
{\PTglyphid{U2-16_0114}}
% 15
{\PTglyphid{U2-16_0115}}
% 16
{\PTglyphid{U2-16_0116}}
% 17
{\PTglyphid{U2-16_0117}}
% 18
{\PTglyphid{U2-16_0118}}
% 19
{\PTglyphid{U2-16_0119}}
% 20
{\PTglyphid{U2-16_0120}}
% 21
{\PTglyphid{U2-16_0121}}
% 22
{\PTglyphid{U2-16_0122}}
% 23
{\PTglyphid{U2-16_0123}}
% 24
{\PTglyphid{U2-16_0124}}
% 25
{\PTglyphid{U2-16_0125}}
% 26
{\PTglyphid{U2-16_0126}}
% 27
{\PTglyphid{U2-16_0127}}
% 28
{\PTglyphid{U2-16_0128}}
% 29
{\PTglyphid{U2-16_0129}}
% 30
{\PTglyphid{U2-16_0201}}
% 31
{\PTglyphid{U2-16_0202}}
% 32
{\PTglyphid{U2-16_0203}}
% 33
{\PTglyphid{U2-16_0204}}
% 34
{\PTglyphid{U2-16_0205}}
% 35
{\PTglyphid{U2-16_0206}}
% 36
{\PTglyphid{U2-16_0207}}
% 37
{\PTglyphid{U2-16_0208}}
% 38
{\PTglyphid{U2-16_0209}}
% 39
{\PTglyphid{U2-16_0210}}
% 40
{\PTglyphid{U2-16_0211}}
% 41
{\PTglyphid{U2-16_0212}}
% 42
{\PTglyphid{U2-16_0213}}
% 43
{\PTglyphid{U2-16_0214}}
% 44
{\PTglyphid{U2-16_0215}}
% 45
{\PTglyphid{U2-16_0216}}
% 46
{\PTglyphid{U2-16_0217}}
% 47
{\PTglyphid{U2-16_0218}}
% 48
{\PTglyphid{U2-16_0219}}
% 49
{\PTglyphid{U2-16_0220}}
% 50
{\PTglyphid{U2-16_0221}}
% 51
{\PTglyphid{U2-16_0222}}
% 52
{\PTglyphid{U2-16_0223}}
% 53
{\PTglyphid{U2-16_0224}}
% 54
{\PTglyphid{U2-16_0225}}
% 55
{\PTglyphid{U2-16_0226}}
% 56
{\PTglyphid{U2-16_0227}}
% 57
{\PTglyphid{U2-16_0228}}
% 58
{\PTglyphid{U2-16_0229}}
% 59
{\PTglyphid{U2-16_0230}}
% 60
{\PTglyphid{U2-16_0231}}
% 61
{\PTglyphid{U2-16_0232}}
% 62
{\PTglyphid{U2-16_0233}}
% 63
{\PTglyphid{U2-16_0234}}
% 64
{\PTglyphid{U2-16_0235}}
% 65
{\PTglyphid{U2-16_0236}}
% 66
{\PTglyphid{U2-16_0237}}
% 67
{\PTglyphid{U2-16_0238}}
% 68
{\PTglyphid{U2-16_0239}}
% 69
{\PTglyphid{U2-16_0240}}
% 70
{\PTglyphid{U2-16_0241}}
% 71
{\PTglyphid{U2-16_0242}}
% 72
{\PTglyphid{U2-16_0243}}
% 73
{\PTglyphid{U2-16_0244}}
% 74
{\PTglyphid{U2-16_0301}}
% 75
{\PTglyphid{U2-16_0302}}
% 76
{\PTglyphid{U2-16_0303}}
% 77
{\PTglyphid{U2-16_0304}}
% 78
{\PTglyphid{U2-16_0305}}
% 79
{\PTglyphid{U2-16_0306}}
% 80
{\PTglyphid{U2-16_0307}}
% 81
{\PTglyphid{U2-16_0308}}
% 82
{\PTglyphid{U2-16_0309}}
% 83
{\PTglyphid{U2-16_0310}}
% 84
{\PTglyphid{U2-16_0311}}
% 85
{\PTglyphid{U2-16_0312}}
% 86
{\PTglyphid{U2-16_0313}}
% 87
{\PTglyphid{U2-16_0314}}
% 88
{\PTglyphid{U2-16_0315}}
% 89
{\PTglyphid{U2-16_0316}}
% 90
{\PTglyphid{U2-16_0317}}
% 91
{\PTglyphid{U2-16_0318}}
% 92
{\PTglyphid{U2-16_0319}}
% 93
{\PTglyphid{U2-16_0320}}
% 94
{\PTglyphid{U2-16_0321}}
% 95
{\PTglyphid{U2-16_0322}}
% 96
{\PTglyphid{U2-16_0323}}
% 97
{\PTglyphid{U2-16_0324}}
% 98
{\PTglyphid{U2-16_0325}}
% 99
{\PTglyphid{U2-16_0326}}
% 100
{\PTglyphid{U2-16_0327}}
% 101
{\PTglyphid{U2-16_0328}}
% 102
{\PTglyphid{U2-16_0329}}
% 103
{\PTglyphid{U2-16_0330}}
% 104
{\PTglyphid{U2-16_0331}}
% 105
{\PTglyphid{U2-16_0332}}
% 106
{\PTglyphid{U2-16_0333}}
% 107
{\PTglyphid{U2-16_0334}}
% 108
{\PTglyphid{U2-16_0335}}
% 109
{\PTglyphid{U2-16_0336}}
% 110
{\PTglyphid{U2-16_0337}}
% 111
{\PTglyphid{U2-16_0338}}
% 112
{\PTglyphid{U2-16_0339}}
% 113
{\PTglyphid{U2-16_0340}}
% 114
{\PTglyphid{U2-16_0341}}
% 115
{\PTglyphid{U2-16_0342}}
% 116
{\PTglyphid{U2-16_0343}}
% 117
{\PTglyphid{U2-16_0344}}
% 118
{\PTglyphid{U2-16_0345}}
% 119
{\PTglyphid{U2-16_0401}}
% 120
{\PTglyphid{U2-16_0402}}
% 121
{\PTglyphid{U2-16_0403}}
% 122
{\PTglyphid{U2-16_0404}}
% 123
{\PTglyphid{U2-16_0405}}
% 124
{\PTglyphid{U2-16_0406}}
% 125
{\PTglyphid{U2-16_0407}}
% 126
{\PTglyphid{U2-16_0408}}
% 127
{\PTglyphid{U2-16_0409}}
% 128
{\PTglyphid{U2-16_0410}}
% 129
{\PTglyphid{U2-16_0411}}
% 130
{\PTglyphid{U2-16_0412}}
% 131
{\PTglyphid{U2-16_0501}}
% 132
{\PTglyphid{U2-16_0601}}
% 133
{\PTglyphid{U2-16_0602}}
% 134
{\PTglyphid{U2-16_0603}}
% 135
{\PTglyphid{U2-16_0701}}
//
\endgl \xe
%%% Local Variables:
%%% mode: latex
%%% TeX-engine: luatex
%%% TeX-master: shared
%%% End:

% //
%\endgl \xe


 \newpage
 
%%%%%%%%%%%%%%%%%%%%%%%%%%%%%%%%%%%%%%%%%%%%%%%%%%%%%%%%%%%%%%%%%%%%%%%%%%%%%%%
% from meta.csv
% 66,Ungler2-17_PT07_357.djvu,Ungler2,17,07,357
% 
%%%%%%%%%%%%%%%%%%%%%%%%%%%%%%%%%%%%%%%%%%%%%%%%%%%%%%%%%%%%%%%%%%%%%%%%%%%%%%%

 
% from dsed4test:
% Ungler2-17_PT07_357_4dsed.txt:Note "17. Pismo tekstowe, rotunda M⁸⁷. Stopień 20 ww. = 75 mm. — Tabl. 357."
% Ungler2-17_PT07_357_4dsed.txt:Note1 "Character set table prepared by Anna Wolińska"

 \pismoPL{Florian Ungler druga drukarnia 17. Pismo tekstowe, rotunda M⁸⁷. Stopień 20 ww. = 75 mm. — Tabl. 357.}
  
 \pismoEN{Florian Ungler second house 17. Rotunda text font, typeface
   M⁸⁷. Type size 20 lines = 75 mm. — Plate 357.}

\plate{357}{VII}{1970}

The plate prepared by Henryk Bułhak.\\
The font table prepared by Henryk Bułhak and Anna Wolińska.

\bigskip

\exampleBib{VII:113}

\bigskip \exampleDesc{ADAM TUSSINUS a Tarnów: ludicium et significatio cometae qui apparuit in fine dierum Iunii sub anno 1533.
Kraków, Florian Ungler, [po 26 VI 1533]. 8⁰.}


 \medskip
 \examplePage{\textit{Karta B₂b.}}

  \bigskip
  \exampleLib{Biblioteka Narodowa. Warszawa.}

 \bigskip \exampleRef{\textit{Estreicher XXXI 425. Wierzbowski 1103.}}

%  \bigskip

 % \exampleDig{\url{https://cyfrowe.mnk.pl/dlibra/publication/943/} page 9}

 % Pismo 14: drugi zestaw. — Pismo 17: kolumna tekstu i pierwszy
 % zestaw. — Pismo 18: nagłówek (wiersz 1). — Pismo 20: nagłówek
 % (wiersz 2). — Rubryka &:
% z pismem 17. — Cyfry 14: z pismem 14.

\bigskip

\fontID{U2-17}{66}

\fontstat{129}

% \exdisplay \bg \gla
 \exdisplay \bg \gla
% 1
{\PTglyph{5}{t66_l01g01.png}}
% 2
{\PTglyph{5}{t66_l01g02.png}}
% 3
{\PTglyph{5}{t66_l01g03.png}}
% 4
{\PTglyph{5}{t66_l01g04.png}}
% 5
{\PTglyph{5}{t66_l01g05.png}}
% 6
{\PTglyph{5}{t66_l01g06.png}}
% 7
{\PTglyph{5}{t66_l01g07.png}}
% 8
{\PTglyph{5}{t66_l01g08.png}}
% 9
{\PTglyph{5}{t66_l01g09.png}}
% 10
{\PTglyph{5}{t66_l01g10.png}}
% 11
{\PTglyph{5}{t66_l01g11.png}}
% 12
{\PTglyph{5}{t66_l01g12.png}}
% 13
{\PTglyph{5}{t66_l01g13.png}}
% 14
{\PTglyph{5}{t66_l01g14.png}}
% 15
{\PTglyph{5}{t66_l01g15.png}}
% 16
{\PTglyph{5}{t66_l01g16.png}}
% 17
{\PTglyph{5}{t66_l01g17.png}}
% 18
{\PTglyph{5}{t66_l01g18.png}}
% 19
{\PTglyph{5}{t66_l01g19.png}}
% 20
{\PTglyph{5}{t66_l01g20.png}}
% 21
{\PTglyph{5}{t66_l01g21.png}}
% 22
{\PTglyph{5}{t66_l01g22.png}}
% 23
{\PTglyph{5}{t66_l01g23.png}}
% 24
{\PTglyph{5}{t66_l01g24.png}}
% 25
{\PTglyph{5}{t66_l01g25.png}}
% 26
{\PTglyph{5}{t66_l01g26.png}}
% 27
{\PTglyph{5}{t66_l01g27.png}}
% 28
{\PTglyph{5}{t66_l02g01.png}}
% 29
{\PTglyph{5}{t66_l02g02.png}}
% 30
{\PTglyph{5}{t66_l02g03.png}}
% 31
{\PTglyph{5}{t66_l02g04.png}}
% 32
{\PTglyph{5}{t66_l02g05.png}}
% 33
{\PTglyph{5}{t66_l02g06.png}}
% 34
{\PTglyph{5}{t66_l02g07.png}}
% 35
{\PTglyph{5}{t66_l02g08.png}}
% 36
{\PTglyph{5}{t66_l02g09.png}}
% 37
{\PTglyph{5}{t66_l02g10.png}}
% 38
{\PTglyph{5}{t66_l02g11.png}}
% 39
{\PTglyph{5}{t66_l02g12.png}}
% 40
{\PTglyph{5}{t66_l02g13.png}}
% 41
{\PTglyph{5}{t66_l02g14.png}}
% 42
{\PTglyph{5}{t66_l02g15.png}}
% 43
{\PTglyph{5}{t66_l02g16.png}}
% 44
{\PTglyph{5}{t66_l02g17.png}}
% 45
{\PTglyph{5}{t66_l02g18.png}}
% 46
{\PTglyph{5}{t66_l02g19.png}}
% 47
{\PTglyph{5}{t66_l02g20.png}}
% 48
{\PTglyph{5}{t66_l02g21.png}}
% 49
{\PTglyph{5}{t66_l02g22.png}}
% 50
{\PTglyph{5}{t66_l02g23.png}}
% 51
{\PTglyph{5}{t66_l02g24.png}}
% 52
{\PTglyph{5}{t66_l02g25.png}}
% 53
{\PTglyph{5}{t66_l02g26.png}}
% 54
{\PTglyph{5}{t66_l02g27.png}}
% 55
{\PTglyph{5}{t66_l02g28.png}}
% 56
{\PTglyph{5}{t66_l02g29.png}}
% 57
{\PTglyph{5}{t66_l02g30.png}}
% 58
{\PTglyph{5}{t66_l02g31.png}}
% 59
{\PTglyph{5}{t66_l02g32.png}}
% 60
{\PTglyph{5}{t66_l02g33.png}}
% 61
{\PTglyph{5}{t66_l02g34.png}}
% 62
{\PTglyph{5}{t66_l02g35.png}}
% 63
{\PTglyph{5}{t66_l02g36.png}}
% 64
{\PTglyph{5}{t66_l02g37.png}}
% 65
{\PTglyph{5}{t66_l02g38.png}}
% 66
{\PTglyph{5}{t66_l03g01.png}}
% 67
{\PTglyph{5}{t66_l03g02.png}}
% 68
{\PTglyph{5}{t66_l03g03.png}}
% 69
{\PTglyph{5}{t66_l03g04.png}}
% 70
{\PTglyph{5}{t66_l03g05.png}}
% 71
{\PTglyph{5}{t66_l03g06.png}}
% 72
{\PTglyph{5}{t66_l03g07.png}}
% 73
{\PTglyph{5}{t66_l03g08.png}}
% 74
{\PTglyph{5}{t66_l03g09.png}}
% 75
{\PTglyph{5}{t66_l03g10.png}}
% 76
{\PTglyph{5}{t66_l03g11.png}}
% 77
{\PTglyph{5}{t66_l03g12.png}}
% 78
{\PTglyph{5}{t66_l03g13.png}}
% 79
{\PTglyph{5}{t66_l03g14.png}}
% 80
{\PTglyph{5}{t66_l03g15.png}}
% 81
{\PTglyph{5}{t66_l03g16.png}}
% 82
{\PTglyph{5}{t66_l03g17.png}}
% 83
{\PTglyph{5}{t66_l03g18.png}}
% 84
{\PTglyph{5}{t66_l03g19.png}}
% 85
{\PTglyph{5}{t66_l03g20.png}}
% 86
{\PTglyph{5}{t66_l03g21.png}}
% 87
{\PTglyph{5}{t66_l03g22.png}}
% 88
{\PTglyph{5}{t66_l03g23.png}}
% 89
{\PTglyph{5}{t66_l03g24.png}}
% 90
{\PTglyph{5}{t66_l03g25.png}}
% 91
{\PTglyph{5}{t66_l03g26.png}}
% 92
{\PTglyph{5}{t66_l03g27.png}}
% 93
{\PTglyph{5}{t66_l03g28.png}}
% 94
{\PTglyph{5}{t66_l03g29.png}}
% 95
{\PTglyph{5}{t66_l03g30.png}}
% 96
{\PTglyph{5}{t66_l03g31.png}}
% 97
{\PTglyph{5}{t66_l03g32.png}}
% 98
{\PTglyph{5}{t66_l03g33.png}}
% 99
{\PTglyph{5}{t66_l03g34.png}}
% 100
{\PTglyph{5}{t66_l03g35.png}}
% 101
{\PTglyph{5}{t66_l03g36.png}}
% 102
{\PTglyph{5}{t66_l04g01.png}}
% 103
{\PTglyph{5}{t66_l04g02.png}}
% 104
{\PTglyph{5}{t66_l04g03.png}}
% 105
{\PTglyph{5}{t66_l04g04.png}}
% 106
{\PTglyph{5}{t66_l04g05.png}}
% 107
{\PTglyph{5}{t66_l04g06.png}}
% 108
{\PTglyph{5}{t66_l04g07.png}}
% 109
{\PTglyph{5}{t66_l04g08.png}}
% 110
{\PTglyph{5}{t66_l04g09.png}}
% 111
{\PTglyph{5}{t66_l04g10.png}}
% 112
{\PTglyph{5}{t66_l04g11.png}}
% 113
{\PTglyph{5}{t66_l04g12.png}}
% 114
{\PTglyph{5}{t66_l04g13.png}}
% 115
{\PTglyph{5}{t66_l04g14.png}}
% 116
{\PTglyph{5}{t66_l04g15.png}}
% 117
{\PTglyph{5}{t66_l04g16.png}}
% 118
{\PTglyph{5}{t66_l04g17.png}}
% 119
{\PTglyph{5}{t66_l04g18.png}}
% 120
{\PTglyph{5}{t66_l04g19.png}}
% 121
{\PTglyph{5}{t66_l04g20.png}}
% 122
{\PTglyph{5}{t66_l04g21.png}}
% 123
{\PTglyph{5}{t66_l04g22.png}}
% 124
{\PTglyph{5}{t66_l04g23.png}}
% 125
{\PTglyph{5}{t66_l04g24.png}}
% 126
{\PTglyph{5}{t66_l04g25.png}}
% 127
{\PTglyph{5}{t66_l04g26.png}}
% 128
{\PTglyph{5}{t66_l04g27.png}}
% 129
{\PTglyph{5}{t66_l04g28.png}}
//
%%% Local Variables:
%%% mode: latex
%%% TeX-engine: luatex
%%% TeX-master: shared
%%% End:

%//
%\glpismo%
 \glpismo
% 1
{\PTglyphid{U2-17_0101}}
% 2
{\PTglyphid{U2-17_0102}}
% 3
{\PTglyphid{U2-17_0103}}
% 4
{\PTglyphid{U2-17_0104}}
% 5
{\PTglyphid{U2-17_0105}}
% 6
{\PTglyphid{U2-17_0106}}
% 7
{\PTglyphid{U2-17_0107}}
% 8
{\PTglyphid{U2-17_0108}}
% 9
{\PTglyphid{U2-17_0109}}
% 10
{\PTglyphid{U2-17_0110}}
% 11
{\PTglyphid{U2-17_0111}}
% 12
{\PTglyphid{U2-17_0112}}
% 13
{\PTglyphid{U2-17_0113}}
% 14
{\PTglyphid{U2-17_0114}}
% 15
{\PTglyphid{U2-17_0115}}
% 16
{\PTglyphid{U2-17_0116}}
% 17
{\PTglyphid{U2-17_0117}}
% 18
{\PTglyphid{U2-17_0118}}
% 19
{\PTglyphid{U2-17_0119}}
% 20
{\PTglyphid{U2-17_0120}}
% 21
{\PTglyphid{U2-17_0121}}
% 22
{\PTglyphid{U2-17_0122}}
% 23
{\PTglyphid{U2-17_0123}}
% 24
{\PTglyphid{U2-17_0124}}
% 25
{\PTglyphid{U2-17_0125}}
% 26
{\PTglyphid{U2-17_0126}}
% 27
{\PTglyphid{U2-17_0127}}
% 28
{\PTglyphid{U2-17_0201}}
% 29
{\PTglyphid{U2-17_0202}}
% 30
{\PTglyphid{U2-17_0203}}
% 31
{\PTglyphid{U2-17_0204}}
% 32
{\PTglyphid{U2-17_0205}}
% 33
{\PTglyphid{U2-17_0206}}
% 34
{\PTglyphid{U2-17_0207}}
% 35
{\PTglyphid{U2-17_0208}}
% 36
{\PTglyphid{U2-17_0209}}
% 37
{\PTglyphid{U2-17_0210}}
% 38
{\PTglyphid{U2-17_0211}}
% 39
{\PTglyphid{U2-17_0212}}
% 40
{\PTglyphid{U2-17_0213}}
% 41
{\PTglyphid{U2-17_0214}}
% 42
{\PTglyphid{U2-17_0215}}
% 43
{\PTglyphid{U2-17_0216}}
% 44
{\PTglyphid{U2-17_0217}}
% 45
{\PTglyphid{U2-17_0218}}
% 46
{\PTglyphid{U2-17_0219}}
% 47
{\PTglyphid{U2-17_0220}}
% 48
{\PTglyphid{U2-17_0221}}
% 49
{\PTglyphid{U2-17_0222}}
% 50
{\PTglyphid{U2-17_0223}}
% 51
{\PTglyphid{U2-17_0224}}
% 52
{\PTglyphid{U2-17_0225}}
% 53
{\PTglyphid{U2-17_0226}}
% 54
{\PTglyphid{U2-17_0227}}
% 55
{\PTglyphid{U2-17_0228}}
% 56
{\PTglyphid{U2-17_0229}}
% 57
{\PTglyphid{U2-17_0230}}
% 58
{\PTglyphid{U2-17_0231}}
% 59
{\PTglyphid{U2-17_0232}}
% 60
{\PTglyphid{U2-17_0233}}
% 61
{\PTglyphid{U2-17_0234}}
% 62
{\PTglyphid{U2-17_0235}}
% 63
{\PTglyphid{U2-17_0236}}
% 64
{\PTglyphid{U2-17_0237}}
% 65
{\PTglyphid{U2-17_0238}}
% 66
{\PTglyphid{U2-17_0301}}
% 67
{\PTglyphid{U2-17_0302}}
% 68
{\PTglyphid{U2-17_0303}}
% 69
{\PTglyphid{U2-17_0304}}
% 70
{\PTglyphid{U2-17_0305}}
% 71
{\PTglyphid{U2-17_0306}}
% 72
{\PTglyphid{U2-17_0307}}
% 73
{\PTglyphid{U2-17_0308}}
% 74
{\PTglyphid{U2-17_0309}}
% 75
{\PTglyphid{U2-17_0310}}
% 76
{\PTglyphid{U2-17_0311}}
% 77
{\PTglyphid{U2-17_0312}}
% 78
{\PTglyphid{U2-17_0313}}
% 79
{\PTglyphid{U2-17_0314}}
% 80
{\PTglyphid{U2-17_0315}}
% 81
{\PTglyphid{U2-17_0316}}
% 82
{\PTglyphid{U2-17_0317}}
% 83
{\PTglyphid{U2-17_0318}}
% 84
{\PTglyphid{U2-17_0319}}
% 85
{\PTglyphid{U2-17_0320}}
% 86
{\PTglyphid{U2-17_0321}}
% 87
{\PTglyphid{U2-17_0322}}
% 88
{\PTglyphid{U2-17_0323}}
% 89
{\PTglyphid{U2-17_0324}}
% 90
{\PTglyphid{U2-17_0325}}
% 91
{\PTglyphid{U2-17_0326}}
% 92
{\PTglyphid{U2-17_0327}}
% 93
{\PTglyphid{U2-17_0328}}
% 94
{\PTglyphid{U2-17_0329}}
% 95
{\PTglyphid{U2-17_0330}}
% 96
{\PTglyphid{U2-17_0331}}
% 97
{\PTglyphid{U2-17_0332}}
% 98
{\PTglyphid{U2-17_0333}}
% 99
{\PTglyphid{U2-17_0334}}
% 100
{\PTglyphid{U2-17_0335}}
% 101
{\PTglyphid{U2-17_0336}}
% 102
{\PTglyphid{U2-17_0401}}
% 103
{\PTglyphid{U2-17_0402}}
% 104
{\PTglyphid{U2-17_0403}}
% 105
{\PTglyphid{U2-17_0404}}
% 106
{\PTglyphid{U2-17_0405}}
% 107
{\PTglyphid{U2-17_0406}}
% 108
{\PTglyphid{U2-17_0407}}
% 109
{\PTglyphid{U2-17_0408}}
% 110
{\PTglyphid{U2-17_0409}}
% 111
{\PTglyphid{U2-17_0410}}
% 112
{\PTglyphid{U2-17_0411}}
% 113
{\PTglyphid{U2-17_0412}}
% 114
{\PTglyphid{U2-17_0413}}
% 115
{\PTglyphid{U2-17_0414}}
% 116
{\PTglyphid{U2-17_0415}}
% 117
{\PTglyphid{U2-17_0416}}
% 118
{\PTglyphid{U2-17_0417}}
% 119
{\PTglyphid{U2-17_0418}}
% 120
{\PTglyphid{U2-17_0419}}
% 121
{\PTglyphid{U2-17_0420}}
% 122
{\PTglyphid{U2-17_0421}}
% 123
{\PTglyphid{U2-17_0422}}
% 124
{\PTglyphid{U2-17_0423}}
% 125
{\PTglyphid{U2-17_0424}}
% 126
{\PTglyphid{U2-17_0425}}
% 127
{\PTglyphid{U2-17_0426}}
% 128
{\PTglyphid{U2-17_0427}}
% 129
{\PTglyphid{U2-17_0428}}
//
\endgl \xe
%%% Local Variables:
%%% mode: latex
%%% TeX-engine: luatex
%%% TeX-master: shared
%%% End:

% //
%\endgl \xe



 \newpage

19 poprawić !!!!
 
%%%%%%%%%%%%%%%%%%%%%%%%%%%%%%%%%%%%%%%%%%%%%%%%%%%%%%%%%%%%%%%%%%%%%%%%%%%%%%%
% from meta.csv
% 67,Ungler2-18_PT07_360.djvu,Ungler2,18,07,360
% 
%%%%%%%%%%%%%%%%%%%%%%%%%%%%%%%%%%%%%%%%%%%%%%%%%%%%%%%%%%%%%%%%%%%%%%%%%%%%%%%

 
% from dsed4test:
% Ungler2-18_PT07_360_4dsed.txt:Note "18. Pismo nagłówkowe i tekstowe, fraktura Neudörffer-Andreae. Stopień 10 ww. == 73 mm. — Tabl. 360."
% Ungler2-18_PT07_360_4dsed.txt:Note1 "Character set table prepared by Anna Wolińska"

 \pismoPL{Florian Ungler druga drukarnia 18. Pismo nagłówkowe i
   tekstowe, fraktura Neudörffer-Andreae. Stopień 10 ww. == 73 mm. —
   Tabl. 360 [drugi zestaw]}
  
 \pismoEN{Florian Ungler second house 18. Neudörffer-Andreae Fraktur
   header and text font. Type size 10 lines = 73 mm. — Plate 360
   [second set].}

\plate{360[2]}{VII}{1970}

The plate prepared by Henryk Bułhak.\\
The font table prepared by Henryk Bułhak and Anna Wolińska.

\bigskip

% \exampleBib{VII:113}

% \bigskip \exampleDesc{ADAM TUSSINUS a Tarnów: ludicium et significatio cometae qui apparuit in fine dierum Iunii sub anno 1533.
% Kraków, Florian Ungler, [po 26 VI 1533]. 8⁰.}


%  \medskip
%  \examplePage{\textit{Karta B₂b.}}

%   \bigskip
%   \exampleLib{Biblioteka Narodowa. Warszawa.}

%  \bigskip \exampleRef{\textit{Estreicher XXXI 425. Wierzbowski 1103.}}

% %  \bigskip

%  % \exampleDig{\url{https://cyfrowe.mnk.pl/dlibra/publication/943/} page 9}

% Pismo 18: drugi zestaw. — Pismo 19: tekst i pierwszy zestaw. — Rubryka t: z pismem 19. — Rubryka A: z pismem 18. — Cyfry 12: z pismem 18. —
% Cyfry 13: z pismem 19.

\bigskip

\fontID{U2-18}{67}

\fontstat{84}

% \exdisplay \bg \gla
 \exdisplay \bg \gla
% 1
{\PTglyph{5}{t67_l01g01.png}}
% 2
{\PTglyph{5}{t67_l01g02.png}}
% 3
{\PTglyph{5}{t67_l01g03.png}}
% 4
{\PTglyph{5}{t67_l01g04.png}}
% 5
{\PTglyph{5}{t67_l01g05.png}}
% 6
{\PTglyph{5}{t67_l01g06.png}}
% 7
{\PTglyph{5}{t67_l01g07.png}}
% 8
{\PTglyph{5}{t67_l01g08.png}}
% 9
{\PTglyph{5}{t67_l01g09.png}}
% 10
{\PTglyph{5}{t67_l01g10.png}}
% 11
{\PTglyph{5}{t67_l01g11.png}}
% 12
{\PTglyph{5}{t67_l01g12.png}}
% 13
{\PTglyph{5}{t67_l01g13.png}}
% 14
{\PTglyph{5}{t67_l01g14.png}}
% 15
{\PTglyph{5}{t67_l01g15.png}}
% 16
{\PTglyph{5}{t67_l01g16.png}}
% 17
{\PTglyph{5}{t67_l01g17.png}}
% 18
{\PTglyph{5}{t67_l01g18.png}}
% 19
{\PTglyph{5}{t67_l01g19.png}}
% 20
{\PTglyph{5}{t67_l01g20.png}}
% 21
{\PTglyph{5}{t67_l01g21.png}}
% 22
{\PTglyph{5}{t67_l01g22.png}}
% 23
{\PTglyph{5}{t67_l01g23.png}}
% 24
{\PTglyph{5}{t67_l01g24.png}}
% 25
{\PTglyph{5}{t67_l01g25.png}}
% 26
{\PTglyph{5}{t67_l02g01.png}}
% 27
{\PTglyph{5}{t67_l02g02.png}}
% 28
{\PTglyph{5}{t67_l02g03.png}}
% 29
{\PTglyph{5}{t67_l02g04.png}}
% 30
{\PTglyph{5}{t67_l02g05.png}}
% 31
{\PTglyph{5}{t67_l02g06.png}}
% 32
{\PTglyph{5}{t67_l02g07.png}}
% 33
{\PTglyph{5}{t67_l02g08.png}}
% 34
{\PTglyph{5}{t67_l02g09.png}}
% 35
{\PTglyph{5}{t67_l02g10.png}}
% 36
{\PTglyph{5}{t67_l02g11.png}}
% 37
{\PTglyph{5}{t67_l02g12.png}}
% 38
{\PTglyph{5}{t67_l02g13.png}}
% 39
{\PTglyph{5}{t67_l02g14.png}}
% 40
{\PTglyph{5}{t67_l02g15.png}}
% 41
{\PTglyph{5}{t67_l02g16.png}}
% 42
{\PTglyph{5}{t67_l02g17.png}}
% 43
{\PTglyph{5}{t67_l02g18.png}}
% 44
{\PTglyph{5}{t67_l02g19.png}}
% 45
{\PTglyph{5}{t67_l02g20.png}}
% 46
{\PTglyph{5}{t67_l02g21.png}}
% 47
{\PTglyph{5}{t67_l02g22.png}}
% 48
{\PTglyph{5}{t67_l02g23.png}}
% 49
{\PTglyph{5}{t67_l02g24.png}}
% 50
{\PTglyph{5}{t67_l02g25.png}}
% 51
{\PTglyph{5}{t67_l02g26.png}}
% 52
{\PTglyph{5}{t67_l02g27.png}}
% 53
{\PTglyph{5}{t67_l02g28.png}}
% 54
{\PTglyph{5}{t67_l02g29.png}}
% 55
{\PTglyph{5}{t67_l02g30.png}}
% 56
{\PTglyph{5}{t67_l02g31.png}}
% 57
{\PTglyph{5}{t67_l02g32.png}}
% 58
{\PTglyph{5}{t67_l02g33.png}}
% 59
{\PTglyph{5}{t67_l02g34.png}}
% 60
{\PTglyph{5}{t67_l02g35.png}}
% 61
{\PTglyph{5}{t67_l02g36.png}}
% 62
{\PTglyph{5}{t67_l02g37.png}}
% 63
{\PTglyph{5}{t67_l02g38.png}}
% 64
{\PTglyph{5}{t67_l02g39.png}}
% 65
{\PTglyph{5}{t67_l02g40.png}}
% 66
{\PTglyph{5}{t67_l02g41.png}}
% 67
{\PTglyph{5}{t67_l02g42.png}}
% 68
{\PTglyph{5}{t67_l02g43.png}}
% 69
{\PTglyph{5}{t67_l03g01.png}}
% 70
{\PTglyph{5}{t67_l03g02.png}}
% 71
{\PTglyph{5}{t67_l03g03.png}}
% 72
{\PTglyph{5}{t67_l03g04.png}}
% 73
{\PTglyph{5}{t67_l03g05.png}}
% 74
{\PTglyph{5}{t67_l03g06.png}}
% 75
{\PTglyph{5}{t67_l04g01.png}}
% 76
{\PTglyph{5}{t67_l04g02.png}}
% 77
{\PTglyph{5}{t67_l04g03.png}}
% 78
{\PTglyph{5}{t67_l04g04.png}}
% 79
{\PTglyph{5}{t67_l04g05.png}}
% 80
{\PTglyph{5}{t67_l04g06.png}}
% 81
{\PTglyph{5}{t67_l04g07.png}}
% 82
{\PTglyph{5}{t67_l04g08.png}}
% 83
{\PTglyph{5}{t67_l04g09.png}}
% 84
{\PTglyph{5}{t67_l04g10.png}}
//
%%% Local Variables:
%%% mode: latex
%%% TeX-engine: luatex
%%% TeX-master: shared
%%% End:

%//
%\glpismo%
 \glpismo
% 1
{\PTglyphid{U2-18_0101}}
% 2
{\PTglyphid{U2-18_0102}}
% 3
{\PTglyphid{U2-18_0103}}
% 4
{\PTglyphid{U2-18_0104}}
% 5
{\PTglyphid{U2-18_0105}}
% 6
{\PTglyphid{U2-18_0106}}
% 7
{\PTglyphid{U2-18_0107}}
% 8
{\PTglyphid{U2-18_0108}}
% 9
{\PTglyphid{U2-18_0109}}
% 10
{\PTglyphid{U2-18_0110}}
% 11
{\PTglyphid{U2-18_0111}}
% 12
{\PTglyphid{U2-18_0112}}
% 13
{\PTglyphid{U2-18_0113}}
% 14
{\PTglyphid{U2-18_0114}}
% 15
{\PTglyphid{U2-18_0115}}
% 16
{\PTglyphid{U2-18_0116}}
% 17
{\PTglyphid{U2-18_0117}}
% 18
{\PTglyphid{U2-18_0118}}
% 19
{\PTglyphid{U2-18_0119}}
% 20
{\PTglyphid{U2-18_0120}}
% 21
{\PTglyphid{U2-18_0121}}
% 22
{\PTglyphid{U2-18_0122}}
% 23
{\PTglyphid{U2-18_0123}}
% 24
{\PTglyphid{U2-18_0124}}
% 25
{\PTglyphid{U2-18_0125}}
% 26
{\PTglyphid{U2-18_0201}}
% 27
{\PTglyphid{U2-18_0202}}
% 28
{\PTglyphid{U2-18_0203}}
% 29
{\PTglyphid{U2-18_0204}}
% 30
{\PTglyphid{U2-18_0205}}
% 31
{\PTglyphid{U2-18_0206}}
% 32
{\PTglyphid{U2-18_0207}}
% 33
{\PTglyphid{U2-18_0208}}
% 34
{\PTglyphid{U2-18_0209}}
% 35
{\PTglyphid{U2-18_0210}}
% 36
{\PTglyphid{U2-18_0211}}
% 37
{\PTglyphid{U2-18_0212}}
% 38
{\PTglyphid{U2-18_0213}}
% 39
{\PTglyphid{U2-18_0214}}
% 40
{\PTglyphid{U2-18_0215}}
% 41
{\PTglyphid{U2-18_0216}}
% 42
{\PTglyphid{U2-18_0217}}
% 43
{\PTglyphid{U2-18_0218}}
% 44
{\PTglyphid{U2-18_0219}}
% 45
{\PTglyphid{U2-18_0220}}
% 46
{\PTglyphid{U2-18_0221}}
% 47
{\PTglyphid{U2-18_0222}}
% 48
{\PTglyphid{U2-18_0223}}
% 49
{\PTglyphid{U2-18_0224}}
% 50
{\PTglyphid{U2-18_0225}}
% 51
{\PTglyphid{U2-18_0226}}
% 52
{\PTglyphid{U2-18_0227}}
% 53
{\PTglyphid{U2-18_0228}}
% 54
{\PTglyphid{U2-18_0229}}
% 55
{\PTglyphid{U2-18_0230}}
% 56
{\PTglyphid{U2-18_0231}}
% 57
{\PTglyphid{U2-18_0232}}
% 58
{\PTglyphid{U2-18_0233}}
% 59
{\PTglyphid{U2-18_0234}}
% 60
{\PTglyphid{U2-18_0235}}
% 61
{\PTglyphid{U2-18_0236}}
% 62
{\PTglyphid{U2-18_0237}}
% 63
{\PTglyphid{U2-18_0238}}
% 64
{\PTglyphid{U2-18_0239}}
% 65
{\PTglyphid{U2-18_0240}}
% 66
{\PTglyphid{U2-18_0241}}
% 67
{\PTglyphid{U2-18_0242}}
% 68
{\PTglyphid{U2-18_0243}}
% 69
{\PTglyphid{U2-18_0301}}
% 70
{\PTglyphid{U2-18_0302}}
% 71
{\PTglyphid{U2-18_0303}}
% 72
{\PTglyphid{U2-18_0304}}
% 73
{\PTglyphid{U2-18_0305}}
% 74
{\PTglyphid{U2-18_0306}}
% 75
{\PTglyphid{U2-18_0401}}
% 76
{\PTglyphid{U2-18_0402}}
% 77
{\PTglyphid{U2-18_0403}}
% 78
{\PTglyphid{U2-18_0404}}
% 79
{\PTglyphid{U2-18_0405}}
% 80
{\PTglyphid{U2-18_0406}}
% 81
{\PTglyphid{U2-18_0407}}
% 82
{\PTglyphid{U2-18_0408}}
% 83
{\PTglyphid{U2-18_0409}}
% 84
{\PTglyphid{U2-18_0410}}
//
\endgl \xe
%%% Local Variables:
%%% mode: latex
%%% TeX-engine: luatex
%%% TeX-master: shared
%%% End:

% //
%\endgl \xe


 \newpage
 
%%%%%%%%%%%%%%%%%%%%%%%%%%%%%%%%%%%%%%%%%%%%%%%%%%%%%%%%%%%%%%%%%%%%%%%%%%%%%%%
% from meta.csv
% 68,Ungler2-19_PT07_360.djvu,Ungler2,19,07,360
% 
%%%%%%%%%%%%%%%%%%%%%%%%%%%%%%%%%%%%%%%%%%%%%%%%%%%%%%%%%%%%%%%%%%%%%%%%%%%%%%%

 
% from dsed4test:
% Ungler2-19_PT07_360_4dsed.txt:Note "19. Pismo tekstowe, fraktura Neudörffer-Andreae. Stopień 20 ww. = 90—91 mm. — Tabl. 360."
% Ungler2-19_PT07_360_4dsed.txt:Note1 "Character set table prepared by Anna Wolińska"

 \pismoPL{Florian Ungler druga drukarnia 19. Pismo tekstowe, fraktura
  Neudörffer-Andreae. Stopień 20 ww. = 90—91 mm. — Tabl. 360 [pierwszy
  zestaw]}
  
 \pismoEN{Florian Ungler second house 19. Neudörffer-Andreae Fraktur
    text font. Type size 20 lines = 90—91 mm. — Plate 360
   [first set].}

\plate{360[1]}{VII}{1970}

The plate prepared by Henryk Bułhak.\\
The font table prepared by Henryk Bułhak and Anna Wolińska.

\bigskip

\exampleBib{VII:83}

\bigskip \exampleDesc{NICOLAUS DE SZADEK: [udicium et significatio cometae visi anno 1531.
Kraków, Florian Ungler, [po VIII] 1531. 8⁰.}

 \medskip
 \examplePage{\textit{Karta B₅b.}}

  \bigskip
  \exampleLib{Biblioteka Jagiellońska. Kraków.}

 \bigskip \exampleRef{\textit{Estreicher XXX 183/4.}}

% %  \bigskip

%  % \exampleDig{\url{https://cyfrowe.mnk.pl/dlibra/publication/943/} page 9}

% Pismo 18: drugi zestaw. — Pismo 19: tekst i pierwszy zestaw. — Rubryka t: z pismem 19. — Rubryka A: z pismem 18. — Cyfry 12: z pismem 18. —
% Cyfry 13: z pismem 19.

\bigskip

\fontID{U2-19}{68}

\fontstat{114}

% \exdisplay \bg \gla
 \exdisplay \bg \gla
% 1
{\PTglyph{5}{t68_l01g01.png}}
% 2
{\PTglyph{5}{t68_l01g02.png}}
% 3
{\PTglyph{5}{t68_l01g03.png}}
% 4
{\PTglyph{5}{t68_l01g04.png}}
% 5
{\PTglyph{5}{t68_l01g05.png}}
% 6
{\PTglyph{5}{t68_l01g06.png}}
% 7
{\PTglyph{5}{t68_l01g07.png}}
% 8
{\PTglyph{5}{t68_l01g08.png}}
% 9
{\PTglyph{5}{t68_l01g09.png}}
% 10
{\PTglyph{5}{t68_l01g10.png}}
% 11
{\PTglyph{5}{t68_l01g11.png}}
% 12
{\PTglyph{5}{t68_l01g12.png}}
% 13
{\PTglyph{5}{t68_l01g13.png}}
% 14
{\PTglyph{5}{t68_l01g14.png}}
% 15
{\PTglyph{5}{t68_l01g15.png}}
% 16
{\PTglyph{5}{t68_l01g16.png}}
% 17
{\PTglyph{5}{t68_l01g17.png}}
% 18
{\PTglyph{5}{t68_l01g18.png}}
% 19
{\PTglyph{5}{t68_l01g19.png}}
% 20
{\PTglyph{5}{t68_l01g20.png}}
% 21
{\PTglyph{5}{t68_l01g21.png}}
% 22
{\PTglyph{5}{t68_l01g22.png}}
% 23
{\PTglyph{5}{t68_l01g23.png}}
% 24
{\PTglyph{5}{t68_l01g24.png}}
% 25
{\PTglyph{5}{t68_l01g25.png}}
% 26
{\PTglyph{5}{t68_l02g01.png}}
% 27
{\PTglyph{5}{t68_l02g02.png}}
% 28
{\PTglyph{5}{t68_l02g03.png}}
% 29
{\PTglyph{5}{t68_l02g04.png}}
% 30
{\PTglyph{5}{t68_l02g05.png}}
% 31
{\PTglyph{5}{t68_l02g06.png}}
% 32
{\PTglyph{5}{t68_l02g07.png}}
% 33
{\PTglyph{5}{t68_l02g08.png}}
% 34
{\PTglyph{5}{t68_l02g09.png}}
% 35
{\PTglyph{5}{t68_l02g10.png}}
% 36
{\PTglyph{5}{t68_l02g11.png}}
% 37
{\PTglyph{5}{t68_l02g12.png}}
% 38
{\PTglyph{5}{t68_l02g13.png}}
% 39
{\PTglyph{5}{t68_l02g14.png}}
% 40
{\PTglyph{5}{t68_l02g15.png}}
% 41
{\PTglyph{5}{t68_l02g16.png}}
% 42
{\PTglyph{5}{t68_l02g17.png}}
% 43
{\PTglyph{5}{t68_l02g18.png}}
% 44
{\PTglyph{5}{t68_l02g19.png}}
% 45
{\PTglyph{5}{t68_l02g20.png}}
% 46
{\PTglyph{5}{t68_l02g21.png}}
% 47
{\PTglyph{5}{t68_l02g22.png}}
% 48
{\PTglyph{5}{t68_l02g23.png}}
% 49
{\PTglyph{5}{t68_l02g24.png}}
% 50
{\PTglyph{5}{t68_l02g25.png}}
% 51
{\PTglyph{5}{t68_l02g26.png}}
% 52
{\PTglyph{5}{t68_l02g27.png}}
% 53
{\PTglyph{5}{t68_l02g28.png}}
% 54
{\PTglyph{5}{t68_l02g29.png}}
% 55
{\PTglyph{5}{t68_l02g30.png}}
% 56
{\PTglyph{5}{t68_l02g31.png}}
% 57
{\PTglyph{5}{t68_l02g32.png}}
% 58
{\PTglyph{5}{t68_l02g33.png}}
% 59
{\PTglyph{5}{t68_l02g34.png}}
% 60
{\PTglyph{5}{t68_l02g35.png}}
% 61
{\PTglyph{5}{t68_l02g36.png}}
% 62
{\PTglyph{5}{t68_l02g37.png}}
% 63
{\PTglyph{5}{t68_l02g38.png}}
% 64
{\PTglyph{5}{t68_l02g39.png}}
% 65
{\PTglyph{5}{t68_l02g40.png}}
% 66
{\PTglyph{5}{t68_l02g41.png}}
% 67
{\PTglyph{5}{t68_l02g42.png}}
% 68
{\PTglyph{5}{t68_l02g43.png}}
% 69
{\PTglyph{5}{t68_l03g01.png}}
% 70
{\PTglyph{5}{t68_l03g02.png}}
% 71
{\PTglyph{5}{t68_l03g03.png}}
% 72
{\PTglyph{5}{t68_l03g04.png}}
% 73
{\PTglyph{5}{t68_l03g05.png}}
% 74
{\PTglyph{5}{t68_l03g06.png}}
% 75
{\PTglyph{5}{t68_l04g01.png}}
% 76
{\PTglyph{5}{t68_l04g02.png}}
% 77
{\PTglyph{5}{t68_l04g03.png}}
% 78
{\PTglyph{5}{t68_l04g04.png}}
% 79
{\PTglyph{5}{t68_l04g05.png}}
% 80
{\PTglyph{5}{t68_l04g06.png}}
% 81
{\PTglyph{5}{t68_l04g07.png}}
% 82
{\PTglyph{5}{t68_l04g08.png}}
% 83
{\PTglyph{5}{t68_l04g09.png}}
% 84
{\PTglyph{5}{t68_l04g10.png}}
//
%%% Local Variables:
%%% mode: latex
%%% TeX-engine: luatex
%%% TeX-master: shared
%%% End:

%//
%\glpismo%
 \glpismo
% 1
{\PTglyphid{U2-19_0101}}
% 2
{\PTglyphid{U2-19_0102}}
% 3
{\PTglyphid{U2-19_0103}}
% 4
{\PTglyphid{U2-19_0104}}
% 5
{\PTglyphid{U2-19_0105}}
% 6
{\PTglyphid{U2-19_0106}}
% 7
{\PTglyphid{U2-19_0107}}
% 8
{\PTglyphid{U2-19_0108}}
% 9
{\PTglyphid{U2-19_0109}}
% 10
{\PTglyphid{U2-19_0110}}
% 11
{\PTglyphid{U2-19_0111}}
% 12
{\PTglyphid{U2-19_0112}}
% 13
{\PTglyphid{U2-19_0113}}
% 14
{\PTglyphid{U2-19_0114}}
% 15
{\PTglyphid{U2-19_0115}}
% 16
{\PTglyphid{U2-19_0116}}
% 17
{\PTglyphid{U2-19_0117}}
% 18
{\PTglyphid{U2-19_0118}}
% 19
{\PTglyphid{U2-19_0119}}
% 20
{\PTglyphid{U2-19_0120}}
% 21
{\PTglyphid{U2-19_0121}}
% 22
{\PTglyphid{U2-19_0122}}
% 23
{\PTglyphid{U2-19_0123}}
% 24
{\PTglyphid{U2-19_0124}}
% 25
{\PTglyphid{U2-19_0125}}
% 26
{\PTglyphid{U2-19_0201}}
% 27
{\PTglyphid{U2-19_0202}}
% 28
{\PTglyphid{U2-19_0203}}
% 29
{\PTglyphid{U2-19_0204}}
% 30
{\PTglyphid{U2-19_0205}}
% 31
{\PTglyphid{U2-19_0206}}
% 32
{\PTglyphid{U2-19_0207}}
% 33
{\PTglyphid{U2-19_0208}}
% 34
{\PTglyphid{U2-19_0209}}
% 35
{\PTglyphid{U2-19_0210}}
% 36
{\PTglyphid{U2-19_0211}}
% 37
{\PTglyphid{U2-19_0212}}
% 38
{\PTglyphid{U2-19_0213}}
% 39
{\PTglyphid{U2-19_0214}}
% 40
{\PTglyphid{U2-19_0215}}
% 41
{\PTglyphid{U2-19_0216}}
% 42
{\PTglyphid{U2-19_0217}}
% 43
{\PTglyphid{U2-19_0218}}
% 44
{\PTglyphid{U2-19_0219}}
% 45
{\PTglyphid{U2-19_0220}}
% 46
{\PTglyphid{U2-19_0221}}
% 47
{\PTglyphid{U2-19_0222}}
% 48
{\PTglyphid{U2-19_0223}}
% 49
{\PTglyphid{U2-19_0224}}
% 50
{\PTglyphid{U2-19_0225}}
% 51
{\PTglyphid{U2-19_0226}}
% 52
{\PTglyphid{U2-19_0227}}
% 53
{\PTglyphid{U2-19_0228}}
% 54
{\PTglyphid{U2-19_0229}}
% 55
{\PTglyphid{U2-19_0230}}
% 56
{\PTglyphid{U2-19_0231}}
% 57
{\PTglyphid{U2-19_0232}}
% 58
{\PTglyphid{U2-19_0233}}
% 59
{\PTglyphid{U2-19_0234}}
% 60
{\PTglyphid{U2-19_0235}}
% 61
{\PTglyphid{U2-19_0236}}
% 62
{\PTglyphid{U2-19_0237}}
% 63
{\PTglyphid{U2-19_0238}}
% 64
{\PTglyphid{U2-19_0239}}
% 65
{\PTglyphid{U2-19_0240}}
% 66
{\PTglyphid{U2-19_0241}}
% 67
{\PTglyphid{U2-19_0242}}
% 68
{\PTglyphid{U2-19_0301}}
% 69
{\PTglyphid{U2-19_0302}}
% 70
{\PTglyphid{U2-19_0303}}
% 71
{\PTglyphid{U2-19_0304}}
% 72
{\PTglyphid{U2-19_0305}}
% 73
{\PTglyphid{U2-19_0306}}
% 74
{\PTglyphid{U2-19_0307}}
% 75
{\PTglyphid{U2-19_0308}}
% 76
{\PTglyphid{U2-19_0309}}
% 77
{\PTglyphid{U2-19_0310}}
% 78
{\PTglyphid{U2-19_0311}}
% 79
{\PTglyphid{U2-19_0312}}
% 80
{\PTglyphid{U2-19_0313}}
% 81
{\PTglyphid{U2-19_0314}}
% 82
{\PTglyphid{U2-19_0315}}
% 83
{\PTglyphid{U2-19_0316}}
% 84
{\PTglyphid{U2-19_0317}}
% 85
{\PTglyphid{U2-19_0318}}
% 86
{\PTglyphid{U2-19_0319}}
% 87
{\PTglyphid{U2-19_0320}}
% 88
{\PTglyphid{U2-19_0321}}
% 89
{\PTglyphid{U2-19_0322}}
% 90
{\PTglyphid{U2-19_0323}}
% 91
{\PTglyphid{U2-19_0324}}
% 92
{\PTglyphid{U2-19_0325}}
% 93
{\PTglyphid{U2-19_0326}}
% 94
{\PTglyphid{U2-19_0327}}
% 95
{\PTglyphid{U2-19_0328}}
% 96
{\PTglyphid{U2-19_0329}}
% 97
{\PTglyphid{U2-19_0330}}
% 98
{\PTglyphid{U2-19_0331}}
% 99
{\PTglyphid{U2-19_0332}}
% 100
{\PTglyphid{U2-19_0333}}
% 101
{\PTglyphid{U2-19_0334}}
% 102
{\PTglyphid{U2-19_0335}}
% 103
{\PTglyphid{U2-19_0336}}
% 104
{\PTglyphid{U2-19_0337}}
% 105
{\PTglyphid{U2-19_0401}}
% 106
{\PTglyphid{U2-19_0402}}
% 107
{\PTglyphid{U2-19_0403}}
% 108
{\PTglyphid{U2-19_0404}}
% 109
{\PTglyphid{U2-19_0405}}
% 110
{\PTglyphid{U2-19_0406}}
% 111
{\PTglyphid{U2-19_0407}}
% 112
{\PTglyphid{U2-19_0408}}
% 113
{\PTglyphid{U2-19_0409}}
% 114
{\PTglyphid{U2-19_0410}}
//
\endgl \xe
%%% Local Variables:
%%% mode: latex
%%% TeX-engine: luatex
%%% TeX-master: shared
%%% End:

% //
%\endgl \xe


 \newpage
 
%%%%%%%%%%%%%%%%%%%%%%%%%%%%%%%%%%%%%%%%%%%%%%%%%%%%%%%%%%%%%%%%%%%%%%%%%%%%%%%
% from meta.csv
% 69,Ungler2-20_PT07_361.djvu,Ungler2,20,07,361
% 
%%%%%%%%%%%%%%%%%%%%%%%%%%%%%%%%%%%%%%%%%%%%%%%%%%%%%%%%%%%%%%%%%%%%%%%%%%%%%%%

 
% from dsed4test:
% Ungler2-20_PT07_361_4dsed.txt:Note "20. Pismo tekstowe, antykwa Qu|(G). Stopień 20 ww. == 90-—91 mm. — Tabl. 361."
% Ungler2-20_PT07_361_4dsed.txt:Note1 "Character set table prepared by Anna Wolińska"

 \pismoPL{Florian Ungler druga drukarnia 20. Pismo tekstowe, antykwa
   Qu|(G). Stopień 20 ww. == 90-—91 mm. — Tabl. 361.}
  
 \pismoEN{Florian Ungler second house 20. Roman text font, typeface Qu|(G). Type size 20 lines = 90—91 mm. — Plate 361.}

\plate{361}{VII}{1970}

The plate prepared by Henryk Bułhak.\\
The font table prepared by Henryk Bułhak and Anna Wolińska.

\bigskip

\exampleBib{VII:101}

\bigskip \exampleDesc{CONSTITUTIONES et articuli synodi Lanciciensis
  anno 1527 celebratae. Item synodi Piotrkoviensis anno 1530 et 1532
  celebratae. Ed. Matthias Drzewicki. Kraków, Florian Ungler, 5 IX
  1532. 4⁰.}

 \medskip
 \examplePage{\textit{Karta tytułowa verso.}}

  \bigskip
  \exampleLib{Biblioteka Jagiellońska. Kraków.}

 \bigskip \exampleRef{\textit{Estreicher XIV 382.}}

% %  \bigskip

 \exampleDig{\url{https://cyfrowe.mnk.pl/dlibra/publication/27252/}
   page 4, \url{https://dbc.wroc.pl/dlibra/publication/13405/} page 6}

  % https://cyfrowe.mnk.pl/dlibra/publication/27252/edition/26931/constitutiones-et-articuli-synodi-lancicien-sis-anno-domini-millesimoquige-n-tesimouigesimoseptimo-celebratae-ite-m-synodi-piotrkouien-sis-anno-domini-millesimoqvingentesimotricesimo-die-lunae-ante-festum-sanctae-margaretae-celebratae-item-synodi-piotrkouien-sis-anno-domini-millesimiqvingentesimotricesimosecundo-celebratae-ecclesia-catholica-leczyca-synod-1527?language=pl

 % https://dbc.wroc.pl/dlibra/publication/13405/edition/11917?language=pl
 
%  Pismo 14: wiersz 1. — Pismo 18: wiersz 2. — Rubryki e, t. — Cyfry 10P. — Inicjał 29.

\bigskip

\fontID{U2-20}{69}

\fontstat{128}

% \exdisplay \bg \gla
 \exdisplay \bg \gla
% 1
{\PTglyph{5}{t69_l01g01.png}}
% 2
{\PTglyph{5}{t69_l01g02.png}}
% 3
{\PTglyph{5}{t69_l01g03.png}}
% 4
{\PTglyph{5}{t69_l01g04.png}}
% 5
{\PTglyph{5}{t69_l01g05.png}}
% 6
{\PTglyph{5}{t69_l01g06.png}}
% 7
{\PTglyph{5}{t69_l01g07.png}}
% 8
{\PTglyph{5}{t69_l01g08.png}}
% 9
{\PTglyph{5}{t69_l01g09.png}}
% 10
{\PTglyph{5}{t69_l01g10.png}}
% 11
{\PTglyph{5}{t69_l01g11.png}}
% 12
{\PTglyph{5}{t69_l01g12.png}}
% 13
{\PTglyph{5}{t69_l01g13.png}}
% 14
{\PTglyph{5}{t69_l01g14.png}}
% 15
{\PTglyph{5}{t69_l01g15.png}}
% 16
{\PTglyph{5}{t69_l01g16.png}}
% 17
{\PTglyph{5}{t69_l01g17.png}}
% 18
{\PTglyph{5}{t69_l01g18.png}}
% 19
{\PTglyph{5}{t69_l01g19.png}}
% 20
{\PTglyph{5}{t69_l01g20.png}}
% 21
{\PTglyph{5}{t69_l01g21.png}}
% 22
{\PTglyph{5}{t69_l01g22.png}}
% 23
{\PTglyph{5}{t69_l01g23.png}}
% 24
{\PTglyph{5}{t69_l01g24.png}}
% 25
{\PTglyph{5}{t69_l01g25.png}}
% 26
{\PTglyph{5}{t69_l01g26.png}}
% 27
{\PTglyph{5}{t69_l02g01.png}}
% 28
{\PTglyph{5}{t69_l02g02.png}}
% 29
{\PTglyph{5}{t69_l02g03.png}}
% 30
{\PTglyph{5}{t69_l02g04.png}}
% 31
{\PTglyph{5}{t69_l02g05.png}}
% 32
{\PTglyph{5}{t69_l02g06.png}}
% 33
{\PTglyph{5}{t69_l02g07.png}}
% 34
{\PTglyph{5}{t69_l02g08.png}}
% 35
{\PTglyph{5}{t69_l02g09.png}}
% 36
{\PTglyph{5}{t69_l02g10.png}}
% 37
{\PTglyph{5}{t69_l02g11.png}}
% 38
{\PTglyph{5}{t69_l02g12.png}}
% 39
{\PTglyph{5}{t69_l02g13.png}}
% 40
{\PTglyph{5}{t69_l02g14.png}}
% 41
{\PTglyph{5}{t69_l02g15.png}}
% 42
{\PTglyph{5}{t69_l02g16.png}}
% 43
{\PTglyph{5}{t69_l02g17.png}}
% 44
{\PTglyph{5}{t69_l02g18.png}}
% 45
{\PTglyph{5}{t69_l02g19.png}}
% 46
{\PTglyph{5}{t69_l02g20.png}}
% 47
{\PTglyph{5}{t69_l02g21.png}}
% 48
{\PTglyph{5}{t69_l02g22.png}}
% 49
{\PTglyph{5}{t69_l02g23.png}}
% 50
{\PTglyph{5}{t69_l02g24.png}}
% 51
{\PTglyph{5}{t69_l02g25.png}}
% 52
{\PTglyph{5}{t69_l02g26.png}}
% 53
{\PTglyph{5}{t69_l02g27.png}}
% 54
{\PTglyph{5}{t69_l02g28.png}}
% 55
{\PTglyph{5}{t69_l02g29.png}}
% 56
{\PTglyph{5}{t69_l02g30.png}}
% 57
{\PTglyph{5}{t69_l02g31.png}}
% 58
{\PTglyph{5}{t69_l02g32.png}}
% 59
{\PTglyph{5}{t69_l02g33.png}}
% 60
{\PTglyph{5}{t69_l02g34.png}}
% 61
{\PTglyph{5}{t69_l02g35.png}}
% 62
{\PTglyph{5}{t69_l02g36.png}}
% 63
{\PTglyph{5}{t69_l02g37.png}}
% 64
{\PTglyph{5}{t69_l02g38.png}}
% 65
{\PTglyph{5}{t69_l02g39.png}}
% 66
{\PTglyph{5}{t69_l02g40.png}}
% 67
{\PTglyph{5}{t69_l03g01.png}}
% 68
{\PTglyph{5}{t69_l03g02.png}}
% 69
{\PTglyph{5}{t69_l03g03.png}}
% 70
{\PTglyph{5}{t69_l03g04.png}}
% 71
{\PTglyph{5}{t69_l03g05.png}}
% 72
{\PTglyph{5}{t69_l03g06.png}}
% 73
{\PTglyph{5}{t69_l03g07.png}}
% 74
{\PTglyph{5}{t69_l03g08.png}}
% 75
{\PTglyph{5}{t69_l03g09.png}}
% 76
{\PTglyph{5}{t69_l03g10.png}}
% 77
{\PTglyph{5}{t69_l03g11.png}}
% 78
{\PTglyph{5}{t69_l03g12.png}}
% 79
{\PTglyph{5}{t69_l03g13.png}}
% 80
{\PTglyph{5}{t69_l03g14.png}}
% 81
{\PTglyph{5}{t69_l03g15.png}}
% 82
{\PTglyph{5}{t69_l03g16.png}}
% 83
{\PTglyph{5}{t69_l03g17.png}}
% 84
{\PTglyph{5}{t69_l03g18.png}}
% 85
{\PTglyph{5}{t69_l03g19.png}}
% 86
{\PTglyph{5}{t69_l03g20.png}}
% 87
{\PTglyph{5}{t69_l03g21.png}}
% 88
{\PTglyph{5}{t69_l03g22.png}}
% 89
{\PTglyph{5}{t69_l03g23.png}}
% 90
{\PTglyph{5}{t69_l03g24.png}}
% 91
{\PTglyph{5}{t69_l03g25.png}}
% 92
{\PTglyph{5}{t69_l03g26.png}}
% 93
{\PTglyph{5}{t69_l03g27.png}}
% 94
{\PTglyph{5}{t69_l03g28.png}}
% 95
{\PTglyph{5}{t69_l03g29.png}}
% 96
{\PTglyph{5}{t69_l03g30.png}}
% 97
{\PTglyph{5}{t69_l03g31.png}}
% 98
{\PTglyph{5}{t69_l03g32.png}}
% 99
{\PTglyph{5}{t69_l03g33.png}}
% 100
{\PTglyph{5}{t69_l03g34.png}}
% 101
{\PTglyph{5}{t69_l04g01.png}}
% 102
{\PTglyph{5}{t69_l04g02.png}}
% 103
{\PTglyph{5}{t69_l04g03.png}}
% 104
{\PTglyph{5}{t69_l04g04.png}}
% 105
{\PTglyph{5}{t69_l04g05.png}}
% 106
{\PTglyph{5}{t69_l04g06.png}}
% 107
{\PTglyph{5}{t69_l04g07.png}}
% 108
{\PTglyph{5}{t69_l04g08.png}}
% 109
{\PTglyph{5}{t69_l04g09.png}}
% 110
{\PTglyph{5}{t69_l04g10.png}}
% 111
{\PTglyph{5}{t69_l04g11.png}}
% 112
{\PTglyph{5}{t69_l04g12.png}}
% 113
{\PTglyph{5}{t69_l04g13.png}}
% 114
{\PTglyph{5}{t69_l04g14.png}}
% 115
{\PTglyph{5}{t69_l04g15.png}}
% 116
{\PTglyph{5}{t69_l04g16.png}}
% 117
{\PTglyph{5}{t69_l04g17.png}}
% 118
{\PTglyph{5}{t69_l04g18.png}}
% 119
{\PTglyph{5}{t69_l04g19.png}}
% 120
{\PTglyph{5}{t69_l04g20.png}}
% 121
{\PTglyph{5}{t69_l04g21.png}}
% 122
{\PTglyph{5}{t69_l04g22.png}}
% 123
{\PTglyph{5}{t69_l04g23.png}}
% 124
{\PTglyph{5}{t69_l04g24.png}}
% 125
{\PTglyph{5}{t69_l04g25.png}}
% 126
{\PTglyph{5}{t69_l04g26.png}}
% 127
{\PTglyph{5}{t69_l04g27.png}}
% 128
{\PTglyph{5}{t69_l04g28.png}}
//
%%% Local Variables:
%%% mode: latex
%%% TeX-engine: luatex
%%% TeX-master: shared
%%% End:

%//
%\glpismo%
 \glpismo
% 1
{\PTglyphid{U2-20_0101}}
% 2
{\PTglyphid{U2-20_0102}}
% 3
{\PTglyphid{U2-20_0103}}
% 4
{\PTglyphid{U2-20_0104}}
% 5
{\PTglyphid{U2-20_0105}}
% 6
{\PTglyphid{U2-20_0106}}
% 7
{\PTglyphid{U2-20_0107}}
% 8
{\PTglyphid{U2-20_0108}}
% 9
{\PTglyphid{U2-20_0109}}
% 10
{\PTglyphid{U2-20_0110}}
% 11
{\PTglyphid{U2-20_0111}}
% 12
{\PTglyphid{U2-20_0112}}
% 13
{\PTglyphid{U2-20_0113}}
% 14
{\PTglyphid{U2-20_0114}}
% 15
{\PTglyphid{U2-20_0115}}
% 16
{\PTglyphid{U2-20_0116}}
% 17
{\PTglyphid{U2-20_0117}}
% 18
{\PTglyphid{U2-20_0118}}
% 19
{\PTglyphid{U2-20_0119}}
% 20
{\PTglyphid{U2-20_0120}}
% 21
{\PTglyphid{U2-20_0121}}
% 22
{\PTglyphid{U2-20_0122}}
% 23
{\PTglyphid{U2-20_0123}}
% 24
{\PTglyphid{U2-20_0124}}
% 25
{\PTglyphid{U2-20_0125}}
% 26
{\PTglyphid{U2-20_0126}}
% 27
{\PTglyphid{U2-20_0201}}
% 28
{\PTglyphid{U2-20_0202}}
% 29
{\PTglyphid{U2-20_0203}}
% 30
{\PTglyphid{U2-20_0204}}
% 31
{\PTglyphid{U2-20_0205}}
% 32
{\PTglyphid{U2-20_0206}}
% 33
{\PTglyphid{U2-20_0207}}
% 34
{\PTglyphid{U2-20_0208}}
% 35
{\PTglyphid{U2-20_0209}}
% 36
{\PTglyphid{U2-20_0210}}
% 37
{\PTglyphid{U2-20_0211}}
% 38
{\PTglyphid{U2-20_0212}}
% 39
{\PTglyphid{U2-20_0213}}
% 40
{\PTglyphid{U2-20_0214}}
% 41
{\PTglyphid{U2-20_0215}}
% 42
{\PTglyphid{U2-20_0216}}
% 43
{\PTglyphid{U2-20_0217}}
% 44
{\PTglyphid{U2-20_0218}}
% 45
{\PTglyphid{U2-20_0219}}
% 46
{\PTglyphid{U2-20_0220}}
% 47
{\PTglyphid{U2-20_0221}}
% 48
{\PTglyphid{U2-20_0222}}
% 49
{\PTglyphid{U2-20_0223}}
% 50
{\PTglyphid{U2-20_0224}}
% 51
{\PTglyphid{U2-20_0225}}
% 52
{\PTglyphid{U2-20_0226}}
% 53
{\PTglyphid{U2-20_0227}}
% 54
{\PTglyphid{U2-20_0228}}
% 55
{\PTglyphid{U2-20_0229}}
% 56
{\PTglyphid{U2-20_0230}}
% 57
{\PTglyphid{U2-20_0231}}
% 58
{\PTglyphid{U2-20_0232}}
% 59
{\PTglyphid{U2-20_0233}}
% 60
{\PTglyphid{U2-20_0234}}
% 61
{\PTglyphid{U2-20_0235}}
% 62
{\PTglyphid{U2-20_0236}}
% 63
{\PTglyphid{U2-20_0237}}
% 64
{\PTglyphid{U2-20_0238}}
% 65
{\PTglyphid{U2-20_0239}}
% 66
{\PTglyphid{U2-20_0240}}
% 67
{\PTglyphid{U2-20_0301}}
% 68
{\PTglyphid{U2-20_0302}}
% 69
{\PTglyphid{U2-20_0303}}
% 70
{\PTglyphid{U2-20_0304}}
% 71
{\PTglyphid{U2-20_0305}}
% 72
{\PTglyphid{U2-20_0306}}
% 73
{\PTglyphid{U2-20_0307}}
% 74
{\PTglyphid{U2-20_0308}}
% 75
{\PTglyphid{U2-20_0309}}
% 76
{\PTglyphid{U2-20_0310}}
% 77
{\PTglyphid{U2-20_0311}}
% 78
{\PTglyphid{U2-20_0312}}
% 79
{\PTglyphid{U2-20_0313}}
% 80
{\PTglyphid{U2-20_0314}}
% 81
{\PTglyphid{U2-20_0315}}
% 82
{\PTglyphid{U2-20_0316}}
% 83
{\PTglyphid{U2-20_0317}}
% 84
{\PTglyphid{U2-20_0318}}
% 85
{\PTglyphid{U2-20_0319}}
% 86
{\PTglyphid{U2-20_0320}}
% 87
{\PTglyphid{U2-20_0321}}
% 88
{\PTglyphid{U2-20_0322}}
% 89
{\PTglyphid{U2-20_0323}}
% 90
{\PTglyphid{U2-20_0324}}
% 91
{\PTglyphid{U2-20_0325}}
% 92
{\PTglyphid{U2-20_0326}}
% 93
{\PTglyphid{U2-20_0327}}
% 94
{\PTglyphid{U2-20_0328}}
% 95
{\PTglyphid{U2-20_0329}}
% 96
{\PTglyphid{U2-20_0330}}
% 97
{\PTglyphid{U2-20_0331}}
% 98
{\PTglyphid{U2-20_0332}}
% 99
{\PTglyphid{U2-20_0333}}
% 100
{\PTglyphid{U2-20_0334}}
% 101
{\PTglyphid{U2-20_0401}}
% 102
{\PTglyphid{U2-20_0402}}
% 103
{\PTglyphid{U2-20_0403}}
% 104
{\PTglyphid{U2-20_0404}}
% 105
{\PTglyphid{U2-20_0405}}
% 106
{\PTglyphid{U2-20_0406}}
% 107
{\PTglyphid{U2-20_0407}}
% 108
{\PTglyphid{U2-20_0408}}
% 109
{\PTglyphid{U2-20_0409}}
% 110
{\PTglyphid{U2-20_0410}}
% 111
{\PTglyphid{U2-20_0411}}
% 112
{\PTglyphid{U2-20_0412}}
% 113
{\PTglyphid{U2-20_0413}}
% 114
{\PTglyphid{U2-20_0414}}
% 115
{\PTglyphid{U2-20_0415}}
% 116
{\PTglyphid{U2-20_0416}}
% 117
{\PTglyphid{U2-20_0417}}
% 118
{\PTglyphid{U2-20_0418}}
% 119
{\PTglyphid{U2-20_0419}}
% 120
{\PTglyphid{U2-20_0420}}
% 121
{\PTglyphid{U2-20_0421}}
% 122
{\PTglyphid{U2-20_0422}}
% 123
{\PTglyphid{U2-20_0423}}
% 124
{\PTglyphid{U2-20_0424}}
% 125
{\PTglyphid{U2-20_0425}}
% 126
{\PTglyphid{U2-20_0426}}
% 127
{\PTglyphid{U2-20_0427}}
% 128
{\PTglyphid{U2-20_0428}}
//
\endgl \xe
%%% Local Variables:
%%% mode: latex
%%% TeX-engine: luatex
%%% TeX-master: shared
%%% End:

% //
%\endgl \xe


 \newpage
 
%%%%%%%%%%%%%%%%%%%%%%%%%%%%%%%%%%%%%%%%%%%%%%%%%%%%%%%%%%%%%%%%%%%%%%%%%%%%%%%
% from meta.csv
% 70,Ungler2-21_PT07_362.djvu,Ungler2,21,07,362
% 
%%%%%%%%%%%%%%%%%%%%%%%%%%%%%%%%%%%%%%%%%%%%%%%%%%%%%%%%%%%%%%%%%%%%%%%%%%%%%%%

 
% from dsed4test:
% Ungler2-21_PT07_362_4dsed.txt:Note "21. Pismo tekstowe, antykwa Qu|(H). Stopień 20 ww. = 113 mm. — Tabl. 362."
% Ungler2-21_PT07_362_4dsed.txt:Note1 "Character set table prepared by Anna Wolińska"

 \pismoPL{Florian Ungler druga drukarnia 21. Pismo tekstowe, antykwa Qu|(H). Stopień 20 ww. = 113
   mm. — Tabl. 362 [pierwszy zestaw].}
  
 \pismoEN{Florian Ungler second house 21. Roman text font, typeface
   Qu|(H). Type size 20 lines = 113 mm. — Plate 362 [first set].}

\plate{361[1]}{VII}{1970}

The plate prepared by Henryk Bułhak.\\
The font table prepared by Henryk Bułhak and Anna Wolińska.

\bigskip

\exampleBib{VII:142}

\bigskip \exampleDesc{GEORGIUS LIBANUS: In sponsalibus Ioachimi
  Brandenburgensis cum Hedvige Sigismundi regis Poloniae filia oratio.
  Kraków, Florian Ungler, [po 7 IX] 1535. 8⁰.}

 \medskip
 \examplePage{\textit{Karta B₄a.}}

  \bigskip
  \exampleLib{Biblioteka Jagiellońska. Kraków.}

 \bigskip \exampleRef{\textit{Estreicher XXI 255. Wierzbowski 2182.}}

% %  \bigskip

 % \exampleDig{\url{https://cyfrowe.mnk.pl/dlibra/publication/27252/}
 %   page 4, \url{https://dbc.wroc.pl/dlibra/publication/13405/} page 6}
 

 % Pismo 21: tekst i pierwszy zestaw. — Pismo 22: drugi zestaw. — Przerywnik 3: z pismem 22.
\bigskip

\fontID{U2-21}{70}

\fontstat{78}

% \exdisplay \bg \gla
 \exdisplay \bg \gla
% 1
{\PTglyph{5}{t70_l01g01.png}}
% 2
{\PTglyph{5}{t70_l01g02.png}}
% 3
{\PTglyph{5}{t70_l01g03.png}}
% 4
{\PTglyph{5}{t70_l01g04.png}}
% 5
{\PTglyph{5}{t70_l01g05.png}}
% 6
{\PTglyph{5}{t70_l01g06.png}}
% 7
{\PTglyph{5}{t70_l01g07.png}}
% 8
{\PTglyph{5}{t70_l01g08.png}}
% 9
{\PTglyph{5}{t70_l01g09.png}}
% 10
{\PTglyph{5}{t70_l01g10.png}}
% 11
{\PTglyph{5}{t70_l01g11.png}}
% 12
{\PTglyph{5}{t70_l01g12.png}}
% 13
{\PTglyph{5}{t70_l01g13.png}}
% 14
{\PTglyph{5}{t70_l01g14.png}}
% 15
{\PTglyph{5}{t70_l01g15.png}}
% 16
{\PTglyph{5}{t70_l01g16.png}}
% 17
{\PTglyph{5}{t70_l01g17.png}}
% 18
{\PTglyph{5}{t70_l01g18.png}}
% 19
{\PTglyph{5}{t70_l01g19.png}}
% 20
{\PTglyph{5}{t70_l02g01.png}}
% 21
{\PTglyph{5}{t70_l02g02.png}}
% 22
{\PTglyph{5}{t70_l02g03.png}}
% 23
{\PTglyph{5}{t70_l02g04.png}}
% 24
{\PTglyph{5}{t70_l02g05.png}}
% 25
{\PTglyph{5}{t70_l02g06.png}}
% 26
{\PTglyph{5}{t70_l02g07.png}}
% 27
{\PTglyph{5}{t70_l02g08.png}}
% 28
{\PTglyph{5}{t70_l02g09.png}}
% 29
{\PTglyph{5}{t70_l02g10.png}}
% 30
{\PTglyph{5}{t70_l02g11.png}}
% 31
{\PTglyph{5}{t70_l02g12.png}}
% 32
{\PTglyph{5}{t70_l02g13.png}}
% 33
{\PTglyph{5}{t70_l02g14.png}}
% 34
{\PTglyph{5}{t70_l02g15.png}}
% 35
{\PTglyph{5}{t70_l02g16.png}}
% 36
{\PTglyph{5}{t70_l02g17.png}}
% 37
{\PTglyph{5}{t70_l02g18.png}}
% 38
{\PTglyph{5}{t70_l02g19.png}}
% 39
{\PTglyph{5}{t70_l02g20.png}}
% 40
{\PTglyph{5}{t70_l02g21.png}}
% 41
{\PTglyph{5}{t70_l02g22.png}}
% 42
{\PTglyph{5}{t70_l02g23.png}}
% 43
{\PTglyph{5}{t70_l02g24.png}}
% 44
{\PTglyph{5}{t70_l02g25.png}}
% 45
{\PTglyph{5}{t70_l02g26.png}}
% 46
{\PTglyph{5}{t70_l02g27.png}}
% 47
{\PTglyph{5}{t70_l02g28.png}}
% 48
{\PTglyph{5}{t70_l02g29.png}}
% 49
{\PTglyph{5}{t70_l02g30.png}}
% 50
{\PTglyph{5}{t70_l03g01.png}}
% 51
{\PTglyph{5}{t70_l03g02.png}}
% 52
{\PTglyph{5}{t70_l03g03.png}}
% 53
{\PTglyph{5}{t70_l03g04.png}}
% 54
{\PTglyph{5}{t70_l03g05.png}}
% 55
{\PTglyph{5}{t70_l03g06.png}}
% 56
{\PTglyph{5}{t70_l03g07.png}}
% 57
{\PTglyph{5}{t70_l03g08.png}}
% 58
{\PTglyph{5}{t70_l03g09.png}}
% 59
{\PTglyph{5}{t70_l03g10.png}}
% 60
{\PTglyph{5}{t70_l03g11.png}}
% 61
{\PTglyph{5}{t70_l03g12.png}}
% 62
{\PTglyph{5}{t70_l03g13.png}}
% 63
{\PTglyph{5}{t70_l03g14.png}}
% 64
{\PTglyph{5}{t70_l03g15.png}}
% 65
{\PTglyph{5}{t70_l03g16.png}}
% 66
{\PTglyph{5}{t70_l03g17.png}}
% 67
{\PTglyph{5}{t70_l03g18.png}}
% 68
{\PTglyph{5}{t70_l03g19.png}}
% 69
{\PTglyph{5}{t70_l03g20.png}}
% 70
{\PTglyph{5}{t70_l03g21.png}}
% 71
{\PTglyph{5}{t70_l03g22.png}}
% 72
{\PTglyph{5}{t70_l03g23.png}}
% 73
{\PTglyph{5}{t70_l03g24.png}}
% 74
{\PTglyph{5}{t70_l03g25.png}}
% 75
{\PTglyph{5}{t70_l03g26.png}}
% 76
{\PTglyph{5}{t70_l03g27.png}}
% 77
{\PTglyph{5}{t70_l03g28.png}}
% 78
{\PTglyph{5}{t70_l03g29.png}}
//
%%% Local Variables:
%%% mode: latex
%%% TeX-engine: luatex
%%% TeX-master: shared
%%% End:

%//
%\glpismo%
 \glpismo
% 1
{\PTglyphid{U2-21_0101}}
% 2
{\PTglyphid{U2-21_0102}}
% 3
{\PTglyphid{U2-21_0103}}
% 4
{\PTglyphid{U2-21_0104}}
% 5
{\PTglyphid{U2-21_0105}}
% 6
{\PTglyphid{U2-21_0106}}
% 7
{\PTglyphid{U2-21_0107}}
% 8
{\PTglyphid{U2-21_0108}}
% 9
{\PTglyphid{U2-21_0109}}
% 10
{\PTglyphid{U2-21_0110}}
% 11
{\PTglyphid{U2-21_0111}}
% 12
{\PTglyphid{U2-21_0112}}
% 13
{\PTglyphid{U2-21_0113}}
% 14
{\PTglyphid{U2-21_0114}}
% 15
{\PTglyphid{U2-21_0115}}
% 16
{\PTglyphid{U2-21_0116}}
% 17
{\PTglyphid{U2-21_0117}}
% 18
{\PTglyphid{U2-21_0118}}
% 19
{\PTglyphid{U2-21_0119}}
% 20
{\PTglyphid{U2-21_0201}}
% 21
{\PTglyphid{U2-21_0202}}
% 22
{\PTglyphid{U2-21_0203}}
% 23
{\PTglyphid{U2-21_0204}}
% 24
{\PTglyphid{U2-21_0205}}
% 25
{\PTglyphid{U2-21_0206}}
% 26
{\PTglyphid{U2-21_0207}}
% 27
{\PTglyphid{U2-21_0208}}
% 28
{\PTglyphid{U2-21_0209}}
% 29
{\PTglyphid{U2-21_0210}}
% 30
{\PTglyphid{U2-21_0211}}
% 31
{\PTglyphid{U2-21_0212}}
% 32
{\PTglyphid{U2-21_0213}}
% 33
{\PTglyphid{U2-21_0214}}
% 34
{\PTglyphid{U2-21_0215}}
% 35
{\PTglyphid{U2-21_0216}}
% 36
{\PTglyphid{U2-21_0217}}
% 37
{\PTglyphid{U2-21_0218}}
% 38
{\PTglyphid{U2-21_0219}}
% 39
{\PTglyphid{U2-21_0220}}
% 40
{\PTglyphid{U2-21_0221}}
% 41
{\PTglyphid{U2-21_0222}}
% 42
{\PTglyphid{U2-21_0223}}
% 43
{\PTglyphid{U2-21_0224}}
% 44
{\PTglyphid{U2-21_0225}}
% 45
{\PTglyphid{U2-21_0226}}
% 46
{\PTglyphid{U2-21_0227}}
% 47
{\PTglyphid{U2-21_0228}}
% 48
{\PTglyphid{U2-21_0229}}
% 49
{\PTglyphid{U2-21_0230}}
% 50
{\PTglyphid{U2-21_0301}}
% 51
{\PTglyphid{U2-21_0302}}
% 52
{\PTglyphid{U2-21_0303}}
% 53
{\PTglyphid{U2-21_0304}}
% 54
{\PTglyphid{U2-21_0305}}
% 55
{\PTglyphid{U2-21_0306}}
% 56
{\PTglyphid{U2-21_0307}}
% 57
{\PTglyphid{U2-21_0308}}
% 58
{\PTglyphid{U2-21_0309}}
% 59
{\PTglyphid{U2-21_0310}}
% 60
{\PTglyphid{U2-21_0311}}
% 61
{\PTglyphid{U2-21_0312}}
% 62
{\PTglyphid{U2-21_0313}}
% 63
{\PTglyphid{U2-21_0314}}
% 64
{\PTglyphid{U2-21_0315}}
% 65
{\PTglyphid{U2-21_0316}}
% 66
{\PTglyphid{U2-21_0317}}
% 67
{\PTglyphid{U2-21_0318}}
% 68
{\PTglyphid{U2-21_0319}}
% 69
{\PTglyphid{U2-21_0320}}
% 70
{\PTglyphid{U2-21_0321}}
% 71
{\PTglyphid{U2-21_0322}}
% 72
{\PTglyphid{U2-21_0323}}
% 73
{\PTglyphid{U2-21_0324}}
% 74
{\PTglyphid{U2-21_0325}}
% 75
{\PTglyphid{U2-21_0326}}
% 76
{\PTglyphid{U2-21_0327}}
% 77
{\PTglyphid{U2-21_0328}}
% 78
{\PTglyphid{U2-21_0329}}
//
\endgl \xe
%%% Local Variables:
%%% mode: latex
%%% TeX-engine: luatex
%%% TeX-master: shared
%%% End:

% //
%\endgl \xe


 \newpage
 
%%%%%%%%%%%%%%%%%%%%%%%%%%%%%%%%%%%%%%%%%%%%%%%%%%%%%%%%%%%%%%%%%%%%%%%%%%%%%%%
% from meta.csv
% 71,Ungler2-22_PT07_362.djvu,Ungler2,22,07,362
% 
%%%%%%%%%%%%%%%%%%%%%%%%%%%%%%%%%%%%%%%%%%%%%%%%%%%%%%%%%%%%%%%%%%%%%%%%%%%%%%%

 
% from dsed4test:
% Ungler2-22_PT07_362_4dsed.txt:Note "22. Pismo tytułowe i nagłówkowe, antykwa Q|u(C). Wysokość 1 ww. = 9 mm. - Tabl. 362."
% Ungler2-22_PT07_362_4dsed.txt:Note1 "Character set table prepared by Henryk Bułhak and Anna Wolińska"


 \pismoPL{Florian Ungler druga drukarnia 22. Pismo tytułowe i nagłówkowe, antykwa Q|u(C). Wysokość 1 ww. = 9 mm. - Tabl. 362 [drugi zestaw].}
  
 \pismoEN{Florian Ungler second house 22. Roman title and header font, typeface
   Q|u(C). Type size 1 line = 362 mm. — Plate 362 [second set].}

\plate{361[2]}{VII}{1970}

The plate prepared by Henryk Bułhak.\\
The font table prepared by Henryk Bułhak and Anna Wolińska.

\bigskip

% \exampleBib{VII:142}

% \bigskip \exampleDesc{142. GEORGIUS LIBANUS: In sponsalibus Ioachimi
%   Brandenburgensis cum Hedvige Sigismundi regis Poloniae filia oratio.
%   Kraków, Florian Ungler, [po 7 IX] 1535. 8⁰.}

%  \medskip
%  \examplePage{\textit{Karta B₄a.}}

%   \bigskip
%   \exampleLib{Biblioteka Jagiellońska. Kraków.}

%  \bigskip \exampleRef{\textit{Estreicher XXI 255. Wierzbowski 2182.}}

% %  \bigskip

 % \exampleDig{\url{https://cyfrowe.mnk.pl/dlibra/publication/27252/}
 %   page 4, \url{https://dbc.wroc.pl/dlibra/publication/13405/} page 6}
 

 % Pismo 21: tekst i pierwszy zestaw. — Pismo 22: drugi zestaw. — Przerywnik 3: z pismem 22.
\bigskip

\fontID{U2-22}{71}

\fontstat{50}

% \exdisplay \bg \gla
 \exdisplay \bg \gla
% 1
{\PTglyph{5}{t71_l01g01.png}}
% 2
{\PTglyph{5}{t71_l01g02.png}}
% 3
{\PTglyph{5}{t71_l01g03.png}}
% 4
{\PTglyph{5}{t71_l01g04.png}}
% 5
{\PTglyph{5}{t71_l01g05.png}}
% 6
{\PTglyph{5}{t71_l01g06.png}}
% 7
{\PTglyph{5}{t71_l01g07.png}}
% 8
{\PTglyph{5}{t71_l01g08.png}}
% 9
{\PTglyph{5}{t71_l01g09.png}}
% 10
{\PTglyph{5}{t71_l01g10.png}}
% 11
{\PTglyph{5}{t71_l01g11.png}}
% 12
{\PTglyph{5}{t71_l01g12.png}}
% 13
{\PTglyph{5}{t71_l01g13.png}}
% 14
{\PTglyph{5}{t71_l01g14.png}}
% 15
{\PTglyph{5}{t71_l02g01.png}}
% 16
{\PTglyph{5}{t71_l02g02.png}}
% 17
{\PTglyph{5}{t71_l02g03.png}}
% 18
{\PTglyph{5}{t71_l02g04.png}}
% 19
{\PTglyph{5}{t71_l02g05.png}}
% 20
{\PTglyph{5}{t71_l02g06.png}}
% 21
{\PTglyph{5}{t71_l02g07.png}}
% 22
{\PTglyph{5}{t71_l02g08.png}}
% 23
{\PTglyph{5}{t71_l02g09.png}}
% 24
{\PTglyph{5}{t71_l02g10.png}}
% 25
{\PTglyph{5}{t71_l02g11.png}}
% 26
{\PTglyph{5}{t71_l02g12.png}}
% 27
{\PTglyph{5}{t71_l02g13.png}}
% 28
{\PTglyph{5}{t71_l02g14.png}}
% 29
{\PTglyph{5}{t71_l02g15.png}}
% 30
{\PTglyph{5}{t71_l02g16.png}}
% 31
{\PTglyph{5}{t71_l02g17.png}}
% 32
{\PTglyph{5}{t71_l02g18.png}}
% 33
{\PTglyph{5}{t71_l02g19.png}}
% 34
{\PTglyph{5}{t71_l02g20.png}}
% 35
{\PTglyph{5}{t71_l02g21.png}}
% 36
{\PTglyph{5}{t71_l03g01.png}}
% 37
{\PTglyph{5}{t71_l03g02.png}}
% 38
{\PTglyph{5}{t71_l03g03.png}}
% 39
{\PTglyph{5}{t71_l03g04.png}}
% 40
{\PTglyph{5}{t71_l03g05.png}}
% 41
{\PTglyph{5}{t71_l03g06.png}}
% 42
{\PTglyph{5}{t71_l03g07.png}}
% 43
{\PTglyph{5}{t71_l03g08.png}}
% 44
{\PTglyph{5}{t71_l03g09.png}}
% 45
{\PTglyph{5}{t71_l03g10.png}}
% 46
{\PTglyph{5}{t71_l03g11.png}}
% 47
{\PTglyph{5}{t71_l03g12.png}}
% 48
{\PTglyph{5}{t71_l03g13.png}}
% 49
{\PTglyph{5}{t71_l03g14.png}}
% 50
{\PTglyph{5}{t71_l03g15.png}}
//
%%% Local Variables:
%%% mode: latex
%%% TeX-engine: luatex
%%% TeX-master: shared
%%% End:

%//
%\glpismo%
 \glpismo
% 1
{\PTglyphid{U2-22_0101}}
% 2
{\PTglyphid{U2-22_0102}}
% 3
{\PTglyphid{U2-22_0103}}
% 4
{\PTglyphid{U2-22_0104}}
% 5
{\PTglyphid{U2-22_0105}}
% 6
{\PTglyphid{U2-22_0106}}
% 7
{\PTglyphid{U2-22_0107}}
% 8
{\PTglyphid{U2-22_0108}}
% 9
{\PTglyphid{U2-22_0109}}
% 10
{\PTglyphid{U2-22_0110}}
% 11
{\PTglyphid{U2-22_0111}}
% 12
{\PTglyphid{U2-22_0112}}
% 13
{\PTglyphid{U2-22_0113}}
% 14
{\PTglyphid{U2-22_0114}}
% 15
{\PTglyphid{U2-22_0201}}
% 16
{\PTglyphid{U2-22_0202}}
% 17
{\PTglyphid{U2-22_0203}}
% 18
{\PTglyphid{U2-22_0204}}
% 19
{\PTglyphid{U2-22_0205}}
% 20
{\PTglyphid{U2-22_0206}}
% 21
{\PTglyphid{U2-22_0207}}
% 22
{\PTglyphid{U2-22_0208}}
% 23
{\PTglyphid{U2-22_0209}}
% 24
{\PTglyphid{U2-22_0210}}
% 25
{\PTglyphid{U2-22_0211}}
% 26
{\PTglyphid{U2-22_0212}}
% 27
{\PTglyphid{U2-22_0213}}
% 28
{\PTglyphid{U2-22_0214}}
% 29
{\PTglyphid{U2-22_0215}}
% 30
{\PTglyphid{U2-22_0216}}
% 31
{\PTglyphid{U2-22_0217}}
% 32
{\PTglyphid{U2-22_0218}}
% 33
{\PTglyphid{U2-22_0219}}
% 34
{\PTglyphid{U2-22_0220}}
% 35
{\PTglyphid{U2-22_0221}}
% 36
{\PTglyphid{U2-22_0301}}
% 37
{\PTglyphid{U2-22_0302}}
% 38
{\PTglyphid{U2-22_0303}}
% 39
{\PTglyphid{U2-22_0304}}
% 40
{\PTglyphid{U2-22_0305}}
% 41
{\PTglyphid{U2-22_0306}}
% 42
{\PTglyphid{U2-22_0307}}
% 43
{\PTglyphid{U2-22_0308}}
% 44
{\PTglyphid{U2-22_0309}}
% 45
{\PTglyphid{U2-22_0310}}
% 46
{\PTglyphid{U2-22_0311}}
% 47
{\PTglyphid{U2-22_0312}}
% 48
{\PTglyphid{U2-22_0313}}
% 49
{\PTglyphid{U2-22_0314}}
% 50
{\PTglyphid{U2-22_0315}}
//
\endgl \xe
%%% Local Variables:
%%% mode: latex
%%% TeX-engine: luatex
%%% TeX-master: shared
%%% End:

% //
%\endgl \xe

 
 \newpage
 
%%%%%%%%%%%%%%%%%%%%%%%%%%%%%%%%%%%%%%%%%%%%%%%%%%%%%%%%%%%%%%%%%%%%%%%%%%%%%%%
% from meta.csv
% 72,Wirzbięta-01_PT09_469.djvu,Wirzbięta,01,09,469
% 
%%%%%%%%%%%%%%%%%%%%%%%%%%%%%%%%%%%%%%%%%%%%%%%%%%%%%%%%%%%%%%%%%%%%%%%%%%%%%%%

 
% from dsed4test:
% Wirzbięta-01_PT09_469_4dsed.txt:Note "1. Pismo nagłówkowe, fraktura H. Schönspergera. Stopień 1 w, = 14,5 mm. — Tabl. 417, 419, 420, 457, 464, 467, 469, 471. [469]"
% Wirzbięta-01_PT09_469_4dsed.txt:Note1 "Character set table prepared by Alodia Kawecka Gryczowa"


 \pismoPL{Maciej Wirzbięta 1. Pismo nagłówkowe, fraktura H. Schönspergera. Stopień 1 w. = 14,5 mm. — Tabl. 417, 419, 420, 457, 464, 467, 469, 471.}
  
 \pismoEN{Maciej Wirzbięta 1. Header font, H. Schönsperger
   Fraktur. Type size 1 line = 14,5 mm. — Plates 417, 419, 420, 457,
   464, 467, 469, 471.}

\plate{469[1]}{IX}{1974}

The plate prepared by Alodia Kawecka Gryczowa.\\
The font table prepared by Alodia Kawecka Gryczowa.

\bigskip

\exampleBib{IX:1}

\bigskip \exampleDesc{[MIKOŁAJ REJ]: Postylla. Kraków, Maciej Wirzbięta, [po 5 I] 1557. 2⁰.}
  
%   Brandenburgensis cum Hedvige Sigismundi regis Poloniae filia oratio.
%   Kraków, Florian Ungler, [po 7 IX] 1535. 8⁰.}

%  \medskip
%  \examplePage{\textit{Karta B₄a.}}
% Karta Pppbsa: przerywnik 8.
%   \bigskip
%   \exampleLib{Biblioteka Jagiellońska. Kraków.}

  \bigskip \exampleRef{\textit{Estreicher XXVI 180. Rostkowska 12.}}

% %  \bigskip

  \exampleDig{\url{https://www.wbc.poznan.pl/dlibra/publication/383976/},
    \url{https://www.dbc.wroc.pl/dlibra/publication/2204/}}.
% https://www.wbc.poznan.pl/dlibra/publication/383976/edition/298223
% https://www.dbc.wroc.pl/dlibra/publication/2204/edition/2248?language=en

 % Pisma 1—5. Rubryki a—qn. Qyfry 1—3. Przerywnik 8.
\bigskip

\fontID{Wi-01}{71}

\fontstat{85}

% \exdisplay \bg \gla
 \exdisplay \bg \gla
% 1
{\PTglyph{5}{t72_l01g01.png}}
% 2
{\PTglyph{5}{t72_l01g02.png}}
% 3
{\PTglyph{5}{t72_l01g03.png}}
% 4
{\PTglyph{5}{t72_l01g04.png}}
% 5
{\PTglyph{5}{t72_l01g05.png}}
% 6
{\PTglyph{5}{t72_l01g06.png}}
% 7
{\PTglyph{5}{t72_l01g07.png}}
% 8
{\PTglyph{5}{t72_l01g08.png}}
% 9
{\PTglyph{5}{t72_l01g09.png}}
% 10
{\PTglyph{5}{t72_l01g10.png}}
% 11
{\PTglyph{5}{t72_l01g11.png}}
% 12
{\PTglyph{5}{t72_l01g12.png}}
% 13
{\PTglyph{5}{t72_l01g13.png}}
% 14
{\PTglyph{5}{t72_l01g14.png}}
% 15
{\PTglyph{5}{t72_l01g15.png}}
% 16
{\PTglyph{5}{t72_l01g16.png}}
% 17
{\PTglyph{5}{t72_l01g17.png}}
% 18
{\PTglyph{5}{t72_l01g18.png}}
% 19
{\PTglyph{5}{t72_l01g19.png}}
% 20
{\PTglyph{5}{t72_l02g01.png}}
% 21
{\PTglyph{5}{t72_l02g02.png}}
% 22
{\PTglyph{5}{t72_l02g03.png}}
% 23
{\PTglyph{5}{t72_l02g04.png}}
% 24
{\PTglyph{5}{t72_l02g05.png}}
% 25
{\PTglyph{5}{t72_l02g06.png}}
% 26
{\PTglyph{5}{t72_l02g07.png}}
% 27
{\PTglyph{5}{t72_l03g01.png}}
% 28
{\PTglyph{5}{t72_l03g02.png}}
% 29
{\PTglyph{5}{t72_l03g03.png}}
% 30
{\PTglyph{5}{t72_l03g04.png}}
% 31
{\PTglyph{5}{t72_l03g05.png}}
% 32
{\PTglyph{5}{t72_l03g06.png}}
% 33
{\PTglyph{5}{t72_l03g07.png}}
% 34
{\PTglyph{5}{t72_l03g08.png}}
% 35
{\PTglyph{5}{t72_l03g09.png}}
% 36
{\PTglyph{5}{t72_l03g10.png}}
% 37
{\PTglyph{5}{t72_l03g11.png}}
% 38
{\PTglyph{5}{t72_l03g12.png}}
% 39
{\PTglyph{5}{t72_l03g13.png}}
% 40
{\PTglyph{5}{t72_l03g14.png}}
% 41
{\PTglyph{5}{t72_l03g15.png}}
% 42
{\PTglyph{5}{t72_l03g16.png}}
% 43
{\PTglyph{5}{t72_l03g17.png}}
% 44
{\PTglyph{5}{t72_l03g18.png}}
% 45
{\PTglyph{5}{t72_l03g19.png}}
% 46
{\PTglyph{5}{t72_l03g20.png}}
% 47
{\PTglyph{5}{t72_l03g21.png}}
% 48
{\PTglyph{5}{t72_l03g22.png}}
% 49
{\PTglyph{5}{t72_l03g23.png}}
% 50
{\PTglyph{5}{t72_l03g24.png}}
% 51
{\PTglyph{5}{t72_l03g25.png}}
% 52
{\PTglyph{5}{t72_l03g26.png}}
% 53
{\PTglyph{5}{t72_l03g27.png}}
% 54
{\PTglyph{5}{t72_l03g28.png}}
% 55
{\PTglyph{5}{t72_l03g29.png}}
% 56
{\PTglyph{5}{t72_l04g01.png}}
% 57
{\PTglyph{5}{t72_l04g02.png}}
% 58
{\PTglyph{5}{t72_l04g03.png}}
% 59
{\PTglyph{5}{t72_l04g04.png}}
% 60
{\PTglyph{5}{t72_l04g05.png}}
% 61
{\PTglyph{5}{t72_l04g06.png}}
% 62
{\PTglyph{5}{t72_l04g07.png}}
% 63
{\PTglyph{5}{t72_l04g08.png}}
% 64
{\PTglyph{5}{t72_l04g09.png}}
% 65
{\PTglyph{5}{t72_l04g10.png}}
% 66
{\PTglyph{5}{t72_l04g11.png}}
% 67
{\PTglyph{5}{t72_l04g12.png}}
% 68
{\PTglyph{5}{t72_l04g13.png}}
% 69
{\PTglyph{5}{t72_l04g14.png}}
% 70
{\PTglyph{5}{t72_l04g15.png}}
% 71
{\PTglyph{5}{t72_l04g16.png}}
% 72
{\PTglyph{5}{t72_l04g17.png}}
% 73
{\PTglyph{5}{t72_l04g18.png}}
% 74
{\PTglyph{5}{t72_l04g19.png}}
% 75
{\PTglyph{5}{t72_l04g20.png}}
% 76
{\PTglyph{5}{t72_l04g21.png}}
% 77
{\PTglyph{5}{t72_l04g22.png}}
% 78
{\PTglyph{5}{t72_l04g23.png}}
% 79
{\PTglyph{5}{t72_l04g24.png}}
% 80
{\PTglyph{5}{t72_l04g25.png}}
% 81
{\PTglyph{5}{t72_l04g26.png}}
% 82
{\PTglyph{5}{t72_l04g27.png}}
% 83
{\PTglyph{5}{t72_l04g28.png}}
% 84
{\PTglyph{5}{t72_l04g29.png}}
% 85
{\PTglyph{5}{t72_l04g30.png}}
//
%%% Local Variables:
%%% mode: latex
%%% TeX-engine: luatex
%%% TeX-master: shared
%%% End:

%//
%\glpismo%
 \glpismo
% 1
{\PTglyphid{Wi-01_0101}}
% 2
{\PTglyphid{Wi-01_0102}}
% 3
{\PTglyphid{Wi-01_0103}}
% 4
{\PTglyphid{Wi-01_0104}}
% 5
{\PTglyphid{Wi-01_0105}}
% 6
{\PTglyphid{Wi-01_0106}}
% 7
{\PTglyphid{Wi-01_0107}}
% 8
{\PTglyphid{Wi-01_0108}}
% 9
{\PTglyphid{Wi-01_0109}}
% 10
{\PTglyphid{Wi-01_0110}}
% 11
{\PTglyphid{Wi-01_0111}}
% 12
{\PTglyphid{Wi-01_0112}}
% 13
{\PTglyphid{Wi-01_0113}}
% 14
{\PTglyphid{Wi-01_0114}}
% 15
{\PTglyphid{Wi-01_0115}}
% 16
{\PTglyphid{Wi-01_0116}}
% 17
{\PTglyphid{Wi-01_0117}}
% 18
{\PTglyphid{Wi-01_0118}}
% 19
{\PTglyphid{Wi-01_0119}}
% 20
{\PTglyphid{Wi-01_0201}}
% 21
{\PTglyphid{Wi-01_0202}}
% 22
{\PTglyphid{Wi-01_0203}}
% 23
{\PTglyphid{Wi-01_0204}}
% 24
{\PTglyphid{Wi-01_0205}}
% 25
{\PTglyphid{Wi-01_0206}}
% 26
{\PTglyphid{Wi-01_0207}}
% 27
{\PTglyphid{Wi-01_0301}}
% 28
{\PTglyphid{Wi-01_0302}}
% 29
{\PTglyphid{Wi-01_0303}}
% 30
{\PTglyphid{Wi-01_0304}}
% 31
{\PTglyphid{Wi-01_0305}}
% 32
{\PTglyphid{Wi-01_0306}}
% 33
{\PTglyphid{Wi-01_0307}}
% 34
{\PTglyphid{Wi-01_0308}}
% 35
{\PTglyphid{Wi-01_0309}}
% 36
{\PTglyphid{Wi-01_0310}}
% 37
{\PTglyphid{Wi-01_0311}}
% 38
{\PTglyphid{Wi-01_0312}}
% 39
{\PTglyphid{Wi-01_0313}}
% 40
{\PTglyphid{Wi-01_0314}}
% 41
{\PTglyphid{Wi-01_0315}}
% 42
{\PTglyphid{Wi-01_0316}}
% 43
{\PTglyphid{Wi-01_0317}}
% 44
{\PTglyphid{Wi-01_0318}}
% 45
{\PTglyphid{Wi-01_0319}}
% 46
{\PTglyphid{Wi-01_0320}}
% 47
{\PTglyphid{Wi-01_0321}}
% 48
{\PTglyphid{Wi-01_0322}}
% 49
{\PTglyphid{Wi-01_0323}}
% 50
{\PTglyphid{Wi-01_0324}}
% 51
{\PTglyphid{Wi-01_0325}}
% 52
{\PTglyphid{Wi-01_0326}}
% 53
{\PTglyphid{Wi-01_0327}}
% 54
{\PTglyphid{Wi-01_0328}}
% 55
{\PTglyphid{Wi-01_0329}}
% 56
{\PTglyphid{Wi-01_0401}}
% 57
{\PTglyphid{Wi-01_0402}}
% 58
{\PTglyphid{Wi-01_0403}}
% 59
{\PTglyphid{Wi-01_0404}}
% 60
{\PTglyphid{Wi-01_0405}}
% 61
{\PTglyphid{Wi-01_0406}}
% 62
{\PTglyphid{Wi-01_0407}}
% 63
{\PTglyphid{Wi-01_0408}}
% 64
{\PTglyphid{Wi-01_0409}}
% 65
{\PTglyphid{Wi-01_0410}}
% 66
{\PTglyphid{Wi-01_0411}}
% 67
{\PTglyphid{Wi-01_0412}}
% 68
{\PTglyphid{Wi-01_0413}}
% 69
{\PTglyphid{Wi-01_0414}}
% 70
{\PTglyphid{Wi-01_0415}}
% 71
{\PTglyphid{Wi-01_0416}}
% 72
{\PTglyphid{Wi-01_0417}}
% 73
{\PTglyphid{Wi-01_0418}}
% 74
{\PTglyphid{Wi-01_0419}}
% 75
{\PTglyphid{Wi-01_0420}}
% 76
{\PTglyphid{Wi-01_0421}}
% 77
{\PTglyphid{Wi-01_0422}}
% 78
{\PTglyphid{Wi-01_0423}}
% 79
{\PTglyphid{Wi-01_0424}}
% 80
{\PTglyphid{Wi-01_0425}}
% 81
{\PTglyphid{Wi-01_0426}}
% 82
{\PTglyphid{Wi-01_0427}}
% 83
{\PTglyphid{Wi-01_0428}}
% 84
{\PTglyphid{Wi-01_0429}}
% 85
{\PTglyphid{Wi-01_0430}}
//
\endgl \xe
%%% Local Variables:
%%% mode: latex
%%% TeX-engine: luatex
%%% TeX-master: shared
%%% End:

% //
%\endgl \xe




 \newpage
 
%%%%%%%%%%%%%%%%%%%%%%%%%%%%%%%%%%%%%%%%%%%%%%%%%%%%%%%%%%%%%%%%%%%%%%%%%%%%%%%
% from meta.csv
% 73,Wirzbięta-01u_PT11_570.djvu,Wirzbięta,01u,11,570
% 
%%%%%%%%%%%%%%%%%%%%%%%%%%%%%%%%%%%%%%%%%%%%%%%%%%%%%%%%%%%%%%%%%%%%%%%%%%%%%%%

 
% from dsed4test:
% Wirzbięta-01u_PT11_570_4dsed.txt:Note "1. Uzupełnienie. [570]"
% Wirzbięta-01u_PT11_570_4dsed.txt:Note1 "Character set table prepared by Alodia Kawecka Gryczowa and Anna Wolińska"
% Wirzbięta-01u_PT11_570_4dsed.txt:Note2 "Photo (taken at  Warsaw Public Library - Central Library of the Masovian Voivodeship) converted to


 \pismoPL{Maciej Wirzbięta 1. Uzupełnienie. [Tabl. 570]}
  
 \pismoEN{Maciej Wirzbięta 1. A supplement.[Plate 570]}

\plate{570[1]}{XI}{1981}

The plate prepared by Alodia Kawecka Gryczowa.\\
The font table prepared by Alodia Kawecka Gryczowa and Anna Wrońska.

\bigskip

% \exampleBib{IX:1}

% \bigskip \exampleDesc{[MIKOŁAJ REJ]: Postylla. Kraków, Maciej Wirzbięta, [po 5 I] 1557. 2⁰.}
  
%   Brandenburgensis cum Hedvige Sigismundi regis Poloniae filia oratio.
%   Kraków, Florian Ungler, [po 7 IX] 1535. 8⁰.}

%  \medskip
%  \examplePage{\textit{Karta B₄a.}}
% Karta Pppbsa: przerywnik 8.
%   \bigskip
%   \exampleLib{Biblioteka Jagiellońska. Kraków.}

%  \bigskip \exampleRef{\textit{Estreicher XXVI 180. Rostkowska 12.}}

% %  \bigskip

  \exampleDig{\url{https://www.wbc.poznan.pl/dlibra/publication/383976/},
    \url{https://www.dbc.wroc.pl/dlibra/publication/2204/}}.
% https://www.wbc.poznan.pl/dlibra/publication/383976/edition/298223
% https://www.dbc.wroc.pl/dlibra/publication/2204/edition/2248?language=en

 % Pisma 1—5. Rubryki a—qn. Qyfry 1—3. Przerywnik 8.
\bigskip

\fontID{Wi-01u}{73}

\fontstat{4}

% \exdisplay \bg \gla
 \exdisplay \bg \gla
% 1
{\PTglyph{5}{t73_l01g01.png}}
% 2
{\PTglyph{5}{t73_l01g02.png}}
% 3
{\PTglyph{5}{t73_l01g03.png}}
% 4
{\PTglyph{5}{t73_l01g04.png}}
//
%%% Local Variables:
%%% mode: latex
%%% TeX-engine: luatex
%%% TeX-master: shared
%%% End:

%//
%\glpismo%
 \glpismo
% 1
{\PTglyphid{Wi-01u0101}}
% 2
{\PTglyphid{Wi-01u0102}}
% 3
{\PTglyphid{Wi-01u0103}}
% 4
{\PTglyphid{Wi-01u0104}}
//
\endgl \xe
%%% Local Variables:
%%% mode: latex
%%% TeX-engine: luatex
%%% TeX-master: shared
%%% End:

% //
%\endgl \xe


 \newpage
 
%%%%%%%%%%%%%%%%%%%%%%%%%%%%%%%%%%%%%%%%%%%%%%%%%%%%%%%%%%%%%%%%%%%%%%%%%%%%%%%
% from meta.csv
% 74,Wirzbięta-02_PT09_469.djvu,Wirzbięta,02,09,469
% 
%%%%%%%%%%%%%%%%%%%%%%%%%%%%%%%%%%%%%%%%%%%%%%%%%%%%%%%%%%%%%%%%%%%%%%%%%%%%%%%

 
% from dsed4test:
% Wirzbięta-02_PT09_469_4dsed.txt:Note "2. Pismo nagłówkowe i tekstowe, fraktura H. Schönspergera. Stopień 20 ww. = ca 160 mm. — Tabl. 420, 464, 469. [469]"
% Wirzbięta-02_PT09_469_4dsed.txt:Note1 "Character set table prepared by Alodia Kawecka Gryczowa"


 \pismoPL{Maciej Wirzbięta 2. Pismo nagłówkowe i tekstowe, fraktura
   H. Schönspergera. Stopień 20 ww. = ca 160 mm. — Tabl. 420, 464,
   469. [drugi zestaw]}
  
 \pismoEN{Maciej Wirzbięta 2. Header font, H. Schönsperger
   Fraktur. Type size 20 lines =  ca 160 mm. — Plates 420, 464, 469. [second set]}

\plate{469[2]}{IX}{1974}

The plate prepared by Alodia Kawecka Gryczowa.\\
The font table prepared by Alodia Kawecka Gryczowa.

\bigskip

\exampleBib{IX:1}

\bigskip \exampleDesc{[MIKOŁAJ REJ]: Postylla. Kraków, Maciej Wirzbięta, [po 5 I] 1557. 2⁰.}
  
%   Brandenburgensis cum Hedvige Sigismundi regis Poloniae filia oratio.
%   Kraków, Florian Ungler, [po 7 IX] 1535. 8⁰.}

%  \medskip
%  \examplePage{\textit{Karta B₄a.}}
% Karta Pppbsa: przerywnik 8.
%   \bigskip
%   \exampleLib{Biblioteka Jagiellońska. Kraków.}

  \bigskip \exampleRef{\textit{Estreicher XXVI 180. Rostkowska 12.}}

% %  \bigskip

  \exampleDig{\url{https://www.wbc.poznan.pl/dlibra/publication/383976/},
    \url{https://www.dbc.wroc.pl/dlibra/publication/2204/}}.
% https://www.wbc.poznan.pl/dlibra/publication/383976/edition/298223
% https://www.dbc.wroc.pl/dlibra/publication/2204/edition/2248?language=en

 % Pisma 1—5. Rubryki a—qn. Qyfry 1—3. Przerywnik 8.
\bigskip

\fontID{Wi-02}{74}

\fontstat{72}

% \exdisplay \bg \gla
 \exdisplay \bg \gla
% 1
{\PTglyph{5}{t74_l01g01.png}}
% 2
{\PTglyph{5}{t74_l01g02.png}}
% 3
{\PTglyph{5}{t74_l01g03.png}}
% 4
{\PTglyph{5}{t74_l01g04.png}}
% 5
{\PTglyph{5}{t74_l01g05.png}}
% 6
{\PTglyph{5}{t74_l01g06.png}}
% 7
{\PTglyph{5}{t74_l01g07.png}}
% 8
{\PTglyph{5}{t74_l01g08.png}}
% 9
{\PTglyph{5}{t74_l01g09.png}}
% 10
{\PTglyph{5}{t74_l01g10.png}}
% 11
{\PTglyph{5}{t74_l01g11.png}}
% 12
{\PTglyph{5}{t74_l01g12.png}}
% 13
{\PTglyph{5}{t74_l01g13.png}}
% 14
{\PTglyph{5}{t74_l01g14.png}}
% 15
{\PTglyph{5}{t74_l01g15.png}}
% 16
{\PTglyph{5}{t74_l01g16.png}}
% 17
{\PTglyph{5}{t74_l01g17.png}}
% 18
{\PTglyph{5}{t74_l01g18.png}}
% 19
{\PTglyph{5}{t74_l01g19.png}}
% 20
{\PTglyph{5}{t74_l01g20.png}}
% 21
{\PTglyph{5}{t74_l01g21.png}}
% 22
{\PTglyph{5}{t74_l01g22.png}}
% 23
{\PTglyph{5}{t74_l02g01.png}}
% 24
{\PTglyph{5}{t74_l02g02.png}}
% 25
{\PTglyph{5}{t74_l02g03.png}}
% 26
{\PTglyph{5}{t74_l02g04.png}}
% 27
{\PTglyph{5}{t74_l02g05.png}}
% 28
{\PTglyph{5}{t74_l02g06.png}}
% 29
{\PTglyph{5}{t74_l02g07.png}}
% 30
{\PTglyph{5}{t74_l02g08.png}}
% 31
{\PTglyph{5}{t74_l02g09.png}}
% 32
{\PTglyph{5}{t74_l02g10.png}}
% 33
{\PTglyph{5}{t74_l02g11.png}}
% 34
{\PTglyph{5}{t74_l02g12.png}}
% 35
{\PTglyph{5}{t74_l02g13.png}}
% 36
{\PTglyph{5}{t74_l02g14.png}}
% 37
{\PTglyph{5}{t74_l02g15.png}}
% 38
{\PTglyph{5}{t74_l02g16.png}}
% 39
{\PTglyph{5}{t74_l02g17.png}}
% 40
{\PTglyph{5}{t74_l02g18.png}}
% 41
{\PTglyph{5}{t74_l02g19.png}}
% 42
{\PTglyph{5}{t74_l02g20.png}}
% 43
{\PTglyph{5}{t74_l02g21.png}}
% 44
{\PTglyph{5}{t74_l02g22.png}}
% 45
{\PTglyph{5}{t74_l02g23.png}}
% 46
{\PTglyph{5}{t74_l02g24.png}}
% 47
{\PTglyph{5}{t74_l02g25.png}}
% 48
{\PTglyph{5}{t74_l02g26.png}}
% 49
{\PTglyph{5}{t74_l02g27.png}}
% 50
{\PTglyph{5}{t74_l02g28.png}}
% 51
{\PTglyph{5}{t74_l02g29.png}}
% 52
{\PTglyph{5}{t74_l02g30.png}}
% 53
{\PTglyph{5}{t74_l02g31.png}}
% 54
{\PTglyph{5}{t74_l02g32.png}}
% 55
{\PTglyph{5}{t74_l02g33.png}}
% 56
{\PTglyph{5}{t74_l02g34.png}}
% 57
{\PTglyph{5}{t74_l02g35.png}}
% 58
{\PTglyph{5}{t74_l02g36.png}}
% 59
{\PTglyph{5}{t74_l02g37.png}}
% 60
{\PTglyph{5}{t74_l02g38.png}}
% 61
{\PTglyph{5}{t74_l03g01.png}}
% 62
{\PTglyph{5}{t74_l03g02.png}}
% 63
{\PTglyph{5}{t74_l03g03.png}}
% 64
{\PTglyph{5}{t74_l03g04.png}}
% 65
{\PTglyph{5}{t74_l03g05.png}}
% 66
{\PTglyph{5}{t74_l03g06.png}}
% 67
{\PTglyph{5}{t74_l03g07.png}}
% 68
{\PTglyph{5}{t74_l03g08.png}}
% 69
{\PTglyph{5}{t74_l03g09.png}}
% 70
{\PTglyph{5}{t74_l03g10.png}}
% 71
{\PTglyph{5}{t74_l03g11.png}}
% 72
{\PTglyph{5}{t74_l03g12.png}}
//
%%% Local Variables:
%%% mode: latex
%%% TeX-engine: luatex
%%% TeX-master: shared
%%% End:

%//
%\glpismo%
 \glpismo
% 1
{\PTglyphid{Wi-02_0101}}
% 2
{\PTglyphid{Wi-02_0102}}
% 3
{\PTglyphid{Wi-02_0103}}
% 4
{\PTglyphid{Wi-02_0104}}
% 5
{\PTglyphid{Wi-02_0105}}
% 6
{\PTglyphid{Wi-02_0106}}
% 7
{\PTglyphid{Wi-02_0107}}
% 8
{\PTglyphid{Wi-02_0108}}
% 9
{\PTglyphid{Wi-02_0109}}
% 10
{\PTglyphid{Wi-02_0110}}
% 11
{\PTglyphid{Wi-02_0111}}
% 12
{\PTglyphid{Wi-02_0112}}
% 13
{\PTglyphid{Wi-02_0113}}
% 14
{\PTglyphid{Wi-02_0114}}
% 15
{\PTglyphid{Wi-02_0115}}
% 16
{\PTglyphid{Wi-02_0116}}
% 17
{\PTglyphid{Wi-02_0117}}
% 18
{\PTglyphid{Wi-02_0118}}
% 19
{\PTglyphid{Wi-02_0119}}
% 20
{\PTglyphid{Wi-02_0120}}
% 21
{\PTglyphid{Wi-02_0121}}
% 22
{\PTglyphid{Wi-02_0122}}
% 23
{\PTglyphid{Wi-02_0201}}
% 24
{\PTglyphid{Wi-02_0202}}
% 25
{\PTglyphid{Wi-02_0203}}
% 26
{\PTglyphid{Wi-02_0204}}
% 27
{\PTglyphid{Wi-02_0205}}
% 28
{\PTglyphid{Wi-02_0206}}
% 29
{\PTglyphid{Wi-02_0207}}
% 30
{\PTglyphid{Wi-02_0208}}
% 31
{\PTglyphid{Wi-02_0209}}
% 32
{\PTglyphid{Wi-02_0210}}
% 33
{\PTglyphid{Wi-02_0211}}
% 34
{\PTglyphid{Wi-02_0212}}
% 35
{\PTglyphid{Wi-02_0213}}
% 36
{\PTglyphid{Wi-02_0214}}
% 37
{\PTglyphid{Wi-02_0215}}
% 38
{\PTglyphid{Wi-02_0216}}
% 39
{\PTglyphid{Wi-02_0217}}
% 40
{\PTglyphid{Wi-02_0218}}
% 41
{\PTglyphid{Wi-02_0219}}
% 42
{\PTglyphid{Wi-02_0220}}
% 43
{\PTglyphid{Wi-02_0221}}
% 44
{\PTglyphid{Wi-02_0222}}
% 45
{\PTglyphid{Wi-02_0223}}
% 46
{\PTglyphid{Wi-02_0224}}
% 47
{\PTglyphid{Wi-02_0225}}
% 48
{\PTglyphid{Wi-02_0226}}
% 49
{\PTglyphid{Wi-02_0227}}
% 50
{\PTglyphid{Wi-02_0228}}
% 51
{\PTglyphid{Wi-02_0229}}
% 52
{\PTglyphid{Wi-02_0230}}
% 53
{\PTglyphid{Wi-02_0231}}
% 54
{\PTglyphid{Wi-02_0232}}
% 55
{\PTglyphid{Wi-02_0233}}
% 56
{\PTglyphid{Wi-02_0234}}
% 57
{\PTglyphid{Wi-02_0235}}
% 58
{\PTglyphid{Wi-02_0236}}
% 59
{\PTglyphid{Wi-02_0237}}
% 60
{\PTglyphid{Wi-02_0238}}
% 61
{\PTglyphid{Wi-02_0301}}
% 62
{\PTglyphid{Wi-02_0302}}
% 63
{\PTglyphid{Wi-02_0303}}
% 64
{\PTglyphid{Wi-02_0304}}
% 65
{\PTglyphid{Wi-02_0305}}
% 66
{\PTglyphid{Wi-02_0306}}
% 67
{\PTglyphid{Wi-02_0307}}
% 68
{\PTglyphid{Wi-02_0308}}
% 69
{\PTglyphid{Wi-02_0309}}
% 70
{\PTglyphid{Wi-02_0310}}
% 71
{\PTglyphid{Wi-02_0311}}
% 72
{\PTglyphid{Wi-02_0312}}
//
\endgl \xe
%%% Local Variables:
%%% mode: latex
%%% TeX-engine: luatex
%%% TeX-master: shared
%%% End:

% //
%\endgl \xe
 

 \newpage
 
%%%%%%%%%%%%%%%%%%%%%%%%%%%%%%%%%%%%%%%%%%%%%%%%%%%%%%%%%%%%%%%%%%%%%%%%%%%%%%%
% from meta.csv
% 75,Wirzbięta-02u_PT11_570.djvu,Wirzbięta,02u,11,570
% 
%%%%%%%%%%%%%%%%%%%%%%%%%%%%%%%%%%%%%%%%%%%%%%%%%%%%%%%%%%%%%%%%%%%%%%%%%%%%%%%

 
% from dsed4test:
% Wirzbięta-02u_PT11_570_4dsed.txt:Note "2. Uzupełnienie. [570]"
% Wirzbięta-02u_PT11_570_4dsed.txt:Note1 "Character set table prepared by Alodia Kawecka Gryczowa and Anna Wolińska"

 \pismoPL{Maciej Wirzbięta 2. Uzupełnienie. [Tabl. 570]}
  
 \pismoEN{Maciej Wirzbięta 2. A supplement.[Plate 570]}

\plate{570[3]}{XI}{1981}

The plate prepared by Alodia Kawecka Gryczowa.\\
The font table prepared by Alodia Kawecka Gryczowa and Anna Wrońska.

\bigskip

% \exampleBib{IX:1}

% \bigskip \exampleDesc{[MIKOŁAJ REJ]: Postylla. Kraków, Maciej Wirzbięta, [po 5 I] 1557. 2⁰.}
  
%   Brandenburgensis cum Hedvige Sigismundi regis Poloniae filia oratio.
%   Kraków, Florian Ungler, [po 7 IX] 1535. 8⁰.}

%  \medskip
%  \examplePage{\textit{Karta B₄a.}}
% Karta Pppbsa: przerywnik 8.
%   \bigskip
%   \exampleLib{Biblioteka Jagiellońska. Kraków.}

%  \bigskip \exampleRef{\textit{Estreicher XXVI 180. Rostkowska 12.}}

% %  \bigskip

  \exampleDig{\url{https://www.wbc.poznan.pl/dlibra/publication/383976/},
    \url{https://www.dbc.wroc.pl/dlibra/publication/2204/}}.
% https://www.wbc.poznan.pl/dlibra/publication/383976/edition/298223
% https://www.dbc.wroc.pl/dlibra/publication/2204/edition/2248?language=en

 % Pisma 1—5. Rubryki a—qn. Qyfry 1—3. Przerywnik 8.
\bigskip

\fontID{Wi-02u}{75}

\fontstat{129?}

% \exdisplay \bg \gla
 \exdisplay \bg \gla
% 1
{\PTglyph{5}{t75_l01g01.png}}
% 2
{\PTglyph{5}{t75_l01g02.png}}
% 3
{\PTglyph{5}{t75_l01g03.png}}
% 4
{\PTglyph{5}{t75_l01g04.png}}
% 5
{\PTglyph{5}{t75_l01g05.png}}
% 6
{\PTglyph{5}{t75_l01g06.png}}
% 7
{\PTglyph{5}{t75_l01g07.png}}
% 8
{\PTglyph{5}{t75_l01g08.png}}
% 9
{\PTglyph{5}{t75_l01g09.png}}
% 10
{\PTglyph{5}{t75_l01g10.png}}
% 11
{\PTglyph{5}{t75_l01g11.png}}
% 12
{\PTglyph{5}{t75_l01g12.png}}
//
%%% Local Variables:
%%% mode: latex
%%% TeX-engine: luatex
%%% TeX-master: shared
%%% End:

%//
%\glpismo%
 \glpismo
% 1
{\PTglyphid{Wi-02u0101}}
% 2
{\PTglyphid{Wi-02u0102}}
% 3
{\PTglyphid{Wi-02u0103}}
% 4
{\PTglyphid{Wi-02u0104}}
% 5
{\PTglyphid{Wi-02u0105}}
% 6
{\PTglyphid{Wi-02u0106}}
% 7
{\PTglyphid{Wi-02u0107}}
% 8
{\PTglyphid{Wi-02u0108}}
% 9
{\PTglyphid{Wi-02u0109}}
% 10
{\PTglyphid{Wi-02u0110}}
% 11
{\PTglyphid{Wi-02u0111}}
% 12
{\PTglyphid{Wi-02u0112}}
//
\endgl \xe
%%% Local Variables:
%%% mode: latex
%%% TeX-engine: luatex
%%% TeX-master: shared
%%% End:

% //
%\endgl \xe


 \newpage
 
%%%%%%%%%%%%%%%%%%%%%%%%%%%%%%%%%%%%%%%%%%%%%%%%%%%%%%%%%%%%%%%%%%%%%%%%%%%%%%%
% from meta.csv
% 76,Wirzbięta-03_PT09_469.djvu,Wirzbięta,03,09,469
% 
%%%%%%%%%%%%%%%%%%%%%%%%%%%%%%%%%%%%%%%%%%%%%%%%%%%%%%%%%%%%%%%%%%%%%%%%%%%%%%%

 
% from dsed4test:
% Wirzbięta-03_PT09_469_4dsed.txt:Note "3. Pismo tekstowe, szwabacha M⁸¹, Stopień 20 ww. = 113 mm. — Tabl. 417, 419, 420, 456—458, 464, 467—470. [469]"
% Wirzbięta-03_PT09_469_4dsed.txt:Note1 "Character set table prepared by Alodia Kawecka Gryczowa"


 \pismoPL{Maciej Wirzbięta 3. Pismo tekstowe, szwabacha M⁸¹, Stopień
   20 ww. = 113 mm. — Tabl. 417, 419, 420, 456—458,
   464, 467—470. [469 trzeci zestaw]}
  
 \pismoEN{Maciej Wirzbięta 3. Text font, H. Schönsperger
   Fraktur. Type size 20 lines =  113 mm. — Plates 417, 419, 420, 456--458, 464, 467-470. [469 third set]}

\plate{469[3]}{IX}{1974}

The plate prepared by Alodia Kawecka Gryczowa.\\
The font table prepared by Alodia Kawecka Gryczowa.

\bigskip

\exampleBib{IX:1}

\bigskip \exampleDesc{[MIKOŁAJ REJ]: Postylla. Kraków, Maciej Wirzbięta, [po 5 I] 1557. 2⁰.}
  
%   Brandenburgensis cum Hedvige Sigismundi regis Poloniae filia oratio.
%   Kraków, Florian Ungler, [po 7 IX] 1535. 8⁰.}

%  \medskip
%  \examplePage{\textit{Karta B₄a.}}
% Karta Pppbsa: przerywnik 8.
%   \bigskip
%   \exampleLib{Biblioteka Jagiellońska. Kraków.}

  \bigskip \exampleRef{\textit{Estreicher XXVI 180. Rostkowska 12.}}

% %  \bigskip

  \exampleDig{\url{https://www.wbc.poznan.pl/dlibra/publication/383976/},
    \url{https://www.dbc.wroc.pl/dlibra/publication/2204/}}.
% https://www.wbc.poznan.pl/dlibra/publication/383976/edition/298223
% https://www.dbc.wroc.pl/dlibra/publication/2204/edition/2248?language=en

 % Pisma 1—5. Rubryki a—qn. Qyfry 1—3. Przerywnik 8.
\bigskip

\fontID{Wi-03}{76}

\fontstat{129?}

% \exdisplay \bg \gla
 \exdisplay \bg \gla
% 1
{\PTglyph{5}{t76_l01g01.png}}
% 2
{\PTglyph{5}{t76_l01g02.png}}
% 3
{\PTglyph{5}{t76_l01g03.png}}
% 4
{\PTglyph{5}{t76_l01g04.png}}
% 5
{\PTglyph{5}{t76_l01g05.png}}
% 6
{\PTglyph{5}{t76_l01g06.png}}
% 7
{\PTglyph{5}{t76_l01g07.png}}
% 8
{\PTglyph{5}{t76_l01g08.png}}
% 9
{\PTglyph{5}{t76_l01g09.png}}
% 10
{\PTglyph{5}{t76_l01g10.png}}
% 11
{\PTglyph{5}{t76_l01g11.png}}
% 12
{\PTglyph{5}{t76_l01g12.png}}
% 13
{\PTglyph{5}{t76_l01g13.png}}
% 14
{\PTglyph{5}{t76_l01g14.png}}
% 15
{\PTglyph{5}{t76_l01g15.png}}
% 16
{\PTglyph{5}{t76_l01g16.png}}
% 17
{\PTglyph{5}{t76_l01g17.png}}
% 18
{\PTglyph{5}{t76_l01g18.png}}
% 19
{\PTglyph{5}{t76_l01g19.png}}
% 20
{\PTglyph{5}{t76_l01g20.png}}
% 21
{\PTglyph{5}{t76_l01g21.png}}
% 22
{\PTglyph{5}{t76_l01g22.png}}
% 23
{\PTglyph{5}{t76_l01g23.png}}
% 24
{\PTglyph{5}{t76_l01g24.png}}
% 25
{\PTglyph{5}{t76_l01g25.png}}
% 26
{\PTglyph{5}{t76_l02g01.png}}
% 27
{\PTglyph{5}{t76_l02g02.png}}
% 28
{\PTglyph{5}{t76_l02g03.png}}
% 29
{\PTglyph{5}{t76_l02g04.png}}
% 30
{\PTglyph{5}{t76_l02g05.png}}
% 31
{\PTglyph{5}{t76_l02g06.png}}
% 32
{\PTglyph{5}{t76_l02g07.png}}
% 33
{\PTglyph{5}{t76_l02g08.png}}
% 34
{\PTglyph{5}{t76_l02g09.png}}
% 35
{\PTglyph{5}{t76_l02g10.png}}
% 36
{\PTglyph{5}{t76_l02g11.png}}
% 37
{\PTglyph{5}{t76_l02g12.png}}
% 38
{\PTglyph{5}{t76_l02g13.png}}
% 39
{\PTglyph{5}{t76_l02g14.png}}
% 40
{\PTglyph{5}{t76_l02g15.png}}
% 41
{\PTglyph{5}{t76_l02g16.png}}
% 42
{\PTglyph{5}{t76_l02g17.png}}
% 43
{\PTglyph{5}{t76_l02g18.png}}
% 44
{\PTglyph{5}{t76_l02g19.png}}
% 45
{\PTglyph{5}{t76_l02g20.png}}
% 46
{\PTglyph{5}{t76_l02g21.png}}
% 47
{\PTglyph{5}{t76_l02g22.png}}
% 48
{\PTglyph{5}{t76_l02g23.png}}
% 49
{\PTglyph{5}{t76_l02g24.png}}
% 50
{\PTglyph{5}{t76_l02g25.png}}
% 51
{\PTglyph{5}{t76_l02g26.png}}
% 52
{\PTglyph{5}{t76_l02g27.png}}
% 53
{\PTglyph{5}{t76_l02g28.png}}
% 54
{\PTglyph{5}{t76_l02g29.png}}
% 55
{\PTglyph{5}{t76_l02g30.png}}
% 56
{\PTglyph{5}{t76_l02g31.png}}
% 57
{\PTglyph{5}{t76_l02g32.png}}
% 58
{\PTglyph{5}{t76_l02g33.png}}
% 59
{\PTglyph{5}{t76_l02g34.png}}
% 60
{\PTglyph{5}{t76_l02g35.png}}
% 61
{\PTglyph{5}{t76_l02g36.png}}
% 62
{\PTglyph{5}{t76_l02g37.png}}
% 63
{\PTglyph{5}{t76_l02g38.png}}
% 64
{\PTglyph{5}{t76_l03g01.png}}
% 65
{\PTglyph{5}{t76_l03g02.png}}
% 66
{\PTglyph{5}{t76_l03g03.png}}
% 67
{\PTglyph{5}{t76_l03g04.png}}
% 68
{\PTglyph{5}{t76_l03g05.png}}
% 69
{\PTglyph{5}{t76_l03g06.png}}
% 70
{\PTglyph{5}{t76_l03g07.png}}
% 71
{\PTglyph{5}{t76_l03g08.png}}
% 72
{\PTglyph{5}{t76_l03g09.png}}
% 73
{\PTglyph{5}{t76_l03g10.png}}
% 74
{\PTglyph{5}{t76_l03g11.png}}
% 75
{\PTglyph{5}{t76_l03g12.png}}
% 76
{\PTglyph{5}{t76_l03g13.png}}
% 77
{\PTglyph{5}{t76_l03g14.png}}
% 78
{\PTglyph{5}{t76_l03g15.png}}
% 79
{\PTglyph{5}{t76_l03g16.png}}
% 80
{\PTglyph{5}{t76_l03g17.png}}
% 81
{\PTglyph{5}{t76_l03g18.png}}
% 82
{\PTglyph{5}{t76_l03g19.png}}
% 83
{\PTglyph{5}{t76_l03g20.png}}
% 84
{\PTglyph{5}{t76_l03g21.png}}
% 85
{\PTglyph{5}{t76_l03g22.png}}
% 86
{\PTglyph{5}{t76_l03g23.png}}
% 87
{\PTglyph{5}{t76_l03g24.png}}
% 88
{\PTglyph{5}{t76_l03g25.png}}
% 89
{\PTglyph{5}{t76_l03g26.png}}
% 90
{\PTglyph{5}{t76_l03g27.png}}
% 91
{\PTglyph{5}{t76_l03g28.png}}
% 92
{\PTglyph{5}{t76_l03g29.png}}
% 93
{\PTglyph{5}{t76_l03g30.png}}
% 94
{\PTglyph{5}{t76_l03g31.png}}
% 95
{\PTglyph{5}{t76_l03g32.png}}
//
%%% Local Variables:
%%% mode: latex
%%% TeX-engine: luatex
%%% TeX-master: shared
%%% End:

%//
%\glpismo%
 \glpismo
% 1
{\PTglyphid{Wi-03_0101}}
% 2
{\PTglyphid{Wi-03_0102}}
% 3
{\PTglyphid{Wi-03_0103}}
% 4
{\PTglyphid{Wi-03_0104}}
% 5
{\PTglyphid{Wi-03_0105}}
% 6
{\PTglyphid{Wi-03_0106}}
% 7
{\PTglyphid{Wi-03_0107}}
% 8
{\PTglyphid{Wi-03_0108}}
% 9
{\PTglyphid{Wi-03_0109}}
% 10
{\PTglyphid{Wi-03_0110}}
% 11
{\PTglyphid{Wi-03_0111}}
% 12
{\PTglyphid{Wi-03_0112}}
% 13
{\PTglyphid{Wi-03_0113}}
% 14
{\PTglyphid{Wi-03_0114}}
% 15
{\PTglyphid{Wi-03_0115}}
% 16
{\PTglyphid{Wi-03_0116}}
% 17
{\PTglyphid{Wi-03_0117}}
% 18
{\PTglyphid{Wi-03_0118}}
% 19
{\PTglyphid{Wi-03_0119}}
% 20
{\PTglyphid{Wi-03_0120}}
% 21
{\PTglyphid{Wi-03_0121}}
% 22
{\PTglyphid{Wi-03_0122}}
% 23
{\PTglyphid{Wi-03_0123}}
% 24
{\PTglyphid{Wi-03_0124}}
% 25
{\PTglyphid{Wi-03_0125}}
% 26
{\PTglyphid{Wi-03_0201}}
% 27
{\PTglyphid{Wi-03_0202}}
% 28
{\PTglyphid{Wi-03_0203}}
% 29
{\PTglyphid{Wi-03_0204}}
% 30
{\PTglyphid{Wi-03_0205}}
% 31
{\PTglyphid{Wi-03_0206}}
% 32
{\PTglyphid{Wi-03_0207}}
% 33
{\PTglyphid{Wi-03_0208}}
% 34
{\PTglyphid{Wi-03_0209}}
% 35
{\PTglyphid{Wi-03_0210}}
% 36
{\PTglyphid{Wi-03_0211}}
% 37
{\PTglyphid{Wi-03_0212}}
% 38
{\PTglyphid{Wi-03_0213}}
% 39
{\PTglyphid{Wi-03_0214}}
% 40
{\PTglyphid{Wi-03_0215}}
% 41
{\PTglyphid{Wi-03_0216}}
% 42
{\PTglyphid{Wi-03_0217}}
% 43
{\PTglyphid{Wi-03_0218}}
% 44
{\PTglyphid{Wi-03_0219}}
% 45
{\PTglyphid{Wi-03_0220}}
% 46
{\PTglyphid{Wi-03_0221}}
% 47
{\PTglyphid{Wi-03_0222}}
% 48
{\PTglyphid{Wi-03_0223}}
% 49
{\PTglyphid{Wi-03_0224}}
% 50
{\PTglyphid{Wi-03_0225}}
% 51
{\PTglyphid{Wi-03_0226}}
% 52
{\PTglyphid{Wi-03_0227}}
% 53
{\PTglyphid{Wi-03_0228}}
% 54
{\PTglyphid{Wi-03_0229}}
% 55
{\PTglyphid{Wi-03_0230}}
% 56
{\PTglyphid{Wi-03_0231}}
% 57
{\PTglyphid{Wi-03_0232}}
% 58
{\PTglyphid{Wi-03_0233}}
% 59
{\PTglyphid{Wi-03_0234}}
% 60
{\PTglyphid{Wi-03_0235}}
% 61
{\PTglyphid{Wi-03_0236}}
% 62
{\PTglyphid{Wi-03_0237}}
% 63
{\PTglyphid{Wi-03_0238}}
% 64
{\PTglyphid{Wi-03_0301}}
% 65
{\PTglyphid{Wi-03_0302}}
% 66
{\PTglyphid{Wi-03_0303}}
% 67
{\PTglyphid{Wi-03_0304}}
% 68
{\PTglyphid{Wi-03_0305}}
% 69
{\PTglyphid{Wi-03_0306}}
% 70
{\PTglyphid{Wi-03_0307}}
% 71
{\PTglyphid{Wi-03_0308}}
% 72
{\PTglyphid{Wi-03_0309}}
% 73
{\PTglyphid{Wi-03_0310}}
% 74
{\PTglyphid{Wi-03_0311}}
% 75
{\PTglyphid{Wi-03_0312}}
% 76
{\PTglyphid{Wi-03_0313}}
% 77
{\PTglyphid{Wi-03_0314}}
% 78
{\PTglyphid{Wi-03_0315}}
% 79
{\PTglyphid{Wi-03_0316}}
% 80
{\PTglyphid{Wi-03_0317}}
% 81
{\PTglyphid{Wi-03_0318}}
% 82
{\PTglyphid{Wi-03_0319}}
% 83
{\PTglyphid{Wi-03_0320}}
% 84
{\PTglyphid{Wi-03_0321}}
% 85
{\PTglyphid{Wi-03_0322}}
% 86
{\PTglyphid{Wi-03_0323}}
% 87
{\PTglyphid{Wi-03_0324}}
% 88
{\PTglyphid{Wi-03_0325}}
% 89
{\PTglyphid{Wi-03_0326}}
% 90
{\PTglyphid{Wi-03_0327}}
% 91
{\PTglyphid{Wi-03_0328}}
% 92
{\PTglyphid{Wi-03_0329}}
% 93
{\PTglyphid{Wi-03_0330}}
% 94
{\PTglyphid{Wi-03_0331}}
% 95
{\PTglyphid{Wi-03_0332}}
//
\endgl \xe
%%% Local Variables:
%%% mode: latex
%%% TeX-engine: luatex
%%% TeX-master: shared
%%% End:

% //
%\endgl \xe



 \newpage
 
%%%%%%%%%%%%%%%%%%%%%%%%%%%%%%%%%%%%%%%%%%%%%%%%%%%%%%%%%%%%%%%%%%%%%%%%%%%%%%%
% from meta.csv
% 77,Wirzbięta-03u_PT11_570.djvu,Wirzbięta,03u,11,570
% 
%%%%%%%%%%%%%%%%%%%%%%%%%%%%%%%%%%%%%%%%%%%%%%%%%%%%%%%%%%%%%%%%%%%%%%%%%%%%%%%

 
% from dsed4test:
% Wirzbięta-03u_PT11_570_4dsed.txt:Note "3. Uzupełnienie. [570]"
% Wirzbięta-03u_PT11_570_4dsed.txt:Note1 "Character set table prepared by Alodia Kawecka Gryczowa and Anna Wolińska"

 \pismoPL{Maciej Wirzbięta 3. Uzupełnienie. [Tabl. 570]}
  
 \pismoEN{Maciej Wirzbięta 3. A supplement. [Plate 570]}

\plate{570[3]}{XI}{1981}

The plate prepared by Alodia Kawecka Gryczowa.\\
The font table prepared by Alodia Kawecka Gryczowa and Anna Wrońska.

\bigskip

% \exampleBib{IX:1}

% \bigskip \exampleDesc{[MIKOŁAJ REJ]: Postylla. Kraków, Maciej Wirzbięta, [po 5 I] 1557. 2⁰.}
  
%   Brandenburgensis cum Hedvige Sigismundi regis Poloniae filia oratio.
%   Kraków, Florian Ungler, [po 7 IX] 1535. 8⁰.}

%  \medskip
%  \examplePage{\textit{Karta B₄a.}}
% Karta Pppbsa: przerywnik 8.
%   \bigskip
%   \exampleLib{Biblioteka Jagiellońska. Kraków.}

%  \bigskip \exampleRef{\textit{Estreicher XXVI 180. Rostkowska 12.}}

% %  \bigskip

  \exampleDig{\url{https://www.wbc.poznan.pl/dlibra/publication/383976/},
    \url{https://www.dbc.wroc.pl/dlibra/publication/2204/}}.
% https://www.wbc.poznan.pl/dlibra/publication/383976/edition/298223
% https://www.dbc.wroc.pl/dlibra/publication/2204/edition/2248?language=en

 % Pisma 1—5. Rubryki a—qn. Qyfry 1—3. Przerywnik 8.
\bigskip

\fontID{Wi-03u}{77}

\fontstat{1}

% \exdisplay \bg \gla
 \exdisplay \bg \gla
% 1
{\PTglyph{5}{t77_l01g01.png}}
//
%%% Local Variables:
%%% mode: latex
%%% TeX-engine: luatex
%%% TeX-master: shared
%%% End:

%//
%\glpismo%
 \glpismo
% 1
{\PTglyphid{Wi-03u0101}}
//
\endgl \xe
%%% Local Variables:
%%% mode: latex
%%% TeX-engine: luatex
%%% TeX-master: shared
%%% End:

% //
%\endgl \xe


 \newpage
 
%%%%%%%%%%%%%%%%%%%%%%%%%%%%%%%%%%%%%%%%%%%%%%%%%%%%%%%%%%%%%%%%%%%%%%%%%%%%%%%
% from meta.csv
% 78,Wirzbięta-03uu_PT11_570.djvu,Wirzbięta,03uu,11,570
% 
%%%%%%%%%%%%%%%%%%%%%%%%%%%%%%%%%%%%%%%%%%%%%%%%%%%%%%%%%%%%%%%%%%%%%%%%%%%%%%%

 
% from dsed4test:
% Wirzbięta-03uu_PT11_570_4dsed.txt:Note "3. Uzupełnienie - cyfry. [570]"
% Wirzbięta-03uu_PT11_570_4dsed.txt:Note1 "Character set table prepared by Alodia Kawecka Gryczowa and Anna Wolińska"

 \pismoPL{Maciej Wirzbięta 3[?]. Uzupełnienie. [Tabl. 570]}
  
 \pismoEN{Maciej Wirzbięta 3[?]. A supplement. [Plate 570]}

\plate{570[4]}{XI}{1981}

The plate prepared by Alodia Kawecka Gryczowa.\\
The font table prepared by Alodia Kawecka Gryczowa and Anna Wrońska.

\bigskip

% \exampleBib{IX:1}

% \bigskip \exampleDesc{[MIKOŁAJ REJ]: Postylla. Kraków, Maciej Wirzbięta, [po 5 I] 1557. 2⁰.}
  
%   Brandenburgensis cum Hedvige Sigismundi regis Poloniae filia oratio.
%   Kraków, Florian Ungler, [po 7 IX] 1535. 8⁰.}

%  \medskip
%  \examplePage{\textit{Karta B₄a.}}
% Karta Pppbsa: przerywnik 8.
%   \bigskip
%   \exampleLib{Biblioteka Jagiellońska. Kraków.}

%  \bigskip \exampleRef{\textit{Estreicher XXVI 180. Rostkowska 12.}}

% %  \bigskip

  \exampleDig{\url{https://www.wbc.poznan.pl/dlibra/publication/383976/},
    \url{https://www.dbc.wroc.pl/dlibra/publication/2204/}}.
% https://www.wbc.poznan.pl/dlibra/publication/383976/edition/298223
% https://www.dbc.wroc.pl/dlibra/publication/2204/edition/2248?language=en

 % Pisma 1—5. Rubryki a—qn. Qyfry 1—3. Przerywnik 8.
\bigskip

\fontID{Wi-03d}{78}

\fontstat{10}

% \exdisplay \bg \gla
 \exdisplay \bg \gla
% 1
{\PTglyph{5}{t78_l01g01.png}}
% 2
{\PTglyph{5}{t78_l01g02.png}}
% 3
{\PTglyph{5}{t78_l01g03.png}}
% 4
{\PTglyph{5}{t78_l01g04.png}}
% 5
{\PTglyph{5}{t78_l01g05.png}}
% 6
{\PTglyph{5}{t78_l01g06.png}}
% 7
{\PTglyph{5}{t78_l01g07.png}}
% 8
{\PTglyph{5}{t78_l01g08.png}}
% 9
{\PTglyph{5}{t78_l01g09.png}}
% 10
{\PTglyph{5}{t78_l01g10.png}}
//
%%% Local Variables:
%%% mode: latex
%%% TeX-engine: luatex
%%% TeX-master: shared
%%% End:

%//
%\glpismo%
 \glpismo
% 1
{\PTglyphid{Wi-03uu0101}}
% 2
{\PTglyphid{Wi-03uu0102}}
% 3
{\PTglyphid{Wi-03uu0103}}
% 4
{\PTglyphid{Wi-03uu0104}}
% 5
{\PTglyphid{Wi-03uu0105}}
% 6
{\PTglyphid{Wi-03uu0106}}
% 7
{\PTglyphid{Wi-03uu0107}}
% 8
{\PTglyphid{Wi-03uu0108}}
% 9
{\PTglyphid{Wi-03uu0109}}
% 10
{\PTglyphid{Wi-03uu0110}}
//
\endgl \xe
%%% Local Variables:
%%% mode: latex
%%% TeX-engine: luatex
%%% TeX-master: shared
%%% End:

% //
%\endgl \xe


 \newpage
 
%%%%%%%%%%%%%%%%%%%%%%%%%%%%%%%%%%%%%%%%%%%%%%%%%%%%%%%%%%%%%%%%%%%%%%%%%%%%%%%
% from meta.csv
% 79,Wirzbięta-04_PT09_469.djvu,Wirzbięta,04,09,469
% 
%%%%%%%%%%%%%%%%%%%%%%%%%%%%%%%%%%%%%%%%%%%%%%%%%%%%%%%%%%%%%%%%%%%%%%%%%%%%%%%

 
% from dsed4test:
% Wirzbięta-04_PT09_469_4dsed.txt:Note "4. Pismo tekstowe, szwabacha M⁸¹, Stopień 20 ww. == 88 mm. — Tabl. 457, 468—470. [469]"
% Wirzbięta-04_PT09_469_4dsed.txt:Note1 "Character set table prepared by Alodia Kawecka Gryczowa"


 \pismoPL{Maciej Wirzbięta 4. Pismo tekstowe, szwabacha M⁸¹, Stopień
   20 ww. = 88 mm. — Tabl. Tabl. 457, 468—470. [469 czwarty zestaw]}
  
 \pismoEN{Maciej Wirzbięta 3. Schwabacher text font M⁸¹.
Type size 20 lines =  88 mm. — Plates 457, 468—470. [469 fourth set]}

\plate{469[4]}{IX}{1974}

The plate prepared by Alodia Kawecka Gryczowa.\\
The font table prepared by Alodia Kawecka Gryczowa.

\bigskip

\exampleBib{IX:1}

\bigskip \exampleDesc{[MIKOŁAJ REJ]: Postylla. Kraków, Maciej Wirzbięta, [po 5 I] 1557. 2⁰.}
  
%   Brandenburgensis cum Hedvige Sigismundi regis Poloniae filia oratio.
%   Kraków, Florian Ungler, [po 7 IX] 1535. 8⁰.}

%  \medskip
%  \examplePage{\textit{Karta B₄a.}}
% Karta Pppbsa: przerywnik 8.
%   \bigskip
%   \exampleLib{Biblioteka Jagiellońska. Kraków.}

  \bigskip \exampleRef{\textit{Estreicher XXVI 180. Rostkowska 12.}}

% %  \bigskip

  \exampleDig{\url{https://www.wbc.poznan.pl/dlibra/publication/383976/},
    \url{https://www.dbc.wroc.pl/dlibra/publication/2204/}}.
% https://www.wbc.poznan.pl/dlibra/publication/383976/edition/298223
% https://www.dbc.wroc.pl/dlibra/publication/2204/edition/2248?language=en

 % Pisma 1—5. Rubryki a—qn. Qyfry 1—3. Przerywnik 8.
\bigskip

\fontID{Wi-04}{79}

\fontstat{129?}

% \exdisplay \bg \gla
 \exdisplay \bg \gla
% 1
{\PTglyph{5}{t79_l01g01.png}}
% 2
{\PTglyph{5}{t79_l01g02.png}}
% 3
{\PTglyph{5}{t79_l01g03.png}}
% 4
{\PTglyph{5}{t79_l01g04.png}}
% 5
{\PTglyph{5}{t79_l01g05.png}}
% 6
{\PTglyph{5}{t79_l01g06.png}}
% 7
{\PTglyph{5}{t79_l01g07.png}}
% 8
{\PTglyph{5}{t79_l01g08.png}}
% 9
{\PTglyph{5}{t79_l01g09.png}}
% 10
{\PTglyph{5}{t79_l01g10.png}}
% 11
{\PTglyph{5}{t79_l01g11.png}}
% 12
{\PTglyph{5}{t79_l01g12.png}}
% 13
{\PTglyph{5}{t79_l01g13.png}}
% 14
{\PTglyph{5}{t79_l01g14.png}}
% 15
{\PTglyph{5}{t79_l01g15.png}}
% 16
{\PTglyph{5}{t79_l01g16.png}}
% 17
{\PTglyph{5}{t79_l01g17.png}}
% 18
{\PTglyph{5}{t79_l01g18.png}}
% 19
{\PTglyph{5}{t79_l01g19.png}}
% 20
{\PTglyph{5}{t79_l01g20.png}}
% 21
{\PTglyph{5}{t79_l01g21.png}}
% 22
{\PTglyph{5}{t79_l01g22.png}}
% 23
{\PTglyph{5}{t79_l01g23.png}}
% 24
{\PTglyph{5}{t79_l01g24.png}}
% 25
{\PTglyph{5}{t79_l02g01.png}}
% 26
{\PTglyph{5}{t79_l02g02.png}}
% 27
{\PTglyph{5}{t79_l02g03.png}}
% 28
{\PTglyph{5}{t79_l02g04.png}}
% 29
{\PTglyph{5}{t79_l02g05.png}}
% 30
{\PTglyph{5}{t79_l02g06.png}}
% 31
{\PTglyph{5}{t79_l02g07.png}}
% 32
{\PTglyph{5}{t79_l02g08.png}}
% 33
{\PTglyph{5}{t79_l02g09.png}}
% 34
{\PTglyph{5}{t79_l02g10.png}}
% 35
{\PTglyph{5}{t79_l02g11.png}}
% 36
{\PTglyph{5}{t79_l02g12.png}}
% 37
{\PTglyph{5}{t79_l02g13.png}}
% 38
{\PTglyph{5}{t79_l02g14.png}}
% 39
{\PTglyph{5}{t79_l02g15.png}}
% 40
{\PTglyph{5}{t79_l02g16.png}}
% 41
{\PTglyph{5}{t79_l02g17.png}}
% 42
{\PTglyph{5}{t79_l02g18.png}}
% 43
{\PTglyph{5}{t79_l02g19.png}}
% 44
{\PTglyph{5}{t79_l02g20.png}}
% 45
{\PTglyph{5}{t79_l02g21.png}}
% 46
{\PTglyph{5}{t79_l02g22.png}}
% 47
{\PTglyph{5}{t79_l02g23.png}}
% 48
{\PTglyph{5}{t79_l02g24.png}}
% 49
{\PTglyph{5}{t79_l02g25.png}}
% 50
{\PTglyph{5}{t79_l02g26.png}}
% 51
{\PTglyph{5}{t79_l02g27.png}}
% 52
{\PTglyph{5}{t79_l02g28.png}}
% 53
{\PTglyph{5}{t79_l02g29.png}}
% 54
{\PTglyph{5}{t79_l02g30.png}}
% 55
{\PTglyph{5}{t79_l02g31.png}}
% 56
{\PTglyph{5}{t79_l02g32.png}}
% 57
{\PTglyph{5}{t79_l02g33.png}}
% 58
{\PTglyph{5}{t79_l02g34.png}}
% 59
{\PTglyph{5}{t79_l02g35.png}}
% 60
{\PTglyph{5}{t79_l02g36.png}}
% 61
{\PTglyph{5}{t79_l02g37.png}}
% 62
{\PTglyph{5}{t79_l02g38.png}}
% 63
{\PTglyph{5}{t79_l02g39.png}}
% 64
{\PTglyph{5}{t79_l02g40.png}}
% 65
{\PTglyph{5}{t79_l02g41.png}}
% 66
{\PTglyph{5}{t79_l03g01.png}}
% 67
{\PTglyph{5}{t79_l03g02.png}}
% 68
{\PTglyph{5}{t79_l03g03.png}}
% 69
{\PTglyph{5}{t79_l03g04.png}}
% 70
{\PTglyph{5}{t79_l03g05.png}}
% 71
{\PTglyph{5}{t79_l03g06.png}}
% 72
{\PTglyph{5}{t79_l03g07.png}}
% 73
{\PTglyph{5}{t79_l03g08.png}}
% 74
{\PTglyph{5}{t79_l03g09.png}}
% 75
{\PTglyph{5}{t79_l03g10.png}}
% 76
{\PTglyph{5}{t79_l03g11.png}}
% 77
{\PTglyph{5}{t79_l03g12.png}}
% 78
{\PTglyph{5}{t79_l03g13.png}}
% 79
{\PTglyph{5}{t79_l03g14.png}}
% 80
{\PTglyph{5}{t79_l03g15.png}}
% 81
{\PTglyph{5}{t79_l03g16.png}}
% 82
{\PTglyph{5}{t79_l03g17.png}}
% 83
{\PTglyph{5}{t79_l03g18.png}}
% 84
{\PTglyph{5}{t79_l03g19.png}}
% 85
{\PTglyph{5}{t79_l03g20.png}}
% 86
{\PTglyph{5}{t79_l03g21.png}}
% 87
{\PTglyph{5}{t79_l03g22.png}}
% 88
{\PTglyph{5}{t79_l03g23.png}}
% 89
{\PTglyph{5}{t79_l03g24.png}}
% 90
{\PTglyph{5}{t79_l03g25.png}}
% 91
{\PTglyph{5}{t79_l03g26.png}}
% 92
{\PTglyph{5}{t79_l03g27.png}}
% 93
{\PTglyph{5}{t79_l03g28.png}}
% 94
{\PTglyph{5}{t79_l03g29.png}}
% 95
{\PTglyph{5}{t79_l03g30.png}}
% 96
{\PTglyph{5}{t79_l03g31.png}}
% 97
{\PTglyph{5}{t79_l03g32.png}}
% 98
{\PTglyph{5}{t79_l03g33.png}}
% 99
{\PTglyph{5}{t79_l03g34.png}}
% 100
{\PTglyph{5}{t79_l03g35.png}}
//
%%% Local Variables:
%%% mode: latex
%%% TeX-engine: luatex
%%% TeX-master: shared
%%% End:

%//
%\glpismo%
 \glpismo
% 1
{\PTglyphid{Wi-04_0101}}
% 2
{\PTglyphid{Wi-04_0102}}
% 3
{\PTglyphid{Wi-04_0103}}
% 4
{\PTglyphid{Wi-04_0104}}
% 5
{\PTglyphid{Wi-04_0105}}
% 6
{\PTglyphid{Wi-04_0106}}
% 7
{\PTglyphid{Wi-04_0107}}
% 8
{\PTglyphid{Wi-04_0108}}
% 9
{\PTglyphid{Wi-04_0109}}
% 10
{\PTglyphid{Wi-04_0110}}
% 11
{\PTglyphid{Wi-04_0111}}
% 12
{\PTglyphid{Wi-04_0112}}
% 13
{\PTglyphid{Wi-04_0113}}
% 14
{\PTglyphid{Wi-04_0114}}
% 15
{\PTglyphid{Wi-04_0115}}
% 16
{\PTglyphid{Wi-04_0116}}
% 17
{\PTglyphid{Wi-04_0117}}
% 18
{\PTglyphid{Wi-04_0118}}
% 19
{\PTglyphid{Wi-04_0119}}
% 20
{\PTglyphid{Wi-04_0120}}
% 21
{\PTglyphid{Wi-04_0121}}
% 22
{\PTglyphid{Wi-04_0122}}
% 23
{\PTglyphid{Wi-04_0123}}
% 24
{\PTglyphid{Wi-04_0124}}
% 25
{\PTglyphid{Wi-04_0201}}
% 26
{\PTglyphid{Wi-04_0202}}
% 27
{\PTglyphid{Wi-04_0203}}
% 28
{\PTglyphid{Wi-04_0204}}
% 29
{\PTglyphid{Wi-04_0205}}
% 30
{\PTglyphid{Wi-04_0206}}
% 31
{\PTglyphid{Wi-04_0207}}
% 32
{\PTglyphid{Wi-04_0208}}
% 33
{\PTglyphid{Wi-04_0209}}
% 34
{\PTglyphid{Wi-04_0210}}
% 35
{\PTglyphid{Wi-04_0211}}
% 36
{\PTglyphid{Wi-04_0212}}
% 37
{\PTglyphid{Wi-04_0213}}
% 38
{\PTglyphid{Wi-04_0214}}
% 39
{\PTglyphid{Wi-04_0215}}
% 40
{\PTglyphid{Wi-04_0216}}
% 41
{\PTglyphid{Wi-04_0217}}
% 42
{\PTglyphid{Wi-04_0218}}
% 43
{\PTglyphid{Wi-04_0219}}
% 44
{\PTglyphid{Wi-04_0220}}
% 45
{\PTglyphid{Wi-04_0221}}
% 46
{\PTglyphid{Wi-04_0222}}
% 47
{\PTglyphid{Wi-04_0223}}
% 48
{\PTglyphid{Wi-04_0224}}
% 49
{\PTglyphid{Wi-04_0225}}
% 50
{\PTglyphid{Wi-04_0226}}
% 51
{\PTglyphid{Wi-04_0227}}
% 52
{\PTglyphid{Wi-04_0228}}
% 53
{\PTglyphid{Wi-04_0229}}
% 54
{\PTglyphid{Wi-04_0230}}
% 55
{\PTglyphid{Wi-04_0231}}
% 56
{\PTglyphid{Wi-04_0232}}
% 57
{\PTglyphid{Wi-04_0233}}
% 58
{\PTglyphid{Wi-04_0234}}
% 59
{\PTglyphid{Wi-04_0235}}
% 60
{\PTglyphid{Wi-04_0236}}
% 61
{\PTglyphid{Wi-04_0237}}
% 62
{\PTglyphid{Wi-04_0238}}
% 63
{\PTglyphid{Wi-04_0239}}
% 64
{\PTglyphid{Wi-04_0240}}
% 65
{\PTglyphid{Wi-04_0241}}
% 66
{\PTglyphid{Wi-04_0301}}
% 67
{\PTglyphid{Wi-04_0302}}
% 68
{\PTglyphid{Wi-04_0303}}
% 69
{\PTglyphid{Wi-04_0304}}
% 70
{\PTglyphid{Wi-04_0305}}
% 71
{\PTglyphid{Wi-04_0306}}
% 72
{\PTglyphid{Wi-04_0307}}
% 73
{\PTglyphid{Wi-04_0308}}
% 74
{\PTglyphid{Wi-04_0309}}
% 75
{\PTglyphid{Wi-04_0310}}
% 76
{\PTglyphid{Wi-04_0311}}
% 77
{\PTglyphid{Wi-04_0312}}
% 78
{\PTglyphid{Wi-04_0313}}
% 79
{\PTglyphid{Wi-04_0314}}
% 80
{\PTglyphid{Wi-04_0315}}
% 81
{\PTglyphid{Wi-04_0316}}
% 82
{\PTglyphid{Wi-04_0317}}
% 83
{\PTglyphid{Wi-04_0318}}
% 84
{\PTglyphid{Wi-04_0319}}
% 85
{\PTglyphid{Wi-04_0320}}
% 86
{\PTglyphid{Wi-04_0321}}
% 87
{\PTglyphid{Wi-04_0322}}
% 88
{\PTglyphid{Wi-04_0323}}
% 89
{\PTglyphid{Wi-04_0324}}
% 90
{\PTglyphid{Wi-04_0325}}
% 91
{\PTglyphid{Wi-04_0326}}
% 92
{\PTglyphid{Wi-04_0327}}
% 93
{\PTglyphid{Wi-04_0328}}
% 94
{\PTglyphid{Wi-04_0329}}
% 95
{\PTglyphid{Wi-04_0330}}
% 96
{\PTglyphid{Wi-04_0331}}
% 97
{\PTglyphid{Wi-04_0332}}
% 98
{\PTglyphid{Wi-04_0333}}
% 99
{\PTglyphid{Wi-04_0334}}
% 100
{\PTglyphid{Wi-04_0335}}
//
\endgl \xe
%%% Local Variables:
%%% mode: latex
%%% TeX-engine: luatex
%%% TeX-master: shared
%%% End:

% //
%\endgl \xe
 
 \newpage
 
%%%%%%%%%%%%%%%%%%%%%%%%%%%%%%%%%%%%%%%%%%%%%%%%%%%%%%%%%%%%%%%%%%%%%%%%%%%%%%%
% from meta.csv
% 79,Wirzbięta-04_PT09_469.djvu,Wirzbięta,04,09,469
% 
%%%%%%%%%%%%%%%%%%%%%%%%%%%%%%%%%%%%%%%%%%%%%%%%%%%%%%%%%%%%%%%%%%%%%%%%%%%%%%%

 
% from dsed4test:
% Wirzbięta-04_PT09_469_4dsed.txt:Note "4. Pismo tekstowe, szwabacha M⁸¹, Stopień 20 ww. == 88 mm. — Tabl. 457, 468—470. [469]"
% Wirzbięta-04_PT09_469_4dsed.txt:Note1 "Character set table prepared by Alodia Kawecka Gryczowa"


 \pismoPL{Maciej Wirzbięta 4. Pismo tekstowe, szwabacha M⁸¹, Stopień
   20 ww. = 88 mm. — Tabl. Tabl. 457, 468—470. [469 czwarty zestaw]}
  
 \pismoEN{Maciej Wirzbięta 3. Schwabacher text font M⁸¹.
Type size 20 lines =  88 mm. — Plates 457, 468—470. [469 fourth set]}

\plate{469[4]}{IX}{1974}

The plate prepared by Alodia Kawecka Gryczowa.\\
The font table prepared by Alodia Kawecka Gryczowa.

\bigskip

\exampleBib{IX:1}

\bigskip \exampleDesc{[MIKOŁAJ REJ]: Postylla. Kraków, Maciej Wirzbięta, [po 5 I] 1557. 2⁰.}
  
%   Brandenburgensis cum Hedvige Sigismundi regis Poloniae filia oratio.
%   Kraków, Florian Ungler, [po 7 IX] 1535. 8⁰.}

%  \medskip
%  \examplePage{\textit{Karta B₄a.}}
% Karta Pppbsa: przerywnik 8.
%   \bigskip
%   \exampleLib{Biblioteka Jagiellońska. Kraków.}

  \bigskip \exampleRef{\textit{Estreicher XXVI 180. Rostkowska 12.}}

% %  \bigskip

  \exampleDig{\url{https://www.wbc.poznan.pl/dlibra/publication/383976/},
    \url{https://www.dbc.wroc.pl/dlibra/publication/2204/}}.
% https://www.wbc.poznan.pl/dlibra/publication/383976/edition/298223
% https://www.dbc.wroc.pl/dlibra/publication/2204/edition/2248?language=en

 % Pisma 1—5. Rubryki a—qn. Qyfry 1—3. Przerywnik 8.
\bigskip

\fontID{Wi-04u}{80}

\fontstat{5}

% \exdisplay \bg \gla
 \exdisplay \bg \gla
% 1
{\PTglyph{5}{t80_l01g01.png}}
% 2
{\PTglyph{5}{t80_l01g02.png}}
% 3
{\PTglyph{5}{t80_l01g03.png}}
% 4
{\PTglyph{5}{t80_l01g04.png}}
% 5
{\PTglyph{5}{t80_l01g05.png}}
//
%%% Local Variables:
%%% mode: latex
%%% TeX-engine: luatex
%%% TeX-master: shared
%%% End:

%//
%\glpismo%
 \glpismo
% 1
{\PTglyphid{Wi-04u0101}}
% 2
{\PTglyphid{Wi-04u0102}}
% 3
{\PTglyphid{Wi-04u0103}}
% 4
{\PTglyphid{Wi-04u0104}}
% 5
{\PTglyphid{Wi-04u0105}}
//
\endgl \xe
%%% Local Variables:
%%% mode: latex
%%% TeX-engine: luatex
%%% TeX-master: shared
%%% End:

% //
%\endgl \xe
 

 \newpage
 
%%%%%%%%%%%%%%%%%%%%%%%%%%%%%%%%%%%%%%%%%%%%%%%%%%%%%%%%%%%%%%%%%%%%%%%%%%%%%%%
% from meta.csv
% 81,Wirzbięta-05_PT09_469.djvu,Wirzbięta,05,09,469
% 
%%%%%%%%%%%%%%%%%%%%%%%%%%%%%%%%%%%%%%%%%%%%%%%%%%%%%%%%%%%%%%%%%%%%%%%%%%%%%%%

 
% from dsed4test:
% Wirzbięta-05_PT09_469_4dsed.txt:Note "5. Pismo komentarzowe, szwabacha M⁸¹. Stopień 20 ww. = 71/2 mm. — Tabl. 456, 457, 469, 470. [469]"
% Wirzbięta-05_PT09_469_4dsed.txt:Note1 "Character set table prepared by Alodia Kawecka Gryczowa"


 \pismoPL{Maciej Wirzbięta 5. Pismo tekstowe, szwabacha M⁸¹, Stopień
   20 ww. = 71/2 mm. — Tabl. 456, 457, 469, 470. [469 piąty zestaw]}
  
 \pismoEN{Maciej Wirzbięta 5. Schwabacher text font M⁸¹.
Type size 20 lines =  71/2 mm. — Plates 456, 457, 469, 470. [469 fifth set]}

\plate{469[5]}{IX}{1974}

The plate prepared by Alodia Kawecka Gryczowa.\\
The font table prepared by Alodia Kawecka Gryczowa.

\bigskip

\exampleBib{IX:1}

\bigskip \exampleDesc{[MIKOŁAJ REJ]: Postylla. Kraków, Maciej Wirzbięta, [po 5 I] 1557. 2⁰.}
  
%   Brandenburgensis cum Hedvige Sigismundi regis Poloniae filia oratio.
%   Kraków, Florian Ungler, [po 7 IX] 1535. 8⁰.}

%  \medskip
%  \examplePage{\textit{Karta B₄a.}}
% Karta Pppbsa: przerywnik 8.
%   \bigskip
%   \exampleLib{Biblioteka Jagiellońska. Kraków.}

  \bigskip \exampleRef{\textit{Estreicher XXVI 180. Rostkowska 12.}}

% %  \bigskip

  \exampleDig{\url{https://www.wbc.poznan.pl/dlibra/publication/383976/},
    \url{https://www.dbc.wroc.pl/dlibra/publication/2204/}}.
% https://www.wbc.poznan.pl/dlibra/publication/383976/edition/298223
% https://www.dbc.wroc.pl/dlibra/publication/2204/edition/2248?language=en

 % Pisma 1—5. Rubryki a—qn. Qyfry 1—3. Przerywnik 8.
\bigskip

\fontID{Wi-05}{80}

\fontstat{129?}

% \exdisplay \bg \gla
 \exdisplay \bg \gla
% 1
{\PTglyph{5}{t81_l01g01.png}}
% 2
{\PTglyph{5}{t81_l01g02.png}}
% 3
{\PTglyph{5}{t81_l01g03.png}}
% 4
{\PTglyph{5}{t81_l01g04.png}}
% 5
{\PTglyph{5}{t81_l01g05.png}}
% 6
{\PTglyph{5}{t81_l01g06.png}}
% 7
{\PTglyph{5}{t81_l01g07.png}}
% 8
{\PTglyph{5}{t81_l01g08.png}}
% 9
{\PTglyph{5}{t81_l01g09.png}}
% 10
{\PTglyph{5}{t81_l01g10.png}}
% 11
{\PTglyph{5}{t81_l01g11.png}}
% 12
{\PTglyph{5}{t81_l01g12.png}}
% 13
{\PTglyph{5}{t81_l01g13.png}}
% 14
{\PTglyph{5}{t81_l01g14.png}}
% 15
{\PTglyph{5}{t81_l01g15.png}}
% 16
{\PTglyph{5}{t81_l01g16.png}}
% 17
{\PTglyph{5}{t81_l01g17.png}}
% 18
{\PTglyph{5}{t81_l01g18.png}}
% 19
{\PTglyph{5}{t81_l01g19.png}}
% 20
{\PTglyph{5}{t81_l01g20.png}}
% 21
{\PTglyph{5}{t81_l01g21.png}}
% 22
{\PTglyph{5}{t81_l01g22.png}}
% 23
{\PTglyph{5}{t81_l01g23.png}}
% 24
{\PTglyph{5}{t81_l02g01.png}}
% 25
{\PTglyph{5}{t81_l02g02.png}}
% 26
{\PTglyph{5}{t81_l02g03.png}}
% 27
{\PTglyph{5}{t81_l02g04.png}}
% 28
{\PTglyph{5}{t81_l02g05.png}}
% 29
{\PTglyph{5}{t81_l02g06.png}}
% 30
{\PTglyph{5}{t81_l02g07.png}}
% 31
{\PTglyph{5}{t81_l02g08.png}}
% 32
{\PTglyph{5}{t81_l02g09.png}}
% 33
{\PTglyph{5}{t81_l02g10.png}}
% 34
{\PTglyph{5}{t81_l02g11.png}}
% 35
{\PTglyph{5}{t81_l02g12.png}}
% 36
{\PTglyph{5}{t81_l02g13.png}}
% 37
{\PTglyph{5}{t81_l02g14.png}}
% 38
{\PTglyph{5}{t81_l02g15.png}}
% 39
{\PTglyph{5}{t81_l02g16.png}}
% 40
{\PTglyph{5}{t81_l02g17.png}}
% 41
{\PTglyph{5}{t81_l02g18.png}}
% 42
{\PTglyph{5}{t81_l02g19.png}}
% 43
{\PTglyph{5}{t81_l02g20.png}}
% 44
{\PTglyph{5}{t81_l02g21.png}}
% 45
{\PTglyph{5}{t81_l02g22.png}}
% 46
{\PTglyph{5}{t81_l02g23.png}}
% 47
{\PTglyph{5}{t81_l02g24.png}}
% 48
{\PTglyph{5}{t81_l02g25.png}}
% 49
{\PTglyph{5}{t81_l02g26.png}}
% 50
{\PTglyph{5}{t81_l02g27.png}}
% 51
{\PTglyph{5}{t81_l02g28.png}}
% 52
{\PTglyph{5}{t81_l02g29.png}}
% 53
{\PTglyph{5}{t81_l02g30.png}}
% 54
{\PTglyph{5}{t81_l02g31.png}}
% 55
{\PTglyph{5}{t81_l02g32.png}}
% 56
{\PTglyph{5}{t81_l02g33.png}}
% 57
{\PTglyph{5}{t81_l02g34.png}}
% 58
{\PTglyph{5}{t81_l02g35.png}}
% 59
{\PTglyph{5}{t81_l03g01.png}}
% 60
{\PTglyph{5}{t81_l03g02.png}}
% 61
{\PTglyph{5}{t81_l03g03.png}}
% 62
{\PTglyph{5}{t81_l03g04.png}}
% 63
{\PTglyph{5}{t81_l03g05.png}}
% 64
{\PTglyph{5}{t81_l03g06.png}}
% 65
{\PTglyph{5}{t81_l03g07.png}}
% 66
{\PTglyph{5}{t81_l03g08.png}}
% 67
{\PTglyph{5}{t81_l03g09.png}}
% 68
{\PTglyph{5}{t81_l03g10.png}}
% 69
{\PTglyph{5}{t81_l03g11.png}}
% 70
{\PTglyph{5}{t81_l03g12.png}}
% 71
{\PTglyph{5}{t81_l03g13.png}}
% 72
{\PTglyph{5}{t81_l03g14.png}}
% 73
{\PTglyph{5}{t81_l03g15.png}}
% 74
{\PTglyph{5}{t81_l03g16.png}}
% 75
{\PTglyph{5}{t81_l03g17.png}}
% 76
{\PTglyph{5}{t81_l03g18.png}}
% 77
{\PTglyph{5}{t81_l03g19.png}}
% 78
{\PTglyph{5}{t81_l03g20.png}}
% 79
{\PTglyph{5}{t81_l03g21.png}}
% 80
{\PTglyph{5}{t81_l03g22.png}}
//
%%% Local Variables:
%%% mode: latex
%%% TeX-engine: luatex
%%% TeX-master: shared
%%% End:

%//
%\glpismo%
 \glpismo
% 1
{\PTglyphid{Wi-05_0101}}
% 2
{\PTglyphid{Wi-05_0102}}
% 3
{\PTglyphid{Wi-05_0103}}
% 4
{\PTglyphid{Wi-05_0104}}
% 5
{\PTglyphid{Wi-05_0105}}
% 6
{\PTglyphid{Wi-05_0106}}
% 7
{\PTglyphid{Wi-05_0107}}
% 8
{\PTglyphid{Wi-05_0108}}
% 9
{\PTglyphid{Wi-05_0109}}
% 10
{\PTglyphid{Wi-05_0110}}
% 11
{\PTglyphid{Wi-05_0111}}
% 12
{\PTglyphid{Wi-05_0112}}
% 13
{\PTglyphid{Wi-05_0113}}
% 14
{\PTglyphid{Wi-05_0114}}
% 15
{\PTglyphid{Wi-05_0115}}
% 16
{\PTglyphid{Wi-05_0116}}
% 17
{\PTglyphid{Wi-05_0117}}
% 18
{\PTglyphid{Wi-05_0118}}
% 19
{\PTglyphid{Wi-05_0119}}
% 20
{\PTglyphid{Wi-05_0120}}
% 21
{\PTglyphid{Wi-05_0121}}
% 22
{\PTglyphid{Wi-05_0122}}
% 23
{\PTglyphid{Wi-05_0123}}
% 24
{\PTglyphid{Wi-05_0201}}
% 25
{\PTglyphid{Wi-05_0202}}
% 26
{\PTglyphid{Wi-05_0203}}
% 27
{\PTglyphid{Wi-05_0204}}
% 28
{\PTglyphid{Wi-05_0205}}
% 29
{\PTglyphid{Wi-05_0206}}
% 30
{\PTglyphid{Wi-05_0207}}
% 31
{\PTglyphid{Wi-05_0208}}
% 32
{\PTglyphid{Wi-05_0209}}
% 33
{\PTglyphid{Wi-05_0210}}
% 34
{\PTglyphid{Wi-05_0211}}
% 35
{\PTglyphid{Wi-05_0212}}
% 36
{\PTglyphid{Wi-05_0213}}
% 37
{\PTglyphid{Wi-05_0214}}
% 38
{\PTglyphid{Wi-05_0215}}
% 39
{\PTglyphid{Wi-05_0216}}
% 40
{\PTglyphid{Wi-05_0217}}
% 41
{\PTglyphid{Wi-05_0218}}
% 42
{\PTglyphid{Wi-05_0219}}
% 43
{\PTglyphid{Wi-05_0220}}
% 44
{\PTglyphid{Wi-05_0221}}
% 45
{\PTglyphid{Wi-05_0222}}
% 46
{\PTglyphid{Wi-05_0223}}
% 47
{\PTglyphid{Wi-05_0224}}
% 48
{\PTglyphid{Wi-05_0225}}
% 49
{\PTglyphid{Wi-05_0226}}
% 50
{\PTglyphid{Wi-05_0227}}
% 51
{\PTglyphid{Wi-05_0228}}
% 52
{\PTglyphid{Wi-05_0229}}
% 53
{\PTglyphid{Wi-05_0230}}
% 54
{\PTglyphid{Wi-05_0231}}
% 55
{\PTglyphid{Wi-05_0232}}
% 56
{\PTglyphid{Wi-05_0233}}
% 57
{\PTglyphid{Wi-05_0234}}
% 58
{\PTglyphid{Wi-05_0235}}
% 59
{\PTglyphid{Wi-05_0301}}
% 60
{\PTglyphid{Wi-05_0302}}
% 61
{\PTglyphid{Wi-05_0303}}
% 62
{\PTglyphid{Wi-05_0304}}
% 63
{\PTglyphid{Wi-05_0305}}
% 64
{\PTglyphid{Wi-05_0306}}
% 65
{\PTglyphid{Wi-05_0307}}
% 66
{\PTglyphid{Wi-05_0308}}
% 67
{\PTglyphid{Wi-05_0309}}
% 68
{\PTglyphid{Wi-05_0310}}
% 69
{\PTglyphid{Wi-05_0311}}
% 70
{\PTglyphid{Wi-05_0312}}
% 71
{\PTglyphid{Wi-05_0313}}
% 72
{\PTglyphid{Wi-05_0314}}
% 73
{\PTglyphid{Wi-05_0315}}
% 74
{\PTglyphid{Wi-05_0316}}
% 75
{\PTglyphid{Wi-05_0317}}
% 76
{\PTglyphid{Wi-05_0318}}
% 77
{\PTglyphid{Wi-05_0319}}
% 78
{\PTglyphid{Wi-05_0320}}
% 79
{\PTglyphid{Wi-05_0321}}
% 80
{\PTglyphid{Wi-05_0322}}
//
\endgl \xe
%%% Local Variables:
%%% mode: latex
%%% TeX-engine: luatex
%%% TeX-master: shared
%%% End:

% //
%\endgl \xe
 

 \newpage
 
%%%%%%%%%%%%%%%%%%%%%%%%%%%%%%%%%%%%%%%%%%%%%%%%%%%%%%%%%%%%%%%%%%%%%%%%%%%%%%%
% from meta.csv
% 82,Wirzbięta-05u_PT11_570.djvu,Wirzbięta,05u,11,570
% 
%%%%%%%%%%%%%%%%%%%%%%%%%%%%%%%%%%%%%%%%%%%%%%%%%%%%%%%%%%%%%%%%%%%%%%%%%%%%%%%

 
% from dsed4test:
% Wirzbięta-05u_PT11_570_4dsed.txt:Note "5. Uzupełnienie. [570]"
% Wirzbięta-05u_PT11_570_4dsed.txt:Note1 "Character set table prepared by Alodia Kawecka Gryczowa and Anna Wolińska"

 \pismoPL{Maciej Wirzbięta 5. Uzupełnienie. [Tabl. 570]}
  
 \pismoEN{Maciej Wirzbięta 5. A supplement. [Plate 570]}

\plate{570[5]}{XI}{1981}

The plate prepared by Alodia Kawecka Gryczowa.\\
The font table prepared by Alodia Kawecka Gryczowa and Anna Wrońska.

\bigskip

% \exampleBib{IX:1}

% \bigskip \exampleDesc{[MIKOŁAJ REJ]: Postylla. Kraków, Maciej Wirzbięta, [po 5 I] 1557. 2⁰.}
  
%   Brandenburgensis cum Hedvige Sigismundi regis Poloniae filia oratio.
%   Kraków, Florian Ungler, [po 7 IX] 1535. 8⁰.}

%  \medskip
%  \examplePage{\textit{Karta B₄a.}}
% Karta Pppbsa: przerywnik 8.
%   \bigskip
%   \exampleLib{Biblioteka Jagiellońska. Kraków.}

%  \bigskip \exampleRef{\textit{Estreicher XXVI 180. Rostkowska 12.}}

% %  \bigskip

  \exampleDig{\url{https://www.wbc.poznan.pl/dlibra/publication/383976/},
    \url{https://www.dbc.wroc.pl/dlibra/publication/2204/}}.
% https://www.wbc.poznan.pl/dlibra/publication/383976/edition/298223
% https://www.dbc.wroc.pl/dlibra/publication/2204/edition/2248?language=en

 % Pisma 1—5. Rubryki a—qn. Qyfry 1—3. Przerywnik 8.
\bigskip

\fontID{Wi-05u}{82}

\fontstat{10}

% \exdisplay \bg \gla
 \exdisplay \bg \gla
% 1
{\PTglyph{5}{t82_l01g01.png}}
% 2
{\PTglyph{5}{t82_l01g02.png}}
% 3
{\PTglyph{5}{t82_l01g03.png}}
% 4
{\PTglyph{5}{t82_l01g04.png}}
% 5
{\PTglyph{5}{t82_l01g05.png}}
% 6
{\PTglyph{5}{t82_l01g06.png}}
% 7
{\PTglyph{5}{t82_l01g07.png}}
% 8
{\PTglyph{5}{t82_l01g08.png}}
% 9
{\PTglyph{5}{t82_l01g09.png}}
% 10
{\PTglyph{5}{t82_l01g10.png}}
//
%%% Local Variables:
%%% mode: latex
%%% TeX-engine: luatex
%%% TeX-master: shared
%%% End:

%//
%\glpismo%
 \glpismo
% 1
{\PTglyphid{Wi-05u0101}}
% 2
{\PTglyphid{Wi-05u0102}}
% 3
{\PTglyphid{Wi-05u0103}}
% 4
{\PTglyphid{Wi-05u0104}}
% 5
{\PTglyphid{Wi-05u0105}}
% 6
{\PTglyphid{Wi-05u0106}}
% 7
{\PTglyphid{Wi-05u0107}}
% 8
{\PTglyphid{Wi-05u0108}}
% 9
{\PTglyphid{Wi-05u0109}}
% 10
{\PTglyphid{Wi-05u0110}}
//
\endgl \xe
%%% Local Variables:
%%% mode: latex
%%% TeX-engine: luatex
%%% TeX-master: shared
%%% End:

% //
%\endgl \xe

 \newpage
 
%%%%%%%%%%%%%%%%%%%%%%%%%%%%%%%%%%%%%%%%%%%%%%%%%%%%%%%%%%%%%%%%%%%%%%%%%%%%%%%
% from meta.csv
% 83,Wirzbięta-09_PT11_562.djvu,Wirzbięta,09,11,562
% 
%%%%%%%%%%%%%%%%%%%%%%%%%%%%%%%%%%%%%%%%%%%%%%%%%%%%%%%%%%%%%%%%%%%%%%%%%%%%%%%

 
% from dsed4test:
% Wirzbięta-09_PT11_562_4dsed.txt:Note "9. Pismo tekstowe, antykwa. Stopień 20 ww. 92 mm. — Tabl. 458, 514, 515, 562, 568. [562]"
% Wirzbięta-09_PT11_562_4dsed.txt:Note1 "Character set table prepared by Anna Wolińska"


 \pismoPL{Maciej Wirzbięta 9. Pismo tekstowe, antykwa. Stopień 20 ww. 92 mm. — Tabl. 458, 514, 515, 562, 568.}
  
 \pismoEN{Maciej Wirzbięta 5. Roman text font.
Type size 20 lines =  92 mm. — Plates 458, 514, 515, 562, 568.}

\plate{562}{XI}{1981}

The plate prepared by Alodia Kawecka Gryczowa.\\
The font table prepared by Alodia Kawecka Gryczowa and Anna Wolińska.

\bigskip

\exampleBib{IX:77}

\bigskip \exampleDesc{ANDRZEJ WOLAN: De libertate politica sive
  civii. — Acc. Augustinus Rotundus [Mieleski]: [Epistola] Generoso
  Domino Andreae Volano... Dat. 10 XII 1571  etc. Kraków, Maciej
  Wirzbięta, [po 5 I| 1572. 4⁰.}


  \medskip
  \examplePage{\textit{Karta O₄b.}}

  \bigskip \exampleRef{\textit{Estreicher XXAIII 245.}}

  \bigskip

  \exampleDig{\url{https://www.dbc.wroc.pl/dlibra/doccontent?id=7489} page  ???}

    %https://wbc.poznan.pl/dlibra/publication/320603/edition/263343 brak!?

% https://neriton.pl/produkt/de-libertate-politica-sive-civili-stanislaw-dubingowicz-o-wolnosci-rzeczypospolitej-slacheckiej-wyd-m-eder-r-mazurkiewicz-red-j-axer/
    

\bigskip

\fontID{Wi-09}{83}

\fontstat{126}

% \exdisplay \bg \gla
 \exdisplay \bg \gla
% 1
{\PTglyph{5}{t83_l01g01.png}}
% 2
{\PTglyph{5}{t83_l01g02.png}}
% 3
{\PTglyph{5}{t83_l01g03.png}}
% 4
{\PTglyph{5}{t83_l01g04.png}}
% 5
{\PTglyph{5}{t83_l01g05.png}}
% 6
{\PTglyph{5}{t83_l01g06.png}}
% 7
{\PTglyph{5}{t83_l01g07.png}}
% 8
{\PTglyph{5}{t83_l01g08.png}}
% 9
{\PTglyph{5}{t83_l01g09.png}}
% 10
{\PTglyph{5}{t83_l01g10.png}}
% 11
{\PTglyph{5}{t83_l01g11.png}}
% 12
{\PTglyph{5}{t83_l01g12.png}}
% 13
{\PTglyph{5}{t83_l01g13.png}}
% 14
{\PTglyph{5}{t83_l01g14.png}}
% 15
{\PTglyph{5}{t83_l01g15.png}}
% 16
{\PTglyph{5}{t83_l01g16.png}}
% 17
{\PTglyph{5}{t83_l01g17.png}}
% 18
{\PTglyph{5}{t83_l01g18.png}}
% 19
{\PTglyph{5}{t83_l01g19.png}}
% 20
{\PTglyph{5}{t83_l01g20.png}}
% 21
{\PTglyph{5}{t83_l01g21.png}}
% 22
{\PTglyph{5}{t83_l01g22.png}}
% 23
{\PTglyph{5}{t83_l01g23.png}}
% 24
{\PTglyph{5}{t83_l01g24.png}}
% 25
{\PTglyph{5}{t83_l02g01.png}}
% 26
{\PTglyph{5}{t83_l02g02.png}}
% 27
{\PTglyph{5}{t83_l02g03.png}}
% 28
{\PTglyph{5}{t83_l02g04.png}}
% 29
{\PTglyph{5}{t83_l02g05.png}}
% 30
{\PTglyph{5}{t83_l02g06.png}}
% 31
{\PTglyph{5}{t83_l02g07.png}}
% 32
{\PTglyph{5}{t83_l02g08.png}}
% 33
{\PTglyph{5}{t83_l02g09.png}}
% 34
{\PTglyph{5}{t83_l02g10.png}}
% 35
{\PTglyph{5}{t83_l02g11.png}}
% 36
{\PTglyph{5}{t83_l02g12.png}}
% 37
{\PTglyph{5}{t83_l02g13.png}}
% 38
{\PTglyph{5}{t83_l02g14.png}}
% 39
{\PTglyph{5}{t83_l02g15.png}}
% 40
{\PTglyph{5}{t83_l02g16.png}}
% 41
{\PTglyph{5}{t83_l02g17.png}}
% 42
{\PTglyph{5}{t83_l02g18.png}}
% 43
{\PTglyph{5}{t83_l02g19.png}}
% 44
{\PTglyph{5}{t83_l02g20.png}}
% 45
{\PTglyph{5}{t83_l02g21.png}}
% 46
{\PTglyph{5}{t83_l02g22.png}}
% 47
{\PTglyph{5}{t83_l02g23.png}}
% 48
{\PTglyph{5}{t83_l02g24.png}}
% 49
{\PTglyph{5}{t83_l02g25.png}}
% 50
{\PTglyph{5}{t83_l02g26.png}}
% 51
{\PTglyph{5}{t83_l02g27.png}}
% 52
{\PTglyph{5}{t83_l02g28.png}}
% 53
{\PTglyph{5}{t83_l02g29.png}}
% 54
{\PTglyph{5}{t83_l02g30.png}}
% 55
{\PTglyph{5}{t83_l02g31.png}}
% 56
{\PTglyph{5}{t83_l02g32.png}}
% 57
{\PTglyph{5}{t83_l02g33.png}}
% 58
{\PTglyph{5}{t83_l02g34.png}}
% 59
{\PTglyph{5}{t83_l02g35.png}}
% 60
{\PTglyph{5}{t83_l02g36.png}}
% 61
{\PTglyph{5}{t83_l02g37.png}}
% 62
{\PTglyph{5}{t83_l02g38.png}}
% 63
{\PTglyph{5}{t83_l03g01.png}}
% 64
{\PTglyph{5}{t83_l03g02.png}}
% 65
{\PTglyph{5}{t83_l03g03.png}}
% 66
{\PTglyph{5}{t83_l03g04.png}}
% 67
{\PTglyph{5}{t83_l03g05.png}}
% 68
{\PTglyph{5}{t83_l03g06.png}}
% 69
{\PTglyph{5}{t83_l03g07.png}}
% 70
{\PTglyph{5}{t83_l03g08.png}}
% 71
{\PTglyph{5}{t83_l03g09.png}}
% 72
{\PTglyph{5}{t83_l03g10.png}}
% 73
{\PTglyph{5}{t83_l03g11.png}}
% 74
{\PTglyph{5}{t83_l03g12.png}}
% 75
{\PTglyph{5}{t83_l03g13.png}}
% 76
{\PTglyph{5}{t83_l03g14.png}}
% 77
{\PTglyph{5}{t83_l03g15.png}}
% 78
{\PTglyph{5}{t83_l03g16.png}}
% 79
{\PTglyph{5}{t83_l03g17.png}}
% 80
{\PTglyph{5}{t83_l03g18.png}}
% 81
{\PTglyph{5}{t83_l03g19.png}}
% 82
{\PTglyph{5}{t83_l03g20.png}}
% 83
{\PTglyph{5}{t83_l03g21.png}}
% 84
{\PTglyph{5}{t83_l03g22.png}}
% 85
{\PTglyph{5}{t83_l03g23.png}}
% 86
{\PTglyph{5}{t83_l03g24.png}}
% 87
{\PTglyph{5}{t83_l03g25.png}}
% 88
{\PTglyph{5}{t83_l03g26.png}}
% 89
{\PTglyph{5}{t83_l03g27.png}}
% 90
{\PTglyph{5}{t83_l03g28.png}}
% 91
{\PTglyph{5}{t83_l03g29.png}}
% 92
{\PTglyph{5}{t83_l03g30.png}}
% 93
{\PTglyph{5}{t83_l03g31.png}}
% 94
{\PTglyph{5}{t83_l03g32.png}}
% 95
{\PTglyph{5}{t83_l03g33.png}}
% 96
{\PTglyph{5}{t83_l04g01.png}}
% 97
{\PTglyph{5}{t83_l04g02.png}}
% 98
{\PTglyph{5}{t83_l04g03.png}}
% 99
{\PTglyph{5}{t83_l04g04.png}}
% 100
{\PTglyph{5}{t83_l04g05.png}}
% 101
{\PTglyph{5}{t83_l04g06.png}}
% 102
{\PTglyph{5}{t83_l04g07.png}}
% 103
{\PTglyph{5}{t83_l04g08.png}}
% 104
{\PTglyph{5}{t83_l04g09.png}}
% 105
{\PTglyph{5}{t83_l04g10.png}}
% 106
{\PTglyph{5}{t83_l04g11.png}}
% 107
{\PTglyph{5}{t83_l04g12.png}}
% 108
{\PTglyph{5}{t83_l04g13.png}}
% 109
{\PTglyph{5}{t83_l04g14.png}}
% 110
{\PTglyph{5}{t83_l04g15.png}}
% 111
{\PTglyph{5}{t83_l04g16.png}}
% 112
{\PTglyph{5}{t83_l04g17.png}}
% 113
{\PTglyph{5}{t83_l04g18.png}}
% 114
{\PTglyph{5}{t83_l04g19.png}}
% 115
{\PTglyph{5}{t83_l04g20.png}}
% 116
{\PTglyph{5}{t83_l04g21.png}}
% 117
{\PTglyph{5}{t83_l04g22.png}}
% 118
{\PTglyph{5}{t83_l04g23.png}}
% 119
{\PTglyph{5}{t83_l04g24.png}}
% 120
{\PTglyph{5}{t83_l04g25.png}}
% 121
{\PTglyph{5}{t83_l04g26.png}}
% 122
{\PTglyph{5}{t83_l04g27.png}}
% 123
{\PTglyph{5}{t83_l04g28.png}}
% 124
{\PTglyph{5}{t83_l04g29.png}}
% 125
{\PTglyph{5}{t83_l04g30.png}}
% 126
{\PTglyph{5}{t83_l04g31.png}}
//
%%% Local Variables:
%%% mode: latex
%%% TeX-engine: luatex
%%% TeX-master: shared
%%% End:

%//
%\glpismo%
 \glpismo
% 1
{\PTglyphid{Wi-09_0101}}
% 2
{\PTglyphid{Wi-09_0102}}
% 3
{\PTglyphid{Wi-09_0103}}
% 4
{\PTglyphid{Wi-09_0104}}
% 5
{\PTglyphid{Wi-09_0105}}
% 6
{\PTglyphid{Wi-09_0106}}
% 7
{\PTglyphid{Wi-09_0107}}
% 8
{\PTglyphid{Wi-09_0108}}
% 9
{\PTglyphid{Wi-09_0109}}
% 10
{\PTglyphid{Wi-09_0110}}
% 11
{\PTglyphid{Wi-09_0111}}
% 12
{\PTglyphid{Wi-09_0112}}
% 13
{\PTglyphid{Wi-09_0113}}
% 14
{\PTglyphid{Wi-09_0114}}
% 15
{\PTglyphid{Wi-09_0115}}
% 16
{\PTglyphid{Wi-09_0116}}
% 17
{\PTglyphid{Wi-09_0117}}
% 18
{\PTglyphid{Wi-09_0118}}
% 19
{\PTglyphid{Wi-09_0119}}
% 20
{\PTglyphid{Wi-09_0120}}
% 21
{\PTglyphid{Wi-09_0121}}
% 22
{\PTglyphid{Wi-09_0122}}
% 23
{\PTglyphid{Wi-09_0123}}
% 24
{\PTglyphid{Wi-09_0124}}
% 25
{\PTglyphid{Wi-09_0201}}
% 26
{\PTglyphid{Wi-09_0202}}
% 27
{\PTglyphid{Wi-09_0203}}
% 28
{\PTglyphid{Wi-09_0204}}
% 29
{\PTglyphid{Wi-09_0205}}
% 30
{\PTglyphid{Wi-09_0206}}
% 31
{\PTglyphid{Wi-09_0207}}
% 32
{\PTglyphid{Wi-09_0208}}
% 33
{\PTglyphid{Wi-09_0209}}
% 34
{\PTglyphid{Wi-09_0210}}
% 35
{\PTglyphid{Wi-09_0211}}
% 36
{\PTglyphid{Wi-09_0212}}
% 37
{\PTglyphid{Wi-09_0213}}
% 38
{\PTglyphid{Wi-09_0214}}
% 39
{\PTglyphid{Wi-09_0215}}
% 40
{\PTglyphid{Wi-09_0216}}
% 41
{\PTglyphid{Wi-09_0217}}
% 42
{\PTglyphid{Wi-09_0218}}
% 43
{\PTglyphid{Wi-09_0219}}
% 44
{\PTglyphid{Wi-09_0220}}
% 45
{\PTglyphid{Wi-09_0221}}
% 46
{\PTglyphid{Wi-09_0222}}
% 47
{\PTglyphid{Wi-09_0223}}
% 48
{\PTglyphid{Wi-09_0224}}
% 49
{\PTglyphid{Wi-09_0225}}
% 50
{\PTglyphid{Wi-09_0226}}
% 51
{\PTglyphid{Wi-09_0227}}
% 52
{\PTglyphid{Wi-09_0228}}
% 53
{\PTglyphid{Wi-09_0229}}
% 54
{\PTglyphid{Wi-09_0230}}
% 55
{\PTglyphid{Wi-09_0231}}
% 56
{\PTglyphid{Wi-09_0232}}
% 57
{\PTglyphid{Wi-09_0233}}
% 58
{\PTglyphid{Wi-09_0234}}
% 59
{\PTglyphid{Wi-09_0235}}
% 60
{\PTglyphid{Wi-09_0236}}
% 61
{\PTglyphid{Wi-09_0237}}
% 62
{\PTglyphid{Wi-09_0238}}
% 63
{\PTglyphid{Wi-09_0301}}
% 64
{\PTglyphid{Wi-09_0302}}
% 65
{\PTglyphid{Wi-09_0303}}
% 66
{\PTglyphid{Wi-09_0304}}
% 67
{\PTglyphid{Wi-09_0305}}
% 68
{\PTglyphid{Wi-09_0306}}
% 69
{\PTglyphid{Wi-09_0307}}
% 70
{\PTglyphid{Wi-09_0308}}
% 71
{\PTglyphid{Wi-09_0309}}
% 72
{\PTglyphid{Wi-09_0310}}
% 73
{\PTglyphid{Wi-09_0311}}
% 74
{\PTglyphid{Wi-09_0312}}
% 75
{\PTglyphid{Wi-09_0313}}
% 76
{\PTglyphid{Wi-09_0314}}
% 77
{\PTglyphid{Wi-09_0315}}
% 78
{\PTglyphid{Wi-09_0316}}
% 79
{\PTglyphid{Wi-09_0317}}
% 80
{\PTglyphid{Wi-09_0318}}
% 81
{\PTglyphid{Wi-09_0319}}
% 82
{\PTglyphid{Wi-09_0320}}
% 83
{\PTglyphid{Wi-09_0321}}
% 84
{\PTglyphid{Wi-09_0322}}
% 85
{\PTglyphid{Wi-09_0323}}
% 86
{\PTglyphid{Wi-09_0324}}
% 87
{\PTglyphid{Wi-09_0325}}
% 88
{\PTglyphid{Wi-09_0326}}
% 89
{\PTglyphid{Wi-09_0327}}
% 90
{\PTglyphid{Wi-09_0328}}
% 91
{\PTglyphid{Wi-09_0329}}
% 92
{\PTglyphid{Wi-09_0330}}
% 93
{\PTglyphid{Wi-09_0331}}
% 94
{\PTglyphid{Wi-09_0332}}
% 95
{\PTglyphid{Wi-09_0333}}
% 96
{\PTglyphid{Wi-09_0401}}
% 97
{\PTglyphid{Wi-09_0402}}
% 98
{\PTglyphid{Wi-09_0403}}
% 99
{\PTglyphid{Wi-09_0404}}
% 100
{\PTglyphid{Wi-09_0405}}
% 101
{\PTglyphid{Wi-09_0406}}
% 102
{\PTglyphid{Wi-09_0407}}
% 103
{\PTglyphid{Wi-09_0408}}
% 104
{\PTglyphid{Wi-09_0409}}
% 105
{\PTglyphid{Wi-09_0410}}
% 106
{\PTglyphid{Wi-09_0411}}
% 107
{\PTglyphid{Wi-09_0412}}
% 108
{\PTglyphid{Wi-09_0413}}
% 109
{\PTglyphid{Wi-09_0414}}
% 110
{\PTglyphid{Wi-09_0415}}
% 111
{\PTglyphid{Wi-09_0416}}
% 112
{\PTglyphid{Wi-09_0417}}
% 113
{\PTglyphid{Wi-09_0418}}
% 114
{\PTglyphid{Wi-09_0419}}
% 115
{\PTglyphid{Wi-09_0420}}
% 116
{\PTglyphid{Wi-09_0421}}
% 117
{\PTglyphid{Wi-09_0422}}
% 118
{\PTglyphid{Wi-09_0423}}
% 119
{\PTglyphid{Wi-09_0424}}
% 120
{\PTglyphid{Wi-09_0425}}
% 121
{\PTglyphid{Wi-09_0426}}
% 122
{\PTglyphid{Wi-09_0427}}
% 123
{\PTglyphid{Wi-09_0428}}
% 124
{\PTglyphid{Wi-09_0429}}
% 125
{\PTglyphid{Wi-09_0430}}
% 126
{\PTglyphid{Wi-09_0431}}
//
\endgl \xe
%%% Local Variables:
%%% mode: latex
%%% TeX-engine: luatex
%%% TeX-master: shared
%%% End:

% //
%\endgl \xe


 \newpage
 
%%%%%%%%%%%%%%%%%%%%%%%%%%%%%%%%%%%%%%%%%%%%%%%%%%%%%%%%%%%%%%%%%%%%%%%%%%%%%%%
% from meta.csv
% 84,Wirzbięta-10+8_PT11_563.djvu,Wirzbięta,10+8,11,563
% 
%%%%%%%%%%%%%%%%%%%%%%%%%%%%%%%%%%%%%%%%%%%%%%%%%%%%%%%%%%%%%%%%%%%%%%%%%%%%%%%

 
% from dsed4test:
% Wirzbięta-10+8_PT11_563_4dsed.txt:Note "10. Pismo tekstowe, antykwa. Stopień 20 ww. == 113/114 mm. Zob. p. 8. — Tabl. 494, 514, 522, 529, % 563, 568. [563]"
% Wirzbięta-10+8_PT11_563_4dsed.txt:Note0 "8. Antykwa polska dorobiona do pisma 10. — Tabl. 458. Zob. p. 10."
% Wirzbięta-10+8_PT11_563_4dsed.txt:Note1 "Character set table prepared by Anna Śliwa"


 \pismoPL{Maciej Wirzbięta 10. Pismo tekstowe, antykwa. Stopień 20 ww. == 113/114 mm. Zob. p. 8. — Tabl. 494, 514, 522, 529,  563, 568.}
 \pismoPL{Maciej Wirzbięta 8. Antykwa polska dorobiona do pisma 10. — Tabl. 458. Zob. p. 10.}
  
 \pismoEN{Maciej Wirzbięta 10. Roman text font. 
Type size 20 lines =  113/114 mm. — See font 8. — Plates 494, 514, 522, 529,  563, 568.}

 \pismoEN{Maciej Wirzbięta 8. Roman Polish font supplementing font 10. Plates. 458. See font 10.}


\plate{562}{XI}{1981}

The plate prepared by Alodia Kawecka Gryczowa.\\
The font table prepared by Alodia Kawecka Gryczowa and Anna Śliwa.

\bigskip

\exampleBib{IX:14}

\bigskip \exampleDesc{[MIKOŁAJ REJ]: Zwierzyniec. [Wyd. 1]. Kraków, Maciej Wirzbięta, [1561/1562]. 4⁰.}


  \medskip
  \examplePage{\textit{Karta A₂b.}}

  \bigskip \exampleRef{\textit{Estreicher XXVI 178, 189-190. Rostkowska 31.}}

  \bigskip

%  \exampleDig{\url{https://polona.pl/preview/75bc2fa5-636e-468d-a04c-f3ce81264f15}



%   Pismo 11: wiersz 1-2, wersalik N w wierszu 3. 

\bigskip

\fontID{Wi-10+8}{84}

\fontstat{105}

% \exdisplay \bg \gla
 \exdisplay \bg \gla
% 1
{\PTglyph{5}{t84_l01g01.png}}
% 2
{\PTglyph{5}{t84_l01g02.png}}
% 3
{\PTglyph{5}{t84_l01g03.png}}
% 4
{\PTglyph{5}{t84_l01g04.png}}
% 5
{\PTglyph{5}{t84_l01g05.png}}
% 6
{\PTglyph{5}{t84_l01g06.png}}
% 7
{\PTglyph{5}{t84_l01g07.png}}
% 8
{\PTglyph{5}{t84_l01g08.png}}
% 9
{\PTglyph{5}{t84_l01g09.png}}
% 10
{\PTglyph{5}{t84_l01g10.png}}
% 11
{\PTglyph{5}{t84_l01g11.png}}
% 12
{\PTglyph{5}{t84_l01g12.png}}
% 13
{\PTglyph{5}{t84_l01g13.png}}
% 14
{\PTglyph{5}{t84_l01g14.png}}
% 15
{\PTglyph{5}{t84_l01g15.png}}
% 16
{\PTglyph{5}{t84_l01g16.png}}
% 17
{\PTglyph{5}{t84_l01g17.png}}
% 18
{\PTglyph{5}{t84_l01g18.png}}
% 19
{\PTglyph{5}{t84_l01g19.png}}
% 20
{\PTglyph{5}{t84_l01g20.png}}
% 21
{\PTglyph{5}{t84_l01g21.png}}
% 22
{\PTglyph{5}{t84_l01g22.png}}
% 23
{\PTglyph{5}{t84_l01g23.png}}
% 24
{\PTglyph{5}{t84_l01g24.png}}
% 25
{\PTglyph{5}{t84_l02g01.png}}
% 26
{\PTglyph{5}{t84_l02g02.png}}
% 27
{\PTglyph{5}{t84_l02g03.png}}
% 28
{\PTglyph{5}{t84_l02g04.png}}
% 29
{\PTglyph{5}{t84_l02g05.png}}
% 30
{\PTglyph{5}{t84_l02g06.png}}
% 31
{\PTglyph{5}{t84_l02g07.png}}
% 32
{\PTglyph{5}{t84_l02g08.png}}
% 33
{\PTglyph{5}{t84_l02g09.png}}
% 34
{\PTglyph{5}{t84_l02g10.png}}
% 35
{\PTglyph{5}{t84_l02g11.png}}
% 36
{\PTglyph{5}{t84_l02g12.png}}
% 37
{\PTglyph{5}{t84_l02g13.png}}
% 38
{\PTglyph{5}{t84_l02g14.png}}
% 39
{\PTglyph{5}{t84_l02g15.png}}
% 40
{\PTglyph{5}{t84_l02g16.png}}
% 41
{\PTglyph{5}{t84_l02g17.png}}
% 42
{\PTglyph{5}{t84_l02g18.png}}
% 43
{\PTglyph{5}{t84_l02g19.png}}
% 44
{\PTglyph{5}{t84_l02g20.png}}
% 45
{\PTglyph{5}{t84_l02g21.png}}
% 46
{\PTglyph{5}{t84_l02g22.png}}
% 47
{\PTglyph{5}{t84_l02g23.png}}
% 48
{\PTglyph{5}{t84_l02g24.png}}
% 49
{\PTglyph{5}{t84_l02g25.png}}
% 50
{\PTglyph{5}{t84_l02g26.png}}
% 51
{\PTglyph{5}{t84_l02g27.png}}
% 52
{\PTglyph{5}{t84_l02g28.png}}
% 53
{\PTglyph{5}{t84_l02g29.png}}
% 54
{\PTglyph{5}{t84_l02g30.png}}
% 55
{\PTglyph{5}{t84_l02g31.png}}
% 56
{\PTglyph{5}{t84_l02g32.png}}
% 57
{\PTglyph{5}{t84_l02g33.png}}
% 58
{\PTglyph{5}{t84_l03g01.png}}
% 59
{\PTglyph{5}{t84_l03g02.png}}
% 60
{\PTglyph{5}{t84_l03g03.png}}
% 61
{\PTglyph{5}{t84_l03g04.png}}
% 62
{\PTglyph{5}{t84_l03g05.png}}
% 63
{\PTglyph{5}{t84_l03g06.png}}
% 64
{\PTglyph{5}{t84_l03g07.png}}
% 65
{\PTglyph{5}{t84_l03g08.png}}
% 66
{\PTglyph{5}{t84_l03g09.png}}
% 67
{\PTglyph{5}{t84_l03g10.png}}
% 68
{\PTglyph{5}{t84_l03g11.png}}
% 69
{\PTglyph{5}{t84_l03g12.png}}
% 70
{\PTglyph{5}{t84_l03g13.png}}
% 71
{\PTglyph{5}{t84_l03g14.png}}
% 72
{\PTglyph{5}{t84_l03g15.png}}
% 73
{\PTglyph{5}{t84_l03g16.png}}
% 74
{\PTglyph{5}{t84_l03g17.png}}
% 75
{\PTglyph{5}{t84_l03g18.png}}
% 76
{\PTglyph{5}{t84_l03g19.png}}
% 77
{\PTglyph{5}{t84_l03g20.png}}
% 78
{\PTglyph{5}{t84_l03g21.png}}
% 79
{\PTglyph{5}{t84_l03g22.png}}
% 80
{\PTglyph{5}{t84_l03g23.png}}
% 81
{\PTglyph{5}{t84_l03g24.png}}
% 82
{\PTglyph{5}{t84_l03g25.png}}
% 83
{\PTglyph{5}{t84_l03g26.png}}
% 84
{\PTglyph{5}{t84_l03g27.png}}
% 85
{\PTglyph{5}{t84_l03g28.png}}
% 86
{\PTglyph{5}{t84_l03g29.png}}
% 87
{\PTglyph{5}{t84_l03g30.png}}
% 88
{\PTglyph{5}{t84_l04g01.png}}
% 89
{\PTglyph{5}{t84_l04g02.png}}
% 90
{\PTglyph{5}{t84_l04g03.png}}
% 91
{\PTglyph{5}{t84_l04g04.png}}
% 92
{\PTglyph{5}{t84_l04g05.png}}
% 93
{\PTglyph{5}{t84_l04g06.png}}
% 94
{\PTglyph{5}{t84_l04g07.png}}
% 95
{\PTglyph{5}{t84_l04g08.png}}
% 96
{\PTglyph{5}{t84_l04g09.png}}
% 97
{\PTglyph{5}{t84_l04g10.png}}
% 98
{\PTglyph{5}{t84_l04g11.png}}
% 99
{\PTglyph{5}{t84_l04g12.png}}
% 100
{\PTglyph{5}{t84_l04g13.png}}
% 101
{\PTglyph{5}{t84_l04g14.png}}
% 102
{\PTglyph{5}{t84_l04g15.png}}
% 103
{\PTglyph{5}{t84_l04g16.png}}
% 104
{\PTglyph{5}{t84_l04g17.png}}
% 105
{\PTglyph{5}{t84_l04g18.png}}
//
%%% Local Variables:
%%% mode: latex
%%% TeX-engine: luatex
%%% TeX-master: shared
%%% End:

%//
%\glpismo%
 \glpismo
% 1
{\PTglyphid{Wi-10+80101}}
% 2
{\PTglyphid{Wi-10+80102}}
% 3
{\PTglyphid{Wi-10+80103}}
% 4
{\PTglyphid{Wi-10+80104}}
% 5
{\PTglyphid{Wi-10+80105}}
% 6
{\PTglyphid{Wi-10+80106}}
% 7
{\PTglyphid{Wi-10+80107}}
% 8
{\PTglyphid{Wi-10+80108}}
% 9
{\PTglyphid{Wi-10+80109}}
% 10
{\PTglyphid{Wi-10+80110}}
% 11
{\PTglyphid{Wi-10+80111}}
% 12
{\PTglyphid{Wi-10+80112}}
% 13
{\PTglyphid{Wi-10+80113}}
% 14
{\PTglyphid{Wi-10+80114}}
% 15
{\PTglyphid{Wi-10+80115}}
% 16
{\PTglyphid{Wi-10+80116}}
% 17
{\PTglyphid{Wi-10+80117}}
% 18
{\PTglyphid{Wi-10+80118}}
% 19
{\PTglyphid{Wi-10+80119}}
% 20
{\PTglyphid{Wi-10+80120}}
% 21
{\PTglyphid{Wi-10+80121}}
% 22
{\PTglyphid{Wi-10+80122}}
% 23
{\PTglyphid{Wi-10+80123}}
% 24
{\PTglyphid{Wi-10+80124}}
% 25
{\PTglyphid{Wi-10+80201}}
% 26
{\PTglyphid{Wi-10+80202}}
% 27
{\PTglyphid{Wi-10+80203}}
% 28
{\PTglyphid{Wi-10+80204}}
% 29
{\PTglyphid{Wi-10+80205}}
% 30
{\PTglyphid{Wi-10+80206}}
% 31
{\PTglyphid{Wi-10+80207}}
% 32
{\PTglyphid{Wi-10+80208}}
% 33
{\PTglyphid{Wi-10+80209}}
% 34
{\PTglyphid{Wi-10+80210}}
% 35
{\PTglyphid{Wi-10+80211}}
% 36
{\PTglyphid{Wi-10+80212}}
% 37
{\PTglyphid{Wi-10+80213}}
% 38
{\PTglyphid{Wi-10+80214}}
% 39
{\PTglyphid{Wi-10+80215}}
% 40
{\PTglyphid{Wi-10+80216}}
% 41
{\PTglyphid{Wi-10+80217}}
% 42
{\PTglyphid{Wi-10+80218}}
% 43
{\PTglyphid{Wi-10+80219}}
% 44
{\PTglyphid{Wi-10+80220}}
% 45
{\PTglyphid{Wi-10+80221}}
% 46
{\PTglyphid{Wi-10+80222}}
% 47
{\PTglyphid{Wi-10+80223}}
% 48
{\PTglyphid{Wi-10+80224}}
% 49
{\PTglyphid{Wi-10+80225}}
% 50
{\PTglyphid{Wi-10+80226}}
% 51
{\PTglyphid{Wi-10+80227}}
% 52
{\PTglyphid{Wi-10+80228}}
% 53
{\PTglyphid{Wi-10+80229}}
% 54
{\PTglyphid{Wi-10+80230}}
% 55
{\PTglyphid{Wi-10+80231}}
% 56
{\PTglyphid{Wi-10+80232}}
% 57
{\PTglyphid{Wi-10+80233}}
% 58
{\PTglyphid{Wi-10+80301}}
% 59
{\PTglyphid{Wi-10+80302}}
% 60
{\PTglyphid{Wi-10+80303}}
% 61
{\PTglyphid{Wi-10+80304}}
% 62
{\PTglyphid{Wi-10+80305}}
% 63
{\PTglyphid{Wi-10+80306}}
% 64
{\PTglyphid{Wi-10+80307}}
% 65
{\PTglyphid{Wi-10+80308}}
% 66
{\PTglyphid{Wi-10+80309}}
% 67
{\PTglyphid{Wi-10+80310}}
% 68
{\PTglyphid{Wi-10+80311}}
% 69
{\PTglyphid{Wi-10+80312}}
% 70
{\PTglyphid{Wi-10+80313}}
% 71
{\PTglyphid{Wi-10+80314}}
% 72
{\PTglyphid{Wi-10+80315}}
% 73
{\PTglyphid{Wi-10+80316}}
% 74
{\PTglyphid{Wi-10+80317}}
% 75
{\PTglyphid{Wi-10+80318}}
% 76
{\PTglyphid{Wi-10+80319}}
% 77
{\PTglyphid{Wi-10+80320}}
% 78
{\PTglyphid{Wi-10+80321}}
% 79
{\PTglyphid{Wi-10+80322}}
% 80
{\PTglyphid{Wi-10+80323}}
% 81
{\PTglyphid{Wi-10+80324}}
% 82
{\PTglyphid{Wi-10+80325}}
% 83
{\PTglyphid{Wi-10+80326}}
% 84
{\PTglyphid{Wi-10+80327}}
% 85
{\PTglyphid{Wi-10+80328}}
% 86
{\PTglyphid{Wi-10+80329}}
% 87
{\PTglyphid{Wi-10+80330}}
% 88
{\PTglyphid{Wi-10+80401}}
% 89
{\PTglyphid{Wi-10+80402}}
% 90
{\PTglyphid{Wi-10+80403}}
% 91
{\PTglyphid{Wi-10+80404}}
% 92
{\PTglyphid{Wi-10+80405}}
% 93
{\PTglyphid{Wi-10+80406}}
% 94
{\PTglyphid{Wi-10+80407}}
% 95
{\PTglyphid{Wi-10+80408}}
% 96
{\PTglyphid{Wi-10+80409}}
% 97
{\PTglyphid{Wi-10+80410}}
% 98
{\PTglyphid{Wi-10+80411}}
% 99
{\PTglyphid{Wi-10+80412}}
% 100
{\PTglyphid{Wi-10+80413}}
% 101
{\PTglyphid{Wi-10+80414}}
% 102
{\PTglyphid{Wi-10+80415}}
% 103
{\PTglyphid{Wi-10+80416}}
% 104
{\PTglyphid{Wi-10+80417}}
% 105
{\PTglyphid{Wi-10+80418}}
//
\endgl \xe
%%% Local Variables:
%%% mode: latex
%%% TeX-engine: luatex
%%% TeX-master: shared
%%% End:

% //
%\endgl \xe
 

 \newpage
 
%%%%%%%%%%%%%%%%%%%%%%%%%%%%%%%%%%%%%%%%%%%%%%%%%%%%%%%%%%%%%%%%%%%%%%%%%%%%%%%
% from meta.csv
% 85,Wirzbięta-11+11a_PT11_564.djvu,Wirzbięta,11+11a,11,564
% 
%%%%%%%%%%%%%%%%%%%%%%%%%%%%%%%%%%%%%%%%%%%%%%%%%%%%%%%%%%%%%%%%%%%%%%%%%%%%%%%

 
% from dsed4test:
% Wirzbięta-11+11a_PT11_564_4dsed.txt:Note "11. Wersaliki tytułowe, antykwa. Wysokość 7 mm. — Tabl. 514, 522, 529, 540, 563, 564, 568. [564]"
% Wirzbięta-11+11a_PT11_564_4dsed.txt:Note0 "11a. Wersaliki tytułowe greckie, dorobione do p. 11. — Tabl. 564."
%Wirzbięta-11+11a_PT11_564_4dsed.txt:Note1 "Character set table prepared by Anna Śliwa"

 \pismoPL{Maciej Wirzbięta 11. Wersaliki tytułowe, antykwa. Wysokość 7 mm. — Tabl. 514, 522, 529, 540, 563, 564, 568.}
 \pismoPL{Maciej Wirzbięta 11a. Wersaliki tytułowe greckie, dorobione do p. 11. — Tabl. 564.}
  
 \pismoEN{Maciej Wirzbięta 10. Roman capital font.  Type size [1
   line?] = 7 mm. — Plates 514, 522, 529, 540, 563, 564, 568.}

 \pismoEN{Maciej Wirzbięta 11a. Greek capital font supplementing font 11. Plates 564}


\plate{564[1]}{XI}{1981}

The plate prepared by Alodia Kawecka Gryczowa.\\
The font table prepared by Alodia Kawecka Gryczowa and Anna Śliwa

\bigskip

\exampleBib{IX:109}

\bigskip \exampleDesc{ALEXANDER GUAGNINUS: Sarmatiae Europeae descriptio.
Kraków, Maciej Wirzbięta, [po 20 VI 1578]. 2⁰.}


  \medskip
  \examplePage{\textit{Karta aaa₁a}}

  \bigskip \exampleRef{\textit{Estreicher XVII 484. Wierzbowski 729. Horyniec 83.}

  \bigskip

  \exampleDig{\url{https://www.dbc.wroc.pl/dlibra/doccontent?id=5021} page  224}

  
%  https://wbc.poznan.pl/dlibra/publication/493318/edition/425366?language=pl brak


%  Pismo Ir, Ira: zestaw 1. —Pismo 12: zestaw 2 i kolumna. —Przerywnik 20.


\bigskip

\fontID{Wi-11+11a}{85}

\fontstat{31}

% \exdisplay \bg \gla
 \exdisplay \bg \gla
% 1
{\PTglyph{5}{t85_l01g01.png}}
% 2
{\PTglyph{5}{t85_l01g02.png}}
% 3
{\PTglyph{5}{t85_l01g03.png}}
% 4
{\PTglyph{5}{t85_l01g04.png}}
% 5
{\PTglyph{5}{t85_l01g05.png}}
% 6
{\PTglyph{5}{t85_l01g06.png}}
% 7
{\PTglyph{5}{t85_l01g07.png}}
% 8
{\PTglyph{5}{t85_l01g08.png}}
% 9
{\PTglyph{5}{t85_l01g09.png}}
% 10
{\PTglyph{5}{t85_l01g10.png}}
% 11
{\PTglyph{5}{t85_l01g11.png}}
% 12
{\PTglyph{5}{t85_l01g12.png}}
% 13
{\PTglyph{5}{t85_l01g13.png}}
% 14
{\PTglyph{5}{t85_l01g14.png}}
% 15
{\PTglyph{5}{t85_l01g15.png}}
% 16
{\PTglyph{5}{t85_l01g16.png}}
% 17
{\PTglyph{5}{t85_l01g17.png}}
% 18
{\PTglyph{5}{t85_l01g18.png}}
% 19
{\PTglyph{5}{t85_l01g19.png}}
% 20
{\PTglyph{5}{t85_l01g20.png}}
% 21
{\PTglyph{5}{t85_l01g21.png}}
% 22
{\PTglyph{5}{t85_l02g01.png}}
% 23
{\PTglyph{5}{t85_l02g02.png}}
% 24
{\PTglyph{5}{t85_l02g03.png}}
% 25
{\PTglyph{5}{t85_l02g04.png}}
% 26
{\PTglyph{5}{t85_l02g05.png}}
% 27
{\PTglyph{5}{t85_l02g06.png}}
% 28
{\PTglyph{5}{t85_l02g07.png}}
% 29
{\PTglyph{5}{t85_l02g08.png}}
% 30
{\PTglyph{5}{t85_l02g09.png}}
% 31
{\PTglyph{5}{t85_l02g10.png}}
//
%%% Local Variables:
%%% mode: latex
%%% TeX-engine: luatex
%%% TeX-master: shared
%%% End:

%//
%\glpismo%
 \glpismo
% 1
{\PTglyphid{Wi-11+11a0101}}
% 2
{\PTglyphid{Wi-11+11a0102}}
% 3
{\PTglyphid{Wi-11+11a0103}}
% 4
{\PTglyphid{Wi-11+11a0104}}
% 5
{\PTglyphid{Wi-11+11a0105}}
% 6
{\PTglyphid{Wi-11+11a0106}}
% 7
{\PTglyphid{Wi-11+11a0107}}
% 8
{\PTglyphid{Wi-11+11a0108}}
% 9
{\PTglyphid{Wi-11+11a0109}}
% 10
{\PTglyphid{Wi-11+11a0110}}
% 11
{\PTglyphid{Wi-11+11a0111}}
% 12
{\PTglyphid{Wi-11+11a0112}}
% 13
{\PTglyphid{Wi-11+11a0113}}
% 14
{\PTglyphid{Wi-11+11a0114}}
% 15
{\PTglyphid{Wi-11+11a0115}}
% 16
{\PTglyphid{Wi-11+11a0116}}
% 17
{\PTglyphid{Wi-11+11a0117}}
% 18
{\PTglyphid{Wi-11+11a0118}}
% 19
{\PTglyphid{Wi-11+11a0119}}
% 20
{\PTglyphid{Wi-11+11a0120}}
% 21
{\PTglyphid{Wi-11+11a0121}}
% 22
{\PTglyphid{Wi-11+11a0201}}
% 23
{\PTglyphid{Wi-11+11a0202}}
% 24
{\PTglyphid{Wi-11+11a0203}}
% 25
{\PTglyphid{Wi-11+11a0204}}
% 26
{\PTglyphid{Wi-11+11a0205}}
% 27
{\PTglyphid{Wi-11+11a0206}}
% 28
{\PTglyphid{Wi-11+11a0207}}
% 29
{\PTglyphid{Wi-11+11a0208}}
% 30
{\PTglyphid{Wi-11+11a0209}}
% 31
{\PTglyphid{Wi-11+11a0210}}
//
\endgl \xe
%%% Local Variables:
%%% mode: latex
%%% TeX-engine: luatex
%%% TeX-master: shared
%%% End:

% //
%\endgl \xe


 \newpage
 
%%%%%%%%%%%%%%%%%%%%%%%%%%%%%%%%%%%%%%%%%%%%%%%%%%%%%%%%%%%%%%%%%%%%%%%%%%%%%%%
% from meta.csv
% 86,Wirzbięta-12_PT11_564.djvu,Wirzbięta,12,11,564
% 
%%%%%%%%%%%%%%%%%%%%%%%%%%%%%%%%%%%%%%%%%%%%%%%%%%%%%%%%%%%%%%%%%%%%%%%%%%%%%%%

 
% from dsed4test:
% Wirzbięta-12_PT11_564_4dsed.txt:Note "12. Pismo nagłówkowe, antykwa. Stopień 5 ww. = 70 mm, wersaliki 10 mm. — Tabl. 522, 529-532, 540, 543, 556, 564. [564]"
% Wirzbięta-12_PT11_564_4dsed.txt:Note1 "Character set table prepared by Anna Śliwa"

 \pismoPL{Maciej Wirzbięta 12. Pismo nagłówkowe, antykwa. Stopień 5
   ww. = 70 mm, wersaliki 10 mm. — Tabl. 522, 529-532, 540, 543, 556,
   564.}

 % \pismoPL{Maciej Wirzbięta 11a. Wersaliki tytułowe
 %   greckie, dorobione do p. 11. — Tabl. 564.}
  
 \pismoEN{Maciej Wirzbięta 12. Roman header font.  Type size 5
   lines = 70 mm, capitals 10 mm. — Plates Tabl. 522, 529-532, 540, 543, 556,
   564.}

% \pismoEN{Maciej Wirzbięta 11a. Greek capital font supplementing font 11. Plates 564}


\plate{564[2]}{XI}{1981}

The plate prepared by Alodia Kawecka Gryczowa.\\
The font table prepared by Alodia Kawecka Gryczowa and Anna Śliwa.

\bigskip

\exampleBib{IX:109}

\bigskip \exampleDesc{ALEXANDER GUAGNINUS: Sarmatiae Europeae descriptio.
Kraków, Maciej Wirzbięta, [po 20 VI 1578]. 2⁰.}


  \medskip
  \examplePage{\textit{Karta aaa₁a}}

  \bigskip \exampleRef{\textit{Estreicher XVII 484. Wierzbowski 729. Horyniec 83.}

  \bigskip

  \exampleDig{\url{https://www.dbc.wroc.pl/dlibra/doccontent?id=5021} page  224}

  
%  https://wbc.poznan.pl/dlibra/publication/493318/edition/425366?language=pl brak


%  Pismo Ir, Ira: zestaw 1. —Pismo 12: zestaw 2 i kolumna. —Przerywnik 20.


\bigskip

\fontID{Wi-12}{86}

\fontstat{31}

% \exdisplay \bg \gla
 \exdisplay \bg \gla
% 1
{\PTglyph{5}{t86_l01g01.png}}
% 2
{\PTglyph{5}{t86_l01g02.png}}
% 3
{\PTglyph{5}{t86_l01g03.png}}
% 4
{\PTglyph{5}{t86_l01g04.png}}
% 5
{\PTglyph{5}{t86_l01g05.png}}
% 6
{\PTglyph{5}{t86_l01g06.png}}
% 7
{\PTglyph{5}{t86_l01g07.png}}
% 8
{\PTglyph{5}{t86_l01g08.png}}
% 9
{\PTglyph{5}{t86_l01g09.png}}
% 10
{\PTglyph{5}{t86_l01g10.png}}
% 11
{\PTglyph{5}{t86_l01g11.png}}
% 12
{\PTglyph{5}{t86_l01g12.png}}
% 13
{\PTglyph{5}{t86_l01g13.png}}
% 14
{\PTglyph{5}{t86_l01g14.png}}
% 15
{\PTglyph{5}{t86_l01g15.png}}
% 16
{\PTglyph{5}{t86_l01g16.png}}
% 17
{\PTglyph{5}{t86_l02g01.png}}
% 18
{\PTglyph{5}{t86_l02g02.png}}
% 19
{\PTglyph{5}{t86_l02g03.png}}
% 20
{\PTglyph{5}{t86_l02g04.png}}
% 21
{\PTglyph{5}{t86_l02g05.png}}
% 22
{\PTglyph{5}{t86_l02g06.png}}
% 23
{\PTglyph{5}{t86_l02g07.png}}
% 24
{\PTglyph{5}{t86_l02g08.png}}
% 25
{\PTglyph{5}{t86_l02g09.png}}
% 26
{\PTglyph{5}{t86_l02g10.png}}
% 27
{\PTglyph{5}{t86_l02g11.png}}
% 28
{\PTglyph{5}{t86_l02g12.png}}
% 29
{\PTglyph{5}{t86_l02g13.png}}
% 30
{\PTglyph{5}{t86_l02g14.png}}
% 31
{\PTglyph{5}{t86_l03g01.png}}
% 32
{\PTglyph{5}{t86_l03g02.png}}
% 33
{\PTglyph{5}{t86_l03g03.png}}
% 34
{\PTglyph{5}{t86_l03g04.png}}
% 35
{\PTglyph{5}{t86_l03g05.png}}
% 36
{\PTglyph{5}{t86_l03g06.png}}
% 37
{\PTglyph{5}{t86_l03g07.png}}
% 38
{\PTglyph{5}{t86_l03g08.png}}
% 39
{\PTglyph{5}{t86_l03g09.png}}
% 40
{\PTglyph{5}{t86_l03g10.png}}
% 41
{\PTglyph{5}{t86_l03g11.png}}
% 42
{\PTglyph{5}{t86_l03g12.png}}
% 43
{\PTglyph{5}{t86_l03g13.png}}
% 44
{\PTglyph{5}{t86_l03g14.png}}
% 45
{\PTglyph{5}{t86_l03g15.png}}
% 46
{\PTglyph{5}{t86_l03g16.png}}
% 47
{\PTglyph{5}{t86_l03g17.png}}
% 48
{\PTglyph{5}{t86_l03g18.png}}
% 49
{\PTglyph{5}{t86_l03g19.png}}
% 50
{\PTglyph{5}{t86_l03g20.png}}
% 51
{\PTglyph{5}{t86_l03g21.png}}
% 52
{\PTglyph{5}{t86_l03g22.png}}
% 53
{\PTglyph{5}{t86_l03g23.png}}
% 54
{\PTglyph{5}{t86_l03g24.png}}
% 55
{\PTglyph{5}{t86_l03g25.png}}
% 56
{\PTglyph{5}{t86_l04g01.png}}
% 57
{\PTglyph{5}{t86_l04g02.png}}
% 58
{\PTglyph{5}{t86_l04g03.png}}
% 59
{\PTglyph{5}{t86_l04g04.png}}
% 60
{\PTglyph{5}{t86_l04g05.png}}
% 61
{\PTglyph{5}{t86_l04g06.png}}
% 62
{\PTglyph{5}{t86_l04g07.png}}
% 63
{\PTglyph{5}{t86_l04g08.png}}
% 64
{\PTglyph{5}{t86_l04g09.png}}
% 65
{\PTglyph{5}{t86_l04g10.png}}
% 66
{\PTglyph{5}{t86_l04g11.png}}
% 67
{\PTglyph{5}{t86_l04g12.png}}
% 68
{\PTglyph{5}{t86_l04g13.png}}
% 69
{\PTglyph{5}{t86_l04g14.png}}
% 70
{\PTglyph{5}{t86_l04g15.png}}
% 71
{\PTglyph{5}{t86_l04g16.png}}
% 72
{\PTglyph{5}{t86_l04g17.png}}
% 73
{\PTglyph{5}{t86_l04g18.png}}
//
%%% Local Variables:
%%% mode: latex
%%% TeX-engine: luatex
%%% TeX-master: shared
%%% End:

%//
%\glpismo%
 \glpismo
% 1
{\PTglyphid{Wi-12_0101}}
% 2
{\PTglyphid{Wi-12_0102}}
% 3
{\PTglyphid{Wi-12_0103}}
% 4
{\PTglyphid{Wi-12_0104}}
% 5
{\PTglyphid{Wi-12_0105}}
% 6
{\PTglyphid{Wi-12_0106}}
% 7
{\PTglyphid{Wi-12_0107}}
% 8
{\PTglyphid{Wi-12_0108}}
% 9
{\PTglyphid{Wi-12_0109}}
% 10
{\PTglyphid{Wi-12_0110}}
% 11
{\PTglyphid{Wi-12_0111}}
% 12
{\PTglyphid{Wi-12_0112}}
% 13
{\PTglyphid{Wi-12_0113}}
% 14
{\PTglyphid{Wi-12_0114}}
% 15
{\PTglyphid{Wi-12_0115}}
% 16
{\PTglyphid{Wi-12_0116}}
% 17
{\PTglyphid{Wi-12_0201}}
% 18
{\PTglyphid{Wi-12_0202}}
% 19
{\PTglyphid{Wi-12_0203}}
% 20
{\PTglyphid{Wi-12_0204}}
% 21
{\PTglyphid{Wi-12_0205}}
% 22
{\PTglyphid{Wi-12_0206}}
% 23
{\PTglyphid{Wi-12_0207}}
% 24
{\PTglyphid{Wi-12_0208}}
% 25
{\PTglyphid{Wi-12_0209}}
% 26
{\PTglyphid{Wi-12_0210}}
% 27
{\PTglyphid{Wi-12_0211}}
% 28
{\PTglyphid{Wi-12_0212}}
% 29
{\PTglyphid{Wi-12_0213}}
% 30
{\PTglyphid{Wi-12_0214}}
% 31
{\PTglyphid{Wi-12_0301}}
% 32
{\PTglyphid{Wi-12_0302}}
% 33
{\PTglyphid{Wi-12_0303}}
% 34
{\PTglyphid{Wi-12_0304}}
% 35
{\PTglyphid{Wi-12_0305}}
% 36
{\PTglyphid{Wi-12_0306}}
% 37
{\PTglyphid{Wi-12_0307}}
% 38
{\PTglyphid{Wi-12_0308}}
% 39
{\PTglyphid{Wi-12_0309}}
% 40
{\PTglyphid{Wi-12_0310}}
% 41
{\PTglyphid{Wi-12_0311}}
% 42
{\PTglyphid{Wi-12_0312}}
% 43
{\PTglyphid{Wi-12_0313}}
% 44
{\PTglyphid{Wi-12_0314}}
% 45
{\PTglyphid{Wi-12_0315}}
% 46
{\PTglyphid{Wi-12_0316}}
% 47
{\PTglyphid{Wi-12_0317}}
% 48
{\PTglyphid{Wi-12_0318}}
% 49
{\PTglyphid{Wi-12_0319}}
% 50
{\PTglyphid{Wi-12_0320}}
% 51
{\PTglyphid{Wi-12_0321}}
% 52
{\PTglyphid{Wi-12_0322}}
% 53
{\PTglyphid{Wi-12_0323}}
% 54
{\PTglyphid{Wi-12_0324}}
% 55
{\PTglyphid{Wi-12_0325}}
% 56
{\PTglyphid{Wi-12_0401}}
% 57
{\PTglyphid{Wi-12_0402}}
% 58
{\PTglyphid{Wi-12_0403}}
% 59
{\PTglyphid{Wi-12_0404}}
% 60
{\PTglyphid{Wi-12_0405}}
% 61
{\PTglyphid{Wi-12_0406}}
% 62
{\PTglyphid{Wi-12_0407}}
% 63
{\PTglyphid{Wi-12_0408}}
% 64
{\PTglyphid{Wi-12_0409}}
% 65
{\PTglyphid{Wi-12_0410}}
% 66
{\PTglyphid{Wi-12_0411}}
% 67
{\PTglyphid{Wi-12_0412}}
% 68
{\PTglyphid{Wi-12_0413}}
% 69
{\PTglyphid{Wi-12_0414}}
% 70
{\PTglyphid{Wi-12_0415}}
% 71
{\PTglyphid{Wi-12_0416}}
% 72
{\PTglyphid{Wi-12_0417}}
% 73
{\PTglyphid{Wi-12_0418}}
//
\endgl \xe
%%% Local Variables:
%%% mode: latex
%%% TeX-engine: luatex
%%% TeX-master: shared
%%% End:

% //
%\endgl \xe
 

 \newpage
 
%%%%%%%%%%%%%%%%%%%%%%%%%%%%%%%%%%%%%%%%%%%%%%%%%%%%%%%%%%%%%%%%%%%%%%%%%%%%%%%
% from meta.csv
% 87,Wirzbięta-13_PT11_565.djvu,Wirzbięta,13,11,565
% 
%%%%%%%%%%%%%%%%%%%%%%%%%%%%%%%%%%%%%%%%%%%%%%%%%%%%%%%%%%%%%%%%%%%%%%%%%%%%%%%

 
% from dsed4test:
% Wirzbięta-13_PT11_565_4dsed.txt:Note "13. Pismo komentarzowe, kursywa. Stopień 20 ww. = 76 mm. — Tabl. 565."
% Wirzbięta-13_PT11_565_4dsed.txt:Note1 "Character set table prepared by Anna Wolińska"

 \pismoPL{Maciej Wirzbięta 13. Pismo komentarzowe, kursywa. Stopień 20 ww. = 76 mm. — Tabl. 565 [pierwszy zestaw].}

 % \pismoPL{Maciej Wirzbięta 11a. Wersaliki tytułowe
 %   greckie, dorobione do p. 11. — Tabl. 564.}
  
 \pismoEN{Maciej Wirzbięta 13. Cursive comment font.  Type size 20
   lines = 76 mm. — Plate 565 [first set].}
   564.}

% \pismoEN{Maciej Wirzbięta 11a. Greek capital font supplementing font 11. Plates 564}


\plate{565[1]}{XI}{1981}

The plate prepared by Alodia Kawecka Gryczowa.\\
The font table prepared by Alodia Kawecka Gryczowa and Anna Wolińska}

\bigskip

% \exampleBib{IX:17}

% \bigskip \exampleDesc{ANDRZEJ FRYCZ MODRZEWSKI: Orichovius sive depulsio calumniarum Stanislai Orichovii.
% [Kraków, Maciej Wirzbięta, po 24 XII 1562]. 8⁰.}


%   \medskip
%   \examplePage{\textit{Karta F₂a}}

% \exampleLib{Biblioteka Narodowa. Warszawa.}
  
  % \bigskip \exampleRef{\textit{Estreicher Estreicher XXII 491. Andrzej Frycz Modrzewski. Bibliografia I 26.}

  % \bigskip

  % \exampleDig{\url{https://www.wbc.poznan.pl/dlibra/publication/494486} page 87}

% Pismo 13: zestaw pierwszy. — Pismo 14: kolumna i zestaw drugi. — Przerywnik 25% pod zestawem pisma 13.



\bigskip

\fontID{Wi-13}{87}

\fontstat{62}

% \exdisplay \bg \gla
 \exdisplay \bg \gla
% 1
{\PTglyph{5}{t87_l01g01.png}}
% 2
{\PTglyph{5}{t87_l01g02.png}}
% 3
{\PTglyph{5}{t87_l01g03.png}}
% 4
{\PTglyph{5}{t87_l01g04.png}}
% 5
{\PTglyph{5}{t87_l01g05.png}}
% 6
{\PTglyph{5}{t87_l01g06.png}}
% 7
{\PTglyph{5}{t87_l01g07.png}}
% 8
{\PTglyph{5}{t87_l01g08.png}}
% 9
{\PTglyph{5}{t87_l01g09.png}}
% 10
{\PTglyph{5}{t87_l01g10.png}}
% 11
{\PTglyph{5}{t87_l01g11.png}}
% 12
{\PTglyph{5}{t87_l01g12.png}}
% 13
{\PTglyph{5}{t87_l01g13.png}}
% 14
{\PTglyph{5}{t87_l01g14.png}}
% 15
{\PTglyph{5}{t87_l01g15.png}}
% 16
{\PTglyph{5}{t87_l01g16.png}}
% 17
{\PTglyph{5}{t87_l01g17.png}}
% 18
{\PTglyph{5}{t87_l01g18.png}}
% 19
{\PTglyph{5}{t87_l01g19.png}}
% 20
{\PTglyph{5}{t87_l01g20.png}}
% 21
{\PTglyph{5}{t87_l02g01.png}}
% 22
{\PTglyph{5}{t87_l02g02.png}}
% 23
{\PTglyph{5}{t87_l02g03.png}}
% 24
{\PTglyph{5}{t87_l02g04.png}}
% 25
{\PTglyph{5}{t87_l02g05.png}}
% 26
{\PTglyph{5}{t87_l02g06.png}}
% 27
{\PTglyph{5}{t87_l02g07.png}}
% 28
{\PTglyph{5}{t87_l02g08.png}}
% 29
{\PTglyph{5}{t87_l02g09.png}}
% 30
{\PTglyph{5}{t87_l02g10.png}}
% 31
{\PTglyph{5}{t87_l02g11.png}}
% 32
{\PTglyph{5}{t87_l02g12.png}}
% 33
{\PTglyph{5}{t87_l02g13.png}}
% 34
{\PTglyph{5}{t87_l02g14.png}}
% 35
{\PTglyph{5}{t87_l02g15.png}}
% 36
{\PTglyph{5}{t87_l02g16.png}}
% 37
{\PTglyph{5}{t87_l02g17.png}}
% 38
{\PTglyph{5}{t87_l02g18.png}}
% 39
{\PTglyph{5}{t87_l02g19.png}}
% 40
{\PTglyph{5}{t87_l02g20.png}}
% 41
{\PTglyph{5}{t87_l02g21.png}}
% 42
{\PTglyph{5}{t87_l02g22.png}}
% 43
{\PTglyph{5}{t87_l02g23.png}}
% 44
{\PTglyph{5}{t87_l02g24.png}}
% 45
{\PTglyph{5}{t87_l02g25.png}}
% 46
{\PTglyph{5}{t87_l02g26.png}}
% 47
{\PTglyph{5}{t87_l02g27.png}}
% 48
{\PTglyph{5}{t87_l02g28.png}}
% 49
{\PTglyph{5}{t87_l02g29.png}}
% 50
{\PTglyph{5}{t87_l02g30.png}}
% 51
{\PTglyph{5}{t87_l02g31.png}}
% 52
{\PTglyph{5}{t87_l02g32.png}}
% 53
{\PTglyph{5}{t87_l02g33.png}}
% 54
{\PTglyph{5}{t87_l02g34.png}}
% 55
{\PTglyph{5}{t87_l02g35.png}}
% 56
{\PTglyph{5}{t87_l03g01.png}}
% 57
{\PTglyph{5}{t87_l03g02.png}}
% 58
{\PTglyph{5}{t87_l03g03.png}}
% 59
{\PTglyph{5}{t87_l03g04.png}}
% 60
{\PTglyph{5}{t87_l03g05.png}}
% 61
{\PTglyph{5}{t87_l03g06.png}}
% 62
{\PTglyph{5}{t87_l04g01.png}}
//
%%% Local Variables:
%%% mode: latex
%%% TeX-engine: luatex
%%% TeX-master: shared
%%% End:

%//
%\glpismo%
 \glpismo
% 1
{\PTglyphid{Wi-13_0101}}
% 2
{\PTglyphid{Wi-13_0102}}
% 3
{\PTglyphid{Wi-13_0103}}
% 4
{\PTglyphid{Wi-13_0104}}
% 5
{\PTglyphid{Wi-13_0105}}
% 6
{\PTglyphid{Wi-13_0106}}
% 7
{\PTglyphid{Wi-13_0107}}
% 8
{\PTglyphid{Wi-13_0108}}
% 9
{\PTglyphid{Wi-13_0109}}
% 10
{\PTglyphid{Wi-13_0110}}
% 11
{\PTglyphid{Wi-13_0111}}
% 12
{\PTglyphid{Wi-13_0112}}
% 13
{\PTglyphid{Wi-13_0113}}
% 14
{\PTglyphid{Wi-13_0114}}
% 15
{\PTglyphid{Wi-13_0115}}
% 16
{\PTglyphid{Wi-13_0116}}
% 17
{\PTglyphid{Wi-13_0117}}
% 18
{\PTglyphid{Wi-13_0118}}
% 19
{\PTglyphid{Wi-13_0119}}
% 20
{\PTglyphid{Wi-13_0120}}
% 21
{\PTglyphid{Wi-13_0201}}
% 22
{\PTglyphid{Wi-13_0202}}
% 23
{\PTglyphid{Wi-13_0203}}
% 24
{\PTglyphid{Wi-13_0204}}
% 25
{\PTglyphid{Wi-13_0205}}
% 26
{\PTglyphid{Wi-13_0206}}
% 27
{\PTglyphid{Wi-13_0207}}
% 28
{\PTglyphid{Wi-13_0208}}
% 29
{\PTglyphid{Wi-13_0209}}
% 30
{\PTglyphid{Wi-13_0210}}
% 31
{\PTglyphid{Wi-13_0211}}
% 32
{\PTglyphid{Wi-13_0212}}
% 33
{\PTglyphid{Wi-13_0213}}
% 34
{\PTglyphid{Wi-13_0214}}
% 35
{\PTglyphid{Wi-13_0215}}
% 36
{\PTglyphid{Wi-13_0216}}
% 37
{\PTglyphid{Wi-13_0217}}
% 38
{\PTglyphid{Wi-13_0218}}
% 39
{\PTglyphid{Wi-13_0219}}
% 40
{\PTglyphid{Wi-13_0220}}
% 41
{\PTglyphid{Wi-13_0221}}
% 42
{\PTglyphid{Wi-13_0222}}
% 43
{\PTglyphid{Wi-13_0223}}
% 44
{\PTglyphid{Wi-13_0224}}
% 45
{\PTglyphid{Wi-13_0225}}
% 46
{\PTglyphid{Wi-13_0226}}
% 47
{\PTglyphid{Wi-13_0227}}
% 48
{\PTglyphid{Wi-13_0228}}
% 49
{\PTglyphid{Wi-13_0229}}
% 50
{\PTglyphid{Wi-13_0230}}
% 51
{\PTglyphid{Wi-13_0231}}
% 52
{\PTglyphid{Wi-13_0232}}
% 53
{\PTglyphid{Wi-13_0233}}
% 54
{\PTglyphid{Wi-13_0234}}
% 55
{\PTglyphid{Wi-13_0235}}
% 56
{\PTglyphid{Wi-13_0301}}
% 57
{\PTglyphid{Wi-13_0302}}
% 58
{\PTglyphid{Wi-13_0303}}
% 59
{\PTglyphid{Wi-13_0304}}
% 60
{\PTglyphid{Wi-13_0305}}
% 61
{\PTglyphid{Wi-13_0306}}
% 62
{\PTglyphid{Wi-13_0401}}
//
\endgl \xe
%%% Local Variables:
%%% mode: latex
%%% TeX-engine: luatex
%%% TeX-master: shared
%%% End:

% //


 \newpage
 
%%%%%%%%%%%%%%%%%%%%%%%%%%%%%%%%%%%%%%%%%%%%%%%%%%%%%%%%%%%%%%%%%%%%%%%%%%%%%%%
% from meta.csv
% 88,Wirzbięta-14_PT11_565.djvu,Wirzbięta,14,11,565
% 
%%%%%%%%%%%%%%%%%%%%%%%%%%%%%%%%%%%%%%%%%%%%%%%%%%%%%%%%%%%%%%%%%%%%%%%%%%%%%%%

 
% from dsed4test:
% Wirzbięta-14_PT11_565_4dsed.txt:Note "14. Pismo tekstowe, kursywa. Stopień 20 ww. = 88/89 mm. — Tabl. 487, 493, 494, 521-523, 540, 561, 565, 568. [565]"
% Wirzbięta-14_PT11_565_4dsed.txt:Note1 "Character set table prepared by Anna Wolińska"

 \pismoPL{Maciej Wirzbięta 14. Pismo tekstowe, kursywa. Stopień 20
   ww. = 88/89 mm. — Tabl. 487, 493, 494, 521-523, 540, 561, 565,
   568. [Tabl. 565 drugi zestaw].}

 % \pismoPL{Maciej Wirzbięta 11a. Wersaliki tytułowe
 %   greckie, dorobione do p. 11. — Tabl. 564.}
  
 \pismoEN{Maciej Wirzbięta 14. Cursive text font.  Type size 20 lines
   = 88/89 mm. — Plate 487, 493, 494, 521-523, 540, 561, 565, 568
   [plate 565 second set].}


% \pismoEN{Maciej Wirzbięta 11a. Greek capital font supplementing font 11. Plates 564}


\plate{565[2]}{XI}{1981}

The plate prepared by Alodia Kawecka Gryczowa.\\
The font table prepared by Alodia Kawecka Gryczowa and Anna Wolińska.

\bigskip

\exampleBib{IX:17}

\bigskip \exampleDesc{ANDRZEJ FRYCZ MODRZEWSKI: Orichovius sive depulsio calumniarum Stanislai Orichovii.
[Kraków, Maciej Wirzbięta, po 24 XII 1562]. 8⁰.}


  \medskip
  \examplePage{\textit{Karta F₂a}}

\exampleLib{Biblioteka Narodowa. Warszawa.}
  
  \bigskip \exampleRef{\textit{Estreicher Estreicher XXII 491. Andrzej Frycz Modrzewski. Bibliografia I 26.}

  \bigskip

  \exampleDig{\url{https://www.wbc.poznan.pl/dlibra/publication/494486} page 87}

% Pismo 13: zestaw pierwszy. — Pismo 14: kolumna i zestaw drugi. — Przerywnik 25% pod zestawem pisma 13.



\bigskip

\fontID{Wi-14}{88}

\fontstat{133}

% \exdisplay \bg \gla
 \exdisplay \bg \gla
% 1
{\PTglyph{5}{t88_l01g01.png}}
% 2
{\PTglyph{5}{t88_l01g02.png}}
% 3
{\PTglyph{5}{t88_l01g03.png}}
% 4
{\PTglyph{5}{t88_l01g04.png}}
% 5
{\PTglyph{5}{t88_l01g05.png}}
% 6
{\PTglyph{5}{t88_l01g06.png}}
% 7
{\PTglyph{5}{t88_l01g07.png}}
% 8
{\PTglyph{5}{t88_l01g08.png}}
% 9
{\PTglyph{5}{t88_l01g09.png}}
% 10
{\PTglyph{5}{t88_l01g10.png}}
% 11
{\PTglyph{5}{t88_l01g11.png}}
% 12
{\PTglyph{5}{t88_l01g12.png}}
% 13
{\PTglyph{5}{t88_l01g13.png}}
% 14
{\PTglyph{5}{t88_l01g14.png}}
% 15
{\PTglyph{5}{t88_l01g15.png}}
% 16
{\PTglyph{5}{t88_l01g16.png}}
% 17
{\PTglyph{5}{t88_l01g17.png}}
% 18
{\PTglyph{5}{t88_l01g18.png}}
% 19
{\PTglyph{5}{t88_l01g19.png}}
% 20
{\PTglyph{5}{t88_l01g20.png}}
% 21
{\PTglyph{5}{t88_l01g21.png}}
% 22
{\PTglyph{5}{t88_l01g22.png}}
% 23
{\PTglyph{5}{t88_l01g23.png}}
% 24
{\PTglyph{5}{t88_l01g24.png}}
% 25
{\PTglyph{5}{t88_l01g25.png}}
% 26
{\PTglyph{5}{t88_l01g26.png}}
% 27
{\PTglyph{5}{t88_l02g01.png}}
% 28
{\PTglyph{5}{t88_l02g02.png}}
% 29
{\PTglyph{5}{t88_l02g03.png}}
% 30
{\PTglyph{5}{t88_l02g04.png}}
% 31
{\PTglyph{5}{t88_l02g05.png}}
% 32
{\PTglyph{5}{t88_l02g06.png}}
% 33
{\PTglyph{5}{t88_l02g07.png}}
% 34
{\PTglyph{5}{t88_l02g08.png}}
% 35
{\PTglyph{5}{t88_l02g09.png}}
% 36
{\PTglyph{5}{t88_l02g10.png}}
% 37
{\PTglyph{5}{t88_l02g11.png}}
% 38
{\PTglyph{5}{t88_l02g12.png}}
% 39
{\PTglyph{5}{t88_l02g13.png}}
% 40
{\PTglyph{5}{t88_l02g14.png}}
% 41
{\PTglyph{5}{t88_l02g15.png}}
% 42
{\PTglyph{5}{t88_l02g16.png}}
% 43
{\PTglyph{5}{t88_l02g17.png}}
% 44
{\PTglyph{5}{t88_l02g18.png}}
% 45
{\PTglyph{5}{t88_l02g19.png}}
% 46
{\PTglyph{5}{t88_l02g20.png}}
% 47
{\PTglyph{5}{t88_l02g21.png}}
% 48
{\PTglyph{5}{t88_l02g22.png}}
% 49
{\PTglyph{5}{t88_l02g23.png}}
% 50
{\PTglyph{5}{t88_l02g24.png}}
% 51
{\PTglyph{5}{t88_l02g25.png}}
% 52
{\PTglyph{5}{t88_l02g26.png}}
% 53
{\PTglyph{5}{t88_l02g27.png}}
% 54
{\PTglyph{5}{t88_l02g28.png}}
% 55
{\PTglyph{5}{t88_l02g29.png}}
% 56
{\PTglyph{5}{t88_l02g30.png}}
% 57
{\PTglyph{5}{t88_l02g31.png}}
% 58
{\PTglyph{5}{t88_l02g32.png}}
% 59
{\PTglyph{5}{t88_l02g33.png}}
% 60
{\PTglyph{5}{t88_l02g34.png}}
% 61
{\PTglyph{5}{t88_l02g35.png}}
% 62
{\PTglyph{5}{t88_l02g36.png}}
% 63
{\PTglyph{5}{t88_l02g37.png}}
% 64
{\PTglyph{5}{t88_l02g38.png}}
% 65
{\PTglyph{5}{t88_l02g39.png}}
% 66
{\PTglyph{5}{t88_l03g01.png}}
% 67
{\PTglyph{5}{t88_l03g02.png}}
% 68
{\PTglyph{5}{t88_l03g03.png}}
% 69
{\PTglyph{5}{t88_l03g04.png}}
% 70
{\PTglyph{5}{t88_l03g05.png}}
% 71
{\PTglyph{5}{t88_l03g06.png}}
% 72
{\PTglyph{5}{t88_l03g07.png}}
% 73
{\PTglyph{5}{t88_l03g08.png}}
% 74
{\PTglyph{5}{t88_l03g09.png}}
% 75
{\PTglyph{5}{t88_l03g10.png}}
% 76
{\PTglyph{5}{t88_l03g11.png}}
% 77
{\PTglyph{5}{t88_l03g12.png}}
% 78
{\PTglyph{5}{t88_l03g13.png}}
% 79
{\PTglyph{5}{t88_l03g14.png}}
% 80
{\PTglyph{5}{t88_l03g15.png}}
% 81
{\PTglyph{5}{t88_l03g16.png}}
% 82
{\PTglyph{5}{t88_l03g17.png}}
% 83
{\PTglyph{5}{t88_l03g18.png}}
% 84
{\PTglyph{5}{t88_l03g19.png}}
% 85
{\PTglyph{5}{t88_l03g20.png}}
% 86
{\PTglyph{5}{t88_l03g21.png}}
% 87
{\PTglyph{5}{t88_l03g22.png}}
% 88
{\PTglyph{5}{t88_l03g23.png}}
% 89
{\PTglyph{5}{t88_l03g24.png}}
% 90
{\PTglyph{5}{t88_l03g25.png}}
% 91
{\PTglyph{5}{t88_l03g26.png}}
% 92
{\PTglyph{5}{t88_l03g27.png}}
% 93
{\PTglyph{5}{t88_l03g28.png}}
% 94
{\PTglyph{5}{t88_l03g29.png}}
% 95
{\PTglyph{5}{t88_l03g30.png}}
% 96
{\PTglyph{5}{t88_l03g31.png}}
% 97
{\PTglyph{5}{t88_l03g32.png}}
% 98
{\PTglyph{5}{t88_l03g33.png}}
% 99
{\PTglyph{5}{t88_l03g34.png}}
% 100
{\PTglyph{5}{t88_l03g35.png}}
% 101
{\PTglyph{5}{t88_l03g36.png}}
% 102
{\PTglyph{5}{t88_l03g37.png}}
% 103
{\PTglyph{5}{t88_l04g01.png}}
% 104
{\PTglyph{5}{t88_l04g02.png}}
% 105
{\PTglyph{5}{t88_l04g03.png}}
% 106
{\PTglyph{5}{t88_l04g04.png}}
% 107
{\PTglyph{5}{t88_l04g05.png}}
% 108
{\PTglyph{5}{t88_l04g06.png}}
% 109
{\PTglyph{5}{t88_l04g07.png}}
% 110
{\PTglyph{5}{t88_l04g08.png}}
% 111
{\PTglyph{5}{t88_l04g09.png}}
% 112
{\PTglyph{5}{t88_l04g10.png}}
% 113
{\PTglyph{5}{t88_l04g11.png}}
% 114
{\PTglyph{5}{t88_l04g12.png}}
% 115
{\PTglyph{5}{t88_l04g13.png}}
% 116
{\PTglyph{5}{t88_l04g14.png}}
% 117
{\PTglyph{5}{t88_l04g15.png}}
% 118
{\PTglyph{5}{t88_l04g16.png}}
% 119
{\PTglyph{5}{t88_l04g17.png}}
% 120
{\PTglyph{5}{t88_l04g18.png}}
% 121
{\PTglyph{5}{t88_l04g19.png}}
% 122
{\PTglyph{5}{t88_l04g20.png}}
% 123
{\PTglyph{5}{t88_l04g21.png}}
% 124
{\PTglyph{5}{t88_l04g22.png}}
% 125
{\PTglyph{5}{t88_l04g23.png}}
% 126
{\PTglyph{5}{t88_l04g24.png}}
% 127
{\PTglyph{5}{t88_l04g25.png}}
% 128
{\PTglyph{5}{t88_l04g26.png}}
% 129
{\PTglyph{5}{t88_l04g27.png}}
% 130
{\PTglyph{5}{t88_l04g28.png}}
% 131
{\PTglyph{5}{t88_l04g29.png}}
% 132
{\PTglyph{5}{t88_l04g30.png}}
% 133
{\PTglyph{5}{t88_l04g31.png}}
//
%%% Local Variables:
%%% mode: latex
%%% TeX-engine: luatex
%%% TeX-master: shared
%%% End:

%//
%\glpismo%
 \glpismo
% 1
{\PTglyphid{Wi-14_0101}}
% 2
{\PTglyphid{Wi-14_0102}}
% 3
{\PTglyphid{Wi-14_0103}}
% 4
{\PTglyphid{Wi-14_0104}}
% 5
{\PTglyphid{Wi-14_0105}}
% 6
{\PTglyphid{Wi-14_0106}}
% 7
{\PTglyphid{Wi-14_0107}}
% 8
{\PTglyphid{Wi-14_0108}}
% 9
{\PTglyphid{Wi-14_0109}}
% 10
{\PTglyphid{Wi-14_0110}}
% 11
{\PTglyphid{Wi-14_0111}}
% 12
{\PTglyphid{Wi-14_0112}}
% 13
{\PTglyphid{Wi-14_0113}}
% 14
{\PTglyphid{Wi-14_0114}}
% 15
{\PTglyphid{Wi-14_0115}}
% 16
{\PTglyphid{Wi-14_0116}}
% 17
{\PTglyphid{Wi-14_0117}}
% 18
{\PTglyphid{Wi-14_0118}}
% 19
{\PTglyphid{Wi-14_0119}}
% 20
{\PTglyphid{Wi-14_0120}}
% 21
{\PTglyphid{Wi-14_0121}}
% 22
{\PTglyphid{Wi-14_0122}}
% 23
{\PTglyphid{Wi-14_0123}}
% 24
{\PTglyphid{Wi-14_0124}}
% 25
{\PTglyphid{Wi-14_0125}}
% 26
{\PTglyphid{Wi-14_0126}}
% 27
{\PTglyphid{Wi-14_0201}}
% 28
{\PTglyphid{Wi-14_0202}}
% 29
{\PTglyphid{Wi-14_0203}}
% 30
{\PTglyphid{Wi-14_0204}}
% 31
{\PTglyphid{Wi-14_0205}}
% 32
{\PTglyphid{Wi-14_0206}}
% 33
{\PTglyphid{Wi-14_0207}}
% 34
{\PTglyphid{Wi-14_0208}}
% 35
{\PTglyphid{Wi-14_0209}}
% 36
{\PTglyphid{Wi-14_0210}}
% 37
{\PTglyphid{Wi-14_0211}}
% 38
{\PTglyphid{Wi-14_0212}}
% 39
{\PTglyphid{Wi-14_0213}}
% 40
{\PTglyphid{Wi-14_0214}}
% 41
{\PTglyphid{Wi-14_0215}}
% 42
{\PTglyphid{Wi-14_0216}}
% 43
{\PTglyphid{Wi-14_0217}}
% 44
{\PTglyphid{Wi-14_0218}}
% 45
{\PTglyphid{Wi-14_0219}}
% 46
{\PTglyphid{Wi-14_0220}}
% 47
{\PTglyphid{Wi-14_0221}}
% 48
{\PTglyphid{Wi-14_0222}}
% 49
{\PTglyphid{Wi-14_0223}}
% 50
{\PTglyphid{Wi-14_0224}}
% 51
{\PTglyphid{Wi-14_0225}}
% 52
{\PTglyphid{Wi-14_0226}}
% 53
{\PTglyphid{Wi-14_0227}}
% 54
{\PTglyphid{Wi-14_0228}}
% 55
{\PTglyphid{Wi-14_0229}}
% 56
{\PTglyphid{Wi-14_0230}}
% 57
{\PTglyphid{Wi-14_0231}}
% 58
{\PTglyphid{Wi-14_0232}}
% 59
{\PTglyphid{Wi-14_0233}}
% 60
{\PTglyphid{Wi-14_0234}}
% 61
{\PTglyphid{Wi-14_0235}}
% 62
{\PTglyphid{Wi-14_0236}}
% 63
{\PTglyphid{Wi-14_0237}}
% 64
{\PTglyphid{Wi-14_0238}}
% 65
{\PTglyphid{Wi-14_0239}}
% 66
{\PTglyphid{Wi-14_0301}}
% 67
{\PTglyphid{Wi-14_0302}}
% 68
{\PTglyphid{Wi-14_0303}}
% 69
{\PTglyphid{Wi-14_0304}}
% 70
{\PTglyphid{Wi-14_0305}}
% 71
{\PTglyphid{Wi-14_0306}}
% 72
{\PTglyphid{Wi-14_0307}}
% 73
{\PTglyphid{Wi-14_0308}}
% 74
{\PTglyphid{Wi-14_0309}}
% 75
{\PTglyphid{Wi-14_0310}}
% 76
{\PTglyphid{Wi-14_0311}}
% 77
{\PTglyphid{Wi-14_0312}}
% 78
{\PTglyphid{Wi-14_0313}}
% 79
{\PTglyphid{Wi-14_0314}}
% 80
{\PTglyphid{Wi-14_0315}}
% 81
{\PTglyphid{Wi-14_0316}}
% 82
{\PTglyphid{Wi-14_0317}}
% 83
{\PTglyphid{Wi-14_0318}}
% 84
{\PTglyphid{Wi-14_0319}}
% 85
{\PTglyphid{Wi-14_0320}}
% 86
{\PTglyphid{Wi-14_0321}}
% 87
{\PTglyphid{Wi-14_0322}}
% 88
{\PTglyphid{Wi-14_0323}}
% 89
{\PTglyphid{Wi-14_0324}}
% 90
{\PTglyphid{Wi-14_0325}}
% 91
{\PTglyphid{Wi-14_0326}}
% 92
{\PTglyphid{Wi-14_0327}}
% 93
{\PTglyphid{Wi-14_0328}}
% 94
{\PTglyphid{Wi-14_0329}}
% 95
{\PTglyphid{Wi-14_0330}}
% 96
{\PTglyphid{Wi-14_0331}}
% 97
{\PTglyphid{Wi-14_0332}}
% 98
{\PTglyphid{Wi-14_0333}}
% 99
{\PTglyphid{Wi-14_0334}}
% 100
{\PTglyphid{Wi-14_0335}}
% 101
{\PTglyphid{Wi-14_0336}}
% 102
{\PTglyphid{Wi-14_0337}}
% 103
{\PTglyphid{Wi-14_0401}}
% 104
{\PTglyphid{Wi-14_0402}}
% 105
{\PTglyphid{Wi-14_0403}}
% 106
{\PTglyphid{Wi-14_0404}}
% 107
{\PTglyphid{Wi-14_0405}}
% 108
{\PTglyphid{Wi-14_0406}}
% 109
{\PTglyphid{Wi-14_0407}}
% 110
{\PTglyphid{Wi-14_0408}}
% 111
{\PTglyphid{Wi-14_0409}}
% 112
{\PTglyphid{Wi-14_0410}}
% 113
{\PTglyphid{Wi-14_0411}}
% 114
{\PTglyphid{Wi-14_0412}}
% 115
{\PTglyphid{Wi-14_0413}}
% 116
{\PTglyphid{Wi-14_0414}}
% 117
{\PTglyphid{Wi-14_0415}}
% 118
{\PTglyphid{Wi-14_0416}}
% 119
{\PTglyphid{Wi-14_0417}}
% 120
{\PTglyphid{Wi-14_0418}}
% 121
{\PTglyphid{Wi-14_0419}}
% 122
{\PTglyphid{Wi-14_0420}}
% 123
{\PTglyphid{Wi-14_0421}}
% 124
{\PTglyphid{Wi-14_0422}}
% 125
{\PTglyphid{Wi-14_0423}}
% 126
{\PTglyphid{Wi-14_0424}}
% 127
{\PTglyphid{Wi-14_0425}}
% 128
{\PTglyphid{Wi-14_0426}}
% 129
{\PTglyphid{Wi-14_0427}}
% 130
{\PTglyphid{Wi-14_0428}}
% 131
{\PTglyphid{Wi-14_0429}}
% 132
{\PTglyphid{Wi-14_0430}}
% 133
{\PTglyphid{Wi-14_0431}}
//
\endgl \xe
%%% Local Variables:
%%% mode: latex
%%% TeX-engine: luatex
%%% TeX-master: shared
%%% End:

% //
%\endgl \xe
 

 \newpage
 
%%%%%%%%%%%%%%%%%%%%%%%%%%%%%%%%%%%%%%%%%%%%%%%%%%%%%%%%%%%%%%%%%%%%%%%%%%%%%%%
% from meta.csv
% 89,Wirzbięta-15_PT11_566.djvu,Wirzbięta,15,11,566
% 
%%%%%%%%%%%%%%%%%%%%%%%%%%%%%%%%%%%%%%%%%%%%%%%%%%%%%%%%%%%%%%%%%%%%%%%%%%%%%%%

 
% from dsed4test:
% Wirzbięta-15_PT11_566_4dsed.txt:Note "15. Pismo tekstowe, kursywa. Stopień 20 ww. = 113/114. mm. — Tabl. 494, 514, 521, 522, 566, 567. [566]"
% Wirzbięta-15_PT11_566_4dsed.txt:Note1 "Character set table prepared by Anna Wolińska"

 \pismoPL{Maciej Wirzbięta 15. Pismo tekstowe, kursywa. Stopień 20 ww. = 113/114. mm. — Tabl. 494, 514, 521, 522, 566, 567.}

 % \pismoPL{Maciej Wirzbięta 11a. Wersaliki tytułowe
 %   greckie, dorobione do p. 11. — Tabl. 564.}
  
 \pismoEN{Maciej Wirzbięta 14. Cursive text font.  Type size 20 lines
   = 113/114 mm. — Plates 494, 514, 521, 522, 566, 567.}


% \pismoEN{Maciej Wirzbięta 11a. Greek capital font supplementing font 11. Plates 564}


\plate{566}{XI}{1981}

The plate prepared by Alodia Kawecka Gryczowa.\\
The font table prepared by Alodia Kawecka Gryczowa and Anna Wolińska}

\bigskip

\exampleBib{IX:77}

\bigskip \exampleDesc{ANDRZEJ WOLAN: De libertate politica sive
  civili. — \textit{Acc}. Augustinus Rotundus [Mieleski]: [Epistola]
  Generoso Domino Andreae Volano... Dat. 10 XII 1571 etc. Kraków,
  Maciej Wirzbięta, [po 5 1] 1572. 4⁰.}


  \medskip
  \examplePage{\textit{Karta F₂a}}

\exampleLib{Biblioteka Jagiellońska. Kraków.}
  
  \bigskip \exampleRef{\textit{Estreicher XXXIII 245.}}

  \bigskip

    \exampleDig{\url{https://www.dbc.wroc.pl/dlibra/doccontent?id=7489} page  ???}


\bigskip

\fontID{Wi-15}{89}

\fontstat{129}

% \exdisplay \bg \gla
 \exdisplay \bg \gla
% 1
{\PTglyph{5}{t89_l01g01.png}}
% 2
{\PTglyph{5}{t89_l01g02.png}}
% 3
{\PTglyph{5}{t89_l01g03.png}}
% 4
{\PTglyph{5}{t89_l01g04.png}}
% 5
{\PTglyph{5}{t89_l01g05.png}}
% 6
{\PTglyph{5}{t89_l01g06.png}}
% 7
{\PTglyph{5}{t89_l01g07.png}}
% 8
{\PTglyph{5}{t89_l01g08.png}}
% 9
{\PTglyph{5}{t89_l01g09.png}}
% 10
{\PTglyph{5}{t89_l01g10.png}}
% 11
{\PTglyph{5}{t89_l01g11.png}}
% 12
{\PTglyph{5}{t89_l01g12.png}}
% 13
{\PTglyph{5}{t89_l01g13.png}}
% 14
{\PTglyph{5}{t89_l01g14.png}}
% 15
{\PTglyph{5}{t89_l01g15.png}}
% 16
{\PTglyph{5}{t89_l01g16.png}}
% 17
{\PTglyph{5}{t89_l01g17.png}}
% 18
{\PTglyph{5}{t89_l01g18.png}}
% 19
{\PTglyph{5}{t89_l01g19.png}}
% 20
{\PTglyph{5}{t89_l01g20.png}}
% 21
{\PTglyph{5}{t89_l01g21.png}}
% 22
{\PTglyph{5}{t89_l01g22.png}}
% 23
{\PTglyph{5}{t89_l01g23.png}}
% 24
{\PTglyph{5}{t89_l01g24.png}}
% 25
{\PTglyph{5}{t89_l01g25.png}}
% 26
{\PTglyph{5}{t89_l02g01.png}}
% 27
{\PTglyph{5}{t89_l02g02.png}}
% 28
{\PTglyph{5}{t89_l02g03.png}}
% 29
{\PTglyph{5}{t89_l02g04.png}}
% 30
{\PTglyph{5}{t89_l02g05.png}}
% 31
{\PTglyph{5}{t89_l02g06.png}}
% 32
{\PTglyph{5}{t89_l02g07.png}}
% 33
{\PTglyph{5}{t89_l02g08.png}}
% 34
{\PTglyph{5}{t89_l02g09.png}}
% 35
{\PTglyph{5}{t89_l02g10.png}}
% 36
{\PTglyph{5}{t89_l02g11.png}}
% 37
{\PTglyph{5}{t89_l02g12.png}}
% 38
{\PTglyph{5}{t89_l02g13.png}}
% 39
{\PTglyph{5}{t89_l02g14.png}}
% 40
{\PTglyph{5}{t89_l02g15.png}}
% 41
{\PTglyph{5}{t89_l02g16.png}}
% 42
{\PTglyph{5}{t89_l02g17.png}}
% 43
{\PTglyph{5}{t89_l02g18.png}}
% 44
{\PTglyph{5}{t89_l02g19.png}}
% 45
{\PTglyph{5}{t89_l02g20.png}}
% 46
{\PTglyph{5}{t89_l02g21.png}}
% 47
{\PTglyph{5}{t89_l02g22.png}}
% 48
{\PTglyph{5}{t89_l02g23.png}}
% 49
{\PTglyph{5}{t89_l02g24.png}}
% 50
{\PTglyph{5}{t89_l02g25.png}}
% 51
{\PTglyph{5}{t89_l02g26.png}}
% 52
{\PTglyph{5}{t89_l02g27.png}}
% 53
{\PTglyph{5}{t89_l02g28.png}}
% 54
{\PTglyph{5}{t89_l02g29.png}}
% 55
{\PTglyph{5}{t89_l02g30.png}}
% 56
{\PTglyph{5}{t89_l02g31.png}}
% 57
{\PTglyph{5}{t89_l02g32.png}}
% 58
{\PTglyph{5}{t89_l02g33.png}}
% 59
{\PTglyph{5}{t89_l02g34.png}}
% 60
{\PTglyph{5}{t89_l03g01.png}}
% 61
{\PTglyph{5}{t89_l03g02.png}}
% 62
{\PTglyph{5}{t89_l03g03.png}}
% 63
{\PTglyph{5}{t89_l03g04.png}}
% 64
{\PTglyph{5}{t89_l03g05.png}}
% 65
{\PTglyph{5}{t89_l03g06.png}}
% 66
{\PTglyph{5}{t89_l03g07.png}}
% 67
{\PTglyph{5}{t89_l03g08.png}}
% 68
{\PTglyph{5}{t89_l03g09.png}}
% 69
{\PTglyph{5}{t89_l03g10.png}}
% 70
{\PTglyph{5}{t89_l03g11.png}}
% 71
{\PTglyph{5}{t89_l03g12.png}}
% 72
{\PTglyph{5}{t89_l03g13.png}}
% 73
{\PTglyph{5}{t89_l03g14.png}}
% 74
{\PTglyph{5}{t89_l03g15.png}}
% 75
{\PTglyph{5}{t89_l03g16.png}}
% 76
{\PTglyph{5}{t89_l03g17.png}}
% 77
{\PTglyph{5}{t89_l03g18.png}}
% 78
{\PTglyph{5}{t89_l03g19.png}}
% 79
{\PTglyph{5}{t89_l03g20.png}}
% 80
{\PTglyph{5}{t89_l03g21.png}}
% 81
{\PTglyph{5}{t89_l03g22.png}}
% 82
{\PTglyph{5}{t89_l03g23.png}}
% 83
{\PTglyph{5}{t89_l03g24.png}}
% 84
{\PTglyph{5}{t89_l03g25.png}}
% 85
{\PTglyph{5}{t89_l03g26.png}}
% 86
{\PTglyph{5}{t89_l03g27.png}}
% 87
{\PTglyph{5}{t89_l03g28.png}}
% 88
{\PTglyph{5}{t89_l03g29.png}}
% 89
{\PTglyph{5}{t89_l03g30.png}}
% 90
{\PTglyph{5}{t89_l03g31.png}}
% 91
{\PTglyph{5}{t89_l04g01.png}}
% 92
{\PTglyph{5}{t89_l04g02.png}}
% 93
{\PTglyph{5}{t89_l04g03.png}}
% 94
{\PTglyph{5}{t89_l04g04.png}}
% 95
{\PTglyph{5}{t89_l04g05.png}}
% 96
{\PTglyph{5}{t89_l04g06.png}}
% 97
{\PTglyph{5}{t89_l04g07.png}}
% 98
{\PTglyph{5}{t89_l04g08.png}}
% 99
{\PTglyph{5}{t89_l04g09.png}}
% 100
{\PTglyph{5}{t89_l04g10.png}}
% 101
{\PTglyph{5}{t89_l04g11.png}}
% 102
{\PTglyph{5}{t89_l04g12.png}}
% 103
{\PTglyph{5}{t89_l04g13.png}}
% 104
{\PTglyph{5}{t89_l04g14.png}}
% 105
{\PTglyph{5}{t89_l04g15.png}}
% 106
{\PTglyph{5}{t89_l04g16.png}}
% 107
{\PTglyph{5}{t89_l04g17.png}}
% 108
{\PTglyph{5}{t89_l04g18.png}}
% 109
{\PTglyph{5}{t89_l04g19.png}}
% 110
{\PTglyph{5}{t89_l04g20.png}}
% 111
{\PTglyph{5}{t89_l04g21.png}}
% 112
{\PTglyph{5}{t89_l04g22.png}}
% 113
{\PTglyph{5}{t89_l04g23.png}}
% 114
{\PTglyph{5}{t89_l04g24.png}}
% 115
{\PTglyph{5}{t89_l04g25.png}}
% 116
{\PTglyph{5}{t89_l04g26.png}}
% 117
{\PTglyph{5}{t89_l04g27.png}}
% 118
{\PTglyph{5}{t89_l04g28.png}}
% 119
{\PTglyph{5}{t89_l04g29.png}}
% 120
{\PTglyph{5}{t89_l05g01.png}}
% 121
{\PTglyph{5}{t89_l05g02.png}}
% 122
{\PTglyph{5}{t89_l05g03.png}}
% 123
{\PTglyph{5}{t89_l05g04.png}}
% 124
{\PTglyph{5}{t89_l05g05.png}}
% 125
{\PTglyph{5}{t89_l05g06.png}}
% 126
{\PTglyph{5}{t89_l05g07.png}}
% 127
{\PTglyph{5}{t89_l05g08.png}}
% 128
{\PTglyph{5}{t89_l05g09.png}}
% 129
{\PTglyph{5}{t89_l05g10.png}}
//
%%% Local Variables:
%%% mode: latex
%%% TeX-engine: luatex
%%% TeX-master: shared
%%% End:

%//
%\glpismo%
 \glpismo
% 1
{\PTglyphid{Wi-15_0101}}
% 2
{\PTglyphid{Wi-15_0102}}
% 3
{\PTglyphid{Wi-15_0103}}
% 4
{\PTglyphid{Wi-15_0104}}
% 5
{\PTglyphid{Wi-15_0105}}
% 6
{\PTglyphid{Wi-15_0106}}
% 7
{\PTglyphid{Wi-15_0107}}
% 8
{\PTglyphid{Wi-15_0108}}
% 9
{\PTglyphid{Wi-15_0109}}
% 10
{\PTglyphid{Wi-15_0110}}
% 11
{\PTglyphid{Wi-15_0111}}
% 12
{\PTglyphid{Wi-15_0112}}
% 13
{\PTglyphid{Wi-15_0113}}
% 14
{\PTglyphid{Wi-15_0114}}
% 15
{\PTglyphid{Wi-15_0115}}
% 16
{\PTglyphid{Wi-15_0116}}
% 17
{\PTglyphid{Wi-15_0117}}
% 18
{\PTglyphid{Wi-15_0118}}
% 19
{\PTglyphid{Wi-15_0119}}
% 20
{\PTglyphid{Wi-15_0120}}
% 21
{\PTglyphid{Wi-15_0121}}
% 22
{\PTglyphid{Wi-15_0122}}
% 23
{\PTglyphid{Wi-15_0123}}
% 24
{\PTglyphid{Wi-15_0124}}
% 25
{\PTglyphid{Wi-15_0125}}
% 26
{\PTglyphid{Wi-15_0201}}
% 27
{\PTglyphid{Wi-15_0202}}
% 28
{\PTglyphid{Wi-15_0203}}
% 29
{\PTglyphid{Wi-15_0204}}
% 30
{\PTglyphid{Wi-15_0205}}
% 31
{\PTglyphid{Wi-15_0206}}
% 32
{\PTglyphid{Wi-15_0207}}
% 33
{\PTglyphid{Wi-15_0208}}
% 34
{\PTglyphid{Wi-15_0209}}
% 35
{\PTglyphid{Wi-15_0210}}
% 36
{\PTglyphid{Wi-15_0211}}
% 37
{\PTglyphid{Wi-15_0212}}
% 38
{\PTglyphid{Wi-15_0213}}
% 39
{\PTglyphid{Wi-15_0214}}
% 40
{\PTglyphid{Wi-15_0215}}
% 41
{\PTglyphid{Wi-15_0216}}
% 42
{\PTglyphid{Wi-15_0217}}
% 43
{\PTglyphid{Wi-15_0218}}
% 44
{\PTglyphid{Wi-15_0219}}
% 45
{\PTglyphid{Wi-15_0220}}
% 46
{\PTglyphid{Wi-15_0221}}
% 47
{\PTglyphid{Wi-15_0222}}
% 48
{\PTglyphid{Wi-15_0223}}
% 49
{\PTglyphid{Wi-15_0224}}
% 50
{\PTglyphid{Wi-15_0225}}
% 51
{\PTglyphid{Wi-15_0226}}
% 52
{\PTglyphid{Wi-15_0227}}
% 53
{\PTglyphid{Wi-15_0228}}
% 54
{\PTglyphid{Wi-15_0229}}
% 55
{\PTglyphid{Wi-15_0230}}
% 56
{\PTglyphid{Wi-15_0231}}
% 57
{\PTglyphid{Wi-15_0232}}
% 58
{\PTglyphid{Wi-15_0233}}
% 59
{\PTglyphid{Wi-15_0234}}
% 60
{\PTglyphid{Wi-15_0301}}
% 61
{\PTglyphid{Wi-15_0302}}
% 62
{\PTglyphid{Wi-15_0303}}
% 63
{\PTglyphid{Wi-15_0304}}
% 64
{\PTglyphid{Wi-15_0305}}
% 65
{\PTglyphid{Wi-15_0306}}
% 66
{\PTglyphid{Wi-15_0307}}
% 67
{\PTglyphid{Wi-15_0308}}
% 68
{\PTglyphid{Wi-15_0309}}
% 69
{\PTglyphid{Wi-15_0310}}
% 70
{\PTglyphid{Wi-15_0311}}
% 71
{\PTglyphid{Wi-15_0312}}
% 72
{\PTglyphid{Wi-15_0313}}
% 73
{\PTglyphid{Wi-15_0314}}
% 74
{\PTglyphid{Wi-15_0315}}
% 75
{\PTglyphid{Wi-15_0316}}
% 76
{\PTglyphid{Wi-15_0317}}
% 77
{\PTglyphid{Wi-15_0318}}
% 78
{\PTglyphid{Wi-15_0319}}
% 79
{\PTglyphid{Wi-15_0320}}
% 80
{\PTglyphid{Wi-15_0321}}
% 81
{\PTglyphid{Wi-15_0322}}
% 82
{\PTglyphid{Wi-15_0323}}
% 83
{\PTglyphid{Wi-15_0324}}
% 84
{\PTglyphid{Wi-15_0325}}
% 85
{\PTglyphid{Wi-15_0326}}
% 86
{\PTglyphid{Wi-15_0327}}
% 87
{\PTglyphid{Wi-15_0328}}
% 88
{\PTglyphid{Wi-15_0329}}
% 89
{\PTglyphid{Wi-15_0330}}
% 90
{\PTglyphid{Wi-15_0331}}
% 91
{\PTglyphid{Wi-15_0401}}
% 92
{\PTglyphid{Wi-15_0402}}
% 93
{\PTglyphid{Wi-15_0403}}
% 94
{\PTglyphid{Wi-15_0404}}
% 95
{\PTglyphid{Wi-15_0405}}
% 96
{\PTglyphid{Wi-15_0406}}
% 97
{\PTglyphid{Wi-15_0407}}
% 98
{\PTglyphid{Wi-15_0408}}
% 99
{\PTglyphid{Wi-15_0409}}
% 100
{\PTglyphid{Wi-15_0410}}
% 101
{\PTglyphid{Wi-15_0411}}
% 102
{\PTglyphid{Wi-15_0412}}
% 103
{\PTglyphid{Wi-15_0413}}
% 104
{\PTglyphid{Wi-15_0414}}
% 105
{\PTglyphid{Wi-15_0415}}
% 106
{\PTglyphid{Wi-15_0416}}
% 107
{\PTglyphid{Wi-15_0417}}
% 108
{\PTglyphid{Wi-15_0418}}
% 109
{\PTglyphid{Wi-15_0419}}
% 110
{\PTglyphid{Wi-15_0420}}
% 111
{\PTglyphid{Wi-15_0421}}
% 112
{\PTglyphid{Wi-15_0422}}
% 113
{\PTglyphid{Wi-15_0423}}
% 114
{\PTglyphid{Wi-15_0424}}
% 115
{\PTglyphid{Wi-15_0425}}
% 116
{\PTglyphid{Wi-15_0426}}
% 117
{\PTglyphid{Wi-15_0427}}
% 118
{\PTglyphid{Wi-15_0428}}
% 119
{\PTglyphid{Wi-15_0429}}
% 120
{\PTglyphid{Wi-15_0501}}
% 121
{\PTglyphid{Wi-15_0502}}
% 122
{\PTglyphid{Wi-15_0503}}
% 123
{\PTglyphid{Wi-15_0504}}
% 124
{\PTglyphid{Wi-15_0505}}
% 125
{\PTglyphid{Wi-15_0506}}
% 126
{\PTglyphid{Wi-15_0507}}
% 127
{\PTglyphid{Wi-15_0508}}
% 128
{\PTglyphid{Wi-15_0509}}
% 129
{\PTglyphid{Wi-15_0510}}
//
\endgl \xe
%%% Local Variables:
%%% mode: latex
%%% TeX-engine: luatex
%%% TeX-master: shared
%%% End:

% //
%\endgl \xe

\end{document}










33,Hochfeder-02_PT01_020bis.djvu,Hochfeder,02,01,020
34,Hochfeder-03_PT01_021.djvu,Hochfeder,03,01,021
35,Hochfeder-04_PT01_022.djvu,Hochfeder,04,01,022
36,Hochfeder-05_PT01_023.djvu,Hochfeder,05,01,023
37,Hochfeder-06_PT01_024.djvu,Hochfeder,06,01,024
38,Hochfeder-07_PT01_024.djvu,Hochfeder,07,01,024
39,Hochfeder-09_PT01_027.djvu,Hochfeder,09,01,027
40,Hochfeder-10_PT01_028.djvu,Hochfeder,10,01,028
41,Hochfeder-11_PT01_028.djvu,Hochfeder,11,01,028
42,Ungler1-01_PT03_112.djvu,Ungler1,01,03,112
43,Ungler1-02_PT03_113.djvu,Ungler1,02,03,113
44,Ungler1-03_PT03_114.djvu,Ungler1,03,03,114
45,Ungler1-04_PT03_115.djvu,Ungler1,04,03,115
46,Ungler1-05_PT03_116.djvu,Ungler1,05,03,116
47,Ungler1-06_PT03_117.djvu,Ungler1,06,03,117
48,Ungler1-07_PT03_118.djvu,Ungler1,07,03,118
49,Ungler1-08_PT03_120.djvu,Ungler1,08,03,120
50,Ungler1-09_PT03_120.djvu,Ungler1,09,03,120
51,Ungler1-10_PT03_119.djvu,Ungler1,10,03,119
52,Ungler2-01_PT05_239.djvu,Ungler2,01,05,239
53,Ungler2-02_PT05_240.djvu,Ungler2,02,05,240
54,Ungler2-03_PT05_241.djvu,Ungler2,03,05,241
55,Ungler2-04_PT05_242.djvu,Ungler2,04,05,242
56,Ungler2-05_PT05_243.djvu,Ungler2,05,05,243
57,Ungler2-06_PT05_241.djvu,Ungler2,06,05,241
58,Ungler2-09_PT05_244.djvu,Ungler2,09,05,244
59,Ungler2-10_PT05_245.djvu,Ungler2,10,05,245
60,Ungler2-11_PT05_245.djvu,Ungler2,11,05,245
61,Ungler2-12_PT05_243.djvu,Ungler2,12,05,243
62,Ungler2-13_PT05_243.djvu,Ungler2,13,05,243
63,Ungler2-14_PT07_357.djvu,Ungler2,14,07,357
64,Ungler2-15_PT07_358.djvu,Ungler2,15,07,358
65,Ungler2-16_PT07_359.djvu,Ungler2,16,07,359
66,Ungler2-17_PT07_357.djvu,Ungler2,17,07,357
67,Ungler2-18_PT07_360.djvu,Ungler2,18,07,360
68,Ungler2-19_PT07_360.djvu,Ungler2,19,07,360
69,Ungler2-20_PT07_361.djvu,Ungler2,20,07,361
70,Ungler2-21_PT07_362.djvu,Ungler2,21,07,362
71,Ungler2-22_PT07_362.djvu,Ungler2,22,07,362
72,Wirzbięta-01_PT09_469.djvu,Wirzbięta,01,09,469
73,Wirzbięta-01u_PT11_570.djvu,Wirzbięta,01u,11,570
74,Wirzbięta-02_PT09_469.djvu,Wirzbięta,02,09,469
75,Wirzbięta-02u_PT11_570.djvu,Wirzbięta,02u,11,570
76,Wirzbięta-03_PT09_469.djvu,Wirzbięta,03,09,469
77,Wirzbięta-03u_PT11_570.djvu,Wirzbięta,03u,11,570
78,Wirzbięta-03uu_PT11_570.djvu,Wirzbięta,03uu,11,570
79,Wirzbięta-04_PT09_469.djvu,Wirzbięta,04,09,469
80,Wirzbięta-04u_PT11_570.djvu,Wirzbięta,04u,11,570
81,Wirzbięta-05_PT09_469.djvu,Wirzbięta,05,09,469
82,Wirzbięta-05u_PT11_570.djvu,Wirzbięta,05u,11,570
83,Wirzbięta-09_PT11_562.djvu,Wirzbięta,09,11,562
84,Wirzbięta-10+8_PT11_563.djvu,Wirzbięta,10+8,11,563
85,Wirzbięta-11+11a_PT11_564.djvu,Wirzbięta,11+11a,11,564
86,Wirzbięta-12_PT11_564.djvu,Wirzbięta,12,11,564
87,Wirzbięta-13_PT11_565.djvu,Wirzbięta,13,11,565
88,Wirzbięta-14_PT11_565.djvu,Wirzbięta,14,11,565
89,Wirzbięta-15_PT11_566.djvu,Wirzbięta,15,11,566


%%% Local Variables: 
%%% coding: utf-8-unix
%%% mode: latex
%%% TeX-master: t
%%% TeX-PDF-mode: t
%%% TeX-engine: xetex
%%% End: 
